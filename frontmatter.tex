% !TeX program = XeLaTeX
% !TeX root = pujavidhanam.tex
\pagenumbering{Roman}
\thispagestyle{empty}
%\begin{titlepage}
%\setlength{\wpYoffset}{1.5cm}
%\ThisCenterWallPaper{0.7}{ShriRama.jpg}
%\vspace*{13.5cm}\centerline{\font\x="Siddhanta:script=deva,mapping=tex-text" at 64pt \x स्तोत्रसङ्ग्रहः}
%\end{titlepage}
%\thispagestyle{empty}\clearpage
\begin{titlepage}
%\ThisCenterWallPaper{0.7}{ShriRama.jpg}
\vspace*{6.5cm}\centerline{\font\x="Siddhanta:script=deva,mapping=tex-text" at 48pt \x पूजा-विधानम्}
\end{titlepage}
\begin{center}
\parbox{10cm}{
\fontspec{Candara}
{\Large \textbf{Colophon}}\\
\noindent This document was typeset using \XeLaTeX, and uses the Siddhanta font extensively. It also uses several \LaTeX\ macros designed by \textit{H.~L.~Prasād}. Practically all the encoding was done with the help of Itranslator 2003 and Ajit Krishnan's mudgala IME (\url{http://www.aupasana.com/}).\\
\vspace*{5cm}

{\large \textbf{Acknowledgements}}\\
The initial encodings of some of these texts were obtained from \url{http://sanskritdocuments.org/} and/or \url{http://prapatti.com/}.  \\
See also \url{http://stotrasamhita.github.io/about/}


\vspace*{1cm}
\begin{center}
{\fontspec[Script=Devanagari]{Siddhanta} स्तोत्रसङ्ग्रहः} is also available online (in PDF format) at: \\
\url{http://stotrasamhita.github.io/}

\vspace*{2cm}

{\scshape{For Personal Use Only\\
 Not For Commercial Printing/Distribution}}
\end{center}
}
\end{center}
\clearemptydoublepage
\setcounter{page}{0}
\pagenumbering{roman}
\renewcommand{\chaptermark}[1]{%
\markboth{#1}{}}
\pdfbookmark[1]{Contents}{Contents}
\begin{center}
\begin{large}
\tableofcontents
\end{large}
\end{center}

\mbox{}\thispagestyle{empty}
\clearpage
\setcounter{page}{0}
\pagenumbering{arabic}
\setlength{\emergencystretch}{3em}
% \fontsize{16pt}{19.2pt}\selectfont
\fontsize{17pt}{20pt}\selectfont
% \fontsize{19pt}{23pt}\selectfont