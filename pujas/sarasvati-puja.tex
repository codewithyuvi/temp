% !TeX program = XeLaTeX
% !TeX root = ../pujavidhanam.tex

\setlength{\parindent}{0pt}
\chapt{श्री-सरस्वती-पूजा}

\sect{पूर्वाङ्गविघ्नेश्वरपूजा}

(आचम्य)
\twolineshloka*
{शुक्लाम्बरधरं विष्णुं शशिवर्णं चतुर्भुजम्}
{प्रसन्नवदनं ध्यायेत् सर्वविघ्नोपशान्तये}
 
प्राणान्  आयम्य।  ॐ भूः + भूर्भुवः॒ सुव॒रोम्।
 
(अप उपस्पृश्य, पुष्पाक्षतान् गृहीत्वा)\\
ममोपात्तसमस्त दुरितक्षयद्वारा \\
श्रीपरमेश्वरप्रीत्यर्थं करिष्यमाणस्य कर्मणः\\
 निर्विघ्नेन परिसमाप्त्यर्थम् आदौ विघ्नेश्वरपूजां करिष्ये।

\twolineshloka*
{ॐ ग॒णानां᳚ त्वा ग॒णप॑तिꣳ हवामहे क॒विं क॑वी॒नामु॑प॒मश्र॑वस्तमम्}
{ज्ये॒ष्ठ॒राजं॒ ब्रह्म॑णां ब्रह्मणस्पत॒ आ नः॑ शृ॒ण्वन्नू॒तिभिः॑ सीद॒ साद॑नम्}
अस्मिन् हरिद्राबिम्बे महागणपतिं ध्यायामि, आवाहयामि।\\


ॐ महागणपतये नमः  आसनं समर्पयामि।\\
पादयोः पाद्यं समर्पयामि। हस्तयोरर्घ्यं समर्पयामि।\\
आचमनीयं समर्पयामि।\\
ॐ भूर्भुवस्सुवः। शुद्धोदकस्नानं समर्पयामि।\\
स्नानानन्तरमाचमनीयं समर्पयामि।\\
वस्त्रार्थमक्षतान् समर्पयामि।\\
यज्ञोपवीताभरणार्थे अक्षतान् समर्पयामि।\\
दिव्यपरिमलगन्धान् धारयामि।\\
गन्धस्योपरि हरिद्राकुङ्कुमं समर्पयामि। अक्षतान् समर्पयामि। \\
पुष्पमालिकां समर्पयामि। पुष्पैः पूजयामि।

\dnsub{अर्चना}
% \setenumerate{label=\devanumber.}
% \renewcommand{\labelenumi}{\devanumber\theenumi.}
\begin{enumerate}%[label=\devanumber\value{enumi}]
\begin{minipage}{0.475\linewidth}   
\item ॐ सुमुखाय नमः
\item ॐ एकदन्ताय नमः
\item ॐ कपिलाय नमः
\item ॐ गजकर्णकाय नमः
\item ॐ लम्बोदराय नमः
\item ॐ विकटाय नमः
\item ॐ विघ्नराजाय नमः
\item ॐ विनायकाय नमः
\item ॐ धूमकेतवे नमः
  \end{minipage}
  \begin{minipage}{0.525\linewidth}
\item ॐ गणाध्यक्षाय नमः
\item ॐ फालचन्द्राय नमः
\item ॐ गजाननाय नमः
\item ॐ वक्रतुण्डाय नमः
\item ॐ शूर्पकर्णाय नमः
\item ॐ हेरम्बाय नमः
\item ॐ स्कन्दपूर्वजाय नमः
\item ॐ सिद्धिविनायकाय नमः
\item ॐ विघ्नेश्वराय नमः
  \end{minipage}
\end{enumerate}
नानाविधपरिमलपत्रपुष्पाणि समर्पयामि॥\\
धूपमाघ्रापयामि।\\
अलङ्कारदीपं सन्दर्शयामि।\\
नैवेद्यम्।\\
ताम्बूलं समर्पयामि।\\
कर्पूरनीराजनं समर्पयामि।\\
कर्पूरनीराजनानन्तरमाचमनीयं समर्पयामि।\\
{वक्रतुण्डमहाकाय कोटिसूर्यसमप्रभ।}\\
{अविघ्नं कुरु मे देव सर्वकार्येषु सर्वदा॥}\\
प्रार्थनाः समर्पयामि।

अनन्तकोटिप्रदक्षिणनमस्कारान् समर्पयामि।\\
छत्त्रचामरादिसमस्तोपचारान् समर्पयामि।\\

 
\sect{प्रधान-पूजा — श्री-सरस्वती-पूजा}

\twolineshloka*
{शुक्लाम्बरधरं विष्णुं शशिवर्णं चतुर्भुजम्}
{प्रसन्नवदनं ध्यायेत् सर्वविघ्नोपशान्तये}
 
\dnsub{सङ्कल्पः}

ममोपात्त-समस्त-दुरित-क्षयद्वारा श्री-परमेश्वर-प्रीत्यर्थं शुभे शोभने मुहूर्ते अद्य ब्रह्मणः
द्वितीयपरार्द्धे श्वेतवराहकल्पे वैवस्वतमन्वन्तरे अष्टाविंशतितमे कलियुगे प्रथमे पादे
जम्बूद्वीपे भारतवर्षे भरतखण्डे मेरोः दक्षिणे पार्श्वे शकाब्दे अस्मिन् वर्तमाने व्यावहारिकाणां प्रभवादीनां षष्ट्याः संवत्सराणां मध्ये (  )\see{app:samvatsara_names} नाम संवत्सरे दक्षिणायने 
वर्ष-ऋतौ (कन्या/तुला)-आश्वयुज-मासे शुक्लपक्षे नवम्यां शुभतिथौ ( )-वासरयुक्तायाम्
(  )\see{app:nakshatra_names} नक्षत्र (  )\see{app:yoga_names} नाम  योग  ( )\see{app:karanam_names} करण युक्तायां च एवं गुण विशेषण विशिष्टायाम्
अस्याम् ( ) शुभतिथौ 
अस्माकं सहकुटुम्बानां क्षेमस्थैर्य-धैर्य-वीर्य-विजय-आयुरारोग्य-ऐश्वर्याभिवृद्ध्यर्थम्
 धर्मार्थकाममोक्ष\-चतुर्विधफलपुरुषार्थसिद्ध्यर्थं पुत्रपौत्राभि\-वृद्ध्यर्थम् इष्टकाम्यार्थसिद्ध्यर्थम्
मम इहजन्मनि पूर्वजन्मनि जन्मान्तरे च सम्पादितानां ज्ञानाज्ञानकृतमहा\-पातकचतुष्टय-व्यतिरिक्तानां रहस्यकृतानां प्रकाशकृतानां सर्वेषां पापानां सद्य अपनोदनद्वारा सकल-पापक्षयार्थं 
श्री-दुर्गा-लक्ष्मी-सरस्वती-प्रीत्यर्थं श्री-दुर्गा-लक्ष्मी-सरस्वती-पूजां करिष्ये।
तदङ्गं कलशपूजां च करिष्ये। 


श्रीविघ्नेश्वराय नमः यथास्थानं प्रतिष्ठापयामि। शोभनार्थे क्षेमाय पुनरागमनाय च।\\
(गणपति-प्रसादं शिरसा गृहीत्वा)

\dnsub{आसन-पूजा}
\centerline{पृथिव्या  मेरुपृष्ठ  ऋषिः।  सुतलं  छन्दः।  कूर्मो  देवता॥}
\twolineshloka*
{पृथ्वि  त्वया  धृता  लोका  देवि  त्वं  विष्णुना  धृता}
{त्वं  च  धारय  मां  देवि  पवित्रं  चाऽऽसनं  कुरु}


\dnsub{घण्टापूजा}
\twolineshloka*
{आगमार्थं तु देवानां गमनार्थं तु रक्षसाम्}
{घण्टारवं करोम्यादौ देवताऽऽह्वानकारणम्}


\dnsub{कलशपूजा}
ॐ कलशाय नमः दिव्यगन्धान् धारयामि।\\
ॐ गङ्गायै नमः। ॐ यमुनायै नमः। ॐ गोदावर्यै नमः।  ॐ सरस्वत्यै नमः। ॐ नर्मदायै नमः। ॐ सिन्धवे नमः। ॐ कावेर्यै नमः।\\
ॐ सप्तकोटिमहातीर्थान्यावाहयामि।\\[-0.25ex]

(अथ कलशं स्पृष्ट्वा जपं कुर्यात्) \\
आपो॒ वा इ॒द सर्वं॒ विश्वा॑ भू॒तान्याप॑ प्रा॒णा वा आप॑ प॒शव॒ आपो\-ऽन्न॒मापोऽमृ॑त॒माप॑ स॒म्राडापो॑ वि॒राडाप॑ स्व॒राडाप॒श्\-छन्दा॒स्यापो॒ ज्योती॒ष्यापो॒ यजू॒ष्याप॑ स॒त्यमाप॒ सर्वा॑ दे॒वता॒ आपो॒ भूर्भुव॒ सुव॒राप॒ ओम्॥\\

\twolineshloka* 
{कलशस्य मुखे विष्णुः कण्ठे रुद्रः समाश्रितः}
{मूले तत्र स्थितो ब्रह्मा मध्ये मातृगणाः स्मृताः}
\threelineshloka* 
{कुक्षौ तु सागराः सर्वे सप्तद्वीपा वसुन्धरा}
{ऋग्वेदोऽथ यजुर्वेदः सामवेदोऽप्यथर्वणः}
{अङ्गैश्च सहिताः सर्वे कलशाम्बुसमाश्रिताः}
\twolineshloka* 
{गङ्गे च यमुने चैव गोदावरि सरस्वति}
{नर्मदे सिन्धुकावेरि जलेऽस्मिन् सन्निधिं कुरु}
\twolineshloka*
{सर्वे समुद्राः सरितः तीर्थानि च ह्रदा नदाः}
{आयान्तु देवपूजार्थं दुरितक्षयकारकाः}

\centerline{ॐ भूर्भुवः॒ सुवो॒ भूर्भुवः॒ सुवो॒ भूर्भुवः॒ सुवः॑।}

(इति कलशजलेन सर्वोपकरणानि आत्मानं च प्रोक्ष्य।)


\dnsub{आत्म-पूजा}
ॐ आत्मने नमः, दिव्यगन्धान् धारयामि।
\begin{multicols}{2}
१. ॐ आत्मने नमः\\
२. ॐ अन्तरात्मने नमः\\
३. ॐ योगात्मने नमः\\
४. ॐ जीवात्मने नमः\\
५. ॐ परमात्मने नमः\\
६. ॐ ज्ञानात्मने नमः
\end{multicols}
समस्तोपचारान् समर्पयामि।

\twolineshloka*
{देहो देवालयः प्रोक्तो जीवो देवः सनातनः}
{त्यजेदज्ञाननिर्माल्यं सोऽहं भावेन पूजयेत्}


\begin{minipage}{\linewidth}
\dnsub{पीठ-पूजा}

\begin{multicols}{2}
\begin{enumerate}
\item ॐ आधारशक्त्यै नमः
\item ॐ मूलप्रकृत्यै नमः
\item ॐ आदिकूर्माय नमः 
\item ॐ आदिवराहाय नमः
\item ॐ अनन्ताय नमः
\item ॐ पृथिव्यै नमः
\item ॐ रत्नमण्डपाय नमः
\item ॐ रत्नवेदिकायै नमः
\item ॐ स्वर्णस्तम्भाय नमः
\item ॐ श्वेतच्छत्त्राय नमः
\item ॐ कल्पकवृक्षाय नमः
\item ॐ क्षीरसमुद्राय नमः 
\item ॐ सितचामराभ्यां नमः
\item ॐ योगपीठासनाय नमः
\end{enumerate}
\end{multicols}

\end{minipage}

\dnsub{गुरु ध्यानम्}

\twolineshloka*
{गुरुर्ब्रह्मा गुरुर्विष्णुर्गुरुर्देवो महेश्वरः}
{गुरुः साक्षात् परं ब्रह्म तस्मै श्री गुरवे नमः}


\sect{षोडशोपचारपूजा}
% \renewcommand{\devAya}{श्री-सरस्वत्यै नमः,}
\renewcommand{\devAya}{श्री-दुर्गा-लक्ष्मी-युक्तायै सरस्वत्यै नमः,}



\begin{center}

\twolineshloka*
{सरस्वतीं सत्यवासां सुधांशुसमविग्रहाम्}
{स्फटिकाक्षत्रजं पद्मं पुस्तकं च शुकं करैः}

\twolineshloka*
{चतुर्भिर्दधतीं देवीं चन्द्रबिम्ब-समाननाम्}
{वल्लभामखिलार्थानां वल्लकीवादनप्रियाम्}

\twolineshloka*
{भारतीं भावये देवीं भाषाणामधिदेवताम्}
{भावितां हृदये सद्भिर्भामिनीं परमेष्ठिनः}
\textbf{अस्मिन् पुस्तक-मण्डले दुर्गालक्ष्मी-युक्तां सरस्वतीं ध्यायामि॥}
\medskip

\twolineshloka*
{चतुर्भुजां चन्द्रवर्णां चतुराननवल्लभाम्}
{आवाहयामि वाणि त्वामाश्रितार्ति-विनाशिनीम्}
\textbf{अस्मिन् पुस्तक-मण्डले दुर्गालक्ष्मी-युक्तां सरस्वतीमावाहयामि॥}
\medskip

\twolineshloka*
{आसनं संगृहाणेद-माश्रिते सकलामरैः}
{आदृते सर्वमुनिभिरार्तिदे सुरवैरिणाम् }
\textbf{\devAya{} आसनं समर्पयामि।}
\medskip

\twolineshloka*
{पाद्यमाद्यन्तशून्यायै वेद्यायै वेदवादिभिः}
{दास्यामि वाणि वरदे देवराजसमर्चिते}
\textbf{\devAya{} पाद्यं समर्पयामि।}
\medskip

\twolineshloka*
{अघहन्त्रि गृहाणेदम् अर्घ्यमष्टाङ्गसंयुतम्}
{अम्बाखिलानां जगताम् अम्बुजासनसुन्दरि}
\textbf{\devAya{} अर्घ्यं समर्पयामि।}
\medskip

\twolineshloka*
{आचम्यतां तोयमिदमाश्रितार्थप्रदायिनि}
{आत्मभूवदनावासे आधिहारिणि ते नमः}
\textbf{\devAya{} आचमनीयं समर्पयामि।}
\medskip

\twolineshloka*
{मधुपर्कं गृहाणेदं मधुसूदनवन्दिते}
{मन्दस्मिते महादेवि महादेवसमर्चिते}
\textbf{\devAya{} मधुपर्कं समर्पयामि।}
\medskip

\twolineshloka*
{पञ्चामृतं गृहाणेदं पञ्चाननसमर्चिते}
{पयोदधिघृतोपेतं पञ्चपातकनाशिनि}
\textbf{\devAya{} पञ्चामृतं समर्पयामि।}
\medskip

\twolineshloka*
{नमस्ते हंसवाहिन्यै नमस्ते धातृपत्निके}
{गन्धोदकेन सम्पूर्णं स्नानं च प्रतिगृह्यताम्}
\textbf{\devAya{} गन्धोदकस्नानं समर्पयामि।}
\medskip

\twolineshloka*
{साध्वीनामग्रतो गण्ये साधुसङ्घसमादृते}
{सरस्वति नमस्तुभ्यं स्नानं स्वीकुरु सम्प्रति}
\textbf{\devAya{} शुद्धोदकस्नानं समर्पयामि।}
\medskip

\twolineshloka*
{दुकूलद्वितयं देवि दुरितापहवैभवे}
{विधिप्रिये गृहाणेदं सुधानिधिसमं शिवे}
\textbf{\devAya{} वस्त्रं समर्पयामि।}
\medskip

\twolineshloka*
{उपवीतं गृहाणेदमुपमाशून्यवैभवे}
{हिरण्यगर्भमहिषि हिरण्मयगुणैः कृतम्}
\textbf{\devAya{} उपवीतं समर्पयामि।}
\medskip

\twolineshloka*
{वर्णरूपे गृहाणेदं स्वर्णवर्यपरिष्कृतम्}
{अर्णवोद्धृतरत्नाढ्यं कर्णभूषादिभूषणम्}
\textbf{\devAya{} आभरणानि समर्पयामि।}
\medskip


\twolineshloka*
{विन्यस्तं नेत्रयोः कान्त्यै सौवीराञ्जनमुत्तमम्}
{सङ्गृहाण सुरश्रेष्ठे वागीश्वरि नमोऽस्तु ते}
\textbf{\devAya{} नेत्राञ्जनं समर्पयामि।}
\medskip

\twolineshloka*
{कुङ्कुमाञ्जन-सिन्दूर-कञ्चुकादिकमम्बिके}
{सौभाग्यद्रव्यमखिलं सुरवन्द्ये गृहाण मे}
\textbf{\devAya{} सौभाग्यद्रव्यं समर्पयामि।}
\medskip

\twolineshloka*
{अन्धकारिप्रियाराध्ये गन्धमुत्तमसौरभम्}
{गृहाण वाणि वरदे गन्धर्वपरिपूजिते}
\textbf{\devAya{} गन्धं समर्पयामि।}
\medskip

\twolineshloka*
{अक्षतांस्त्वं गृहाणेमान् अहतानमरार्चिते}
{अक्षतेऽद्भुतरूपाढ्ये यक्षराजादिवन्दिते}
\textbf{\devAya{} अक्षतान् समर्पयामि।}
\medskip

\twolineshloka*
{पुन्नाग-जाती-मल्ल्यादि-पुष्पजातं गृहाण मे}
{पुमर्थदायिनि परे पुस्तकाढ्य-कराम्बुजे}
\textbf{\devAya{} पुष्पाणि समर्पयामि।}
\end{center}


\dnsub{अङ्गपूजा}
\begin{longtable}{ll@{— }l}
 १. & पावनायै नमः & पादौ पूजयामि\\
 २. & गिरे नमः & गुल्फौ पूजयामि\\
 ३. & जगद्वन्द्यायै नमः &  जङ्घे पूजयामि\\
 ४. & जलजासनायै नमः & जानुनी पूजयामि\\
 ५. & उत्तमायै नमः & ऊरू  पूजयामि\\
 ६. & कमलासनप्रियायै नमः & कटिं पूजयामि\\
 ७. & नानाविद्यायै नमः & नाभिं पूजयामि\\
 ८. & वाण्यै नमः & वक्षः पूजयामि\\
 ९. & कुरङ्गाक्ष्यै नमः & कुचौ पूजयामि\\
१०. & कलारूपिण्यै नमः &  कण्ठं पूजयामि\\
११. & भाषायै नमः & बाहून् पूजयामि\\
१२. & चिरन्तनायै नमः & चुबुकं पूजयामि\\
१३. & मुग्धस्मितायै नमः & मुखं पूजयामि\\
१४. & लोलेक्षणायै नमः & लोचने पूजयामि\\
१५. & कलायै नमः & ललाटं पूजयामि\\
१६. & वर्णरूपायै नमः & कर्णौ पूजयामि\\
१७. & करुणायै नमः & कचान् पूजयामि\\
१८. & शिवायै नमः & शिरः पूजयामि\\
१९. & सरस्वत्यै नमः &  सर्वाण्यङ्गानि पूजयामि\\
\end{longtable}

\begingroup
\centering
\setlength{\columnseprule}{1pt}
\let\chapt\sect
\needspace{6em}
\input{../namavali-manjari/100/Durga_108.tex}
\input{../namavali-manjari/100/Lakshmi_108.tex}
\input{../namavali-manjari/100/Saraswati_108.tex}

\endgroup

\textbf{\devAya{} नानाविध\-परिमल\-पत्र-पुष्पाणि समर्पयामि।}

\sect{उत्तराङ्गपूजा}
\begin{center}

\twolineshloka*
{गुग्गुलागरु-कस्तूरी-गोरोचन-समन्वितम्}
{धूपं गृहाण देवेशि भक्तिं त्वय्यचलां कुरु}
\textbf{\devAya{} धूपमाघ्रापयामि।}
\medskip

\twolineshloka*
{कामधेनुसमुद्भूत-घृतवर्ति-समन्वितम्}
{दीपं गृहाण कल्याणि कमलासनवल्लभे}
\textbf{\devAya{} अलङ्कारदीपं सन्दर्शयामि।}
\medskip

ॐ भूर्भुवः॒ सुवः॑। + ब्र॒ह्मणे॒ स्वाहा᳚।
\twolineshloka*
{नैवेद्यं षड्रसोपेतं शर्करामधुसंयुतम्}
{पायसान्नं च भारत्यै ददामि प्रतिगृह्यताम्}

\twolineshloka*
{अपूपान् विविधान् स्वादून् शालिपिष्टोपपाचितान्}
{मृदुलान् गुडसम्मिश्रान् भक्ष्यादींश्च ददामि ते}
\textbf{\devAya{} ( ) नैवेद्यं निवेदयामि।}
\medskip
मध्ये मध्ये अमृतपानीयं समर्पयामि।
नैवेद्यानन्तरम् आचमनीयं समर्पयामि।

\twolineshloka*
{पूगीफल-समायुक्तं नागवल्लीदलैर्युतम्}
{कर्पूरचूर्ण-संयुक्तं ताम्बूलं प्रतिगृह्यताम्}
\textbf{\devAya{} कर्पूर-ताम्बूलं समर्पयामि।}
\medskip

\twolineshloka*
{नीराजनं गृहाण त्वं जगदानन्ददायिनि}
{जगत्तिमिरमार्तण्डमण्डले ते नमो नमः}
\textbf{\devAya{} कर्पूर-नीराजनं सन्दर्शयामि।}
\textbf{कर्पूरनीराजनानन्तरम् आचमनीयं समर्पयामि।}
\medskip

\twolineshloka*
{शारदे लोकमातस्त्वम् आश्रिताभीष्टदायिनि}
{पुष्पाञ्जलिं गृहाण त्वं मया भक्त्या समर्पितम्}
\textbf{श्री-वरलक्ष्म्यै नमः मन्त्रपुष्पाञ्जलिं समर्पयामि।}
\textbf{स्वर्णपुष्पं समर्पयामि।}

\twolineshloka*
{यानि कानि च पापानि जन्मान्तरकृतानि च}
{तानि तानि विनश्यन्ति प्रदक्षिण-पदे पदे}

\twolineshloka*
{पाहि पाहि जगद्वन्द्ये नमस्ते भक्तवत्सले}
{नमस्तुभ्यं नमस्तुभ्यं नमस्तुभ्यं नमो नमः}

\twolineshloka*
{सरस्वति नमस्तुभ्यं वरदे भक्तवत्सले}
{त्राहि मां नरकाद् घोरात् ब्रह्मपत्नि नमोऽस्तु ते}
\textbf{\devAya{} नमस्कारान् समर्पयामि।}\medskip

\sect{प्रार्थना}

\fourlineindentedshloka*
{या कुन्देन्दु-तुषारहारधवला या शुभ्रवस्त्रावृता}
{या वीणा-वरदण्डमण्डितकरा या श्वेतपद्मासना}
{या ब्रह्माच्युतशङ्करप्रभृतिभिर्देवैः सदा पूजिता}
{सा मां पातु सरस्वती भगवती निःशेषजाड्यापहा}

\fourlineindentedshloka*
{सरस्वती सरसिजकेसरप्रभा}
{तपस्विनी सितकमलासनप्रिया}
{घनस्तनी कमलविलोललोचना}
{मनस्विनी भवतु वरप्रसादिनी}

\twolineshloka*
{पाशाङ्कुशधरा वाणी वीणापुस्तकधारिणी}
{मम वक्त्रे वसेन्नित्यं सन्तुष्टाऽस्तु च सर्वदा}

\twolineshloka*
{चतुर्दशसु विद्यासु रमते या सरस्वती}
{सा देवी कृपया मह्यं जिह्वासिद्धिं करोतु च}

\textbf{प्रार्थनाः समर्पयामि।}

\sect{उपायनदानम्}
ब्राह्मणपूजा।

\twolineshloka*
{शुक्लाम्बरधरं विष्णुं शशिवर्णं चतुर्भुजम्}
{प्रसन्नवदनं ध्यायेत् सर्वविघ्नोपशान्तये}

अद्य-पूर्वोक्त-एवं-गुण-सकल-विशेषेण-विशिष्टायां अस्यां नवम्यां शुभतिथौ श्री-दुर्गा-लक्ष्मी-सरस्वती-देवतामुद्दिश्य अद्य मया अनुष्ठित-श्री-दुर्गा-लक्ष्मी-सरस्वती-पूजान्ते  
श्री-दुर्गा-लक्ष्मी-सरस्वती-प्रीत्यर्थं उपायनदानं करिष्ये।

श्री-दुर्गा-लक्ष्मी-सरस्वती-स्वरूपस्य ब्राह्मणस्य इदम् आसनम्। इमे गन्धाः। सकलाराधनैः स्वर्चितम्।

\twolineshloka*
{भारती प्रतिगृह्णातु भारतीयं ददाति च}
{भारती तारिका द्वाभ्यां भारत्यै ते नमो नमः}

इदम् उपायनं सदक्षिणाकं सताम्बूलं श्री-दुर्गा-लक्ष्मी-सरस्वती-प्रीतिं कामयमानः श्री-दुर्गा-लक्ष्मी-सरस्वती-स्वरूपाय ब्राह्मणाय तुभ्यमहं सम्प्रददे न मम।

\twolineshloka*
{हिरण्यगर्भगर्भस्थं हेमबीजं विभावसोः}
{अनन्तपुण्यफलदम् अतः शान्तिं प्रयच्छ मे}

श्री-दुर्गा-लक्ष्मी-सरस्वती-पूजाकाले अस्मिन् मया क्रियमाण\\
श्री-दुर्गा-लक्ष्मी-सरस्वती-पूजायां यद्देयमुपायनदानं तत्प्रत्याम्नायार्थं हिरण्यं\\
श्री-दुर्गा-लक्ष्मी-सरस्वती-प्रीतिं कामयमानः\\
मनसोद्दिष्टाय ब्राह्मणाय तुभ्यमहं सम्प्रददे नमः न मम।\\ 
अनया पूजया श्री-दुर्गा-लक्ष्मी-सरस्वत्यः प्रीयन्ताम्। 
\end{center}

\sect{अपराध-क्षमापनम्}

\twolineshloka*
{यस्य स्मृत्या च नामोक्त्या तपः-पूजा-क्रियादिषु}
{न्यूनं सम्पूर्णतां याति सद्यो वन्दे तमच्युतम्}

\fourlineindentedshloka*
{कायेन वाचा मनसेन्द्रियैर्वा}
{बुद्‌ध्याऽऽत्मना वा प्रकृतेः स्वभावात्}
{करोमि यद्यत् सकलं परस्मै}
{नारायणायेति समर्पयामि}

\centerline{सर्वं तत्सद्ब्रह्मार्पणमस्तु।}

\closesub

\input{../stotra-sangrahah/stotras/big/SaraswatiSahasranamaStotram.tex} 

\closesection