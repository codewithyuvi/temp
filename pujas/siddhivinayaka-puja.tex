% !TeX program = XeLaTeX
% !TeX root = ../pujavidhanam.tex

\setlength{\parindent}{0pt}
\chapt{श्री-सिद्धिविनायक-पूजा}
\centerline{\small{(मूलम्—श्री-व्रतराजः)}}

\sect{पूर्वाङ्गविघ्नेश्वरपूजा}

(आचम्य)
\twolineshloka*
{शुक्लाम्बरधरं विष्णुं शशिवर्णं चतुर्भुजम्}
{प्रसन्नवदनं ध्यायेत् सर्वविघ्नोपशान्तये}
 
प्राणान्  आयम्य।  ॐ भूः + भूर्भुवः॒ सुव॒रोम्।
 
(अप उपस्पृश्य, पुष्पाक्षतान् गृहीत्वा)\\
ममोपात्तसमस्त दुरितक्षयद्वारा \\
श्रीपरमेश्वरप्रीत्यर्थं करिष्यमाणस्य कर्मणः\\
 निर्विघ्नेन परिसमाप्त्यर्थम् आदौ विघ्नेश्वरपूजां करिष्ये।

\twolineshloka*
{ॐ ग॒णानां᳚ त्वा ग॒णप॑तिꣳ हवामहे क॒विं क॑वी॒नामु॑प॒मश्र॑वस्तमम्}
{ज्ये॒ष्ठ॒राजं॒ ब्रह्म॑णां ब्रह्मणस्पत॒ आ नः॑ शृ॒ण्वन्नू॒तिभिः॑ सीद॒ साद॑नम्}
अस्मिन् हरिद्राबिम्बे महागणपतिं ध्यायामि, आवाहयामि।\\


ॐ महागणपतये नमः  आसनं समर्पयामि।\\
पादयोः पाद्यं समर्पयामि। हस्तयोरर्घ्यं समर्पयामि।\\
आचमनीयं समर्पयामि।\\
ॐ भूर्भुवस्सुवः। शुद्धोदकस्नानं समर्पयामि।\\
स्नानानन्तरमाचमनीयं समर्पयामि।\\
वस्त्रार्थमक्षतान् समर्पयामि।\\
यज्ञोपवीताभरणार्थे अक्षतान् समर्पयामि।\\
दिव्यपरिमलगन्धान् धारयामि।\\
गन्धस्योपरि हरिद्राकुङ्कुमं समर्पयामि। अक्षतान् समर्पयामि। \\
पुष्पमालिकां समर्पयामि। पुष्पैः पूजयामि।

\dnsub{अर्चना}
% \setenumerate{label=\devanumber.}
% \renewcommand{\labelenumi}{\devanumber\theenumi.}
\begin{enumerate}%[label=\devanumber\value{enumi}]
\begin{minipage}{0.475\linewidth}   
\item ॐ सुमुखाय नमः
\item ॐ एकदन्ताय नमः
\item ॐ कपिलाय नमः
\item ॐ गजकर्णकाय नमः
\item ॐ लम्बोदराय नमः
\item ॐ विकटाय नमः
\item ॐ विघ्नराजाय नमः
\item ॐ विनायकाय नमः
\item ॐ धूमकेतवे नमः
  \end{minipage}
  \begin{minipage}{0.525\linewidth}
\item ॐ गणाध्यक्षाय नमः
\item ॐ फालचन्द्राय नमः
\item ॐ गजाननाय नमः
\item ॐ वक्रतुण्डाय नमः
\item ॐ शूर्पकर्णाय नमः
\item ॐ हेरम्बाय नमः
\item ॐ स्कन्दपूर्वजाय नमः
\item ॐ सिद्धिविनायकाय नमः
\item ॐ विघ्नेश्वराय नमः
  \end{minipage}
\end{enumerate}
नानाविधपरिमलपत्रपुष्पाणि समर्पयामि॥\\
धूपमाघ्रापयामि।\\
अलङ्कारदीपं सन्दर्शयामि।\\
नैवेद्यम्।\\
ताम्बूलं समर्पयामि।\\
कर्पूरनीराजनं समर्पयामि।\\
कर्पूरनीराजनानन्तरमाचमनीयं समर्पयामि।\\
{वक्रतुण्डमहाकाय कोटिसूर्यसमप्रभ।}\\
{अविघ्नं कुरु मे देव सर्वकार्येषु सर्वदा॥}\\
प्रार्थनाः समर्पयामि।

अनन्तकोटिप्रदक्षिणनमस्कारान् समर्पयामि।\\
छत्त्रचामरादिसमस्तोपचारान् समर्पयामि।\\


\sect{प्रधान-पूजा — श्री-सिद्धिविनायक-पूजा}

\twolineshloka*
{शुक्लाम्बरधरं विष्णुं शशिवर्णं चतुर्भुजम्}
{प्रसन्नवदनं ध्यायेत् सर्वविघ्नोपशान्तये}

प्राणान् आयम्य। ॐ भूः + भूर्भुवः॒ सुव॒रोम्।

\dnsub{सङ्कल्पः}

ममोपात्तसमस्तदुरितक्षयद्वारा श्रीपरमेश्वरप्रीत्यर्थं शुभे शोभने मुहूर्ते अद्यब्रह्मणः
द्वितीयपरार्द्धे श्वेतवराहकल्पे वैवस्वतमन्वन्तरे अष्टाविंशतितमे कलियुगे प्रथमे पादे
जम्बूद्वीपे भारतवर्षे भरतखण्डे मेरोः दक्षिणेपार्श्वे शकाब्दे अस्मिन् वर्तमाने व्यावहारिके
प्रभवादि षष्टिसंवत्सराणां मध्ये ( )\see{app:samvatsara_names} नाम संवत्सरे दक्षिणायने 
वर्ष-ऋतौ (सिंह/कन्या)-भाद्रपद-मासे शुक्लपक्षे चतुर्थ्यां शुभतिथौ ( ) वासरयुक्तायाम्
( )\see{app:nakshatra_names} नक्षत्र ( )\see{app:yoga_names} नाम योग 
(वणिजा/भद्रा)\see{app:karanam_names}-करण युक्तायां च एवं गुण विशेषण विशिष्टायाम्
अस्याम् चतुर्थ्यां शुभतिथौ 
अस्माकं सकुटुम्बानां क्षेमस्थैर्य-धैर्य-वीर्य-विजय-आयुरारोग्य-ऐश्वर्याभिवृद्ध्यर्थम्
धर्मार्थकाममोक्ष\-चतुर्विधफलपुरुषार्थसिद्ध्यर्थं पुत्रपौत्राभि\-वृद्ध्यर्थम् इष्टकाम्यार्थसिद्ध्यर्थं
मम इहजन्मनि पूर्वजन्मनि जन्मान्तरे च सम्पादितानां ज्ञानाज्ञानकृतमहा\-पातकचतुष्टय-व्यतिरिक्तानां 
रहस्यकृतानां प्रकाशकृतानां सर्वेषां पापानां सद्य अपनोदनद्वारा 
सकल-पापक्षयार्थं श्री-सिद्धिविनायक-प्रसादसिद्ध्यर्थं 
यथाशक्ति-ध्यानावाहनादिषोडशोपचारैः श्री-सिद्धिविनायक-पूजां करिष्ये। तदङ्गं कलशपूजां च करिष्ये। 

श्रीविघ्नेश्वराय नमः यथास्थानं प्रतिष्ठापयामि। शोभनार्थे क्षेमाय पुनरागमनाय च।\\
(गणपति प्रसादं शिरसा गृहीत्वा)

\dnsub{आसन-पूजा}
\centerline{पृथिव्या  मेरुपृष्ठ  ऋषिः।  सुतलं  छन्दः।  कूर्मो  देवता॥}
\twolineshloka*
{पृथ्वि  त्वया  धृता  लोका  देवि  त्वं  विष्णुना  धृता}
{त्वं  च  धारय  मां  देवि  पवित्रं  चाऽऽसनं  कुरु}


\dnsub{घण्टापूजा}
\twolineshloka*
{आगमार्थं तु देवानां गमनार्थं तु रक्षसाम्}
{घण्टारवं करोम्यादौ देवताऽऽह्वानकारणम्}


\dnsub{कलशपूजा}
ॐ कलशाय नमः दिव्यगन्धान् धारयामि।\\
ॐ गङ्गायै नमः। ॐ यमुनायै नमः। ॐ गोदावर्यै नमः।  ॐ सरस्वत्यै नमः। ॐ नर्मदायै नमः। ॐ सिन्धवे नमः। ॐ कावेर्यै नमः।\\
ॐ सप्तकोटिमहातीर्थान्यावाहयामि।\\[-0.25ex]

(अथ कलशं स्पृष्ट्वा जपं कुर्यात्) \\
आपो॒ वा इ॒द सर्वं॒ विश्वा॑ भू॒तान्याप॑ प्रा॒णा वा आप॑ प॒शव॒ आपो\-ऽन्न॒मापोऽमृ॑त॒माप॑ स॒म्राडापो॑ वि॒राडाप॑ स्व॒राडाप॒श्\-छन्दा॒स्यापो॒ ज्योती॒ष्यापो॒ यजू॒ष्याप॑ स॒त्यमाप॒ सर्वा॑ दे॒वता॒ आपो॒ भूर्भुव॒ सुव॒राप॒ ओम्॥\\

\twolineshloka* 
{कलशस्य मुखे विष्णुः कण्ठे रुद्रः समाश्रितः}
{मूले तत्र स्थितो ब्रह्मा मध्ये मातृगणाः स्मृताः}
\threelineshloka* 
{कुक्षौ तु सागराः सर्वे सप्तद्वीपा वसुन्धरा}
{ऋग्वेदोऽथ यजुर्वेदः सामवेदोऽप्यथर्वणः}
{अङ्गैश्च सहिताः सर्वे कलशाम्बुसमाश्रिताः}
\twolineshloka* 
{गङ्गे च यमुने चैव गोदावरि सरस्वति}
{नर्मदे सिन्धुकावेरि जलेऽस्मिन् सन्निधिं कुरु}
\twolineshloka*
{सर्वे समुद्राः सरितः तीर्थानि च ह्रदा नदाः}
{आयान्तु देवपूजार्थं दुरितक्षयकारकाः}

\centerline{ॐ भूर्भुवः॒ सुवो॒ भूर्भुवः॒ सुवो॒ भूर्भुवः॒ सुवः॑।}

(इति कलशजलेन सर्वोपकरणानि आत्मानं च प्रोक्ष्य।)


\dnsub{आत्म-पूजा}
ॐ आत्मने नमः, दिव्यगन्धान् धारयामि।
\begin{multicols}{2}
१. ॐ आत्मने नमः\\
२. ॐ अन्तरात्मने नमः\\
३. ॐ योगात्मने नमः\\
४. ॐ जीवात्मने नमः\\
५. ॐ परमात्मने नमः\\
६. ॐ ज्ञानात्मने नमः
\end{multicols}
समस्तोपचारान् समर्पयामि।

\twolineshloka*
{देहो देवालयः प्रोक्तो जीवो देवः सनातनः}
{त्यजेदज्ञाननिर्माल्यं सोऽहं भावेन पूजयेत्}


\begin{minipage}{\linewidth}
\dnsub{पीठ-पूजा}

\begin{multicols}{2}
\begin{enumerate}
\item ॐ आधारशक्त्यै नमः
\item ॐ मूलप्रकृत्यै नमः
\item ॐ आदिकूर्माय नमः 
\item ॐ आदिवराहाय नमः
\item ॐ अनन्ताय नमः
\item ॐ पृथिव्यै नमः
\item ॐ रत्नमण्डपाय नमः
\item ॐ रत्नवेदिकायै नमः
\item ॐ स्वर्णस्तम्भाय नमः
\item ॐ श्वेतच्छत्त्राय नमः
\item ॐ कल्पकवृक्षाय नमः
\item ॐ क्षीरसमुद्राय नमः 
\item ॐ सितचामराभ्यां नमः
\item ॐ योगपीठासनाय नमः
\end{enumerate}
\end{multicols}

\end{minipage}

\dnsub{गुरु ध्यानम्}

\twolineshloka*
{गुरुर्ब्रह्मा गुरुर्विष्णुर्गुरुर्देवो महेश्वरः}
{गुरुः साक्षात् परं ब्रह्म तस्मै श्री गुरवे नमः}


\dnsub{प्राण-प्रतिष्ठा}

असुनीते पुनरिति ऋचं पठित्वा गर्भाधानादिपञ्चदशसंस्कार सिद्धयर्थं पञ्चदशप्रणवावृत्तीः करिष्ये इति सङ्कल्प्य पञ्चदशवारं प्रणवमावर्त्य तच्चक्षुर्देवहितम् इति मन्त्रेण देवस्याज्येन नेत्रोन्मीलनं कृत्वा पञ्चोपचारैः पूजनं कुर्यात्। 
आसनविधिं कृत्वा पुरुषसूक्त-न्यासान् विधाय पूजनमारभेत्॥ 

\sect{षोडशोपचारपूजा}
\renewcommand{\devAya}{श्री-सिद्धिविनायकाय नमः,}
\begin{center}

\twolineshloka*
{करिष्ये गणनाथस्य व्रतं सम्पत्करं शुभम्}
{भक्तानामिष्टवरदं सर्वमङ्गल-कारणम्}

\twolineshloka*
{एकदन्तं शूर्पकर्णं गजवक्त्रं चतुर्भुजम्}
{पाशाङ्कुशधरं देवं ध्यायेत् सिद्धिविनायकम्}

\twolineshloka*
{ध्यायेद् देवं महाकायं तप्तकाञ्चनसन्निभम्}
% वन्दे गजाननं देवं तप्तकाञ्चन सन्निभम्|
{चतुर्भुजं महाकायं सर्वाभरणभूषितम्}

\twolineshloka*
{दन्ताक्षमाला-परशु-पूर्णमोदक-हस्तकम्}
{मोदकासक्त-शुण्डाग्रम् एकदन्तं विनायकम्}
\textbf{अस्मिन् बिम्बे/प्रतिमायां/चित्रपटे श्री-सिद्धिविनायकं ध्यायामि।}

\twolineshloka*
{आवाहयामि विघ्नेश सुरराजार्चितेश्वर}
{अनाथनाथ सर्वज्ञ पूजार्थं गणनायक}

\twolineshloka*
{स॒हस्र॑शीर्‌षा॒ पुरु॑षः। स॒ह॒स्रा॒क्षः स॒हस्र॑पात्}
{स भूमिं॑ वि॒श्वतो॑ वृ॒त्वा। अत्य॑तिष्ठद्दशाङ्गु॒लम्}
\textbf{अस्मिन् बिम्बे/प्रतिमायां/चित्रपटे श्री-सिद्धिविनायकम् आवाहयामि।}
\medskip

\twolineshloka*
{विचित्ररत्नरचितं दिव्यास्तरणसंयुतम्}
{स्वर्णसिंहासनं चारु गृहाण सुरपूजित}

\twolineshloka*
{पुरु॑ष ए॒वेदꣳ सर्वम्᳚। यद्भू॒तं यच्च॒ भव्यम्᳚}
{उ॒तामृ॑त॒त्वस्येशा॑नः। यदन्ने॑नाति॒रोह॑ति}
\textbf{\devAya{} आसनं समर्पयामि।}
\medskip

\twolineshloka*
{सर्वतीर्थसमानीतं पाद्यं गन्धादिसंयुतम्}
{विघ्नराज गृहाणेनदं भगवन् भक्तवत्सल}

\twolineshloka*
{ए॒तावा॑नस्य महि॒मा। अतो॒ ज्यायाꣴश्च॒ पूरु॑षः}
{पादो᳚ऽस्य॒ विश्वा॑ भू॒तानि॑। त्रि॒पाद॑स्या॒मृतं॑ दि॒वि}
\textbf{\devAya{} पाद्यं समर्पयामि।}
\medskip

\twolineshloka*
{अर्घ्यं च फलसंयुक्तं गन्धपुष्पाक्षतैर्युतम्}
{गणाध्यक्ष नमस्तेऽस्तु गृहाण करुणानिधे}

\twolineshloka*
{त्रि॒पादू॒र्ध्व उदै॒त्पुरु॑षः। पादो᳚ऽस्ये॒हाऽऽभ॑वा॒त्पुनः॑}
{ततो॒ विश्व॒ङ्व्य॑क्रामत्। सा॒श॒ना॒न॒श॒ने अ॒भि}
\textbf{\devAya{} अर्घ्यं समर्पयामि।}
\medskip

\twolineshloka*
{विनायक नमस्तुभ्यं त्रिदशैरभिवन्दित}
{गङ्गाहृतेन तोयेन शीघ्रमाचमनं कुरु}

\twolineshloka*
{तस्मा᳚द्वि॒राड॑जायत। वि॒राजो॒ अधि॒ पूरु॑षः}
{स जा॒तो अत्य॑रिच्यत। प॒श्चाद्भूमि॒मथो॑ पु॒रः}
\textbf{\devAya{} आचमनीयं समर्पयामि।}
\medskip

\twolineshloka*
{दध्याज्यमधुसंयुक्तं मधुपर्कं मयाऽऽहृतम्}
{गृहाण सर्वलोकेश गणनाथ नमोऽस्तु ते}

\twolineshloka*
{यत्पुरु॑षेण ह॒विषा᳚। दे॒वा य॒ज्ञमत॑न्वत}
{व॒स॒न्तो अ॑स्याऽऽसी॒दाज्यम्᳚। ग्री॒ष्म इ॒ध्मः श॒रद्ध॒विः}
\textbf{\devAya{} मधुपर्कं समर्पयामि।}
\medskip

\twolineshloka*
{पयो दधि घृतं चैव शर्करामधुसंयुतम्}
{पञ्चामृतं गृहाणेदं स्नानाय गणनायक}

\textbf{\devAya{} पञ्चामृतस्नानम् समर्पयामि।}
\medskip

\twolineshloka*
{गङ्गादिसर्वतीर्थेभ्य आनीतं तोयमुत्तमम्}
{भक्त्या समर्पितं तुभ्यं स्नानायाभीष्टदायक}

\twolineshloka*
{स॒प्तास्या॑ऽऽसन् परि॒धयः॑। त्रिः स॒प्त स॒मिधः॑ कृ॒ताः}
{दे॒वा यद्य॒ज्ञं त॑न्वा॒नाः। अब॑ध्न॒न् पु॑रुषं प॒शुम्}
\textbf{\devAya{} शुद्धोदकस्नानं समर्पयामि। स्नानानन्तरम् आचमनीयं समर्पयामि।}
\medskip

\twolineshloka*
{रक्तवस्त्रयुगं देव दिव्यं काञ्चनसम्भवम्}
{सर्वप्रद गृहाणेदं लम्बोदर हरात्मज}

\twolineshloka*
{तं य॒ज्ञं ब॒र्{}हिषि॒ प्रौक्षन्॑। पुरु॑षं जा॒तम॑ग्र॒तः}
{तेन॑ दे॒वा अय॑जन्त। सा॒ध्या ऋष॑यश्च॒ ये}
\textbf{\devAya{} वस्त्रं समर्पयामि।}
\medskip

\twolineshloka*
{राजतं ब्रह्मसूत्रं च काञ्चनं चोत्तरीयकम्}
{गृहाण चारु सर्वज्ञ भक्तानां वरदो भव}

\twolineshloka*
{तस्मा᳚द्य॒ज्ञाथ्स॑र्व॒हुतः॑। सम्भृ॑तं पृषदा॒ज्यम्}
{प॒शूꣴस्ताꣴश्च॑क्रे वाय॒व्यान्॑। आ॒र॒ण्यान्ग्रा॒म्याश्च॒ ये}
\textbf{\devAya{} यज्ञोपवीतं समर्पयामि।}
\medskip

\twolineshloka*
{उद्यद्भास्करसङ्काशं सन्ध्यावदरुणं प्रभो}
{वीरालङ्करणं दिव्यं सिन्दूरं प्रतिगृह्यताम्}
\textbf{\devAya{} सिन्दूरं समर्पयामि।}
\medskip

\twolineshloka*
{नानाविधानि दिव्यानि नानारत्नोज्ज्वलानि च}
{भूषणानि गृहाणेश पार्वतीप्रियनन्दन}
\textbf{\devAya{} आभरणानि समर्पयामि।}
\medskip

\twolineshloka*
{कस्तूरीरोचनाचन्द्रकुङ्कुमैश्च समन्वितम्}
{विलेपनं सुरश्रेष्ठ चन्दनं प्रतिगृह्यताम्}

\twolineshloka*
{तस्मा᳚द्य॒ज्ञाथ्स॑र्व॒हुतः॑। ऋचः॒ सामा॑नि जज्ञिरे}
{छन्दाꣳसि जज्ञिरे॒ तस्मा᳚त्। यजु॒स्तस्मा॑दजायत}
\textbf{\devAya{} दिव्यपरिमलगन्धान् धारयामि। गन्धस्योपरि हरिद्राकुङ्कुमं समर्पयामि।}
\medskip

\twolineshloka*
{रक्ताक्षतांश्च देवेश गृहाण द्विरदानन}
{ललाटपटले चन्द्रस्तस्योपरि विधार्यताम्}
\textbf{\devAya{} अक्षतान् समर्पयामि।}
\medskip

\twolineshloka*
{माल्यादीनि सुगन्धीनि मालत्यादीनि मे प्रभो}
{मयाऽऽहृतानि पुष्पाणि पूजार्थं प्रतिगृह्यताम्}

\twolineshloka*
{करवीरैर्जातिकुसुमैश्चम्पकैर्बकुलैः शुभैः}
{शतपत्रैश्च कह्लारैरर्चयेद् गणनायकम्}

\twolineshloka*
{तस्मा॒दश्वा॑ अजायन्त। ये के चो॑भ॒याद॑तः}
{गावो॑ ह जज्ञिरे॒ तस्मा᳚त्। तस्मा᳚ज्जा॒ता अ॑जा॒वयः॑}
\textbf{\devAya{} पुष्पाणि समर्पयामि।} पुष्पैः पूजयामि।

\dnsub{अङ्गपूजा}
\begin{longtable}{ll@{~नमः — }l@{~पूजयामि}}
    १. & पार्वतीनन्दनाय & पादौ\\
    २. & गणेशाय & गुल्फौ\\
    ३. & जगद्धात्रे & जङ्घे\\
    ४. & जगद्वल्लभाय & जानुनी\\
    ५. & उमापुत्राय & ऊरू\\
    ६. & विकटाय & कटिं\\
    ७. & गुहाग्रजाय & गुह्यं\\
    ८. & महत्तमाय & मेढ्रं\\
    ९. & नाथाय & नाभिं\\
    १०. & उत्तमाय & उदरं\\
    ११. & विनायकाय & वक्षः\\
    १२. & पाशच्छिदे & पार्श्वौ\\
    १३. & हेरम्बाय & हृदयं\\
    १४. & कपिलाय & कण्ठं\\
    १५. & स्कन्दाग्रजाय & स्कन्धौ\\
    १६. & हरसुताय & हस्तान्\\
    १७. & ब्रह्मचारिणे & बाहून्\\
    १८. & सुमुखाय & मुखं\\
    १९. & एकदन्ताय & दन्तौ\\
    २०. & विघ्ननेत्रे & नेत्रे\\
    २१. & शूर्पकर्णाय & कर्णौ\\
    २२. & फालचन्द्राय & फालं\\
    २३. & नागाभरणाय & नासिकां\\
    २४. & चिरन्तनाय & चुबुकं\\
    २५. & स्थूलौष्ठाय & ओष्ठौ\\
    २६. & गलन्मदाय & गण्डौ\\
    २७. & कपिलाय & कचान्\\
    २८. & शिवप्रियाय & शिरः\\
    २९. & सर्वमङ्गलासुताय & सर्वाण्यङ्गानि\\
\end{longtable}

\dnsub{एकविंशति-पत्र-पूजा}

\begin{longtable}{ll@{~नमः — }l@{-पत्रं समर्पयामि}}
    १. & सुमुखाय & मालती\\
    २. & उमापुत्राय & माची\\
    ३. & हेरम्बाय & बृहती\\
    ४. & लम्बोदराय & बिल्व\\
    ५. & द्विरदाननाय & दूर्वा\\
    ६. & धूमकेतवे & दुर्धूर\\
    ७. & बृहते & बदरी\\
    ८. & अपवर्गदाय & अपामार्ग\\
    ९. & द्वैमातुराय & तुलसी\\
    १०. & चिरन्तनाय & चूत\\
    ११. & कपिलाय & करवीर\\
    १२. & विष्णुस्तुताय & विष्णुक्रान्त\\
    १३. & अमलाय & आमलकी\\
    १४. & महते & मरुवक\\
    १५. & सिन्धूराय & सिन्धूर\\
    १६. & गजाननाय & जाती\\
    १७. & गण्डगलन्मदाय & गण्डली\\
    १८. & शङ्करीप्रियाय & शमी\\
    १९. & भृङ्गराजत्कटाय & भृङ्गराज\\
    २०. & अर्जुनदन्ताय & अर्जुन\\
    २१. & अर्कप्रभाय & अर्क\\
\end{longtable}


\dnsub{एकविंशति-पुष्प-पूजा}
\begin{longtable}{ll@{-गणपतये नमः — }ll@{-पुष्पं समर्पयामि}}
    १. & पञ्चास्य & पुन्नाग&\\
    २. & महा & मन्दार&\\
    ३. & धीर & दाडिमी&\\
    ४. & विष्वक्सेन & वकुल&\\
    ५. & आमोद & अमृणाल&\\
    ६. & प्रमथ & पाटली&\\
    ७. & रुद्र & द्रोण&\\
    ८. & विद्या & दुर्धूर&\\
    ९. & विघ्न & चम्पक&\\
    १०. & दुरित & रसाल&\\
    ११. & कामितार्थ & केतकी&\\
    १२. & सम्मोह & माधवी&\\
    १३. & विष्णु & श्यामक&\\
    १४. & ईश & अर्क&\\
    १५. & गजास्य & कह्लार&\\
    १६. & सर्वसिद्धि & सेवन्तिका&\\
    १७. & वीर & बिल्व&\\
    १८. & कन्दर्प & करवीर&\\
    १९. & उच्छिष्ट & कुन्द&\\
    २०. & ब्रह्म & पारिजात&\\
    २१. & ज्ञान & जाती&\\
\end{longtable}

\dnsub{एकविंशति-दूर्वायुग्म-पूजा}
\begin{longtable}{ll@{~नमः — दूर्वायुग्मं समर्पयामि।}}
    १. & गणाधिपाय \\
    २. & पाशाङ्कुशधराय \\
    ३. & आखुवाहनाय \\
    ४. & विनायकाय \\
    ५. & ईशपुत्राय \\
    ६. & सर्वसिद्धि-प्रदाय \\
    ७. & एकदन्ताय \\
    ८. & इभवक्त्राय \\
    ९. & मूषिकवाहनाय \\
    १०. & कुमारगुरवे \\
    ११. & कपिलवर्णाय \\
    १२. & ब्रह्मचारिणे \\
    १३. & मोदकहस्ताय \\
    १४. & सुरश्रेष्ठाय \\
    १५. & गजनासिकाय \\
    १६. & कपित्थफल-प्रियाय \\
    १७. & गजमुखाय \\
    १८. & सुप्रसन्नाय \\
    १९. & सुराग्रजाय \\
    २०. & उमापुत्राय \\
    २१. & स्कन्दप्रियाय \\
\end{longtable}

\needspace{3em}
\begingroup
\setlength{\columnseprule}{1pt}
\let\chapt\sect
\input{../namavali-manjari/100/Ganapati_108.tex}

\endgroup

\twolineshloka*
{दशाङ्गं गुग्गुलुं धूपं सुगन्धं च मनोहरम्}
{गृहाण सर्वदेवेश उमापुत्र नमोऽस्तु ते} 

\twolineshloka*
{यत्पुरु॑षं॒ व्य॑दधुः। क॒ति॒धा व्य॑कल्पयन्}
{मुखं॒ किम॑स्य॒ कौ बा॒हू। कावू॒रू पादा॑वुच्येते}
\textbf{\devAya{} धूपमाघ्रापयामि।}
\medskip

\twolineshloka*
{सर्वज्ञ सर्वलोकेश त्रैलोक्यतिमिरापह}
{गृहाण मङ्गलं दीपं रुद्रप्रिय नमोऽस्तु ते}

\twolineshloka*
{ब्रा॒ह्म॒णो᳚ऽस्य॒ मुख॑मासीत्। बा॒हू रा॑ज॒न्यः॑ कृ॒तः}
{ऊ॒रू तद॑स्य॒ यद्वैश्यः॑। प॒द्भ्याꣳ शू॒द्रो अ॑जायत}

उद्दी᳚प्यस्व जातवेदोऽप॒घ्नन्निर्ऋ॑तिं॒ मम॑।\\
प॒शूꣴश्च॒ मह्य॒माव॑ह॒ जीव॑नं च॒ दिशो॑ दिश॥\\
मा नो॑ हिꣳसीज्जातवेदो॒ गामश्वं॒ पुरु॑षं॒ जग॑त्।\\
अबि॑भ्र॒दग्न॒ आग॑हि श्रि॒या मा॒ परि॑पातय॥ \\

\textbf{\devAya{} अलङ्कारदीपं सन्दर्शयामि।\\
दीपानन्तरम् आचमनीयं समर्पयामि।}
\medskip

ॐ भूर्भुवः॒ सुवः॑। + ब्र॒ह्मणे॒ स्वाहा᳚।
\twolineshloka*
{च॒न्द्रमा॒ मन॑सो जा॒तः। चक्षोः॒ सूर्यो॑ अजायत}
{मुखा॒दिन्द्र॑श्चा॒ग्निश्च॑। प्रा॒णाद्वा॒युर॑जायत}

\twolineshloka*
{नानाखाद्यमयं दिव्यं नैवेद्यं ते निवेदितम्}
{मया भक्त्या शिवापुत्र गृहाण गणनायक}

\textbf{\devAya{} (	) निवेदयामि, \\}
अमृतापिधानमसि। निवेदनानन्तरम् आचमनीयं समर्पयामि।\medskip

\twolineshloka*
{एलोशीरलवङ्गादिकर्पूरपरिवासितम्}
{प्राशनार्थं कृतं तोयं गृहाण गणनायक}
मध्ये मध्ये पानीयं समर्पयामि। उत्तरापोशनं मुखप्रक्षालनं च समर्पयामि।\medskip

\twolineshloka*
{मलयाचलसम्भूतं कर्पूरेण समन्वितम्}
{करोद्वर्तनकं चारु गृह्यतां जगतः पते} 
करोद्वर्तनम् समर्पयामि।

\twolineshloka*
{बीजपूराम्रपनसखजूरीकदलीफलम्} 
{नारिकेलफलं दिव्यं गृहाण गणनायक} 

\twolineshloka*
{इदं फलं मया देव स्थापितं पुरतस्तव}
{तेन मे सफलावाप्तिर्भवेज्जन्मनि जन्मनि}
\textbf{\devAya{} फलं समर्पयामि।}
\medskip

\twolineshloka*
{एकविंशतिसङ्ख्याकान् मोदकान् घृतपाचितान्} 
{नैवेद्यं सफलं दद्यान्नमस्ते विघ्ननाशिने} 
\textbf{\devAya{} मोदकान् समर्पयामि।}
\medskip

\twolineshloka*
{पूगीफलं महद्दिव्यं नागवल्ल्या दलैर्युतम्}
{कर्पूरैलासमायुक्तं ताम्बूलं प्रतिगृह्यताम्}

\twolineshloka*
{नाभ्या॑ आसीद॒न्तरि॑क्षम्। शी॒र्ष्णो द्यौः सम॑वर्तत}
{प॒द्भ्यां भूमि॒र्दिशः॒ श्रोत्रा᳚त्। तथा॑ लो॒काꣳ अ॑कल्पयन्}
\textbf{\devAya{} कर्पूरताम्बूलं समर्पयामि।}
\medskip

% \twolineshloka*
% {हिरण्यगर्भगर्भस्थं हेमबीजं विभावसोः}
% {अनन्तपुण्यफलदम् अतः शान्तिं प्रयच्छ मे}

% \textbf{\devAya{} दक्षिणां समर्पयामि।}
% \medskip

\twolineshloka*
{वज्रमाणिक्यवैदूर्यमुक्ताविद्रुममण्डितम्}
{पुष्परागसमायुक्तं भूषणं प्रतिगृह्यताम्} 

\textbf{\devAya{} भूषणानि समर्पयामि।}
\medskip

\twolineshloka*
{दूर्वायुग्मं गृहीत्वा तु गन्धपुष्पाक्षतैर्युतम्} 
{पूजयेत्सिद्धिविघ्नेशं प्रत्येकं पूर्वनामभिः} 

\twolineshloka*
{गणाधिप नमस्तेऽस्तु उमापुत्राघनाशन}
{एकदन्तेभवक्रेति तथा मूषकवाहन}

\twolineshloka*
{विनायकेशपुत्रेति सर्वसिद्धिप्रदायक}
{कुमारगुरवे नित्यं पूजनीयः प्रयत्नतः}

इति दूर्वार्पणम्॥

नमो व्रातपतये नमो गणपतये नमः प्रमथपतये नमस्ते अस्तु लम्बोदरायैकदन्ताय विघ्नविनाशिने शिवसुताय श्रीवरदमू‍र्तये॑ नमो॒ नमः॥

\twolineshloka*
{विघ्नेश्वर विशालाक्ष सर्वाभीष्टफलप्रद}
{प्रदक्षिणं करोमि त्वां सर्वान्कामान् प्रयच्छ मे}

\twolineshloka*
{चन्द्रादित्यौ च धरणी विद्युदग्निस्तथैव च}
{त्वमेव सर्वतेजांसि आर्तिक्यं प्रतिगृह्यताम्} 

\twolineshloka*
{वेदा॒हमे॒तं पुरु॑षं म॒हान्तम्᳚। आ॒दि॒त्यव॑र्णं॒ तम॑स॒स्तु पा॒रे}
{सर्वा॑णि रू॒पाणि॑ वि॒चित्य॒ धीरः॑। नामा॑नि कृ॒त्वाऽभि॒वद॒न्॒ यदास्ते᳚}
\textbf{\devAya{} समस्त-अपराध-क्षमापनार्थं कर्पूरनीराजनं दर्शयामि।\\}
कर्पूरनीरजनानन्तरम् आचमनीयं समर्पयामि।\\
रक्षां धारयामि। पुष्पैः पूजयामि।
\medskip

\twolineshloka*
{धा॒ता पु॒रस्ता॒द्यमु॑दाज॒हार॑। श॒क्रः प्रवि॒द्वान् प्र॒दिश॒श्चत॑स्रः}
{तमे॒वं वि॒द्वान॒मृत॑ इ॒ह भ॑वति। नान्यः पन्था॒ अय॑नाय विद्यते}

यो॑ऽपां पुष्पं॒ वेद॑। पुष्प॑वान् प्र॒जावा᳚न् पशु॒मान् भ॑वति।\\
च॒न्द्रमा॒ वा अ॒पां पुष्पम्᳚। पुष्प॑वान् प्र॒जावा᳚न् पशु॒मान् भ॑वति।\\
य ए॒वं वेद॑। यो॑ऽपामा॒यत॑नं॒ वेद॑। आ॒यत॑नवान् भवति।\medskip

ओं᳚ तद्ब्र॒ह्म। ओं᳚ तद्वा॒युः। ओं᳚ तदा॒त्मा।\\
ओं᳚ तथ्स॒त्यम्‌। ओं᳚ तथ्सर्वम्᳚‌। ओं᳚ तत्पुरो॒र्नमः॥\medskip

अन्तश्चरति॑ भूते॒षु॒ गुहायां वि॑श्वमू॒र्तिषु।\\
त्वं यज्ञस्त्वं वषट्कारस्त्वमिन्द्रस्त्वꣳ\\
रुद्रस्त्वं विष्णुस्त्वं ब्रह्म त्वं॑ प्रजा॒पतिः।\\
त्वं त॑दाप॒ आपो॒ ज्योती॒ रसो॒ऽमृतं॒ ब्रह्म॒ भूर्भुवः॒ सुव॒रोम्‌॥\medskip

\twolineshloka*
{यो वेदादौ स्व॑रः प्रो॒क्तो॒ वे॒दान्ते॑ च प्र॒तिष्ठि॑तः}
{तस्य॑ प्र॒कृति॑लीन॒स्य॒ यः॒ परः॑ स म॒हेश्व॑रः}
\textbf{\devAya{} वेदोक्तमन्त्रपुष्पाञ्जलिं समर्पयामि।}
\medskip

\twolineshloka*
{नमस्ते विघ्नसंहत्रे नमस्ते ईप्सितप्रद} 
{नमस्ते देवदेवेश नमस्ते गणनायक} 

\twolineshloka*
{विनायकेशपुत्रस्त्वं गजराज सुरोत्तम}
{देहि मे सकलान् कामान् वन्दे सिद्धिविनायक} 

\twolineshloka*
{स॒प्तास्या॑ऽऽसन् परि॒धयः॑। त्रिः स॒प्त स॒मिधः॑ कृ॒ताः}
{दे॒वा यद्य॒ज्ञं त॑न्वा॒नाः। अब॑ध्न॒न् पु॑रुषं प॒शुम्}
\textbf{\devAya{} नमस्कारान् समर्पयामि।}
\medskip

\twolineshloka*
{य॒ज्ञेन॑ य॒ज्ञम॑यजन्त दे॒वाः। तानि॒ धर्मा॑णि प्रथ॒मान्या॑सन्}
{ते ह॒ नाकं॑ महि॒मानः॑ सचन्ते। यत्र॒ पूर्वे॑ सा॒ध्याः सन्ति॑ दे॒वाः}
\textbf{\devAya{} छत्रचामर-नृत्त-गीत-वाद्यादि समस्तराजोपचारान् समर्पयामि।}
\medskip

\twolineshloka*
{यन्मयाऽऽचरितं देव व्रतमेतत् सुदुर्लभम्}
{गणेश त्वं प्रसन्नः सन् सफलं कुरु सर्वदा}

\twolineshloka*
{विनायक गणेशान सर्वदेवनमस्कृत}
{पार्वतीप्रिय विघ्नेश मम विघ्नान्निवारय} 

\twolineshloka*
{नमो नमो गणेशाय नमस्ते विश्वरूपिणे}
{निर्विघ्नं कुरु मे कामं नमामि त्वां गजानन}

\twolineshloka*
{अगजाननपद्मार्कं गजाननमहर्निशम्}
{अनेकदं तं भक्तानाम् एकदन्तमुपास्महे}

\twolineshloka* 
{विनायक वरं देहि महात्मन् मोदकप्रिय}
{अविघ्नं कुरु मे देव सर्वकार्येषु सर्वदा}

प्रार्थनाः समर्पयामि।

\dnsub{अर्घ्यम्}
\resetShloka
ममोपात्तसमस्तदुरितक्षयद्वारा श्री-सिद्धिविनायक-प्रीत्यर्थं श्री-सिद्धिविनायक-पूजान्ते क्षीरार्घ्यप्रदानं करिष्ये॥

(हस्ते साक्षतपुष्पं क्षीरं गृहीत्वा)

\twolineshloka
{अर्घ्यं गृहाण हेरम्ब वरप्रद विनायक}
{गन्धपुष्पाक्षतैर्युक्तं भक्त्या दत्तं मया प्रभो}
\textbf{\devAya{} इदमर्घ्यम् इदमर्घ्यम् इदमर्घ्यम्।}

\twolineshloka
{नमस्ते भिन्नदन्ताय नमस्ते हरसूनवे}
{इदमर्घ्यं प्रदास्यामि गृहाण गणनायक}
\textbf{\devAya{} इदमर्घ्यम् इदमर्घ्यम् इदमर्घ्यम्।}

\twolineshloka
{नमस्तुभ्यं गणेशाय नमस्ते विघ्ननायक}
{पुनरर्घ्यं प्रदास्यामि गृहाण गणनायक}
\textbf{\devAya{} इदमर्घ्यम् इदमर्घ्यम् इदमर्घ्यम्।}

\dnsub{उपायन-दानम्}
\centerline{आचार्य पूजा}

अद्यपूर्वोक्त-एवङ्गुण-विशेषण-विशिष्टायामस्यां चतुर्थ्यां शुभतिथौ श्री-सिद्धिविनायक-पूजा-फलसिद्ध्यर्थं ब्राह्मणपूजाम् उपायन-दानं च करिष्ये॥ 
श्री-महागणपति-स्वरूपस्य ब्राह्मणस्य इदमासनम्। गन्धादि-सकलाराधनैः स्वर्चितम्॥

\twolineshloka*
{[अथैकविंशतिं गृह्य मोदकान् घृतपाचितान्}
{स्थापयित्वा गणाध्यक्षसमीपे कुरुनन्दन}

\twolineshloka*
{दश विप्राय दातव्याः स्थापयेद् दश आत्मनि}
{एकं गणाधिपे दद्यात् सघृतं मोदकं शुभम्]}

\centerline{वायनमन्त्रः}

\twolineshloka*
{दशानां मोदकानां च फलदक्षिणया युतम्}
{विप्राय फलसिद्धयर्थं वायनं प्रददाम्यहम्} 

\centerline{प्रतिमा-दान-मन्त्रः}

\twolineshloka*
{विनायकस्य प्रतिमां वस्त्रयुग्मेन वेष्टिताम्}
{तुभ्यं सम्प्रददे विप्र प्रीयतां में गजाननः}

\twolineshloka*
{गणेशः प्रतिगृह्णाति गणेशो वै ददाति च}
{गणेशस्तारकोभाभ्यां गणेशाय नमो नमः}

\twolineshloka*
{हिरण्यगर्भगर्भस्थं हेमबीजं विभावसोः}
{अनन्तपुण्यफलदम् अतः शान्तिं प्रयच्छ मे}

श्री-विनायक-चतुर्थी-पुण्यकाले अस्मिन् मया क्रियमाण-श्री-सिद्धिविनायक-पूजायां
यद्देयमुपायनदानं तत्प्रत्यायाम्नार्थं हिरण्यं श्री-सिद्धिविनायक-प्रीतिं
कामयमानः मनसोद्दिष्टाय ब्राह्मणाय सम्प्रददे नमः न मम।

अनया पूजया श्री-सिद्धिविनायकः प्रीयताम्। 

\sect{प्रार्थना}

\input{../stotra-sangrahah/stotras/ganesha/MahaGaneshaPancharatnam.tex}
\input{../stotra-sangrahah/stotras/ganesha/GaneshaBhujangam.tex}
\input{../stotra-sangrahah/stotras/big/VakratundaMahaganapatiSahasranamaStotram.tex}

\sect{अपराध-क्षमापनम्}

\twolineshloka*
{यस्य स्मृत्या च नामोक्त्या तपः पूजाक्रियादिषु} 
{न्यूनं सम्पूर्णतां याति सद्यो वन्दे गजाननम्} 

अनया पूजया श्री-सिद्धिविनायकः प्रीयताम्॥ 

\fourlineindentedshloka*
{कायेन वाचा मनसेन्द्रियैर्वा}
{बुद्‌ध्याऽऽत्मना वा प्रकृतेः स्वभावात्}
{करोमि यद्यत् सकलं परस्मै}
{नारायणायेति समर्पयामि}

ॐ तत्सद्ब्रह्मार्पणमस्तु।

\sect{कथा}

\twolineshloka
{शौनकाद्या ऋषिगणा नैमिषारण्यवासिनः}
{सूतं पौराणिकं श्रेष्ठमिदमूचुर्वचस्तदा}%॥१॥

\uvacha{ऋषय ऊचुः}

\twolineshloka
{निर्विघ्ने तु कार्याणि कथं सिध्यन्ति सूतज}
{अर्थसिद्धिः कथं नॄणां पुत्रसौभाग्यसम्पदः}%॥२॥

\twolineshloka
{दम्पत्योः कलहे चैव बन्धुभेदे तथा नृणाम्}
{उदासीनेषु लोकेषु कथं सुमुखता भवेत्}%॥३॥

\twolineshloka
{विद्यारम्भे तथा नॄणां वाणिज्ये च कृषौ तथा}
{नृपतेः परचक्रं च जयसिद्धिः कथं भवेत्}%॥४॥

\twolineshloka
{कां देवतां नमस्कृत्य कार्यसिद्धिर्भवेन्नृणाम्}
{एतत् समस्तं विस्तार्य ब्रूहि मे सूत पृच्छतः}%॥५॥

\uvacha{सूत उवाच}
\twolineshloka
{सन्नद्धयोः पुरा विप्राः कुरुपाण्डवसेनयोः}
{पृष्टवान् देवकीपुत्रं कुन्तीपुत्रो युधिष्ठिरः}%॥६॥ 

\uvacha{युधिष्ठिर उवाच}

\twolineshloka
{निर्विघ्नेन जयं मह्यं वद त्वां देवकीसुत}
{कां देवतां नमस्कृत्य सम्यग्राज्यं लभेमहि}%॥७॥

\uvacha{कृष्ण उवाच}

\twolineshloka
{पूजयस्व गणाध्यक्षमुमा-मल-समुद्भवम्}
{तस्मिन् सम्पूजिते देवे ध्रुवं राज्यमवाप्स्यसि}%॥८॥

\uvacha{युधिष्ठिर उवाच}

\twolineshloka
{देव केन विधानेन पूजनीयो गणाधिपः}
{पूजितस्तु तिथौ कस्यां सिद्धिदो गणपो भवेत्}%॥९॥

\uvacha{कृष्ण उवाच}

\twolineshloka
{मासि भाद्रपदे शुक्ले चतुर्थ्यां पूजयेन्नृप}
{मासि माघे श्रावणे वा मार्गशीर्षेऽथवा भवेत्}%॥१०॥

\twolineshloka
{गजवक्त्रं तु शुक्लायां चतुर्थ्यां पूजयेन्नृप}
{यदा चोत्पद्यते भक्तिस्तदा पूज्यो गणाधिपः}%॥११॥

\twolineshloka
{प्रातः शुक्लतिलैः स्नात्वा मध्याह्ने पूजयेन्नृप}
{निष्कमात्रसुवर्णेन तदर्धार्धेन वा पुनः}%॥१२॥

\twolineshloka
{स्वशक्त्या गणनाथस्य स्वर्णरौप्यमयाकृतिम्}
{अथवा मृन्मयीं कुर्याद् वित्तशाठ्यं न कारयेत्}%॥१३॥

\twolineshloka
{एकदन्तं शूर्पकर्णं गजवक्त्रं चतुर्भुजम्}
{पाशाङ्कुशधरं देवं ध्यायेत् सिद्धिविनायकम्}%॥१४॥

\twolineshloka
{ध्यात्वा चानेन मन्त्रेण स्नाप्य पञ्चामृतैः पृथक्}
{गणाध्यक्षेति नाम्ना वै गन्धं दद्याच्च भक्तितः}%॥१५॥

\twolineshloka
{आवाहनार्थे पाद्यं च दत्त्वा पश्चात् प्रयत्नतः}
{रक्तवस्त्रयुगं सर्वप्रदं दद्याच्च भक्तितः}%॥१६॥

\twolineshloka
{विनायकेति पुष्पाणि धूपं चोमासुताय च}
{दीपं रुद्रप्रियायेति नैवेद्यं विघ्ननाशिने}%॥१७॥

\twolineshloka
{किञ्चित् सुवर्णं पूजां च ताम्बूलं च समर्पयेत्}
{ततो दूर्वाङ्कुरान् गृह्य विंशति चैकमेव हि}%॥१८॥

\twolineshloka
{पूजनीयः प्रयत्नेन एभिर्नामपदैः पृथक्}
{गणाधिप नमस्तेऽस्तु उमापुत्राघनाशन}%॥१९॥

\twolineshloka
{विनायकेशपुत्रेति सर्वसिद्धिप्रदायक}
{एकदन्तेभवक्रेति तथा मूषकवाहन}%॥२०॥

\twolineshloka
{कुमारगुरवे तुभ्यं पूजनीयः प्रयत्नतः}
{दूर्वायुग्मं गृहीत्वा तु गन्धपुष्पाक्षतैर्युतम्}%॥२१॥

\twolineshloka
{एकैकेन तु नाम्ना वै दत्त्वैकं सर्वनामभिः}
{अथैकविंशतिं गृह्य मोदकान् घृतपाचितान्}%॥२२॥

\twolineshloka
{स्थापयित्वा गणाध्यक्षसमीपे कुरुनन्दन}
{दश विप्राय दातव्याः स्वयं ग्राह्यास्तथा दश}%॥२३॥

\twolineshloka
{एकं गणाधिपे दद्यात् सनैवेद्यं नृपोत्तम}
{विनायकस्य प्रतिमां ब्राह्मणाय निवेदयेत्}%॥२४॥

\twolineshloka
{विनायकस्य प्रतिमां वस्त्रयुग्मेन वेष्टिताम्}
{तुभ्यं सम्प्रददे विप्र प्रीयतां मे गजाननः}%॥२५॥

\twolineshloka
{विनायक गणेश त्वं सर्वदेवनमस्कृत}
{पार्वतीप्रिय विघ्नेश मम विघ्नं विनाशय}%॥२६॥

\twolineshloka
{गणेशः प्रतिगृह्णाति गणेशो वै ददाति च}
{गणेशस्तारकोभाभ्यां गणेशाय नमो नमः}%॥२७॥

\twolineshloka
{कृत्वा नैमित्तिकं कर्म पूजयेदिष्टदेवताम्}
{ब्राह्मणान् भोजयेत् पश्चाद् भुञ्जीयात् तैलवर्जितम्}%॥२८॥

\twolineshloka
{एवं कृते धर्मराज गणनाथस्य पूजने}
{विजयस्ते भवेन्नूनं सत्यं सत्यं मयोदितम्}%॥२९॥

\twolineshloka
{त्रिपुरं हन्तुकामेन पूजितः शूलपाणिना}
{शक्रेण पूजितः पूर्वं वृत्रासुरवधेच्छया}%॥३०॥

\twolineshloka
{अन्वेषयन्त्या भर्तारं पूजितोऽहल्यया पुरा}
{नलस्यान्वेषणार्थाय दमयन्त्या पुराऽऽर्चितः}%॥३१॥

\twolineshloka
{रघुनाथेन तद्वच्च सीतायान्वेषणे पुरा}
{द्रष्टुं सीतां महाभागां वीरेण च हनूमता}%॥३२॥

\twolineshloka
{भगीरथेन तद्वच्च गङ्गामानयता पुरा}
{अमृतोत्पादनार्थाय तथा देवासुरैरपि}%॥३३॥

\twolineshloka
{अमृतं हरता पूर्वं वैनतेयेन पक्षिणा}
{आराधितो गणाध्यक्षो ह्यमृतं च हृतं बलात्}%॥३४॥

\twolineshloka
{रुक्मिणीहेतुकामेन पूजितोऽसौ मया प्रभुः}
{तस्य प्रसादाद्राजेन्द्र रुक्मिणीं प्राप्तवानहम्}%॥३५॥

\twolineshloka
{यदा पूर्वं हि दैत्येन हृतो रुक्मिणिनन्दनः}
{आराधितो मया तद्वद् रुक्मिण्या सहितेन च}%॥३६॥

\twolineshloka
{कुष्ठव्याधियुतेनाथ साम्बेनाऽऽराधितः पुरा}
{जयकामस्तथा शीघ्रं त्वमाराधय शाङ्करिम्}%॥३७॥

\twolineshloka
{विद्याकामो लभेद् विद्यां धनकामो धनं तथा}
{जयं च जयकामस्तु पुत्रार्थी विन्दते सुतान्}%॥३८॥

\twolineshloka
{पतिकामा च भर्तारं सौभाग्यं च सुवासिनी}
{विधवा पूजयित्वा तु वैधव्यं नाप्नुयात्क्वचित्}%॥३९॥

\twolineshloka
{वैष्णव्याद्यासु दीक्षासु आदौ पूज्यो गणाधिपः}
{तस्मिन् सम्पूजिते विष्णुरीशो भानुस्तथा ह्युमा}%॥४०॥ 

\twolineshloka
{हव्यवाहमुखा देवाः पूजिताः स्युर्न संशयः}
{चण्डिकाद्या मातृगणाः परितुष्टा भवन्ति च}%॥४१॥

\twolineshloka
{तस्मिन्सम्पूजिते विप्रा भक्त्या सिद्धिविनायके}
{एवं कृते धर्मराज गणनाथस्य पूजने}%॥४२॥

\twolineshloka
{प्राप्स्यसि त्वं स्वकं राज्यं हत्वा शत्रून् रणाजिरे}
{सिध्यन्ति सर्वकार्याणि नात्र कार्या विचारणा}%॥४३॥

\twolineshloka
{एवमुक्तस्तु कृष्णेन सानुजः पाण्डुनन्दनः}
{पूजयामास देवस्य पुत्रं त्रिपुरघातिनः}%॥४४॥

शत्रुसङ्घं निहत्यासौ प्राप्तवान्राज्यमोजसा।

\uvacha{सूत उवाच}
\onelineshloka
{यः पूजयेन्मन्दभाग्यो गणेशं सिद्धिदायकम्}%॥४५॥

\twolineshloka
{सिध्यन्ति तस्य कार्याणि मनसा चिन्तितान्यपि}
{ख्यातिं गमिष्यते तेन नाम्ना सिद्धिविनायकः}%॥४६॥

\twolineshloka
{य इदं शृणुयान्नित्यं श्रावयेद् वा समाहितः}
{सिध्यन्ति सर्वकार्याणि विनायकप्रसादतः}%॥४७॥  

॥इति सिद्धिविनायकव्रतं भविष्योक्तं सम्पूर्णम्॥

\dnsub{सारभूतः श्लोकः}
“सिंहः प्रसेनम् अवधीत् सिंहो जाम्बवता हतः।\\
सुकुमारक मा रोदीः तव ह्येष स्यमन्तकः॥”\\

\dnsub{स्यमन्तकोपाख्यानपठने प्रमाणवचनानि}
कन्यादित्ये चतुर्थ्यां च शुक्ले चन्द्रस्य दर्शनम्।\\
मिथ्याभिदूषणं कुर्यात् तस्मात् पश्येन्न तं तदा॥\\
तद्दोषशान्तये जाप्यं विष्णुनोक्तं स्यमन्तकम्॥\\
(व्रतचूडामण्यादौ व्रतग्रन्थेषु)\\

ये शृण्वन्ति तवाख्यानं स्यमन्तकमणीयकम्॥ \\
चन्द्रस्य चरितं सर्वं तेषां दोषो न जायते॥१३२॥\\
(अत्रैव अग्रे उपाख्याने)\\

\sect{स्यमन्तकोपाख्यानम्}

\dnsub{सारभूतः श्लोकः}
“सिंहः प्रसेनम् अवधीत् सिंहो जाम्बवता हतः।\\
सुकुमारक मा रोदीः तव ह्येष स्यमन्तकः॥”\\

\dnsub{स्यमन्तकोपाख्यानपठने प्रमाणवचनानि}
कन्यादित्ये चतुर्थ्यां च शुक्ले चन्द्रस्य दर्शनम्।\\
मिथ्याभिदूषणं कुर्यात् तस्मात् पश्येन्न तं तदा॥\\
तद्दोषशान्तये जाप्यं विष्णुनोक्तं स्यमन्तकम्॥\\
(व्रतचूडामण्यादौ व्रतग्रन्थेषु)\\

ये शृण्वन्ति तवाख्यानं स्यमन्तकमणीयकम्॥ \\
चन्द्रस्य चरितं सर्वं तेषां दोषो न जायते॥१३२॥\\
(अत्रैव अग्रे उपाख्याने)\\

\uvacha{नन्दिकेश्वर उवाच}

\twolineshloka
{शृणुष्वैकाग्रचित्तः सन् व्रतं गाणेश्वरं महत्}
{चतुर्थ्यां शुक्लपक्षे तु सदा कार्यं प्रयत्नतः}%॥१॥

\twolineshloka
{सनत्कुमार योगीन्द्र यदीच्छेच्छुभमात्मनः}
{नारी वा पुरुषो वाऽपि यः कुर्याद् विधिवद् व्रतम्}%॥२॥

\twolineshloka
{मोचयत्याशु विप्रेन्द्र सङ्कष्टाद् व्रतिनं हि तत्}
{अपवादहरं चैव सर्वविघ्नप्रणाशनम्}%॥३॥

\twolineshloka
{कान्तारे विषमे वाऽपि रणे राजकुलेऽथवा}
{सर्वसिद्धिकरं विद्धि व्रतानामुत्तमं व्रतम्}%॥४॥

\twolineshloka
{गजाननप्रियं चाथ त्रिषु लोकेषु विश्रुतम्}
{अतो न विद्यते ब्रह्मन् सर्वसङ्कष्टनाशनम्}%॥५॥

\uvacha{सनत्कुमार उवाच}

\twolineshloka
{केन चादौ पुरा चीर्णं मर्त्यलोकं कथं गतम्}
{एतत्समस्तं विस्तार्य ब्रूहि गाणेश्वरं व्रतम्}%॥६॥

\uvacha{नन्दिकेश्वर उवाच}
\twolineshloka
{चक्रे व्रतं जगन्नाथो वासुदेवः प्रतापवान्}
{आदिष्टं नारदेनैव वृथालाञ्छनमुक्तये}%॥७॥

\uvacha{सनत्कुमार उवाच}
\twolineshloka
{षड्गुणैश्वर्य-सम्पन्नः सृष्टिसंहार-कारकः} 
{वासुदेवो जगद्व्यापी प्राप्तवान् लाञ्छनं कथम्}%॥८॥

एतदाश्चर्यमाख्यानं ब्रूहि त्वं नन्दिकेश्वर।

\uvacha{नन्दिकेश्वर उवाच}
\onelineshloka{भूमिभारनिवृत्त्यर्थं वसुदेवसुतावुभौ}%॥९॥

\twolineshloka
{रामकृष्णौ समुत्पन्नौ पद्मनाभ-फणीश्वरौ}
{जरासन्धभयात् कृष्णो द्वारकां समकल्पयत्}%॥१०॥

\twolineshloka
{विश्वकर्माणमाहूय पुरीं हाटकनिर्मिताम्}
{तत्र षोडशसाहस्रं स्त्रीणां चैव शताधिकम्}%॥११॥

\twolineshloka
{भवनानि मनोज्ञानि तेषां मध्ये व्यकल्पयत्}
{पारिजाततरुं मध्ये तासां भोगाय कल्पयत्}%॥१२॥

\twolineshloka
{यादवानां गृहास्तत्र षट् पञ्चाशच्च कोटयः}
{अन्येऽपि बहवो लोका वसन्ति विगतज्वराः}%॥१३॥

\twolineshloka
{यत् किञ्चित् त्रिषु लोकेषु सुन्दरं तत्र दृश्यते}
{सत्राजितप्रसेनाख्यौ पुत्रावुग्रस्य विश्रुतौ}%॥१४॥

\twolineshloka
{अम्भोधितीरमासाद्य तन्मनस्कतया च सः}
{सत्राजितस्तपस्तेपे सूर्यमुद्दिश्य बुद्धिमान्}%॥१५॥

\twolineshloka
{व्रतं निरशनं गृह्य सूर्यसम्बद्धलोचनः}
{ततः प्रसन्नो भगवान् सत्राजितपुरः स्थितः}%॥१६॥

\twolineshloka
{सत्राजितोऽपि तुष्टाव दृष्ट्वा देवं दिवाकरम्}
{तेजोराशे नमस्तेऽस्तु नमस्ते सर्वतोमुख}%॥१७॥

\twolineshloka
{विश्वव्यापिन् नमस्तेऽस्तु नमस्ते विश्वरूपिणे}
{काश्यपेय नमस्तेऽस्तु हरिदश्व नमोऽस्तु ते}%॥१८॥

\twolineshloka
{ग्रहराज नमस्तेऽस्तु नमस्ते चण्डरोचिषे}
{वेदत्रय नमस्तेऽस्तु सर्वदेव नमोऽस्तु ते}%॥१९॥

\twolineshloka
{प्रसीद पाहि देवेश सुदृष्ट्या मां दिवाकर}
{इत्थं संस्तूयमानोऽसौ देवदेवो दिवाकरः}%॥२०॥

{स्निग्धगम्भीरमधुरं सत्राजितमुवाच ह।}

\uvacha{सूर्य उवाच}

\onelineshloka{वरं ब्रूहि प्रदास्यामि यत् ते मनसि वर्तते}%॥२१॥

{सत्राजित महाभाग तुष्टोऽहं तव निश्चयात्।}

\uvacha{सत्राजित उवाच}

\onelineshloka
{स्यमन्तकमणिं देहि यदि तुष्टोऽसि भास्कर}%॥२२॥

{ददौ तस्य च तद् रत्नं स्वकण्ठादवतार्य सः।}

\uvacha{भास्कर उवाच}
\onelineshloka
{भाराष्टकं शातकुम्भं स्रवतेऽसौ महामणिः}%॥२३॥

\threelineshloka
{शुचिष्मता सदा धार्यं रत्नमेतन्महोत्तमम्}
{सत्राजित क्षणेनैतदशुचिं हन्ति मानवम्}
{इत्युक्त्वाऽन्तर्दधे देवस्तेजोराशिर्दिवाकरः}%॥२४॥

\fourlineindentedshloka
{तत्कण्ठरत्नज्वलमानरूपी}
{पुरीं स कृष्णस्य विवेश सत्वरम्} 
{दृष्ट्वा तु लोका मनसा दिवाकरं}
{सञ्चिन्तयन्तो हि विमुष्टदृष्टयः}%॥२५॥

\fourlineindentedshloka
{समागतोऽयं हरिदश्वदीधिति-}
{र्जनार्दनं द्रष्टुमसंशयेन} 
{नायं सहस्रांशुरितीह लोकाः}
{सत्राजितोऽयं मणिकण्ठभास्वान्}%॥२६॥

\twolineshloka
{स्यमन्तकं महारत्नं दृष्ट्वा तत्कण्ठमण्डले}
{स्पृहां चक्रे जगन्नाथो न जहार मणिं तु सः}%॥२७॥

\twolineshloka
{सत्राजितो जातभयो याचयिष्यति मां हरिः}
{प्रसेनाय ददौ भ्रात्रे धार्योऽयं शुचिना त्वया}%॥२८॥

\twolineshloka
{एकदा कण्ठदेशेऽसौ क्षिप्त्वा तं मणिमुत्तमम्}
{मृगया क्रीडनार्थाय ययौ कृष्णेन संयुतः}%॥२९॥

\twolineshloka
{अश्वारूढोऽशुचिश्चासौ हतः सिंहेन तत्क्षणात्}
{रत्नमादाय सिंहोऽपि गच्छन् जाम्बवता हतः}%॥३०॥

\twolineshloka
{नीत्वा स विवरे रत्नं ददौ पुत्राय जाम्बवान्}
{पुरीं विवेश कृष्णोऽपि स्वकैः सर्वैः समावृतः}%॥३१॥

\twolineshloka
{प्रसेनोऽद्यापि नाऽऽयाति हतः कृष्णेन निश्चितम्}
{मणिलोभेन हा कष्टं बान्धवः पापिना हतः}%॥३२॥

\twolineshloka
{द्वारकावासिनः सर्वे जना ऊचुः परस्परम्}
{वृथापवादसन्तप्तः कृष्णोऽपि निरगाच्छनैः}%॥३३॥

\twolineshloka
{सहैव तैर्गतोऽरण्यं दृष्ट्वा सिंहेन पातितम्}
{प्रसेनं वाहनयुतं तत्पदानुचरः शनैः}%॥३४॥

\twolineshloka
{ऋक्षेण निहतं दृष्ट्वा कृष्णश्चर्क्षबिलं गतः}
{विवेश योजनशतमन्धकारं स्वतेजसा}%॥३५॥

\twolineshloka
{निवारयन् ददर्शाग्रे प्रासादं बद्धभूमिकम्}
{तं कुमारं जाम्बवतो दोलायाममितद्युतिम्}%॥३६॥

\twolineshloka
{माणिक्यं लम्बमानं च ददर्श भगवान् हरिः}
{रूपयौवनसम्पन्नां कन्यां जाम्बवतीं पुनः}%॥३७॥

\threelineshloka
{दोलां दोलयमानां च ददर्श कमलेक्षणः}
{महान्तं विस्मयं चक्रे दृष्ट्वा तां चारुहासिनीम्}
{दोलां दोलयमाना सा जगौ गीतमिदं मुहुः}%॥३८॥

\begingroup
\bfseries
\twolineshloka
{सिंहः प्रसेनमवधीत् सिंहो जाम्बवता हतः}
{सुकुमारक मा रोदीस्तव ह्येष स्यमन्तकः}%॥३९॥
\endgroup

\twolineshloka
{मदनज्वरदाहार्ता दृष्ट्वा तं कमलेक्षणम्}
{उवाच ललितं बाला गम्यतां गम्यतामिति}%॥४०॥

\twolineshloka
{रत्नं गृहीत्वा वेगेन यावच्छेते तु जाम्बवान्}
{इत्याकर्ण्य वचः शौरिः शङ्खं दध्मौ प्रतापवान्}%॥४१॥

\twolineshloka
{आकर्ण्य सहसोत्थाय युयुधे ऋक्षराट् ततः}
{तयोर्युद्धमभूद्घोरं हरिजाम्बवतोस्तदा}%॥४२॥

\twolineshloka
{द्वारकावासिनः सर्वे गतास्ते सप्तमे दिने}
{मृतः कृष्णो भक्षितो वा निःसन्दिग्धं विचार्य च}%॥४३॥

\twolineshloka
{परलोकक्रियां चक्रुः परेतस्य तु ते तदा}
{एकविंशद्दिनं यावद् बाहुप्रहरणो विभुः}%॥४४॥

\twolineshloka
{युयुधे तेन ऋक्षेण युद्धकर्मणि तोषितः}
{जाम्बवान् प्राक्तनं स्मृत्वा दृष्ट्वा देवबलं महत्}%॥४५॥

\uvacha{जाम्बवानुवाच}

\twolineshloka
{अजेयोऽहं सुरैः सर्वैर्यक्षराक्षसदानवैः}
{त्वया जितोऽहं देवेश देवस्त्वमसि निश्चितम्}%॥४६॥

\twolineshloka
{जाने त्वां वैष्णवं तेजो नान्यथा बलमीदृशम्}
{इति प्रसाद्य देवेशं ददौ माणिक्यमुत्तमम्}%॥४७॥

\twolineshloka
{सुतां जाम्बवतीं नाम भार्यार्थं वरवर्णिनीम्}
{पाणिं वै ग्राहयामास देवदेवं च जाम्बवान्}%॥४८॥

\twolineshloka
{मणिमादाय देवोऽपि जाम्बवत्याऽपि संयुतः}
{तद्वृत्तान्तं समाचष्टे द्वारकावासिनां स्वयम्}%॥४९॥

\twolineshloka
{सत्राजितस्य माणिक्यं दत्तवान् संसदि स्थितः}
{मिथ्यापवादसंशुद्धिं प्राप्तवान् मधुसूदनः}%॥५०॥

\twolineshloka
{सत्राजितोऽपि सन्त्रस्तः कृष्णाय प्रददौ सुताम्}
{सत्यभामां महाबुद्धिस्तदा सर्वगुणान्विताम्}%॥५१॥

\twolineshloka
{शतधन्वाक्रूरमुखा यादवा दुष्टमानसाः}
{सत्राजितेन ते वैरं चक्रू रत्नाभिलाषिणः}%॥५२॥

\twolineshloka
{दुरात्मा शतधन्वाऽपि गते कृष्णे च कुत्रचित्}
{सत्राजितं निहत्याशु मणिं जग्राह पापधीः}%॥५३॥

\twolineshloka
{कृष्णस्य पुरतः सत्या समाचष्टे विचेष्टितम्}
{अन्तर्हृष्टो बहिःकोपी कृष्णः कपटनायकः}%॥५४॥

\twolineshloka
{बलदेवपुरो वाक्यमुवाच धरणीधरः}
{हत्वा सत्राजितं दुष्टो मणिमादाय गच्छति}%॥५५॥

\twolineshloka
{निहत्य शतधन्वानं गृह्णीमो रत्नमावयोः}
{मम भोग्यं च तद् रत्नं भविष्यति सुनिश्चितम्}%॥५६॥

\twolineshloka
{एतच्छ्रुत्वा  भयत्रस्तः शतधन्वाऽपि यादवः}
{आहूयाक्रूरनामानं माणिक्यं प्रददौ च सः}%॥५७॥

\twolineshloka
{आरुह्य वडवां वेगान्निर्गतो दक्षिणां दिशम्}
{रथस्थावनुगच्छेतां तदा रामजनार्दनौ}%॥५८॥

\twolineshloka
{शतयोजनमात्रेण ममार वडवा तदा}
{पलायमानो निहतः पदातिस्तु पदातिना}%॥५९॥

\twolineshloka
{रथस्थे बलदेवे तु हरिणा रत्नलोभतः}
{न दृष्टं तत्र तद्रत्नं बलदेवपुरोऽवदत्}%॥६०॥

\twolineshloka
{तदाकर्ण्य महारोषादुवाच वचनं बली}
{कपटी त्वं सदा कृष्ण लोभी पापी सुनिश्चितम्}%॥६१॥

\twolineshloka
{अर्थाय स्वजनं हंसि कस्त्वां बन्धुः समाश्रयेत्}
{अनेकशपथैः कृष्णो बलदेवं प्रसादयत्}%॥६२॥

\twolineshloka
{सोऽपि धिक् कष्टमित्युक्त्वा ययौ वैदर्भमण्डलम्}
{कृष्णोऽपि रथमारुह्य द्वारकां प्रययौ पुनः}%॥६३॥

\twolineshloka
{तथैवोचुर्जनाः सर्वे न साधीयानयं हरिः}
{निष्कासितो रत्नलोभाज्ज्येष्ठो भ्राता बलो बली}%॥६४॥

\twolineshloka
{तच्छ्रुत्वा दीनवदनः पापीयानिव संस्थितः}
{वृथाभिशापात् सन्तप्तो बभूव स जगत्पतिः}%॥६५॥

\twolineshloka
{अक्रूरोऽपि विनिष्क्रम्य तीर्थयात्रानिमित्ततः}
{काशीं गत्वा सुखेनासौ यजन् यज्ञपतिं प्रभुम्}%॥६६॥

\twolineshloka
{तोषमुत्पादयामास तेन द्रव्येण बुद्धिमान्}
{सुरालयगृहैश्चित्रैर्नगरं समकल्पयत्}%॥६७॥

\twolineshloka
{न दुर्भिक्षं न वै रोग ईतयो न च विड्वरम्}
{शुचिना धार्यते यत्र मणिः सूर्यस्य निश्चितम्}%॥६८॥

\twolineshloka
{जानन्नपि हि तत् सर्वं मानुषं भावमाश्रितः}
{लोकाचारं तथा मायामज्ञानं च समाश्रितः}%॥६९॥

\twolineshloka
{बन्धुवैरं समुत्पन्नं लाञ्छनं समुपस्थितम्}
{वृथापवादबहुलं जायमानं कथं सहे}%॥७०॥

\twolineshloka
{इति चिन्तातुरं कृष्णं नारदः समुपस्थितः}
{गृहीत्वा तत्कृतां पूजां सुखासीनस्ततोऽब्रवीत्}%॥७१॥

\uvacha{नारद उवाच}

\twolineshloka
{किमर्थं खिद्यसे देव किं वा ते शोककारणम्}
{यथावृत्तं समाचष्टे नारदाय च केशवः}%॥७२॥

\uvacha{नारद उवाच}

\twolineshloka
{जानामि कारणं देव यदर्थं लाञ्छनं तव}
{त्वया भाद्रपदे शुक्लचतुर्थ्यां चन्द्रदर्शनम्}%॥७३॥

{कृतं तेन समुत्पन्नं लाञ्छनं तु वृथैव हि।}

\uvacha{श्रीकृष्ण उवाच}

\onelineshloka
{वद नारद मे शीघ्रं को दोषश्चन्द्रदर्शने}%॥७४}

{किमर्थं तु द्वितीयायां तस्य कुर्वन्ति दर्शनम्।}

\uvacha{नारद उवाच}

\onelineshloka
{गणनाथेन संशप्तश्चन्द्रमा रूपगर्वतः}%॥७५॥

{त्वद्दर्शने नराणां हि वृथानिन्दा भविष्यति।}

\uvacha{श्रीकृष्ण उवाच}
\onelineshloka
{किमर्थं गणनाथेन शप्तश्चन्द्रः सुधामयः}%॥७६॥

{इदमाख्यानकं श्रेष्ठं यथावद् वक्तुमर्हसि।} 

\uvacha{नारद उवाच}
\onelineshloka
{गणानामाधिपत्ये च रुद्रेण विहितः पुरा}%॥७७॥

\twolineshloka
{अणिमा महिमा चैव लघिमा गरिमा तथा}
{प्राप्तिः प्राकाम्यमीशित्वं वशित्वं चाष्टसिद्धयः}%॥७८॥

\twolineshloka
{भार्यार्थं प्रददौ देवो गणेशस्य प्रजापतिः}
{पूजयित्वा गणाध्यक्षं स्तुतिं कर्तुं प्रचक्रमे}%॥७९॥

\uvacha{ब्रह्मोवाच}

\twolineshloka
{गजवक्त्र गणाध्यक्ष लम्बोदर वरप्रद}
{विघ्नाधीश्वर देवेश सृष्टिसंहारकारक}%॥८०॥

\twolineshloka
{यः पूजयेद् गणाध्यक्षं मोदकाद्यैः प्रयत्नतः}
{तस्य प्रजायते सिद्धिर्निविघ्नेन न संशयः}%॥८१॥

\twolineshloka
{असम्पूज्य गणाध्यक्षं ये वाञ्छन्ति सुरासुराः}
{न तेषां जायते सिद्धिः कल्पकोटिशतैरपि}%॥८२॥

\twolineshloka
{त्वद्भक्त्या तु गणाध्यक्ष विष्णुः पालयते सदा}
{रुद्रोऽपि संहरत्याशु त्वद्भक्त्यैव करोम्यहम्}%॥८३॥

\twolineshloka
{इत्थं संस्तूयमानोऽसौ देवदेवो गजाननः}
{उवाच परमप्रीतो ब्रह्माणं जगतां पतिम्}%॥८४॥

\uvacha{श्रीगणेश उवाच}

{वरं ब्रूहि प्रदास्यामि यत् ते मनसि वर्तते।}

\uvacha{ब्रह्मोवाच}
\onelineshloka
{क्रियमाणस्य मे सृष्टिर्निविघ्नं जायतां प्रभो}%॥८५॥

\twolineshloka
{एवमस्त्विति देवोऽसौ गृहीत्वा मोदकान् करे}
{सत्यलोकात् समागच्छन् स्वेच्छया गगने शनैः}%॥८६॥

\twolineshloka
{चन्द्रलोकं समासाद्य चलितो गणनायकः}
{उपहासं तदा चक्रे सोमो रूपमदान्वितः}%॥८७॥

\twolineshloka
{तं दृष्ट्वा कोपताम्राक्षो गणनाथः शशाप ह}
{दर्शनीयः सुरूपोऽहं सुन्दरश्चाहमित्यथ}%॥८८॥

\twolineshloka
{गर्वितोऽसि शशाङ्क त्वं फलं प्राप्स्यसि सत्वरम्}
{अद्यप्रभृति लोकास्त्वां न हि पश्यन्ति पापिनम्}%॥८९॥

\twolineshloka
{ये पश्यन्ति प्रमादेन त्वां नरा मृगलाञ्छनम्}
{मिथ्याभिशापसंयुक्ता भविष्यन्तीह ते ध्रुवम्}%॥९०॥

\twolineshloka
{हाहाकारो महाञ्जातः श्रुत्वा शापं च भीषणम्}
{अत्यन्तं म्लानवदनश्चन्द्रो जलमथाविशत्}%॥९१॥

\twolineshloka
{कुमुदं कौमुदीनाथः स्थितस्तत्र कृतालयः}
{ततो देवर्षिगन्धर्वा निराशा दीनमानसाः}%॥९२॥

\twolineshloka
{तुरासाहं पुरोधाय जग्मुस्ते तं पितामहम्}
{देवं शशंसुश्चन्द्रस्य गणेशस्य च चेष्टितम्}%॥९३॥

\twolineshloka
{दत्तः शापो गणेशेन कथयामासुरादरात्}
{विचार्य भगवान् ब्रह्मा तान् सुरानिदमब्रवीत्}%॥९४॥

\twolineshloka
{गणेशशापो देवेन्द्र शक्यते केन वाऽन्यथा}
{कर्तुं रुद्रेण न मया विष्णुना चापि निश्चितम्}%॥९५॥

\twolineshloka
{तमेव देवदेवेशं व्रजध्वं शरणं सुराः}
{स एव शापमोक्षं च करिष्यति न संशयः}%॥९६॥

\uvacha{देवा ऊचुः}

\twolineshloka
{केनोपायेन वरदो गजवक्त्रो गणेश्वरः}
{पितामह महाप्राज्ञ तदस्माकं वद प्रभो}%॥९७॥

\uvacha{पितामह उवाच}

\twolineshloka
{चतुर्थ्यां देवदेवोऽसौ पूजनीयः प्रयत्नतः}
{कृष्णपक्षे विशेषेण नक्तं कुर्याच्च तद् व्रतम्}%॥९८॥

\twolineshloka
{अपूपैर्घृतसंयुक्तैर्मोदकैः परितोषयेत्}
{मधुरान्नं हविष्यं च स्वयं भुञ्जीत वाग्यतः}%॥९९॥

\twolineshloka
{स्वर्णरूपं गणेशस्य दातव्यं द्विजसत्तम}
{शक्त्या च दक्षिणां दद्याद् वित्तशाठ्यं न कारयेत्}%॥१००॥

\twolineshloka
{एवं श्रुत्वा च तैः सर्वैर्गीष्पतिः प्रेषितस्तदा}
{स गत्वा कथयामास चन्द्राय ब्रह्मणोदितम्}%॥१०१॥

\twolineshloka
{व्रतं चक्रे ततश्चन्द्रो यथोक्तं ब्रह्मणा पुरा}
{आविर्बभूव भगवान् गणेशो व्रततोषितः}%॥१०२॥

\fourlineindentedshloka
{तं क्रीडमानं गणनायकं च}
{तुष्टाव दृष्ट्वा तु कलानिधानः}
{त्वं कारणं कारणकारणानां}
{वेत्तासि वेद्यं च विभो प्रसीद}%॥१०३॥

\fourlineindentedshloka
{प्रसीद देवेश जगन्निवास}
{गणेश लम्बोदर वक्रतुण्ड}
{विरिञ्चिनारायणपूज्यमान}
{क्षमस्व मे गर्वकृतं च हास्यम्}%॥१०४॥

\fourlineindentedshloka
{ये त्वामसम्पूज्य गणेश नूनं}
{वाच्छन्ति मूढाः स्वकृतार्थसिद्धिम्}
{ते दैवनष्टा निभृतं च लोके}
{ज्ञातो मया ते सकलः प्रभावः}%॥१०५॥

\fourlineindentedshloka
{ये चाप्युदासीनतरास्तु पापाः}
{ते यान्ति वासं नरके सदैव}
{हेरम्ब लम्बोदर मे क्षमस्व}
{दुश्चेष्टितं तत् करुणासमुद्र}%॥१०६॥

\twolineshloka
{एवं संस्तूयमानोऽसौ चन्द्रेणाह गजाननः}
{तुष्टोऽहं तव दास्यामि वरं ब्रूहि निशाकर}%॥१०७॥

\uvacha{चन्द्र उवाच}

\twolineshloka
{लोकानां दर्शनीयोऽहं भवामि पुनरेव हि}
{विशापोऽहं भविष्यामि त्वत्प्रसादाद् गणेश्वर}%॥१०८॥

\uvacha{गणेश उवाच}

\twolineshloka
{वरमन्यं प्रदास्यामि नैतद् देयं मया तव}
{ततो ब्रह्मादयः सर्वे समाजग्मुर्भयार्दिताः}%॥१०९॥

\twolineshloka
{विशापं कुरु देवेश प्रार्थयामो वयं तव}
{विशापमकरोच्चन्द्रं कमलासनगौरवात्}%॥११०॥

\twolineshloka
{भाद्रशुक्लचतुर्थ्यां तु ये पश्यन्ति सदैव हि}
{मिथ्यापवादमावर्षं प्राप्स्यन्तीह न संशयः}%॥१११॥

\twolineshloka
{मासादौ पूर्वमेव त्वां ये पश्यन्ति सदा जनाः}
{भद्रायां शुक्लपक्षस्य तेषां दोषो न जायते}%॥११२॥

\twolineshloka
{तदाप्रभृति लोकोऽयं द्वितीयायां कृतादरः}
{पुनरेव तु पप्रच्छ कलावान् गणनायकम्}%॥११३॥

{केनोपायेन देवेश तुष्टो भवसि तद्वद।}

\uvacha{गणेश उवाच}
\onelineshloka{यश्च कृष्णचतुर्थ्यां तु मोदकाद्यैः प्रपूज्य माम्}%॥११४॥

\twolineshloka
{रोहिण्या सहितं त्वां च समभ्यर्च्यार्घ्यदानतः}
{यथाशक्त्या च मद्रूपं स्वर्णेन परिकल्पितम्}%॥११५॥

\twolineshloka
{दत्त्वा द्विजाय भुञ्जीत कथां श्रुत्वा विधानतः}
{सदा तस्य करिष्यामि सङ्कष्टस्य निवारणम्}%॥११६॥

\twolineshloka
{भाद्रशुक्लचतुर्थ्यां तु मृन्मयी प्रतिमा शुभा}
{हेमाभावे तु कर्तव्या नानापुष्पैः प्रपूज्य माम्}%॥११७॥

\twolineshloka
{ब्राह्मणान् भोजयेत् पश्चाज्जागरं च विशेषतः}
{स्थापयेदव्रणं कुम्भं धान्यस्योपरि शोभितम्}%॥११८॥

\twolineshloka
{यथाशक्त्या च मद्रूपं शातकुम्भेन निर्मितम्}
{वस्त्रद्वयसमाच्छन्नं मोदकाद्यैः प्रपूज्य माम्}%॥११९॥

\twolineshloka
{रक्ताम्बरधरो मर्त्यो ब्रह्मचर्यव्रतः शुचिः}
{रोहिणीसहितं त्वां च पूजयेत् स्थाप्य मत्पुरः}%॥१२०॥

\twolineshloka
{रजतस्य तु रूपं ते कृत्वा शक्त्या विनिर्मितम्}
{वस्त्रं शिवप्रियायेति उपवस्त्रं गणाधिपे}%॥१२१॥

\twolineshloka
{गन्धं लम्बोदरायेति पुष्पं सिद्धिप्रदायके}
{धूपं गजमुखायेति दीपं मूषकवाहने}%॥१२२॥

\twolineshloka
{विघ्ननाथाय नैवेद्यं फलं सर्वार्थसिद्धिदे}
{ताम्बूलं कामरूपाय दक्षिणां धनदाय च}%॥१२३॥

\twolineshloka
{इक्षुदण्डैर्मोदकैश्च होमं कुर्याच्च नामभिः}
{विसर्जनं ततः कुर्यात् सर्वसिद्धिप्रदायकम्}%॥१२४॥

\twolineshloka
{एवं सम्पूज्य विघ्नेशं कथां श्रुत्वा विधानतः}
{मन्त्रेणानेन तत् सर्वं ब्राह्मणाय निवेदयेत्}%॥१२५॥

\twolineshloka
{दानेनानेन देवेश प्रीतो भव गणेश्वर}
{सर्वत्र सर्वदा देव निर्विघ्नं कुरु सर्वदा}%॥१२६॥

\twolineshloka
{मानोन्नतिं च राज्यं च पुत्रपौत्रान् प्रदेहि मे}
{गाश्च धान्यं च वासांसि दद्यात् सर्वं स्वशक्तितः}%॥१२७॥

\twolineshloka
{दत्त्वा तु ब्राह्मणे सर्वं स्वयं भुञ्जीत वाग्यतः}
{मोदकापूपमधुरं लवणक्षारवर्जितम्}%॥१२८॥

\twolineshloka
{एवं करोति यश्चन्द्र तस्याहं सर्वदा जयम्}
{सिद्धिं च धनधान्ये च दादामि विपुलां प्रजाम्}%॥१२९॥

\twolineshloka
{इत्युक्त्वान्तर्दधे देवो विघ्नराजो विनायकः}
{तद् व्रतं कुरु कृष्ण त्वं ततः सिद्धिमवाप्स्यसि}%॥१३०॥

\twolineshloka
{नारदेनैवमुक्तस्तु व्रतं चक्रे हरिः स्वयम्}
{मिथ्यापवादं निर्मृज्य ततः कृष्णोऽभवच्छुचिः}%॥१३१॥

\twolineshloka
{ये शृण्वन्ति तवाख्यानं स्यमन्तकमणीयकम्}
{चन्द्रस्य चरितं सर्वं तेषां दोषो न जायते}%॥१३२॥

\twolineshloka
{भाद्रशुक्लचतुर्थ्यां तु क्वचिच्चन्द्रस्य दर्शनम्}
{जातं तत्परिहारार्थं श्रोतव्यं सर्वमेव हि}%॥१३३॥

\threelineshloka
{यदा यदा मनःकष्टं सन्देह उपजायते} 
{तदा तदा च श्रोतव्यमाख्यानं कष्टनाशनम्}
{एवमुक्त्वा गतो देवो गणेशः कृष्णतोषितः}%॥१३४॥

\fourlineindentedshloka
{यदा यदा पश्यति कार्यमुत्थितं}
{नारी नरश्चाथ करोति तद् व्रतम्}
{सिध्यन्ति कार्याणि मनेप्सितानि}
{किं दुर्लभं विघ्नहरे प्रसन्ने}%॥१३५॥

\end{center}

॥इति श्रीस्कन्दपुराणे नन्दिकेश्वरसनत्कुमारसंवादे स्यमन्तकोपाख्यानं सम्पूर्णम्॥

