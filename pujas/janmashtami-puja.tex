% !TeX program = XeLaTeX
% !TeX root = ../pujavidhanam.tex

\chapt{श्रीकृष्णजन्माष्टमी-पूजा}
\centerline{\small{(मूलम्—श्री-व्रतराजः)}}
\setlength{\parindent}{0pt}

व्रतपूर्वदिने दन्तधावनपूर्वकं कृतैकभक्तो व्रतदिने कृतनित्यक्रियो देवताः प्रार्थयेत्—

\threelineshloka*
{सूर्यः सोमो यमः कालसन्ध्या भूतान्यहःक्षपा}
{पवनो दिक्पतिर्भूमिराकाशं खेचरा नराः}
{ब्रह्मशासनमास्थाय कल्पन्तामिह सन्निधिम्}

इत्युक्त्वा सफलं पुष्पाक्षतजलपूर्णं ताम्रपात्रमादाय मासपक्षाद्युल्लिख्य अमुकफलकामं पापक्षयकामो वा 
कृष्णप्रीतये कृष्णजन्माष्टमीव्रतं करिष्ये इति सङ्कल्प्य।

\twolineshloka*
{वासुदेवं समुद्दिश्य सर्वपापप्रशान्तये}
{उपवासं करिष्यामि कृष्णाष्टम्यां नभस्यहम्}

\twolineshloka*
{अद्य कृष्णाष्टमी देवी नभश्चन्द्रं सरोहिणीम्}
{अर्चयित्वोपवासेन भोक्ष्येऽहमपरेऽहनि}

\twolineshloka*
{एनसो मोक्षकामोऽस्मि यगोविन्दवियोनिजम्}
{तन्मे मुञ्चतु मां त्राहि पतितं शोकसागरे}

\twolineshloka*
{आजन्ममरणं यावद् यन्मया दुष्कृतं कृतम्}
{तत्प्रणाशय गोविन्द प्रसीद पुरुषोत्तम}
%१ यन्मे बियोनिजं विविधजन्मजं एन इति शेषः। तन्मां मुञ्चतु इत्यन्वयः। विभोजनमित्यपि पाठः तत्र यन्मे विभोजनमुपवासस्तन्मां मुञ्चतु मोचयत्वित्यर्थः।

इत्युक्त्वा पात्रस्थं जलं निक्षिपेत्।

ततः कदली-स्तम्भ-वासोभि-राम्र-पल्लव-युत-सजल-पूर्ण-कलशेर्दीपैः पुष्प-मालाभिर्\-युतमगुरु-धूपित-मग्नि-खड्ग-कृष्णच्छाग-रक्षामणि-द्वार-न्यस्त-मुसलादि-युतं मङ्गलोपेतं षष्ठया देव्याधिष्ठितं देवक्याः
सूतिकागृहं विधाय तस्य समन्ताद्भित्तिषु कुसुमाञ्जलीन् देवगन्धर्वादीन् खड्ग-चर्मधर-वसुदेव-देवकी-नन्द-यशोदा-गर्ग-गोपी-गोपान्-कंस-नियुक्तान्
गो-धेनु-कुञ्जरान्-यमुनां तन्मध्ये कालियमन्यच्च तत्कालीनं गोकुलचरितं 
यथासम्भवं लिखित्वा सूतिकागृहमध्ये प्रछदपटावृतं मञ्चकं स्थापयित्वा
मध्याह्ने नद्यादौ तिलैः स्नात्वा अर्धरात्रे श्रीकृष्णं सपरिवारं सुपूजयेत्॥


\sect{पूर्वाङ्गविघ्नेश्वरपूजा}

(आचम्य)
\twolineshloka*
{शुक्लाम्बरधरं विष्णुं शशिवर्णं चतुर्भुजम्}
{प्रसन्नवदनं ध्यायेत् सर्वविघ्नोपशान्तये}
 
प्राणान्  आयम्य।  ॐ भूः + भूर्भुवः॒ सुव॒रोम्।
 
(अप उपस्पृश्य, पुष्पाक्षतान् गृहीत्वा)\\
ममोपात्तसमस्त दुरितक्षयद्वारा \\
श्रीपरमेश्वरप्रीत्यर्थं करिष्यमाणस्य कर्मणः\\
 निर्विघ्नेन परिसमाप्त्यर्थम् आदौ विघ्नेश्वरपूजां करिष्ये।

\twolineshloka*
{ॐ ग॒णानां᳚ त्वा ग॒णप॑तिꣳ हवामहे क॒विं क॑वी॒नामु॑प॒मश्र॑वस्तमम्}
{ज्ये॒ष्ठ॒राजं॒ ब्रह्म॑णां ब्रह्मणस्पत॒ आ नः॑ शृ॒ण्वन्नू॒तिभिः॑ सीद॒ साद॑नम्}
अस्मिन् हरिद्राबिम्बे महागणपतिं ध्यायामि, आवाहयामि।\\


ॐ महागणपतये नमः  आसनं समर्पयामि।\\
पादयोः पाद्यं समर्पयामि। हस्तयोरर्घ्यं समर्पयामि।\\
आचमनीयं समर्पयामि।\\
ॐ भूर्भुवस्सुवः। शुद्धोदकस्नानं समर्पयामि।\\
स्नानानन्तरमाचमनीयं समर्पयामि।\\
वस्त्रार्थमक्षतान् समर्पयामि।\\
यज्ञोपवीताभरणार्थे अक्षतान् समर्पयामि।\\
दिव्यपरिमलगन्धान् धारयामि।\\
गन्धस्योपरि हरिद्राकुङ्कुमं समर्पयामि। अक्षतान् समर्पयामि। \\
पुष्पमालिकां समर्पयामि। पुष्पैः पूजयामि।

\dnsub{अर्चना}
% \setenumerate{label=\devanumber.}
% \renewcommand{\labelenumi}{\devanumber\theenumi.}
\begin{enumerate}%[label=\devanumber\value{enumi}]
\begin{minipage}{0.475\linewidth}   
\item ॐ सुमुखाय नमः
\item ॐ एकदन्ताय नमः
\item ॐ कपिलाय नमः
\item ॐ गजकर्णकाय नमः
\item ॐ लम्बोदराय नमः
\item ॐ विकटाय नमः
\item ॐ विघ्नराजाय नमः
\item ॐ विनायकाय नमः
\item ॐ धूमकेतवे नमः
  \end{minipage}
  \begin{minipage}{0.525\linewidth}
\item ॐ गणाध्यक्षाय नमः
\item ॐ फालचन्द्राय नमः
\item ॐ गजाननाय नमः
\item ॐ वक्रतुण्डाय नमः
\item ॐ शूर्पकर्णाय नमः
\item ॐ हेरम्बाय नमः
\item ॐ स्कन्दपूर्वजाय नमः
\item ॐ सिद्धिविनायकाय नमः
\item ॐ विघ्नेश्वराय नमः
  \end{minipage}
\end{enumerate}
नानाविधपरिमलपत्रपुष्पाणि समर्पयामि॥\\
धूपमाघ्रापयामि।\\
अलङ्कारदीपं सन्दर्शयामि।\\
नैवेद्यम्।\\
ताम्बूलं समर्पयामि।\\
कर्पूरनीराजनं समर्पयामि।\\
कर्पूरनीराजनानन्तरमाचमनीयं समर्पयामि।\\
{वक्रतुण्डमहाकाय कोटिसूर्यसमप्रभ।}\\
{अविघ्नं कुरु मे देव सर्वकार्येषु सर्वदा॥}\\
प्रार्थनाः समर्पयामि।

अनन्तकोटिप्रदक्षिणनमस्कारान् समर्पयामि।\\
छत्त्रचामरादिसमस्तोपचारान् समर्पयामि।\\


\sect{प्रधान-पूजा — श्रीकृष्ण-पूजा}


\twolineshloka*
{शुक्लाम्बरधरं विष्णुं शशिवर्णं चतुर्भुजम्}
{प्रसन्नवदनं ध्यायेत् सर्वविघ्नोपशान्तये}

प्राणान्  आयम्य।  ॐ भूः + भूर्भुवः॒ सुव॒रोम्।

\dnsub{ सङ्कल्पः}
\setlength{\parindent}{30pt}


ममोपात्तसमस्तदुरितक्षयद्वारा श्रीपरमेश्वरप्रीत्यर्थं शुभे शोभने मुहूर्ते अद्यब्रह्मणः
द्वितीयपरार्द्धे श्वेतवराहकल्पे वैवस्वतमन्वन्तरे अष्टाविंशतितमे कलियुगे प्रथमे पादे
जम्बूद्वीपे भारतवर्षे भरतखण्डे मेरोः दक्षिणेपार्श्वे शकाब्दे अस्मिन् वर्तमाने व्यावहारिके
प्रभवादि षष्टिसंवत्सराणां मध्ये (  ) नाम संवत्सरे दक्षिणायने वर्ष-ऋतौ  (कर्कटक/सिंह)-श्रावण-मासे 
कृष्णपक्षे अष्टम्यां शुभतिथौ (इन्दु/भौम/बुध/गुरु/भृगु /स्थिर/भानु) वासरयुक्तायाम्
(कृत्तिका/रोहिणी/मृगशीर्ष) नक्षत्रयुक्तायां ()-योग ()-करण-युक्तायां च एवं गुण-विशेषण-विशिष्टायाम्
अस्याम् नवम्यां शुभतिथौ अस्माकं सहकुटुम्बानां क्षेमस्थैर्य-धैर्य-वीर्य-विजय आयुरारोग्य ऐश्वर्याभिवृद्ध्यर्थम्
धर्मार्थकाममोक्ष\-चतुर्विधफलपुरुषार्थसिद्ध्यर्थं पुत्रपौत्राभिवृद्ध्यर्थम् इष्टकाम्यार्थसिद्ध्यर्थम्
मम इहजन्मनि पूर्वजन्मनि जन्मान्तरे च सम्पादितानां ज्ञानाज्ञानकृतमहा\-पातकचतुष्टय
व्यतिरिक्तानां रहस्यकृतानां प्रकाशकृतानां सर्वेषां पापानां सद्य अपनोदनद्वारा सकल 
पापक्षयार्थं 
श्रावण-कृष्ण-जन्माष्मी-पुण्यकाले देवकी-सहित-श्रीकृष्ण-प्रीत्यर्थं देवकी-सहित-श्रीकृष्ण-प्रसाद-सिध्यर्थं कल्पोक्त-प्रकारेण देवकी-सहित-श्रीकृष्ण-पूजां करिष्ये।

तदङ्गं कलशपूजां च करिष्ये।


श्रीविघ्नेश्वराय नमः यथास्थानं प्रतिष्ठापयामि।
(गणपति प्रसादं शिरसा गृहीत्वा)

\dnsub{आसन-पूजा}
\centerline{पृथिव्या  मेरुपृष्ठ  ऋषिः।  सुतलं  छन्दः।  कूर्मो  देवता॥}
\twolineshloka*
{पृथ्वि  त्वया  धृता  लोका  देवि  त्वं  विष्णुना  धृता}
{त्वं  च  धारय  मां  देवि  पवित्रं  चाऽऽसनं  कुरु}


\dnsub{घण्टापूजा}
\twolineshloka*
{आगमार्थं तु देवानां गमनार्थं तु रक्षसाम्}
{घण्टारवं करोम्यादौ देवताऽऽह्वानकारणम्}


\dnsub{कलशपूजा}
ॐ कलशाय नमः दिव्यगन्धान् धारयामि।\\
ॐ गङ्गायै नमः। ॐ यमुनायै नमः। ॐ गोदावर्यै नमः।  ॐ सरस्वत्यै नमः। ॐ नर्मदायै नमः। ॐ सिन्धवे नमः। ॐ कावेर्यै नमः।\\
ॐ सप्तकोटिमहातीर्थान्यावाहयामि।\\[-0.25ex]

(अथ कलशं स्पृष्ट्वा जपं कुर्यात्) \\
आपो॒ वा इ॒द सर्वं॒ विश्वा॑ भू॒तान्याप॑ प्रा॒णा वा आप॑ प॒शव॒ आपो\-ऽन्न॒मापोऽमृ॑त॒माप॑ स॒म्राडापो॑ वि॒राडाप॑ स्व॒राडाप॒श्\-छन्दा॒स्यापो॒ ज्योती॒ष्यापो॒ यजू॒ष्याप॑ स॒त्यमाप॒ सर्वा॑ दे॒वता॒ आपो॒ भूर्भुव॒ सुव॒राप॒ ओम्॥\\

\twolineshloka* 
{कलशस्य मुखे विष्णुः कण्ठे रुद्रः समाश्रितः}
{मूले तत्र स्थितो ब्रह्मा मध्ये मातृगणाः स्मृताः}
\threelineshloka* 
{कुक्षौ तु सागराः सर्वे सप्तद्वीपा वसुन्धरा}
{ऋग्वेदोऽथ यजुर्वेदः सामवेदोऽप्यथर्वणः}
{अङ्गैश्च सहिताः सर्वे कलशाम्बुसमाश्रिताः}
\twolineshloka* 
{गङ्गे च यमुने चैव गोदावरि सरस्वति}
{नर्मदे सिन्धुकावेरि जलेऽस्मिन् सन्निधिं कुरु}
\twolineshloka*
{सर्वे समुद्राः सरितः तीर्थानि च ह्रदा नदाः}
{आयान्तु देवपूजार्थं दुरितक्षयकारकाः}

\centerline{ॐ भूर्भुवः॒ सुवो॒ भूर्भुवः॒ सुवो॒ भूर्भुवः॒ सुवः॑।}

(इति कलशजलेन सर्वोपकरणानि आत्मानं च प्रोक्ष्य।)


\dnsub{आत्म-पूजा}
ॐ आत्मने नमः, दिव्यगन्धान् धारयामि।
\begin{multicols}{2}
१. ॐ आत्मने नमः\\
२. ॐ अन्तरात्मने नमः\\
३. ॐ योगात्मने नमः\\
४. ॐ जीवात्मने नमः\\
५. ॐ परमात्मने नमः\\
६. ॐ ज्ञानात्मने नमः
\end{multicols}
समस्तोपचारान् समर्पयामि।

\twolineshloka*
{देहो देवालयः प्रोक्तो जीवो देवः सनातनः}
{त्यजेदज्ञाननिर्माल्यं सोऽहं भावेन पूजयेत्}


\begin{minipage}{\linewidth}
\dnsub{पीठ-पूजा}

\begin{multicols}{2}
\begin{enumerate}
\item ॐ आधारशक्त्यै नमः
\item ॐ मूलप्रकृत्यै नमः
\item ॐ आदिकूर्माय नमः 
\item ॐ आदिवराहाय नमः
\item ॐ अनन्ताय नमः
\item ॐ पृथिव्यै नमः
\item ॐ रत्नमण्डपाय नमः
\item ॐ रत्नवेदिकायै नमः
\item ॐ स्वर्णस्तम्भाय नमः
\item ॐ श्वेतच्छत्त्राय नमः
\item ॐ कल्पकवृक्षाय नमः
\item ॐ क्षीरसमुद्राय नमः 
\item ॐ सितचामराभ्यां नमः
\item ॐ योगपीठासनाय नमः
\end{enumerate}
\end{multicols}

\end{minipage}

\dnsub{गुरु ध्यानम्}

\twolineshloka*
{गुरुर्ब्रह्मा गुरुर्विष्णुर्गुरुर्देवो महेश्वरः}
{गुरुः साक्षात् परं ब्रह्म तस्मै श्री गुरवे नमः}


\sect{श्रीकृष्णजन्माष्टमी-षोडशोपचारपूजा}

\begin{center}

\fourlineindentedshloka*
{गायद्भिः किन्नराद्यैः सततपरिवृता वेणुवीणानिनादैः}
{भृङ्गारादर्श-कुन्तप्रवर-कृतकरैः किङ्करैः सेव्यमाना}
{पर्यङ्के स्वास्तृते या मुदिततरमुखी पुत्रिणी सम्यगास्ते}
{सा देवी देवमाता जयति सुवदना देवकी दिव्यरूपा} 
इति देवकीं ध्यात्वा।

\twolineshloka*
{ध्यायामि बालकं सुप्तं मात्रङ्के स्तनपायिनम्}
{श्रीवत्स-वक्षसं शान्तं नीलोत्पल-दलच्छविम्}
एवं देवक्या सहितं श्रीकृष्णं ध्यात्वा।

ॐ नमो देव्यै श्रियै नमः इति श्रियं ध्यात्वा, आवाह्य।

ॐ नमो वसुदेवाय नमः इति देवकीसहितं वसुदेवं ध्यात्वा, आवाह्य।

ॐ नमो नन्दाय नमः इति यशोदासहितं नन्दं ध्यात्वा, आवाह्य।

ॐ नमो बलदेवाय नमः इति श्रीकृष्णसहितं बलदेवं ध्यात्वा, आवाह्य।

ॐ नमश्चण्डिकायै नमः इति चण्डिकां ध्यात्वा आवाह्य।

ततः श्रीकृष्णपूजां कुर्यात्। ध्यानम्—

\renewcommand{\devAya}{सपरिवाराय कृष्णाय नमः,}
\medskip

\twolineshloka*
{कृष्णं चतुर्भुजं देवं शङ्खचक्रगदाधरम्}
{पीताम्बरयुगोपतं लक्ष्मीयुक्तं विभूषितम्}

\twolineshloka*
{लसत्कौस्तुभ-शोभाढ्यं मेघश्यामं सुलोचनम्}
{ध्यायामि पुण्डरीकाक्षं जगदानन्दकारकम्}
\medskip

\twolineshloka*
{स॒हस्र॑शीर्‌षा॒ पुरु॑षः। स॒ह॒स्रा॒क्षः स॒हस्र॑पात्}
{स भूमिं॑ वि॒श्वतो॑ वृ॒त्वा। अत्य॑तिष्ठद्दशाङ्गु॒लम्}

\twolineshloka*
{आगच्छ देवदेवेश जगद्योने रमापते}
{बिम्बे चास्मिन्नधिष्ठाने सन्निधेहि कृपां कुरु}
श्रीकृष्णमावाहयामि।
\medskip

\twolineshloka*
{पुरु॑ष ए॒वेद सर्वम्। यद्भू॒तं यच्च॒ भव्यम्}
{उ॒तामृ॑त॒त्वस्येशा॑नः। यदन्ने॑नाति॒रोह॑ति}

\twolineshloka*
{देवदेव जगन्नाथ गरुडासनसंस्थित} 
{गृहाणासनकं दिव्यं जगद्धातर्नमोऽस्तु ते}
\devAya{} आसनं समर्पयामि।
\medskip

\twolineshloka*
{ए॒तावा॑नस्य महि॒मा। अतो॒ ज्यायाश्च॒ पूरु॑षः}
{पादोऽस्य॒ विश्वा॑ भू॒तानि॑। त्रि॒पाद॑स्या॒मृतं॑ दि॒वि}

\twolineshloka*
{नानातीर्थाहृतं शुद्धं निर्मलं पुष्पमिश्रितम्}
{पाद्यं गृहाण दैत्यारे विश्वरूप नमोऽस्तु ते}
\devAya{} पाद्यं समर्पयामि।
\medskip

\twolineshloka*
{त्रि॒पादू॒र्ध्व उदै॒त्पुरु॑षः। पादोऽस्ये॒हाऽऽभ॑वा॒त्पुन॑}
{ततो॒ विश्व॒ङ्व्य॑क्रामत्। सा॒श॒ना॒न॒श॒ने अ॒भि}

\twolineshloka*
{गन्धपुष्पाक्षतोपेतं फलेन च समन्वितम्}
{अर्घ्यं गृहाण देवेश मया दत्तं हि भक्तितः}
\devAya{} अर्घ्यं समर्पयामि।

\twolineshloka*
{तस्माद्वि॒राड॑जायत। वि॒राजो॒ अधि॒ पूरु॑षः}
{स जा॒तो अत्य॑रिच्यत। प॒श्चाद्भूमि॒मथो॑ पु॒रः}

\twolineshloka*
{गङ्गादिसर्वतीर्थेभ्यो मयाऽऽनीतं सुशीतलम्‌}
{गृहाणाचमनं देव विश्वकाय नमोऽस्तु ते}
\devAya{} आचमनीयं समर्पयामि।
\medskip


\twolineshloka*
{यत्पुरु॑षेण ह॒विषा। दे॒वा य॒ज्ञमत॑न्वत}
{व॒स॒न्तो अ॑स्याऽऽसी॒दाज्यम्। ग्री॒ष्म इ॒ध्मः श॒रद्ध॒विः}

\twolineshloka*
{दधि क्षौद्रं घृतं शुद्धं कपिलायाः सुगन्धि यत्‌}
{सुस्वादु मधुरं शौरे मधुपर्कं गृहाण भोः}
\devAya{} मधुपर्कं समर्पयामि।\medskip


\twolineshloka*
{पञ्चामृतेन स्नपनं करिष्यामि सुरोत्तम}
{क्षीरोदधिनिवासाय लक्ष्मीकान्ताय ते नमः}
\devAya{} पञ्चामृतस्नानं समर्पयामि।
\medskip

\twolineshloka*
{स॒प्तास्या॑ऽऽसन्  परि॒धय॑। त्रिः स॒प्त स॒मिध॑ कृ॒ताः}
{दे॒वा यद्य॒ज्ञं त॑न्वा॒नाः। अब॑ध्न॒न् पु॑रुषं प॒शुम्}

\twolineshloka*
{मन्दाकिनी गौतमी च यमुना च सरस्वती}
{ताभ्यः स्नानार्थमानीतं गृहाण शिशिरं जलम्‌}
\devAya{} स्नानं समर्पयामि। \\
स्नानानन्तरमाचमनीयं समर्पयामि। \\
\medskip

\twolineshloka*
{तं य॒ज्ञं ब॒र्{}हिषि॒ प्रौक्षन्। पुरु॑षं जा॒तम॑ग्र॒तः}
{तेन॑ दे॒वा अय॑जन्त। सा॒ध्या ऋष॑यश्च॒ ये}

\twolineshloka*
{शुद्ध-जाम्बूनद-प्रख्ये तटिद्भासुर-रोचिषी}
{मयोपपादिते तुभ्यं वाससी च गृहाण भोः}
\devAya{} वस्रं समर्पयामि।\\
\medskip

\twolineshloka*
{तस्माद्य॒ज्ञाथ्स॑र्व॒हुत॑। सम्भृ॑तं पृषदा॒ज्यम्}
{प॒शूस्ताश्च॑क्रे वाय॒व्यान्। आ॒र॒ण्यान्ग्रा॒म्याश्च॒ ये}

\twolineshloka*
{दामोदर नमस्तेस्तु त्राहि मां भवसागरात्}
{ब्रह्मसूत्रं मया दत्तं गृहाण पुरुषोत्तम}
\devAya{} यज्ञोपवीतं समर्पयामि । 
\medskip

\twolineshloka*
{किरीटकुण्डलादीनि काञ्चीवलययुग्मकम्‌}
{कौस्तुभं वनमालाञ्च भूषणानि भजस्व भोः}
\devAya{} आभरणानि समर्पयामि। \\
\medskip

\twolineshloka*
{तस्माद्य॒ज्ञाथ्स॑र्व॒हुत॑। ऋच॒ सामा॑नि जज्ञिरे}
{छन्दासि जज्ञिरे॒ तस्मात्। यजु॒स्तस्मा॑दजायत}

\twolineshloka*
{मलयाचलसम्भूतं गन्धसारं मनोहरम्}
{हृदयानन्दनं चारु प्रीत्यर्थे प्रतिगृह्यताम्}
\devAya{} चन्दनं समर्पयामि।\\
\medskip

\twolineshloka*
{तस्मा॒दश्वा॑ अजायन्त। ये के चो॑भ॒याद॑तः}
{गावो॑ ह जज्ञिरे॒ तस्मात्। तस्माज्जा॒ता अ॑जा॒वय॑}

\twolineshloka*
{मालतीचम्पकादीनि यूथिकावकुलानि च}
{तुलसीपत्रमिश्राणि गृहाण सुरसत्तम}
\devAya{} पुष्पाणि समर्पयामि।\\

\dnsub{अङ्गपूजा}
\begin{longtable}{ll@{—}l}
१. & गोविन्दाय नमः & पादौ पूजयामि \\
२. & माधवाय नमः &  जङ्घे पूजयामि \\
३. & मधुसूदनाय नमः & कटी पूजयामि \\
४. & पद्मनाभाय नमः & नाभिं पूजयामि \\
५. & हृषीकेशाय नमः & हृदयं पूजयामि \\
६. & सङ्कर्षणाय नमः & स्तनौ पूजयामि \\
७. & वामनाय नमः & बाहू पूजयामि \\
८. & दैत्यसूदनाय नमः & हस्तौ पूजयामि \\
९. & हरिकेशाय नमः & नमः कण्ठं पूजयामि \\
१०. & चारुमुखाय नमः & मुखं पूजयामि \\
११. & त्रिविक्रमाय नमः & नासिकां पूजयामि \\
१२. & पुण्डरीकाक्षाय नमः & नेत्रे पूजयामि \\
१३. & नृसिंहाय नमः & श्रोत्रे पूजयामि \\
१४. & उपेन्द्राय नमः & ललाटं पूजयामि \\
१५. & हरये नमः & शिरः पूजयामि \\
१६. & श्रीकृष्णाय नमः & सर्वाणि अङ्गानि पूजयामि \\
\end{longtable}

\dnsub{चतुर्विंशति नामपूजा}
\begin{multicols}{2}
\begin{enumerate}
\item ॐ केशवाय नमः
\item ॐ नारायणाय नमः
\item ॐ माधवाय नमः
\item ॐ गोविन्दाय नमः
\item ॐ विष्णवे नमः 
\item ॐ मधुसूदनाय नमः
\item ॐ त्रिविक्रमाय नमः
\item ॐ वामनाय नमः
\item ॐ श्रीधराय नमः
\item ॐ हृषीकेशाय नमः
\item ॐ पद्मनाभाय नमः
\item ॐ दामोदराय नमः
\item ॐ सङ्कर्षणाय नमः
\item ॐ वासुदेवाय नमः
\item ॐ प्रद्युम्नाय नमः
\item ॐ अनिरुद्धाय नमः
\item ॐ पुरुषोत्तमाय नमः
\item ॐ अधोक्षजाय नमः
\item ॐ नृसिंहाय नमः
\item ॐ अच्युताय नमः
\item ॐ जनार्दनाय नमः
\item ॐ उपेन्द्राय नमः 
\item ॐ हरये नमः
\item ॐ श्रीकृष्णाय नमः
\end{enumerate}
\end{multicols}

\begingroup
\centering
\setlength{\columnseprule}{1pt}
\let\chapt\sect
\needspace{6em}
\input{../namavali-manjari/100/Krishna_108.tex}
\endgroup

\sect{उत्तराङ्गपूजा}

\twolineshloka*
{यत्पुरु॑षं॒ व्य॑दधुः। क॒ति॒धा व्य॑कल्पयन्}
{मुखं॒ किम॑स्य॒ कौ बा॒हू। कावू॒रू पादा॑वुच्येते}

\twolineshloka*
{वनस्पतुरसोद्भूतं कालागरु-समन्वितम्}
{धूपं गृहाण गोविन्द गुणसागर गोपते}
\devAya{} धूपमाघ्रापयामि।\medskip

\twolineshloka*
{ब्रा॒ह्म॒णोऽस्य॒ मुख॑मासीत्। बा॒हू रा॑ज॒न्य॑ कृ॒तः}
{ऊ॒रू तद॑स्य॒ यद्वैश्य॑। प॒द्भ्या शू॒द्रो अ॑जायत}

\twolineshloka*
{यज्ञेश्वराय देवाय तथा यज्ञोद्भवाय च}
{यज्ञानां पतये नाथ गोविन्दाय नमो नमः}

\twolineshloka*
{साज्यं त्रिवर्तिसंयुक्तं वह्निना योजितं मया}
{दीपं गृहाण देवेश त्रैलोक्यतिमिरापह}
\devAya{} दीपं दर्शयामि।\\
\medskip

ॐ भूर्भुवः॒ सुवः॑। + ब्र॒ह्मणे॒ स्वाहा᳚।
\twolineshloka*
{च॒न्द्रमा॒ मन॑सो जा॒तः। चक्षो॒ सूर्यो॑ अजायत}
{मुखा॒दिन्द्र॑श्चा॒ग्निश्च॑। प्रा॒णाद्वा॒युर॑जायत}


\twolineshloka*
{विश्वेश्वराय विश्वाय तथा विश्वोद्भवाय च}
{विश्वस्य पतये तुभ्यं गोविन्दाय नमो नमः}

\twolineshloka*
{शाल्योदनं पायसं च सिताघृतविमिश्रितम्}
{नानापक्वान्नसंयुक्तं नैवेद्यं प्रतिगृह्यताम्}
\devAya{} सूपसहितं शाल्योदनं शाकोपदंसं निवेदयामि।\\
मध्ये मध्ये अमृतपानीयं समर्पयामि । उत्तरापोशनं समर्पयामि।\\
हस्तप्रक्षालनं समर्पयामि । पादप्रक्षालनं समर्पयामि । गण्डूषं समर्पयामि।\\
हस्तप्रक्षालनं समर्पयामि । आचमनीयं समर्पयामि।\\
\devAya{} शष्कुली-लवनचिपिट-गुडचिपिट-गुड-शुण्ठी-नवनीतादीनि,\\
जम्बूफल-तिन्त्रिणीफल-प्रभृतीनि नानाफलानि च निवेदयामि।
\medskip

\twolineshloka*
{नाभ्या॑ आसीद॒न्तरि॑क्षम्। शी॒र्ष्णो द्यौः सम॑वर्तत}
{प॒द्भ्यां भूमि॒र्दिश॒ श्रोत्रात्। तथा॑ लो॒का अ॑कल्पयन्}
\twolineshloka*
{पूगीफलसमायुक्तं नागवल्ली-दलैर्युतम्}
{कर्पूरचूर्णसंयुक्तं ताम्बूलं प्रतिगृह्यताम्}
\devAya{} कर्पूरताम्बूलं निवेदयामि।\\
\medskip


\twolineshloka*
{वेदा॒हमे॒तं पुरु॑षं म॒हान्तम्। आ॒दि॒त्यव॑र्णं॒ तम॑स॒स्तु पा॒रे}
{सर्वा॑णि रू॒पाणि॑ वि॒चित्य॒ धीर॑। नामा॑नि कृ॒त्वाऽभि॒वद॒\an{} यदास्ते}

\twolineshloka*
{नीराजयेत्ततो भक्त्या मङ्गलं समुदीरयन्}
{जय-मङ्गल-निर्घोषैर्देवदेवं समर्चयेत्}
\devAya{} कर्पूरनीराजनं दर्शयामि।\\
\medskip

\twolineshloka*
{धा॒ता पु॒रस्ता॒द्यमु॑दाज॒हार॑। श॒क्रः प्रवि॒द्वान्  प्र॒दिश॒श्चत॑स्रः}
{तमे॒वं वि॒द्वान॒मृत॑ इ॒ह भ॑वति। नान्यः पन्था॒ अय॑नाय विद्यते}

यो॑ऽपां पुष्पं॒ वेद॑। पुष्प॑वान् प्र॒जावान् पशु॒मान् भ॑वति।\\
च॒न्द्रमा॒ वा अ॒पां पुष्पम्। पुष्प॑वान् प्र॒जावान् पशु॒मान् भ॑वति।\\
य ए॒वं वेद॑। यो॑ऽपामा॒यत॑नं॒ वेद॑। आ॒यत॑नवान् भवति।\medskip

ओं तद्ब्र॒ह्म। ओं तद्वा॒युः। ओं तदा॒त्मा।\\ ओं᳚ तथ्स॒त्यम्‌।
ओं᳚ तथ्सर्वम्᳚‌। ओं तत्पुरो॒र्नमः॥\medskip

अन्तश्चरति॑ भूते॒षु॒ गुहायां वि॑श्वमू॒र्तिषु। \\
त्वं यज्ञस्त्वं वषट्कारस्त्वमिन्द्रस्त्व\\ रुद्रस्त्वं विष्णुस्त्वं ब्रह्म त्वं॑ प्रजा॒पतिः। \\
त्वं त॑दाप॒ आपो॒ ज्योती॒ रसो॒ऽमृतं॒ ब्रह्म॒ भूर्भुवः॒ सुव॒रोम्‌॥\medskip

\devAya{} वेदोक्तमन्त्रपुष्पाञ्जलिं समर्पयामि।\medskip

\twolineshloka*
{दत्वा पुष्पाञ्जलिं चैव प्रदक्षिणपुरस्सरम्}
{प्रणमेद्दण्डवद्भूमौ भक्तिप्रह्वः पुनः पुनः}
\medskip

\twolineshloka*
{स्तुत्वा नानाविधैः स्तोत्रैः प्रार्थयेत जगत्पतिम्}
{नमस्तुभ्यं जगन्नाथ देवकीतनय प्रभो}

\twolineshloka*
{वसुदेवात्मजानन्त यशोदानन्दवर्धन}
{गोविन्द गोकुलाधार गोपीकान्त नमोऽस्तु ते}

\devAya{} प्रदक्षिणनमस्कारान् समर्पयामि।
\medskip

% गारुडे तु-
% यज्ञाय यज्ञेश्वराय यज्ञपतये यज्ञसंभवाय गोविन्दाय नमोनम इति अध्ये॥

% सर्वेषां यज्ञपदानां स्थाने योगपदयुक्तोऽयमेव मन्त्रः स्नाने॥

% तथैव विश्वपदयुक्तो नैवेद्यौतिथैव धर्मपदयुक्तः स्वाहान्तस्तिलहोमे॥
% विश्वपदयुक्त एव शयने॥
% सोमपदयुक्तश्चन्द्रपूजायाम् इति मन्त्रा उक्ताः॥
% ततो गव्यघृतेनाग्नौ वसोर्धारा, क्वचिद्गुडघृतेनेति॥


ततो जातकर्मनालच्छेदषष्ठीपूजानामकरणकर्माणि सङ्क्षेपेण कार्याणि।

\dnsub{अर्घ्यप्रदानम्}

\twolineshloka*
{ततस्तु दापयेदर्घ्यम् इन्दोरुदयतः शुचिः}
{कृष्णाय प्रथमं दद्याद्देवकीसहिताय च}

\twolineshloka*
{नालिकेरेण शुद्धेन दद्यादर्घ्यं विचक्षणः}
{कृष्णाय परया भक्त्या शङ्खे कृत्वा विधानतः}
\medskip

\twolineshloka*
{जातः कंसवधार्थाय भूभारोत्तारणाय च}
{पाण्डवानां हितार्थाय धर्मसंस्थापनाय च}

\twolineshloka*
{कौरवाणां विनाशाय दैत्यानां निधनाय च}
{गृहाणार्घ्यं मया दत्तं देवकीजनित प्रभो}
\devAya{} इदमर्घ्यमिदमर्घ्यमिदमर्घ्यम्॥\medskip
\medskip\\
पूजिताभ्यः सर्वाभ्यो देवताभ्यः तत्तन्नाममन्त्रेण अर्घ्यं दद्यात् \\

\begin{longtable}{ll@{—}l}
१.& ॐ देवक्यै नमः & इदमर्घ्यम्\\
२.& ॐ वसुदेवाय नमः & इदमर्घ्यम्\\
३.& ॐ रोहिण्यै नमः & इदमर्घ्यम्	\\
४.& ॐ सबलायै नमः & इदमर्घ्यम्	\\
५.& ॐ सात्यक्यै नमः & इदमर्घ्यम्\\
६.& ॐ उद्धवाय नमः & इदमर्घ्यम्	\\
७.& ॐ अक्रूराय नमः & इदमर्घ्यम्\\
८.& ॐ उग्रसेनादि-यादवेभ्यो नमः & इदमर्घ्यम्\\
९.& ॐ नन्दाय नमः & इदमर्घ्यम्\\
१०.& ॐ यशोदायै नमः & इदमर्घ्यम्	\\
११.& ॐ तत्कालप्रसूताभ्यः गोपगोपिकाभ्यो नमः & इदमर्घ्यम्\\
१३.& ॐ कालिन्द्यै नमः & इदमर्घ्यम्\\
१४.& ॐ काल्यै नमः & इदमर्घ्यम्\\

\end{longtable}

इति पृथक्पृथगर्घ्यं दत्वा॥\\
\medskip


ततश्चन्द्रोदये रोहिणीयुतं चन्द्रं स्थण्डिले प्रतिमायां वा नाममन्त्रेण सम्पूज्य।


\twolineshloka*
{ततस्तु रोहिणीयुक्तं चन्द्रं सम्पूज्य भक्तितः}
{स्तुत्वा तु स्तोत्रमन्त्रेण चन्द्रायार्घ्यं प्रदापयेत्}
\medskip

आप्यायस्वेति मन्त्रेण देवसमीपे चन्दनबिम्बे रोहिणीसहितं\\
चन्द्रमावाह्य षोडशोपचारैः सम्पूजयेत्॥\\
चन्द्रप्रार्थना–\\


% \twolineshloka*
% {शङ्खे तोयं समादाय सपुष्पकुशचन्दनम्}
% {जानुभ्यामवनीं गत्वा चन्द्रायार्घ्यं निवेदयेत्}

% क्षीरोदार्णवसम्भूत अत्रिगोत्रसमुद्भव॥
% गृहाणार्घ्यं शशाङ्केदं रोहिण्या सहितो मम॥
% इति अय॑म्॥

\twolineshloka*
{ज्योत्स्नायाः पतये तुभ्यं ज्योतिषां पतये नमः}
{नमस्ते रोहिणीकान्त सुधावास नमोऽस्तु ते}

\twolineshloka*
{नमो मण्डलदीपाय शिरोरत्नाय धूर्जटे}
{कलाभिर्वर्धमानाय नमश्चन्द्राय चारवे}
इति प्रणमेत्।

\twolineshloka*
{ज्योत्स्नापते नमस्तुभ्यं नमस्ते ज्योतिषां पते}
{नमस्ते रोहिणीकान्त नमस्ते युवमोहन}
इति स्तुत्वा\\
\medskip

\twolineshloka*
{शङ्खे कृत्वा ततस्तोयं सपुष्पफलचन्दनम्}
{जानुभ्यामवनीं गत्वा चन्द्रायार्घ्यं निवेदयेत्}
चन्द्रार्घ्यमन्त्रः–\\

\twolineshloka*
{क्षीरोदार्णवसम्भूत अत्रिनेत्रसमुद्भव}
{रोहिणीश गृहाणार्घ्यं रमाभ्रातर्मनःपते}
रोहिणीसहिताय चन्द्राय नमः इदमर्घ्यम् (त्रिः)\\

इत्यर्घ्यं दत्वा देवकीसहिताय कृष्णाय छत्रचामराद्युपचारं कृत्वा\\पूजां समाप्य कृष्णावतारघट्टं पठेत्॥

\twolineshloka*
{य॒ज्ञेन॑ य॒ज्ञम॑यजन्त दे॒वाः। तानि॒ धर्मा॑णि प्रथ॒मान्या॑सन्}
{ते ह॒ नाकं॑ महि॒मान॑ सचन्ते। यत्र॒ पूर्वे॑ सा॒ध्याः सन्ति॑ दे॒वाः}
\devAya{} छत्त्रचामरादिसमस्तोपचारान् समर्पयामि।\medskip

\twolineshloka
{अनघं वामनं शौरि वैकुण्ठं पुरुषोत्तमम्}
{वासुदेवं हृषीकेशं माधवं मधुसूदनम्}

\twolineshloka
{वराहं पुण्डरीकाक्षं नृसिंह दैत्यसूदनम्}
{दामोदरं पद्मनाभं केशवं गरुडध्वजम्}

\twolineshloka
{गोविन्दमच्युतं कृष्णमनन्तमपराजितम्}
{अधोक्षजं जगबीजं सर्गस्थित्यन्तकारणम्}

\twolineshloka
{अनादिनिधनं विष्णुं त्रिलोकेशं त्रिविक्रमम्}
{नारायणं चतुर्बाहुं शङ्खचक्रगदाधरम्}

\twolineshloka
{पीताम्बरधरं नित्यं वनमालाविभूषितम्}
{श्रीवत्साङ्कं जगत्सेतुं श्रीकृष्णं श्रीधरं हरिम्}

\twolineshloka
{शरणं त्वां प्रपद्येहं सर्वकामार्थसिद्धये}
{प्रणममामि सदा देवं वासुदेवं जगत्पतिम्}

इति मन्त्रैः प्रणम्य॥


\resetShloka
\threelineshloka
{त्राहि मां सर्वलोकेश हरे संसारसागरात्}
{त्राहि मां सर्वपापन्न दुःखशोकार्णवात्प्रभो}
{सर्वलोकेश्वर त्राहि पतितं मां भवार्णवे}

\twolineshloka
{देवकीनन्दन श्रीश हरे संसारसागरात्}
{त्राहि मां सर्वदुःखघ्न रोगशोकार्णवाद्धरे}

\twolineshloka
{दुर्वृत्ताबायसे विष्णो ये स्मरन्ति सकृत्सकृत्}
{सोऽहं देवातिदुर्वृत्तस्त्राहि मां शोकसागरात्}

\twolineshloka
{पुष्कराक्ष निमग्नोऽहं मायाव्यज्ञानसागरे}
{त्राहि मां देवदेवेश त्वत्तो नान्योऽस्ति रक्षिता}

\twolineshloka
{यद्वाल्ये यंञ्च कौमारे यौवने यच्च वार्द्धके}
{तत्पुण्यं वृद्धिमायातु पापं हर हलायुध}
इति मन्त्रैः प्रार्थयेत्॥

ततः स्तोत्रं पठन् पुराणश्रवणादिना जागरं कुर्यात्॥

द्वितीयेऽह्नि प्रातःकाले स्नानादिनित्यकर्म कृत्वा पूर्ववद्देवं पूजयित्वा ब्राह्मणान् भोजयेत्॥

तेभ्यः सुवर्णधेनुवस्त्रादि दत्त्वा कृष्णो मे प्रीयतामिति वदेत्॥

\twolineshloka*
{यं देवं देवकी देवी वसुदेवादजीजनत्}
{भौमस्य ब्रह्मणो गुप्त्यै तस्मै ब्रह्मात्मने नमः}

\twolineshloka*
{नमस्ते वासुदेवाय गोब्राह्मणहिताय च}
{शान्तिरस्तु शिवं चास्तु इत्युक्त्वा मां विसर्जयेत्}

इति प्रतिमामुद्वास्य तां ब्राह्मणाय दत्त्वा पारणं कृत्वा व्रतं समापयेत्॥

सर्वस्मै सर्वेश्वराय सर्वेषां पतये सर्वसम्भवाय गोविन्दाय नमो नम इति पारणे॥

भूताय भूतपतये नम इति समापने मन्त्रः॥

इति पूजाविधिः॥

\twolineshloka*
{हिरण्यगर्भगर्भस्थं हेमबीजं विभावसोः}
{अनन्तपुण्यफलदम् अतः शान्तिं प्रयच्छ मे}

श्री-कृष्णजन्माष्टमी-पुण्यकाले अस्मिन् मया क्रियमाण सपरिवार-कृष्णपूजायां यद्देयमुपायनदानं तत्प्रतिनिधित्वेन हिरण्यं सपरिवार-श्री-कृष्ण-प्रीतिं कामयमानः मनसोद्दिष्टाय ब्राह्मणाय सम्प्रददे नमः न मम। 

अनया पूजया सपरिवार-श्रीकृष्णः प्रीयताम्। 
 
\fourlineindentedshloka*
{कायेन वाचा मनसेन्द्रियैर्वा}
{बुद्‌ध्याऽऽत्मना वा प्रकृतेः स्वभावात्}
{करोमि यद्यत् सकलं परस्मै}
{नारायणायेति समर्पयामि}

ॐ तत्सद्ब्रह्मार्पणमस्तु।

\end{center}

% !TeX program = XeLaTeX
% !TeX root = ../../pujavidhanam.tex
\begingroup
\setlength{\parindent}{0pt}
\sect{जन्माष्टमी-व्रत-कथा}
\uvacha{युधिष्ठिर उवाच}

\twolineshloka
{जन्माष्टमीव्रतं ब्रूहि विस्तरेण ममाच्युत}
{कस्मिन्काले समुत्पन्नं किं पुण्यं को विधिः स्मृतः}%॥१॥

\uvacha{श्रीकृष्ण उवाच}
\twolineshloka
{मल्लयुद्धे परावृत्ते शमिते कुकुरान्धके}
{स्वजनैर्बन्धुभिः स्त्रीभिः समः स्निग्धैः समावृते}%॥२॥

\twolineshloka
{हते कंसासुरे दुष्टे मथुरायां युधिष्ठिर}
{देवकी मां परिष्वज्य कृत्वोत्सङ्गे रुरोद ह}%॥३॥

\twolineshloka
{वसुदेवोऽपि तत्रैव वात्सल्यात्प्ररुरोद ह}
{समालिङ्ग्याश्रुवदनः पुत्र पुत्रेत्युवाच ह}%॥४॥

\twolineshloka
{सगद्गदस्वरो दीनो बाष्पपर्याकुलेक्षणः}
{बलभद्रं च मां चैव परिष्वज्य मुदा पुनः}%॥५॥

\twolineshloka
{अद्य मे सफलं जन्म जीवितं च सुजीवितम्}
{उभाभ्यामद्य पुत्राभ्यां समुद्भूतः समागमः}%॥६॥

\twolineshloka
{एवं हर्षेण दाम्पत्यं हृष्टं पुष्टं तदा ह्यभूत्}
{प्रणिपत्य जनाः सर्वे बभूवुस्ते प्रहर्षिताः}%॥७॥

\onelineshloka*
{एवं महोत्सवं दृष्ट्वा मामूचुर्मधुसूदनम्}
\uvacha{जना ऊचुः}
\onelineshloka
{प्रसादः क्रियतामस्य लोकस्याऽऽर्तस्य दुःखहन्}%॥८॥

\twolineshloka
{यस्मिन्दिने च प्रासूत देवकी त्वां जनार्दन}
{तद्दिनं देहि वैकुण्ठ कुर्मस्तत्र महोत्सवम्}%॥९॥

\twolineshloka
{एवं स्तुतो जनौघेन वासुदेवो मयेक्षितः}
{विलोक्य बलभद्रं च मां च हृष्टतनूरुहः}%॥१०॥

\twolineshloka
{उवाच स ममादेशाल्लोकाञ्जन्माष्टमीव्रतम्}
{मथुरायां ततः पश्चात् पार्थ सम्यक् प्रकाशितम्}%॥११॥

\twolineshloka
{कुर्वन्तु ब्राह्मणाः सर्वे व्रतं जन्माष्टमीदिने}
{क्षत्रिया वैश्यजातीयाः शूद्रा येऽन्येऽपि धर्मिणः}%॥१२॥

\uvacha{युधिष्ठिर उवाच}
\twolineshloka
{कीदृशं तद्व्रतं देवदेव सर्वैरनुष्ठितम्}
{जन्माष्टमीति संज्ञं च पवित्रं पापनाशनम्}%॥१३॥

\twolineshloka
{येन त्वं तुष्टिमायासि कार्त्स्न्येन प्रभवाव्यय}
{एतन्मे तत्त्वतो ब्रूहि सविधानं सविस्तरम्}%॥१४॥

\uvacha{श्रीकृष्ण उवाच}
\twolineshloka
{मासि भाद्रपदेष्टम्यां निशीथे कृष्णपक्षके}
{शशाङ्के वृषराशिस्थे ऋक्षे रोहिणीसंज्ञके}%॥१५॥

\twolineshloka
{योगेऽस्मिन्वसुदेवाद्धि देवकी मामजीजनत्}
{भगवत्याश्च तत्रैव क्रियते सुमहोत्सवः}%॥१६॥

\twolineshloka
{योगेऽस्मिन्कथितेऽष्टम्यां सिंहराशिगते रवौ}
{सप्तम्यां लघुभुक् कुर्याद्दन्तधावनपूर्वकम्}%॥१७॥

\twolineshloka
{उपवासस्य नियमं रात्रौ स्वप्याज्जितेन्द्रियः}
{केवलेनोपवासेन तस्मिञ्जन्मदिने मम}%॥१८॥

\twolineshloka
{सप्तजन्मकृतात्पापान्मुच्यते नात्र संशयः}
{उपावृत्तस्य पापेभ्यो यस्तु वासोगुणैः सह}%॥१९॥

\twolineshloka
{उपवासः स विज्ञेयः सर्वभोगविवर्जितः}
{ततोऽष्टम्यां तिलैः स्नात्वा नद्यादौ विमले जले}%॥२०॥

\twolineshloka
{सुदेशे शोभनं कुर्याद्देवक्याः सूतिकागृहम्}
{सितपीतैस्तथा रक्तैः कर्बुरैरितरैरपि}%॥२१॥

\twolineshloka
{वासोभिः शोभितं कृत्वा समन्तात्कलशैर्नवैः}
{पुष्पैः फलैरनेकैश्च दीपालिभिरितस्ततः}%॥२२॥

\twolineshloka
{पुष्पमालाविचित्रं च चन्दनागुरुधूपितम्}
{अतिरम्यमनौपम्यं रक्षामणिविभूषितम्}%॥२३॥

\twolineshloka
{हरिवंशस्य चरितं गोकुलं च विलेखयेत्}
{ततो वादिबनिनदेर्वीणावेणुरवाकुलम्}%॥२४॥

\twolineshloka
{नृत्यगीतक्रमोपेतं मङ्गलैश्च समन्ततः}
{वेष्टकारीं लोहखड्गं कृष्णछागं च यत्नतः}%॥२५॥

\twolineshloka
{द्वारे विन्यस्य मुसलं रक्षितं रक्षपालकैः}
{षष्ठया देव्याधिष्ठितं च तद्गृहं चोत्सवैस्तथा}%॥२६॥

\twolineshloka
{एवंविभवसारेण कृत्वा तत्सूतिकागृहम्}
{तन्मध्ये प्रतिमा स्थाप्या सा चाप्यष्टविधा स्मृता}%॥२७॥

\twolineshloka
{काञ्चनी राजती ताम्री पैत्तली मृन्मयी तथा}
{वार्क्षी मणिमयी चैव वर्णकैर्लिखिता तथा}%॥२८॥

\twolineshloka
{सर्वलक्षणसम्पूर्णा पर्यङ्के चाष्टशल्यके}
{प्रतप्तकाञ्चनाभासां महार्हां सुतपस्विनीम्}%॥२९॥

\twolineshloka
{प्रसूतां च प्रसुप्तां च स्थापयन्मञ्चकोपरि}
{मां तत्र बालकं सुप्तं पर्यङ्के स्तनपायिनम्}%॥३०॥

\twolineshloka
{श्रीवत्सवक्षसं शान्तं नीलोत्पलदलच्छविम्}
{यशोदा तत्र चैकस्मिन् प्रदेशे सूतिकागृहे}%॥३१॥

\twolineshloka
{तद्वच्च कल्पयेत् पार्थ प्रसूतां वरकन्यकाम्}
{तथैव मम पार्श्वस्थाः कृताञ्जलिपुटा नृप}%॥३२॥

\twolineshloka
{देवा ग्रहास्तथा नागा यक्षविद्याधराभराः}
{प्रणताः पुष्पमालाग्रचारुहस्ताः सुरासुराः}%॥३३॥

\twolineshloka
{सञ्चरन्त इवाऽऽकाशे प्रहारैरुदितोदितैः}
{वसुदेवोऽपि तत्रैव खड्गचर्मधरः स्थितः}%॥३४॥

\twolineshloka
{कश्यपो वसुदेवोऽयमदितिश्चैव देवकी}
{शेगे वै बलदेवोऽयं यशोदादितिरन्वभूत्}%॥३५॥

\twolineshloka
{नन्दः प्रजापतिर्दक्षो गर्गश्चापि चतुर्मुखः}
{गोप्यश्चाप्सरसश्चैव गोपाश्चापि दिवौकसः}%॥३६॥

\twolineshloka
{एषोऽवतारो राजेन्द्र कंसोऽयं कालनेमिजः}
{तत्र कंसनियुक्ताश्च मोहिता योगनिद्रया}%॥३७॥

\twolineshloka
{गोधेनुकुञ्जराश्चैव दानवाः शस्त्रपाणयः}
{नृत्यतश्चाप्सरोभिस्ते गन्धर्वा गीततत्पराः}%॥३८॥

\twolineshloka
{लेखनीयश्च तत्रैव कालियो यमुनाह्रदे}
{इत्येवमादि यत्किञ्चिद्विद्यते चरितं मम}%॥३९॥

\twolineshloka
{लेखयित्वा प्रयत्नेन पूजयेद्भक्तितत्परः}
{रम्यमेवं बीजपूरैः पुष्पमालादिशोभितम्}%॥४०॥

\threelineshloka
{कालदेशोद्भवैः पुष्पैः फलैश्चापि युधिष्ठिर}
{पाद्यार्घ्यैः पूजयेद्भक्त्या गन्धपुष्पाक्षतैः सह}
{मन्त्रेणानेन कौन्तेय देवकीं पूजयेन्नरः}%॥४१॥

\fourlineindentedshloka
{गायद्भिः किन्नराद्यैः सततपरिवृता वेणुवीणानिनादैः}
{भृङ्गारादर्शकुम्भप्रवरवृतकरैः किङ्करैः सेव्यमाना}
{पर्यङ्के स्वास्तृते यामुदिततरमुखी पुत्रिणी सम्यगास्ते}
{सा देवी देवमाता जयतु च ससुता देवकी कान्तरूपा}%॥४२॥


\twolineshloka
{पादावभ्यञ्जयन्ती श्रीदेवक्याश्चरणान्तिके}
{निषण्णा पङ्कजे पूज्या दिव्यगन्धातुलेपनैः}%॥४३॥


\twolineshloka
{पङ्कजैः पूजयेद्देवीं नमो देव्यै श्रिया इति}
{देववत्से नमस्तेऽस्तु कृष्णोत्पादनतत्परा}%॥४४॥


\twolineshloka
{पापक्षयकरा देवी तुष्टिं यातु मयाऽर्चिता}
{प्रणवादिनमोऽन्तं च पृथङ्नामानुकीर्तनम्}%॥४५॥


\twolineshloka
{कुर्यात्पूजा विधिज्ञश्च सर्वपापापनुत्तये}
{देवक्यै वसुदेवाय वासुदेवाय चैव हि}%॥४६॥


\twolineshloka
{बलदेवाय नन्दाय यशोदायै पृथक् पृथकू}
{क्षीरादिस्नपनं कृत्वा चन्दनेनानुलेपयेत्}%॥४७॥


\twolineshloka
{विध्यन्तरमपीच्छन्ति केचिदत्रैव सूरयः}
{चन्द्रोदये शशाङ्काय अर्घ्यं दत्त्वा हरिं स्मरन्}%॥४८॥


\twolineshloka
{अनघं वामनं शौरिं वैकुण्ठं पुरुषोत्तमम्}
{वासुदेवं हृषीकेशं माधवं मधुसूदनम्}%॥४९॥


\twolineshloka
{वराहं पुण्डरीकाक्षं नृसिंहं ब्रह्मणः प्रियम्}
{समस्तस्यापि जगतः सृष्टिस्थित्यन्तकारकम्}%॥५०॥


\twolineshloka
{अनादिनिधनं विष्णुः त्रैलोक्येशं त्रिविक्रमम्}
{नारायणं चतुर्बाहुं शङ्खचक्रगदाधरम्}%॥५१॥


\twolineshloka
{पीताम्बरधरं नित्यं वनमालाविभूषितम्}
{श्रीवत्साक्षं जगत्सेतुं श्रीपतिं श्रीधरं हरिम्}%॥५२॥


\twolineshloka
{योगेश्वराय देवाय योगिनां पतये नमः}
{योगोद्भवाय नित्याय गोविन्दाय नमो नमः}%॥५३॥


\twolineshloka
{यज्ञेश्वराय देवाय तथा यज्ञोद्भवाय च}
{यज्ञानां पतये नाथ गोविन्दाय नमो नमः}%॥५४॥


\twolineshloka
{विश्वेश्वराय विश्वाय तथा विश्वोद्भवाय च}
{विश्वस्य पतये तुभ्यं गोविन्दाय नमो नमः}%॥५५॥


\twolineshloka
{जगन्नाथ नमस्तुभ्यं संसारभयनाशन}
{जगदीशाय देवाय भूतानां पतये नमः}%॥५६॥


\twolineshloka
{धर्मेश्वराय धर्माय सम्भवाय जगत्पते}
{धर्मज्ञाय च देवाय गोविन्दाय नमो नमः}%॥५७॥


\twolineshloka
{एताभ्यां चैव मन्त्राभ्यां नैवेद्यं शयनं तथा}
{चन्द्रायार्घ्यं च मन्त्रेण अनेनैवाथ दापयेत्}%॥५८॥


\twolineshloka
{क्षीरोदार्णवसम्भूत अत्रिगोत्रसमुद्भव}
{गृहाणार्घ्यं शशाङ्केश रोहिण्या सहितो मम}%॥५९॥


\twolineshloka
{ज्योस्नापते नमस्तुभ्यं ज्योतिषां पतये नमः}
{नमस्ते रोहिणीकान्त अर्घ्यं नः प्रतिगृह्यताम्}%॥६०॥


\twolineshloka
{स्थण्डिले स्थापयेद्देवं शशाङ्कं रोहिणीयुतम्}
{देवक्या वसुदेवं च नन्दं चैव यशोदया}%॥६१॥


\twolineshloka
{बलदेवं मया सार्धं भक्त्या परमया नृप}
{सम्पूज्य विधिवद्देहि किं नाऽऽप्नोत्यतिदुर्लभम्}%॥६२॥


\twolineshloka
{एकादशीनां विंशत्यः कोटयो याः प्रकीर्तिताः}
{ताभिः कृष्णाष्टमी तुल्या ततोऽनन्तचतुर्दशी}%॥६३॥


\twolineshloka
{अर्धरात्रे वसोर्धारां पातयेद् द्रव्यसर्पिषा}
{ततो वर्धापयेन्नालं षष्ठीनामादिकं मम}%॥६४॥


\twolineshloka
{कर्तव्यं तत्क्षणाद्रात्रौ प्रभाते नवमीदिने}
{यथा मम तथा कार्यो भगवत्या महोत्सवः}%॥६५॥


\twolineshloka
{ब्राह्मणान् भोजयेद्भक्त्या तेभ्यो दद्याच्च दक्षिणाम्}
{हिरण्यं मेदिनीं गावो वासांसि कुसुमानि च}%॥६६॥


\twolineshloka
{यद्यदिष्टतमं तत्तत्कृष्णो मे प्रीयतामिति}
{यं देवं देवकी देवीं वसुदेवादजीजनत्}%॥६७॥


\twolineshloka
{भौमस्य ब्रह्मणो गुप्त्यै तस्मै ब्रह्मात्मने नमः}
{नमस्ते वासुदेवाय गोब्राह्मणहिताय च}%॥६८॥


\twolineshloka
{शान्तिरस्तु शिवं चास्तु इत्युक्त्वा मां विसर्जयेत्}
{ततो बन्धुजनौघं च दीनानाथांश्च भोजयेत्}%॥६९॥


\twolineshloka
{भोजयित्वा सुशान्तांस्तान् स्वयं भुञ्जीत वाग्यतः}
{एवं यः कुरुते देव्या देवक्याः सुमहोत्सवम्}%॥७०॥


\twolineshloka
{प्रतिवर्षं विधानेन मद्भक्तो धर्मनन्दन}
{नरो वा यदि वा नारी यथोक्तं लभते फलम्}%॥७१॥


\twolineshloka
{पुत्रसन्तानमारोग्यं सौभाग्यमतुलं लभेत्}
{इह धर्मरतिर्भूत्वा मृतो वैकुण्ठमाप्नुयात्}%॥७२॥


\twolineshloka
{तत्र देवविमानेन वर्षलक्षं युधिष्ठिर}
{भोगान्नानाविधान् भुक्त्वा पुण्यशेषादिहागतः}%॥७३॥


\twolineshloka
{सर्वकामसमृद्धे च सर्वाशुभविवर्जिते}
{कुले नृपतिशीलानां जायते हृच्छयोपमः}%॥७४॥


\twolineshloka
{यस्मिन सदैव देशे तु लिखितं तु पटार्पितम्}
{मम जन्मदिनं भक्त्या सर्वालङ्कारभूषितम्}%॥७५॥


\twolineshloka
{पूज्यते पाण्डवश्रेष्ठ जनैरुत्सवसंयुतैः}
{परचक्रभयं तत्र न कदाऽपि भवेत्पुनः}%॥७६॥


\twolineshloka
{पर्जन्यः कामवर्षी स्यादीतिभ्यो न भयं भवेत्}
{गृहे वा पूज्यते यत्र देवक्याश्चरितं मम}%॥७७॥


\twolineshloka
{तत्र सर्वं समृद्धं स्यान्नोपसर्गादिकं भवेत्}
{पशुभ्यो नकुलाद्व्यालात्पापरोगाच्च पातकात्}%॥७८॥

\threelineshloka
{राजतश्चोरतो वाऽपि न कदाचिद्भयं भवेत्}
{संसर्गेणापि यो भक्त्या व्रतं पश्येदनाकुलम्}
{सोऽपि पापविनिर्मुक्तः प्रयाति हरिमन्दिरम्}%॥७९॥

\fourlineindentedshloka
{जन्माष्टमीं जनमनोनयनाभिरामा}
{पापापहां सपदि नन्दितनन्दगोपाम्}
{यो देवकी सुतयुतां च भजेद्धि भक्त्या}
{पुत्रानवाप्य समुपैति पदं स विष्णोः}%॥८०॥

\centerline{॥इति भविष्योत्तरे जन्माष्टमीव्रतकथा॥}


\sect{शिष्टाचारप्राप्ता जन्माष्टमीव्रतकथा}

\uvacha{व्यास उवाच}

\twolineshloka
{निवृत्ते भारते युद्धे कृतशौचो युधिष्ठिरः}
{उवाच वाक्यं धर्मात्मा कृष्णं देवकिनन्दनम्}%॥१॥

\uvacha{युधिष्ठिर उवाच}
\twolineshloka
{त्वत्प्रसादात्तु गोविन्द निहताः शत्रवो रणे}
{कर्णश्च निहतः सैन्ये त्वत्प्रसादात्किरीटिना}%॥२॥

\twolineshloka
{जेता को युधि भीष्मस्य यस्य मृत्युर्न विद्यते}
{अजेयोऽपि जितः सोऽपि त्वत्प्रसादाज्जनार्दन}%॥३॥

\twolineshloka
{प्राप्तं निष्कण्टकं राज्यं कृत्वा कर्म सुदुष्करम्}
{आचारो दण्डनीतिश्च राजधर्माः क्रियान्विताः}%॥४॥

\twolineshloka
{अधुना श्रोतुमिच्छामि शुभं जन्माष्टमीव्रतम्}
{जन्माष्टमी व्रतं ब्रूहि विस्तरेण ममाच्युत}%॥५॥

\onelineshloka*
{कुतः काले समुत्पन्नं किं पुण्यं को विधिः स्मृतः}
\uvacha{श्रीकृष्ण उवाच}
\onelineshloka
{शृणु राजन्प्रवक्ष्यामि व्रतानामुत्तमं व्रतम्}%॥६॥

\twolineshloka
{यतः प्रभृति विख्यातं फलेन विधिनान्वितम्}
{राजवंशसमुत्पन्नैर्दैत्यानीकैः सुपीडिता}%॥७॥

\twolineshloka
{धरा भारसमाक्रान्ता ब्रह्माणं शरणं ययौ}
{ज्ञात्वा तदा प्रभुर्ब्रह्मा भूमेर्भारं समाहितः}%॥८॥

\twolineshloka
{श्वेतदीपं समागत्य सर्वदेवसमन्वितः}
{समाहितमतिर्ब्रह्मा मां तुष्टाव विशां पते}%॥९॥

\twolineshloka
{स्तुत्या तयाऽहं सम्प्रीतस्तेषां दृग्गोचरोऽभवम्}
{दृष्ट्वा मां प्रणिपत्याऽऽशु भक्तिभावसमन्विताः}%॥१०॥

\twolineshloka
{ब्रह्माणमग्रतः कृत्वा तुष्टाः सर्वे दिवौकसः}
{विजिज्ञपुर्महराज भूमिभारापनुत्तये}%॥११॥

\twolineshloka
{उपधार्य तदा तेषां वचनं चान्वचिन्तयम्}
{केनोपायन हन्तव्या दानवाः क्षत्रियोद्भवाः}%॥१२॥

\twolineshloka
{स्वधर्मनिरताः सर्वे महाबलपराक्रमाः}
{ततो निश्चित्य मनसा ब्रह्माणमहमब्रुवम्}%॥१३॥

\twolineshloka
{वसुदेवो देवकी च प्रजाकामौ पुरा नृप}
{भक्त्या मां भजमानौ तौ तप्तवन्तौ महत्तपः}%॥१४॥

\twolineshloka
{तयोः प्रसन्नः सुप्रीतो याचतं वरमुत्तमम्}
{अब्रुवं तावपि ततो वरयामासतुः किल}%॥१५॥

\twolineshloka
{यदि देव प्रसन्नोऽसि त्वादृशौ नौ भवेत्सुतः}
{तथेति च मया ताभ्यामुक्तं प्रीतेन चेतसा}%॥१६॥

\twolineshloka
{तत्कामपूरणार्थाय सम्भविष्याम्यहं तयोः}
{दिवौकसोऽपि स्वांशेन सम्भवन्तु सुरस्त्रियः}%॥१७॥

\twolineshloka
{योगमाया च नन्दस्य यशोदायां भविष्यति}
{देवक्या जठरे गर्भमनन्तं धाम मामकम्}%॥१८॥

\twolineshloka
{सन्निकृष्य च सा तूर्णं रोहिण्या जठरं नयेत्}
{इति सन्दिश्य तान् सर्वानहमन्तर्हितोऽभवम्}%॥१९॥

\twolineshloka
{ततो देवैः समं ब्रह्मा तां दिशं प्रणिपत्य च}
{आश्वास्य च महीं देवीं वरधाम्नि जगाम ह}%॥२०॥

\threelineshloka
{ततोऽहं देवकीगर्भमाविशं स्वेन तेजसा}
{हतेषु षट्सु बालेषु देवक्या औग्रसेनिना}
{कारागृहस्थितायाश्च वसुदेवेन वै सह}%॥२१॥

\twolineshloka
{गतेऽर्धरात्रसमये सुप्ते सर्वजने निशि}
{भाद्रे मास्यसिते पक्षेऽष्टम्यां ब्रह्मर्क्षसंयुजि}%॥२२॥

\twolineshloka
{सर्वग्रहशुभे काले प्रसन्नहृदयाशये}
{आविरासं निजेनैव रुपेण ह्यवनीपते}%॥२३॥

\twolineshloka
{वसुदेवोऽपि मां दृष्ट्वा हर्षशोकसमन्वितः}
{भीतः कंसादतितरां तुष्टाव च कृताञ्जलिः}%॥२४॥

\onelineshloka*
{पुनः पुनः प्रणम्याथ प्रार्थयामास सादरम्}
\uvacha{वसुदेव उवाच}
\onelineshloka
{अलौकिकमिदं रूपं दुर्दर्शं योगिनामपि}%॥२५॥

\twolineshloka
{यत्तेजसाऽरिष्टगृहमभवत्सम्प्रकाशितम्}
{उद्धिजे भगवत्कंसाद्यो मे बालानघातयत्}%॥२६॥

\twolineshloka
{उपसंहर तस्माच्च एतद्रूपमलौकिकम्}
{शङ्खचक्रगदापद्मलसत्कौस्तुभमालिनम्}%॥२७॥

\twolineshloka
{किरीटहारमुकुटकेयूरवलयाङ्कितम्}
{तडिद्वसनसंवीतक्वणत्काञ्चनमेखलम्}%॥२८॥

\twolineshloka
{स्फुरद्राजीवताम्राक्षं स्निग्धाञ्जनसमप्रभम्}
{महामरकतस्वच्छं कोटिसूर्यसमप्रभम्}%॥२९॥

\uvacha{कृष्ण उवाच}
\twolineshloka
{एवं सम्प्रार्थितो राजन्वसुदेवेन वै तदा}
{तेनैव निजरूपेण भूत्वाऽहं प्राकृतः शिशुः}%॥३०॥

\twolineshloka
{नय मां गोकुलमिति वसुदेवमचोदयम्}
{समादायागमत्सोऽपि नन्दगोकुलमञ्जसा}%॥३१॥

\twolineshloka
{द्वाराण्यपाकृतान्यासन्मत्प्रभावात्स्वयं प्रभो}
{ददौ मार्गं च कालिन्दीजलकल्लोलमालिनी}%॥३२॥

\twolineshloka
{ततो यशोदाशयने न्यस्य माऽऽनकदुन्दुभिः}
{तत्पर्यङ्के स्थितां गृह्य दारिकामगमत्पुनः}%॥३३॥

\twolineshloka
{द्वाराणि पिहितान्यासन् पूर्ववन्निगडं ततः}
{विन्यस्य पादयोरास्ते शयने न्यस्य दारिकाम्}%॥३४॥

\twolineshloka
{ततो रुरोद महता स्वरेणाऽऽपूर्य सा दिशः}
{तस्या रुदितशब्देन उत्थिता रक्षका गृहात्}%॥३५॥

\twolineshloka
{कंसायाऽऽगत्य चाचख्युः प्रसूता देवकीति च}
{सोऽपि तल्पात्समुत्थाय भयेनातीव विह्वलः}%॥३६॥

\twolineshloka
{जगाम सूतिकागेहं देवक्याः प्रस्खलन्पथि}
{दारिकां शयनाद्गृह्य रुदत्याश्चैव स्वस्वसुः}%॥३७॥

\twolineshloka
{अपोथयच्छिलापृष्ठे साऽपि तस्य कराच्च्युता}
{उवाच कंसमाभाष्य देवी ह्याकाशगा सती}%॥३८॥

\twolineshloka
{किं मया हतया मन्द जातः कुत्रापि ते रिपुः}
{इत्युक्तः सोऽप्यभूत्कंसः परमोद्विग्नमानसः}%॥३९॥

\twolineshloka
{आज्ञापयामास ततो बालानां कदनाय वै}
{दानवा अपि बालानां कदनं चक्रुरुद्यताः}%॥४०॥

\twolineshloka
{वनेषूपवने चैव पुरग्रामव्रजेष्वपि}
{अहं च गोकुले स्थित्वा पूतनां बालघातिनीम्}%॥४१॥

\twolineshloka
{स्तनं दातुं प्रवृत्तां च प्राणैः सममशोषयम्}
{तृणावर्तबकारिष्टान् धेनुकं केशिनं तथा}%॥४२॥

\twolineshloka
{अन्यानपि खलान् हत्वा स्वप्रभावमदर्शयम्}
{ततश्च मथुरां गत्वा हत्वा कंसादिदानवान्}%॥४३॥

\twolineshloka
{ज्ञातीनां परमं हर्षं कृतवानस्मि सादरम्}
{देवकीवसुदेवौ च परिष्वज्य मुदा मम}%॥४४॥

\twolineshloka
{आनन्दर्जैर्जलैर्मूर्ध्नि सेचयामासतुर्नृप}
{तस्मिन् रङ्गवरे मल्लान् हत्वा चाणूरमुख्यकान्}%॥४५॥

\twolineshloka
{गजं कुवलयापीडं कंसभ्रातॄननेकशः}
{एवं हतेऽसुरे कंसे सर्वलोकैककण्टके}%॥४६॥

\twolineshloka
{अन्येषु दुष्टदैत्येषु सर्वलोकभयङ्करम्}
{लोकाः समुत्सुकाः सर्वे मांसमेत्योचुरादृताः}%॥४७॥

\twolineshloka
{कृष्ण कृष्ण महायोगिन् भक्तानामभयप्रद}
{प्रलयात्पाहि नो देव शरणागतवत्सलः}%॥४८॥

\twolineshloka
{अनाथनाथ सर्वज्ञ सर्वभूतहिते रत}
{किञ्चिद् विज्ञाप्यतेऽस्माभिस्तन्नो वक्तुं त्वमर्हसि}%॥४९॥

\twolineshloka
{तव जन्मदिनं लोके न ज्ञातं केनचित्क्वचित्}
{ज्ञात्वा च तत्त्वतः सर्वे कुर्मो वर्धापनोत्सवम्}%॥५०॥

\twolineshloka
{तेषां दृष्ट्वा तु तां भक्तिं श्रद्धामपि च सौहृदम्}
{मया जन्मदिनं तेभ्यः ख्यातं निर्मलचेतसा}%॥५१॥

\twolineshloka
{श्रुत्वा तेऽपि तथा चक्रुर्विधिना येन तच्छृणु}
{पार्थ तद्दिवसे प्राप्ते दन्तधावनपूर्वकम्}%॥५२॥

\twolineshloka
{स्नात्वा पुण्यजले शुद्धे वाससी परिधाय च}
{निर्वर्त्यावश्यकं कर्म व्रतसङ्कल्पमाचरेत्}%॥५३॥

\twolineshloka
{अद्य स्थित्वा निराहारः श्वोभूते तु परेऽहनि}
{भोक्ष्यामि पुण्डरीकाक्ष शरणं मे भवाव्यय}%॥५४॥

\twolineshloka
{गृहीत्वा नियमं चैव सम्पाद्यार्चनसाधनम्}
{मण्डपं शोभनं कृत्वा फलपुष्पादिभिर्युतम्}%॥५५॥

\twolineshloka
{तस्मिन्मां पूजयेद्भक्त्या गन्धपुष्पादिभिः क्रमात्}
{उपचारैः षोडशभिर्द्वादशाक्षरविद्यया}%॥५६॥

\twolineshloka
{सद्यःप्रसूतां जननीं वसुदेवं च मारिषः}
{बलदेवसमायुक्तां रोहिणीं गुणशोभिनीम्}%॥५७॥

\twolineshloka
{नन्दं यशोदां गोपीश्च गोपान् गाश्चैव सर्वशः}
{गोकुलं यमुनां चैव योगमायां च दारिकाम्}%॥५८॥

\twolineshloka
{यशोदाशयने सुप्तां सद्योजातां वरप्रभाम्}
{एवं सम्पूजयेत्सम्यङ्नाममन्त्रैः पृथक्पृथक्}%॥५९॥

\twolineshloka
{सुवर्णरौप्यताम्रारमृदादिभिरलङ्कृताः}
{काष्ठपाषाणरचिताश्चित्रमय्योऽथ लेखिताः}%॥६॥

\twolineshloka
{प्रतिमा विविधाः प्रोक्तास्तासु चान्यतमां यजेत्}
{रात्रौ जागरणं कुर्याद्गीतनृत्यादिभिः सह}%॥६॥

\twolineshloka
{पुराणैः स्तोत्रपाठैश्च जातनामादिसूत्सवैः}
{श्वभूते पारणं कुर्याद् द्विजान् सम्भोज्य यत्नतः}%॥६॥

\twolineshloka
{एवं कृते महाराज व्रतानामुत्तमे व्रते}
{सर्वान्कामानवाप्नोति विष्णुलोके महीयते}%॥६३॥

\twolineshloka
{मोहान्न कुरुते यस्तु याति संसारगह्वरे}
{तस्मात्कुर्वन्प्रयत्नेन निष्पापो जायते नरः}%॥६४॥

\twolineshloka
{अत्रैवोदाहरन्तीममितिहासं पुरातनम्}
{अङ्गदेशोद्भवो राजा मित्रजिन्नाम नामतः}%॥६५॥

\twolineshloka
{तस्य पुत्रो महातेजः सत्यजित्सत्पथे स्थितः}
{पालयामास धर्मज्ञो विधिवद्रञ्जयन्प्रजाः}%॥६६॥

\twolineshloka
{तस्यैवं वर्तमानस्य कदाचिद्दैवयोगतः}
{पापण्डैः सह संवासो बभूव बहुवासरम्}%॥६७॥

\twolineshloka
{तत्संसर्गात्स नृपतिरधर्मनिरतोऽभवत्}
{वेदशास्त्रपुराणानि विनिन्द्य बहुशो नृप}%॥६८॥

\twolineshloka
{ब्राह्मणेषु तथा धर्मे विद्वेषं परमं गतः}
{एवं बहुतिथे काले गते भरतसत्तम}%॥६९॥

\twolineshloka
{कालेन निधनप्राप्तो यमदूतवशं गतः}
{बद्धा पाशैर्नीयमानो यमदूतैर्यमान्तिकम्}%॥७०॥

\twolineshloka
{पीडितस्ताड्यमानोऽसौ दुष्टसङ्गवशं गतः}
{नरके पतितः पापो यातनां बहुवत्सरम्}%॥७१॥

\twolineshloka
{भुक्त्वा पापस्य शेषेण पैशाची योनिमास्थितः}
{तृषाक्षुधासमाक्रान्तो भ्रमन्स मरुधन्वसु}%॥७२॥

\twolineshloka
{कस्यचित्त्वथ वैश्यस्य देहमाविश्य संस्थितः}
{सह तेनैव सम्प्राप्तो मथुरा पुण्यदां पुरीम्}%॥७३॥

\twolineshloka
{तत्रत्यैरक्षकैः सोऽथ तद्देहात्तु बहिष्कृतः}
{बभ्राम विपिने सोऽपि ऋषीणामाश्रमेष्वपि}%॥७४॥

\twolineshloka
{कदाचिद् दैवयोगेन मम जन्माष्टमीदिने}
{क्रियमाणां महापूजां व्रतिभिर्मुनिभिर्द्विजैः}%॥७५॥

\twolineshloka
{रात्रौ जागरणं चैव नामसङ्कीर्तनादिभिः}
{ददर्श सर्वं विधिवच्छुश्राव च हरेः कथाः}%॥७६॥

\twolineshloka
{निष्पापस्तत्क्षणादेव शुद्धनिर्मलमानसः}
{प्रेतदेहं समुत्सृज्य विष्णुलोकं विमानतः}%॥७७॥

\twolineshloka
{मम दूतैः समानीतो दिव्यभोगसमन्वितः}
{मम सान्निध्यमापन्नो व्रतस्यास्य प्रभावतः}%॥७८॥

\twolineshloka
{नित्यमेव व्रतं चैतत् पुराणे सार्वकालिकम्}
{गीयते विधिवत्सम्यङ्मुनिभिस्तत्त्वदर्शिभिः}%॥७९॥

\threelineshloka
{सार्वकालिकमेवैतत्कृत्वा कामानवाप्नुयात्}
{एतत्ते सर्वमाख्यातं व्रतानामुत्तमं व्रतम्}
{मम सान्निध्यकृद्राजन्किं भूयः श्रोतुमिच्छसि}%॥८०॥

\centerline{॥इति भविष्ये जन्माष्टमीव्रतकथा}


\sect{व्रतोद्यापनम्}

\uvacha{युधिष्ठिर उवाच}
\twolineshloka
{उद्यापनविधिं ब्रूहि सर्वदेव दयानिधे}
{येन सम्पूर्णतां याति व्रतमेतदनुत्तमम्}

\uvacha{श्रीकृष्ण उवाच}
\twolineshloka
{पूर्णां तिथिमनुप्राप्य वित्तचित्तादिसंयुतः}
{पूर्वेद्युरेकभक्ताशी स्वपेन्मां संस्मरन्हदि}

\twolineshloka
{प्रातरुत्थाय संस्मृत्य पुण्यश्लोकान् समाहितः}
{निर्वर्त्यावश्यकं कर्म ब्राह्मणान्स्वस्ति वाचयेत्}

\twolineshloka
{गुरुमानीय धर्मज्ञं वेदवेदाङ्गपारगम्}
{वृणुयादृत्विजश्चैव वस्त्रालङ्करणादिभिः}

\twolineshloka
{पलेन वा तदर्धेन तदर्धार्धेन वा पुनः}
{शक्त्या वाऽपि नृपश्रेष्ठ वित्तशाठ्यविवर्जितः}

\twolineshloka
{सौवर्णीं प्रतिमां कुर्यात्पाद्यार्घ्याचमनीयकम्}
{पात्रं सम्पाद्य विधिवत्पूजोपकरणं तथा}

\twolineshloka
{गोचर्ममात्रं संलिप्य मध्ये मण्डलमाचरेत्}
{ब्रह्माद्या देवतास्तत्र स्थापयित्वा प्रपूजयेत्}

\twolineshloka
{मण्डपं रचयेत्तत्र कदलीस्तम्भमण्डितम्}
{चतुर्द्वारसमोपेतं फलपुष्पादिशोभितम्}

\twolineshloka
{वितानं तत्र बध्नीयाद्विचित्रं चैव शोभनम्}
{मण्डले स्थापयेत्कुम्भं ताम्रं वा मृन्मयं शुचिम्}

\twolineshloka
{तस्योपरि न्यसेत्पात्रं राजतं वैष्णवं तु वा}
{वाससाऽऽच्छाद्य कौन्तेय पूजयेत्तत्र मां बुधः}

\twolineshloka
{उपचारैः षोडशभिर्मन्त्रैरेतैः समाहितः}
{ध्यात्वाऽऽवाह्यामृतीकृत्य स्वागतादिभिरादरात्}

\threelineshloka
{ध्यायेच्चतुर्भुजं देवं शङ्खचक्रगदाधरम्}
{पीताम्बरयुगोपेतं लक्ष्मीयुक्तं विभूषितम्}
{लसत्कौस्तुभशोभाढ्यं मेघश्यामं सुलोचनम्}
ध्यानम्॥

\twolineshloka
{आगच्छ देवदेवेश जगद्योने रमापते}
{शुद्धे ह्यस्मिन्नधिष्ठाने सन्निधेहि कृपां कुरु}
आवाहनम्॥

\twolineshloka
{देवदेव जगन्नाथ गरुडासनसंस्थित}
{गृहाण चाऽऽसनं दिव्यं जगद्धातर्नमोऽस्तु ते}
आसनम्॥

\twolineshloka
{नानातीर्थाहृतं तोयं निर्मलं पुष्षमिश्रितम्}
{पाद्यं गृहाण देवेश विश्वरूप नमोऽस्तु ते}
पाद्यम्॥

\twolineshloka
{गङ्गादिसर्वतीर्थेभ्यो भक्त्याऽऽनीतं सुशीतलम्}
{गन्धपुष्पाक्षतोपेतं गृहाणार्घ्यं नमोऽस्तु ते}
अर्घ्यम्॥

\twolineshloka
{कृष्णावेणीसमुद्भूतं कालिन्दीजलसंयुतम्}
{गृहाणाऽऽचमनं देव विश्वकाय नमोऽस्तु ते}
आचमनम्॥

\twolineshloka
{दधि क्षौद्रं घृतं शुद्धं कपिलायाः सुगन्धि यत्}
{सुस्वादु मधुरं शौरे मधुपर्कं गृहाण मे}
मधुपर्कम्॥
पुनराचमनम्॥

\twolineshloka
{पञ्चामृतेन स्नपनं करिष्यामि सुरोत्तम}
{क्षीरोदधिनिवासाय लक्ष्मीकान्ताय ते नमः}
पञ्चामृतस्नानम्॥

\twolineshloka
{मन्दाकिनी गौतमी च यमुना च सरस्वती}
{ताभ्यः स्नानार्थमानीतं गृहाण शिशिरं जलम्}
स्नानम्॥
पुनराचमनम्॥

\twolineshloka
{शुद्धजाम्बूनदप्रख्ये तडिद्भासुररोचिषी}
{मयोपपादिते तुभ्यं वाससी प्रतिगृह्यताम्}
वस्त्रयुग्मम्॥

यज्ञोपवीतमिति यज्ञोपवीतम्॥

\twolineshloka
{किरीटकुण्डलादीनि काञ्चीवलययुग्मकम्}
{कौस्तुभं वनमालां च भूषणानि भजस्व मे}
भूषणानि॥

\twolineshloka
{मलयाचलसम्भूतं घनसारं मनोहरम्}
{हृदयानन्दनं चारु चन्दनं प्रतिगृह्यताम्}
चन्दनम्॥

अक्षताश्च सुरश्रेष्ठेति कुङ्कुमाक्षतान्।

\twolineshloka
{मालतीचम्पकादीनि यूथिकाबकुलानि च}
{तुलसीपत्रमिश्राणि गृहाण सुरसत्तम}
पुष्पाणि॥


अथाङ्गपूजा-

अघनाशनाय नमः — पादौ पूजयामि।\\
वामनाय नमः — गुल्फौ पूजयामि।\\
शौरये नमः — जङ्घे पूजयामि।\\
वैकुण्ठवासिने नमः — ऊरू पूजयामि।\\
पुरुषोत्तमाय नमः — मेढ्रं पूजयामि।\\
वासुदेवाय नमः — कटिं पूजयामि।\\
हृषीकेशाय नमः — नाभिं पूजयामि।\\
माधवाय नमः — हृदयं पूजयामि।\\
मधुसूदनाय नमः — कण्ठं पूजयामि।\\
वराहाय नमः — बाहून् पूजयामि।\\
नृसिंहाय नमः — हस्तौ पूजयामि।\\
दैत्यसूदनाय नमः — मुखं पूजयामि।\\
दामोदराय नमः — नासिकां पूजयामि।\\
पुण्डरीकाक्षाय नमः — नेत्रे पूजयामि।\\
गरुडध्वजाय नमः — श्रोत्रे पूजयामि।\\
गोविन्दाय नमः — ललाटं पूजयामि।\\
अच्युताय नमः — शिरः पूजयामि।\\
कृष्णाय नमः — सर्वाङ्गं पूजयामि॥\\


अथ परिवारदेवतापूजा—

\twolineshloka
{देवकीं वसुदेवं च रोहिणीं सबलां तथा}
{सात्यकिं चोद्धवाक्रूरोग्रसेनादियादवान्}

\twolineshloka
{नन्दं यशोदां तत्कालप्रसूतां गोपगोपिकाः}
{कालिन्दीं कालियं चैव पूजयेन्नाममन्त्रतः}

\twolineshloka
{वनस्पतिरसोद्भतं कालागुरुसमन्वितम्}
{धूपं गृहाण गोविन्द गुणसागर गोपते}
धूपम्॥

साज्यं त्रिवर्तिसंयुक्तम्॥\\
दीपम्॥

\twolineshloka
{शाल्योदनं पायसं च सिताघृतविमिश्रितम्}
{नानापक्कासंयुक्तं नैवेद्यं प्रतिगृह्यताम्}
नैवेद्यम्।
उत्तरापोशनम्॥

इदं फलमिति फलम्॥

पूगीफलमिति ताम्बूलम्॥

हिरण्यगर्भेति दक्षिणाम्॥

\twolineshloka
{नीराजयेत्ततो भक्त्या मङ्गलं समुदीरयन्}
{जयमङ्गलनिर्घोषैर्देवदेवं समर्चयेत्}
नीराजनम्॥


\twolineshloka
{दत्त्वा पुष्पाञ्जलिं चैव प्रदक्षिणपुरःसरम्}
{प्रणमेद्दण्डवद् भूभौ भक्तिप्रह्वः पुनः पुनः}

स्तुत्वा नानाविधैः स्तोत्रैः प्रार्थयेत जगत्पतिम्॥

\twolineshloka
{नमस्तुभ्यं जगन्नाथ देवकीतनय प्रभो}
{वसुदेवात्मजानन्त यशोदानन्दवर्धन}

\twolineshloka
{गोविन्द गोकुलाधार गोपीकान्त नमोऽस्तु ते}
{ततस्तु दापयेदर्घ्यमिन्दोरुदयतः शुचिः}

\twolineshloka
{कृष्णाय प्रथमं दद्याद् देवकीसहिताय च}
{नालिकेरेण शुद्धेन मुक्तमर्घ्यं विचक्षण}

कृष्णाय परया भक्त्या शङ्खे कृत्वा विधानतः॥

\twolineshloka
{जातः कंसवधार्थाय भूभारोत्तारणाय च}
{कौरवाणां विनाशाय दैत्यानां निधनाय च}

\twolineshloka
{पाण्डवानां हितार्थाय धर्मसंस्थापनाय च}
{गृहाणाऱ्या मया दत्तं देवकीसहितो हरे}
कृष्णार्घ्यमन्त्रः॥

\twolineshloka
{शङ्खे कृत्वा ततस्तोयं सपुष्पफलचन्दनम्}
{जानुभ्यामवनिं गत्वा चन्द्रायार्घ्यं निवेदयेत्}

\twolineshloka
{क्षीरोदार्णवसम्भूत अत्रिगोत्रसमुद्भव}
{गृहाणार्घ्यं मया दत्तं रोहिण्या सहित प्रभो}

\twolineshloka
{ज्योत्स्नापते नमस्तुभ्यं नमस्ते ज्योतिषां पते}
{नमस्ते रोहिणीकान्त गृहाणार्घ्यं नमोऽस्तु ते}
चन्द्रार्घ्यमन्त्रः॥

\twolineshloka
{इत्थं सम्पूज्य देवेशं रात्रौ जागरणं चरेत्}
{गीतनृत्यादिना चैव पुराणश्रवणादिभिः}

\twolineshloka
{प्रत्यूषे विमले स्नात्वा पूजयित्वा जगद्गुरुम्}
{पायसेन तिलाज्यैश्च मूलमन्त्रेण भक्तितः}

\twolineshloka
{अष्टोत्तरशतं हुत्वा ततः पुरुषसूक्ततः}
{इदं विष्णुरिति प्रोक्त्वा जुहुयाद्वै घृताहुतीः}

\twolineshloka
{होमशेषं समाप्याथ पूर्णाहुतिपुरःसरम्}
{आचार्यं पूजयेद्भक्त्या भूषणाच्छादनादिभिः}

\twolineshloka
{गामेकां कपिलां दद्याद् व्रतसम्पूर्तिहेतवे}
{पयस्विनीं सुशीलां च सवत्सां सगुणां तथा}

\twolineshloka
{स्वर्णशृङ्गीं रौप्यखुरांं कांस्यदोहनिकायुताम्}
{रत्नपुच्छां ताम्रपृष्ठीं स्वर्णघण्टासमन्विताम्}

\twolineshloka
{वस्त्रच्छन्नां दक्षिणाढ्यामेवं सम्पूर्णतां व्रजेत्}
{कपिलाया अभावे तु गौरन्याऽपि प्रदीयते}

\twolineshloka
{ततो दद्याच्च ऋत्विग्भ्योऽन्येभ्यश्चैव यथाविधि}
{शय्यां सोपस्करां दद्याद् व्रतसम्पूर्तिहेतवे}

\twolineshloka
{ब्राह्मणान्भोजयेत्पश्चादष्टौ तेभ्यश्च दक्षिणाम्}
{कलशानन्नसम्पूर्णान्दद्याच्चैव समाहितः}

\twolineshloka
{दीनान्धकृपणांश्चैव यथार्हं प्रतिपूजयेत्}
{प्राप्यानुज्ञां तथा तेभ्यो भुञ्जीत सह बन्धुभिः}

\twolineshloka
{एवं कृते महाराज व्रतोद्यापनकर्मणि}
{निष्पापस्तत्क्षणादेव जायते विबुधोपमः}

\twolineshloka
{पुत्रपौत्रसमायुक्तो धनधान्यसमन्वितः}
{भुक्त्वा भोगांश्चिरं कालमन्ते मम पुरं व्रजेत्}

॥इति श्रीभविष्यपुराणे कृष्णयुधिष्ठिरसंवादे जन्माष्टमीव्रतोद्यापनं सम्पूर्णम्॥

\endgroup

\sect{श्रीमद्भागवते महापुराणे दशमस्कन्धे पूर्वार्धे तृतीयोऽध्यायः}

\uvacha{श्रीशुक उवाच}

\twolineshloka
{अथ सर्वगुणोपेतः कालः परमशोभनः}
{यर्ह्येवाजनजन्मर्क्षं शान्तर्क्षग्रहतारकम्} %1

\twolineshloka
{दिशः प्रसेदुर्गगनं निर्मलोडुगणोदयम्}
{मही मङ्गलभूयिष्ठ पुरग्रामव्रजाकरा} %2

\twolineshloka
{नद्यः प्रसन्नसलिला ह्रदा जलरुहश्रियः}
{द्विजालिकुलसन्नाद स्तवका वनराजयः} %3

\twolineshloka
{ववौ वायुः सुखस्पर्शः पुण्यगन्धवहः शुचिः}
{अग्नयश्च द्विजातीनां शान्तास्तत्र समिन्धत} %4

\twolineshloka
{मनांस्यासन्प्रसन्नानि साधूनामसुरद्रुहाम्}
{जायमानेऽजने तस्मिन्नेदुर्दुन्दुभयः समम्} %5

\twolineshloka
{जगुः किन्नरगन्धर्वास्तुष्टुवुः सिद्धचारणाः}
{विद्याधर्यश्च ननृतुरप्सरोभिः समं मुदा} %6

\twolineshloka
{मुमुचुर्मुनयो देवाः सुमनांसि मुदान्विताः}
{मन्दं मन्दं जलधरा जगर्जुरनुसागरम्} %7

\threelineshloka
{निशीथे तम उद्भूते जायमाने जनार्दने}
{देवक्यां देवरूपिण्यां विष्णुः सर्वगुहाशयः}
{आविरासीद्यथा प्राच्यां दिशीन्दुरिव पुष्कलः}

\fourlineindentedshloka
{तमद्भुतं बालकमम्बुजेक्षणं}
{चतुर्भुजं शङ्खगदाद्युदायुधम्}
{श्रीवत्सलक्ष्मं गलशोभिकौस्तुभं}
{पीताम्बरं सान्द्रपयोदसौभगम्} %9

\fourlineindentedshloka
{महार्हवैदूर्यकिरीटकुण्डल-}
{त्विषा परिष्वक्तसहस्रकुन्तलम्}
{उद्दामकाञ्च्यङ्गदकङ्कणादिभिर्-}
{विरोचमानं वसुदेव ऐक्षत} %10

\fourlineindentedshloka
{स विस्मयोत्फुल्लविलोचनो हरिं}
{सुतं विलोक्यानकदुन्दुभिस्तदा}
{कृष्णावतारोत्सवसम्भ्रमोऽस्पृशन्}
{मुदा द्विजेभ्योऽयुतमाप्लुतो गवाम्} %11

\fourlineindentedshloka
{अथैनमस्तौदवधार्य पूरुषं}
{परं नताङ्गः कृतधीः कृताञ्जलिः}
{स्वरोचिषा भारत सूतिकागृहं}
{विरोचयन्तं गतभीः प्रभाववित्} %12

\uvacha{श्रीवसुदेव उवाच}


\twolineshloka
{विदितोऽसि भवान्साक्षात्पुरुषः प्रकृतेः परः}
{केवलानुभवानन्द स्वरूपः सर्वबुद्धिदृक्} %13

\twolineshloka
{स एव स्वप्रकृत्येदं सृष्ट्वाग्रे त्रिगुणात्मकम्}
{तदनु त्वं ह्यप्रविष्टः प्रविष्ट इव भाव्यसे} %14

\twolineshloka
{यथेमेऽविकृता भावास्तथा ते विकृतैः सह}
{नानावीर्याः पृथग्भूता विराजं जनयन्ति हि} %15

\twolineshloka
{सन्निपत्य समुत्पाद्य दृश्यन्तेऽनुगता इव}
{प्रागेव विद्यमानत्वान्न तेषामिह सम्भवः} %16

\fourlineindentedshloka
{एवं भवान्बुद्ध्यनुमेयलक्षणैर्-}
{ग्राह्यैर्गुणैः सन्नपि तद्गुणाग्रहः}
{अनावृतत्वाद्बहिरन्तरं न ते}
{सर्वस्य सर्वात्मन आत्मवस्तुनः} %17

\fourlineindentedshloka
{य आत्मनो दृश्यगुणेषु सन्निति}
{व्यवस्यते स्वव्यतिरेकतोऽबुधः}
{विनानुवादं न च तन्मनीषितं}
{सम्यग्यतस्त्यक्तमुपाददत्पुमान्} %18

\fourlineindentedshloka
{त्वत्तोऽस्य जन्मस्थितिसंयमान्विभो}
{वदन्त्यनीहादगुणादविक्रियात्}
{त्वयीश्वरे ब्रह्मणि नो विरुध्यते}
{त्वदाश्रयत्वादुपचर्यते गुणैः} %19

\fourlineindentedshloka
{स त्वं त्रिलोकस्थितये स्वमायया}
{बिभर्षि शुक्लं खलु वर्णमात्मनः}
{सर्गाय रक्तं रजसोपबृंहितं}
{कृष्णं च वर्णं तमसा जनात्यये} %20

\fourlineindentedshloka
{त्वमस्य लोकस्य विभो रिरक्षिषुर्-}
{गृहेऽवतीर्णोऽसि ममाखिलेश्वर}
{राजन्यसंज्ञासुरकोटियूथपैर्-}
{निर्व्यूह्यमाना निहनिष्यसे चमूः} %21

\fourlineindentedshloka
{अयं त्वसभ्यस्तव जन्म नौ गृहे}
{श्रुत्वाग्रजांस्ते न्यवधीत्सुरेश्वर}
{स तेऽवतारं पुरुषैः समर्पितं}
{श्रुत्वाधुनैवाभिसरत्युदायुधः} %22

\uvacha{श्रीशुक उवाच}


\twolineshloka
{अथैनमात्मजं वीक्ष्य महापुरुषलक्षणम्}
{देवकी तमुपाधावत्कंसाद्भीता सुविस्मिता} %23

\uvacha{श्रीदेवक्युवाच}


\fourlineindentedshloka
{रूपं यत्तत्प्राहुरव्यक्तमाद्यं}
{ब्रह्म ज्योतिर्निर्गुणं निर्विकारम्}
{सत्तामात्रं निर्विशेषं निरीहं}
{स त्वं साक्षाद्विष्णुरध्यात्मदीपः} %24

\fourlineindentedshloka
{नष्टे लोके द्विपरार्धावसाने}
{महाभूतेष्वादिभूतं गतेषु}
{व्यक्तेऽव्यक्तं कालवेगेन याते}
{भवानेकः शिष्यतेऽशेषसंज्ञः} %25

\fourlineindentedshloka
{योऽयं कालस्तस्य तेऽव्यक्तबन्धो}
{चेष्टामाहुश्चेष्टते येन विश्वम्}
{निमेषादिर्वत्सरान्तो महीयांस्}
{तं त्वेशानं क्षेमधाम प्रपद्ये} %26

\fourlineindentedshloka
{मर्त्यो मृत्युव्यालभीतः पलायन्}
{लोकान्सर्वान्निर्भयं नाध्यगच्छत्}
{त्वत्पादाब्जं प्राप्य यदृच्छयाद्य}
{सुस्थः शेते मृत्युरस्मादपैति} %27

\fourlineindentedshloka
{स त्वं घोरादुग्रसेनात्मजान्नस्-}
{त्राहि त्रस्तान्भृत्यवित्रासहासि}
{रूपं चेदं पौरुषं ध्यानधिष्ण्यं}
{मा प्रत्यक्षं मांसदृशां कृषीष्ठाः} %28

\twolineshloka
{जन्म ते मय्यसौ पापो मा विद्यान्मधुसूदन}
{समुद्विजे भवद्धेतोः कंसादहमधीरधीः} %29

\twolineshloka
{उपसंहर विश्वात्मन्नदो रूपमलौकिकम्}
{शङ्खचक्रगदापद्म श्रिया जुष्टं चतुर्भुजम्} %30

\fourlineindentedshloka
{विश्वं यदेतत्स्वतनौ निशान्ते}
{यथावकाशं पुरुषः परो भवान्}
{बिभर्ति सोऽयं मम गर्भगोऽभू-}
{दहो नृलोकस्य विडम्बनं हि तत्} %31

\uvacha{श्रीभगवानुवाच}


\twolineshloka
{त्वमेव पूर्वसर्गेऽभूः पृश्निः स्वायम्भुवे सति}
{तदायं सुतपा नाम प्रजापतिरकल्मषः} %32

\twolineshloka
{युवां वै ब्रह्मणादिष्टौ प्रजासर्गे यदा ततः}
{सन्नियम्येन्द्रियग्रामं तेपाथे परमं तपः} %33

\twolineshloka
{वर्षवातातपहिम घर्मकालगुणाननु}
{सहमानौ श्वासरोध विनिर्धूतमनोमलौ} %34

\twolineshloka
{शीर्णपर्णानिलाहारावुपशान्तेन चेतसा}
{मत्तः कामानभीप्सन्तौ मदाराधनमीहतुः} %35

\twolineshloka
{एवं वां तप्यतोस्तीव्रं तपः परमदुष्करम्}
{दिव्यवर्षसहस्राणि द्वादशेयुर्मदात्मनोः} %36

\twolineshloka
{तदा वां परितुष्टोऽहममुना वपुषानघे}
{तपसा श्रद्धया नित्यं भक्त्या च हृदि भावितः} %37

\twolineshloka
{प्रादुरासं वरदराड्युवयोः कामदित्सया}
{व्रियतां वर इत्युक्ते मादृशो वां वृतः सुतः} %38

\twolineshloka
{अजुष्टग्राम्यविषयावनपत्यौ च दम्पती}
{न वव्राथेऽपवर्गं मे मोहितौ देवमायया} %39

\twolineshloka
{गते मयि युवां लब्ध्वा वरं मत्सदृशं सुतम्}
{ग्राम्यान्भोगानभुञ्जाथां युवां प्राप्तमनोरथौ} %40

\twolineshloka
{अदृष्ट्वान्यतमं लोके शीलौदार्यगुणैः समम्}
{अहं सुतो वामभवं पृश्निगर्भ इति श्रुतः} %41

\twolineshloka
{तयोर्वां पुनरेवाहमदित्यामास कश्यपात्}
{उपेन्द्र इति विख्यातो वामनत्वाच्च वामनः} %42

\twolineshloka
{तृतीयेऽस्मिन्भवेऽहं वै तेनैव वपुषाथ वाम्}
{जातो भूयस्तयोरेव सत्यं मे व्याहृतं सति} %43

\twolineshloka
{एतद्वां दर्शितं रूपं प्राग्जन्मस्मरणाय मे}
{नान्यथा मद्भवं ज्ञानं मर्त्यलिङ्गेन जायते} %44

\twolineshloka
{युवां मां पुत्रभावेन ब्रह्मभावेन चासकृत्}
{चिन्तयन्तौ कृतस्नेहौ यास्येथे मद्गतिं पराम्} %45

\uvacha{श्रीशुक उवाच}


\twolineshloka
{इत्युक्त्वासीद्धरिस्तूष्णीं भगवानात्ममायया}
{पित्रोः सम्पश्यतोः सद्यो बभूव प्राकृतः शिशुः} %46

\fourlineindentedshloka
{ततश्च शौरिर्भगवत्प्रचोदितः}
{सुतं समादाय स सूतिकागृहात्}
{यदा बहिर्गन्तुमियेष तर्ह्यजा}
{या योगमायाजनि नन्दजायया} %47

\fourlineindentedshloka
{तया हृतप्रत्ययसर्ववृत्तिषु}
{द्वाःस्थेषु पौरेष्वपि शायितेष्वथ}
{द्वारश्च सर्वाः पिहिता दुरत्यया}
{बृहत्कपाटायसकीलशृङ्खलैः} %48

\fourlineindentedshloka
{ताः कृष्णवाहे वसुदेव आगते}
{स्वयं व्यवर्यन्त यथा तमो रवेः}
{ववर्ष पर्जन्य उपांशुगर्जितः}
{शेषोऽन्वगाद्वारि निवारयन्फणैः} %49

\fourlineindentedshloka
{मघोनि वर्षत्यसकृद्यमानुजा}
{गम्भीरतोयौघजवोर्मिफेनिला}
{भयानकावर्तशताकुला नदी}
{मार्गं ददौ सिन्धुरिव श्रियः पतेः} %50

\fourlineindentedshloka
{नन्दव्रजं शौरिरुपेत्य तत्र तान्}
{गोपान्प्रसुप्तानुपलभ्य निद्रया}
{सुतं यशोदाशयने निधाय तत्}
{सुतामुपादाय पुनर्गृहानगात्} %51

\twolineshloka
{देवक्याः शयने न्यस्य वसुदेवोऽथ दारिकाम्}
{प्रतिमुच्य पदोर्लोहमास्ते पूर्ववदावृतः} %52

\twolineshloka
{यशोदा नन्दपत्नी च जातं परमबुध्यत}
{न तल्लिङ्गं परिश्रान्ता निद्रयापगतस्मृतिः}


॥इति श्रीमद्भागवते महापुराणे पारमहंस्यां संहितायां दशमस्कन्धे पूर्वार्धे तृतीयोऽध्यायः॥



\closesection
