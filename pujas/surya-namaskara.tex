% !TeX program = XeLaTeX
% !TeX root = pUjA.tex

\setlength{\parindent}{0pt}

\chapt{श्रीसूर्य-नमस्कारः}

\sect{पूर्वाङ्गविघ्नेश्वरपूजा}

(आचम्य)
\twolineshloka*
{शुक्लाम्बरधरं विष्णुं शशिवर्णं चतुर्भुजम्}
{प्रसन्नवदनं ध्यायेत् सर्वविघ्नोपशान्तये}
 
प्राणान्  आयम्य।  ॐ भूः + भूर्भुवः॒ सुव॒रोम्।
 
(अप उपस्पृश्य, पुष्पाक्षतान् गृहीत्वा)\\
ममोपात्तसमस्त दुरितक्षयद्वारा \\
श्रीपरमेश्वरप्रीत्यर्थं करिष्यमाणस्य कर्मणः\\
 निर्विघ्नेन परिसमाप्त्यर्थम् आदौ विघ्नेश्वरपूजां करिष्ये।

\twolineshloka*
{ॐ ग॒णानां᳚ त्वा ग॒णप॑तिꣳ हवामहे क॒विं क॑वी॒नामु॑प॒मश्र॑वस्तमम्}
{ज्ये॒ष्ठ॒राजं॒ ब्रह्म॑णां ब्रह्मणस्पत॒ आ नः॑ शृ॒ण्वन्नू॒तिभिः॑ सीद॒ साद॑नम्}
अस्मिन् हरिद्राबिम्बे महागणपतिं ध्यायामि, आवाहयामि।\\


ॐ महागणपतये नमः  आसनं समर्पयामि।\\
पादयोः पाद्यं समर्पयामि। हस्तयोरर्घ्यं समर्पयामि।\\
आचमनीयं समर्पयामि।\\
ॐ भूर्भुवस्सुवः। शुद्धोदकस्नानं समर्पयामि।\\
स्नानानन्तरमाचमनीयं समर्पयामि।\\
वस्त्रार्थमक्षतान् समर्पयामि।\\
यज्ञोपवीताभरणार्थे अक्षतान् समर्पयामि।\\
दिव्यपरिमलगन्धान् धारयामि।\\
गन्धस्योपरि हरिद्राकुङ्कुमं समर्पयामि। अक्षतान् समर्पयामि। \\
पुष्पमालिकां समर्पयामि। पुष्पैः पूजयामि।

\dnsub{अर्चना}
% \setenumerate{label=\devanumber.}
% \renewcommand{\labelenumi}{\devanumber\theenumi.}
\begin{enumerate}%[label=\devanumber\value{enumi}]
\begin{minipage}{0.475\linewidth}   
\item ॐ सुमुखाय नमः
\item ॐ एकदन्ताय नमः
\item ॐ कपिलाय नमः
\item ॐ गजकर्णकाय नमः
\item ॐ लम्बोदराय नमः
\item ॐ विकटाय नमः
\item ॐ विघ्नराजाय नमः
\item ॐ विनायकाय नमः
\item ॐ धूमकेतवे नमः
  \end{minipage}
  \begin{minipage}{0.525\linewidth}
\item ॐ गणाध्यक्षाय नमः
\item ॐ फालचन्द्राय नमः
\item ॐ गजाननाय नमः
\item ॐ वक्रतुण्डाय नमः
\item ॐ शूर्पकर्णाय नमः
\item ॐ हेरम्बाय नमः
\item ॐ स्कन्दपूर्वजाय नमः
\item ॐ सिद्धिविनायकाय नमः
\item ॐ विघ्नेश्वराय नमः
  \end{minipage}
\end{enumerate}
नानाविधपरिमलपत्रपुष्पाणि समर्पयामि॥\\
धूपमाघ्रापयामि।\\
अलङ्कारदीपं सन्दर्शयामि।\\
नैवेद्यम्।\\
ताम्बूलं समर्पयामि।\\
कर्पूरनीराजनं समर्पयामि।\\
कर्पूरनीराजनानन्तरमाचमनीयं समर्पयामि।\\
{वक्रतुण्डमहाकाय कोटिसूर्यसमप्रभ।}\\
{अविघ्नं कुरु मे देव सर्वकार्येषु सर्वदा॥}\\
प्रार्थनाः समर्पयामि।

अनन्तकोटिप्रदक्षिणनमस्कारान् समर्पयामि।\\
छत्त्रचामरादिसमस्तोपचारान् समर्पयामि।\\


\sect{प्रधान-पूजा - श्री-सूर्यनारायण-पूजा}

\twolineshloka*
{शुक्लाम्बरधरं विष्णुं शशिवर्णं चतुर्भुजम्}
{प्रसन्नवदनं ध्यायेत् सर्वविघ्नोपशान्तये}
 
प्राणान्  आयम्य।  ॐ भूः + भूर्भुवः॒ सुव॒रोम्।

\dnsub{सङ्कल्पः}

ममोपात्त-समस्त-दुरित-क्षयद्वारा श्री-परमेश्वर-प्रीत्यर्थं शुभे शोभने मुहूर्ते अद्य ब्रह्मणः
द्वितीयपरार्द्धे श्वेतवराहकल्पे वैवस्वतमन्वन्तरे अष्टाविंशतितमे कलियुगे प्रथमे पादे
जम्बूद्वीपे भारतवर्षे भरतखण्डे मेरोः दक्षिणे पार्श्वे शकाब्दे अस्मिन् वर्तमाने व्यावहारिके
प्रभवादि षष्टिसंवत्सराणां मध्ये (	) नाम संवत्सरे उत्तरायणे / दक्षिणायने  (  )-ऋतौ  (  ) मासे 
शुक्लपक्षे (  ) शुभतिथौ (इन्दु/भौम/बुध/गुरु/भृगु/स्थिर/भानु) वासरयुक्तायाम्
(  ) नक्षत्रयुक्तायां (  )-योग (  )-करण-युक्तायां च एवं गुण-विशेषण-विशिष्टायाम्
अस्याम् (  ) शुभतिथौ अस्माकं सहकुटुम्बानां क्षेमस्थैर्य-धैर्य-वीर्य-विजय-आयुरारोग्य-ऐश्वर्याभिवृद्ध्यर्थम्
 धर्मार्थकाममोक्ष\-चतुर्विधफलपुरुषार्थसिद्ध्यर्थं पुत्रपौत्राभिवृद्ध्यर्थम् इष्टकाम्यार्थसिद्ध्यर्थम्
मम इहजन्मनि पूर्वजन्मनि जन्मान्तरे च सम्पादितानां ज्ञानाज्ञानकृतमहा\-पातकचतुष्टय-व्यतिरिक्तानां रहस्यकृतानां प्रकाशकृतानां सर्वेषां पापानां सद्य अपनोदनद्वारा सकल-पापक्षयार्थं 
श्री-छाया-सुवर्चलाम्बा-समेत-श्री-सूर्यनारायणप्रीत्यर्थं
श्री-सूर्यनारायण-प्रसाद-सिद्ध्यर्थं श्री-सूर्यनारायण-पूज-पुरस्सरं तृचकल्पेन अरुणप्रश्नेन च श्री-सूर्यनमस्कारान् 
करिष्ये। तदङ्गं कलशपूजां च करिष्ये।


श्रीविघ्नेश्वराय नमः यथास्थानं प्रतिष्ठापयामि।
(गणपति-प्रसादं शिरसा गृहीत्वा)

\dnsub{आसन-पूजा}
\centerline{पृथिव्या  मेरुपृष्ठ  ऋषिः।  सुतलं  छन्दः।  कूर्मो  देवता॥}
\twolineshloka*
{पृथ्वि  त्वया  धृता  लोका  देवि  त्वं  विष्णुना  धृता}
{त्वं  च  धारय  मां  देवि  पवित्रं  चाऽऽसनं  कुरु}


\dnsub{घण्टापूजा}
\twolineshloka*
{आगमार्थं तु देवानां गमनार्थं तु रक्षसाम्}
{घण्टारवं करोम्यादौ देवताऽऽह्वानकारणम्}


\dnsub{कलशपूजा}
ॐ कलशाय नमः दिव्यगन्धान् धारयामि।\\
ॐ गङ्गायै नमः। ॐ यमुनायै नमः। ॐ गोदावर्यै नमः।  ॐ सरस्वत्यै नमः। ॐ नर्मदायै नमः। ॐ सिन्धवे नमः। ॐ कावेर्यै नमः।\\
ॐ सप्तकोटिमहातीर्थान्यावाहयामि।\\[-0.25ex]

(अथ कलशं स्पृष्ट्वा जपं कुर्यात्) \\
आपो॒ वा इ॒द सर्वं॒ विश्वा॑ भू॒तान्याप॑ प्रा॒णा वा आप॑ प॒शव॒ आपो\-ऽन्न॒मापोऽमृ॑त॒माप॑ स॒म्राडापो॑ वि॒राडाप॑ स्व॒राडाप॒श्\-छन्दा॒स्यापो॒ ज्योती॒ष्यापो॒ यजू॒ष्याप॑ स॒त्यमाप॒ सर्वा॑ दे॒वता॒ आपो॒ भूर्भुव॒ सुव॒राप॒ ओम्॥\\

\twolineshloka* 
{कलशस्य मुखे विष्णुः कण्ठे रुद्रः समाश्रितः}
{मूले तत्र स्थितो ब्रह्मा मध्ये मातृगणाः स्मृताः}
\threelineshloka* 
{कुक्षौ तु सागराः सर्वे सप्तद्वीपा वसुन्धरा}
{ऋग्वेदोऽथ यजुर्वेदः सामवेदोऽप्यथर्वणः}
{अङ्गैश्च सहिताः सर्वे कलशाम्बुसमाश्रिताः}
\twolineshloka* 
{गङ्गे च यमुने चैव गोदावरि सरस्वति}
{नर्मदे सिन्धुकावेरि जलेऽस्मिन् सन्निधिं कुरु}
\twolineshloka*
{सर्वे समुद्राः सरितः तीर्थानि च ह्रदा नदाः}
{आयान्तु देवपूजार्थं दुरितक्षयकारकाः}

\centerline{ॐ भूर्भुवः॒ सुवो॒ भूर्भुवः॒ सुवो॒ भूर्भुवः॒ सुवः॑।}

(इति कलशजलेन सर्वोपकरणानि आत्मानं च प्रोक्ष्य।)


\dnsub{आत्म-पूजा}
ॐ आत्मने नमः, दिव्यगन्धान् धारयामि।
\begin{multicols}{2}
१. ॐ आत्मने नमः\\
२. ॐ अन्तरात्मने नमः\\
३. ॐ योगात्मने नमः\\
४. ॐ जीवात्मने नमः\\
५. ॐ परमात्मने नमः\\
६. ॐ ज्ञानात्मने नमः
\end{multicols}
समस्तोपचारान् समर्पयामि।

\twolineshloka*
{देहो देवालयः प्रोक्तो जीवो देवः सनातनः}
{त्यजेदज्ञाननिर्माल्यं सोऽहं भावेन पूजयेत्}


\begin{minipage}{\linewidth}
\dnsub{पीठ-पूजा}

\begin{multicols}{2}
\begin{enumerate}
\item ॐ आधारशक्त्यै नमः
\item ॐ मूलप्रकृत्यै नमः
\item ॐ आदिकूर्माय नमः 
\item ॐ आदिवराहाय नमः
\item ॐ अनन्ताय नमः
\item ॐ पृथिव्यै नमः
\item ॐ रत्नमण्डपाय नमः
\item ॐ रत्नवेदिकायै नमः
\item ॐ स्वर्णस्तम्भाय नमः
\item ॐ श्वेतच्छत्त्राय नमः
\item ॐ कल्पकवृक्षाय नमः
\item ॐ क्षीरसमुद्राय नमः 
\item ॐ सितचामराभ्यां नमः
\item ॐ योगपीठासनाय नमः
\end{enumerate}
\end{multicols}

\end{minipage}

\dnsub{गुरु ध्यानम्}

\twolineshloka*
{गुरुर्ब्रह्मा गुरुर्विष्णुर्गुरुर्देवो महेश्वरः}
{गुरुः साक्षात् परं ब्रह्म तस्मै श्री गुरवे नमः}

 
\sect{कुम्भे वरुण-सूर्य-नारायण-पूजा}
% \begin{center}

ॐ। इ॒मं मे॑ वरुण श्रुधी॒ हव॑म॒द्या च॑ मृडय। त्वाम॑व॒स्युराच॑के॥ तत्त्वा॑ यामि॒ ब्रह्म॑णा॒ वन्द॑मान॒स्तदाशा᳚स्ते॒ यज॑मानो ह॒विर्भिः॑। अहे॑डमानो वरुणे॒ह बो॒ध्युरु॑शꣳस॒ मा न॒ आयुः॒ प्रमो॑षीः॥

अस्मिन् कुम्भे सकल-तीर्थाधिपतिं वरुणं ध्यायामि। वरुणमावाहयामि।

वरुणाय नमः। आसनं समर्पयामि।

पाद्यं, अर्घ्यं, आचमनीयं, स्नानं, स्नानानन्तरमाचमनीयं, वस्त्रं, उपवीतं, गन्धं, गन्धोपरि अक्षतान्, पुष्पाणि समर्पयामि।

वरुणाय नमः। प्रचेतसे नमः। सुरूपिणे नमः। अपां पतये नमः। मकरवाहनाय नमः। जलाधिपतये नमः। पाशहस्ताय नमः। वरुणाय नमः। नानाविधपत्रपुष्पाणि समर्पयामि॥

वरुणाय नमः समस्तोपचारान् समर्पयामि।

ॐ। आ स॒त्येन॒ रज॑सा॒ वर्त॑मानो निवे॒शय॑न्न॒मृतं॒ मर्त्यं॑ च। हि॒र॒ण्यये॑न सवि॒ता रथे॒नाऽदे॒वो या॑ति॒ भुव॑ना वि॒पश्यन्॑। 

\fourlineindentedshloka*
{सूर्यं सुन्दरलोकनाथममृतं वेदान्तसारं शिवम्}
{ज्ञानं ब्रह्ममयं सुरेशममलं लोकैकचित्तं प्रभुम्}
{इन्द्रादित्यनराधिपं सुरगुरुं त्रैलोक्यचूडामणिं}
{विष्णुब्रह्मशिवस्वरूपहृदयं वन्दे सदा भास्करम्}

अस्मिन् कुम्भे श्री-छाया-सुवर्चलाम्बा-समेत-श्री-सूर्यनारायणं ध्यायामि। श्री-छाया-सुवर्चलाम्बा-समेत-श्री-सूर्यनारायणम् आवाहयामि।

आसनं समर्पयामि। पाद्यं समर्पयामि। अर्घ्यं समर्पयामि। आचमनीयं समर्पयामि। मधुपर्कं समर्पयामि। शुद्धोदकस्नानं समर्पयामि। स्नानानन्तरं आचमनीयं समर्पयामि। वस्त्रं समर्पयामि। उपवीतं समर्पयामि। आभरणम् समर्पयामि। गन्धान् धारयामि। गन्धस्योपरि हरिद्राकुङ्कुमं समर्पयमि। अक्षतान् समर्पयामि। पुष्पाणि समर्पयामि।

\dnsub{अर्चना}

\begin{multicols}{2}
\begin{enumerate}
\item ॐ मित्राय नमः
\item ॐ रवये नमः
\item ॐ सूर्याय नमः
\item ॐ भानवे नमः
\item ॐ खगाय नमः
\item ॐ पूष्णे नमः
\item ॐ हिरण्यगर्भाय नमः
\item ॐ मरीचये नमः
\item ॐ आदित्याय नमः
\item ॐ सवित्रे नमः
\item ॐ अर्काय नमः
\item ॐ भास्कराय नमः

\end{enumerate}
\end{multicols}

\begingroup
\setlength{\columnseprule}{1pt}
\let\chapt\sect
\input{../namavali-manjari/100/Surya_108.tex}
\endgroup


 श्री-छाया-सुवर्चलाम्बा-समेत-श्री-सूर्यनारायण-स्वामिने नमः नानाविध\-परिमल\-पत्र\-पुष्पाणि समर्पयामि।

\dnsub{उत्तराङ्गपूजा}

\renewcommand{\devAya}{श्री-छाया-सुवर्चलाम्बा-समेत-श्री-सूर्यनारायण-स्वामिने नमः}

धूपमाघ्रापयामि। अलङ्कारदीपं सन्दर्शयामि।
\devAya{} नैवेद्यं निवेदयामि। निवेदनान्तरम् आचमनीयं समर्पयामि।
\devAya{} कर्पूरताम्बूलं समर्पयामि।

भा॒स्क॒राय॑ वि॒द्महे॑ महद्युतिक॒राय॑ धीमहि। 
तन्नो॑ आदित्यः प्रचो॒दया᳚त्। 

\devAya{} समस्त अपराध क्षमापनार्थं कर्पूरनीराजनं दर्शयामि। कर्पूरनीरजनानन्तरम् आचमनीयं समर्पयामि। रक्षां धारयामि।
\devAya{} वेदोक्तमन्त्रपुष्पाञ्जलिं समर्पयामि। स्वर्णपुष्पं समर्पयामि।
\devAya{} प्रदक्षिणनमस्काराः समर्पयामि।

\dnsub{न्यासः}

ओं अस्य श्रीसूर्यनमस्कार-महामन्त्रस्य, कण्वपुत्रः प्रस्कन्न ऋषिः, अनुष्टुप् छन्दः, श्रीसूर्यनारायणो देवता।

ह्रां बीजम्, ह्रीं शक्तिः, ह्रूं कीलकम्। श्रीसूर्यनारायण-प्रसाद-सिद्ध्यर्थे नमस्कारे विनियोगः।

ह्रां अङ्गुष्ठाभ्यां नमः।\\
ह्रीं तर्जनीभ्यां नमः।\\
ह्रूं मध्यमाभ्यां नमः।\\
ह्रैं अनामिकाभ्यां नमः।\\
ह्रौं कनिष्ठिकाभ्यां नमः।\\
ह्रः करतलकरपृष्ठाभ्यां नमः।\\

ह्रां हृदयाय नमः।\\
ह्रीं शिरसे स्वाहा।\\
ह्रूं शिखायै वषट्।\\
ह्रैं कवचाय हुम्।\\
ह्रौं नेत्रत्रयाय वौषट्।\\
ह्रः अस्त्राय फट्।\\

भूर्भुवस्सुवरोमिति दिग्बन्धः॥\\

\dnsub{ध्यानम्}
\fourlineindentedshloka*
{उदयगिरिमुपेतं भास्करं पद्महस्तं}
{सकलभुवननेत्रं नूतनरत्नोपधेयम्}
{तिमिरकरिमृगेन्द्रं बोधकं पद्मिनीनां}
{सुरगुरुमभिवन्दे सुन्दरं विश्वरूपम्}

लं-पृथिव्यात्मने गन्धं समर्पयामि।\\
हं- आकाशात्मने पुष्पाणि समर्पयामि।\\
यं-वाय्वात्मने धूपमाघ्रापयामि।\\
रं-वह्न्यात्मने दीपं दर्शयामि।\\
वं-अमृतात्मने अमृतोपहारं निवेदयामि।\\
सं-सर्वात्मने सर्वोपचारान् समर्पयामि॥\\


\dnsub{नमस्कारा:}

ॐ ग॒णानां᳚ त्वा ग॒णप॑तिꣳ हवामहे क॒विं क॑वी॒नामु॑प॒मश्र॑\-वस्तमम्। 
ज्ये॒ष्ठ॒राजं॒ ब्रह्म॑णां ब्रह्मणस्पत॒ आ नः॑ शृ॒ण्वन्नू॒तिभिः॑ सीद॒ साद॑नम्॥\\
ॐ महागणपतये॒ नमः॑॥ 

\twolineshloka*
{उमाकोमल-हस्ताब्ज-सम्भावित-ललाटकम्}
{हिरण्यकुण्डलं वन्दे कुमारं पुष्करस्रजम्}
ॐ श्री-वल्ली-देवसेना-समेत-सुब्रह्मण्यस्वामिने नमः॥


\twolineshloka*
{गुरुर्ब्रह्मा गुरुर्विष्णुर्गुरुर्देवो महेश्वरः}
{गुरुः साक्षात्परं ब्रह्म तस्मै श्री-गुरवे नमः}

\twolineshloka*
{गुरवे सर्वलोकानां भिषजे भवरोगिणाम्}
{निधये सर्वविद्यानां दक्षिणामूर्तये नमः}

\twolineshloka*
{विनतातनयो देवः कर्मसाक्षी सुरेश्वरः}
{सप्ताश्वः सप्तरज्जुश्च अरुणो मे प्रसीदतु}

\twolineshloka*
{रज्जुवेत्रकशापाणिं प्रसन्नं कश्यपात्मजम्}
{सर्वाभरणदीप्ताङ्गमरुणं प्रणमाम्यहम्}

ॐ कर्मसाक्षिणे अरुणाय नमः॥

अ॒ग्निमी᳚ळे पु॒रोहि॑तं य॒ज्ञस्य॑ दे॒वमृ॒त्विजम्᳚। होता᳚रं रत्न॒-धात॑मम्॥ ऋग्वेदात्मने सूर्यनारायण-स्वामिने नमः॥

इ॒षेत्वो॒र्जे त्वा॑ वा॒यवः॑ स्थो पा॒यवः॑ स्थ दे॒वो वः॑ सवि॒ता प्रार्प॑यतु॒ श्रेष्ठ॑तमाय॒ कर्म॑णे॥ यजुर्वेदात्मने सूर्यनारायण-स्वामिने नमः॥

अग्न॒ आया॑हि वी॒तये॑ गृणा॒नो ह॒व्यदा॑तये। नि होता॑ सथ्सि ब॒र्हिषि॑॥ सामवेदात्मने सूर्यनारायण-स्वामिने नमः॥

शन्नो॑ दे॒वीर॒भिष्ट॑य॒ आपो॑ भवन्तु पी॒तये᳚। शं योर॒भिस्र॑वन्तु नः॥ अथर्ववेदात्मने सूर्यनारायण-स्वामिने नमः॥


\centerline{\normalsize (तैत्तिरीयारण्यके प्रश्नः – १० (महानारयणोपनिषत्))}

घृणिः सूर्य॑ आदि॒त्यो न प्रभा॑ वा॒त्यक्ष॑रम्। 
मधु॑ क्षरन्ति॒ तद्र॑सम्। 
स॒त्यं वै तद्रस॒मापो॒ ज्योती॒रसो॒ऽमृतं॒ ब्रह्म॒ भूर्भुवः॒ सुव॒रोम्॥ \devAya{}॥ %॥५४॥


\centerline{\normalsize (तैत्तिरीयसंहितायां काण्डः १/प्रश्नः – ४)}
%1.4.31.1
त॒रणि॑र्वि॒श्वद॑र्\mbox{}शतो ज्योति॒ष्कृद॑सि सूर्य। विश्व॒मा भा॑सि रोच॒नम्॥ उ॒प॒या॒मगृ॑हीतो\-ऽसि॒ सूर्या॑य त्वा॒ भ्राज॑स्वत ए॒ष ते॒ योनिः॒ सूर्या॑य त्वा॒ भ्राज॑स्वते॥३२॥
\devAya{}॥


ॐ ह्राम्।  उ॒द्यन्न॒द्य मि॑त्रमहः।  मित्राय नमः॥

ॐ ह्रीम्।  आ॒रोह॒न्नुत्त॑रां॒ दिवम्᳚। रवये नमः॥

ॐ ह्रूम्।  हृ॒द्रो॒गं मम॑ सूर्य। सूर्याय नमः॥

ॐ ह्रैम्।  ह॒रि॒माणं॑ च नाशय। भानवे नमः॥

ॐ ह्रौम्।  शुके॑षु मे हरि॒माणम्᳚। खगाय नमः॥

ॐ ह्रः। रो॒प॒णाका॑सु दध्मसि॥ पूष्णे नमः॥

ॐ ह्राम्।  अथो॑ हारिद्र॒वेषु॑ मे। हिरण्यगर्भाय नमः॥

ॐ ह्रीम्।  ह॒रि॒माणं॒ नि द॑ध्मसि। मरीचये नमः॥

ॐ ह्रूम्।  उद॑गाद॒यमा॑दि॒त्यः। आदित्याय नमः॥

ॐ ह्रैम्।  विश्वे॑न॒ सह॑सा स॒ह। सवित्रे नमः॥

ॐ ह्रौम्।  द्वि॒षन्तं॒ मम॑ र॒न्धयन्॑। अर्काय नमः॥

ॐ ह्रः। मो अ॒हं द्वि॑ष॒तो र॑धम्। भास्कराय नमः॥


ॐ ह्रां ह्रीम्। उ॒द्यन्न॒द्य मि॑त्रमहः।  आ॒रोह॒न्नुत्त॑रां॒ दिवम्᳚। मित्र-रविभ्यां नमः॥

ॐ ह्रूं ह्रैम्। हृ॒द्रो॒गं मम॑ सूर्य। ह॒रि॒माणं॑ च नाशय। सूर्य-भानुभ्यां नमः॥

ॐ ह्रौं ह्रः। शुके॑षु मे हरि॒माणम्᳚। रो॒प॒णाका॑सु दध्मसि॥ खग-पूषभ्यां नमः॥

ॐ ह्रां ह्रीम्। अथो॑ हारिद्र॒वेषु॑ मे। ह॒रि॒माणं॒ नि द॑ध्मसि। हिरण्यगर्भ-मरीचिभ्यां नमः॥

ॐ ह्रूं ह्रैम्। उद॑गाद॒यमा॑दि॒त्यः। विश्वे॑न॒ सह॑सा स॒ह। आदित्य-सवितृभ्यां नमः॥

ॐ ह्रौं ह्रः। द्वि॒षन्तं॒ मम॑ र॒न्धयन्॑। मो अ॒हं द्वि॑ष॒तो र॑धम्। अर्क-भास्कराभ्यां नमः॥


ॐ ह्रां ह्रीं ह्रूं ह्रैम्। उ॒द्यन्न॒द्य मि॑त्रमहः। आ॒रोह॒न्नुत्त॑रां॒ दिवम्᳚। हृ॒द्रो॒गं मम॑ सूर्य। ह॒रि॒माणं॑ च नाशय। मित्र-रवि-सूर्य-भानुभ्यो नमः॥

ॐ ह्रौं ह्रः ह्रां ह्रीम्। शुके॑षु मे हरि॒माणम्᳚। रो॒प॒णाका॑सु दध्मसि॥ अथो॑ हारिद्र॒वेषु॑ मे। ह॒रि॒माणं॒ नि द॑ध्मसि। खग-पूष-हिरण्यगर्भ-मरीचिभ्यो नमः॥

ॐ ह्रूं ह्रैं ह्रौं ह्रः। उद॑गाद॒यमा॑दि॒त्यः। विश्वे॑न॒ सह॑सा स॒ह। द्वि॒षन्तं॒ मम॑ र॒न्धयन्॑। मो अ॒हं द्वि॑ष॒तो र॑धम्। आदित्य-सवित्रर्क-भास्करेभ्यो नमः॥

ॐ ह्रां ह्रीं ह्रूं ह्रैं ह्रौं ह्रः।
उ॒द्यन्न॒द्य मि॑त्रमहः। आ॒रोह॒न्नुत्त॑रां॒ दिवम्᳚। हृ॒द्रो॒गं मम॑ सूर्य। ह॒रि॒माणं॑ च नाशय। शुके॑षु मे हरि॒माणम्᳚। रो॒प॒णाका॑सु दध्मसि॥ 
मित्र-रवि-सूर्य-भानु-खग-पूषभ्यो नमः॥


ॐ ह्रां ह्रीं ह्रूं ह्रैं ह्रौं ह्रः।
अथो॑ हारिद्र॒वेषु॑ मे। ह॒रि॒माणं॒ नि द॑ध्मसि। उद॑गाद॒यमा॑दि॒त्यः। विश्वे॑न॒ सह॑सा स॒ह। द्वि॒षन्तं॒ मम॑ र॒न्धयन्॑। मो अ॒हं द्वि॑ष॒तो र॑धम्।
हिरण्यगर्भ-मरीच्यादित्य-सवित्रर्क-भास्करेभ्यो नमः॥

\centerline{\normalsize (तैत्तिरीयब्राह्मणे अष्टकं – ३/प्रश्नः – ७/अनुवाकः – ६/ पञ्चादयः ७६-७७)}

ॐ ह्रां ह्रीं ह्रूं ह्रैं ह्रौं ह्रः। ॐ ह्रां ह्रीं ह्रूं ह्रैं ह्रौं ह्रः। 
उ॒द्यन्न॒द्य मि॑त्रमहः। 
आ॒रोह॒न्नुत्त॑रां॒ दिवम्᳚।
हृ॒द्रो॒गं मम॑ सूर्य।
ह॒रि॒माणं॑ च नाशय।
शुके॑षु मे हरि॒माणम्᳚।
रो॒प॒णाका॑सु दध्मसि॥
अथो॑ हारिद्र॒वेषु॑ मे।
ह॒रि॒माणं॒ नि द॑ध्मसि।
उद॑गाद॒यमा॑दि॒त्यः।
विश्वे॑न॒ सह॑सा स॒ह।
द्वि॒षन्तं॒ मम॑ र॒न्धयन्॑।
मो अ॒हं द्वि॑ष॒तो र॑धम्।
मित्र-रवि-सूर्य-भानु-खग-पूष-हिरण्यगर्भ-मरीच्यादित्य-सवित्रर्क-भास्करेभ्यो नमः॥

\dnsub{आदित्यमण्डले परब्रह्मोपासनम्}
\centerline{\normalsize (तैत्तिरीयारण्यके प्रश्नः – १० (महानारयणोपनिषत्))}
आ॒दि॒त्यो वा ए॒ष ए॒तन्म॒ण्डलं॒ तप॑ति॒ तत्र॒ ता ऋच॒स्तदृ॒चा म॑ण्डल॒ꣳ॒ स ऋ॒चां लो॒कोऽथ॒ य ए॒ष ए॒तस्मि॑न्म॒ण्डले॒ऽर्चिर्दी॒प्यते॒ तानि॒ सामा॑नि॒ स सा॒म्नां म॒ण्डल॒ꣳ॒ स सा॒म्नां लो॒कोऽथ॒ य ए॒ष ए॒तस्मि॑न्म॒ण्डले॒ऽर्चिषि॒ पुरु॑ष॒स्तानि॒ यजूꣳ॑षि॒ स यजु॑षा मण्डल॒ꣳ॒ स यजु॑षां लो॒कः सैषा त्र॒य्येव॑ वि॒द्या त॑पति॒ य ए॒षो᳚ऽन्तरा॑दि॒त्ये हि॑र॒ण्मयः॒ पुरु॑षः॥३१॥
\devAya{}॥
%६.१४.०


\dnsub{आदित्यपुरुषस्य सर्वात्मकत्वप्रदर्शनम्}
\centerline{\normalsize (तैत्तिरीयारण्यके प्रश्नः – १० (महानारयणोपनिषत्))}
आ॒दि॒त्यो वै तेज॒ ओजो॒ बलं॒ यश॒श्चक्षुः॒ श्रोत्र॑मा॒त्मा मनो॑ म॒न्युर्मनु॑र्मृ॒त्युः स॒त्यो मि॒त्रो वा॒युरा॑का॒शः प्रा॒णो लो॑कपा॒लः कः किं कं तथ्स॒त्यमन्न॑म॒मृतो॑ जी॒वो विश्वः॑ कत॒मः स्व॑य॒म्भु ब्रह्मै॒तदमृ॑त ए॒ष पुरु॑ष ए॒ष भू॒ताना॒मधि॑पति॒र्ब्रह्म॑णः॒ सायु॑ज्यꣳ सलो॒कता॑माप्नोत्ये॒तासा॑मे॒व दे॒वता॑ना॒ꣳ॒ सायु॑ज्यꣳ सा॒र्ष्टिताꣳ॑ समानलो॒कता॑माप्नोति॒ य ए॒वं वेदे᳚त्युप॒निषत्॥३२॥
\devAya{}॥
%६.१५.०

\begingroup
\setlength{\parindent}{1.5em}
\newcounter{anuvakam}
\makeatletter
  \def\vhrulefill#1{\leavevmode\leaders\hrule\@height#1\hfill \kern\z@}
\makeatother
\newcommand{\anuvakamend}[1][]{\refstepcounter{anuvakam}%[-1ex]
\newline\centerline{\devAya{}।}
\centerline{\textbf{ॐ नमो नारायणाय॥}}
\newline\mbox{}
\baselineskip=12pt\nolinebreak[4]\vhrulefill{1.6pt}\raisebox{-3pt}{\bfseries{[\devanumber{\arabic{anuvakam}}]}}%\hrulefill
\vspace{-1pt}
}
\sect{अरुणप्रश्नः}
\renewcommand{\sect}[1]{}
\input{../vedamantra-book/aranyakas/ArunaPrashnah.tex}
\endgroup
\begingroup
\let\chapt\sect
\input{../vedamantra-book/mantras/NavagrahaSuktam.tex}
\endgroup


\dnsub{प्रार्थना}
\begin{center}

\twolineshloka*
{भानो भास्कर मार्तण्ड चण्डरश्मे दिवाकर}
{आयुरारोग्यमैश्वर्यं श्रियं पुत्रांश्च देहि मे}

\twolineshloka*
{धृतपद्मद्वयं भानुं तेजोमण्डलमध्यगम्}
{सर्वाधिव्याधिशमनं छायाश्लिष्टतनुं भजे}

\twolineshloka*
{सौरमण्डलमध्यस्थं साम्बं संसारभेषजम्}
{नीलग्रीवं विरूपाक्षं नमामि शिवमव्ययम्}

\fourlineindentedshloka*
{ध्येयः सदा सवितृमण्डल-मध्यवर्ती}
{नारायणः सरसि-जासन-सन्निविष्टः}
{केयूरवान् मकरकुण्डलवान् किरीटी}
{हारी हिरण्मयवपुर्धृतशङ्खचक्रः}

\twolineshloka*
{शङ्ख-चक्र-गदापाणे द्वारकानिलयाच्युत}
{गोविन्द पुण्डरीकाक्ष रक्ष मां शरणागतम्}

\twolineshloka*
{आकाशात् पतितं तोयं यथा गच्छति सागरम्}
{सर्वदेवनमस्कारः केशवं प्रतिगच्छति}

श्री-केशवं प्रतिगच्छत्यों नम इति॥



\fourlineindentedshloka*
{कायेन वाचा मनसेन्द्रियैर्वा}
{बुद्‌ध्याऽऽत्मना वा प्रकृतेः स्वभावात्}
{करोमि यद्यत् सकलं परस्मै}
{नारायणायेति समर्पयामि}

अनेन पूजनेन सपरिवार-भगवान्-सूर्यः प्रीयताम्। \\
\end{center}

\centerline{ॐ तत्सद्ब्रह्मार्पणमस्तु।}

\begingroup
\let\chapt\sect
\begin{center}
\input{../stotra-sangrahah/stotras/navagraha/AdityaHrdayam.tex}
\input{../stotra-sangrahah/stotras/navagraha/DwadasharyaSuryaStuti.tex}
\end{center}
\endgroup

\closesection
