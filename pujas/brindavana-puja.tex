% !TeX program = XeLaTeX
% !TeX root = ../pujavidhanam.tex

\setlength{\parindent}{0pt}
\chapt{बृन्दावनपूजा (तुलसी--विष्णु-पूजा)} 

\sect{पूर्वाङ्गविघ्नेश्वरपूजा}

(आचम्य)
\twolineshloka*
{शुक्लाम्बरधरं विष्णुं शशिवर्णं चतुर्भुजम्}
{प्रसन्नवदनं ध्यायेत् सर्वविघ्नोपशान्तये}
 
प्राणान्  आयम्य।  ॐ भूः + भूर्भुवः॒ सुव॒रोम्।
 
(अप उपस्पृश्य, पुष्पाक्षतान् गृहीत्वा)\\
ममोपात्तसमस्त दुरितक्षयद्वारा \\
श्रीपरमेश्वरप्रीत्यर्थं करिष्यमाणस्य कर्मणः\\
 निर्विघ्नेन परिसमाप्त्यर्थम् आदौ विघ्नेश्वरपूजां करिष्ये।

\twolineshloka*
{ॐ ग॒णानां᳚ त्वा ग॒णप॑तिꣳ हवामहे क॒विं क॑वी॒नामु॑प॒मश्र॑वस्तमम्}
{ज्ये॒ष्ठ॒राजं॒ ब्रह्म॑णां ब्रह्मणस्पत॒ आ नः॑ शृ॒ण्वन्नू॒तिभिः॑ सीद॒ साद॑नम्}
अस्मिन् हरिद्राबिम्बे महागणपतिं ध्यायामि, आवाहयामि।\\


ॐ महागणपतये नमः  आसनं समर्पयामि।\\
पादयोः पाद्यं समर्पयामि। हस्तयोरर्घ्यं समर्पयामि।\\
आचमनीयं समर्पयामि।\\
ॐ भूर्भुवस्सुवः। शुद्धोदकस्नानं समर्पयामि।\\
स्नानानन्तरमाचमनीयं समर्पयामि।\\
वस्त्रार्थमक्षतान् समर्पयामि।\\
यज्ञोपवीताभरणार्थे अक्षतान् समर्पयामि।\\
दिव्यपरिमलगन्धान् धारयामि।\\
गन्धस्योपरि हरिद्राकुङ्कुमं समर्पयामि। अक्षतान् समर्पयामि। \\
पुष्पमालिकां समर्पयामि। पुष्पैः पूजयामि।

\dnsub{अर्चना}
% \setenumerate{label=\devanumber.}
% \renewcommand{\labelenumi}{\devanumber\theenumi.}
\begin{enumerate}%[label=\devanumber\value{enumi}]
\begin{minipage}{0.475\linewidth}   
\item ॐ सुमुखाय नमः
\item ॐ एकदन्ताय नमः
\item ॐ कपिलाय नमः
\item ॐ गजकर्णकाय नमः
\item ॐ लम्बोदराय नमः
\item ॐ विकटाय नमः
\item ॐ विघ्नराजाय नमः
\item ॐ विनायकाय नमः
\item ॐ धूमकेतवे नमः
  \end{minipage}
  \begin{minipage}{0.525\linewidth}
\item ॐ गणाध्यक्षाय नमः
\item ॐ फालचन्द्राय नमः
\item ॐ गजाननाय नमः
\item ॐ वक्रतुण्डाय नमः
\item ॐ शूर्पकर्णाय नमः
\item ॐ हेरम्बाय नमः
\item ॐ स्कन्दपूर्वजाय नमः
\item ॐ सिद्धिविनायकाय नमः
\item ॐ विघ्नेश्वराय नमः
  \end{minipage}
\end{enumerate}
नानाविधपरिमलपत्रपुष्पाणि समर्पयामि॥\\
धूपमाघ्रापयामि।\\
अलङ्कारदीपं सन्दर्शयामि।\\
नैवेद्यम्।\\
ताम्बूलं समर्पयामि।\\
कर्पूरनीराजनं समर्पयामि।\\
कर्पूरनीराजनानन्तरमाचमनीयं समर्पयामि।\\
{वक्रतुण्डमहाकाय कोटिसूर्यसमप्रभ।}\\
{अविघ्नं कुरु मे देव सर्वकार्येषु सर्वदा॥}\\
प्रार्थनाः समर्पयामि।

अनन्तकोटिप्रदक्षिणनमस्कारान् समर्पयामि।\\
छत्त्रचामरादिसमस्तोपचारान् समर्पयामि।\\


\sect{प्रधान-पूजा — तुलसी-पूजा}

\twolineshloka*
{शुक्लाम्बरधरं विष्णुं शशिवर्णं चतुर्भुजम्}
{प्रसन्नवदनं ध्यायेत् सर्वविघ्नोपशान्तये}

प्राणान्  आयम्य।  ॐ भूः + भूर्भुवः॒ सुव॒रोम्।


शुक्लाम्बरधरं + शान्तये + श्री-परमेश्वर-प्रीत्यर्थं शुभे + तिथौ तुलसी महाविष्णु प्रसादसिद्धयर्थं तुलसी महाविष्णुपूजां करिष्ये।
इति सङ्कल्प्य विघ्नेशमुद्वास्य कलशपूजां कृत्वा, तुलसीं विष्णुं च ध्यायेत्। 


\dnsub{सङ्कल्पः}

ममोपात्त-समस्त-दुरित-क्षयद्वारा श्रीपरमेश्वर\-प्रीत्यर्थं शुभे शोभने मुहूर्ते अद्य ब्रह्मणः
द्वितीयपरार्धे श्वेतवराहकल्पे वैवस्वतमन्वन्तरे अष्टाविंशतितमे कलियुगे प्रथमे पादे
जम्बूद्वीपे भारतवर्षे भरत\-खण्डे मेरोः दक्षिणे पार्श्वे अस्मिन् वर्तमाने व्यावहारिकाणां
प्रभवादीनां षष्ट्याः संवत्सराणां मध्ये \mbox{(~~~)}-नाम संवत्सरे दक्षिणायने 
{शरद्}-ऋतौ  {तुला/वृश्चिक}-मासे कार्तिक-शुक्ल-पक्षे द्वादश्यां शुभतिथौ
\mbox{(~~~)}-वासरयुक्तायां 
\mbox{(~~~)}-नक्षत्र\-युक्तायाम्
\mbox{(~~~)}-योग\-युक्तायां
\mbox{(~~~)}-करण\-युक्तायाम् एवं-गुण-विशेषण-विशिष्टायाम् अस्यां द्वादश्यां शुभतिथौ 

अस्माकं सहकुटुम्बानां क्षेमस्थैर्य-धैर्य-वीर्य-विजय-आयुरारोग्य-ऐश्वर्याभिवृद्ध्यर्थम्
धर्मार्थकाममोक्ष\-चतुर्विधफलपुरुषार्थसिद्ध्यर्थं पुत्रपौत्राभि\-वृद्ध्यर्थम् इष्टकाम्यार्थसिद्ध्यर्थम्
मम इहजन्मनि पूर्वजन्मनि जन्मान्तरे च सम्पादितानां ज्ञानाज्ञानकृतमहा\-पातकचतुष्टय-व्यतिरिक्तानां रहस्यकृतानां प्रकाशकृतानां सर्वेषां पापानां सद्य अपनोदनद्वारा सकल-पापक्षयार्थं श्रीमहाविष्णु-तुलसी-प्रीत्यर्थं यावच्छक्ति ध्यानावाहनादि षोडशोपचार-श्रीमहाविष्णु-तुलसी-पूजां करिष्ये। तदङ्गं कलशपूजां च करिष्ये।


श्रीविघ्नेश्वराय नमः यथास्थानं प्रतिष्ठापयामि।\\
(गणपति-प्रसादं शिरसा गृहीत्वा)

\dnsub{आसन-पूजा}
\centerline{पृथिव्या  मेरुपृष्ठ  ऋषिः।  सुतलं  छन्दः।  कूर्मो  देवता॥}
\twolineshloka*
{पृथ्वि  त्वया  धृता  लोका  देवि  त्वं  विष्णुना  धृता}
{त्वं  च  धारय  मां  देवि  पवित्रं  चाऽऽसनं  कुरु}


\dnsub{घण्टापूजा}
\twolineshloka*
{आगमार्थं तु देवानां गमनार्थं तु रक्षसाम्}
{घण्टारवं करोम्यादौ देवताऽऽह्वानकारणम्}


\dnsub{कलशपूजा}
ॐ कलशाय नमः दिव्यगन्धान् धारयामि।\\
ॐ गङ्गायै नमः। ॐ यमुनायै नमः। ॐ गोदावर्यै नमः।  ॐ सरस्वत्यै नमः। ॐ नर्मदायै नमः। ॐ सिन्धवे नमः। ॐ कावेर्यै नमः।\\
ॐ सप्तकोटिमहातीर्थान्यावाहयामि।\\[-0.25ex]

(अथ कलशं स्पृष्ट्वा जपं कुर्यात्) \\
आपो॒ वा इ॒द सर्वं॒ विश्वा॑ भू॒तान्याप॑ प्रा॒णा वा आप॑ प॒शव॒ आपो\-ऽन्न॒मापोऽमृ॑त॒माप॑ स॒म्राडापो॑ वि॒राडाप॑ स्व॒राडाप॒श्\-छन्दा॒स्यापो॒ ज्योती॒ष्यापो॒ यजू॒ष्याप॑ स॒त्यमाप॒ सर्वा॑ दे॒वता॒ आपो॒ भूर्भुव॒ सुव॒राप॒ ओम्॥\\

\twolineshloka* 
{कलशस्य मुखे विष्णुः कण्ठे रुद्रः समाश्रितः}
{मूले तत्र स्थितो ब्रह्मा मध्ये मातृगणाः स्मृताः}
\threelineshloka* 
{कुक्षौ तु सागराः सर्वे सप्तद्वीपा वसुन्धरा}
{ऋग्वेदोऽथ यजुर्वेदः सामवेदोऽप्यथर्वणः}
{अङ्गैश्च सहिताः सर्वे कलशाम्बुसमाश्रिताः}
\twolineshloka* 
{गङ्गे च यमुने चैव गोदावरि सरस्वति}
{नर्मदे सिन्धुकावेरि जलेऽस्मिन् सन्निधिं कुरु}
\twolineshloka*
{सर्वे समुद्राः सरितः तीर्थानि च ह्रदा नदाः}
{आयान्तु देवपूजार्थं दुरितक्षयकारकाः}

\centerline{ॐ भूर्भुवः॒ सुवो॒ भूर्भुवः॒ सुवो॒ भूर्भुवः॒ सुवः॑।}

(इति कलशजलेन सर्वोपकरणानि आत्मानं च प्रोक्ष्य।)


\dnsub{आत्म-पूजा}
ॐ आत्मने नमः, दिव्यगन्धान् धारयामि।
\begin{multicols}{2}
१. ॐ आत्मने नमः\\
२. ॐ अन्तरात्मने नमः\\
३. ॐ योगात्मने नमः\\
४. ॐ जीवात्मने नमः\\
५. ॐ परमात्मने नमः\\
६. ॐ ज्ञानात्मने नमः
\end{multicols}
समस्तोपचारान् समर्पयामि।

\twolineshloka*
{देहो देवालयः प्रोक्तो जीवो देवः सनातनः}
{त्यजेदज्ञाननिर्माल्यं सोऽहं भावेन पूजयेत्}


\begin{minipage}{\linewidth}
\dnsub{पीठ-पूजा}

\begin{multicols}{2}
\begin{enumerate}
\item ॐ आधारशक्त्यै नमः
\item ॐ मूलप्रकृत्यै नमः
\item ॐ आदिकूर्माय नमः 
\item ॐ आदिवराहाय नमः
\item ॐ अनन्ताय नमः
\item ॐ पृथिव्यै नमः
\item ॐ रत्नमण्डपाय नमः
\item ॐ रत्नवेदिकायै नमः
\item ॐ स्वर्णस्तम्भाय नमः
\item ॐ श्वेतच्छत्त्राय नमः
\item ॐ कल्पकवृक्षाय नमः
\item ॐ क्षीरसमुद्राय नमः 
\item ॐ सितचामराभ्यां नमः
\item ॐ योगपीठासनाय नमः
\end{enumerate}
\end{multicols}

\end{minipage}

\dnsub{गुरु ध्यानम्}

\twolineshloka*
{गुरुर्ब्रह्मा गुरुर्विष्णुर्गुरुर्देवो महेश्वरः}
{गुरुः साक्षात् परं ब्रह्म तस्मै श्री गुरवे नमः}


\sect{षोडशोपचार-पूजा}
\begin{center}

\twolineshloka*
{ध्यायामि तुलसीं देवीं श्यामां कमललोचनाम्}
{प्रसन्नवदनाम्भोजां वरदाम् अभयप्रदाम्}
\textbf{अस्मिन् क्षुपे तुलसीं ध्यायामि।}
\medskip

\twolineshloka*
{ध्यायामि विष्णुं वरदं तुलसीप्रियवल्लभम्}
{पीताम्बरं पद्मनेत्रं वासुदेवं वरप्रदम्}
\textbf{अस्मिन् आमलक-स्कन्धे महाविष्णुं ध्यायामि।}
\medskip

\twolineshloka*
{वासुदेवप्रिये देवि सर्वदेवस्वरूपिणि}
{आगच्छ पूजाभवने सदा सन्निहिता भव}

\twolineshloka*
{आगच्छाऽऽगच्छ देवेश तेजोराशे जगत्पते}
{क्रियमाणां मया पूजां वासुदेव गृहाण भोः} 
\textbf{तुलसीविष्णू आवाहयामि।}
\medskip

\twolineshloka*
{नानारत्नसमायुक्तं कार्तस्वरविभूषितम्}
{आसनं कृपया विष्णो तुलसि प्रतिगृह्यताम्}
\textbf{तुलसी-विष्णुभ्यां नमः, आसनं समर्पयामि।}
\medskip

\twolineshloka*
{नानानदीसमानीतं सुवर्णकलशस्थितम्}
{पाद्यं गृहाण तुलसि पापं मे विनिवारय}

\twolineshloka*
{गङ्गादिसर्वतीर्थेभ्यो वासुदेव मयाऽऽहृतम्}
{तोयमेतत्सुखस्पर्शं पाद्यार्थं प्रतिगृह्यताम्}
\textbf{तुलसी-विष्णुभ्यां नमः, पाद्यं समर्पयामि।}
\medskip

\twolineshloka*
{अर्घ्यं गृहाण देवि त्वं अच्युतप्रियवल्लभे}
{अक्षतादिसमायुक्तं अक्षय्यफलदायिनि}

\twolineshloka*
{नमस्ते देवदेवेश नमस्ते कमलापते}
{नमस्ते सर्वविनुत गृहाणार्घ्यं नमोऽस्तु ते}
\textbf{तुलसी-विष्णुभ्यां नमः, अर्घ्यं समर्पयामि।}
\medskip

\begin{minipage}{\linewidth}
\centering
\twolineshloka*
{गृहाणाचमनार्थाय विष्णुवक्षः स्थलालये}
{स्वच्छं तोयमिदं देवि सर्वपापविनाशिनि}

\twolineshloka*
{कर्पूरवासितं तोयं गङ्गादिभ्यः समाहृतम्}
{आचम्यतां जगन्नाथ मया दत्तं च भक्तितः}
\textbf{तुलसी-विष्णुभ्यां नमः, आचमनीयं समर्पयामि।}
\medskip
\end{minipage}

\twolineshloka*
{मधुपर्कं गृहाणेमं मधुसूदनवल्लभे}
{मधुदध्याज्यसंयुक्तं महापापविनाशिनि}

\twolineshloka*
{दध्याज्यमधुसंयुक्तं मधुपर्कं मयाऽऽहृतम्}
{गृहाण विष्णो वरद लक्ष्मीकान्त नमोऽस्तु ते}
\textbf{तुलसी-विष्णुभ्यां नमः, मधुपर्कं समर्पयामि।}
\medskip

\twolineshloka*
{पञ्चामृतं गृहाणेदं पञ्चपातकनाशिनि}
{दधिक्षीरसमायुक्तं दामोदरकुटुम्बिनि}

\twolineshloka*
{मध्वाज्यशर्करायुक्तं दधिक्षीरसमन्वितम्}
{पञ्चामृतं गृहाणेदं भक्तानामिष्टदायक}
\textbf{तुलसी-विष्णुभ्यां नमः, पञ्चामृतं समर्पयामि।}
\medskip

\twolineshloka*
{गङ्गागोदावरीकृष्णातुङ्गादिभ्यः समाहृतम्}
{सलिलं देवि तुलसि स्नानार्थं प्रतिगृह्यताम्}

\twolineshloka*
{गङ्गा कृष्णा च यमुना नर्मदा च सरस्वती}
{तुङ्गा गोदावरी वेणी क्षिप्रा सिन्धुर्घटप्रभा}

\twolineshloka*
{तापी पयोष्णी सरयूस्ताभ्यः स्नानार्थमाहृतम्}
{तोयमेतत्सुखस्पर्शं स्नानीयं गृह्यतां हरे}
\textbf{तुलसी-विष्णुभ्यां नमः, स्नानं समर्पयामि।}
\medskip

\twolineshloka*
{पीताम्बरमिदं दिव्यं पातकव्रजनाशिनि}
{पीताम्बरप्रिये देवि परिधत्स्व परात्परे}

\twolineshloka*
{सर्वभूषाधिके सौम्ये लोकलज्जानिवारणे}
{वाससी प्रतिगृह्णातु लक्ष्मीजानिरधोक्षजः}
\textbf{तुलसी-विष्णुभ्यां नमः, वस्त्रं समर्पयामि।}
\medskip

\twolineshloka*
{भूषणानि वरार्हाणि गृह्णीतं तुलसीश्वर}
{किरीटहारकेयूरकटकानि हरेऽमृते}
\textbf{तुलसी-विष्णुभ्यां नमः, आभरणानि समर्पयामि।}
\medskip

\twolineshloka*
{चन्दनागरुकर्पूरकस्तूरीकुङ्कुमान्वितम्}
{गन्धं स्वीकुरुतं देवौ रमेशहरिवल्लभे}
\textbf{तुलसी-विष्णुभ्यां नमः, गन्धान् धारयामि।}
\medskip

\twolineshloka*
{मल्लिकाकुन्दमन्दारजाजीवकुलचम्पकैः}
{शतपत्रैश्च कह्लारैरर्चये तुलसीहरी}
\textbf{तुलसी-विष्णुभ्यां नमः, पुष्पाणि समर्पयामि।}
\medskip


\sect{अङ्ग-पूजा}

\begin{tabular}{ll@{}l}
बृन्दायै &  अच्युताय नमः & - पादौ पूजयामि।\\
तुलस्यै & अनन्ताय   नमः  &- गुल्फौ पूजयामि।\\
जनार्दनप्रियायै &  तुलसीकान्ताय नमः & - जङ्घे पूजयामि।\\
जन्मनाशिन्यै &  गङ्गाधरपदाय नमः & - जानुनी पूजयामि \\
उत्तमायै &  उत्तमाय नमः & - ऊरू पूजयामि।\\
कमलाक्ष्यै &  कमलाक्षाय नमः & - कटिं पूजयामि।\\
नारायण्यै &  नारायणाय नमः & - नाभिं पूजयामि।\\
उन्नतायै &  उन्नताय नमः & - उदरं पूजयामि।\\
वरदायै &  वरदाय नमः & - वक्षः पूजयामि।\\
स्तव्यायै &  स्तव्याय नमः & - स्तनौ कौस्तुभं पूजयामि।\\
चतुर्भुजायै &  चतुर्भुजाय नमः & - भुजान् पूजयामि।\\
कम्बुकण्ठ्यै &  वनमालिने नमः & - कण्ठं पूजयामि।\\
कल्मषघ्न्यै &  कल्मषघ्नाय नमः & - कर्णौ पूजयामि।\\
मुनिप्रियायै &  मुनिप्रियाय नमः & - नेत्रे पूजयामि\\
शुभप्रदायै &  शुभप्रदाय नमः & - शिरः पूजयामि।\\
सर्वार्थदायिन्यै &  सर्वार्थदायिने नमः & - सर्वाण्यङ्गानि पूजयामि। \\
\end{tabular}


\dnsub{चतुर्विंशति नामपूजा}
\begin{multicols}{2}
\begin{enumerate}
\item ॐ केशवाय नमः
\item ॐ नारायणाय नमः
\item ॐ माधवाय नमः
\item ॐ गोविन्दाय नमः
\item ॐ विष्णवे नमः
\item ॐ मधुसूदनाय नमः
\item ॐ त्रिविक्रमाय नमः
\item ॐ वामनाय नमः
\item ॐ श्रीधराय नमः
\item ॐ हृषीकेशाय नमः
\item ॐ पद्मनाभाय नमः
\item ॐ दामोदराय नमः
\item ॐ सङ्कर्षणाय नमः
\item ॐ वासुदेवाय नमः
\item ॐ प्रद्युम्नाय नमः
\item ॐ अनिरुद्धाय नमः
\item ॐ पुरुषोत्तमाय नमः
\item ॐ अधोक्षजाय नमः
\item ॐ नृसिंहाय नमः
\item ॐ अच्युताय नमः
\item ॐ जनार्दनाय नमः
\item ॐ उपेन्द्राय नमः
\item ॐ हरये नमः
\item ॐ श्रीकृष्णाय नमः
\end{enumerate}
\end{multicols}

\begingroup
% \centering
\setlength{\columnseprule}{1pt}
\let\chapt\sect
\input{../namavali-manjari/100/Tulasi_108.tex}

\textbf{श्री-तुलसी-विष्णुभ्यां नमः}, नानाविध-परिमल-पत्र-पुष्पाणि समर्पयामि।
\endgroup



\twolineshloka*
{धूपं गृहाण वरदे दशाङ्गेन सुवासितम्}
{तुलस्यमृतसम्भूते धूतपापे नमोऽस्तु ते}

\twolineshloka*
{दशाङ्गो गुग्गुलूपेतः सुगन्धः सुमनोहरः}
{श्रीवत्साङ्क हृषीकेश धूपोऽयं प्रतिगृह्यताम्}
\textbf{तुलसी-विष्णुभ्यां नमः, धूपम् आघ्रापयामि।}
\medskip

\twolineshloka*
{वर्तित्रययुतं दीप्तं गोघृतेन समन्वितम्}
{दीपं देवि गृहाणेमं दैत्यारिहृदयस्थिते}

\twolineshloka*
{साज्यं त्रिवर्तिसंयुक्तं दीप्तं देव जनार्दन}
{गृहाण मङ्गलं दीपं त्रैलोक्यतिमिरं हर}
\textbf{तुलसी-विष्णुभ्यां नमः, दीपं दर्शयामि।}
\medskip

\twolineshloka*
{नानाभक्ष्यैश्च भोज्यैश्च फलैः क्षीरघृतादिभिः}
{नैवेद्यं गृह्यतां युक्तं नारायणमनःप्रिये}

\twolineshloka*
{भोज्यं चतुर्विधं चोष्यभक्ष्यसूपफलैर्युतम्}
{दधिमध्वाज्यसंयुक्तं गृह्यतामम्बुजेक्षण}
\textbf{तुलसी-विष्णुभ्यां नमः, महानैवेद्यं निवेदयामि।}
\medskip

\twolineshloka*
{कर्पूरचूर्णताम्बूलवल्लीपूगफलैर्युतम्}
{जगतः पितरावेतत्ताम्बूलं प्रतिगृह्यताम्}
\textbf{तुलसी-विष्णुभ्यां नमः, ताम्बूलं समर्पयामि।}
\medskip

\twolineshloka*
{नीराजनं गृहाणेदं कर्पूरैः कलितं मया}
{तुलस्यमृतसम्भूते गृहाण हरिवल्लभे}

\twolineshloka*
{चन्द्रादित्यौ च नक्षत्रं विद्युदग्निस्त्वमेव च}
{त्वमेव सर्वज्योतींषि कुर्यां नीराजनं हरे}
\textbf{तुलसी-विष्णुभ्यां नमः, कर्पूर-नीराजनं दर्शयामि।}
\medskip

\twolineshloka*
{प्रकृष्टपापनाशाय प्रकृष्टफलसिद्धये}
{युवां प्रदक्षिणी कुर्वे तुलसीशौ प्रसीदतम्}
\textbf{तुलसी-विष्णुभ्यां नमः, प्रदक्षिणं समर्पयामि।}
\medskip

\fourlineindentedshloka*
{नमोऽस्तु पीयूषसमुद्भवायै}{नमोऽस्तु पद्माक्षमनः प्रियायै}
{नमोऽस्तु जन्माप्यय-भीतिहन्त्र्यै}{नमस्तुलस्यै जगतां जनन्यै}

\twolineshloka*
{शङ्खचक्रगदापाणे द्वारकानिलयाच्युत}
{गोविन्द पुण्डरीकाक्ष रक्ष मां शरणागत(ता)म्}
\textbf{तुलसी-विष्णुभ्यां नमः, नमस्कारान् समर्पयामि।}
\medskip

\twolineshloka*
{पुष्पाञ्जलिं गृहाणेदं पङ्कजाक्षस्य वल्लभे}
{नमस्ते देवि तुलसि नताभीष्टफलप्रदे}

\twolineshloka*
{मन्दारनीलोत्पलकुन्दजाती पुन्नागमल्लीकरवीरपद्मैः}
{पुष्पाञ्जलिं ते जगदेकबन्धो हरे त्वदङ्घ्रौ विनिवेशयामि}
\textbf{तुलसी-विष्णुभ्यां नमः, मन्त्रपुष्पाञ्जलिं समर्पयामि।}
\medskip

\twolineshloka*
{आयुरारेग्यमतुलमैश्वर्यं पुत्रसम्पदः}
{देहि मे सकलान्कामान् तुलस्यमृतसम्भवे}

\fourlineindentedshloka*
{नमो नमः सुखवरपूजिताङ्घ्रये}
{नमो नमो निरुपममङ्गलात्मने}
{नमो नमो विपुलपदैकसिद्धये}
{नमो नमः परमदयानिधे हरे}
\nopagebreak[4]\textbf{तुलसी-विष्णुभ्यां नमः, प्रार्थनाः समर्पयामि।}
\medskip

\twolineshloka*
{नमस्ते देवि तुलसि नमस्ते मोक्षदायिनि}
{इदमर्घ्यं प्रदास्यामि सुप्रीता वरदा भव}

\twolineshloka*
{लक्ष्मीपते नमस्तुभ्यं तुलसीदामभूषण}
{इदमर्घ्यं प्रदास्यामि गृहाण गरुडध्वज}

\textbf{श्री-तुलस्यै महाविष्णवे च नमः}, इदमर्घ्यमिदमर्घ्यमिदमर्घ्यम्।

\twolineshloka*
{नमस्ते देवि तुलसि माधवेन समन्विता}
{प्रयच्छ सकलान्कामान् द्वादश्यां पूजिता मया}


अनेन पूजनेन श्री-तुलसी-विष्णू प्रीयेताम्।\\

\end{center}

\section{तुलसीविवाहविधिः}

(इक्षुदण्डनिर्मिते पुष्पाद्यलङ्कृते मण्टपे विवाहः।

तुलसी हरिद्राचन्दनकुङ्कुमपुष्पाद्यालङ्कृता स्वीयवेद्यां प्रथमं पूज्यते।

शुक्लाम्बरधरं + परमेश्वरप्रीत्यर्थं शुभे शोभने मुहूर्ते + शुभतिथौ तुलस्याः विष्णुना सह विवाहोत्सवमाचरिष्ये।

(अप उपस्पृश्य)

विष्णुं विवाहार्थे वरार्ह-वस्त्रालङ्करण-पुष्पमालादिभिः अलङ्कृत्य वाद्यघोषगीतपुरस्सरं विवाहमण्टपमानीय कर्ता नारिकेल-कदलीफलताम्बूलादिभिः उपसृत्य मण्टपे आसने प्रतिष्ठाप्य प्रार्थयेत्।

\twolineshloka*
{आगच्छ भगवन् देव अर्हयिष्यामि केशव}
{तुभ्यं ददामि तुलसीं प्रतीच्छन् कामदो भव}


आसनमिदं, अलङ्क्रियताम्।
पाद्यं समर्पयामि।

अर्घ्यं समर्पयामि, आचमनीयं समर्पयामि।
मधुपर्कं समर्पयामि।
हरिद्रालेप-मङ्गलविधिं समर्पयामि।
तैलाभ्यङ्गपूर्वकं मङ्गलस्नानं समर्पयामि।
वस्त्रालङ्करणपुष्पमालाः समर्पयामि।
गन्धान् धारयामि। गन्धोपरि हरिद्राकुङ्कुमं समर्पयामि।
पुष्पैः पूजयामि।

तुलसी-विष्णुभ्यां नमः, नानाविधपरिमलपत्रपुष्पाणि समर्पयामि।

(वधूवरौ परस्परमभिमुखौ स्थापयित्वा)

... गोत्रोद्भवां ... शर्मणः प्रपौत्रीं, ... शर्मणः पौत्रीं, ... शर्मणः पुत्रीं 
तुलसीनाम्नीम् इमां कन्यकां अजाय 
परब्रह्मणे श्री-विष्णवे वराय प्रतिपादयामि। (त्रिः उक्त्वा) 

\twolineshloka*
{अनादिमध्यनिधन त्रैलोक्यप्रतिपालक}
{इमां गृहाण तुलसीं विवाहविधिनेश्वर}

\twolineshloka*
{पार्वती-बीजसम्भूतां बृन्दाभस्मनि संस्थिताम्}
{अनादिमध्यनिधनां वल्लभां ते ददाम्यहम्}

\twolineshloka*
{पयोघृतैश्च सेवाभिः कन्यावद्वर्धितां मया}
{त्वत्प्रियां तुलसीं तुभ्यं ददामि त्वं गृहाण भोः}

वाद्यघोष-वेदस्वस्तिवाचन-मङ्गलाशीर्भिः उभौ मेलयित्वा गीतादिभिः सन्तोषयेत्।

सायमपि पुनः पूजां कृत्वा स्त्रीधनं यथाशक्तिं दद्यात्।

विवाहोत्सवपूर्तौ\textsf{---}\hfill\mbox{}

\twolineshloka*
{वैकुण्ठं गच्छ भगवन् तुलस्या सहितः प्रभो}
{मत्कृतं पूजनं गृह्य सन्तुष्टो भव सर्वदा}

\twolineshloka*
{गच्छ गच्छ सुरश्रेष्ठ स्वस्थानं परमेश्वर}
{यत्र ब्रह्मादयो देवाः तत्र गच्छ जनार्दन}


इति उभौ अभ्यनुज्ञापयेत् मङ्गलारार्तिकेन सह। तुलसीं यथापूर्वं रक्षेत्।

\fourlineindentedshloka*
{कायेन वाचा मनसेन्द्रियैर्वा}
{बुद्‌ध्याऽऽत्मना वा प्रकृतेः स्वभावात्}
{करोमि यद्यत् सकलं परस्मै}
{नारायणायेति समर्पयामि}

ॐ तत्सद्ब्रह्मार्पणमस्तु।

\closesection