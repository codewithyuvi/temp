% !TeX program = XeLaTeX
% !TeX root = ../pujavidhanam.tex

\setlength{\parindent}{0pt}
\chapt{एकादशीव्रतम् — श्री-महाविष्णुपूजा}

\sect{पूर्वाङ्गविघ्नेश्वरपूजा}

(आचम्य)
\twolineshloka*
{शुक्लाम्बरधरं विष्णुं शशिवर्णं चतुर्भुजम्}
{प्रसन्नवदनं ध्यायेत् सर्वविघ्नोपशान्तये}
 
प्राणान्  आयम्य।  ॐ भूः + भूर्भुवः॒ सुव॒रोम्।
 
(अप उपस्पृश्य, पुष्पाक्षतान् गृहीत्वा)\\
ममोपात्तसमस्त दुरितक्षयद्वारा \\
श्रीपरमेश्वरप्रीत्यर्थं करिष्यमाणस्य कर्मणः\\
 निर्विघ्नेन परिसमाप्त्यर्थम् आदौ विघ्नेश्वरपूजां करिष्ये।

\twolineshloka*
{ॐ ग॒णानां᳚ त्वा ग॒णप॑तिꣳ हवामहे क॒विं क॑वी॒नामु॑प॒मश्र॑वस्तमम्}
{ज्ये॒ष्ठ॒राजं॒ ब्रह्म॑णां ब्रह्मणस्पत॒ आ नः॑ शृ॒ण्वन्नू॒तिभिः॑ सीद॒ साद॑नम्}
अस्मिन् हरिद्राबिम्बे महागणपतिं ध्यायामि, आवाहयामि।\\


ॐ महागणपतये नमः  आसनं समर्पयामि।\\
पादयोः पाद्यं समर्पयामि। हस्तयोरर्घ्यं समर्पयामि।\\
आचमनीयं समर्पयामि।\\
ॐ भूर्भुवस्सुवः। शुद्धोदकस्नानं समर्पयामि।\\
स्नानानन्तरमाचमनीयं समर्पयामि।\\
वस्त्रार्थमक्षतान् समर्पयामि।\\
यज्ञोपवीताभरणार्थे अक्षतान् समर्पयामि।\\
दिव्यपरिमलगन्धान् धारयामि।\\
गन्धस्योपरि हरिद्राकुङ्कुमं समर्पयामि। अक्षतान् समर्पयामि। \\
पुष्पमालिकां समर्पयामि। पुष्पैः पूजयामि।

\dnsub{अर्चना}
% \setenumerate{label=\devanumber.}
% \renewcommand{\labelenumi}{\devanumber\theenumi.}
\begin{enumerate}%[label=\devanumber\value{enumi}]
\begin{minipage}{0.475\linewidth}   
\item ॐ सुमुखाय नमः
\item ॐ एकदन्ताय नमः
\item ॐ कपिलाय नमः
\item ॐ गजकर्णकाय नमः
\item ॐ लम्बोदराय नमः
\item ॐ विकटाय नमः
\item ॐ विघ्नराजाय नमः
\item ॐ विनायकाय नमः
\item ॐ धूमकेतवे नमः
  \end{minipage}
  \begin{minipage}{0.525\linewidth}
\item ॐ गणाध्यक्षाय नमः
\item ॐ फालचन्द्राय नमः
\item ॐ गजाननाय नमः
\item ॐ वक्रतुण्डाय नमः
\item ॐ शूर्पकर्णाय नमः
\item ॐ हेरम्बाय नमः
\item ॐ स्कन्दपूर्वजाय नमः
\item ॐ सिद्धिविनायकाय नमः
\item ॐ विघ्नेश्वराय नमः
  \end{minipage}
\end{enumerate}
नानाविधपरिमलपत्रपुष्पाणि समर्पयामि॥\\
धूपमाघ्रापयामि।\\
अलङ्कारदीपं सन्दर्शयामि।\\
नैवेद्यम्।\\
ताम्बूलं समर्पयामि।\\
कर्पूरनीराजनं समर्पयामि।\\
कर्पूरनीराजनानन्तरमाचमनीयं समर्पयामि।\\
{वक्रतुण्डमहाकाय कोटिसूर्यसमप्रभ।}\\
{अविघ्नं कुरु मे देव सर्वकार्येषु सर्वदा॥}\\
प्रार्थनाः समर्पयामि।

अनन्तकोटिप्रदक्षिणनमस्कारान् समर्पयामि।\\
छत्त्रचामरादिसमस्तोपचारान् समर्पयामि।\\


\sect{प्रधान पूजा - एकादशीपूजा}

\twolineshloka*
{शुक्लाम्बरधरं विष्णुं शशिवर्णं चतुर्भुजम्}
{प्रसन्नवदनं ध्यायेत् सर्वविघ्नोपशान्तये}
 
प्राणान्  आयम्य।  ॐ भूः + भूर्भुवः॒ सुव॒रोम्।

\dnsub{सङ्कल्पः}

ममोपात्तसमस्तदुरितक्षयद्वारा श्रीपरमेश्वरप्रीत्यर्थं शुभे शोभने मुहूर्ते अद्यब्रह्मणः
द्वितीयपरार्द्धे श्वेतवराहकल्पे वैवस्वतमन्वन्तरे अष्टाविंशतितमे कलियुगे प्रथमे पादे
जम्बूद्वीपे भारतवर्षे भरतखण्डे मेरोः दक्षिणेपार्श्वे शकाब्दे अस्मिन् वर्तमाने व्यावहारिके
 प्रभवादि षष्टिसंवत्सराणां मध्ये (	)\see{app:samvatsara_names} नाम संवत्सरे उत्तरायणे / दक्षिनायने 
(ग्रीष्म / वर्ष / शरद् / हेमन्त / शिशिर / वसन्त) ऋतौ  (मेष / वृषभ / मिथुन / कर्कटक / सिंह / कन्या / तुला / 
वृश्चिक / धनुर् / मकर / कुम्भ / मीन) मासे (शुक्ल / कृष्ण) पक्षे (एकादश्यां / द्वादश्यां) शुभतिथौ
(इन्दु / भौम / बुध / गुरु / भृगु / स्थिर / भानु) वासरयुक्तायाम्
(  )\see{app:nakshatra_names} नक्षत्र (  )\see{app:yoga_names} नाम  योग  (  ) करण युक्तायां च एवं\-गुण\-विशेषण\-विशिष्टायाम्
अस्याम् (एकादश्यां / द्वादश्यां) शुभतिथौ 
अस्माकं सहकुटुम्बानां क्षेमस्थैर्य-धैर्य-वीर्य-विजय आयुरारोग्य ऐश्वर्याभिवृद्ध्यर्थम्
 धर्मार्थकाममोक्ष\-चतुर्विधफलपुरुषार्थसिद्ध्यर्थं पुत्रपौत्राभि\-वृद्ध्यर्थम् इष्टकाम्यार्थसिद्ध्यर्थम्
मम इहजन्मनि पूर्वजन्मनि जन्मान्तरे च सम्पादितानां ज्ञानाज्ञानकृतमहा\-पातकचतुष्टय
व्यतिरिक्तानां रहस्यकृतानां प्रकाशकृतानां सर्वेषां पापानां सद्य अपनोदनद्वारा सकल 
पापक्षयार्थं श्रीभूमिनीलासमेतश्रीमहाविष्णुप्रीत्यर्थं यावच्छक्ति ध्यानावाहनादि 
षोडशोपचारपूजां करिष्ये तदङ्गं कलशपूजां च करिष्ये।


श्रीविघ्नेश्वराय नमः यथास्थानं प्रतिष्ठापयामि।\\
(गणपति प्रसादं शिरसा गृहीत्वा)

\renewcommand{\devaName}{विष्णु}
\dnsub{आसन-पूजा}
\centerline{पृथिव्या  मेरुपृष्ठ  ऋषिः।  सुतलं  छन्दः।  कूर्मो  देवता॥}
\twolineshloka*
{पृथ्वि  त्वया  धृता  लोका  देवि  त्वं  विष्णुना  धृता}
{त्वं  च  धारय  मां  देवि  पवित्रं  चाऽऽसनं  कुरु}


\dnsub{घण्टापूजा}
\twolineshloka*
{आगमार्थं तु देवानां गमनार्थं तु रक्षसाम्}
{घण्टारवं करोम्यादौ देवताऽऽह्वानकारणम्}


\dnsub{कलशपूजा}
ॐ कलशाय नमः दिव्यगन्धान् धारयामि।\\
ॐ गङ्गायै नमः। ॐ यमुनायै नमः। ॐ गोदावर्यै नमः।  ॐ सरस्वत्यै नमः। ॐ नर्मदायै नमः। ॐ सिन्धवे नमः। ॐ कावेर्यै नमः।\\
ॐ सप्तकोटिमहातीर्थान्यावाहयामि।\\[-0.25ex]

(अथ कलशं स्पृष्ट्वा जपं कुर्यात्) \\
आपो॒ वा इ॒द सर्वं॒ विश्वा॑ भू॒तान्याप॑ प्रा॒णा वा आप॑ प॒शव॒ आपो\-ऽन्न॒मापोऽमृ॑त॒माप॑ स॒म्राडापो॑ वि॒राडाप॑ स्व॒राडाप॒श्\-छन्दा॒स्यापो॒ ज्योती॒ष्यापो॒ यजू॒ष्याप॑ स॒त्यमाप॒ सर्वा॑ दे॒वता॒ आपो॒ भूर्भुव॒ सुव॒राप॒ ओम्॥\\

\twolineshloka* 
{कलशस्य मुखे विष्णुः कण्ठे रुद्रः समाश्रितः}
{मूले तत्र स्थितो ब्रह्मा मध्ये मातृगणाः स्मृताः}
\threelineshloka* 
{कुक्षौ तु सागराः सर्वे सप्तद्वीपा वसुन्धरा}
{ऋग्वेदोऽथ यजुर्वेदः सामवेदोऽप्यथर्वणः}
{अङ्गैश्च सहिताः सर्वे कलशाम्बुसमाश्रिताः}
\twolineshloka* 
{गङ्गे च यमुने चैव गोदावरि सरस्वति}
{नर्मदे सिन्धुकावेरि जलेऽस्मिन् सन्निधिं कुरु}
\twolineshloka*
{सर्वे समुद्राः सरितः तीर्थानि च ह्रदा नदाः}
{आयान्तु देवपूजार्थं दुरितक्षयकारकाः}

\centerline{ॐ भूर्भुवः॒ सुवो॒ भूर्भुवः॒ सुवो॒ भूर्भुवः॒ सुवः॑।}

(इति कलशजलेन सर्वोपकरणानि आत्मानं च प्रोक्ष्य।)


\dnsub{आत्म-पूजा}
ॐ आत्मने नमः, दिव्यगन्धान् धारयामि।
\begin{multicols}{2}
१. ॐ आत्मने नमः\\
२. ॐ अन्तरात्मने नमः\\
३. ॐ योगात्मने नमः\\
४. ॐ जीवात्मने नमः\\
५. ॐ परमात्मने नमः\\
६. ॐ ज्ञानात्मने नमः
\end{multicols}
समस्तोपचारान् समर्पयामि।

\twolineshloka*
{देहो देवालयः प्रोक्तो जीवो देवः सनातनः}
{त्यजेदज्ञाननिर्माल्यं सोऽहं भावेन पूजयेत्}


\begin{minipage}{\linewidth}
\dnsub{पीठ-पूजा}

\begin{multicols}{2}
\begin{enumerate}
\item ॐ आधारशक्त्यै नमः
\item ॐ मूलप्रकृत्यै नमः
\item ॐ आदिकूर्माय नमः 
\item ॐ आदिवराहाय नमः
\item ॐ अनन्ताय नमः
\item ॐ पृथिव्यै नमः
\item ॐ रत्नमण्डपाय नमः
\item ॐ रत्नवेदिकायै नमः
\item ॐ स्वर्णस्तम्भाय नमः
\item ॐ श्वेतच्छत्त्राय नमः
\item ॐ कल्पकवृक्षाय नमः
\item ॐ क्षीरसमुद्राय नमः 
\item ॐ सितचामराभ्यां नमः
\item ॐ योगपीठासनाय नमः
\end{enumerate}
\end{multicols}

\end{minipage}

\dnsub{गुरु ध्यानम्}

\twolineshloka*
{गुरुर्ब्रह्मा गुरुर्विष्णुर्गुरुर्देवो महेश्वरः}
{गुरुः साक्षात् परं ब्रह्म तस्मै श्री गुरवे नमः}


\begin{center}

\sect{षोडशोपचारपूजा}

\threelineshloka*
 {ध्यायेत् चतुर्भुजं देवं शङ्खचक्रगदाधरम्}
{पीताम्बरयुगोपेतं लक्ष्मीयुक्तं विभूषितम्}
{लसत्कौस्तुभशोभाढ्यं मेघश्यामं सुलोचनम्}
अस्मिन् बिम्बे श्रीभूमिनीलासमेतं महाविष्णुं ध्यायामि।
\medskip

\twolineshloka*
{स॒हस्र॑शीर्‌षा॒ पुरु॑षः। स॒ह॒स्रा॒क्षः स॒हस्र॑पात्}
{स भूमिं॑ वि॒श्वतो॑ वृ॒त्वा। अत्य॑तिष्ठद्दशाङ्गु॒लम्}
अस्मिन् बिम्बे श्रीभूमिनीलासमेतं महाविष्णुम् आवाहयामि।
\medskip

 \twolineshloka*
 {पुरु॑ष ए॒वेद सर्वम्। यद्भू॒तं यच्च॒ भव्यम्}
 {उ॒तामृ॑त॒त्वस्येशा॑नः। यदन्ने॑नाति॒रोह॑ति}
 आसनं समर्पयामि।\medskip

\twolineshloka*
{ए॒तावा॑नस्य महि॒मा। अतो॒ ज्यायाश्च॒ पूरु॑षः}
{पादोऽस्य॒ विश्वा॑ भू॒तानि॑। त्रि॒पाद॑स्या॒मृतं॑ दि॒वि}
 पाद्यं समर्पयामि।\medskip
 
\twolineshloka*
{त्रि॒पादू॒र्ध्व उदै॒त्पुरु॑षः। पादोऽस्ये॒हाऽऽभ॑वा॒त्पुन॑}
{ततो॒ विश्व॒ङ्व्य॑क्रामत्। सा॒श॒ना॒न॒श॒ने अ॒भि}
 अर्घ्यं समर्पयामि।\medskip

\twolineshloka*
{तस्माद्वि॒राड॑जायत। वि॒राजो॒ अधि॒ पूरु॑षः}
{स जा॒तो अत्य॑रिच्यत। प॒श्चाद्भूमि॒मथो॑ पु॒रः}
 आचमनीयं समर्पयामि।\medskip

\twolineshloka*
{यत्पुरु॑षेण ह॒विषा। दे॒वा य॒ज्ञमत॑न्वत}
{व॒स॒न्तो अ॑स्याऽऽसी॒दाज्यम्। ग्री॒ष्म इ॒ध्मः श॒रद्ध॒विः}
मधुपर्कं समर्पयामि।\medskip

 \twolineshloka*
 {स॒प्तास्या॑ऽऽसन्  परि॒धय॑। त्रिः स॒प्त स॒मिध॑ कृ॒ताः}
 {दे॒वा यद्य॒ज्ञं त॑न्वा॒नाः। अब॑ध्न॒न् पु॑रुषं प॒शुम्}
 शुद्धोदकस्नानं समर्पयामि। स्नानानन्तरम् आचमनीयं समर्पयामि।\medskip

 \twolineshloka*
 {तं य॒ज्ञं ब॒र्{}हिषि॒ प्रौक्षन्। पुरु॑षं जा॒तम॑ग्र॒तः}
 {तेन॑ दे॒वा अय॑जन्त। सा॒ध्या ऋष॑यश्च॒ ये}
 वस्त्रं समर्पयामि।\medskip

\twolineshloka*
{तस्माद्य॒ज्ञाथ्स॑र्व॒हुत॑। सम्भृ॑तं पृषदा॒ज्यम्}
{प॒शूस्ताश्च॑क्रे वाय॒व्यान्। आ॒र॒ण्यान्ग्रा॒म्याश्च॒ ये}
 यज्ञोपवीतं समर्पयामि।\medskip

\twolineshloka*
{तस्माद्य॒ज्ञाथ्स॑र्व॒हुत॑। ऋच॒ सामा॑नि जज्ञिरे}
{छन्दासि जज्ञिरे॒ तस्मात्। यजु॒स्तस्मा॑दजायत}
 दिव्यपरिमलगन्धान् धारयामि। गन्धस्योपरि हरिद्राकुङ्कुमं समर्पयामि। अक्षतान् समर्पयामि।\medskip

\twolineshloka*
{तस्मा॒दश्वा॑ अजायन्त। ये के चो॑भ॒याद॑तः}
{गावो॑ ह जज्ञिरे॒ तस्मात्। तस्माज्जा॒ता अ॑जा॒वय॑}
 पुष्पाणि समर्पयामि।  पुष्पैः पूजयामि।

\dnsub{अङ्गपूजा}
\begin{longtable}{ll@{— }l}
१.&	ॐ वराहाय नमः & पादौ पूजयामि	\\
२.&	सङ्कर्षणाय नमः & गुल्फौ पूजयामि\\
३.&	कालात्मने नमः & जानुनी पूजयामि	\\
४.&	विश्वरूपाय नमः & जङ्घे पूजयामि\\
५.&	क्रोढाय नमः & ऊरू पूजयामि	\\
६.&	भोक्त्रे नमः & कटिं पूजयामि	\\
७.&	विष्णवे नमः & मेढ्रं पूजयामि		\\
८.&	हिरण्यगर्भाय नमः & नाभिं पूजयामि\\
९.&	श्रीवत्सधारिणे नमः & कुक्षिं पूजयामि	\\
१०.& परमात्मने नमः & हृदयं पूजयामि\\
११.& सर्वास्त्रधारिणे नमः & वक्षः पूजयामि	\\
१२.& वनमालिने नमः & कण्ठं पूजयामि\\
१३.& सर्वात्मने नमः & मुखं पूजयामि	\\
१४.&	 सहस्राक्षाय नमः & नेत्राणि पूजयामि\\
१५.& सुप्रभाय नमः & ललाटं पूजयामि	\\
१६.& चम्पकनासिकाय नमः & नासिकां पूजयामि	\\
१७.& सर्वेशाय नमः & कर्णौ पूजयामि	\\
१८.& सहस्रशिरसे नमः & शिरः पूजयामि\\
१९.& नीलमेघनिभाय नमः & केशान् पूजयामि	\\
२०.& महापुरुषाय नमः & सर्वाणि अङ्गानि पूजयामि	\\
\end{longtable}

\dnsub{चतुर्विंशति नामपूजा}
\begin{multicols}{2}
\begin{enumerate}
\item ॐ केशवाय नमः
\item ॐ नारायणाय नमः
\item ॐ माधवाय नमः
\item ॐ गोविन्दाय नमः
\item ॐ विष्णवे नमः	
\item ॐ मधुसूदनाय नमः
\item ॐ त्रिविक्रमाय नमः
\item ॐ वामनाय नमः
\item ॐ श्रीधराय नमः
\item ॐ हृषीकेशाय नमः
\item ॐ पद्मनाभाय नमः
\item ॐ दामोदराय नमः
\item ॐ सङ्कर्षणाय नमः
\item ॐ वासुदेवाय नमः
\item ॐ प्रद्युम्नाय नमः
\item ॐ अनिरुद्धाय नमः
\item ॐ पुरुषोत्तमाय नमः
\item ॐ अधोक्षजाय नमः
\item ॐ नृसिंहाय नमः
\item ॐ अच्युताय नमः
\item ॐ जनार्दनाय नमः
\item ॐ उपेन्द्राय नमः 
\item ॐ हरये नमः
\item ॐ श्रीकृष्णाय नमः
\end{enumerate}
\end{multicols}
\clearpage

\begingroup
\setlength{\columnseprule}{1pt}
\let\chapt\sect
\input{../namavali-manjari/1000/Vishnu_1000.tex}
\input{../namavali-manjari/100/Krishna_108.tex}
\endgroup


  
\sect{उत्तराङ्गपूजा}

\twolineshloka*
{यत्पुरु॑षं॒ व्य॑दधुः। क॒ति॒धा व्य॑कल्पयन्}
{मुखं॒ किम॑स्य॒ कौ बा॒हू। कावू॒रू पादा॑वुच्येते}
श्री भूमीनीलासमेतमहाविष्णवे नमः धूपमाघ्रापयामि।\medskip
 
\twolineshloka*
{ब्रा॒ह्म॒णोऽस्य॒ मुख॑मासीत्। बा॒हू रा॑ज॒न्य॑ कृ॒तः}
{ऊ॒रू तद॑स्य॒ यद्वैश्य॑। प॒द्भ्या शू॒द्रो अ॑जायत}
उद्दीप्यस्व जातवेदोऽप॒घ्नन्निर्ऋ॑तिं॒ मम॑।\\
 प॒शूश्च॒ मह्य॒माव॑ह॒ जीव॑नं च॒ दिशो॑ दिश॥ \\
मा नो॑ हिसीज्जातवेदो॒ गामश्वं॒ पुरु॑षं॒ जग॑त्।\\
अबि॑भ्र॒दग्न॒ आग॑हि श्रि॒या मा॒ परि॑पातय॥ \\
श्री भूमीनीलासमेतमहाविष्णवे नमः अलङ्कारदीपं सन्दर्शयामि।\medskip

ॐ भूर्भुवः॒ सुवः॑। + ब्र॒ह्मणे॒ स्वाहा᳚।
 \twolineshloka*
 {च॒न्द्रमा॒ मन॑सो जा॒तः। चक्षो॒ सूर्यो॑ अजायत}
 {मुखा॒दिन्द्र॑श्चा॒ग्निश्च॑। प्रा॒णाद्वा॒युर॑जायत}
श्रीभूमिनीलासमेत महाविष्णवे नमः (	) निवेदयामि, \\
अमृतापिधानमसि। निवेदनानन्तरम् आचमनीयं समर्पयामि।\medskip

\twolineshloka*
{नाभ्या॑ आसीद॒न्तरि॑क्षम्। शी॒र्ष्णो द्यौः सम॑वर्तत}
{प॒द्भ्यां भूमि॒र्दिश॒ श्रोत्रात्। तथा॑ लो॒का अ॑कल्पयन्}

\twolineshloka*
{पूगीफलसमायुक्तं नागवल्लीदलैर्युतम्}
{कर्पूरचूर्णसंयुक्तं ताम्बूलं प्रतिगृह्यताम्}
श्री भूमीनीलासमेतमहाविष्णवे नमः कर्पूरताम्बूलं समर्पयामि।\medskip

\twolineshloka*
{वेदा॒हमे॒तं पुरु॑षं म॒हान्तम्। आ॒दि॒त्यव॑र्णं॒ तम॑स॒स्तु पा॒रे}
{सर्वा॑णि रू॒पाणि॑ वि॒चित्य॒ धीर॑। नामा॑नि कृ॒त्वाऽभि॒वद॒\an{} यदास्ते}
श्री भूमीनीलासमेतमहाविष्णवे नमः समस्त अपराध क्षमापनार्थं कर्पूरनीराजनं दर्शयामि।\\
कर्पूरनीरजनानन्तरम् आचमनीयं समर्पयामि।\medskip

\twolineshloka*
 {धा॒ता पु॒रस्ता॒द्यमु॑दाज॒हार॑। श॒क्रः प्रवि॒द्वान्  प्र॒दिश॒श्चत॑स्रः}
 {तमे॒वं वि॒द्वान॒मृत॑ इ॒ह भ॑वति। नान्यः पन्था॒ अय॑नाय विद्यते}

 यो॑ऽपां पुष्पं॒ वेद॑। पुष्प॑वान् प्र॒जावान् पशु॒मान् भ॑वति।\\
च॒न्द्रमा॒ वा अ॒पां पुष्पम्। पुष्प॑वान् प्र॒जावान् पशु॒मान् भ॑वति।\\
य ए॒वं वेद॑। यो॑ऽपामा॒यत॑नं॒ वेद॑। आ॒यत॑नवान् भवति।\medskip

ओं तद्ब्र॒ह्म। ओं तद्वा॒युः। ओं तदा॒त्मा।\\ ओं᳚ तथ्स॒त्यम्‌।
ओं᳚ तथ्सर्वम्᳚‌। ओं तत्पुरो॒र्नमः॥\medskip

अन्तश्चरति॑ भूते॒षु॒ गुहायां वि॑श्वमू॒र्तिषु। \\
त्वं यज्ञस्त्वं वषट्कारस्त्वमिन्द्रस्त्व\\ रुद्रस्त्वं विष्णुस्त्वं ब्रह्म त्वं॑ प्रजा॒पतिः। \\
त्वं त॑दाप॒ आपो॒ ज्योती॒ रसो॒ऽमृतं॒ ब्रह्म॒ भूर्भुवः॒ सुव॒रोम्‌॥\medskip

\medskip

श्री भूमीनीलासमेतमहाविष्णवे नमः वेदोक्तमन्त्रपुष्पाञ्जलिं समर्पयामि।\medskip

\twolineshloka*
{सुवर्णरजतैर्युक्तं चामीकरविनिर्मितम्}
{स्वर्णपुष्पं प्रदास्यामि गृह्यतां मधुसूदन}
स्वर्णपुष्पं समर्पयामि।\medskip

\twolineshloka*
{प्रदक्षिणं करोम्यद्य पापानि नुत माधव}
{मयार्पितान्यशेषाणि परिगृह्य कृपां कुरु}

\twolineshloka*
{यानि कानि च पापानि जन्मान्तरकृतानि च}
{तानि तानि विनश्यन्ति प्रदक्षिण पदे पदे}

\twolineshloka*
{नमस्ते देवदेवेश नमस्ते भक्तवत्सल}
{नमस्ते पुण्डरीकाक्ष वासुदेवाय ते नमः}

\twolineshloka*
{नमः सर्वहितार्थाय जगदाधाररूपिणे}
{साष्टाङ्गोऽयं प्रणामोऽस्तु जगन्नाथ मया कृतः}
अनन्तकोटिप्रदक्षिणनमस्कारान् समर्पयामि।\medskip

\twolineshloka*
{य॒ज्ञेन॑ य॒ज्ञम॑यजन्त दे॒वाः। तानि॒ धर्मा॑णि प्रथ॒मान्या॑सन्}
{ते ह॒ नाकं॑ महि॒मान॑ सचन्ते। यत्र॒ पूर्वे॑ सा॒ध्याः सन्ति॑ दे॒वाः}
छत्त्रचामरादिसमस्तोपचारान् समर्पयामि।\medskip

\dnsub{अर्घ्यप्रदानम्}
ममोपात्त समस्तदुरितक्षयद्वारा श्रीपरमेश्वरप्रीत्यर्थम् एकादशीपुण्यकाले महाविष्णुपूजान्ते क्षीरार्घ्यप्रदानं करिष्ये॥
\medskip

\twolineshloka*
{एकादश्यामुपोष्यैव पारणात् पूर्वकालतः}
{इदमर्घ्यं प्रदास्यामि गृहाण सुरवन्दित}
	महाविष्णवे नमः इदमर्घ्यमिदमर्घ्यमिदमर्घ्यम्॥\medskip

\twolineshloka*
{नमोऽस्तु केशवादिभ्यः सर्वलोकैकवन्दिताः}
{इदमर्घ्यं प्रदास्यामि सुप्रीतो भव सर्वदा}
	केशवादिभ्यः इदमर्घ्यमिदमर्घ्यमिदमर्घ्यम्।\medskip

\twolineshloka*
{कूर्मरूपाय देवाय मत्स्यरूप नमोऽस्तुते।}
{नीलमेघस्वरूपाय अर्घ्यं दत्तं मया प्रभो}
	विष्णवे नमः इदमर्घ्यमिदमर्घ्यमिदमर्घ्यम्॥\medskip

\twolineshloka*
{क्षीरोद्भवे महालक्ष्मि सुप्रसन्ने सुरेश्वरि}
{सर्वप्रदे जगद्वन्द्ये गृह्णीदार्घ्यमिदं रमे॥}
	महालक्ष्म्यै नमः इदमर्घ्यमिदमर्घ्यमिदमर्घ्यम्।\\
अनेन अर्घ्यप्रदानेन भगवान् सर्वात्मकः\\ श्री लक्ष्मीनारायणः प्रीयताम्।\medskip

\twolineshloka*
{हिरण्यगर्भगर्भस्थं हेमबीजं विभावसोः}
{अनन्तपुण्यफलदम् अतः शान्तिं प्रयच्छ मे}

एकादशीपुण्यकाले अस्मिन् मया क्रियमाण\\
महाविष्णुपूजायां यद्देयमुपायनदानं तत्प्रत्यायाम्नार्थं हिरण्यं\\
श्रीभूमिनीलासमेत श्री महाविष्णुप्रीतिं कामयमानः\\
मनसोद्दिष्टाय ब्राह्मणाय सम्प्रददे नमः न मम।\\ 
अनया पूजया श्रीभूमिनीलासमेतः श्रीमहाविष्णुः प्रीयताम्। 
 
\dnsub{विसर्जनम्}

\twolineshloka*
{यस्य स्मृत्या च नामोक्त्या तपः पूजा क्रियादिषु}
{न्यूनं सम्पूर्णतां याति सद्यो वन्दे तमच्युतम्}

\twolineshloka*
{इदं व्रतं मया देव कृतं प्रीत्यै तव प्रभो}
{न्यूनं सम्पूर्णतां यातु त्वत्प्रसादाज्जनार्द्दन}\medskip

अस्मात् बिम्बात् श्रीभूमिनीलासमेतश्रीमहाविष्णुं यथास्थानं प्रतिष्ठापयामि (अक्षतानर्पित्वा देवमुत्सर्जयेत्।)\\
अनया पूजया श्रीभूमिनीलासमेतः श्रीमहाविष्णुः प्रीयताम्।\medskip

\fourlineindentedshloka*
{कायेन वाचा मनसेन्द्रियैर्वा}
{बुद्‌ध्याऽऽत्मना वा प्रकृतेः स्वभावात्}
{करोमि यद्यत् सकलं परस्मै}
{नारायणायेति समर्पयामि}


ॐ तत्सद्ब्रह्मार्पणमस्तु।\medskip

\twolineshloka* 
{सालग्रामशिलावारि पापहारि शरीरिणाम्}
{आजन्मकृतपापानां प्रायश्चित्तं दिने दिने}

\twolineshloka*
{अकालमृत्युहरणं सर्वव्याधिनिवारणम्}
{सर्वपापक्षयकरं विष्णुपादोदकं शुभम्}
 इति तीर्थं पीत्वा शिरसि प्रसादं धारयेत्।

\end{center}

\dnsub{उत्तरस्मिन्   दिने पारणम्}

\twolineshloka*
{अज्ञानतिमिरान्धस्य व्रतेनानेन केशव}
{प्रसीद सुमुखो नाथ ज्ञानदृष्टिप्रदो भव}

\closesection

