% !TeX program = XeLaTeX
% !TeX root = ../pujavidhanam.tex

\sect{प्रधान-पूजा — साम्ब-परमेश्वर-पूजा (द्वितीय-यामः)}

\twolineshloka*
{शुक्लाम्बरधरं विष्णुं शशिवर्णं चतुर्भुजम्}
{प्रसन्नवदनं ध्यायेत् सर्वविघ्नोपशान्तये}
 
प्राणान्  आयम्य।  ॐ भूः + भूर्भुवः॒ सुव॒रोम्।

ममोपात्त-समस्त-दुरित-क्षयद्वारा श्री-परमेश्वर-प्रीत्यर्थं शुभे शोभने मुहूर्ते अद्य ब्रह्मणः
द्वितीयपरार्धे श्वेतवराहकल्पे वैवस्वतमन्वन्तरे अष्टाविंशतितमे कलियुगे प्रथमे पादे
जम्बूद्वीपे भारतवर्षे भरतखण्डे मेरोः दक्षिणे पार्श्वे शकाब्दे अस्मिन् वर्तमाने व्यावहारिकणां
 प्रभवादि षष्ट्याः संवत्सराणां मध्ये \mbox{(~~~)}\see{app:samvatsara_names} नाम संवत्सरे \textbf{उत्तरायणे} 
\textbf{शिशिर}-ऋतौ  \textbf{कुम्भ}-मासे \textbf{कृष्ण}-पक्षे त्र्योदश्यां/चतुर्दश्यां शुभतिथौ
(इन्दु / भौम / बुध / गुरु / भृगु / स्थिर / भानु) वासरयुक्तायाम्
\mbox{(~~~)}\see{app:nakshatra_names} नक्षत्र \mbox{(~~~)}\see{app:yoga_names} नाम  योग  \mbox{(~~~)} करण युक्तायां च एवं गुण विशेषण विशिष्टायाम्
अस्याम् (त्र्योदश्यां/चतुर्दश्यां) शुभतिथौ 
अस्माकं सहकुटुम्बानां क्षेमस्थैर्य-धैर्य-वीर्य-विजय-आयुरारोग्य-ऐश्वर्याभिवृद्ध्यर्थम्
 धर्मार्थकाममोक्ष\-चतुर्विधफलपुरुषार्थसिद्ध्यर्थं पुत्रपौत्राभि\-वृद्ध्यर्थम् इष्टकाम्यार्थसिद्ध्यर्थम्
मम इहजन्मनि पूर्वजन्मनि जन्मान्तरे च सम्पादितानां ज्ञानाज्ञानकृतमहा\-पातकचतुष्टय-व्यतिरिक्तानां रहस्यकृतानां प्रकाशकृतानां सर्वेषां पापानां सद्य अपनोदनद्वारा सकल-पापक्षयार्थं शिवरात्रौ श्री-साम्ब-परमेश्वर-प्रीत्यर्थं द्वितीय-यामपूजां करिष्ये।

\sect{षोडशोपचार-पूजा}

\fourlineindentedshloka*
{ध्यायेन्नित्यं महेशं रजतगिरिनिभं चारुचन्द्रावतंसम्}
{रत्नाकल्पोज्ज्वलाङ्गं परशुमृगवराभीतिहस्तं प्रसन्नम्}
{पद्मासीनं समन्तात् स्तुतममरगणैर्व्याघ्रकृत्तिं वसानम्}
{विश्वाद्यं विश्वबीजं निखिलभयहरं पञ्चवक्त्रं त्रिनेत्रम्}

अस्मिन् बिम्बे श्री-साम्ब-परमेश्वरं ध्यायामि। 

नम॑स्ते रुद्र म॒न्यव॑ उ॒तो त॒ इष॑वे॒ नमः॑। नम॑स्ते अस्तु॒ धन्व॑ने बा॒हुभ्या॑मु॒त ते॒ नमः॑॥ ॐ ह्रीं न॒मः शि॒वाय॑। स॒द्योजा॒तं प्र॑पद्यामि। आवाहयामि॥१॥

या त॒ इषुः॑ शि॒वत॑मा शि॒वं ब॒भूव॑ ते॒ धनुः॑। शि॒वा श॑र॒व्या॑ या तव॒ तया॑ नो रुद्र मृडय॥ ॐ ह्रीं न॒मः शि॒वाय॑। स॒द्योजा॒ताय॒ वै नमो॒ नमः॑। आसनं समर्पयामि॥२॥

या ते॑ रुद्र शि॒वा त॒नूरघो॒राऽपा॑पकाशिनी। तया॑ नस्त॒नुवा॒ शन्त॑मया॒ गिरि॑शन्ता॒\-भिचा॑कशीहि॥ ॐ ह्रीं न॒मः शि॒वाय॑। भ॒वे भ॑वे॒ नाति॑ भवे भवस्व॒ माम्। पादयोः पाद्यं समर्पयामि॥३॥

यामिषुं॑ गिरिशन्त॒ हस्ते॒ बिभ॒र्ष्यस्त॑वे। शि॒वां गि॑रित्र॒ तां कु॑रु॒ मा हिꣳ॑सीः॒ पुरु॑षं॒ जग॑त्॥ ॐ ह्रीं न॒मः शि॒वाय॑। भ॒वोद्भ॑वाय॒ नमः॑॥ अर्घ्यं समर्पयामि॥४॥

शि॒वेन॒ वच॑सा त्वा॒ गिरि॒शाच्छा॑वदामसि। यथा॑ नः॒ सर्व॒मिज्जग॑दय॒क्ष्मꣳ सु॒मना॒ अस॑त्॥ ॐ ह्रीं न॒मः शि॒वाय॑। वा॒म॒दे॒वाय॒ नमः॑। आचमनीयं समर्पयामि॥५॥

अध्य॑वोचदधिव॒क्ता प्र॑थ॒मो दैव्यो॑ भि॒षक्। अहीꣴ॑श्च॒ सर्वा᳚ञ्ज॒म्भय॒-न्थ्सर्वा᳚श्च यातुधा॒न्यः॑॥ ॐ ह्रीं न॒मः शि॒वाय॑। ज्ये॒ष्ठाय॒ नमः॑। मधुपर्कं समर्पयामि॥६॥

अ॒सौ यस्ता॒म्रो अ॑रु॒ण उ॒त ब॒भ्रुः सु॑म॒ङ्गलः॑। ये चे॒माꣳ रु॒द्रा अ॒भितो॑ दि॒क्षु श्रि॒ताः स॑हस्र॒शोऽवै॑षा॒ꣳ॒ हेड॑ ईमहे॥ ॐ ह्रीं न॒मः शि॒वाय॑। श्रे॒ष्ठाय॒ नमः॑। स्नानं समर्पयामि। 

% !TeX program = XeLaTeX
% !TeX root = ../vedamantrabook.tex


\sect{महान्यासः}

\sect{पञ्चाङ्गरुद्रन्यासः रावणोक्ता पञ्चाङ्गप्रार्थना-सहितम्}

\twolineshloka
{ओङ्कारमन्त्रसंयुक्तं नित्यं ध्यायन्ति योगिनः}
{कामदं मोक्षदं तस्मै नकाराय नमो नमः}

नम॑स्ते रुद्र म॒न्यव॑ उ॒तो त॒ इष॑वे॒ नमः॑। नम॑स्ते अस्तु॒ धन्व॑ने बा॒हुभ्या॑मु॒त ते॒ नमः॑॥ या त॒ इषुः॑ शि॒वत॑मा शि॒वं ब॒भूव॑ ते॒ धनुः॑। शि॒वा श॑र॒व्या॑ या तव॒ तया॑ नो रुद्र मृडय॥\\
{\scriptsize (EAST)}\\
कं खं गं घं ङं। यरलवशषसहोम्।ॐ नमो भगवते॑ रुद्रा॒य। पूर्वाङ्गरुद्राय नमः। 
\medskip
\twolineshloka
{महादेवं महात्मानं महापातकनाशनम्}
{महापापहरं वन्दे मकाराय नमो नमः}

अपै॑तु मृ॒त्युर॒मृतं॑ न॒ आग॑न्वैवस्व॒तो नो॒ अभ॑यं कृणोतु।
प॒र्णं वन॒स्पते॑रिवा॒भिनः॑ शीयताꣳ र॒यिः स च॑ तान्नः॒ शची॒पतिः॑।\\
{\scriptsize (SOUTH)}\\
चं छं जं झं ञं। यरलवशषसहोम्।ॐ नमो भगवते॑ रुद्रा॒य। दक्षिणाङ्गरुद्राय नमः।
\medskip
\twolineshloka
{शिवं शान्तं जगन्नाथं लोकानुग्रहकारणम्}
{शिवमेकं परं वन्दे शिकाराय नमो नमः}

ॐ। निध॑नपतये॒ नमः। निध॑नपतान्तिकाय॒ नमः। ऊर्ध्वाय॒ नमः। ऊर्ध्वलिङ्गाय॒ नमः। हिरण्याय॒ नमः। हिरण्यलिङ्गाय॒ नमः। सुवर्णाय॒ नमः। सुवर्णलिङ्गाय॒ नमः। दिव्याय॒ नमः। दिव्यलिङ्गाय॒ नमः। भवाय॒ नमः। भवलिङ्गाय॒ नमः। शर्वाय॒ नमः। शर्वलिङ्गाय॒ नमः। शिवाय॒ नमः। शिवलिङ्गाय॒ नमः। ज्वलाय॒ नमः। ज्वललिङ्गाय॒ नमः। आत्माय॒ नमः। आत्मलिङ्गाय॒ नमः। परमाय॒ नमः। परमलिङ्गाय॒ नमः। एतथ्सोमस्य॑ सूर्य॒स्य॒ सर्वलिङ्गꣴ॑ स्थाप॒य॒ति॒ पाणिमन्त्रं॑ पवि॒त्रम्।\\
{\scriptsize (WEST)}\\
टं ठं डं ढं णं। यरलवशषसहोम्।ॐ नमो भगवते॑ रुद्रा॒य। पश्चिमाङ्गरुद्राय नमः।
\medskip
\twolineshloka
{वाहनं वृषभो यस्य वासुकिः कण्ठभूषणम्}
{वामे शक्तिधरं वन्दे वकाराय नमो नमः}

यो रु॒द्रो अ॒ग्नौ यो अ॒फ्सु य ओष॑धीषु॒ यो रु॒द्रो विश्वा॒ भुव॑नाऽऽवि॒वेश॒ तस्मै॑ रु॒द्राय॒ नमो॑ अस्तु॥ \\
{\scriptsize (NORTH)}\\
तं थं दं धं नं। यरलवशषसहोम्।ॐ नमो भगवते॑ रुद्रा॒य। उत्तराङ्गरुद्राय नमः।
\medskip
\twolineshloka
{यत्र कुत्र स्थितं देवं सर्वव्यापिनमीश्वरम्}
{यल्लिङ्गं पूजयेन्नित्यं यकाराय नमो नमः}
प्राणानां ग्रन्थिरसि रुद्रो मा॑ विशा॒न्तकः। तेनान्नेना᳚प्याय॒स्व॥ नमो रुद्राय विष्णवे मृत्यु॑र्मे पा॒हि।\\
{\scriptsize (UPWARDS)}\\
पं फं बं भं मं। यरलवशषसहोम्।ॐ नमो भगवते॑ रुद्रा॒य। ऊर्ध्वाङ्गरुद्राय नमः।

{\small \closesection}

\sect{पञ्चाङ्गमुखन्यासः रावणोक्ता पञ्चमुखप्रार्थना-सहितम्}
\begin{center}
तत्पुरु॑षाय वि॒द्महे॑ महादे॒वाय॑ धीमहि।\\
तन्नो॑ रुद्रः प्रचो॒दया᳚त्॥

\fourlineindentedshloka*
{संवर्ताग्नि-तटित्प्रदीप्त-कनक-प्रस्पर्धि-तेजोरुणम्}
{गम्भीरध्वनि-सामवेदजनकं ताम्राधरं सुन्दरम्}
{अर्धेन्दुद्युति-लोल-पिङ्गल-जटाभार-प्रबोद्धोदकम्}
{वन्दे सिद्धसुरासुरेन्द्रनमितं पूर्वं मुखं शूलिनः}

ॐ नमो भगवते॑ रुद्रा॒य। पूर्वाङ्गमुखाय नमः। {\scriptsize (EAST)}\\[1em]

\twolineshloka*
{अ॒घोरे᳚भ्योऽथ॒ घोरे᳚भ्यो॒ घोर॒घोर॑तरेभ्यः}
{सर्वे᳚भ्यः सर्व॒शर्वे᳚भ्यो॒ नम॑स्ते अस्तु रु॒द्ररू॑पेभ्यः}

\fourlineindentedshloka*
{कालाभ्रभ्रमराञ्जन-द्युतिनिभं व्यावृत्तपिङ्गेक्षणम्}
{कर्णोद्भासित-भोगिमस्तकमणि-प्रोद्भिन्नदंष्ट्राङ्कुरम्}
{सर्पप्रोतकपाल-शुक्तिशकल-व्याकीर्णताशेखरम्}
{वन्दे दक्षिणमीश्वरस्य वदनं चाथर्वनादोदयम्}

ॐ नमो भगवते॑ रुद्रा॒य। दक्षिणाङ्गमुखाय नमः। {\scriptsize (SOUTH)}\\[1em]

स॒द्योजा॒तं प्र॑पद्या॒मि॒ स॒द्योजा॒ताय॒ वै नमो॒ नमः॑।\\
भ॒वे भ॑वे॒ नाति॑ भवे भवस्व॒ माम्। भ॒वोद्भ॑वाय॒ नमः॑॥ 

\fourlineindentedshloka*
{प्रालेयाचलमिन्दुकुन्द-धवलं गोक्षीरफेनप्रभम्}
{भस्माभ्यङ्गमनङ्गदेहदहन-ज्वालावली-लोचनम्}
{विष्णु-ब्रह्म-मरुद्गणार्चितपदं ऋग्वेदनादोदयम्}
{वन्देऽहं सकलं कलङ्करहितं स्थाणोर्मुखं पश्चिमम्}

ॐ नमो भगवते॑ रुद्रा॒य। पश्चिमाङ्गमुखाय नमः। {\scriptsize (WEST)}\\[1em]

वा॒म॒दे॒वाय॒ नमो᳚ ज्ये॒ष्ठाय॒ नमः॑ श्रे॒ष्ठाय॒ नमो॑ रु॒द्राय॒ नमः॒ काला॑य॒\\ नमः॒ कल॑विकरणाय॒ नमो॒ बल॑विकरणाय॒ नमो॒ बला॑य॒ नमो॒ बल॑प्रमथनाय॒ नमः॒ सर्व॑भूतदमनाय॒ नमो॑ म॒नोन्म॑नाय॒ नमः॑॥

\fourlineindentedshloka*
{गौरं कुङ्कुमपङ्कितं सुतिलकं व्यापाण्डुमण्डस्थलम्}
{भ्रूविक्षेप-कटाक्षवीक्षण-लसत्-संसक्तकर्णोत्पलम्}
{स्निग्धं बिम्बफलाधरं प्रहसितं नीलालकालङ्कृतम्}
{वन्दे याजुषवेदघोषजनकं वक्त्रं हरस्योत्तरम्}

ॐ नमो भगवते॑ रुद्रा॒य। उत्तराङ्गमुखाय नमः। {\scriptsize (NORTH)}\\[1em]

ईशानः सर्व॑विद्या॒ना॒मीश्वरः सर्व॑भूता॒नां॒ ब्रह्माधि॑पति॒र्ब्रह्म॒णोऽधि॑पति॒र्ब्रह्मा॑ शि॒वो मे॑ अस्तु सदाशि॒वोम्॥

\fourlineindentedshloka*
{व्यक्ताव्यक्तनिरूपितं च परमं षट्त्रिंशतत्त्वाधिकम्}
{तस्मादुत्तर-तत्त्वमक्षरमिति ध्येयं सदा योगिभिः}
{ओङ्कारादि समस्तमन्त्रजनकं सूक्ष्मातिसूक्ष्मं परम्}
{वन्दे पञ्चममीश्वरस्य वदनं खव्यापि तेजोमयम्}

ॐ नमो भगवते॑ रुद्रा॒य। ऊर्ध्वाङ्गमुखाय नमः। {\scriptsize (UPWARDS)}\\[1em]

\end{center}

%{\small \closesection}

\sect{केशादिपादान्त (प्रथमो) न्यासः}

 या ते॑ रुद्र शि॒वा त॒नूरघो॒राऽपा॑पकाशिनी।\\
 तया॑ नस्त॒नुवा॒ शन्त॑मया॒ गिरि॑शन्ता॒भिचा॑कशीहि॥\\ शिखायै नमः॥ {\scriptsize (TUFT)}

अ॒स्मिन् म॑ह॒त्य॑र्ण॒वे᳚ऽन्तरि॑क्षे भ॒वा अधि॑।\\
 तेषाꣳ॑ सहस्रयोज॒नेऽव॒धन्वा॑नि तन्मसि॥ \\
शिरसे नमः॥ {\scriptsize (TOP OF HEAD)}

स॒हस्रा॑णि सहस्र॒शो ये रु॒द्रा अधि॒ भूम्या᳚म्।\\
 तेषाꣳ॑ सहस्रयोज॒नेऽव॒धन्वा॑नि तन्मसि॥\\
ललाटाय नमः॥ {\scriptsize (FOREHEAD)}

ह॒ꣳ॒सः शु॑चि॒षद्वसु॑रन्तरिक्ष॒सद्धोता॑ वेदि॒षदति॑थिर्दुरोण॒सत्।\\
नृ॒षद्व॑र॒सदृ॑त॒सद्व्यो॑म॒सद॒ब्जा गो॒जा ऋ॑त॒जा अ॑द्रि॒जा ऋ॒तं बृ॒हत्॥\\
भ्रुवोर्मध्याय नमः॥ {\scriptsize (MIDDLE OF EYEBROWS)}


त्र्य॑म्बकं यजामहे सुग॒न्धिं पु॑ष्टि॒वर्ध॑नम्।\\
 उ॒र्वा॒रु॒कमि॑व॒ बन्ध॑नान्मृ॒त्योर्मु॑क्षीय॒ माऽमृता᳚त्॥\\
नेत्राभ्यां नमः॥ {\scriptsize (EYES)}

नमः॒ स्रुत्या॑य च॒ पथ्या॑य च॒ नमः॑ का॒ट्या॑य च नी॒प्या॑य च॒ %\\
 नमः॒ सूद्या॑य च सर॒स्या॑य च॒ नमो॑ ना॒द्याय॑ च वैश॒न्ताय॑ च।\\
कर्णाभ्यां नमः॥ {\scriptsize (EARS)}

मा न॑स्तो॒के तन॑ये॒ मा न॒ आयु॑षि॒ मा नो॒ गोषु॒ मा नो॒ अश्वे॑षु रीरिषः।\\
 वी॒रान्मा नो॑ रुद्र भामि॒तोऽव॑धीर्‌ह॒विष्म॑न्तो॒ नम॑सा विधेम ते॥ \\
नासिकायै\footnote{नासिकाभ्यां} नमः॥ {\scriptsize (NOSE)}

अ॒व॒तत्य॒ धनु॒स्त्वꣳ सह॑स्राक्ष॒ शते॑षुधे॥\\
 नि॒शीर्य॑ श॒ल्यानां॒ मुखा॑ शि॒वो नः॑ सु॒मना॑ भव।\\
मुखाय नमः॥ {\scriptsize (FACE)}

 नील॑ग्रीवाः शिति॒कण्ठाः᳚ श॒र्वा अ॒धः, क्ष॑माच॒राः।\\
 तेषाꣳ॑ सहस्रयोज॒नेऽव॒धन्वा॑नि तन्मसि॥\\
कण्ठाय नमः॥ {\scriptsize (NECK)}

नील॑ग्रीवाः शिति॒कण्ठा॒ दिवꣳ॑ रु॒द्रा उप॑श्रिताः।\\
 तेषाꣳ॑ सहस्रयोज॒नेऽव॒धन्वा॑नि तन्मसि॥\\
उपकण्ठाय नमः॥ {\scriptsize (LOWER NECK)}

नम॑स्ते अ॒स्त्वायु॑धा॒याना॑तताय धृ॒ष्णवे᳚।\\
 उ॒भाभ्या॑मु॒त ते॒ नमो॑ बा॒हुभ्यां॒ तव॒ धन्व॑ने॥\\
बाहुभ्यां नमः॥ {\scriptsize (SHOULDERS)}


 या ते॑ हे॒तिर्मी॑ढुष्टम॒ हस्ते॑ ब॒भूव॑ ते॒ धनुः॑।\\
 तया॒ऽस्मान् वि॒श्वत॒स्त्वम॑य॒क्ष्मया॒ परि॑ब्भुज॥\\
उपबाहुभ्यां नमः॥ {\scriptsize (ELBOW TO WRIST)}

परि॑ णो रु॒द्रस्य॑ हे॒तिर्वृ॑णक्तु॒ परि॑ त्वे॒षस्य॑ दुर्म॒तिर॑घा॒योः।\\
 अव॑ स्थि॒रा म॒घव॑द्भ्यस्तनुष्व॒ मीढ्व॑स्तो॒काय॒ तन॑याय मृडय॥\\
मणिबन्धाभ्यां नमः॥ {\scriptsize (WRISTS)}

 ये ती॒र्थानि॑ प्र॒चर॑न्ति सृ॒काव॑न्तो निष॒ङ्गिणः॑।\\
 तेषाꣳ॑ सहस्रयोज॒नेऽव॒धन्वा॑नि तन्मसि॥\\
हस्ताभ्यां नमः॥ {\scriptsize (HANDS)}


स॒द्योजा॒तं प्र॑पद्या॒मि॒ स॒द्योजा॒ताय॒ वै नमो॒ नमः॑।\\
भ॒वे भ॑वे॒ नाति॑ भवे भवस्व॒ माम्। भ॒वोद्भ॑वाय॒ नमः॑॥ \\
अङ्गुष्ठाभ्यां नमः॥ {\scriptsize (ROLL RING FINGERS ON THUMBS)}

वा॒म॒दे॒वाय॒ नमो᳚ ज्ये॒ष्ठाय॒ नमः॑ श्रे॒ष्ठाय॒ नमो॑ रु॒द्राय॒ नमः॒ काला॑य॒ नमः॒ कल॑विकरणाय॒ नमो॒ बल॑विकरणाय॒ नमो॒ बला॑य॒ नमो॒ बल॑प्रमथनाय॒ नमः॒ सर्व॑भूतदमनाय॒ नमो॑ म॒नोन्म॑नाय॒ नमः॑॥\\
तर्जनीभ्यां नमः॥ {\scriptsize (ROLL THUMBS ON INDEX FINGERS)}

अ॒घोरे᳚भ्योऽथ॒ घोरे᳚भ्यो॒ घोर॒घोर॑तरेभ्यः।\\
सर्वे᳚भ्यः सर्व॒शर्वे᳚भ्यो॒ नम॑स्ते अस्तु रु॒द्ररू॑पेभ्यः॥\\
मध्यमाभ्यां नमः॥ {\scriptsize (ROLL THUMBS ON MIDDLE FINGERS)}

तत्पुरु॑षाय वि॒द्महे॑ महादे॒वाय॑ धीमहि।\\
तन्नो॑ रुद्रः प्रचो॒दया᳚त्॥\\
अनामिकाभ्यां नमः॥ {\scriptsize (ROLL THUMBS ON RING FINGERS)}

ईशानः सर्व॑विद्या॒ना॒मीश्वरः सर्व॑भूता॒नां॒ ब्रह्माधि॑पति॒र्ब्रह्म॒णो\-ऽधि॑पति॒र्ब्रह्मा॑ शि॒वो मे॑ अस्तु सदाशि॒वोम्॥\\
कनिष्ठिकाभ्यां नमः॥ {\scriptsize (ROLL THUMBS ON LITTLE FINGERS)}

नमो हिरण्यबाहवे हिरण्यवर्णाय हिरण्यरूपाय हिरण्यपतये\-ऽम्बिकापतय उमापतये पशुपतये॑ नमो॒ नमः॥\\
करतलकरपृष्ठाभ्यां नमः॥{\scriptsize (RUB PALMS OVER ONE ANOTHER, FRONT AND BACK)}

नमो॑ वः किरि॒केभ्यो॑ दे॒वाना॒ꣳ॒ हृद॑येभ्यः॥\\
हृदयाय नमः॥ {\scriptsize (HEART)}

नमो॑ ग॒णेभ्यो॑ ग॒णप॑तिभ्यश्च वो॒ नमः॑॥\\
पृष्ठाय नमः॥ {\scriptsize (BACK)}

नम॒स्तक्ष॑भ्यो रथका॒रेभ्य॑श्च वो॒ नमः॑॥\\
कक्षाभ्यां नमः॥ {\scriptsize (ARMPIT TO WAIST)}

नमो॒ हिर॑ण्यबाहवे सेना॒न्ये॑ दि॒शां च॒ पत॑ये॒ नमः॑॥\\
पार्श्वाभ्यां नमः॥ {\scriptsize (TRUNK)}

 विज्यं॒ धनुः॑ कप॒र्दिनो॒ विश॑ल्यो॒ बाण॑वाꣳ उ॒त।\\
 अने॑शन्न॒स्येष॑व आ॒भुर॑स्य निष॒ङ्गथिः॑॥\\
जठराय नमः॥ {\scriptsize (STOMACH)}


हि॒र॒ण्य॒ग॒र्भः सम॑वर्त॒ताग्रे॑ भू॒तस्य॑ जा॒तः पति॒रेक॑ आसीत्।\\
सदा॑धार पृथि॒वीं द्यामु॒तेमां कस्मै॑ दे॒वाय॑ ह॒विषा॑ विधेम॥\\
नाभ्यै नमः॥ {\scriptsize (NAVEL)}

मीढु॑ष्टम॒ शिव॑तम शि॒वो नः॑ सु॒मना॑ भव।\\
 प॒र॒मे वृ॒क्ष आयु॑धं नि॒धाय॒ कृत्तिं॒ वसा॑न॒ आ च॑र॒ पिना॑कं॒ बिभ्र॒दा ग॑हि॥\\
कट्यै नमः॥ {\scriptsize (WAIST)}

 ये भू॒ताना॒मधि॑पतयो विशि॒खासः॑ कप॒र्दिनः॑। \\
तेषाꣳ॑ सहस्रयोज॒नेऽव॒धन्वा॑नि तन्मसि॥ \\
गुह्याय नमः॥ {\scriptsize (UPPER REPRODUCTIVE ORGANS)}

ये अन्ने॑षु वि॒विध्य॑न्ति॒ पात्रे॑षु॒ पिब॑तो॒ जना\sn{}।\\
तेषाꣳ॑ सहस्रयोज॒नेऽव॒धन्वा॑नि तन्मसि॥ \\
अण्डाभ्यां नमः॥ {\scriptsize (LOWER REPRODUCTIVE ORGANS)}

स शि॒रा जा॒तवे॑दा अ॒क्षरं॑ पर॒मं प॒दम्।\\
वेदा॑ना॒ꣳ॒ शिर॑सि मा॒ता॒ आ॒युष्मन्तं॑ करोतु॒ माम्॥\\
अपानाय नमः॥ {\scriptsize (ANUS)}

मा नो॑ म॒हान्त॑मु॒त मा नो॑ अर्भ॒कं मा न॒ उक्ष॑न्तमु॒त मा न॑ उक्षि॒तम्।\\
 मा नो॑ वधीः पि॒तरं॒ मोत मा॒तरं॑ प्रि॒या मा न॑स्त॒नुवो॑ रुद्र रीरिषः॥\\
ऊरुभ्यां नमः॥ {\scriptsize (THIGHS)}

ए॒ष ते॑ रुद्रभा॒गस्तं जु॑षस्व॒ तेना॑व॒सेन॑ प॒रो मूज॑व॒तोऽती॒\-ह्यव॑ततधन्वा॒ पिना॑कहस्तः॒ कृत्ति॑वासाः॥\\
जानुभ्यां नमः॥ {\scriptsize (KNEES)}

स॒ꣳ॒सृ॒ष्ट॒जिथ्सो॑म॒पा बा॑हुश॒ध्यू᳚र्ध्वध॑न्वा॒ प्रति॑हिताभि॒रस्ता᳚।\\
बृह॑स्पते॒ परि॑दीया॒ रथे॑न रक्षो॒हाऽमित्राꣳ॑ अप॒बाध॑मानः॥\\
जङ्घाभ्यां नमः॥ {\scriptsize (KNEE TO ANKLES)}

विश्वं॑ भू॒तं भुव॑नं चि॒त्रं ब॑हु॒धा जा॒तं जाय॑मानं च॒ यत्।\\
सर्वो॒ ह्ये॑ष रु॒द्रस्तस्मै॑ रु॒द्राय॒ नमो॑ अस्तु॥\\
गुल्फाभ्यां नमः॥ {\scriptsize (ANKLES)}

ये प॒थां प॑थि॒रक्ष॑य ऐलबृ॒दा य॒व्युधः॑।\\
तेषाꣳ॑ सहस्रयोज॒नेऽव॒धन्वा॑नि तन्मसि॥ \\
पादाभ्यां नमः॥ {\scriptsize (FEET)}


अध्य॑वोचदधिव॒क्ता प्र॑थ॒मो दैव्यो॑ भि॒षक्।\\
अहीꣴ॑श्च॒ सर्वा᳚ञ्ज॒म्भय॒-न्थ्सर्वा᳚श्च यातुधा॒न्यः॑॥\\
कवचाय हुम्॥ {\scriptsize (CROSS HANDS ACROSS CHEST WITH TIPS OF FINGERS TOUCHING SHOULDERS)}


नमो॑ बि॒ल्मिने॑ च कव॒चिने॑ च॒ नमः॑ श्रु॒ताय॑ च श्रुतसे॒नाय॑ च॥\\
उपकवचाय हुम्॥ {\scriptsize (REPEAT THE ABOVE AT ELBOW LEVEL)}

नमो॑ अस्तु॒ नील॑ग्रीवाय सहस्रा॒क्षाय॑ मी॒ढुषे᳚। \\
अथो॒ ये अ॑स्य॒ सत्वा॑नो॒ऽहं तेभ्यो॑ऽकरं॒ नमः॑॥ \\
नेत्रत्रयाय वौषट्॥ {\scriptsize (TOUCH INDEX, MIDDLE, RING FINGERSACROSS THE THREE EYES)}


प्र मु॑ञ्च॒ धन्व॑न॒स्त्वमु॒भयो॒रार्त्नि॑यो॒र्ज्याम्।\\
याश्च॑ ते॒ हस्त॒ इष॑वः॒ परा॒ ता भ॑गवो वप॥\\ 
अस्त्राय फट्॥ {\scriptsize (SLAP INDEXAND MIDDLE FINGERS OF RIGHT HAND ON LEFT PALM)}

य ए॒ताव॑न्तश्च॒ भूयाꣳ॑सश्च॒ दिशो॑ रु॒द्रा वि॑तस्थि॒रे।\\
 तेषाꣳ॑ सहस्रयोज॒नेऽव॒धन्वा॑नि तन्मसि॥ \\
इति दिग्बन्धः॥ {\scriptsize (SNAP MIDDLE AND THUMB WITH CLICKING SOUNDS AROUND SELF)}

{\small \closesection}

\sect{मूर्धादिपादान्त दशाक्षरी दशाङ्ग (द्वितीयो) न्यासः}
ॐ मूर्ध्ने नमः। नं नासिकाय\footnote{नासिकाभ्यां} नमः। मों ललाटाय नमः। भं मुखाय नमः। गं कण्ठाय नमः। वं हृदयाय नमः। तें दक्षिणहस्ताय नमः। रुं वामहस्ताय नमः। द्रां नाभ्यै नमः। यं पादाभ्यां नमः।

\sect{पादादिमूर्धान्त पञ्चाङ्ग (तृतीयो) न्यासः}
स॒द्योजा॒तं प्र॑पद्या॒मि॒ स॒द्योजा॒ताय॒ वै नमो॒ नमः॑।\\
भ॒वे भ॑वे॒ नाति॑ भवे भवस्व॒ माम्। भ॒वोद्भ॑वाय॒ नमः॑॥ \\
पादाभ्यां नमः॥

वा॒म॒दे॒वाय॒ नमो᳚ ज्ये॒ष्ठाय॒ नमः॑ श्रे॒ष्ठाय॒ नमो॑ रु॒द्राय॒ नमः॒ काला॑य॒ नमः॒ कल॑विकरणाय॒ नमो॒ बल॑विकरणाय॒ नमो॒ बला॑य॒ नमो॒ बल॑प्रमथनाय॒ नमः॒ सर्व॑भूतदमनाय॒ नमो॑ म॒नोन्म॑नाय॒ नमः॑॥\\
ऊरुभ्यां नमः॥ 

अ॒घोरे᳚भ्योऽथ॒ घोरे᳚भ्यो॒ घोर॒घोर॑तरेभ्यः।\\
सर्वे᳚भ्यः सर्व॒शर्वे᳚भ्यो॒ नम॑स्ते अस्तु रु॒द्ररू॑पेभ्यः॥\\
हृदयाय नमः॥ 

तत्पुरु॑षाय वि॒द्महे॑ महादे॒वाय॑ धीमहि।\\
तन्नो॑ रुद्रः प्रचो॒दया᳚त्॥\\
मुखाय नमः॥ 

ईशानः सर्व॑विद्या॒ना॒मीश्वरः सर्व॑भूता॒नां॒ ब्रह्माधि॑पति॒र्ब्रह्म॒णो\-ऽधि॑पति॒र्ब्रह्मा॑ शि॒वो मे॑ अस्तु सदाशि॒वोम्॥\\
हंस हंस मूर्ध्ने नमः॥ \\
{\small \closesection}

\sect{हंसगायत्री}
अस्य श्री हंसगायत्री महामन्त्रस्य। अव्यक्त परब्रह्म ऋषिः। अव्यक्त गायत्री छन्दः। परमहंसो देवता॥ हंसां बीजम्। हंसीं शक्तिः। हंसूं कीलकम्॥\\
परमहंस-प्रसाद-सिद्ध्यर्थे जपे विनियोगः॥

हंसां अङ्गुष्ठाभ्यां नमः।
हंसीं तर्जनीभ्यां नमः।
हंसूं मध्यमाभ्यां नमः।
हंसैं अनामिकाभ्यां नमः।
हंसौं कनिष्ठिकाभ्यां नमः।
हंसः करतलकरपृष्ठाभ्यां नमः।
हंसां हृदयाय नमः।
हंसीं शिरसे स्वाहा।
हंसूं शिखायै वषट्।
हंसैं कवचाय हुम्।
हंसौं नेत्रत्रयाय वौषट्।
हंसः अस्त्राय फट्।
भूर्भुवस्सुवरोम् इति दिग्बन्धः॥

\centerline{\textbf{॥ध्यानम्॥}}

\twolineshloka*
{गमागमस्थं गमनादिशून्यं चिद्रूपदीपं तिमिरापहारम्}
{पश्यामि ते सर्वजनान्तरस्थं नमामि हंसं परमात्मरूपम्}
\smallskip
\centerline{हंसहंसात् परमहंसः सोऽहं हंसः॥}
\smallskip
\centerline{हं॒स॒ हं॒साय॑ वि॒द्महे॑ परमहं॒साय॑ धीमहि।}
\centerline{तन्नो॑ हंसः प्रचो॒दया᳚त्॥ } 
\begin{flushright}
\vspace{-5.7ex}{\small (एवं त्रिः)}
\end{flushright}

\twolineshloka*
{हंस हंसेति यो ब्रूयाद्धंसो नाम सदाशिवः}
{एवं न्यासविधिं कृत्वा ततः सम्पुटमारभेत्}

{\small \closesection}

\sect{दिक् सम्पुटन्यासः}
\newcounter{dik}
\newcommand{\indradi}[9]{\refstepcounter{dik}
ॐ भूर्भुवः॒ सुव॒रोम्।\\
\lbrack#1\rbrack\  #2। #5।\\#6॥\\%
#2 भूर्भुवः॒ सुवः॑। #4 #7 साङ्गाय सायुधाय सशक्तिपरिवाराय सर्वालङ्कार\-भूषिताय उमामहेश्वर\-पार्षदाय नमः।\\%
#8दिग्भागे #9 #2 #4 नमः। #3 सुप्रीतो वरदो भवतु॥ \lbrack#3 संरक्षतु॥\rbrack\hfill॥\devanumber{\arabic{dik}}॥}

\indradi{ॐ}{लं}{इन्द्रः}{इन्द्राय}%
{त्रा॒तार॒मिन्द्र॑मवि॒तार॒मिन्द्र॒ꣳ॒ हवे॑हवे सु॒हव॒ꣳ॒ शूर॒मिन्द्रम्᳚}%
{हु॒वे नु श॒क्रं पु॑रुहू॒तमिन्द्रꣴ॑ स्व॒स्ति नो॑ म॒घवा॑ धा॒त्विन्द्रः॑}%
{वज्रहस्ताय सुराधिपतय ऐरावतवाहनाय}{पूर्व}{ललाटस्थाने}

\indradi{नं}{रं}{अग्निः}{अग्नये}%
{त्वन्नो॑ अग्ने॒ वरु॑णस्य वि॒द्वान् दे॒वस्य॒ हेडोऽव॑ यासिसीष्ठाः}%
{यजि॑ष्ठो॒ वह्नि॑तमः॒ शोशु॑चानो॒ विश्वा॒ द्वेषाꣳ॑सि॒ प्रमु॑मुग्ध्य॒स्मत्}%
{शक्तिहस्ताय तेजोऽधिपतये\-ऽजवाहनाय}{आग्नेय}{नेत्रयोः स्थाने}

\indradi{मों}{हं}{यमः}{यमाय}%
{सु॒गं नः॒ पन्था॒मभ॑यं कृणोतु। यस्मि॒न्नक्ष॑त्रे य॒म एति॒ राजा᳚}%
{यस्मि॑न्नेनम॒भ्यषि॑ञ्चन्त दे॒वाः। तद॑स्य चि॒त्रꣳ ह॒विषा॑ यजाम}%
{दण्डहस्ताय धर्माधिपतये महिषवाहनाय}{दक्षिण}{कर्णयोः स्थाने}

\indradi{भं}{षं}{निर्‌ऋतिः}{निर्‌ऋतये}%
{असु॑न्वन्त॒मय॑जमानमिच्छ स्ते॒नस्ये॒त्यां तस्क॑र॒स्यान्वे॑षि}%
{अ॒न्यम॒स्मदि॑च्छ॒ सा त॑ इ॒त्या नमो॑ देवि निर्‌ऋते॒ तुभ्य॑मस्तु}%
{खड्गहस्ताय रक्षोधिपतये नरवाहनाय}{निर्‌ऋति}{मुखस्थाने}

\indradi{गं}{वं}{वरुणः}{वरुणाय}%
{तत्त्वा॑ यामि॒ ब्रह्म॑णा॒ वन्द॑मान॒स्तदा शा᳚स्ते॒ यज॑मानो ह॒विर्भिः॑}%
{अहे॑डमानो वरुणे॒ह बो॒ध्युरु॑शꣳस॒ मा न॒ आयुः॒ प्रमो॑षीः}%
{पाशहस्ताय जलाधिपतये मकरवाहनाय}{पश्चिम}{बाह्वोः स्थाने}

\indradi{वं}{यं}{वायुः}{वायवे}%
{आ नो॑ नि॒युद्भिः॑ श॒तिनी॑भिरध्व॒रम्। स॒ह॒स्रिणी॑भि॒रुप॑\-याहि य॒ज्ञम्}%
{वायो॑ अ॒स्मिन् ह॒विषि॑ मादयस्व। यू॒यं पा॑त स्व॒स्तिभिः॒ सदा॑ नः}%
{साङ्कुशध्वजहस्ताय प्राणाधिपतये मृगवाहनाय}{वायव्य}{नासिकास्थाने\footnote{नासिकयोः स्थाने}}

\indradi{तें}{सं}{सोमः}{सोमाय}%
{व॒यꣳ सो॑म व्र॒ते तव॑। मन॑स्त॒नूषु॒ बिभ्र॑तः}%
{प्र॒जाव॑न्तो अशीमहि}%। इ॒न्द्रा॒णी दे॒वी सु॒भगा॑ सु॒पत्नी᳚}%
{अमृतकलशहस्ताय नक्षत्राधिपतये अश्ववाहनाय}{उत्तर}{हृदयस्थाने}

\indradi{रुं}{शं}{ईशानः}{ईशानाय}%
{(ऋक्) तमीशा᳚नं॒ जग॑तस्त॒स्थुष॒स्पतिम्᳚। धि॒यं॒ जि॒न्वमव॑से हूमहे व॒यम्}%
{पू॒षा नो॒ यथा॒ वेद॑सा॒मस॑द्वृ॒धे र॑क्षि॒ता पा॒युरद॑ब्धः स्व॒स्तये᳚}
{त्रिशूलहस्ताय भूताधिपतये वृषभवाहनाय}{ईशान}{नाभिस्थाने}

\indradi{द्रां}{खं}{ब्रह्मा}{ब्रह्मणे}%
{(ऋक्) अ॒स्मे रु॒द्रा मे॒हना॒ पर्व॑तासो वृत्र॒हत्ये॒ भर॑हूतौ स॒जोषाः᳚ }%
{यः शंस॑ते स्तुव॒ते धायि॑ प॒ज्र इन्द्र॑ज्येष्ठा अ॒स्माँ अ॑वन्तु दे॒वाः}%
{पद्महस्ताय विद्याधिपतये हंसवाहनाय}{ऊर्ध्व}{मूर्ध्निस्थाने}


\indradi{यं}{ह्रीं}{विष्णुः}{विष्णवे}%
{स्यो॒ना पृ॑थिवि॒ भवा॑ऽनृक्ष॒रा नि॒वेश॑नी}%
{यच्छा॑नः॒ शर्म॑ स॒प्रथाः᳚}%
{चक्रहस्ताय नागाधिपतये गरुडवाहनाय}{अधो}{पादयोः स्थाने}

{\small \closesection}
%\newpage
\sect{षोडशाङ्गरौद्रीकरणम्}
\newcounter{rau}
\newcommand{\raudri}[3]{\refstepcounter{rau}%
ॐ भूर्भुवः॒ सुवः॑। ॐ #1। नमः॑ श॒म्भवे॑ च मयो॒भवे॑ च॒ नमः॑ शङ्क॒राय॑ च मयस्क॒राय॑ च॒ नमः॑ शि॒वाय॑ च शि॒वत॑राय च॥\\
#2रौद्रे॒णानी॑केन पा॒हि मा᳚ऽग्ने पिपृ॒हि मा॒ मा मा॑ हिꣳसीः॥
#1 [ॐ] भूर्भुवः॒ सुव॒रोम्। #3स्थाने रुद्राय नमः॥\devanumber{\arabic{rau}}॥\\[1ex]}
\raudri{अं}{वि॒भूर॑सि प्र॒वाह॑णो॒ }{शिखा}
\raudri{आं}{वह्नि॑रसि हव्य॒वाह॑नो॒ }{शिर}
\raudri{इं}{श्वा॒त्रो॑ऽसि॒ प्रचे॑ता॒ }{मूर्ध्नि}
\raudri{ईं}{तु॒थो॑ऽसि वि॒श्ववे॑दा॒ }{ललाट}
\raudri{उं}{उ॒शिग॑सि क॒वी }{नेत्रयोः\footnote{भ्रू} }
\raudri{ऊं}{अङ्घा॑रिरसि॒ बम्भा॑री॒ }{कर्णयोः }
\raudri{ऋं}{अ॒व॒स्युर॑सि॒ दुव॑स्वा॒न् }{मुख}
\raudri{ॠं}{शु॒न्ध्यूर॑सि मार्जा॒लीयो॒ }{कण्ठ}
\raudri{ऌं}{स॒म्राड॑सि कृ॒शानू॒ }{बाह्वोः }
\raudri{ॡं}{प॒रि॒षद्यो॑ऽसि॒ पव॑मानो॒ }{हृदय}
\raudri{एं}{प्र॒तक्वा॑ऽसि॒ नभ॑स्वा॒न् }{नाभि}
\raudri{ऐं}{अस॑म्मृष्टोऽसि हव्य॒सूदो॒ }{कटि}
\raudri{ॐ}{ऋ॒तधा॑माऽसि॒ सुव॑र्ज्योती॒ }{ऊरु}
\raudri{औं}{ब्रह्म॑ज्योतिरसि॒ सुव॑र्धामा॒ }{जानु}
\raudri{अं}{अ॒जो᳚ऽस्येक॑पा॒द्}{जङ्घा}
\raudri{अः}{अहि॑रसि बु॒ध्नियो॒ }{पादयोः }
त्वगस्थिगतैः सर्वपापैः प्रमुच्यते। सर्वभूतेष्वपराजितो भवति।\\
ततो भूत-प्रेत-पिशाच-ब्रह्मराक्षस-यक्ष-यमदूत-शाकिनी-डाकिनी-सर्प-श्वापद-वृश्चिक-तस्कराद्युपद्रवाद्युपघाताः। सर्वे ज्वलन्तं पश्यन्तु। मां रक्षन्तु। यजमानं रक्षन्तु। सर्वान् महाजनान् रक्षन्तु॥\\
{\small\closesection}

%\vfill
%\newpage
\sect{गुह्यादि मस्तकान्तं षडङ्ग (चतुर्थो) न्यासः}

मनो॒ ज्योति॑र्जुषता॒माज्यं॒ विच्छि॑न्नं य॒ज्ञꣳ समि॒मं द॑धातु।\\ या इ॒ष्टा उ॒षसो॑ नि॒म्रुच॑श्च॒ ताः सन्द॑धामि ह॒विषा॑ घृ॒तेन॑॥ गुह्याय नमः॥१॥

अबो᳚ध्य॒ग्निः स॒मिधा॒ जना॑नां॒ प्रति॑ धे॒नुमि॑वाऽऽय॒तीमु॒षासम्᳚।\\
य॒ह्वा इ॑व॒ प्रव॒यामु॒ज्जिहा॑नाः॒ प्रभा॒नवः॑ सिस्रते॒ नाक॒मच्छ॑॥ नाभ्यै नमः॥२॥

अ॒ग्निर्मू॒र्धा दि॒वः क॒कुत्पतिः॑ पृथि॒व्या अ॒यम्।\\
 अ॒पाꣳ रेताꣳ॑सि जिन्वति॥ हृदयाय नमः॥३॥

मू॒र्धानं॑ दि॒वो अ॑र॒तिं पृ॑थि॒व्या वै᳚श्वान॒रमृ॒ताय॑ जा॒तम॒ग्निम्।\\
क॒विꣳ स॒म्राज॒मति॑थिं॒ जना॑नामा॒सन्ना पात्रं॑ जनयन्त दे॒वाः॥ कण्ठाय नमः॥४॥

मर्मा॑णि ते॒ वर्म॑भिश्छादयामि॒ सोम॑स्त्वा॒ राजा॒ऽमृते॑ना॒भि\-व॑स्ताम्।\\
उ॒रोर्वरी॑यो॒ वरि॑वस्ते अस्तु॒ जय॑न्तं॒ त्वामनु॑ मदन्तु दे॒वाः॥ मुखाय नमः॥५॥

जा॒तवे॑दा॒ यदि॑ वा पाव॒कोऽसि॑। वै॒श्वा॒न॒रो यदि॑ वा वैद्यु॒तोऽसि॑।\\
शं प्र॒जाभ्यो॒ यज॑मानाय लो॒कम्। ऊर्जं॒ पुष्टिं॒ दद॑द॒भ्याव॑वृथ्स्व॥ शिरसे नमः॥६॥

{\small \closesection}

\sect{आत्मरक्षा}
\centerline{\normalsize (तैत्तिरीयब्राह्मणे अष्टकं – २/प्रश्नः – ३/अनुवाकः – ११)}

ब्रह्मा᳚त्म॒न्वद॑सृजत। तद॑कामयत। समा॒त्मना॑ पद्ये॒येति॑।\\
आत्म॒न्नात्म॒न्नित्याम॑न्त्रयत। तस्मै॑ दश॒मꣳ हू॒तः प्रत्य॑शृणोत्। स दश॑हूतोऽभवत्। दश॑हूतो ह॒ वै नामै॒षः। तं वा ए॒तं दश॑हूत॒ꣳ॒ सन्तम्᳚। दश॑हो॒तेत्याच॑क्षते प॒रोक्षे॑ण। प॒रोक्ष॑प्रिया इव॒ हि दे॒वाः॥

आत्म॒न्नात्म॒न्नित्याम॑न्त्रयत। तस्मै॑ सप्त॒मꣳ हू॒तः प्रत्य॑शृणोत्। स स॒प्तहू॑तोऽभवत्। स॒प्तहू॑तो ह॒ वै नामै॒षः। तं वा ए॒तꣳ स॒प्तहू॑त॒ꣳ॒ सन्तम्᳚। स॒प्तहो॒तेत्याच॑क्षते प॒रोक्षे॑ण। प॒रोक्ष॑प्रिया इव॒ हि दे॒वाः॥

आत्म॒न्नात्म॒न्नित्याम॑न्त्रयत। तस्मै॑ ष॒ष्ठꣳ हू॒तः प्रत्य॑शृणोत्। स षड्ढू॑तोऽभवत्। षड्ढू॑तो ह॒ वै नामै॒षः। तं वा ए॒तꣳ षड्ढू॑त॒ꣳ॒ सन्तम्᳚। षड्ढो॒तेत्याच॑क्षते प॒रोक्षे॑ण। प॒रोक्ष॑प्रिया इव॒ हि दे॒वाः॥

आत्म॒न्नात्म॒न्नित्याम॑न्त्रयत। तस्मै॑ पञ्च॒मꣳ हू॒तः प्रत्य॑शृणोत्। स पञ्च॑हूतोऽभवत्। पञ्च॑हूतो ह॒ वै नामै॒षः। तं वा ए॒तं पञ्च॑हूत॒ꣳ॒ सन्तम्᳚। पञ्च॑हो॒तेत्याच॑क्षते प॒रोक्षे॑ण। प॒रोक्ष॑प्रिया इव॒ हि दे॒वाः॥

आत्म॒न्नात्म॒न्नित्याम॑न्त्रयत। तस्मै॑ चतु॒र्थꣳ हू॒तः प्रत्य॑शृणोत्। स चतु॑र्‌हूतोऽभवत्। चतु॑र्‌हूतो ह॒ वै नामै॒षः। तं वा ए॒तं चतु॑र्‌हूत॒ꣳ॒ सन्तम्᳚। चतु॑र्हो॒तेत्याच॑क्षते प॒रोक्षे॑ण। प॒रोक्ष॑प्रिया इव॒ हि दे॒वाः॥

तम॑ब्रवीत्। त्वं वै मे॒ नेदि॑ष्ठꣳ हू॒तः प्रत्य॑श्रौषीः। त्वयै॑नानाख्या॒तार॒ इति॑। तस्मा॒न्नु है॑ना॒ꣴ॒श्चतु॑र्‌होतार॒ इत्याच॑क्षते। तस्मा᳚च्छुश्रू॒षुः पु॒त्राणा॒ꣳ॒ हृद्य॑तमः। नेदि॑ष्ठो॒ हृद्य॑तमः। नेदि॑ष्ठो॒ ब्रह्म॑णो भवति। य ए॒वं वेद॑।\\
आत्मने नमः॥

%{\small \closesection}

\newcounter{ssk}
\newcommand{\ssankalpa}[1]{\refstepcounter{ssk} 
#1तन्मे॒ मनः॑ शि॒वस॑ङ्क॒ल्पम॑स्तु॥\devanumber{\arabic{ssk}}॥}
\newcommand{\ssankalpaalign}[2]{
\setcounter{shlokacount}{\value{ssk}}

\twolineshloka{#1}{#2तन्मे॒ मनः॑ शि॒वस॑ङ्क॒ल्पम॑स्तु}
\refstepcounter{ssk}
\pagebreak[0]}

\sect{शिवसङ्कल्पः}
\begin{center}
\ssankalpaalign{येने॒दं भू॒तं भुव॑नं भवि॒ष्यत् परि॑गृहीतम॒मृते॑न॒ सर्वम्᳚}
{येन॑ य॒ज्ञस्त्रा॒यते॑ स॒प्तहो॑ता॒ }

\ssankalpa{येन॒ कर्मा॑णि प्र॒चर॑न्ति॒ धीरा॒ यतो॑ वा॒चा मन॑सा॒ चारु॒ यन्ति॑। यथ्स॒म्मित॒मनु॑सं॒यन्ति॑ प्रा॒णिन॒स्}

\ssankalpa{येन॒ कर्मा᳚ण्य॒पसो॑ मनी॒षिणो॑ य॒ज्ञे कृ॒ण्वन्ति॑ वि॒दथे॑षु॒ धीराः᳚। यद॑पू॒र्वं य॒क्षम॒न्तः प्र॒जानां॒ }

\ssankalpa{यत्प्र॒ज्ञान॑मु॒त चेतो॒ धृति॑श्च॒ यज्ज्योति॑र॒न्तर॒मृतं॑ प्र॒जासु॑। यस्मा॒न्न ऋ॒ते किं च॒ न कर्म॑ क्रि॒यते॒ }

\ssankalpa{सु॒षा॒र॒थिरश्वा॑निव॒ यन्म॑नु॒ष्या᳚न्नेनी॒यते॒ऽभीशु॑भिर्वा॒जिन॑ इव। हृत्प्रति॑ष्ठं॒ यद॑जिरं॒ जवि॑ष्ठं॒ }

\ssankalpa{यस्मि॒नृचः॒ साम॒ यजूꣳ॑षि यस्मि॒न् प्रति॑ष्ठिता रथना॒भावि॑वा॒राः। यस्मिꣴ॑श्चि॒त्तꣳ सर्व॒मोतं॑ प्र॒जानां॒ }

\ssankalpa{यदत्र॑ ष॒ष्ठं त्रि॒शतꣳ॑ सु॒वीरं॑ य॒ज्ञस्य॑ गु॒ह्यं नव॑नाव॒माय्यम्᳚। दश॑ पञ्च त्रि॒ꣳ॒शतं॒ यत्परं॑ च॒ }

\ssankalpa{यज्जाग्र॑तो दू॒रमु॒दैति॒ दैवं॒ तदु॑ सु॒प्तस्य॒ तथै॒वैति॑। दू॒र॒ङ्ग॒मं ज्योति॑षां॒ ज्योति॒रेकं॒ }

\ssankalpa{येने॒दं विश्वं॒ जग॑तो ब॒भूव॒ ये दे॒वापि॑ मह॒तो जा॒तवे॑दाः। तदे॒वाग्निस्तम॑सो॒ ज्योति॒रेकं॒ }

\ssankalpa{येन॒ द्यौः पृ॑थि॒वी चा॒न्तरि॑क्षं च॒ ये पर्व॑ताः प्र॒दिशो॒ दिश॑श्च। येने॒दं जग॒द्व्याप्तं॑ प्र॒जानां॒ }

\ssankalpa{ये म॑नो॒ हृद॑यं॒ ये च॑ दे॒वा ये दि॒व्या आपो॒ ये सू᳚र्यर॒श्मिः। ते श्रोत्रे॒ चक्षु॑षी स॒ञ्चर॑न्तं॒ }

\ssankalpaalign{अचि॑न्त्यं॒ चाप्र॑मेयं॒ च॒ व्य॒क्ता॒व्यक्त॑परं च॒ यत्}
{सूक्ष्मा᳚थ्सूक्ष्मत॑रं ज्ञे॒यं॒ }

\ssankalpa{एका॑ च द॒श श॒तं च॑ स॒हस्रं॑ चा॒युतं॑ च\\ नि॒युतं॑ च प्र॒युतं॒ चार्बु॑दं च॒ न्य॑र्बुदं च समु॒द्रश्च॒ मध्यं॒ चान्त॑श्च परा॒र्धश्च॒ }

\ssankalpa{ये प॑ञ्च प॒ञ्चाद॒श श॒तꣳ स॒हस्र॑म॒युतं॒ न्य॑र्बुदं च। ते अ॑ग्निचि॒त्येष्ट॑का॒स्तꣳ शरी॑रं॒ }

\ssankalpa{वेदा॒हमे॒तं पुरु॑षं म॒हान्त॑मादि॒त्यव॑र्णं॒ तम॑सः॒ पर॑स्तात्। यस्य॒ योनिं॒ परि॒पश्य॑न्ति॒ धीरा॒स्}

\ssankalpa{यस्ये॒दं धीराः᳚ पु॒नन्ति॑ क॒वयो᳚ ब्र॒ह्माण॑मे॒तं त्वा॑ वृणत॒ इन्दु᳚म्। स्था॒व॒रं जङ्ग॑मं॒ द्यौरा॑का॒शं }

\ssankalpaalign{परा᳚त्प॒रत॑रं चै॒व॒ य॒त्परा᳚च्चैव॒ यत्प॑रम्}
{यत्परा᳚त्पर॑तो ज्ञे॒यं॒ }

\ssankalpaalign{परा᳚त्प॒रत॑रो ब्र॒ह्मा॒ त॒त्परा᳚त्पर॒तो ह॑रिः}
{त॒त्परा॑त्पर॑तोऽधी॒श॒स्}

\ssankalpaalign{या वे॑दा॒दिषु॑ गाय॒त्री॒ स॒र्वव्या॑पी म॒हेश्व॑री}
{ऋग्यजुः॑ सामा॑थर्वै॒श्च॒ }

\ssankalpaalign{यो वै॑ दे॒वं म॑हादे॒वं॒ प्र॒णवं॑ पर॒मेश्व॑रम्}
{यः सर्वे॑ सर्व॑वेदै॒श्च॒ }

\ssankalpaalign{प्रय॑तः॒ प्रण॑वोङ्का॒रं॒ प्र॒णवं॑ पुरु॒षोत्त॑मम्}
{ओङ्का॑रं॒ प्रण॑वात्मा॒नं॒ }

\ssankalpaalign{योऽसौ॑ स॒र्वेषु॑ वेदे॒षु॒ प॒ठ्यते᳚ ह्यज॒ इश्व॑रः}
{अ॒कायो॑ निर्गु॑णो ह्या॒त्मा॒ }

\ssankalpaalign{गोभि॒र्जुष्टं॒ धने॑न॒ ह्यायु॑षा च॒ बले॑न च}
{प्र॒जया॑ प॒शुभिः॑ पुष्करा॒क्षं }

\ssankalpaalign{कैला॑स॒शिख॑रे र॒म्ये॒ श॒ङ्कर॑स्य शि॒वाल॑ये}
{दे॒वता᳚स्तत्र॑ मोद॒न्ते॒ }

\ssankalpa{वि॒श्वत॑श्चक्षुरु॒त वि॒श्वतो॑मुखो वि॒श्वतो॑हस्त उ॒त वि॒श्वत॑स्पात्। सं बा॒हुभ्यां॒ नम॑ति॒ सम्पत॑त्रै॒र्द्यावा॑पृथि॒वी ज॒नय॑न्दे॒व एक॒स्}

\ssankalpa{त्र्य॑म्बकं यजामहे सुग॒न्धिं पु॑ष्टि॒वर्ध॑नम्।\\ उ॒र्वा॒रु॒कमि॑व॒ बन्ध॑नान्मृ॒त्योर्मु॑क्षीय॒ माऽमृता॒त्}

\ssankalpaalign{च॒तुरो॑ वे॒दान॑धीयी॒त॒ स॒र्वशा᳚स्त्रम॒यं वि॑दुः}
{इ॒ति॒हा॒सपु॑राणा॒नां॒ }

\ssankalpa{मा नो॑ म॒हान्त॑मु॒त मा नो॑ अर्भ॒कं मा न॒ उक्ष॑न्तमु॒त मा न॑\linebreak उक्षि॒तम्। मा नो॑ वधीः पि॒तरं॒ मोत मा॒तरं॑ प्रि॒या मा\linebreak न॑स्त॒नुवो॑ रुद्र रीरिष॒स्}

\ssankalpa{मा न॑स्तो॒के तन॑ये॒ मा न॒ आयु॑षि॒ मा नो॒ गोषु॒ मा नो॒\linebreak अश्वे॑षु रीरिषः। वी॒रान्मा नो॑ रुद्र भामि॒तोऽव॑धीर्‌ह॒विष्म॑न्तो॒ नम॑सा विधेम ते॒ }

\ssankalpa{ऋ॒तꣳ स॒त्यं प॑रं ब्र॒ह्म॒ पु॒रुषं॑ कृष्ण॒पिङ्ग॑लम्।\\  ऊ॒र्ध्वरे॑तं वि॑रूपा॒क्षं॒  वि॒श्वरू॑पाय॒ वै नमो॒\\ नम॒स्}

\ssankalpa{कद्रु॒द्राय॒ प्रचे॑तसे मी॒ढुष्ट॑माय॒ तव्य॑से।\\ वो॒ चेम॒ शन्त॑मꣳ हृ॒दे। सर्वो॒ ह्ये॑ष रु॒द्रस्तस्मै॑ रु॒द्राय॒ नमो॑ अस्तु॒ }

\ssankalpa{ ब्रह्म॑जज्ञा॒नं प्र॑थ॒मं पु॒रस्ता॒द्विसी॑म॒तः सु॒रुचो॑ वे॒न आ॑वः।\\ सबु॒ध्निया॑ उप॒मा अ॑स्य वि॒ष्ठाः स॒तश्च॒ योनि॒मस॑तश्च॒\\ विव॒स्}

\ssankalpa{यः प्रा॑ण॒तो नि॑मिष॒तो म॑हि॒त्वैक॒ इद्राजा॒ जग॑तो ब॒भूव॑।\\ य ईशे॑ अ॒स्य द्वि॒पद॒श्चतु॑ष्पदः॒ कस्मै॑ दे॒वाय॑\\ ह॒विषा॑ विधेम॒ }

\ssankalpa{य आ᳚त्म॒दा ब॑ल॒दा यस्य॒ विश्व॑ उ॒पास॑ते प्र॒शिषं॒ यस्य॑ दे॒वाः।\\ यस्य॑ छा॒याऽमृतं॒ यस्य॑ मृ॒त्युः कस्मै॑ दे॒वाय॑\\ ह॒विषा॑ विधेम॒ }

\ssankalpa{यो रु॒द्रो अ॒ग्नौ यो अ॒फ्सु य ओष॑धीषु॒ यो रु॒द्रो विश्वा॒ भुव॑नाऽऽवि॒वेश॒ तस्मै॑ रु॒द्राय॒ नमो॑ अस्तु॒ }

\ssankalpa{ग॒न्ध॒द्वा॒रां दु॑राध॒\ar{}षां॒ नि॒त्यपु॑ष्टां करी॒षिणी᳚म्। ई॒श्वरीꣳ॑ सर्व॑भूता॒नां॒\\ तामि॒होप॑ह्वये॒ श्रियं }

य इ॒दꣳ शिव॑सङ्कल्पꣳ स॒दा ध्याय॑न्ति॒ ब्राह्म॑णाः।\\
 ते परं॒ मोक्षं॑ गमिष्यन्ति॒ तन्मे॒ मनः॑ शि॒वस॑ङ्क॒ल्पम॑स्तु॥

हृदयाय नमः॥
\end{center}
{\small \closesection}

\sect{पुरुषसूक्तम्}
ॐ स॒हस्र॑शीर्‌षा॒ पुरु॑षः। स॒ह॒स्रा॒क्षः स॒हस्र॑पात्। स भूमिं॑ वि॒श्वतो॑ वृ॒त्वा। अत्य॑तिष्ठद्दशाङ्गु॒लम्॥ पुरु॑ष ए॒वेदꣳ सर्वम्᳚। यद्भू॒तं यच्च॒ भव्यम्᳚। उ॒तामृ॑त॒त्वस्येशा॑नः। यदन्ने॑नाति॒रोह॑ति॥ ए॒तावा॑नस्य महि॒मा। अतो॒ ज्यायाꣴ॑श्च॒ पूरु॑षः।

 पादो᳚ऽस्य॒ विश्वा॑ भू॒तानि॑। त्रि॒पाद॑स्या॒मृतं॑ दि॒वि॥ त्रि॒पादू॒र्ध्व उदै॒त्पुरु॑षः। पादो᳚ऽस्ये॒हाऽऽभ॑वा॒त्पुनः॑। ततो॒ विश्व॒ङ्व्य॑क्रामत्। सा॒श॒ना॒न॒श॒ने अ॒भि॥ तस्मा᳚द्वि॒राड॑जायत। वि॒राजो॒ अधि॒ पूरु॑षः। स जा॒तो अत्य॑रिच्यत। प॒श्चाद्भूमि॒मथो॑ पु॒रः॥

 यत्पुरु॑षेण ह॒विषा᳚। दे॒वा य॒ज्ञमत॑न्वत। व॒स॒न्तो अ॑स्यासी॒दाज्यम्᳚। ग्री॒ष्म इ॒ध्मः श॒रद्ध॒विः॥ स॒प्तास्या॑ऽऽ\-सन्परि॒धयः॑। त्रिः स॒प्त स॒मिधः॑ कृ॒ताः। दे॒वा यद्य॒ज्ञं त॑न्वा॒नाः। अब॑ध्न॒न्पुरु॑षं प॒शुम्॥ तं य॒ज्ञं ब॒\ar हिषि॒ प्रौक्ष\sn। पुरु॑षं जा॒तम॑ग्र॒तः।

 तेन॑ दे॒वा अय॑जन्त। सा॒ध्या ऋष॑यश्च॒ ये॥ तस्मा᳚द्य॒ज्ञाथ्स॑र्व॒\-हुतः॑। सम्भृ॑तं पृषदा॒ज्यम्। प॒शूꣳस्ताꣳश्च॑क्रे वाय॒व्या\sn{}। आ॒र॒ण्यान्ग्रा॒म्याश्च॒ ये॥ तस्मा᳚द्य॒ज्ञाथ्स॑र्व॒हुतः॑। ऋचः॒ सामा॑नि जज्ञिरे। छन्दाꣳ॑सि जज्ञिरे॒ तस्मा᳚त्।

 यजु॒स्तस्मा॑दजायत॥ तस्मा॒दश्वा॑ अजायन्त। ये के चो॑भ॒याद॑तः। गावो॑ ह जज्ञिरे॒ तस्मा᳚त्। तस्मा᳚ज्जा॒ता अ॑जा॒वयः॑॥ यत्पुरु॑षं॒ व्य॑दधुः। क॒ति॒धा व्य॑कल्पयन्। मुखं॒ किम॑स्य॒ कौ बा॒हू। कावू॒रू पादा॑वुच्येते॥ ब्रा॒ह्म॒णो᳚ऽस्य॒ मुख॑मासीत्। बा॒हू रा॑ज॒न्यः॑ कृ॒तः।

 ऊ॒रू तद॑स्य॒ यद्वैश्यः॑। प॒द्भ्याꣳ शू॒द्रो अ॑जायत॥ च॒न्द्रमा॒ मन॑सो जा॒तः। चक्षोः॒ सूर्यो॑ अजायत। मुखा॒दिन्द्र॑श्चा॒ग्निश्च॑। प्रा॒णाद्वा॒युर॑जायत॥ नाभ्या॑ आसीद॒न्तरि॑क्षम्। शी॒र्ष्णो द्यौः सम॑वर्तत। प॒द्भ्यां भूमि॒र्दिशः॒ श्रोत्रा᳚त्। तथा॑ लो॒काꣳ अ॑कल्पयन्॥

 वेदा॒हमे॒तं पुरु॑षं म॒हान्तम्᳚। आ॒दि॒त्यव॑र्णं॒ तम॑स॒स्तु॒ पा॒रे॥ सर्वा॑णि रू॒पाणि॑ वि॒चित्य॒ धीरः॑। नामा॑नि कृ॒त्वाऽभि॒वद॒न् यदास्ते᳚॥ धा॒ता पु॒रस्ता॒द्यमु॑दाज॒हार॑। श॒क्रः प्रवि॒द्वान्प्र॒दिश॒श्चत॑स्रः। तमे॒वं वि॒द्वान॒मृत॑ इ॒ह भ॑वति। नान्यः पन्था॒ अय॑नाय विद्यते॥ य॒ज्ञेन॑ य॒ज्ञम॑यजन्त दे॒वाः। तानि॒ धर्मा॑णि प्रथ॒मान्या॑सन्। ते ह॒ नाकं॑ महि॒मानः॑ सचन्ते। यत्र॒ पूर्वे॑ सा॒ध्याः सन्ति॑ दे॒वाः॥ 

शिरसे स्वाहा॥

{\small \closesection}

\sect{उत्तरनारायणम्}

अ॒द्भ्यः सम्भू॑तः पृथि॒व्यै रसा᳚च्च। वि॒श्वक॑र्मणः॒ सम॑वर्त॒ताधि॑। तस्य॒ त्वष्टा॑ वि॒दध॑द्रू॒पमे॑ति। तत्पुरु॑षस्य॒ विश्व॒माजा॑न॒मग्रे᳚॥ वेदा॒हमे॒तं पुरु॑षं म॒हान्तम्᳚। आ॒दि॒त्यव॑र्णं॒ तम॑सः॒ पर॑स्तात्। तमे॒वं वि॒द्वान॒मृत॑ इ॒ह भ॑वति। नान्यः पन्था॑ विद्य॒तेय॑ऽनाय॥ प्र॒जाप॑तिश्चरति॒ गर्भे॑ अ॒न्तः। अ॒जाय॑मानो बहु॒धा विजा॑यते। 

तस्य॒ धीराः॒ परि॑जानन्ति॒ योनिम्᳚। मरी॑चीनां प॒दमि॑च्छन्ति वे॒धसः॑॥ यो दे॒वेभ्य॒ आत॑पति। यो दे॒वानां᳚ पु॒रोहि॑तः। पूर्वो॒ यो दे॒वेभ्यो॑ जा॒तः। नमो॑ रु॒चाय॒ ब्राह्म॑ये॥ रुचं॑ ब्रा॒ह्मं ज॒नय॑न्तः। दे॒वा अग्रे॒ तद॑ब्रुवन्। यस्त्वै॒वं ब्रा᳚ह्म॒णो वि॒द्यात्। तस्य॑ दे॒वा अस॒न् वशे᳚॥ ह्रीश्च॑ ते ल॒क्ष्मीश्च॒ पत्न्यौ᳚। अ॒हो॒रा॒त्रे पा॒र्श्वे। नक्ष॑त्राणि रू॒पम्। अ॒श्विनौ॒ व्यात्तम्᳚। इ॒ष्टं म॑निषाण। अ॒मुं म॑निषाण। सर्वं॑ मनिषाण॥ 

शिखायै वषट्॥

{\small \closesection}

\sect{अप्रतिरथम्}
\centerline{\normalsize (तैत्तिरीयसंहितायां काण्डः – ४/प्रश्नः – ६/अनुवाकः – ४)}

आ॒शुः शिशा॑नो वृष॒भो न यु॒ध्मो घ॑नाघ॒नः क्षोभ॑णश्चर्‌षणी॒नाम्। स॒ङ्क्रन्द॑नोऽनिमि॒ष ए॑कवी॒रः श॒तꣳ सेना॑ अजयत् सा॒कमिन्द्रः॑। स॒ङ्क्रन्द॑नेनानिमि॒षेण॑ जि॒ष्णुना॑ युत्का॒रेण॑ दुश्च्यव॒नेन॑ धृ॒ष्णुना᳚। तदिन्द्रे॑ण जयत॒ तथ्स॑हध्वं॒ युधो॑ नर॒ इषु॑हस्तेन॒ वृष्णा᳚। स इषु॑हस्तैः॒ सनि॑ष॒ङ्गिभि॑र्व॒शी सꣴस्र॑ष्टा॒ स युध॒ इन्द्रो॑ ग॒णेन॑। स॒ꣳ॒सृ॒ष्ट॒जिथ्सो॑म॒पा बा॑हुश॒ध्यू᳚र्ध्वध॑न्वा॒ प्रति॑हिताभि॒रस्ता᳚।

बृह॑स्पते॒ परि॑ दीया॒ रथे॑न रक्षो॒हाऽमित्राꣳ॑ अप॒बाध॑मानः। प्र॒भ॒ञ्जन्थ्सेनाः᳚ प्रमृ॒णो यु॒धा जय॑न्न॒स्माक॑मेध्यवि॒ता रथा॑नाम्। गो॒त्र॒भिदं॑ गो॒विदं॒ वज्र॑बाहुं॒ जय॑न्त॒मज्म॑ प्रमृ॒णन्त॒मोज॑सा। इ॒मꣳ स॑जाता॒ अनु॑ वीरयध्व॒मिन्द्रꣳ॑ सखा॒योऽनु॒ सꣳ र॑भध्वम्। ब॒ल॒वि॒ज्ञा॒यः स्थवि॑रः॒ प्रवी॑रः॒ सह॑स्वान् वा॒जी सह॑मान उ॒ग्रः। अ॒भिवी॑रो अ॒भिस॑त्त्वा सहो॒जा जैत्र॑मिन्द्र॒ रथ॒मा ति॑ष्ठ गो॒वित्। अ॒भि गो॒त्राणि॒ सह॑सा॒ गाह॑मानोऽदा॒यो वी॒रः श॒तम॑न्यु॒रिन्द्रः॑।

दु॒श्च्य॒व॒न पृ॑तना॒षाड॑यु॒ध्यो᳚ऽस्माक॒ꣳ॒ सेना॑ अवतु॒ प्र यु॒थ्सु। इन्द्र॑ आसां ने॒ता बृह॒स्पति॒र्दक्षि॑णा य॒ज्ञः पु॒र ए॑तु॒ सोमः॑। दे॒व॒से॒नाना॑मभिभञ्जती॒नां जय॑न्तीनां म॒रुतो॑ य॒न्त्वग्रे᳚। इन्द्र॑स्य॒ वृष्णो॒ वरु॑णस्य॒ राज्ञ॑ आदि॒त्यानां᳚ म॒रुता॒ꣳ॒ शर्ध॑ उ॒ग्रम्। म॒हाम॑नसां भुवनच्य॒वानां॒ घोषो॑ दे॒वानां॒ जय॑ता॒मुद॑स्थात्। अ॒स्माक॒मिन्द्रः॒ समृ॑तेषु ध्व॒जेष्व॒स्माकं॒ या इष॑व॒स्ता ज॑यन्तु।

अ॒स्माकं॑ वी॒रा उत्त॑रे भवन्त्व॒स्मानु॑ देवा अवता॒ हवे॑षु। उद्ध॑र्‌षय मघव॒न्नायु॑धा॒न्युत् सत्त्व॑नां माम॒कानां॒ महाꣳ॑सि। उद्वृ॑त्रहन् वा॒जिनां॒ वाजि॑ना॒न्युद्रथा॑नां॒ जय॑तामेतु॒ घोषः॑। उप॒ प्रेत॒ जय॑ता नरः स्थि॒रा वः॑ सन्तु बा॒हवः॑। इन्द्रो॑ वः॒ शर्म॑ यच्छत्त्वनाधृ॒ष्या यथाऽस॑थ। अव॑सृष्टा॒ परा॑ पत॒ शर॑व्ये॒ ब्रह्म॑सꣳशिता।

गच्छा॒मित्रा॒न् प्रवि॑श॒ मैषां॒ कं च॒नोच्छि॑षः। मर्मा॑णि ते॒ वर्म॑भिश्छादयामि॒ सोम॑स्त्वा॒ राजा॒ऽमृते॑ना॒भिव॑स्ताम्। उ॒रोर्वरी॑यो॒ वरि॑वस्ते अस्तु॒ जय॑न्तं॒ त्वामनु॑ मदन्तु दे॒वाः। यत्र॑ बा॒णाः स॒म्पत॑न्ति कुमा॒रा वि॑शि॒खा इ॑व। इन्द्रो॑ न॒स्तत्र॑ वृत्र॒हा वि॑श्वा॒हा शर्म॑ यच्छतु॥
कवचाय हुम्॥

{\small \closesection}

\sect{प्रतिपूरुषम् (सं॰)}
\centerline{\normalsize (तैत्तिरीयसंहितायां काण्डः – १/प्रश्नः – ८/अनुवाकः – ६)}

प्र॒ति॒पू॒रु॒षमेक॑कपाला॒न्निर्व॑प॒त्येक॒मति॑रिक्तं॒ याव॑न्तो गृ॒ह्याः᳚ स्मस्तेभ्यः॒ कम॑करं पशू॒नाꣳ शर्मा॑सि॒ शर्म॒ यज॑मानस्य॒ शर्म॑ मे य॒च्छैक॑ ए॒व रु॒द्रो न द्वि॒तीया॑य तस्थ आ॒खुस्ते॑ रुद्र प॒शुस्तं जु॑षस्वै॒ष ते॑ रुद्र भा॒गः स॒ह स्वस्राऽम्बि॑कया॒ तं जु॑षस्व भेष॒जं गवेऽश्वा॑य॒ पुरु॑षाय भेष॒जमथो॑ अ॒स्मभ्यं॑ भेष॒जꣳ सुभे॑षजं॒ यथाऽस॑ति। सु॒गं मे॒षाय॑ मे॒ष्या॑ अवा᳚म्ब रु॒द्रम॑दिम॒ह्यव॑ दे॒वं त्र्य॑म्बकम्। यथा॑ नः॒ श्रेय॑सः॒ कर॒द्यथा॑ नो॒ वस्य॑सः॒ कर॒द्यथा॑ नः पशु॒मतः॒ कर॒द्यथा॑ नो व्यवसा॒यया᳚त्। त्र्य॑म्बकं यजामहे सुग॒न्धिं पु॑ष्टि॒वर्ध॑नम्। उ॒र्वा॒रु॒कमि॑व॒ बन्ध॑नान्मृ॒त्योर्मु॑क्षीय॒ माऽमृता᳚त्॥ ए॒ष ते॑ रुद्रभा॒गस्तं जु॑षस्व॒ तेना॑व॒सेन॑ प॒रो मूज॑व॒तोऽती॒ह्यव॑ततधन्वा॒ पिना॑कहस्तः॒ कृत्ति॑वासाः॥


\sect{प्रतिपूरुषम् (ब्रा॰)}
\centerline{\normalsize (तैत्तिरीयब्राह्मणे अष्टकं – १/प्रश्नः – ६/अनुवाकः – १०)}

प्र॒ति॒पू॒रु॒षमेक॑कपाला॒न्निर्व॑पति। जा॒ता ए॒व प्र॒जा रु॒द्रान्नि॒रव॑दयते। एक॒मति॑रिक्तम्। ज॒नि॒ष्यमा॑णा ए॒व प्र॒जा रु॒द्रान्नि॒रव॑दयते। एक॑कपाला भवन्ति। ए॒क॒धैव रु॒द्रं नि॒रव॑दयते। नाभिघा॑रयति। यद॑भिघा॒रये᳚त्। अ॒न्त॒र॒व॒चा॒रिणꣳ॑ रु॒द्रं कु॑र्यात्। ए॒को॒ल्मु॒केन॑ यन्ति। 
तद्धि रु॒द्रस्य॑ भाग॒धेयम्᳚। इ॒मां दिशं॑ यन्ति। ए॒षा वै रु॒द्रस्य॒ दिक्। स्वाया॑मे॒व दि॒शि रु॒द्रं नि॒रव॑दयते। रु॒द्रो वा अ॑प॒शुका॑या॒ आहु॑त्यै॒ नाति॑ष्ठत। अ॒सौ ते॑ प॒शुरिति॒ निर्दि॑शे॒द्यं द्वि॒ष्यात्। यमे॒व द्वेष्टि॑। तम॑स्मै प॒शुं निर्दि॑शति। यदि॒ न द्वि॒ष्यात्। आ॒खुस्ते॑ प॒शुरिति॑ ब्रूयात्। 
न ग्रा॒म्यान् प॒शून् हि॒नस्ति॑। नाऽऽर॒ण्यान्। च॒तु॒ष्प॒थे जु॑होति। ए॒ष वा अ॑ग्नी॒नां पड्बी॑शो॒ नाम॑। अ॒ग्नि॒वत्ये॒व जु॑होति। म॒ध्य॒मेन॑ प॒र्णेन॑ जुहोति। स्रुग्घ्ये॑षा। अथो॒ खलु॑। अ॒न्त॒मेनै॒व हो॑त॒व्यम्᳚। अ॒न्त॒त ए॒व रु॒द्रं नि॒रव॑दयते। 
ए॒ष ते॑ रुद्र भा॒गः स॒ह स्वस्राऽम्बि॑क॒येत्या॑ह। श॒रद्वा अ॒स्याम्बि॑का॒ स्वसा᳚। तया॒ वा ए॒ष हि॑नस्ति। यꣳ हि॒नस्ति॑। तयै॒वैनꣳ॑ स॒ह श॑मयति। भे॒ष॒जं गव॒ इत्या॑ह। याव॑न्त ए॒व ग्रा॒म्याः प॒शवः॑। तेभ्यो॑ भेष॒जं क॑रोति। अवा᳚म्ब रु॒द्रम॑दिम॒हीत्या॑ह। आ॒शिष॑मे॒वैतामाशा᳚स्ते। 
त्र्य॑म्बकं यजामह॒ इत्या॑ह। मृ॒त्योर्मु॑क्षीय॒ माऽमृता॒दिति॒ वावैतदा॑ह। उत्कि॑रन्ति। भग॑स्य लीफ्सन्ते। मूते॑ कृ॒त्वाऽऽस॑जन्ति। यथा॒ जनं॑ य॒ते॑ऽव॒सं क॒रोति॑। ता॒दृगे॒व तत्। ए॒ष ते॑ रुद्र भा॒ग इत्या॑ह नि॒रव॑त्यै। अप्र॑तीक्ष॒माय॑न्ति। अ॒पः परि॑षिञ्चति। \mbox{रु॒द्रस्या॒न्तर्‌\mbox{}हि॑त्यै}। प्र वा ए॒ते᳚स्माल्लो॒काच्च्य॑वन्ते। ये त्र्य॑म्बकै॒श्चर॑न्ति। आ॒दि॒त्यं च॒रुं पुन॒रेत्य॒ निर्व॑पति। इ॒यं वा अदि॑तिः। अ॒स्यामे॒व प्रति॑तिष्ठन्ति। 

नेत्रत्रयाय वौषट्॥

{\small \closesection}



%\newpage
\sect{शतरुद्रीयम् (सं॰)}
\centerline{\normalsize (तैत्तिरीयसंहितायां काण्डः – १/प्रश्नः – ३/अनुवाकः – १४)}

त्वम॑ग्ने रु॒द्रो असु॑रो म॒हो दि॒वस्त्वꣳ शर्धो॒ मारु॑तं पृ॒क्ष ई॑शिषे। त्वं वातै॑ररु॒णैर्या॑सि शङ्ग॒यस्त्वं पू॒षा वि॑ध॒तः पा॑सि॒ नु त्मना᳚।
आ वो॒ राजा॑नमध्व॒रस्य॑ रु॒द्रꣳ होता॑रꣳ सत्य॒यज॒ꣳ॒ रोद॑स्योः। अ॒ग्निं पु॒रा त॑नयि॒त्नोर॒चित्ता॒द्धिर॑ण्यरूप॒मव॑से कृणुध्वम्। अ॒ग्निर्‌होता॒ नि ष॑सादा॒ यजी॑यानु॒पस्थे॑ मा॒तुः सु॑र॒भावु॑ लो॒के। युवा॑ क॒वि पु॑रुनि॒ष्ठ ऋ॒तावा॑ ध॒र्ता कृ॑ष्टी॒नामु॒त मध्य॑ इ॒द्धः। 

सा॒ध्वीम॑कर्दे॒ववी॑तिं नो अ॒द्य य॒ज्ञस्य॑ जि॒ह्वाम॑विदाम॒ गुह्या᳚म्। स आयु॒राऽगा᳚थ्सुर॒भिर्वसा॑नो भ॒द्राम॑कर्दे॒वहू॑तिं नो अ॒द्य। अक्र॑न्दद॒ग्निः स्त॒नय॑न्निव॒ द्यौः क्षामा॒ रेरि॑हद्वी॒रुधः॑ सम॒ञ्जन्न्। स॒द्यो ज॑ज्ञा॒नो वि हीमि॒द्धो अख्य॒दा रोद॑सी भा॒नुना॑ भात्य॒न्तः। त्वे वसू॑नि पुर्वणीक होतर्दो॒षा वस्तो॒रेरि॑रे य॒ज्ञिया॑सः। 

क्षामे॑व॒ विश्वा॒ भुव॑नानि॒ यस्मि॒न्थ्सꣳ सौभ॑गानि दधि॒रे पा॑व॒के। तुभ्यं॒ ता अ॑ङ्गिरस्तम॒ विश्वाः᳚ सुक्षि॒तयः॒ पृथ॑क्। अग्ने॒ कामा॑य येमिरे। अ॒श्याम॒ तं काम॑मग्ने॒ तवो॒त्य॑श्याम॑ र॒यिꣳ र॑यिवः सु॒वीरम्᳚। अ॒श्याम॒ वाज॑म॒भि वा॒जय॑न्तो॒ऽश्याम॑ द्यु॒म्नम॑जरा॒ऽजर॑न्ते। श्रेष्ठं॑ यविष्ठ भार॒ताऽग्ने᳚ द्यु॒मन्त॒मा भ॑र। 

वसो॑ पुरु॒स्पृहꣳ॑ र॒यिम्। स श्वि॑ता॒नस्त॑न्य॒तू रो॑चन॒स्था अ॒जरे॑भि॒र्नान॑दद्भि॒र्यवि॑ष्ठः। यः पा॑व॒कः पु॑रु॒तमः॑ पु॒रूणि॑ पृ॒थून्य॒ग्निर॑नु॒याति॒ भर्व\snn{}। आयु॑ष्टे वि॒श्वतो॑ दधद॒यम॒ग्निर्वरे᳚ण्यः। पुन॑स्ते प्रा॒ण आय॑ति॒ परा॒ यक्ष्मꣳ॑ सुवामि ते। आ॒यु॒र्दा अ॑ग्ने ह॒विषो॑ जुषा॒णो घृ॒तप्र॑तीको घृ॒तयो॑निरेधि। घृ॒तं पी॒त्वा मधु॒ चारु॒ गव्यं॑ पि॒तेव॑ पु॒त्रम॒भिर॑क्षतादि॒मम्।

तस्मै॑ ते प्रति॒हर्य॑ते॒ जात॑वेदो॒ विच॑र्‌षणे। अग्ने॒ जना॑मि सुष्टु॒तिम्। दि॒वस्परि॑ प्रथ॒मं ज॑ज्ञे अ॒ग्निर॒स्मद्द्वि॒तीयं॒ परि॑ जा॒तवे॑दाः। तृ॒तीय॑म॒फ्सु नृ॒मणा॒ अज॑स्र॒मिन्धा॑न एनं जरते स्वा॒धीः। शुचिः॑ पावक॒ वन्द्योऽग्ने॑ बृ॒हद्विरो॑चसे। त्वं घृ॒तेभि॒राहु॑तः। दृ॒शा॒नो रु॒क्म उ॒र्व्या व्य॑द्यौद्दु॒र्मर्‌ष॒मायुः॑ श्रि॒ये रु॑चा॒नः। अ॒ग्निर॒मृतो॑ अभव॒द्वयो॑भि॒र्यदे॑नं॒ द्यौरज॑नयथ्सु॒रेताः᳚। 

आ यदि॒षे नृ॒पतिं॒ तेज॒ आन॒ट्छुचि॒ रेतो॒ निषि॑क्तं॒ द्यौर॒भीके᳚।
अ॒ग्निः शर्ध॑मनव॒द्यं युवा॑नꣴ स्वा॒धियं॑ जनयथ्सू॒दय॑च्च। स तेजी॑यसा॒ मन॑सा॒ त्वोत॑ उ॒त शि॑क्ष स्वप॒त्यस्य॑ शि॒क्षोः। अग्ने॑ रा॒यो नृत॑मस्य॒ प्रभू॑तौ भू॒याम॑ ते सुष्टु॒तय॑श्च॒ वस्वः॑। अग्ने॒ सह॑न्त॒मा भ॑र द्यु॒म्नस्य॑ प्रा॒सहा॑ र॒यिम्।

 विश्वा॒ यश्च॑र्‌ष॒णीर॒भ्या॑सा वाजे॑षु सा॒सह॑त्। 
तम॑ग्ने पृतना॒ सहꣳ॑ र॒यिꣳ स॑हस्व॒ आ भ॑र। त्वꣳ हि स॒त्यो अद्भु॑तो दा॒ता वाज॑स्य॒ गोम॑तः। उ॒क्षान्ना॑य व॒शान्ना॑य॒ सोम॑पृष्ठाय वे॒धसे᳚। स्तोमै᳚र्विधेमा॒ग्नये᳚। व॒द्मा हि सू॑नो॒ अस्य॑द्म॒सद्वा॑ च॒क्रे अ॒ग्निर्ज॒नुषाऽज्माऽन्नम्᳚। स त्वं न॑ ऊर्जसन॒ ऊर्जं॑ धा॒  राजे॑व जेरवृ॒के क्षे᳚ष्य॒न्तः।

अग्न॒ आयूꣳ॑षि पवस॒ आ सु॒वोर्ज॒मिषं॑ च नः। आ॒रे बा॑धस्व दु॒च्छुना᳚म्। अग्ने॒ पव॑स्व॒ स्वपा॑ अ॒स्मे वर्चः॑ सु॒वीर्यम्᳚। दध॒त्पोषꣳ॑ र॒यिं मयि॑।
अग्ने॑ पावक रो॒चिषा॑ म॒न्द्रया॑ देव जि॒ह्वया᳚। आ दे॒वान् व॑क्षि॒ यक्षि॑ च। स नः॑ पावक दीदि॒वोऽग्ने॑ दे॒वाꣳ इ॒हाऽऽव॑ह। उप॑ य॒ज्ञꣳ ह॒विश्च॑ नः। अ॒ग्निः शुचि॑ व्रततमः॒ शुचि॒र्विप्रः॒ शुचिः॑ क॒विः। शुची॑ रोचत॒ आहु॑तः। उद॑ग्ने॒ शुच॑य॒स्तव॑ शु॒क्रा भ्राज॑न्त ईरते। तव॒ ज्योतीꣴ॑ष्य॒र्चयः॑॥


%ꣳ ꣳ॑ ꣳ॒
\sect{शतरुद्रीयम् (ब्रा॰)}
\centerline{\normalsize (तैत्तिरीयब्राह्मणे काठके प्रश्नः – २/अनुवाकः – २)}
 त्वम॑ग्ने रु॒द्रो असु॑रो म॒हो दि॒वः। त्वꣳ शर्धो॒ मारु॑तं पृ॒क्ष ई॑शिषे।
 त्वं वातै॑ररु॒णैर्या॑सि शङ्ग॒यः। त्वं पू॒षा वि॑ध॒तः पा॑सि॒ नु त्मना᳚।
 देवा॑ दे॒वेषु॑ श्रयध्वम्। प्रथ॑मा द्वि॒तीये॑षु श्रयध्वम्।
 द्विती॑यास्तृ॒तीये॑षु श्रयध्वम्। तृती॑याश्चतु॒र्थेषु॑ श्रयध्वम्।
 च॒तु॒र्थाः प॑ञ्च॒मेषु॑ श्रयध्वम्। प॒ञ्च॒माः ष॒ष्ठेषु॑ श्रयध्वम्॥
%%
 ष॒ष्ठाः स॑प्त॒मेषु॑ श्रयध्वम्। स॒प्त॒मा अ॑ष्ट॒मेषु॑ श्रयध्वम्।
 अ॒ष्ट॒मा न॑व॒मेषु॑ श्रयध्वम्। न॒व॒मा द॑श॒मेषु॑ श्रयध्वम्।
 द॒श॒मा ए॑काद॒शेषु॑ श्रयध्वम्। ए॒का॒द॒शा द्वा॑द॒शेषु॑ श्रयध्वम्।
 द्वा॒द॒शास्त्र॑योद॒शेषु॑ श्रयध्वम्। त्र॒यो॒द॒शाश्च॑तुर्द॒शेषु॑ श्रयध्वम्।
 च॒तु॒र्द॒शाः प॑ञ्चद॒शेषु॑ श्रयध्वम्। प॒ञ्च॒द॒शाः षो॑ड॒शेषु॑ श्रयध्वम्॥
%%
 षो॒ड॒शाः स॑प्तद॒शेषु॑ श्रयध्वम्। स॒प्त॒द॒शा अ॑ष्टाद॒शेषु॑ श्रयध्वम्। अ॒ष्टा॒द॒शा ए॑कान्नवि॒ꣳ॒शेषु॑ श्रयध्वम्। ए॒का॒न्न॒वि॒ꣳ॒शा वि॒ꣳ॒शेषु॑ श्रयध्वम्। वि॒ꣳ॒शा ए॑कवि॒ꣳ॒शेषु॑ श्रयध्वम्। ए॒क॒वि॒ꣳ॒शा द्वा॑वि॒ꣳ॒शेषु॑ श्रयध्वम्। द्वा॒वि॒ꣳ॒शास्त्र॑योवि॒ꣳ॒शे॑षु श्रयध्वम्। त्र॒यो॒वि॒ꣳ॒शाश्च॑तुर्वि॒ꣳ॒शेषु॑ श्रयध्वम्। च॒तु॒र्वि॒ꣳ॒शाः प॑ञ्चवि॒ꣳ॒शेषु॑ श्रयध्वम्। प॒ञ्च॒वि॒ꣳ॒शाः ष॑ड्वि॒ꣳ॒शेषु॑ श्रयध्वम्॥
%%
 ष॒ड्वि॒ꣳ॒शाः स॑प्तवि॒ꣳ॒शेषु॑ श्रयध्वम्। स॒प्त॒वि॒ꣳ॒शा अ॑ष्टावि॒ꣳ॒शेषु॑ श्रयध्वम्।
 अ॒ष्टा॒वि॒ꣳ॒शा ए॑कान्नत्रि॒ꣳ॒शेषु॑ श्रयध्वम्। ए॒का॒न्न॒त्रि॒ꣳ॒शास्त्रि॒ꣳ॒शेषु॑ श्रयध्वम्।
 त्रि॒ꣳ॒शा ए॑कत्रि॒ꣳ॒शेषु॑ श्रयध्वम्। ए॒क॒त्रि॒ꣳ॒शा द्वा᳚त्रि॒ꣳ॒शेषु॑ श्रयध्वम्।
 द्वा॒त्रि॒ꣳ॒शास्त्र॑यस्त्रि॒ꣳ॒शेषु॑ श्रयध्वम्। देवा᳚स्त्रिरेकादशा॒स्त्रिस्त्र॑यस्त्रिꣳशाः।
 उत्त॑रे भवत। उत्त॑रवर्त्मान॒ उत्त॑रसत्वानः।
 यत्का॑म इ॒दं जु॒होमि॑। तन्मे॒ समृ॑ध्यताम्।
 व॒यꣴ स्या॑म॒ पत॑यो रयी॒णाम्। भूर्भुवः॒ स्वः॑ स्वाहा᳚॥\\
ॐ भूर्भुवस्सुवरोमिति दिग्बन्धः॥ अस्त्राय फट्॥

%{\small \closesection}
%\clearpage

\sect{पञ्चाङ्गम्}
ह॒ꣳ॒सः शु॑चि॒षद्वसु॑रन्तरिक्ष॒सद्धोता॑ वेदि॒षदति॑थिर्दुरोण॒सत्।\\
नृ॒षद्व॑र॒सदृ॑त॒सद्व्यो॑म॒सद॒ब्जा गो॒जा ऋ॑त॒जा अ॑द्रि॒जा ऋ॒तं बृ॒हत्॥

प्रतद्विष्णुः॑ स्तवते वी॒र्या॑य। मृ॒गो न भी॒मः कु॑च॒रो गि॑रि॒ष्ठाः।\\
 यस्यो॒रुषु॑ त्रि॒षु वि॒क्रम॑णेषु। अधि॑क्षि॒यन्ति॒ भुव॑नानि॒ विश्वा᳚॥

 त्र्य॑म्बकं यजामहे सुग॒न्धिं पु॑ष्टि॒वर्ध॑नम्।\\
 उ॒र्वा॒रु॒कमि॑व॒ बन्ध॑नान्मृ॒त्योर्मु॑क्षीय॒ माऽमृता᳚त्॥ 

 तथ्स॑वि॒तुर्वृ॑णीमहे। व॒यं दे॒वस्य॒ भोज॑नम्‌।\\
 श्रेष्ठꣳ॑ सर्व॒धात॑मम्‌। तुरं॒ भग॑स्य धीमहि॥ 

विष्णु॒र्योनिं॑ कल्पयतु। त्वष्टा॑ रू॒पाणि॑ पिꣳशतु।\\
आसि॑ञ्चतु प्र॒जाप॑तिः। धा॒ता गर्भं॑ दधातु ते॥

{\small \closesection}

\sect{अष्टाङ्ग-नमस्काराः}
\resetShloka
\newcounter{ashtanga}
\newcommand{\um}[1]{\refstepcounter{ashtanga}\centerline{\lbrack#1\rbrack\  उमामहेश्वराभ्यां नमः॥\devanumber{\arabic{ashtanga}}॥}\smallskip}

\twolineshloka*
{हि॒र॒ण्य॒ग॒र्भः सम॑वर्त॒ताग्रे॑ भू॒तस्य॑ जा॒तः पति॒रेक॑ आसीत्}
{सदा॑धार पृथि॒वीं द्यामु॒तेमां कस्मै॑ दे॒वाय॑ ह॒विषा॑ विधेम}
\um{उरसा}


\twolineshloka*
{यः प्रा॑ण॒तो नि॑मिष॒तो म॑हि॒त्वैक॒ इद्राजा॒ जग॑तो ब॒भूव॑}
{य ईशे॑ अ॒स्य द्वि॒पद॒श्चतु॑ष्पदः॒ कस्मै॑ दे॒वाय॑ ह॒विषा॑ विधेम}
\um{शिरसा}

\twolineshloka
{ब्रह्म॑जज्ञा॒नं प्र॑थ॒मं पु॒रस्ता॒द्विसी॑म॒तः सु॒रुचो॑ वे॒न आ॑वः}
{सबु॒ध्निया॑ उप॒मा अ॑स्य वि॒ष्ठाः स॒तश्च॒ योनि॒मस॑तश्च॒ विवः॑}
\um{दृष्ट्या}

म॒ही द्यौः पृ॑थि॒वी च॑ न इ॒मं य॒ज्ञं मि॑मिक्षताम्। पि॒पृ॒तान्नो॒ भरी॑मभिः॥\\
\um{मनसा}

\twolineshloka*
{उप॑श्वासय पृथि॒वीमु॒त द्यां पु॑रु॒त्रा ते॑ मनुतां॒ विष्ठि॑तं॒ जग॑त्}
{स दु॑न्दुभे स॒जूरिन्द्रे॑ण दे॒वैर्दू॒राद्दवी॑यो॒ अप॑सेध॒ शत्रू\sn{}}
\um{वचसा}

अग्ने॒ नय॑ सु॒पथा॑ रा॒ये अ॒स्मान् विश्वा॑नि देव व॒युना॑नि वि॒द्वान्। 
यु॒यो॒ध्य॑स्मज्जु॑हुरा॒णमेनो॒ भूयि॑ष्ठां ते॒ नम॑ उक्तिं विधेम॥\\
\um{पद्भ्याम्}

या ते॑ अग्ने॒ रुद्रि॑या त॒नूस्तया॑ नः पाहि॒ तस्या᳚स्ते॒ स्वाहा᳚॥
\um{कराभ्याम्}

इ॒मं य॑मप्रस्त॒रमाहि सीदाऽङ्गि॑रोभिः पि॒तृभिः॑ संविदा॒नः। आत्वा॒ मन्त्राः᳚ कविश॒स्ता व॑हन्त्वे॒ना रा॑जन् ह॒विषा॑ मादयस्व॥\\
\um{कर्णाभ्याम्}
{\small \closesection}
%\pagebreak[4]

\sect{लघुन्यासे श्री रुद्रध्यानम्}
अथाऽऽत्मानं शिवात्मानं श्री रुद्र रूपं ध्यायेत्॥

शुद्धस्फटिकसङ्काशं त्रिनेत्रं पञ्चवक्त्रकम्।\\
गङ्गाधरं दशभुजं सर्वाभरणभूषितम्॥

नीलग्रीवं शशाङ्काङ्कं नागयज्ञोपवीतिनम्।\\
व्याघ्रचर्मोत्तरीयं च वरेण्यमभयप्रदम्॥

कमण्डल्वक्षसूत्राणां धारिणं शूलपाणिनम्।\\
ज्वलन्तं पिङ्गलजटाशिखामुद्योतधारिणम्॥

वृषस्कन्धसमारूढम् उमादेहार्धधारिणम्।\\
अमृते  नाप्लुतं शान्तं दिव्यभोगसमन्वितम्॥

दिग्देवता समायुक्तं सुरासुरनमस्कृतम्।\\
नित्यं च शाश्वतं शुद्धं ध्रुवमक्षरमव्ययम्॥

सर्वव्यापिनमीशानं रुद्रं वै विश्वरूपिणम्।\\
एवं ध्यात्वा द्विजः सम्यक् ततो यजनमारभेत्॥

अथातो रुद्र स्नानार्चनाभिषेकविधिं व्याख्यास्यामः। आदित एव तीर्थे स्नात्वा उदेत्य शुचिः
प्रयतो ब्रह्मचारी शुक्लवासा ईशानस्य प्रतिकृतिं कृत्वा तस्य दक्षिणप्रत्यग्देशे देवाभिमुखः स्थित्वा आत्मनि देवताः स्थापयेत्॥

\sect{लघुन्यासे देवता-स्थापनम्}
प्रजनने ब्रह्मा तिष्ठतु। पादयोर्विष्णुस्तिष्ठतु। 
हस्तयोर्हरस्तिष्ठतु। बाह्वोरिन्द्रस्तिष्ठतु। 
जठरे अग्निस्तिष्ठतु। हृदये शिवस्तिष्ठतु। 
कण्ठे वसवस्तिष्ठन्तु। वक्त्रे सरस्वती तिष्ठतु। 
नासिकयोर्\-वायुस्तिष्ठतु। नयनयोश्चन्द्रादित्यौ तिष्ठेताम्। 
कर्णयोरश्विनौ तिष्ठेताम्। ललाटे रुद्रास्तिष्ठन्तु। 
मूर्ध्न्यादित्यास्तिष्ठन्तु। शिरसि महादेवस्तिष्ठतु। 
शिखायां वामदेवस्तिष्ठतु। पृष्ठे पिनाकी तिष्ठतु। 
पुरतः शूली तिष्ठतु। पार्श्वयोः शिवाशङ्करौ तिष्ठेताम्। 
सर्वतो वायुस्तिष्ठतु। ततो बहिः सर्वतोऽग्निर्ज्वालामाला-परिवृतस्तिष्ठतु।
सर्वेष्वङ्गेषु सर्वा देवता यथास्थानं तिष्ठन्तु। मां रक्षन्तु।
\lbrack सर्वान् महाजनान् सकुटुम्बं रक्षन्तु॥\rbrack

अ॒ग्निर्मे॑ वा॒चि श्रि॒तः।   वाग्घृद॑ये।   हृद॑यं॒ मयि॑।   अ॒हम॒मृते᳚।   अ॒मृतं॒ ब्रह्म॑णि। (जिह्वा)

 वा॒युर्मे᳚ प्रा॒णे श्रि॒तः।   प्रा॒णो हृद॑ये।   हृद॑यं॒ मयि॑।   अ॒हम॒मृते᳚।   अ॒मृतं॒ ब्रह्म॑णि। (नासिका)

   सूर्यो॑ मे॒ चक्षु॑षि श्रि॒तः।   चक्षु॒र्‌॒हृद॑ये।   हृद॑यं॒ मयि॑।   अ॒हम॒मृते᳚।   अ॒मृतं॒ ब्रह्म॑णि। (नेत्रे)

   च॒न्द्रमा॑ मे॒ मन॑सि श्रि॒तः।   मनो॒ हृद॑ये।   हृद॑यं॒ मयि॑।   अ॒हम॒मृते᳚।   अ॒मृतं॒ ब्रह्म॑णि। (वक्षः)

   दिशो॑ मे॒ श्रोत्रे᳚ श्रि॒ताः।   श्रोत्र॒ꣳ॒ हृद॑ये।   हृद॑यं॒ मयि॑।   अ॒हम॒मृते᳚।   अ॒मृतं॒ ब्रह्म॑णि। (श्रोत्रे)

   आपो॑ मे॒ रेत॑सि श्रि॒ताः।   रेतो॒ हृद॑ये।   हृद॑यं॒ मयि॑।   अ॒हम॒मृते᳚।   अ॒मृतं॒ ब्रह्म॑णि। (गुह्यम्)

   पृ॒थि॒वी मे॒ शरी॑रे श्रि॒ता।   शरी॑र॒ꣳ॒ हृद॑ये।   हृद॑यं॒ मयि॑।   अ॒हम॒मृते᳚।   अ॒मृतं॒ ब्रह्म॑णि। (शरीरम्)

   ओ॒ष॒धि॒व॒न॒स्प॒तयो॑ मे॒ लोम॑सु श्रि॒ताः।   लोमा॑नि॒ हृद॑ये।   हृद॑यं॒ मयि॑।   अ॒हम॒मृते᳚।   अ॒मृतं॒ ब्रह्म॑णि। (लोमानि)

   इन्द्रो॑ मे॒ बले᳚ श्रि॒तः।   बल॒ꣳ॒ हृद॑ये।   हृद॑यं॒ मयि॑।   अ॒हम॒मृते᳚।   अ॒मृतं॒ ब्रह्म॑णि। (बाहू)

   प॒र्जन्यो॑ मे मू॒र्ध्नि श्रि॒तः।   मू॒र्धा हृद॑ये।   हृद॑यं॒ मयि॑।   अ॒हम॒मृते᳚।   अ॒मृतं॒ ब्रह्म॑णि। (शिरः)

   ईशा॑नो मे म॒न्यौ श्रि॒तः।   म॒न्युर्‌हृद॑ये।   हृद॑यं॒ मयि॑।    अ॒हम॒मृते᳚।   अ॒मृतं॒ ब्रह्म॑णि। (हृदयम्)

   आ॒त्मा म॑ आ॒त्मनि॑ श्रि॒तः।   आ॒त्मा हृद॑ये।   हृद॑यं॒ मयि॑।   अ॒हम॒मृते᳚।   अ॒मृतं॒ ब्रह्म॑णि।
(हृदयम्)

   पुन॑र्म आ॒त्मा पुन॒रायु॒रागा᳚त्।   पुनः॑ प्रा॒णः पुन॒राकू॑त॒मागा᳚त्।   वै॒श्वा॒न॒रो र॒श्मिभि॑र्वावृधा॒नः।   अ॒न्तस्ति॑ष्ठत्व॒मृत॑स्य गो॒पाः॥ (सर्वाण्यङ्गानि संस्पृश्य स्थापनं कृत्वा मानसैराराधयेत्॥)
{\small \closesection}

\sect{आत्मपूजा}

\twolineshloka*
{आराधितो मनुष्यैस्त्वं सिद्धैर्देवासुरादिभिः}
{आराधयामि भक्त्या त्वाऽनुग्रहाण महेश्वर}

\sect{कलशेषु साम्बपरमेश्वर ध्यानम्}


\twolineshloka
{ध्यायेन्निरामयं वस्तुं सर्गस्थितिलयादिकम्}
{निर्गुणं निष्कलं नित्यं मनोवाचामगोचरम्}

\twolineshloka
{गङ्गाधरं शशिधरं जटामकुटशोभितम्}
{श्वेतभूतित्रिपुण्ड्रेण विराजितललाटकम्}

\twolineshloka
{लोचनत्रयसम्पन्नं स्वर्णकुण्डलशोभितम्}
{स्मेराननं चतुर्बाहुं मुक्ताहारोपशोभितम्}

\twolineshloka
{अक्षमालां सुधाकुम्भं चिन्मयीं मुद्रिकामपि}
{पुस्तकं च भुजैर्दिव्यैर्दधानं पार्वतीपतिम्}

\twolineshloka
{श्वेताम्बरधरं श्वेतं रत्नसिंहासनस्थितम्}
{सर्वाभीष्टप्रदातारं वटमूलनिवासिनम्}

\threelineshloka
{वामाङ्कसंस्थितां गौरीं बालार्कायुतसन्निभाम्}
{जपाकुसुमसाहस्रसमानश्रियमीश्वरीम्}
{सुवर्णरत्नखचितमकुटेन विराजिताम्}

\twolineshloka
{ललाटपट्टसंराजत्संलग्नतिलकाञ्चिताम्}
{राजीवायतनेत्रान्तां नीलोत्पलदलेक्षणाम्}

\twolineshloka
{सन्तप्तहेमखचित ताटङ्काभरणान्विताम्}
{ताम्बूलचर्वणरतरक्तजिह्वाविराजिताम्}

\twolineshloka
{पताकाभरणोपेतां मुक्ताहारोपशोभिताम्}
{स्वर्णकङ्कणसंयुक्तैश्चतुर्भिर्बाहुभिर्युताम्}

\twolineshloka
{सुवर्णरत्नखचित-काञ्चीदामविराजिताम्}
{कदलीललितस्तम्भ-सन्निभोरुयुगान्विताम्}

\twolineshloka
{श्रिया विराजितपदां भक्तत्राणपरायणाम्}
{अन्योन्याश्लिष्टहृद्बाहू गौरीशङ्करसंज्ञकम्}

\twolineshloka
{सनातनं परं ब्रह्म परमात्मानमव्ययम्}
{आवाहयामि जगतामीश्वरं परमेश्वरम्}

\twolineshloka
{मङ्गलायतनं देवं युवानमति सुन्दरम्}
{ध्यायेत्कल्पतरोर्मूले सुखासीनं सहोमया}

\twolineshloka
{आगच्छाऽऽगच्छ भगवन् देवेश परमेश्वर}
{सच्चिदानन्द भूतेश पार्वतीश नमोऽस्तु ते}

{\small \closesection}


% \sect{षोडशोपचार पूजा}

% नम॑स्ते रुद्र म॒न्यव॑ उ॒तो त॒ इष॑वे॒ नमः॑। नम॑स्ते अस्तु॒ धन्व॑ने बा॒हुभ्या॑मु॒त ते॒ नमः॑॥ ॐ ह्रीं न॒मः शि॒वाय॑। स॒द्योजा॒तं प्र॑पद्यामि।

% या त॒ इषुः॑ शि॒वत॑मा शि॒वं ब॒भूव॑ ते॒ धनुः॑। शि॒वा श॑र॒व्या॑ या तव॒ तया॑ नो रुद्र मृडय॥ ॐ ह्रीं न॒मः शि॒वाय॑। स॒द्योजा॒ताय॒ वै नमो॒ नमः॑। आसनं समर्पयामि॥२॥

% या ते॑ रुद्र शि॒वा त॒नूरघो॒राऽपा॑पकाशिनी। तया॑ नस्त॒नुवा॒ शन्त॑मया॒ गिरि॑शन्ता॒\-भिचा॑कशीहि॥ ॐ ह्रीं न॒मः शि॒वाय॑। भ॒वे भ॑वे॒ नाति॑ भवे भवस्व॒ माम्। पादयोः पाद्यं समर्पयामि॥३॥

% यामिषुं॑ गिरिशन्त॒ हस्ते॒ बिभ॒र्ष्यस्त॑वे। शि॒वां गि॑रित्र॒ तां कु॑रु॒ मा हिꣳ॑सीः॒ पुरु॑षं॒ जग॑त्॥ ॐ ह्रीं न॒मः शि॒वाय॑। भ॒वोद्भ॑वाय॒ नमः॑॥ अर्घ्यं समर्पयामि॥४॥

% शि॒वेन॒ वच॑सा त्वा॒ गिरि॒शाच्छा॑वदामसि। यथा॑ नः॒ सर्व॒मिज्जग॑दय॒क्ष्मꣳ सु॒मना॒ अस॑त्॥ ॐ ह्रीं न॒मः शि॒वाय॑। वा॒म॒दे॒वाय॒ नमः॑। आचमनीयं समर्पयामि॥५॥

% अध्य॑वोचदधिव॒क्ता प्र॑थ॒मो दैव्यो॑ भि॒षक्। अहीꣴ॑श्च॒ सर्वा᳚ञ्ज॒म्भय॒-न्थ्सर्वा᳚श्च यातुधा॒न्यः॑॥ ॐ ह्रीं न॒मः शि॒वाय॑। ज्ये॒ष्ठाय॒ नमः॑। मधुपर्कं समर्पयामि॥६॥

% अ॒सौ यस्ता॒म्रो अ॑रु॒ण उ॒त ब॒भ्रुः सु॑म॒ङ्गलः॑। ये चे॒माꣳ रु॒द्रा अ॒भितो॑ दि॒क्षु श्रि॒ताः स॑हस्र॒शोऽवै॑षा॒ꣳ॒ हेड॑ ईमहे॥ ॐ ह्रीं न॒मः शि॒वाय॑। श्रे॒ष्ठाय॒ नमः॑। स्नानं समर्पयामि। स्नानानन्तरम् आचमनीयं समर्पयामि॥७॥

% अ॒सौ यो॑ऽव॒सर्प॑ति॒ नील॑ग्रीवो॒ विलो॑हितः। उ॒तैनं॑ गो॒पा अ॑दृश॒न्न॒दृ॑शन्नुदहा॒र्यः॑। उ॒तैनं॒ विश्वा॑ भू॒तानि॒ स दृ॒ष्टो मृ॑डयाति नः॥ ॐ ह्रीं न॒मः शि॒वाय॑। रु॒द्राय॒ नमः॑। वस्त्रोत्तरीयं समर्पयामि॥८॥

% नमो॑ अस्तु॒ नील॑ग्रीवाय सहस्रा॒क्षाय॑ मी॒ढुषे᳚। अथो॒ ये अ॑स्य॒ सत्वा॑नो॒ऽहं तेभ्यो॑ऽकरं॒ नमः॑॥ ॐ ह्रीं न॒मः शि॒वाय॑। काला॑य॒ नमः॑। यज्ञोपवीताभरणानि समर्पयामि॥९॥

% प्र मु॑ञ्च॒ धन्व॑न॒स्त्वमु॒भयो॒रार्त्नि॑यो॒र्ज्याम्। याश्च॑ ते॒ हस्त॒ इष॑वः॒ परा॒ ता भ॑गवो वप॥ ॐ ह्रीं न॒मः शि॒वाय॑। कल॑विकरणाय॒ नमः॑। दिव्यपरिमलगन्धान् धारयामि। गन्धस्योपरि अक्षतान् समर्पयामि॥१०॥

% अ॒व॒तत्य॒ धनु॒स्त्वꣳ सह॑स्राक्ष॒ शते॑षुधे। नि॒शीर्य॑ श॒ल्यानां॒ मुखा॑ शि॒वो नः॑ सु॒मना॑ भव॥ ॐ ह्रीं न॒मः शि॒वाय॑। बल॑विकरणाय॒ नमः॑। पुष्पैः पूजयामि॥११॥


% ॐ भवाय देवाय नमः। ॐ शर्वाय देवाय नमः। ॐ ईशानाय देवाय नमः। ॐ पशुपतये देवाय नमः। ॐ रुद्राय देवाय नमः। ॐ उग्राय देवाय नमः। ॐ भीमाय देवाय नमः। ॐ महते देवाय नमः॥
% ॐ भवस्य देवस्य पत्न्यै नमः। ॐ शर्वस्य देवस्य पत्न्यै नमः। ॐ ईशानस्य देवस्य पत्न्यै नमः। ॐ पशुपतेर्देवस्य पत्न्यै नमः। ॐ रुद्रस्य देवस्य पत्न्यै नमः। ॐ उग्रस्य देवस्य पत्न्यै नमः। ॐ भीमस्य देवस्य पत्न्यै नमः। ॐ महतो देवस्य पत्न्यै नमः॥ 

% विज्यं॒ धनुः॑ कप॒र्दिनो॒ विश॑ल्यो॒ बाण॑वाꣳ उ॒त। अने॑शन्न॒\-स्येष॑व आ॒भुर॑स्य निष॒ङ्गथिः॑॥ ॐ ह्रीं न॒मः शि॒वाय॑। बला॑य॒ नमः॑। धूपमाघ्रापयामि॥१२॥

% या ते॑ हे॒तिर्मी॑ढुष्टम॒ हस्ते॑ ब॒भूव॑ ते॒ धनुः॑। तया॒ऽस्मान् वि॒श्वत॒स्त्वम॑य॒क्ष्मया॒ परि॑ब्भुज॥ ॐ ह्रीं न॒मः शि॒वाय॑। बल॑प्रमथनाय॒ नमः॑। अलङ्कारदीपं सन्दर्शयामि॥१३॥

% ॐ भूर्भुवः॒ सुवः॑। + ब्र॒ह्मणे॒ स्वाहा᳚। नम॑स्ते अ॒स्त्वायु॑धा॒याना॑तताय धृ॒ष्णवे᳚। उ॒भाभ्या॑मु॒त ते॒ नमो॑ बा॒हुभ्यां॒ तव॒ धन्व॑ने॥ ॐ ह्रीं न॒मः शि॒वाय॑। सर्व॑भूतदमनाय॒ नमः॑। () निवेदयामि। मध्ये मध्ये अमृतपानीयं समर्पयामि। अमृतापिधानमसि।\\
% हस्तप्रक्षालनं समर्पयामि। पादप्रक्षालनं समर्पयामि। निवेदनानन्तरम् आचमनीयं समर्पयामि॥१४॥

% परि॑ ते॒ धन्व॑नो हे॒तिर॒स्मान्वृ॑णक्तु वि॒श्वतः॑। अथो॒ य इ॑षु॒धिस्तवा॒ऽ॒ऽ॒रे अ॒स्मन्नि धे॑हि॒ तम्॥ ॐ ह्रीं न॒मः शि॒वाय॑। म॒नोन्म॑नाय॒ नमः॑। कर्पूरताम्बूलं समर्पयामि॥१५॥

% नम॑स्ते अस्तु भगवन् विश्वेश्व॒राय॑ महादे॒वाय॑ त्र्यम्ब॒काय॑ त्रिपुरान्त॒काय॑ त्रिकाग्निका॒लाय॑ कालाग्निरु॒द्राय॑ नीलक॒ण्ठाय॑ मृत्युञ्ज॒याय॑ सर्वेश्व॒राय॑ सदाशि॒वाय॑ श्रीमन्महादे॒वाय॒ नमः॑॥ कर्पूरनीराजनं दर्शयामि॥१६॥

% \dnsub{रक्षा}
% बृ॒हथ्साम॑ क्षत्र॒भृद्वृ॒द्ध वृ॑ष्णियं त्रि॒ष्टुभौजः॑ शुभि॒तमु॒ग्रवी॑रम्।
% इन्द्र॒स्तोमे॑न पञ्चद॒शेन॒ मध्य॑मि॒दं वाते॑न॒ सग॑रेण रक्ष॥

% रक्षां धारयामि॥


% {\small \closesection}


% \sect{श्रीरुद्रनाम त्रिशती}
% नमो॒ हिर॑ण्यबाहवे॒ नमः॑। से॒ना॒न्ये॑  नमः॑। \\
% दि॒शां च॒ पत॑ये॒ नमः॑। नमो॑ वृ॒क्षेभ्यो॒ नमः॑। \\
% हरि॑केशेभ्यो॒  नमः॑। प॒शू॒नां पत॑ये॒  नमः॑। \\
% नमः॑ स॒स्पिञ्ज॑राय॒ नमः॑। त्विषी॑मते॒ नमः॑। \\
% प॒थी॒नां पत॑ये॒ नमः॑। नमो॑ बभ्लु॒शाय॒ नमः॑। \\
% वि॒व्या॒धिने॒ नमः॑। अन्ना॑नां॒ पत॑ये॒ नमः॑।  \\
% नमो॒ हरि॑केशाय॒ नमः॑। उ॒प॒वी॒तिने॒ नमः॑। \\
% पु॒ष्टानां॒ पत॑ये॒ नमः॑। नमो॑ भ॒वस्य॑ हे॒त्यै नमः॑। \\
% जग॑तां॒ पत॑ये॒ नमः॑। नमो॑ रु॒द्राय॒ नमः॑। \\
% आ॒त॒ता॒विने॒ नमः॑। क्षेत्रा॑णां॒ पत॑ये॒ नमः॑।  \\
% नमः॑ सू॒ताय॒ नमः॑। अह॑न्त्याय॒ नमः॑। \\
% वना॑नां॒  पत॑ये॒ नमः॑। नमो॒ रोहि॑ताय॒ नमः॑। \\
% स्थ॒पत॑ये नमः॑। वृ॒क्षाणां॒ पत॑ये॒ नमः॑। \\
% नमो॑ म॒न्त्रिणे॒ नमः॑। वा॒णि॒जाय॒ नमः॑। \\
% कक्षा॑णां॒ पत॑ये॒ नमः॑। नमो॑ भुव॒न्तये॒ नमः॑। \\
% वा॒रि॒व॒स्कृ॒ताय॒ नमः॑। ओष॑धीनां॒ पत॑ये॒ नमः॑। \\
% नम॑ उ॒च्चैर्घो॑षाय॒ नमः॑। आ॒क्र॒न्दय॑ते॒ नमः॑। \\
% प॒त्ती॒नां पत॑ये॒ नमः॑। नमः॑ कृथ्स्नवी॒ताय॒ नमः॑। \\
% धाव॑ते॒ नमः॑। सत्त्व॑नां॒ पत॑ये॒ नमः॑॥\\

% नमः॒ सह॑मानाय॒ नमः॑। नि॒व्या॒धिने॒ नमः॑। \\
% आ॒व्या॒धिनी॑नां॒ पत॑ये॒ नमः॑। नमः॑ ककु॒भाय॒ नमः॑। \\
% नि॒ष॒ङ्गिणे॒ नमः॑। स्ते॒नानां॒ पत॑ये॒ नमः॑। \\
% नमो॑ निष॒ङ्गिणे॒ नमः॑। इ॒षु॒धि॒मते॒ नमः॑। \\
% तस्क॑राणां॒ पत॑ये॒ नमः॑। नमो॒ वञ्च॑ते॒ नमः॑। \\
% प॒रि॒वञ्च॑ते॒ नमः॑। स्ता॒यू॒नां पत॑ये॒ नमः॑। \\
% नमो॑ निचे॒रवे॒ नमः॑। प॒रि॒च॒राय॒ नमः॑। \\
% अर॑ण्यानां॒ पत॑ये॒ नमः॑। नमः॑ सृका॒विभ्यो॒ नमः॑। \\
% जिघाꣳ॑सद्भ्यो॒ नमः॑। मु॒ष्ण॒तां पत॑ये॒ नमः॑। \\
% नमो॑ऽसि॒मद्भ्यो॒ नमः॑। नक्तं॒ चर॑द्भ्यो॒ नमः॑। \\
% प्र॒कृ॒न्तानां॒ पत॑ये॒ नमः॑। नम॑ उष्णी॒षिने॒ नमः॑। \\
% गि॒रि॒च॒राय॒ नमः॑। कु॒लु॒ञ्चानां॒ पत॑ये॒  नमः॑। \\
% नम॒ इषु॑मद्भ्यो॒  नमः॑। ध॒न्वा॒विभ्य॑श्च॒  नमः॑। वो॒  नमः॑। \\
% नम॑ आतन्वा॒नेभ्यो॒ नमः॑। प्र॒ति॒दधा॑नेभ्यश्च॒  नमः॑। वो॒ नमः॑।\\
% नम॑ आ॒यच्छ॑द्भ्यो॒  नमः॑।  वि॒सृ॒जद्भ्य॑श्च॒  नमः॑।  वो॒ नमः॑।\\
% नमोऽस्य॑द्भ्यो॒  नमः॑। विध्य॑द्भ्यश्च॒  नमः॑।  वो॒ नमः॑।\\
% नम॒ आसी॑नेभ्यो॒  नमः॑। शया॑नेभ्यश्च॒  नमः॑। वो॒ नमः॑।\\
% नमः॑ स्व॒पद्भ्यो॒  नमः॑। जाग्र॑द्भ्यश्च॒  नमः॑।  वो॒ नमः॑।\\
% नम॒स्तिष्ठ॑द्भ्यो॒  नमः॑। धाव॑द्भ्यश्च॒  नमः॑।  वो॒ नमः॑।\\
% नमः॑ स॒भाभ्यो॒  नमः॑। स॒भाप॑तिभ्यश्च॒ नमः॑। वो॒ नमः॑।\\
% नमो॒ अश्वे᳚भ्यो॒  नमः॑। अश्व॑पतिभ्यश्च॒  नमः॑। वो॒ नमः॑॥\\


% नम॑ आव्या॒धिनी᳚भ्यो॒  नमः॑। वि॒विध्य॑न्तीभ्यश्च॒  नमः॑।  वो॒ नमः॑। \\
% नम॒ उग॑णाभ्यो॒ नमः॑। तृ॒ꣳ॒ह॒तीभ्य॑श्च॒ नमः॑। वो॒ नमः॑। \\
% नमो॑ गृ॒थ्सेभ्यो॒ नमः॑। गृ॒थ्सप॑तिभ्यश्च॒ नमः॑। वो॒ नमः॑। \\
% नमो॒ व्राते᳚भ्यो॒ नमः॑। व्रात॑पतिभ्यश्च॒ नमः॑। वो॒ नमः॑। \\
% नमो॑ ग॒णेभ्यो॒ नमः॑। ग॒णप॑तिभ्यश्च॒ नमः॑। वो॒ नमः॑।\\
% नमो॒ विरू॑पेभ्यो॒ नमः॑। वि॒श्वरू॑पेभ्यश्च॒ नमः॑। वो॒ नमः॑। \\
% नमो॑ म॒हद्भ्यो॒ नमः॑। क्षु॒ल्ल॒केभ्य॑श्च॒ नमः॑। वो॒ नमः॑।\\
% नमो॑ र॒थिभ्यो॒ नमः॑। अ॒र॒थेभ्य॑श्च॒ नमः॑। वो॒ नमः॑। \\
% नमो॒ रथे᳚भ्यो॒ नमः॑। रथ॑पतिभ्यश्च॒ नमः॑। वो॒ नमः॑।\\
% नमः॒ सेना᳚भ्यो॒ नमः॑। से॒ना॒निभ्य॑श्च॒ नमः॑। वो॒ नमः॑। \\
% नमः॑ क्ष॒त्तृभ्यो॒ नमः॑। स॒ङ्ग्र॒ही॒तृभ्य॑श्च॒ नमः॑। वो॒ नमः॑। \\
% नम॒स्तक्ष॑भ्यो॒ नमः॑। र॒थ॒का॒रेभ्य॑श्च॒ नमः॑। वो॒ नमः॑। \\
% नमः॒ कुला॑लेभ्यो॒ नमः॑। क॒र्मारे᳚भ्यश्च॒ नमः॑। वो॒ नमः॑। \\
% नमः॑ पु॒ञ्जिष्टे᳚भ्यो॒ नमः॑। नि॒षा॒देभ्य॑श्च॒ नमः॑। वो॒ नमः॑। \\
% नम॑ इषु॒कृद्भ्यो॒ नमः॑। ध॒न्व॒कृद्भ्य॑श्च॒ नमः॑। वो॒ नमः॑।\\
% नमो॑ मृग॒युभ्यो॒ नमः॑। श्व॒निभ्य॑श्च॒ नमः॑। वो॒ नमः॑। \\
% नमः॒ श्वभ्यो॒ नमः॑। श्वप॑तिभ्यश्च॒ नमः॑। वो॒ नमः॑॥ \\


% नमो॑ भ॒वाय॑ च॒ नमः॑। रु॒द्राय॑ च॒ नमः॑। \\
% नमः॑ श॒र्वाय॑ च॒ नमः॑। प॒शु॒पत॑ये च॒ नमः॑।\\
% नमो॒ नील॑ग्रीवाय च॒ नमः॑। शि॒ति॒कण्ठा॑य च॒ नमः॑। \\
% नमः॑ कप॒र्दिने॑ च॒ नमः॑। व्यु॑प्तकेशाय च॒ नमः॑।\\
% नमः॑ सहस्रा॒क्षाय॑ च॒ नमः॑। श॒तध॑न्वने च॒ नमः॑। \\
% नमो॑ गिरि॒शाय॑ च॒ नमः॑। शि॒पि॒वि॒ष्टाय॑ च॒ नमः॑।\\
% नमो॑ मी॒ढुष्ट॑माय च॒ नमः॑। इषु॑मते च॒ नमः॑। \\
% नमो᳚ ह्र॒स्वाय॑ च॒ नमः॑। वा॒म॒नाय॑ च॒ नमः॑।\\
% नमो॑ बृह॒ते च॒ नमः॑। वर्षी॑यसे च॒ नमः॑। \\
% नमो॑ वृ॒द्धाय॑ च॒ नमः॑। सं॒वृध्व॑ने च॒ नमः॑। \\
% नमो॒ अग्रि॑याय च॒ नमः॑। प्र॒थ॒माय॑ च॒ नमः॑। \\
% नम॑ आ॒शवे॑ च॒ नमः॑। अ॒जि॒राय॑ च॒ नमः॑।\\
% नमः॒ शीघ्रि॑याय च॒ नमः॑। शीभ्या॑य च॒ नमः॑। \\
% नम॑ ऊ॒र्म्या॑य च॒ नमः॑। अ॒व॒स्व॒न्या॑य च॒ नमः॑। \\
% नमः॑ स्त्रोत॒स्या॑य च॒ नमः॑। द्वीप्या॑य च॒ नमः॑॥\\


% नमो᳚ ज्ये॒ष्ठाय॑ च॒ नमः॑। क॒नि॒ष्ठाय॑ च॒ नमः॑। \\
% नमः॑ पूर्व॒जाय॑ च॒ नमः॑। अ॒प॒र॒जाय॑ च॒ नमः॑। \\
% नमो॑ मध्य॒माय॑ च॒ नमः॑। अ॒प॒ग॒ल्भाय॑ च॒ नमः॑। \\
% नमो॑ जघ॒न्या॑य च॒ नमः॑। बुध्नि॑याय च॒ नमः॑।\\
% नमः॑ सो॒भ्या॑य च॒ नमः॑। प्र॒ति॒स॒र्या॑य च॒ नमः॑। \\
% नमो॒ याम्या॑य च॒ नमः॑। क्षेम्या॑य च॒ नमः॑। \\
% नम॑ उर्व॒र्या॑य च॒ नमः॑। खल्या॑य च॒ नमः॑। \\
% नमः॒ श्लोक्या॑य च॒ नमः॑। अ॒व॒सा॒न्या॑य च॒ नमः॑। \\
% नमो॒ वन्या॑य च॒ नमः॑। कक्ष्या॑य च॒ नमः॑। \\
% नमः॑ श्र॒वाय॑ च॒ नमः॑। प्र॒ति॒श्र॒वाय॑ च॒ नमः॑। \\
% नम॑ आ॒शुषे॑णाय च॒ नमः॑। आ॒शुर॑थाय च॒ नमः॑। \\
% नमः॒ शूरा॑य च॒ नमः॑। अ॒व॒भि॒न्द॒ते च॒ नमः॑। \\
% नमो॑ व॒र्मिणे॑ च॒ नमः॑। व॒रू॒थिने॑ च॒ नमः॑। \\
% नमो॑ बि॒ल्मिने॑ च॒ नमः॑। क॒व॒चिने॑ च॒ नमः॑। \\
% नमः॑ श्रु॒ताय॑ च॒ नमः॑। श्रु॒त॒से॒नाय॑ च॒ नमः॑॥ \\


% नमो॑ दुन्दु॒भ्या॑य च॒ नमः॑। आ॒ह॒न॒न्या॑य च॒ नमः॑। \\
% नमो॑ धृ॒ष्णवे॑ च॒ नमः॑। प्र॒मृ॒शाय॑ च॒ नमः॑।\\
% नमो॑ दू॒ताय॑ च॒ नमः॑। प्रहि॑ताय च॒ नमः॑। \\
% नमो॑ निष॒ङ्गिणे॑ च॒ नमः॑। इ॒षु॒धि॒मते॑ च॒ नमः॑।\\
% नम॑स्ती॒क्ष्णेष॑वे च॒ नमः॑। आ॒यु॒धिने॑ च॒ नमः॑। \\
% नमः॑ स्वायु॒धाय॑ च॒ नमः॑। सु॒धन्व॑ने च॒ नमः॑।\\
% नमः॒ स्रुत्या॑य च॒ नमः॑। पथ्या॑य च॒ नमः॑। \\
% नमः॑ का॒ट्या॑य च॒ नमः॑। नी॒प्या॑य च॒ नमः॑।\\
% नमः॒ सूद्या॑य च॒ नमः॑। स॒र॒स्या॑य च॒ नमः॑। \\
% नमो॑ ना॒द्याय॑ च॒ नमः॑। वै॒श॒न्ताय॑ च॒ नमः॑।  \\
% नमः॒ कूप्या॑य च॒ नमः॑। अ॒व॒ट्या॑य च॒ नमः॑। \\
% नमो॒ वर्ष्या॑य च॒ नमः॑। अ॒व॒र्ष्याय॑ च॒ नमः॑। \\
% नमो॑ मे॒घ्या॑य च॒ नमः॑। वि॒द्यु॒त्या॑य च॒ नमः॑।\\
% नम॑ ई॒ध्रिया॑य च॒ नमः॑। आ॒त॒प्या॑य च॒ नमः॑।\\
% नमो॒ वात्या॑य च॒ नमः॑। रेष्मि॑याय च॒ नमः॑। \\
% नमो॑ वास्त॒व्या॑य च॒ नमः॑। वा॒स्तु॒पाय॑ च॒ नमः॑॥ \\


% नमः॒ सोमा॑य च॒ नमः॑। रु॒द्राय॑ च॒ नमः॑। \\
% नम॑स्ता॒म्राय॑ च॒ नमः॑। अ॒रु॒णाय॑ च॒ नमः॑।\\
% नमः॑ श॒ङ्गाय॑ च॒ नमः॑। प॒शु॒पत॑ये च॒ नमः॑। \\
% नम॑ उ॒ग्राय॑ च॒ नमः॑। भी॒माय॑ च॒ नमः॑। \\
% नमो॑ अग्रेव॒धाय॑ च॒ नमः॑। दू॒रे॒व॒धाय॑ च॒ नमः॑।\\
% नमो॑ ह॒न्त्रे च॒ नमः॑। हनी॑यसे च॒ नमः॑। \\
% नमो॑ वृ॒क्षेभ्यो॒ नमः॑। हरि॑केशेभ्यो॒ नमः॑।\\
% नम॑स्ता॒राय॒ नमः॑। नमः॑ श॒म्भवे॑ च॒ नमः॑। \\
% म॒यो॒भवे॑ च॒ नमः॑। नमः॑ शङ्क॒राय॑ च॒ नमः॑। \\
% म॒य॒स्क॒राय॑ च॒ नमः॑। नमः॑ शि॒वाय॑  च॒ नमः॑। \\
% शि॒वत॑राय च॒ नमः॑। नम॒स्तीर्थ्या॑य च॒ नमः॑। \\
% कूल्या॑य च॒ नमः॑। नमः॑ पा॒र्या॑य च॒ नमः॑। \\
% अ॒वा॒र्या॑य च॒ नमः॑। नमः॑ प्र॒तर॑णाय च॒ नमः॑। \\
% उ॒त्तर॑णाय च॒ नमः॑। नम॑ आता॒र्या॑य च॒ नमः॑। \\
% आ॒ला॒द्या॑य च॒ नमः॑। नमः॒ शष्प्या॑य च॒ नमः॑। \\
% फेन्या॑य च॒ नमः॑। नमः॑ सिक॒त्या॑य च॒ नमः॑। \\
% प्र॒वा॒ह्या॑य च॒ नमः॑॥ \\


% नम॑ इरि॒ण्या॑य च॒ नमः॑। प्र॒प॒थ्या॑य च॒ नमः॑। \\
% नमः॑ किꣳशि॒लाय॑ च॒ नमः॑। क्षय॑णाय च॒ नमः॑। \\
% नमः॑ कप॒र्दिने॑ च॒ नमः॑। पु॒ल॒स्तये॑ च॒ नमः॑।\\
% नमो॒ गोष्ठ्या॑य च॒ नमः॑। गृह्या॑य च॒ नमः॑। \\
% नम॒स्तल्प्या॑य च॒ नमः॑। गेह्या॑य च॒ नमः॑। \\
% नमः॑ का॒ट्या॑य च॒ नमः॑। ग॒ह्व॒रे॒ष्ठाय॑ च॒ नमः॑।\\
% नमो᳚ ह्रद॒य्या॑य च॒ नमः॑। नि॒वे॒ष्प्या॑य च॒ नमः॑। \\
% नमः॑ पाꣳस॒व्या॑य च॒ नमः॑। र॒ज॒स्या॑य च॒ नमः॑।\\
% नमः॒ शुष्क्या॑य च॒ नमः॑। ह॒रि॒त्या॑य च॒ नमः॑। \\
% नमो॒ लोप्या॑य च॒ नमः॑। उ॒ल॒प्या॑य च॒ नमः॑। \\
% नम॑ ऊ॒र्व्या॑य च॒ नमः॑। सू॒र्म्या॑य च॒ नमः॑। \\
% नमः॑ प॒र्ण्या॑य च॒ नमः॑। प॒र्ण॒श॒द्या॑य च॒ नमः॑। \\
% नमो॑ऽपगु॒रमा॑णाय च॒ नमः॑। अ॒भि॒घ्न॒ते च॒ नमः॑। \\
% नम॑ आख्खिद॒ते च॒ नमः॑। प्र॒ख्खि॒द॒ते च॒ नमः॑। नमो॑ वो॒ नमः॑। \\
% कि॒रि॒केभ्यो॒ नमः॑। दे॒वाना॒ꣳ॒ हृद॑येभ्यो॒ नमः॑। \\
% नमो॑ विक्षीण॒केभ्यो॒ नमः॑। नमो॑ विचिन्व॒त्केभ्यो॒ नमः॑। \\
% नम॑ आनिर्‌ह॒तेभ्यो॒ नमः॑। नम॑ आमीव॒त्केभ्यो॒ नमः॑। \\
% {\small \closesection}

\sect{प्रदक्षिणम्}

द्रापे॒ अन्ध॑सस्पते॒ दरि॑द्र॒न्नील॑लोहित। ए॒षां पुरु॑षाणामे॒षां प॑शू॒नां मा भेर्माऽरो॒ मो ए॑षां॒ किं च॒नाऽऽम॑मत्॥ या ते॑ रुद्र शि॒वा त॒नूः शि॒वा वि॒श्वाह॑भेषजी। शि॒वा रु॒द्रस्य॑ भेष॒जी तया॑ नो मृड जी॒वसे᳚॥ इ॒माꣳ रु॒द्राय॑ त॒वसे॑ कप॒र्दिने᳚ क्ष॒यद्वी॑राय॒ प्रभ॑रामहे म॒तिम्॥ यथा॑ नः॒ शमस॑द्द्वि॒पदे॒ चतु॑ष्पदे॒ विश्वं॑ पु॒ष्टं ग्रामे॑ अ॒स्मिन्नना॑तुरम्॥ मृ॒डा नो॑ रुद्रो॒त नो॒ मय॑स्कृधि क्ष॒यद्वी॑राय॒ नम॑सा विधेम ते। यच्छं च॒ योश्च॒ मनु॑राय॒जे पि॒ता तद॑श्याम॒ तव॑ रुद्र॒ प्रणी॑तौ॥ मा नो॑ म॒हान्त॑मु॒त मा नो॑ अर्भ॒कं मा न॒ उक्ष॑न्तमु॒त मा न॑ उक्षि॒तम्। मा नो॑ वधीः पि॒तरं॒ मोत मा॒तरं॑ प्रि॒या मा न॑स्त॒नुवो॑ रुद्र रीरिषः॥ मा न॑स्तो॒के तन॑ये॒ मा न॒ आयु॑षि॒ मा नो॒ गोषु॒ मा नो॒ अश्वे॑षु रीरिषः। वी॒रान्मा नो॑ रुद्र भामि॒तोऽव॑धीर्‌ह॒विष्म॑न्तो॒ नम॑सा विधेम ते॥ आ॒रात्ते॑ गो॒घ्न उ॒त पू॑रुष॒घ्ने क्ष॒यद्वी॑राय सु॒म्नम॒स्मे ते॑ अस्तु। रक्षा॑ च नो॒ अधि॑ च देव ब्रू॒ह्यधा॑ च नः॒ शर्म॑ यच्छ द्वि॒बर्‌हाः᳚॥ स्तु॒हि श्रु॒तं ग॑र्त॒सदं॒ युवा॑नं मृ॒गं न भी॒ममु॑पह॒त्नुमु॒ग्रम्। मृ॒डा ज॑रि॒त्रे रु॑द्र॒ स्तवा॑नो अ॒न्यन्ते॑ अ॒स्मन्निव॑पन्तु॒ सेनाः᳚॥ परि॑णो रु॒द्रस्य॑ हे॒तिर्वृ॑णक्तु॒ परि॑ त्वे॒षस्य॑ दुर्म॒तिर॑घा॒योः। अव॑ स्थि॒रा म॒घव॑द्भ्यस्तनुष्व॒ मीढ्व॑स्तो॒काय॒ तन॑याय मृडय॥ मीढु॑ष्टम॒ शिव॑तम शि॒वो नः॑ सु॒मना॑ भव। प॒र॒मे वृ॒क्ष आयु॑धं नि॒धाय॒ कृत्तिं॒ वसा॑न॒ आ च॑र॒ पिना॑कं॒ बिभ्र॒दा ग॑हि॥ विकि॑रिद॒ विलो॑हित॒ नम॑स्ते अस्तु भगवः। यास्ते॑ स॒हस्रꣳ॑ हे॒तयो॒ऽन्यम॒स्मन्निव॑पन्तु॒ ताः॥ स॒हस्रा॑णि सहस्र॒धा बा॑हु॒वोस्तव॑ हे॒तयः॑। तासा॒मीशा॑नो भगवः परा॒चीना॒ मुखा॑ कृधि॥
प्रदक्षिणं कृत्वा॥


{\small \closesection}


\sect{नमस्काराः}

स॒हस्रा॑णि सहस्र॒शो ये रु॒द्रा अधि॒ भूम्या᳚म्। तेषाꣳ॑ सहस्रयोज॒ने\-ऽव॒धन्वा॑नि तन्मसि। श्री महादेवादिभ्यो नमः॥१॥

अ॒स्मिन् म॑ह॒त्य॑र्ण॒वे᳚ऽन्तरि॑क्षे भ॒वा अधि॑। तेषाꣳ॑ सहस्रयोज॒ने\-ऽव॒धन्वा॑नि तन्मसि। श्री महादेवादिभ्यो नमः॥२॥

नील॑ग्रीवाः शिति॒कण्ठाः᳚ श॒र्वा अ॒धः, क्ष॑माच॒राः। तेषाꣳ॑ सहस्रयोज॒ने\-ऽव॒धन्वा॑नि तन्मसि। श्री महादेवादिभ्यो नमः॥३॥

नील॑ग्रीवाः शिति॒कण्ठा॒ दिवꣳ॑ रु॒द्रा उप॑श्रिताः। तेषाꣳ॑ सहस्रयोज॒ने\-ऽव॒धन्वा॑नि तन्मसि। श्री महादेवादिभ्यो नमः॥४॥

ये वृ॒क्षेषु॑ स॒स्पिञ्ज॑रा॒ नील॑ग्रीवा॒ विलो॑हिताः। तेषाꣳ॑ सहस्रयोज॒ने\-ऽव॒धन्वा॑नि तन्मसि। श्री महादेवादिभ्यो नमः॥५॥

ये भू॒ताना॒मधि॑पतयो विशि॒खासः॑ कप॒र्दिनः॑। तेषाꣳ॑ सहस्रयोज॒ने\-ऽव॒धन्वा॑नि तन्मसि। श्री महादेवादिभ्यो नमः॥६॥

ये अन्ने॑षु वि॒विध्य॑न्ति॒ पात्रे॑षु॒ पिब॑तो॒ जना\sn। तेषाꣳ॑ सहस्रयोज॒ने\-ऽव॒धन्वा॑नि तन्मसि। श्री महादेवादिभ्यो नमः॥७॥

ये प॒थां प॑थि॒रक्ष॑य ऐलबृ॒दा य॒व्युधः॑। तेषाꣳ॑ सहस्रयोज॒ने\-ऽव॒धन्वा॑नि तन्मसि। श्री महादेवादिभ्यो नमः॥८॥

ये ती॒र्थानि॑ प्र॒चर॑न्ति सृ॒काव॑न्तो निष॒ङ्गिणः॑। तेषाꣳ॑ सहस्रयोज॒ने\-ऽव॒धन्वा॑नि तन्मसि। श्री महादेवादिभ्यो नमः॥९॥

य ए॒ताव॑न्तश्च॒ भूयाꣳ॑सश्च॒ दिशो॑ रु॒द्रा वि॑तस्थि॒रे। तेषाꣳ॑ सहस्रयोज॒ने\-ऽव॒धन्वा॑नि तन्मसि। श्री महादेवादिभ्यो नमः॥१०॥

नमो॑ रु॒द्रेभ्यो॒ ये पृ॑थि॒व्यां येषा॒मन्न॒मिष॑व॒स्तेभ्यो॒ दश॒ प्राची॒र्दश॑ दक्षि॒णा दश॑ प्र॒तीची॒र्दशो\-दी॑ची॒र्दशो॒र्ध्वास्तेभ्यो॒ नम॒स्ते नो॑ मृडयन्तु॒ ते यं द्वि॒ष्मो यश्च॑ नो॒ द्वेष्टि॒ तं वो॒ जम्भे॑ दधामि। श्री महादेवादिभ्यो नमः॥११॥ 

नमो॑ रु॒द्रेभ्यो॒ ये᳚ऽन्तरि॑क्षे॒ येषां॒ वात॒ इष॑व॒स्तेभ्यो॒ दश॒ प्राची॒र्दश॑ दक्षि॒णा दश॑ प्र॒तीची॒र्दशो\-दी॑ची॒र्दशो॒र्ध्वास्तेभ्यो॒ नम॒स्ते नो॑ मृडयन्तु॒ ते यं द्वि॒ष्मो यश्च॑ नो॒ द्वेष्टि॒ तं वो॒ जम्भे॑ दधामि। श्री महादेवादिभ्यो नमः॥१२॥ 

नमो॑ रु॒द्रेभ्यो॒ ये दि॒वि येषां᳚ व॒\ar षमिष॑व॒स्तेभ्यो॒ दश॒ प्राची॒र्दश॑ दक्षि॒णा दश॑ प्र॒तीची॒र्दशो\-दी॑ची॒र्दशो॒र्ध्वास्तेभ्यो॒ नम॒स्ते नो॑ मृडयन्तु॒ ते यं द्वि॒ष्मो यश्च॑ नो॒ द्वेष्टि॒ तं वो॒ जम्भे॑ दधामि। श्री महादेवादिभ्यो नमः॥१३॥ 

नमस्कारान् कृत्वा॥

{\small \closesection}

\sect{चमकानुवाकैः प्रार्थना}

अग्ना॑विष्णू स॒जोष॑से॒मा व॑र्धन्तु वां॒ गिरः॑। द्यु॒म्नैर्वाजे॑भि॒रा\-ग॑तम्॥ 
वाज॑श्च मे प्रस॒वश्च॑ मे॒ प्रय॑तिश्च मे॒ प्रसि॑तिश्च मे धी॒तिश्च॑ मे॒ क्रतु॑श्च मे॒ स्वर॑श्च मे॒ श्लोक॑श्च मे श्रा॒वश्च॑ मे॒ श्रुति॑श्च मे॒ ज्योति॑श्च मे॒ सुव॑श्च मे प्रा॒णश्च॑ मेऽपा॒नश्च॑ मे व्या॒नश्च॒ मेऽसु॑श्च मे चि॒त्तं च॑ म॒ आधी॑तं च मे॒ वाक्च॑ मे॒ मन॑श्च मे॒ चक्षु॑श्च मे॒ श्रोत्रं॑ च मे॒ दक्ष॑श्च मे॒ बलं॑ च म॒ ओज॑श्च मे॒ सह॑श्च म॒ आयु॑श्च मे ज॒रा च॑ म आ॒त्मा च॑ मे त॒नूश्च॑ मे॒ शर्म॑ च मे॒ वर्म॑ च॒ मेऽङ्गा॑नि च मे॒ऽस्थानि॑ च मे॒ परूꣳ॑षि च मे॒ शरी॑राणि च मे॥१॥ 

ज्यैष्ठ्यं॑ च म॒ आधि॑पत्यं च मे म॒न्युश्च॑ मे॒ भाम॑श्च॒ मेऽम॑श्च॒ मेऽम्भ॑श्च मे जे॒मा च॑ मे महि॒मा च॑ मे वरि॒मा च॑ मे प्रथि॒मा च॑ मे व॒र्ष्मा च॑ मे द्राघु॒या च॑ मे वृ॒द्धं च॑ मे॒ वृद्धि॑श्च मे स॒त्यं च॑ मे श्र॒द्धा च॑ मे॒ जग॑च्च मे॒ धनं॑ च मे॒ वश॑श्च मे॒ त्विषि॑श्च मे क्री॒डा च॑ मे॒ मोद॑श्च मे जा॒तं च॑ मे जनि॒ष्यमा॑णं च मे सू॒क्तं च॑ मे सुकृ॒तं च॑ मे वि॒त्तं च॑ मे॒ वेद्यं॑ च मे भू॒तं च॑ मे भवि॒ष्यच्च॑ मे सु॒गं च॑ मे सु॒पथं॑ च म ऋ॒द्धं च॑ म॒ ऋद्धि॑श्च मे कॢ॒प्तं च॑ मे॒ कॢप्ति॑श्च मे म॒तिश्च॑ मे सुम॒तिश्च॑ मे॥२॥ 

शं च॑ मे॒ मय॑श्च मे प्रि॒यं च॑ मेऽनुका॒मश्च॑ मे॒ काम॑श्च मे सौमन॒सश्च॑ मे भ॒द्रं च॑ मे॒ श्रेय॑श्च मे॒ वस्य॑श्च मे॒ यश॑श्च मे॒ भग॑श्च मे॒ द्रवि॑णं च मे य॒न्ता च॑ मे ध॒र्ता च॑ मे॒ क्षेम॑श्च मे॒ धृति॑श्च मे॒ विश्वं॑ च मे॒ मह॑श्च मे सं॒विच्च॑ मे॒ ज्ञात्रं॑ च मे॒ सूश्च॑ मे प्र॒सूश्च॑ मे॒ सीरं॑ च मे ल॒यश्च॑ म ऋ॒तं च॑ मे॒ऽमृतं॑ च मेऽय॒क्ष्मं च॒ मेऽना॑मयच्च मे जी॒वातु॑श्च मे दीर्घायु॒त्वं च॑ मेऽनमि॒त्रं च॒ मेऽभ॑यं च मे सु॒गं च॑ मे॒ शय॑नं च मे सू॒षा च॑ मे सु॒दिनं॑ च मे॥३॥ 

ऊर्क्च॑ मे सू॒नृता॑ च मे॒ पय॑श्च मे॒ रस॑श्च मे घृ॒तं च॑ मे॒ मधु॑ च मे॒ सग्धि॑श्च मे॒ सपी॑तिश्च मे कृ॒षिश्च॑ मे॒ वृष्टि॑श्च मे॒ जैत्रं॑ च म॒ औद्भि॑द्यं च मे र॒यिश्च॑ मे॒ राय॑श्च मे पु॒ष्टं च॑ मे॒ पुष्टि॑श्च मे वि॒भु च॑ मे प्र॒भु च॑ मे ब॒हु च॑ मे॒ भूय॑श्च मे पू॒र्णं च॑ मे पू॒र्णत॑रं च॒ मेऽक्षि॑तिश्च मे॒ कूय॑वाश्च॒ मेऽन्नं॑ च॒ मेऽक्षु॑च्च मे व्री॒हय॑श्च मे॒ यवा᳚श्च मे॒ माषा᳚श्च मे॒ तिला᳚श्च मे मु॒द्गाश्च॑ मे ख॒ल्वा᳚श्च मे गो॒धूमा᳚श्च मे म॒सुरा᳚श्च मे प्रि॒यङ्ग॑वश्च॒ मेऽण॑वश्च मे श्या॒माका᳚श्च मे नी॒वारा᳚श्च मे॥४॥ 

अश्मा॑ च मे॒ मृत्ति॑का च मे गि॒रय॑श्च मे॒ पर्व॑ताश्च मे॒ सिक॑ताश्च मे॒ वन॒स्पत॑यश्च मे॒ हिर॑ण्यं च॒ मेऽय॑श्च मे॒ सीसं॑ च मे॒ त्रपु॑श्च मे श्या॒मं च॑ मे लो॒हं च॑ मे॒ऽग्निश्च॑ म॒ आप॑श्च मे वी॒रुध॑श्च म॒ ओष॑धयश्च मे कृष्टप॒च्यं च॑ मेऽकृष्टप॒च्यं च॑ मे ग्रा॒म्याश्च॑ मे प॒शव॑ आर॒ण्याश्च॑ य॒ज्ञेन॑ कल्पन्तां वि॒त्तं च॑ मे॒ वित्ति॑श्च मे भू॒तं च॑ मे॒ भूति॑श्च मे॒ वसु॑ च मे वस॒तिश्च॑ मे॒ कर्म॑ च मे॒ शक्ति॑श्च॒ मेऽर्थ॑श्च म॒ एम॑श्च म॒ इति॑श्च मे॒ गति॑श्च मे॥५॥ 

अ॒ग्निश्च॑ म॒ इन्द्र॑श्च मे॒ सोम॑श्च म॒ इन्द्र॑श्च मे सवि॒ता च॑ म॒ इन्द्र॑श्च मे॒ सर॑स्वती च म॒ इन्द्र॑श्च मे पू॒षा च॑ म॒ इन्द्र॑श्च मे॒ बृह॒स्पति॑श्च म॒ इन्द्र॑श्च मे मि॒त्रश्च॑ म॒ इन्द्र॑श्च मे॒ वरु॑णश्च म॒ इन्द्र॑श्च मे॒ त्वष्टा॑ च म॒ इन्द्र॑श्च मे धा॒ता च॑ म॒ इन्द्र॑श्च मे॒ विष्णु॑श्च म॒ इन्द्र॑श्च मे॒ऽश्विनौ॑ च म॒ इन्द्र॑श्च मे म॒रुत॑श्च म॒ इन्द्र॑श्च मे॒ विश्वे॑ च मे दे॒वा इन्द्र॑श्च मे पृथि॒वी च॑ म॒ इन्द्र॑श्च मे॒ऽन्तरि॑क्षं च म॒ इन्द्र॑श्च मे॒ द्यौश्च॑ म॒ इन्द्र॑श्च मे॒ दिश॑श्च म॒ इन्द्र॑श्च मे मू॒र्धा च॑ म॒ इन्द्र॑श्च मे प्र॒जाप॑तिश्च म॒ इन्द्र॑श्च मे॥६॥ 

अ॒ꣳ॒शुश्च॑ मे र॒श्मिश्च॒ मेऽदा᳚भ्यश्च॒ मेऽधि॑पतिश्च म उपा॒ꣳ॒शुश्च॑ मेऽन्तर्या॒मश्च॑ म ऐन्द्रवाय॒वश्च॑ मे मैत्रावरु॒णश्च॑ म आश्वि॒नश्च॑ मे प्रतिप्र॒स्थान॑श्च मे शु॒क्रश्च॑ मे म॒न्थी च॑ म आग्रय॒णश्च॑ मे वैश्वदे॒वश्च॑ मे ध्रु॒वश्च॑ मे वैश्वान॒रश्च॑ म ऋतुग्र॒हाश्च॑ मेऽतिग्रा॒ह्या᳚श्च म ऐन्द्रा॒ग्नश्च॑ मे वैश्वदे॒वश्च॑ मे मरुत्व॒तीया᳚श्च मे माहे॒न्द्रश्च॑ म आदि॒त्यश्च॑ मे सावि॒त्रश्च॑ मे सारस्व॒तश्च॑ मे पौ॒ष्णश्च॑ मे पात्नीव॒तश्च॑ मे हारियोज॒नश्च॑ मे॥७॥ 

इ॒ध्मश्च॑ मे ब॒र्हिश्च॑ मे॒ वेदि॑श्च मे॒ धिष्णि॑याश्च मे॒ स्रुच॑श्च मे चम॒साश्च॑ मे॒ ग्रावा॑णश्च मे॒ स्वर॑वश्च म उपर॒वाश्च॑ मेऽधि॒षव॑णे च मे द्रोणकल॒शश्च॑ मे वाय॒व्या॑नि च मे पूत॒भृच्च॑ म आधव॒नीय॑श्च म॒ आग्नी᳚ध्रं च मे हवि॒र्धानं॑ च मे गृ॒हाश्च॑ मे॒ सद॑श्च मे पुरो॒डाशा᳚श्च मे पच॒ताश्च॑ मेऽवभृ॒थश्च॑ मे स्वगाका॒रश्च॑ मे॥८॥ 

अ॒ग्निश्च॑ मे घ॒र्मश्च॑ मे॒ऽर्कश्च॑ मे॒ सूर्य॑श्च मे प्रा॒णश्च॑ मेऽश्वमे॒धश्च॑ मे पृथि॒वी च॒ मेऽदि॑तिश्च मे॒ दिति॑श्च मे॒ द्यौश्च॑ मे॒ शक्व॑रीर॒ङ्गुल॑यो॒ दिश॑श्च मे य॒ज्ञेन॑ कल्पन्ता॒मृक्च॑ मे॒ साम॑ च मे॒ स्तोम॑श्च मे॒ यजु॑श्च मे दी॒क्षा च॑ मे॒ तप॑श्च म ऋ॒तुश्च॑ मे व्र॒तं च॑ मेऽहोरा॒त्रयो᳚र्वृ॒ष्ट्या बृ॑हद्रथन्त॒रे च॑ मे य॒ज्ञेन॑ कल्पेताम्॥९॥ 

गर्भा᳚श्च मे व॒थ्साश्च॑ मे॒ त्र्यवि॑श्च मे त्र्य॒वी च॑ मे दित्य॒वाच्च॑ मे दित्यौ॒ही च॑ मे॒ पञ्चा॑विश्च मे पञ्चा॒वी च॑ मे त्रिव॒थ्सश्च॑ मे त्रिव॒थ्सा च॑ मे तुर्य॒वाच्च॑ मे तुर्यौ॒ही च॑ मे पष्ठ॒वाच्च॑ मे पष्ठौ॒ही च॑ म उ॒क्षा च॑ मे व॒शा च॑ म ऋष॒भश्च॑ मे वे॒हच्च॑ मेऽन॒ड्वां च॑ मे धे॒नुश्च॑ म॒ आयुर्य॒ज्ञेन॑ कल्पतां प्रा॒णो य॒ज्ञेन॑ कल्पतामपा॒नो य॒ज्ञेन॑ कल्पतां व्या॒नो य॒ज्ञेन॑ कल्पतां॒ चक्षु॑र्य॒ज्ञेन॑ कल्पता॒ꣴ॒ श्रोत्रं॑ य॒ज्ञेन॑ कल्पतां॒ मनो॑ य॒ज्ञेन॑ कल्पतां॒ वाग्य॒ज्ञेन॑ कल्पतामा॒त्मा य॒ज्ञेन॑ कल्पतां य॒ज्ञो य॒ज्ञेन॑ कल्पताम्॥१०॥ 

एका॑ च मे ति॒स्रश्च॑ मे॒ पञ्च॑ च मे स॒प्त च॑ मे॒ नव॑ च म॒ एका॑दश च मे॒ त्रयो॑दश च मे॒ पञ्च॑दश च मे स॒प्तद॑श च मे॒ नव॑दश च म॒ एक॑विꣳशतिश्च मे॒ त्रयो॑विꣳशतिश्च मे॒ पञ्च॑विꣳशतिश्च मे स॒प्तविꣳ॑शतिश्च मे॒ नव॑विꣳशतिश्च म॒ एक॑त्रिꣳशच्च मे॒ त्रय॑स्त्रिꣳशच्च मे॒ चत॑स्रश्च मे॒ऽष्टौ च॑ मे॒ द्वाद॑श च मे॒ षोड॑श च मे विꣳश॒तिश्च॑ मे॒ चतु॑र्विꣳशतिश्च मे॒ऽष्टाविꣳ॑शतिश्च मे॒ द्वात्रिꣳ॑शच्च मे॒ षट्त्रिꣳ॑शच्च मे चत्वारि॒ꣳ॒शच्च॑ मे॒ चतु॑श्चत्वारिꣳशच्च मे॒ऽष्टाच॑त्वारिꣳशच्च मे॒ वाज॑श्च प्रस॒वश्चा॑पि॒जश्च॒ क्रतु॑श्च॒ सुव॑श्च मू॒र्धा च॒ व्यश्नि॑यश्चान्त्याय॒नश्चान्त्य॑श्च भौव॒नश्च॒ भुव॑न॒श्चाधि॑पतिश्च॥११॥ 

महादेवादिभ्यो नमः॥ समस्तोपचारान् समर्पयामि॥

इडा॑ देव॒हूर्मनु॑र्यज्ञ॒नीर्बृह॒स्पति॑रुक्थाम॒दानि॑ शꣳसिष॒द्विश्वे॑दे॒वाः सू᳚क्त॒वाचः॒ पृथि॑वि मात॒र्मा मा॑ हिꣳसी॒र्मधु॑ मनिष्ये॒ मधु॑ जनिष्ये॒ मधु॑ वक्ष्यामि॒ मधु॑ वदिष्यामि॒ मधु॑मतीं दे॒वेभ्यो॒ वाच॑मुद्यासꣳ शुश्रू॒षेण्यां᳚ मनु॒ष्ये᳚भ्य॒स्तं मा॑ दे॒वा अ॑वन्तु शो॒भायै॑ पि॒तरोऽनु॑मदन्तु॥ 

\centerline{॥ॐ शान्तिः॒ शान्तिः॒ शान्तिः॑॥}

{\small \closesection}

अ॒घोरे᳚भ्योऽथ॒ घोरे᳚भ्यो॒ घोर॒घोर॑तरेभ्यः।\\
सर्वे᳚भ्यः सर्व॒शर्वे᳚भ्यो॒ नम॑स्ते अस्तु रु॒द्ररू॑पेभ्यः॥

तत्पुरु॑षाय वि॒द्महे॑ महादे॒वाय॑ धीमहि।\\
तन्नो॑ रुद्रः प्रचो॒दया᳚त्॥

ईशानः सर्व॑विद्या॒ना॒मीश्वरः सर्व॑भूता॒नां॒ ब्रह्माधि॑पति॒र्ब्रह्म॒णो\-ऽधि॑पति॒र्ब्रह्मा॑ शि॒वो मे॑ अस्तु सदाशि॒वोम्॥

तत्पुरु॑षाय वि॒द्महे॑ महादे॒वाय॑ धीमहि।\\
तन्नो॑ रुद्रः प्रचो॒दया᳚त्॥ (दशवारं जपेत्।)

महादेवादिभ्यो नमः॥ समस्तोपचारान् समर्पयामि॥

{\small \closesection}

\sect{प्रार्थना}

{\small \closesection}

\sect{श्रीरुद्रजपः}

अस्य श्री रुद्राध्याय-प्रश्न-महामन्त्रस्य। अघोर ऋषिः।\\
अनुष्टुप् छन्दः। सङ्कर्षणमूर्तिस्वरूपो योऽसावादित्यः परमपुरुषः स एष रुद्रो देवता॥

नमः शिवायेति बीजम्। शिवतरायेति शक्तिः।\\
महादेवायेति कीलकम्।\\
श्री साम्बसदाशिवप्रसादसिद्ध्यर्थे जपे विनियोगः॥\\


\centerline{॥करन्यासः॥}
ॐ अग्निहोत्रात्मने अङ्गुष्ठाभ्यां नमः।\\
दर्शपूर्णमासात्मने तर्जनीभ्यां नमः।\\
चातुर्मास्यात्मने मध्यमाभ्यां नमः।\\
निरूढपशुबन्धात्मने अनामिकाभ्यां नमः।\\
ज्योतिष्टोमात्मने कनिष्ठिकाभ्यां नमः।\\
सर्वक्रत्वात्मने करतलकरपृष्ठाभ्यां  नमः।\\
%\pagebreak[4]


\centerline{॥अङ्गन्यासः॥}
अग्निहोत्रात्मने हॄदयाय नमः।\\
दर्शपूर्णमासात्मने शिरसे स्वाहा।\\
चातुर्मास्यात्मने शिखायै वषट्।\\
निरूढपशुबन्धात्मने कवचाय हुं।\\
ज्योतिष्टोमात्मने नेत्रत्रयाय वौषट्।\\
सर्वक्रत्वात्मने अस्त्राय फट्।\\
भूर्भुवस्सुवरोमिति दिग्बन्धः।\\

\sect{ध्यानम्}
\setlength{\shlokaspaceskip}{2pt}

\fourlineindentedshloka*
{आपाताल-नभः-स्थलान्त-भुवन-ब्रह्माण्डमाविस्फुरत्}
{ज्योतिः स्फाटिक-लिङ्ग-मौलि-विलसत्-पूर्णेन्दु-वान्तामृतैः}
{अस्तोकाप्लुतमेकमीशमनिशं रुद्रानुवाकान् जपन्}
{ध्यायेदीप्सितसिद्धये ध्रुवपदं विप्रोऽभिषिञ्चेच्छिवम्}

\fourlineindentedshloka*
{ब्रह्माण्ड-व्याप्त-देहा भसित-हिमरुचा भासमाना भुजङ्गैः}
{कण्ठे कालाः कपर्दा-कलित-शशि-कलाश्चण्ड-कोदण्ड-हस्ताः}
{त्र्यक्षा रुद्राक्षमालाः प्रकटित-विभवाः शाम्भवा मूर्तिभेदाः}
{रुद्राः श्रीरुद्रसूक्त-प्रकटित-विभवा नः प्रयच्छन्तु सौख्यम्}

\centerline{॥पञ्चपूजा॥}

लं पृथिव्यात्मने गन्धं समर्पयामि।\\
हं आकाशात्मने पूष्पैः पूजयामि।\\
यं वाय्वात्मने धूपमाघ्रापयामि।\\
रं अग्न्यात्मने दीपं दर्शयामि।\\
वं अमृतात्मने अमृतं महानैवेद्यं निवेदयामि।\\
सं सर्वात्मने सर्वोपचारपूजां समर्पयामि।

ॐ ग॒णानां᳚ त्वा ग॒णप॑तिꣳ हवामहे क॒विं क॑वी॒नामु॑प॒\-मश्र॑वस्तमम्। \\
ज्ये॒ष्ठ॒राजं॒ ब्रह्म॑णां ब्रह्मणस्पत॒ आ नः॑ शृ॒ण्वन्नू॒तिभिः॑ सीद॒ साद॑नम्॥ \\
ॐ महागणपतये॒ नमः॑॥ 


शं च॑ मे॒ मय॑श्च मे प्रि॒यं च॑ मेऽनुका॒मश्च॑ मे॒ काम॑श्च मे सौमन॒सश्च॑ मे भ॒द्रं च॑ मे॒ श्रेय॑श्च मे॒ वस्य॑श्च मे॒ यश॑श्च मे॒ भग॑श्च मे॒ द्रवि॑णं च मे य॒न्ता च॑ मे ध॒र्ता च॑ मे॒ क्षेम॑श्च मे॒ धृति॑श्च मे॒ विश्वं॑ च मे॒ मह॑श्च मे सं॒विच्च॑ मे॒ ज्ञात्रं॑ च मे॒ सूश्च॑ मे प्र॒सूश्च॑ मे॒ सीरं॑ च मे ल॒यश्च॑ म ऋ॒तं च॑ मे॒ऽमृतं॑ च मेऽय॒क्ष्मं च॒ मेऽना॑मयच्च मे जी॒वातु॑श्च मे दीर्घायु॒त्वं च॑ मेऽनमि॒त्रं च॒ मेऽभ॑यं च मे सु॒गं च॑ मे॒ शय॑नं च मे सू॒षा च॑ मे सु॒दिनं॑ च मे॥
\centerline{॥ॐ शान्तिः॒ शान्तिः॒ शान्तिः॑॥}



\newcounter{cj}%chamakajapa
\newcommand{\cham}[3]{\refstepcounter{cj}%
#1\\
अनेन \textbf{#2}-वार-जपेन भगवान् सर्वात्मकः\\
\underline{\textbf{#3}} सुप्रीतः सुप्रसन्नो वरदो भवतु॥\devanumber{\arabic{cj}}॥}

\newcommand{\rudram}[1]{\sect{रुद्रप्रश्नः}


ॐ नमो भगवते॑ रुद्रा॒य॥\\
 नम॑स्ते रुद्र म॒न्यव॑ उ॒तो त॒ इष॑वे॒ नमः॑। 
% 
\sect{रुद्रप्रश्नः}

ॐ नमो भगवते॑ रुद्रा॒य॥\\
 नम॑स्ते रुद्र म॒न्यव॑ उ॒तो त॒ इष॑वे॒ नमः॑। नम॑स्ते अस्तु॒ धन्व॑ने बा॒हुभ्या॑मु॒त ते॒ नमः॑॥ या त॒ इषुः॑ शि॒वत॑मा शि॒वं ब॒भूव॑ ते॒ धनुः॑। शि॒वा श॑र॒व्या॑ या तव॒ तया॑ नो रुद्र मृडय॥ या ते॑ रुद्र शि॒वा त॒नूरघो॒राऽपा॑पकाशिनी। तया॑ नस्त॒नुवा॒ शन्त॑मया॒ गिरि॑शन्ता॒\-भिचा॑कशीहि॥ यामिषुं॑ गिरिशन्त॒ हस्ते॒ बिभ॒र्ष्यस्त॑वे। शि॒वां गि॑रित्र॒ तां कु॑रु॒ मा हिꣳ॑सीः॒ पुरु॑षं॒ जग॑त्॥ शि॒वेन॒ वच॑सा त्वा॒ गिरि॒शाच्छा॑वदामसि। यथा॑ नः॒ सर्व॒मिज्जग॑दय॒क्ष्मꣳ सु॒मना॒ अस॑त्॥ अध्य॑वोचदधिव॒क्ता प्र॑थ॒मो दैव्यो॑ भि॒षक्।  अहीꣴ॑श्च॒ सर्वा᳚ञ्ज॒म्भय॒-न्थ्सर्वा᳚श्च यातुधा॒न्यः॑॥ अ॒सौ यस्ता॒म्रो अ॑रु॒ण उ॒त ब॒भ्रुः सु॑म॒ङ्गलः॑। ये चे॒माꣳ रु॒द्रा अ॒भितो॑ दि॒क्षु श्रि॒ताः स॑हस्र॒शोऽवै॑षा॒ꣳ॒ हेड॑ ईमहे॥ अ॒सौ यो॑ऽव॒सर्प॑ति॒ नील॑ग्रीवो॒ विलो॑हितः। उ॒तैनं॑ गो॒पा अ॑दृश॒न्न॒दृ॑शन्नुदहा॒र्यः॑॥ उ॒तैनं॒ विश्वा॑ भू॒तानि॒ स दृ॒ष्टो मृ॑डयाति नः। नमो॑ अस्तु॒ नील॑ग्रीवाय सहस्रा॒क्षाय॑ मी॒ढुषे᳚॥ अथो॒ ये अ॑स्य॒ सत्वा॑नो॒ऽहं तेभ्यो॑ऽकरं॒ नमः॑। प्र मु॑ञ्च॒ धन्व॑न॒स्त्वमु॒भयो॒रार्त्नि॑यो॒र्ज्याम्॥ याश्च॑ ते॒ हस्त॒ इष॑वः॒ परा॒ ता भ॑गवो वप। अ॒व॒तत्य॒ धनु॒स्त्वꣳ सह॑स्राक्ष॒ शते॑षुधे॥ नि॒शीर्य॑ श॒ल्यानां॒ मुखा॑ शि॒वो नः॑ सु॒मना॑ भव। विज्यं॒ धनुः॑ कप॒र्दिनो॒ विश॑ल्यो॒ बाण॑वाꣳ उ॒त॥
 अने॑शन्न॒\-स्येष॑व आ॒भुर॑स्य निष॒ङ्गथिः॑। या ते॑ हे॒तिर्मी॑ढुष्टम॒ हस्ते॑ ब॒भूव॑ ते॒ धनुः॑॥ तया॒ऽस्मान् वि॒श्वत॒स्त्वम॑य॒क्ष्मया॒ परि॑ब्भुज। नम॑स्ते अ॒स्त्वायु॑धा॒याना॑तताय धृ॒ष्णवे᳚॥ उ॒भाभ्या॑मु॒त ते॒ नमो॑ बा॒हुभ्यां॒ तव॒ धन्व॑ने। परि॑ ते॒ धन्व॑नो हे॒तिर॒स्मान्वृ॑णक्तु वि॒श्वतः॑॥ अथो॒ य इ॑षु॒धिस्तवा॒ऽ॒ऽ॒रे अ॒स्मन्नि धे॑हि॒ तम्॥१॥

 
% \lbrack नम॑स्ते अस्तु भगवन् विश्वेश्व॒राय॑ महादे॒वाय॑ त्र्यम्ब॒काय॑ त्रिपुरान्त॒काय॑ त्रिका\lbrack ला\rbrack ग्निका॒लाय॑ कालाग्निरु॒द्राय॑ नीलक॒ण्ठाय॑ मृत्युञ्ज॒याय॑ सर्वेश्व॒राय॑ सदाशि॒वाय॑ श्रीमन्महादे॒वाय॒ नमः॑॥\rbrack

नमो॒ हिर॑ण्यबाहवे सेना॒न्ये॑ दि॒शां च॒ पत॑ये॒ नमो॒ नमो॑ वृ॒क्षेभ्यो॒ हरि॑केशेभ्यः पशू॒नां पत॑ये॒ नमो॒ नमः॑ स॒स्पिञ्ज॑राय॒ त्विषी॑मते पथी॒नां पत॑ये॒ नमो॒ नमो॑ बभ्लु॒शाय॑ विव्या॒धिने\-ऽन्ना॑नां॒ पत॑ये॒ नमो॒ नमो॒ हरि॑केशायोपवी॒तिने॑ पु॒ष्टानां॒ पत॑ये॒ नमो॒ नमो॑ भ॒वस्य॑ हे॒त्यै जग॑तां॒ पत॑ये॒ नमो॒ नमो॑ रु॒द्राया॑तता॒विने॒ क्षेत्रा॑णां॒ पत॑ये॒ नमो॒ नमः॑ सू॒तायाह॑न्त्याय॒ वना॑नां॒ पत॑ये॒ नमो॒ नमो॒ रोहि॑ताय स्थ॒पत॑ये वृ॒क्षाणां॒ पत॑ये॒ नमो॒ नमो॑ म॒न्त्रिणे॑ वाणि॒जाय॒ कक्षा॑णां॒ पत॑ये॒ नमो॒ नमो॑ भुव॒न्तये॑ वारिवस्कृ॒तायौष॑धीनां॒ पत॑ये॒ नमो॒ नम॑ उ॒च्चैर्घो॑षा\-याक्र॒न्दय॑ते पत्ती॒नां पत॑ये॒ नमो॒ नमः॑ कृथ्स्नवी॒ताय॒ धाव॑ते॒ सत्व॑नां॒ पत॑ये॒ नमः॑॥२॥ 


नमः॒ सह॑मानाय निव्या॒धिन॑ आव्या॒धिनी॑नां॒ पत॑ये॒ नमो॒ नमः॑ ककु॒भाय॑ निष॒ङ्गिणे᳚ स्ते॒नानां॒ पत॑ये॒ नमो॒ नमो॑ निष॒ङ्गिण॑ इषुधि॒मते॒ तस्क॑राणां॒ पत॑ये॒ नमो॒ नमो॒ वञ्च॑ते परि॒वञ्च॑ते स्तायू॒नां पत॑ये॒ नमो॒ नमो॑ निचे॒रवे॑ परिच॒रायार॑ण्यानां॒ पत॑ये॒ नमो॒ नमः॑ सृका॒विभ्यो॒ जिघाꣳ॑सद्भ्यो मुष्ण॒तां पत॑ये॒ नमो॒ नमो॑ऽसि॒मद्भ्यो॒ नक्तं॒ चर॑द्भ्यः प्रकृ॒न्तानां॒ पत॑ये॒ नमो॒ नम॑ उष्णी॒षिणे॑ गिरिच॒राय॑ कुलु॒ञ्चानां॒ पत॑ये॒ नमो॒ नम॒ इषु॑मद्भ्यो धन्वा॒विभ्य॑श्च वो॒ नमो॒ नम॑ आतन्वा॒नेभ्यः॑ प्रति॒दधा॑नेभ्यश्च वो॒ नमो॒ नम॑ आ॒यच्छ॑द्भ्यो विसृ॒जद्भ्य॑श्च वो॒ नमो॒ नमोऽस्य॑द्भ्यो॒ विध्य॑द्भ्यश्च वो॒ नमो॒ नम॒ आसी॑नेभ्यः॒ शया॑नेभ्यश्च वो॒ नमो॒ नमः॑ स्व॒पद्भ्यो॒ जाग्र॑द्भ्यश्च वो॒ नमो॒ नम॒स्तिष्ठ॑द्भ्यो॒ धाव॑द्भ्यश्च वो॒ नमो॒ नमः॑ स॒भाभ्यः॑ स॒भाप॑तिभ्यश्च वो॒ नमो॒ नमो॒ अश्वे॒भ्योऽश्व॑पतिभ्यश्च वो॒ नमः॑॥३॥ 

नम॑ आव्या॒धिनी᳚भ्यो वि॒विध्य॑न्तीभ्यश्च वो॒ नमो॒ नम॒ उग॑णाभ्यस्तृꣳ\-ह॒तीभ्य॑श्च वो॒ नमो॒ नमो॑ गृ॒थ्सेभ्यो॑ गृ॒थ्सप॑तिभ्यश्च वो॒ नमो॒ नमो॒ व्राते᳚भ्यो॒ व्रात॑पतिभ्यश्च वो॒ नमो॒ नमो॑ ग॒णेभ्यो॑ ग॒णप॑तिभ्यश्च वो॒ नमो॒ नमो॒ विरू॑पेभ्यो वि॒श्वरू॑पेभ्यश्च वो॒ नमो॒ नमो॑ म॒हद्भ्यः॑, क्षुल्ल॒केभ्य॑श्च वो॒ नमो॒ नमो॑ र॒थिभ्यो॑ऽर॒थेभ्य॑श्च वो॒ नमो॒ नमो॒ रथे᳚भ्यो॒ रथ॑पतिभ्यश्च वो॒ नमो॒ नमः॒ सेना᳚भ्यः सेना॒निभ्य॑श्च वो॒ नमो॒ नमः॑, क्ष॒त्तृभ्यः॑ सङ्ग्रही॒तृभ्य॑श्च वो॒ नमो॒ नम॒स्तक्ष॑भ्यो रथका॒रेभ्य॑श्च वो॒ नमो॒ नमः॒ कुला॑लेभ्यः क॒र्मारे᳚भ्यश्च वो॒ नमो॒ नमः॑ पु॒ञ्जिष्टे᳚भ्यो निषा॒देभ्य॑श्च वो॒ नमो॒ नम॑ इषु॒कृद्भ्यो॑ धन्व॒कृद्भ्य॑श्च वो॒ नमो॒ नमो॑ मृग॒युभ्यः॑ श्व॒निभ्य॑श्च वो॒ नमो॒ नमः॒ श्वभ्यः॒ श्वप॑तिभ्यश्च वो॒ नमः॑॥४॥ 

नमो॑ भ॒वाय॑ च रु॒द्राय॑ च॒ नमः॑ श॒र्वाय॑ च पशु॒पत॑ये च॒ नमो॒ नील॑ग्रीवाय च शिति॒कण्ठा॑य च॒ नमः॑ कप॒र्दिने॑ च॒ व्यु॑प्तकेशाय च॒ नमः॑ सहस्रा॒क्षाय॑ च श॒तध॑न्वने च॒ नमो॑ गिरि॒शाय॑ च शिपिवि॒ष्टाय॑ च॒ नमो॑ मी॒ढुष्ट॑माय॒ चेषु॑मते च॒ नमो᳚ ह्र॒स्वाय॑ च वाम॒नाय॑ च॒ नमो॑ बृह॒ते च॒ वर्‌षी॑यसे च॒ नमो॑ वृ॒द्धाय॑ च सं॒वृध्व॑ने च॒ नमो॒ अग्रि॑याय च प्रथ॒माय॑ च॒ नम॑ आ॒शवे॑ चाजि॒राय॑ च॒ नमः॒ शीघ्रि॑याय च॒ शीभ्या॑य च॒ नम॑ ऊ॒र्म्या॑य चावस्व॒न्या॑य च॒ नमः॑ स्रोत॒स्या॑य च॒ द्वीप्या॑य च॥५॥ 

नमो᳚ ज्ये॒ष्ठाय॑ च कनि॒ष्ठाय॑ च॒ नमः॑ पूर्व॒जाय॑ चापर॒जाय॑ च॒ नमो॑ मध्य॒माय॑ चापग॒ल्भाय॑ च॒ नमो॑ जघ॒न्या॑य च॒ बुध्नि॑याय च॒ नमः॑ सो॒भ्या॑य च प्रतिस॒र्या॑य च॒ नमो॒ याम्या॑य च॒ क्षेम्या॑य च॒ नम॑ उर्व॒र्या॑य च॒ खल्या॑य च॒ नमः॒ श्लोक्या॑य चावसा॒न्या॑य च॒ नमो॒ वन्या॑य च॒ कक्ष्या॑य च॒ नमः॑ श्र॒वाय॑ च प्रतिश्र॒वाय॑ च॒ नम॑ आ॒शुषे॑णाय चा॒शुर॑थाय च॒ नमः॒ शूरा॑य चावभिन्द॒ते च॒ नमो॑ व॒र्मिणे॑ च वरू॒थिने॑ च॒ नमो॑ बि॒ल्मिने॑ च कव॒चिने॑ च॒ नमः॑ श्रु॒ताय॑ च श्रुतसे॒नाय॑ च॥६॥ 

नमो॑ दुन्दु॒भ्या॑य चाऽऽहन॒न्या॑य च॒ नमो॑ धृ॒ष्णवे॑ च प्रमृ॒शाय॑ च॒ नमो॑ दू॒ताय॑ च॒ प्रहि॑ताय च॒ नमो॑ निष॒ङ्गिणे॑ चेषुधि॒मते॑ च॒ नम॑स्ती॒क्ष्णेष॑वे चाऽऽयु॒धिने॑ च॒ नमः॑ स्वायु॒धाय॑ च सु॒धन्व॑ने च॒ नमः॒ स्रुत्या॑य च॒ पथ्या॑य च॒ नमः॑ का॒ट्या॑य च नी॒प्या॑य च॒ नमः॒ सूद्या॑य च सर॒स्या॑य च॒ नमो॑ ना॒द्याय॑ च वैश॒न्ताय॑ च॒ नमः॒ कूप्या॑य चाव॒ट्या॑य च॒ नमो॒ वर्ष्या॑य चाव॒र्ष्याय॑ च॒ नमो॑ मे॒घ्या॑य च विद्यु॒त्या॑य च॒ नम॑ ई॒ध्रिया॑य चाऽऽत॒प्या॑य च॒ नमो॒ वात्या॑य च॒ रेष्मि॑याय च॒ नमो॑ वास्त॒व्या॑य च वास्तु॒पाय॑ च॥७॥ 

नमः॒ सोमा॑य च रु॒द्राय॑ च॒ नम॑स्ता॒म्राय॑ चारु॒णाय॑ च॒ नमः॑ श॒ङ्गाय॑ च पशु॒पत॑ये च॒ नम॑ उ॒ग्राय॑ च भी॒माय॑ च॒ नमो॑ अग्रेव॒धाय॑ च दूरेव॒धाय॑ च॒ नमो॑ ह॒न्त्रे च॒ हनी॑यसे च॒ नमो॑ वृ॒क्षेभ्यो॒ हरि॑केशेभ्यो॒ नम॑स्ता॒राय॒ नमः॑  श॒म्भवे॑ च मयो॒भवे॑ च॒ नमः॑ शङ्क॒राय॑ च मयस्क॒राय॑ च॒ नमः॑ शि॒वाय॑ च शि॒वत॑राय च॒ नम॒स्तीर्थ्या॑य च॒ कूल्या॑य च॒ नमः॑ पा॒र्या॑य चावा॒र्या॑य च॒ नमः॑ प्र॒तर॑णाय चो॒त्तर॑णाय च॒ नम॑ आता॒र्या॑य चाऽऽला॒द्या॑य च॒ नमः॒ शष्प्या॑य च॒ फेन्या॑य च॒ नमः॑ सिक॒त्या॑य च प्रवा॒ह्या॑य च॥८॥ 

नम॑ इरि॒ण्या॑य च प्रप॒थ्या॑य च॒ नमः॑ किꣳशि॒लाय॑ च॒ क्षय॑णाय च॒ नमः॑ कप॒र्दिने॑ च पुल॒स्तये॑ च॒ नमो॒ गोष्ठ्या॑य च॒ गृह्या॑य च॒ नम॒स्तल्प्या॑य च॒ गेह्या॑य च॒ नमः॑ का॒ट्या॑य च गह्वरे॒ष्ठाय॑ च॒ नमो᳚ ह्रद॒य्या॑य च निवे॒ष्प्या॑य च॒ नमः॑ पाꣳस॒व्या॑य च रज॒स्या॑य च॒ नमः॒ शुष्क्या॑य च हरि॒त्या॑य च॒ नमो॒ लोप्या॑य चोल॒प्या॑य च॒ नम॑ ऊ॒र्व्या॑य च सू॒र्म्या॑य च॒ नमः॑ प॒र्ण्या॑य च पर्णश॒द्या॑य च॒ नमो॑ऽपगु॒रमा॑णाय चाभिघ्न॒ते च॒ नम॑ आख्खिद॒ते च॑ प्रख्खिद॒ते च॒ नमो॑ वः किरि॒केभ्यो॑ दे॒वाना॒ꣳ॒ हृद॑येभ्यो॒ नमो॑ विक्षीण॒केभ्यो॒ नमो॑ विचिन्व॒त्केभ्यो॒ नम॑ आनिर्‌ह॒तेभ्यो॒ नम॑ आमीव॒त्केभ्यः॑॥९॥ 

द्रापे॒ अन्ध॑सस्पते॒ दरि॑द्र॒न्नील॑लोहित। ए॒षां पुरु॑षाणामे॒षां प॑शू॒नां मा भेर्माऽरो॒ मो ए॑षां॒ किं च॒नाऽऽम॑मत्॥ या ते॑ रुद्र शि॒वा त॒नूः शि॒वा वि॒श्वाह॑भेषजी। शि॒वा रु॒द्रस्य॑ भेष॒जी तया॑ नो मृड जी॒वसे᳚॥ इ॒माꣳ रु॒द्राय॑ त॒वसे॑ कप॒र्दिने᳚ क्ष॒यद्वी॑राय॒ प्रभ॑रामहे म॒तिम्॥ यथा॑ नः॒ शमस॑द्द्वि॒पदे॒ चतु॑ष्पदे॒ विश्वं॑ पु॒ष्टं ग्रामे॑ अ॒स्मिन्नना॑तुरम्॥ मृ॒डा नो॑ रुद्रो॒त नो॒ मय॑स्कृधि क्ष॒यद्वी॑राय॒ नम॑सा विधेम ते। यच्छं च॒ योश्च॒ मनु॑राय॒जे पि॒ता तद॑श्याम॒ तव॑ रुद्र॒ प्रणी॑तौ॥ मा नो॑ म॒हान्त॑मु॒त मा नो॑ अर्भ॒कं मा न॒ उक्ष॑न्तमु॒त मा न॑ उक्षि॒तम्। मा नो॑ वधीः पि॒तरं॒ मोत मा॒तरं॑ प्रि॒या मा न॑स्त॒नुवो॑ रुद्र रीरिषः॥ मा न॑स्तो॒के तन॑ये॒ मा न॒ आयु॑षि॒ मा नो॒ गोषु॒ मा नो॒ अश्वे॑षु रीरिषः। वी॒रान्मा नो॑ रुद्र भामि॒तो व॑धीर्‌ह॒विष्म॑न्तो॒ नम॑सा विधेम ते॥ आ॒रात्ते॑ गो॒घ्न उ॒त पू॑रुष॒घ्ने क्ष॒यद्वी॑राय सु॒म्नम॒स्मे ते॑ अस्तु। रक्षा॑ च नो॒ अधि॑ च देव ब्रू॒ह्यधा॑ च नः॒ शर्म॑ यच्छ द्वि॒बर्‌हाः᳚॥ स्तु॒हि श्रु॒तं ग॑र्त॒सदं॒ युवा॑नं मृ॒गं न भी॒ममु॑पह॒त्नुमु॒ग्रम्। मृ॒डा ज॑रि॒त्रे रु॑द्र॒ स्तवा॑नो अ॒न्यं ते॑ अ॒स्मन्नि व॑पन्तु॒ सेनाः᳚॥ परि॑णो रु॒द्रस्य॑ हे॒तिर्वृ॑णक्तु॒ परि॑ त्वे॒षस्य॑ दुर्म॒तिर॑घा॒योः। अव॑ स्थि॒रा म॒घव॑द्भ्यस्तनुष्व॒ मीढ्व॑स्तो॒काय॒ तन॑याय मृडय॥ मीढु॑ष्टम॒ शिव॑तम शि॒वो नः॑ सु॒मना॑ भव। प॒र॒मे वृ॒क्ष आयु॑धं नि॒धाय॒ कृत्तिं॒ वसा॑न॒ आ च॑र॒ पिना॑कं॒ बिभ्र॒दा ग॑हि॥ विकि॑रिद॒ विलो॑हित॒ नम॑स्ते अस्तु भगवः। यास्ते॑ स॒हस्रꣳ॑ हे॒तयो॒ऽन्यम॒स्मन्नि व॑पन्तु॒ ताः॥ स॒हस्रा॑णि सहस्र॒धा बा॑हु॒वोस्तव॑ हे॒तयः॑। तासा॒मीशा॑नो भगवः परा॒चीना॒ मुखा॑ कृधि॥१०॥

स॒हस्रा॑णि सहस्र॒शो ये रु॒द्रा अधि॒ भूम्या᳚म्। तेषाꣳ॑ सहस्रयोज॒नेऽव॒ धन्वा॑नि तन्मसि॥ अ॒स्मिन् म॑ह॒त्य॑र्ण॒वे᳚\-ऽन्तरि॑क्षे भ॒वा अधि॑॥ नील॑ग्रीवाः शिति॒कण्ठाः᳚ श॒र्वा अ॒धः, क्ष॑माच॒राः॥ नील॑ग्रीवाः शिति॒कण्ठा॒ दिवꣳ॑ रु॒द्रा उप॑श्रिताः॥ ये वृ॒क्षेषु॑ स॒स्पिञ्ज॑रा॒ नील॑ग्रीवा॒ विलो॑हिताः॥ ये भू॒ताना॒मधि॑पतयो विशि॒खासः॑ कप॒र्दिनः॑॥ ये अन्ने॑षु वि॒विध्य॑न्ति॒ पात्रे॑षु॒ पिब॑तो॒ जना\sn ॥ ये प॒थां प॑थि॒रक्ष॑य ऐलबृ॒दा य॒व्युधः॑॥ ये ती॒र्थानि॑ प्र॒चर॑न्ति सृ॒काव॑न्तो निष॒ङ्गिणः॑॥ य ए॒ताव॑न्तश्च॒ भूयाꣳ॑सश्च॒ दिशो॑ रु॒द्रा वि॑तस्थि॒रे॥ तेषाꣳ॑ सहस्रयोज॒नेऽव॒ धन्वा॑नि तन्मसि॥ नमो॑ रु॒द्रेभ्यो॒ ये पृ॑थि॒व्यां ये᳚ऽन्तरि॑क्षे॒ ये दि॒वि येषा॒मन्नं॒ वातो॑ व॒\ar षमिष॑व॒स्तेभ्यो॒ दश॒ प्राची॒र्दश॑ दक्षि॒णा दश॑ प्र॒तीची॒र्दशो\-दी॑ची॒र्दशो॒र्ध्वास्तेभ्यो॒ नम॒स्ते नो॑ मृडयन्तु॒ ते यं द्वि॒ष्मो यश्च॑ नो॒ द्वेष्टि॒ तं वो॒ जम्भे॑ दधामि॥११॥ 

त्र्य॑म्बकं यजामहे सुग॒न्धिं पु॑ष्टि॒वर्ध॑नम्।\\ उ॒र्वा॒रु॒कमि॑व॒ बन्ध॑नान्मृ॒त्योर्मु॑क्षीय॒ माऽमृता᳚त्॥ 

$\ldots$\\
#1\\
$\ldots$\\

त्र्य॑म्बकं यजामहे सुग॒न्धिं पु॑ष्टि॒वर्ध॑नम्।\\ उ॒र्वा॒रु॒कमि॑व॒ बन्ध॑नान्मृ॒त्योर्मु॑क्षीय॒ माऽमृता᳚त्॥ 

}

% प्रथमं गन्धतोयं च द्वितीयं पञ्चगव्यकम्।
% पञ्चामृतं तृतीयं स्यात् घृतस्नानं चतुर्थकम्॥
% पञ्चमं पयसा स्नानं दधिस्नानं तु षष्ठकम्।
% सप्तमं मधुना स्नानम् इक्षुसारमथाष्टमम्॥
% नवमं फलसारं च दशमं नाळिकेरकम्।
% एकादशं गन्धतोयं द्वादशं तीर्थमुच्यते॥

%%% 1
\rudram{\textbf{अभिषेकः}—\textbf{प्रथमं गन्धतोयं च}}

अग्ना॑विष्णू स॒जोष॑से॒मा व॑र्धन्तु वां॒ गिरः॑। द्यु॒म्नैर्वाजे॑भि॒रा\-ग॑तम्॥ 
वाज॑श्च मे प्रस॒वश्च॑ मे॒ प्रय॑तिश्च मे॒ प्रसि॑तिश्च मे धी॒तिश्च॑ मे॒ क्रतु॑श्च मे॒ स्वर॑श्च मे॒ श्लोक॑श्च मे श्रा॒वश्च॑ मे॒ श्रुति॑श्च मे॒ ज्योति॑श्च मे॒ सुव॑श्च मे प्रा॒णश्च॑ मेऽपा॒नश्च॑ मे व्या॒नश्च॒ मेऽसु॑श्च मे चि॒त्तं च॑ म॒ आधी॑तं च मे॒ वाक्च॑ मे॒ मन॑श्च मे॒ चक्षु॑श्च मे॒ श्रोत्रं॑ च मे॒ दक्ष॑श्च मे॒ बलं॑ च म॒ ओज॑श्च मे॒ सह॑श्च म॒ आयु॑श्च मे ज॒रा च॑ म आ॒त्मा च॑ मे त॒नूश्च॑ मे॒ शर्म॑ च मे॒ वर्म॑ च मे॒ऽङ्गा॑नि च मे॒ऽस्थानि॑ च मे॒ परूꣳ॑षि च मे॒ शरी॑राणि च मे॥

\cham{यो दे॒वानां᳚ प्रथ॒मं पु॒रस्ता॒द्विश्वा॒धियो॑ रु॒द्रो महर्‌षिः॑।\\
हि॒र॒ण्य॒ग॒र्भं प॑श्यत॒ जाय॑मान॒ꣳ॒ स नो॑ दे॒वः शु॒भया॒ स्मृत्याः॒ संयु॑नक्तु॥}{प्रथम}{महादेवः}

%%% 2
\rudram{\textbf{अभिषेकः}—\textbf{द्वितीयं पञ्चगव्यकम्।}}

ज्यैष्ठ्यं॑ च म॒ आधि॑पत्यं च मे म॒न्युश्च॑ मे॒ भाम॑श्च॒ मेऽम॑श्च॒ मेऽम्भ॑श्च मे जे॒मा च॑ मे महि॒मा च॑ मे वरि॒मा च॑ मे प्रथि॒मा च॑ मे व॒र्ष्मा च॑ मे द्राघु॒या च॑ मे वृ॒द्धं च॑ मे॒ वृद्धि॑श्च मे स॒त्यं च॑ मे श्र॒द्धा च॑ मे॒ जग॑च्च मे॒ धनं॑ च मे॒ वश॑श्च मे॒ त्विषि॑श्च मे क्री॒डा च॑ मे॒ मोद॑श्च मे जा॒तं च॑ मे जनि॒ष्यमा॑णं च मे सू॒क्तं च॑ मे सुकृ॒तं च॑ मे वि॒त्तं च॑ मे॒ वेद्यं॑ च मे भू॒तं च॑ मे भवि॒ष्यच्च॑ मे सु॒गं च॑ मे सु॒पथं॑ च म ऋ॒द्धं च॑ म॒ ऋद्धि॑श्च मे कॢ॒प्तं च॑ मे॒ कॢप्ति॑श्च मे म॒तिश्च॑ मे सुम॒तिश्च॑ मे॥


\cham{यस्मा॒त्परं॒ नाप॑र॒मस्ति॒ किञ्चि॒द्यस्मा॒न्नाणी॑यो॒ न ज्यायो᳚स्ति॒ कश्चि॑त्।\\
वृ॒क्ष इ॑व स्तब्धो दि॒वि ति॑ष्ठ॒त्येक॒स्तेने॒दं पू॒र्णं पुरु॑षेण॒ सर्वम्᳚॥}{द्वितीय}{शिवः}

%%% 3
\rudram{\textbf{अभिषेकः}—\textbf{पञ्चामृतं तृतीयं स्यात्}}

शं च॑ मे॒ मय॑श्च मे प्रि॒यं च॑ मेऽनुका॒मश्च॑ मे॒ काम॑श्च मे सौमन॒सश्च॑ मे भ॒द्रं च॑ मे॒ श्रेय॑श्च मे॒ वस्य॑श्च मे॒ यश॑श्च मे॒ भग॑श्च मे॒ द्रवि॑णं च मे य॒न्ता च॑ मे ध॒र्ता च॑ मे॒ क्षेम॑श्च मे॒ धृति॑श्च मे॒ विश्वं॑ च मे॒ मह॑श्च मे सं॒विच्च॑ मे॒ ज्ञात्रं॑ च मे॒ सूश्च॑ मे प्र॒सूश्च॑ मे॒ सीरं॑ च मे ल॒यश्च॑ म ऋ॒तं च॑ मे॒ऽमृतं॑ च मेऽय॒क्ष्मं च॒ मेऽना॑मयच्च मे जी॒वातु॑श्च मे दीर्घायु॒त्वं च॑ मेऽनमि॒त्रं च॒ मेऽभ॑यं च मे सु॒गं च॑ मे॒ शय॑नं च मे सू॒षा च॑ मे सु॒दिनं॑ च मे॥

\cham{न कर्म॑णा न प्र॒जया॒ धने॑न त्यागे॑नैके अमृत॒त्वमा॑न॒शुः।\\
परे॑ण॒ नाकं॒ निहि॑तं॒ गुहा॑यां वि॒भ्राज॑दे॒तद्यत॑यो वि॒शन्ति॑॥}{तृतीय}{श्री रुद्रः}

%%% 4
\rudram{\textbf{अभिषेकः}—\textbf{घृतस्नानं चतुर्थकम्॥}}

ऊर्क्च॑ मे सू॒नृता॑ च मे॒ पय॑श्च मे॒ रस॑श्च मे घृ॒तं च॑ मे॒ मधु॑ च मे॒ सग्धि॑श्च मे॒ सपी॑तिश्च मे कृ॒षिश्च॑ मे॒ वृष्टि॑श्च मे॒ जैत्रं॑ च म॒ औद्भि॑द्यं च मे र॒यिश्च॑ मे॒ राय॑श्च मे पु॒ष्टं च॑ मे॒ पुष्टि॑श्च मे वि॒भु च॑ मे प्र॒भु च॑ मे ब॒हु च॑ मे॒ भूय॑श्च मे पू॒र्णं च॑ मे पू॒र्णत॑रं च॒ मेऽक्षि॑तिश्च मे॒ कूय॑वाश्च॒ मेऽन्नं॑ च॒ मेऽक्षु॑च्च मे व्री॒हय॑श्च मे॒ यवा᳚श्च मे॒ माषा᳚श्च मे॒ तिला᳚श्च मे मु॒द्गाश्च॑ मे ख॒ल्वा᳚श्च मे गो॒धूमा᳚श्च मे म॒सुरा᳚श्च मे प्रि॒यङ्ग॑वश्च॒ मेऽण॑वश्च मे श्या॒मका᳚श्च मे नी॒वारा᳚श्च मे॥ 

\cham{वे॒दा॒न्त॒वि॒ज्ञान॒सुनि॑श्चिता॒र्थाः सन्न्या॑स यो॒गाद्यत॑यः शुद्ध॒सत्त्वाः᳚।\\
ते ब्र॑ह्मलो॒के तु॒ परा᳚न्तकाले॒ परा॑मृता॒त्परि॑मुच्यन्ति॒ सर्वे᳚॥}{चतुर्थ}{शङ्करः}

%%% 5
\rudram{\textbf{अभिषेकः}—\textbf{पञ्चमं पयसा स्नानं}}

अश्मा॑ च मे॒ मृत्ति॑का च मे गि॒रय॑श्च मे॒ पर्व॑ताश्च मे॒ सिक॑ताश्च मे॒ वन॒स्पत॑यश्च मे॒ हिर॑ण्यं च॒ मेऽय॑श्च मे॒ सीसं॑ च मे॒ त्रपु॑श्च मे श्या॒मं च॑ मे लो॒हं च॑ मे॒ऽग्निश्च॑ म॒ आप॑श्च मे वी॒रुध॑श्च म॒ ओष॑धयश्च मे कृष्टप॒च्यं च॑ मेऽकृष्टप॒च्यं च॑ मे ग्रा॒म्याश्च॑ मे प॒शव॑ आर॒ण्याश्च॑ य॒ज्ञेन॑ कल्पन्तां वि॒त्तं च॑ मे॒ वित्ति॑श्च मे भू॒तं च॑ मे॒ भूति॑श्च मे॒ वसु॑ च मे वस॒तिश्च॑ मे॒ कर्म॑ च मे॒ शक्ति॑श्च॒ मेऽर्थ॑श्च म॒ एम॑श्च म॒ इति॑श्च मे॒ गति॑श्च मे॥

\cham{द॒ह्रं॒ वि॒पा॒पं प॒रमे᳚ऽश्मभूतं॒ यत्पु॑ण्डरी॒कं पु॒रम॑ध्यस॒ꣴ॒स्थम्।\\
त॒त्रा॒पि॒ द॒ह्रं ग॒गनं॑ विशोक॒स्तस्मि॑न् यद॒न्तस्तदुपा॑सित॒व्यम्॥}{पञ्चम}{नीललोहितः}

%%% 6
\rudram{\textbf{अभिषेकः}—\textbf{दधिस्नानं तु षष्ठकम्।}}

अ॒ग्निश्च॑ म॒ इन्द्र॑श्च मे॒ सोम॑श्च म॒ इन्द्र॑श्च मे सवि॒ता च॑ म॒ इन्द्र॑श्च मे॒ सर॑स्वती च म॒ इन्द्र॑श्च मे पू॒षा च॑ म॒ इन्द्र॑श्च मे॒ बृह॒स्पति॑श्च म॒ इन्द्र॑श्च मे मि॒त्रश्च॑ म॒ इन्द्र॑श्च मे॒ वरु॑णश्च म॒ इन्द्र॑श्च मे॒ त्वष्टा॑ च म॒ इन्द्र॑श्च मे धा॒ता च॑ म॒ इन्द्र॑श्च मे॒ विष्णु॑श्च म॒ इन्द्र॑श्च मे॒ऽश्विनौ॑ च म॒ इन्द्र॑श्च मे म॒रुत॑श्च म॒ इन्द्र॑श्च मे॒ विश्वे॑ च मे दे॒वा इन्द्र॑श्च मे पृथि॒वी च॑ म॒ इन्द्र॑श्च मे॒ऽन्तरि॑क्षं च म॒ इन्द्र॑श्च मे॒ द्यौश्च॑ म॒ इन्द्र॑श्च मे॒ दिश॑श्च म॒ इन्द्र॑श्च मे मू॒र्धा च॑ म॒ इन्द्र॑श्च मे प्र॒जाप॑तिश्च म॒ इन्द्र॑श्च मे॥

\cham{यो वेदादौ स्व॑रः प्रो॒क्तो॒ वे॒दान्ते॑ च प्र॒तिष्ठि॑तः।\\
तस्य॑ प्र॒कृति॑लीन॒स्य॒ यः॒ परः॑ स म॒हेश्व॑रः॥}{षष्ठम}{ईशानः}

%%% 7
\rudram{\textbf{अभिषेकः}—\textbf{सप्तमं मधुना स्नानम्}}

अ॒ꣳ॒शुश्च॑ मे र॒श्मिश्च॒ मेऽदा᳚भ्यश्च॒ मेऽधि॑पतिश्च म उपा॒ꣳ॒शुश्च॑ मेऽन्तर्या॒मश्च॑ म ऐन्द्रवाय॒वश्च॑ मे मैत्रावरु॒णश्च॑ म आश्वि॒नश्च॑ मे प्रतिप्र॒स्थान॑श्च मे शु॒क्रश्च॑ मे म॒न्थी च॑ म आग्रय॒णश्च॑ मे वैश्वदे॒वश्च॑ मे ध्रु॒वश्च॑ मे वैश्वान॒रश्च॑ म ऋतुग्र॒हाश्च॑ मेऽतिग्रा॒ह्या᳚श्च म ऐन्द्रा॒ग्नश्च॑ मे वैश्वदे॒वश्च॑ मे मरुत्व॒तीया᳚श्च मे माहे॒न्द्रश्च॑ म आदि॒त्यश्च॑ मे सावि॒त्रश्च॑ मे सारस्व॒तश्च॑ मे पौ॒ष्णश्च॑ मे पात्नीव॒तश्च॑ मे हारियोज॒नश्च॑ मे॥ 

\cham{स॒द्योजा॒तं प्र॑पद्या॒मि॒ स॒द्योजा॒ताय॒ वै नमो॒ नमः॑।\\
भ॒वे भ॑वे॒ नाति॑ भवे भवस्व॒ माम्। भ॒वोद्भ॑वाय॒ नमः॑॥}{सप्तम}{विजयः}

%%% 8
\rudram{\textbf{अभिषेकः}—\textbf{इक्षुसारमथाष्टमम्॥}}

इ॒ध्मश्च॑ मे ब॒र्हिश्च॑ मे॒ वेदि॑श्च मे॒ धिष्णि॑याश्च मे॒ स्रुच॑श्च मे चम॒साश्च॑ मे॒ ग्रावा॑णश्च मे॒ स्वर॑वश्च म उपर॒वाश्च॑ मेऽधि॒षव॑णे च मे द्रोणकल॒शश्च॑ मे वाय॒व्या॑नि च मे पूत॒भृच्च॑ म आधव॒नीय॑श्च म॒ आग्नी᳚ध्रं च मे हवि॒र्धानं॑ च मे गृ॒हाश्च॑ मे॒ सद॑श्च मे पुरो॒डाशा᳚श्च मे पच॒ताश्च॑ मेऽवभृ॒थश्च॑ मे स्वगाका॒रश्च॑ मे॥ 

%%% 9
\rudram{\textbf{अभिषेकः}—\textbf{नवमं फलसारं च}}

\cham{वा॒म॒दे॒वाय॒ नमो᳚ ज्ये॒ष्ठाय॒ नमः॑ श्रे॒ष्ठाय॒ नमो॑ रु॒द्राय॒ नमः॒ काला॑य॒ नमः॒ कल॑विकरणाय॒ नमो॒ बल॑विकरणाय॒ नमो॒ बला॑य॒ नमो॒ बल॑प्रमथनाय॒ नमः॒ सर्व॑भूतदमनाय॒ नमो॑ म॒नोन्म॑नाय॒ नमः॑॥}{अष्टम}{भीमः}

अ॒ग्निश्च॑ मे घ॒र्मश्च॑ मे॒ऽर्कश्च॑ मे॒ सूर्य॑श्च मे प्रा॒णश्च॑ मेऽश्वमे॒धश्च॑ मे पृथि॒वी च॒ मेऽदि॑तिश्च मे॒ दिति॑श्च मे॒ द्यौश्च॑ मे॒ शक्व॑रीर॒ङ्गुल॑यो॒ दिश॑श्च मे य॒ज्ञेन॑ कल्पन्ता॒मृक्च॑ मे॒ साम॑ च मे॒ स्तोम॑श्च मे॒ यजु॑श्च मे दी॒क्षा च॑ मे॒ तप॑श्च म ऋ॒तुश्च॑ मे व्र॒तं च॑ मेऽहोरा॒त्रयो᳚र्वृ॒ष्ट्या बृ॑हद्रथन्त॒रे च॑ मे य॒ज्ञेन॑ कल्पेताम्॥ 

\cham{अ॒घोरे᳚भ्योऽथ॒ घोरे᳚भ्यो॒ घोर॒घोर॑तरेभ्यः।\\
सर्वे᳚भ्यः सर्व॒शर्वे᳚भ्यो॒ नम॑स्ते अस्तु रु॒द्ररू॑पेभ्यः॥}{नवम}{देवदेवः}

%%% 10
\rudram{\textbf{अभिषेकः}—\textbf{दशमं नाळिकेरकम्।}}

गर्भा᳚श्च मे व॒थ्साश्च॑ मे॒ त्र्यवि॑श्च मे त्र्य॒वी च॑ मे दित्य॒वाच्च॑ मे दित्यौ॒ही च॑ मे॒ पञ्चा॑विश्च मे पञ्चा॒वी च॑ मे त्रिव॒थ्सश्च॑ मे त्रिव॒थ्सा च॑ मे तुर्य॒वाच्च॑ मे तुर्यौ॒ही च॑ मे पष्ठ॒वाच्च॑ मे पष्ठौ॒ही च॑ म उ॒क्षा च॑ मे व॒शा च॑ म ऋष॒भश्च॑ मे वे॒हच्च॑ मेऽन॒ड्वां च॑ मे धे॒नुश्च॑ म॒ आयुर्य॒ज्ञेन॑ कल्पतां प्रा॒णो य॒ज्ञेन॑ कल्पतामपा॒नो य॒ज्ञेन॑ कल्पतां व्या॒नो य॒ज्ञेन॑ कल्पतां॒ चक्षु॑र्‌य॒ज्ञेन॑ कल्पता॒ꣴ॒ श्रोत्रं॑ य॒ज्ञेन॑ कल्पतां॒ मनो॑ य॒ज्ञेन॑ कल्पतां॒ वाग्य॒ज्ञेन॑ कल्पतामा॒त्मा य॒ज्ञेन॑ कल्पतां य॒ज्ञो य॒ज्ञेन॑ कल्पताम्॥ 

\cham{तत्पुरु॑षाय वि॒द्महे॑ महादे॒वाय॑ धीमहि।
तन्नो॑ रुद्रः प्रचो॒दया᳚त्॥}{दशम}{भवोद्भवः}

%%% 11
\rudram{\textbf{अभिषेकः}—\textbf{एकादशं गन्धतोयं द्वादशं तीर्थमुच्यते॥}}

यो रु॒द्रो अ॒ग्नौ यो अ॒फ्सु य ओष॑धीषु॒ यो रु॒द्रो विश्वा॒ भुव॑नाऽऽवि॒वेश॒ तस्मै॑ रु॒द्राय॒ नमो॑ अस्तु॥ तमु॑ष्टु॒हि॒ यः स्वि॒षुः सु॒धन्वा॒ यो विश्व॑स्य॒ क्षय॑ति भेष॒जस्य॑। 

(ऋक्)   यक्ष्वा᳚म॒हे सौ᳚मन॒साय॑ रु॒द्रं नमो᳚भिर्दे॒वमसु॑रं दुवस्य॥ अ॒यं मे॒ हस्तो॒ भग॑वान॒यं मे॒ भग॑वत्तरः। अ॒यं मे᳚ वि॒श्वभे᳚षजो॒ऽयं शि॒वाभि॑मर्शनः॥

ये ते॑ स॒हस्र॑म॒युतं॒ पाशा॒ मृत्यो॒ मर्त्या॑य॒ हन्त॑वे। तान् य॒ज्ञस्य॑ मा॒यया॒ सर्वा॒नव॑ यजामहे। मृ॒त्यवे॒ स्वाहा॑ मृ॒त्यवे॒ स्वाहा᳚॥ ओं नमो भगवते रुद्राय विष्णवे मृत्यु॑र्मे पा॒हि। प्राणानां ग्रन्थिरसि रुद्रो मा॑ विशा॒न्तकः। तेनान्नेना᳚प्याय॒स्व॥ नमो रुद्राय विष्णवे मृत्यु॑र्मे पा॒हि॥\rbrack
 
एका॑ च मे ति॒स्रश्च॑ मे॒ पञ्च॑ च मे स॒प्त च॑ मे॒ नव॑ च म॒ एका॑दश च मे॒ त्रयो॑दश च मे॒ पञ्च॑दश च मे स॒प्तद॑श च मे॒ नव॑दश च म॒ एक॑विꣳशतिश्च मे॒ त्रयो॑विꣳशतिश्च मे॒ पञ्च॑विꣳशतिश्च मे स॒प्तविꣳ॑शतिश्च मे॒ नव॑विꣳशतिश्च म॒ एक॑त्रिꣳशच्च मे॒ त्रय॑स्त्रिꣳशच्च मे॒ चत॑स्रश्च मे॒ऽष्टौ च॑ मे॒ द्वाद॑श च मे॒ षोड॑श च मे विꣳश॒तिश्च॑ मे॒ चतु॑र्विꣳशतिश्च मे॒ऽष्टाविꣳ॑शतिश्च मे॒ द्वात्रिꣳ॑शच्च मे॒ षट्त्रिꣳ॑शच्च मे चत्वारि॒ꣳ॒शच्च॑ मे॒ चतु॑श्चत्वारिꣳशच्च मे॒ऽष्टाच॑त्वारिꣳशच्च मे॒ वाज॑श्च प्रस॒वश्चा॑पि॒जश्च॒ क्रतु॑श्च॒ सुव॑श्च मू॒र्धा च॒ व्यश्नि॑यश्चान्त्याय॒नश्चान्त्य॑श्च भौव॒नश्च॒ भुव॑न॒श्चाधि॑पतिश्च॥ 

\cham{ईशानः सर्व॑विद्या॒ना॒मीश्वरः सर्व॑भूता॒नां॒ ब्रह्माधि॑पति॒र्ब्रह्म॒णो\-ऽधि॑पति॒र्ब्रह्मा॑ शि॒वो मे॑ अस्तु सदाशि॒वोम्॥}{एकादश}{आदित्यात्मकः श्री रुद्रः}

\lbrack इडा॑ देव॒हूर्मनु॑र्यज्ञ॒नीर्बृह॒स्पति॑रुक्थाम॒दानि॑ शꣳसिष॒द्विश्वे॑\-दे॒वाः सू᳚क्त॒वाचः॒ पृथि॑वि मात॒र्मा मा॑ हिꣳसी॒र्मधु॑ मनिष्ये॒ मधु॑ जनिष्ये॒ मधु॑ वक्ष्यामि॒ मधु॑ वदिष्यामि॒ मधु॑मतीं दे॒वेभ्यो॒ वाच॑मुद्यासꣳ शुश्रू॒षेण्यां᳚ मनु॒ष्ये᳚भ्य॒स्तं मा॑ दे॒वा अ॑वन्तु शो॒भायै॑ पि॒तरोऽनु॑मदन्तु॥\rbrack

\centerline{॥ॐ शान्तिः॒ शान्तिः॒ शान्तिः॑॥}

{\small \closesection}


% रुद्रम्। चमकम्। पुरुषसूक्तम्॥

स्नानानन्तरम् आचमनीयं समर्पयामि॥७॥

साक्षतजलेन तर्पणं कार्यम्॥

ॐ भवं देवं तर्पयामि। ॐ शर्वं देवं तर्पयामि। ॐ ईशानं देवं तर्पयामि। ॐ पशुपतिं देवं तर्पयामि। ॐ रुद्रं देवं तर्पयामि। ॐ उग्रं देवं तर्पयामि। ॐ भीमं देवं तर्पयामि। ॐ महान्तं देवं तर्पयामि॥

ॐ भवस्य देवस्य पत्नीं तर्पयामि। ॐ शर्वस्य देवस्य पत्नीं तर्पयामि। ॐ ईशानस्य देवस्य पत्नीं तर्पयामि। ॐ पशुपतेर्देवस्य पत्नीं तर्पयामि। ॐ रुद्रस्य देवस्य पत्नीं तर्पयामि। ॐ उग्रस्य देवस्य पत्नीं तर्पयामि। ॐ भीमस्य देवस्य पत्नीं तर्पयामि। ॐ महतो देवस्य पत्नीं तर्पयामि॥


अ॒सौ यो॑ऽव॒सर्प॑ति॒ नील॑ग्रीवो॒ विलो॑हितः। उ॒तैनं॑ गो॒पा अ॑दृश॒न्न॒दृ॑शन्नुदहा॒र्यः॑। उ॒तैनं॒ विश्वा॑ भू॒तानि॒ स दृ॒ष्टो मृ॑डयाति नः॥ ॐ ह्रीं न॒मः शि॒वाय॑। रु॒द्राय॒ नमः॑। वस्त्रोत्तरीयं समर्पयामि॥८॥

नमो॑ अस्तु॒ नील॑ग्रीवाय सहस्रा॒क्षाय॑ मी॒ढुषे᳚। अथो॒ ये अ॑स्य॒ सत्वा॑नो॒ऽहं तेभ्यो॑ऽकरं॒ नमः॑॥ ॐ ह्रीं न॒मः शि॒वाय॑। काला॑य॒ नमः॑। यज्ञोपवीताभरणानि समर्पयामि॥९॥

प्र मु॑ञ्च॒ धन्व॑न॒स्त्वमु॒भयो॒रार्त्नि॑यो॒र्ज्याम्। याश्च॑ ते॒ हस्त॒ इष॑वः॒ परा॒ ता भ॑गवो वप॥ ॐ ह्रीं न॒मः शि॒वाय॑। कल॑विकरणाय॒ नमः॑। दिव्यपरिमलगन्धान् धारयामि। गन्धस्योपरि अक्षतान् समर्पयामि॥१०॥

अ॒व॒तत्य॒ धनु॒स्त्वꣳ सह॑स्राक्ष॒ शते॑षुधे। नि॒शीर्य॑ श॒ल्यानां॒ मुखा॑ शि॒वो नः॑ सु॒मना॑ भव॥ ॐ ह्रीं न॒मः शि॒वाय॑। बल॑विकरणाय॒ नमः॑। पुष्पैः पूजयामि॥११॥


ॐ भवाय देवाय नमः। ॐ शर्वाय देवाय नमः। \\
ॐ ईशानाय देवाय नमः। ॐ पशुपतये देवाय नमः।\\
ॐ रुद्राय देवाय नमः। ॐ उग्राय देवाय नमः।\\
ॐ भीमाय देवाय नमः। ॐ महते देवाय नमः॥\\
ॐ भवस्य देवस्य पत्न्यै नमः। ॐ शर्वस्य देवस्य पत्न्यै नमः।\\
ॐ ईशानस्य देवस्य पत्न्यै नमः। ॐ पशुपतेर्देवस्य पत्न्यै नमः।\\
ॐ रुद्रस्य देवस्य पत्न्यै नमः। ॐ उग्रस्य देवस्य पत्न्यै नमः।\\
ॐ भीमस्य देवस्य पत्न्यै नमः। ॐ महतो देवस्य पत्न्यै नमः॥ \\
% \end{center}

\begingroup
\centering
\setlength{\columnseprule}{1pt}
\let\chapt\sect
\input{../namavali-manjari/300/Rudra_300_compact.tex}

\endgroup
{\small \closesection}


\begingroup
\centering
\setlength{\columnseprule}{1pt}
\let\chapt\sect
\input{../namavali-manjari/100/Shiva_108.tex}

\endgroup

\sect{उत्तराङ्ग-पूजा}



विज्यं॒ धनुः॑ कप॒र्दिनो॒ विश॑ल्यो॒ बाण॑वाꣳ उ॒त। अने॑शन्न॒\-स्येष॑व आ॒भुर॑स्य निष॒ङ्गथिः॑॥ ॐ ह्रीं न॒मः शि॒वाय॑। बला॑य॒ नमः॑। धूपमाघ्रापयामि॥१२॥

या ते॑ हे॒तिर्मी॑ढुष्टम॒ हस्ते॑ ब॒भूव॑ ते॒ धनुः॑। तया॒ऽस्मान् वि॒श्वत॒स्त्वम॑य॒क्ष्मया॒ परि॑ब्भुज॥ ॐ ह्रीं न॒मः शि॒वाय॑। बल॑प्रमथनाय॒ नमः॑। अलङ्कारदीपं सन्दर्शयामि॥१३॥

ॐ भूर्भुवः॒ सुवः॑। + ब्र॒ह्मणे॒ स्वाहा᳚। नम॑स्ते अ॒स्त्वायु॑धा॒याना॑तताय धृ॒ष्णवे᳚। उ॒भाभ्या॑मु॒त ते॒ नमो॑ बा॒हुभ्यां॒ तव॒ धन्व॑ने॥ ॐ ह्रीं न॒मः शि॒वाय॑। सर्व॑भूतदमनाय॒ नमः॑। () निवेदयामि। मध्ये मध्ये अमृतपानीयं समर्पयामि। अमृतापिधानमसि।\\
हस्तप्रक्षालनं समर्पयामि। पादप्रक्षालनं समर्पयामि। निवेदनानन्तरम् आचमनीयं समर्पयामि॥१४॥

परि॑ ते॒ धन्व॑नो हे॒तिर॒स्मान्वृ॑णक्तु वि॒श्वतः॑। अथो॒ य इ॑षु॒धिस्तवा॒ऽ॒ऽ॒रे अ॒स्मन्नि धे॑हि॒ तम्॥ ॐ ह्रीं न॒मः शि॒वाय॑। म॒नोन्म॑नाय॒ नमः॑। कर्पूरताम्बूलं समर्पयामि॥१५॥

नम॑स्ते अस्तु भगवन् विश्वेश्व॒राय॑ महादे॒वाय॑ त्र्यम्ब॒काय॑ त्रिपुरान्त॒काय॑ त्रिकाग्निका॒लाय॑ कालाग्निरु॒द्राय॑ नीलक॒ण्ठाय॑ मृत्युञ्ज॒याय॑ सर्वेश्व॒राय॑ सदाशि॒वाय॑ श्रीमन्महादे॒वाय॒ नमः॑॥ कर्पूरनीराजनं दर्शयामि॥१६॥

\dnsub{रक्षा}
बृ॒हथ्साम॑ क्षत्र॒भृद्वृ॒द्ध वृ॑ष्णियं त्रि॒ष्टुभौजः॑ शुभि॒तमु॒ग्रवी॑रम्।
इन्द्र॒स्तोमे॑न पञ्चद॒शेन॒ मध्य॑मि॒दं वाते॑न॒ सग॑रेण रक्ष॥

रक्षां धारयामि॥

\dnsub{नमस्काराः}
ॐ भवाय देवाय नमः। 

ॐ शर्वाय देवाय नमः। 

ॐ ईशानाय देवाय नमः। 

ॐ पशुपतये देवाय नमः। 

ॐ रुद्राय देवाय नमः। 

ॐ उग्राय देवाय नमः। 

ॐ भीमाय देवाय नमः। 

ॐ महते देवाय नमः॥


ईशानः सर्व॑विद्या॒ना॒मीश्वरः सर्व॑भूता॒नां॒ ब्रह्माधि॑पति॒र्ब्रह्म॒णो\-ऽधि॑पति॒र्ब्रह्मा॑ शि॒वो मे॑ अस्तु सदाशि॒वोम्॥ ॐ ह्रीं न॒मः शि॒वाय॑। मन्त्रपुष्पाञ्जलिं समर्पयामि।   


शिवाय नमः। रुद्राय नमः। पशुपतये नमः। नीलकण्ठाय नमः। महेश्वराय नमः। हरिकेशाय नमः। विरूपाक्षाय नमः। पिनाकिने नमः। त्रिपुरान्तकाय नमः। शम्भवे नमः। शूलिने नमः। महादेवाय नमः। इति द्वादशनामभिर्द्वादशपुष्पाञ्जलीन् दत्त्वा॥

प्रदक्षिणनमस्कारान् कृत्वा॥

\dnsub{अर्घ्यप्रदानम्}
ममोपात्त-समस्त-दुरित-क्षयद्वारा श्रीपरमेश्वरप्रीत्यर्थम् शिवरात्रौ द्वितीय-याम-पूजान्ते क्षीरार्घ्यप्रदानं करिष्ये॥

\twolineshloka*
{शिवरात्रिव्रतं देव पूजाजपपरायणः}
{करोमि विधिवद्दत्तं गृहाणार्घ्यं नमोऽस्तुते}

इत्यर्घ्यं दत्त्वा।
\medskip

\twolineshloka*
{मया कृतान्यनेकानि पापानि हर शङ्कर}
{गृहाणार्घ्यमुमाकान्त शिवरात्रौ प्रसीद मे॥२}

\dnsub{पूजानिवेदनम्}

\threelineshloka*
{पूर्वे नन्दि महाकालौ गणभृङ्गी च दक्षिणे}
{वृषस्कन्दौ पश्चिमे ते देशकालौ तथोत्तरे}
{गङ्गा च यमुना पार्श्वे पूजां गृह्ण नमोऽस्तुते॥२}


\twolineshloka*
{यत्किञ्चित् कुर्महे देव सदा सुकृतदुष्कृतम्}
{तन्मे शिवपदस्थस्य भुङ्क्ष्व क्षपय शङ्कर}

\twolineshloka*
{शिवो दाता शिवो भोक्ता शिवः सर्वमिदं जगत्}
{शिवो जयति सर्वत्र यः शिवः सोऽहमेव हि}

इति प्रार्थ्य॥

 देवहृदयस्थं पादस्थं च पुष्पमादाय प्रणम्य देवेन दत्तमिति ध्यात्वा॥

\threelineshloka*
{प्रपन्नं पाहि मामीश भीतमृत्युमहार्णवात्}
{त्वयोपभुक्त-स्रग्-गन्ध-वासो-लङ्कार-चर्चिताः}
{उच्छिष्टभोजिनो दासास्तव मायां जयेम हि}

इति मूर्ध्नि धृत्वा॥

\twolineshloka*
{मन्त्रहीनं क्रियाहीनं भक्तिहीनं महेश्वर}
{यत्कृतं तु मया देव परिपूर्णं तदस्तु ते}


अनेन पूजनेन साम्बसदाशिवः प्रीयताम्। 

\fourlineindentedshloka*
{कायेन वाचा मनसेन्द्रियैर्वा}
{बुद्‌ध्याऽऽत्मना वा प्रकृतेः स्वभावात्}
{करोमि यद्यत् सकलं परस्मै}
{नारायणायेति समर्पयामि}


ॐ तत्सद्ब्रह्मार्पणमस्तु।\medskip
