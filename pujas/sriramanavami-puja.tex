% !TeX program = XeLaTeX
% !TeX root = pUjA.tex

\setlength{\parindent}{0pt}

\chapt{श्री-रामनवमी-पूजा}

\sect{पूर्वाङ्गविघ्नेश्वरपूजा}

(आचम्य)
\twolineshloka*
{शुक्लाम्बरधरं विष्णुं शशिवर्णं चतुर्भुजम्}
{प्रसन्नवदनं ध्यायेत् सर्वविघ्नोपशान्तये}
 
प्राणान्  आयम्य।  ॐ भूः + भूर्भुवः॒ सुव॒रोम्।
 
(अप उपस्पृश्य, पुष्पाक्षतान् गृहीत्वा)\\
ममोपात्तसमस्त दुरितक्षयद्वारा \\
श्रीपरमेश्वरप्रीत्यर्थं करिष्यमाणस्य कर्मणः\\
 निर्विघ्नेन परिसमाप्त्यर्थम् आदौ विघ्नेश्वरपूजां करिष्ये।

\twolineshloka*
{ॐ ग॒णानां᳚ त्वा ग॒णप॑तिꣳ हवामहे क॒विं क॑वी॒नामु॑प॒मश्र॑वस्तमम्}
{ज्ये॒ष्ठ॒राजं॒ ब्रह्म॑णां ब्रह्मणस्पत॒ आ नः॑ शृ॒ण्वन्नू॒तिभिः॑ सीद॒ साद॑नम्}
अस्मिन् हरिद्राबिम्बे महागणपतिं ध्यायामि, आवाहयामि।\\


ॐ महागणपतये नमः  आसनं समर्पयामि।\\
पादयोः पाद्यं समर्पयामि। हस्तयोरर्घ्यं समर्पयामि।\\
आचमनीयं समर्पयामि।\\
ॐ भूर्भुवस्सुवः। शुद्धोदकस्नानं समर्पयामि।\\
स्नानानन्तरमाचमनीयं समर्पयामि।\\
वस्त्रार्थमक्षतान् समर्पयामि।\\
यज्ञोपवीताभरणार्थे अक्षतान् समर्पयामि।\\
दिव्यपरिमलगन्धान् धारयामि।\\
गन्धस्योपरि हरिद्राकुङ्कुमं समर्पयामि। अक्षतान् समर्पयामि। \\
पुष्पमालिकां समर्पयामि। पुष्पैः पूजयामि।

\dnsub{अर्चना}
% \setenumerate{label=\devanumber.}
% \renewcommand{\labelenumi}{\devanumber\theenumi.}
\begin{enumerate}%[label=\devanumber\value{enumi}]
\begin{minipage}{0.475\linewidth}   
\item ॐ सुमुखाय नमः
\item ॐ एकदन्ताय नमः
\item ॐ कपिलाय नमः
\item ॐ गजकर्णकाय नमः
\item ॐ लम्बोदराय नमः
\item ॐ विकटाय नमः
\item ॐ विघ्नराजाय नमः
\item ॐ विनायकाय नमः
\item ॐ धूमकेतवे नमः
  \end{minipage}
  \begin{minipage}{0.525\linewidth}
\item ॐ गणाध्यक्षाय नमः
\item ॐ फालचन्द्राय नमः
\item ॐ गजाननाय नमः
\item ॐ वक्रतुण्डाय नमः
\item ॐ शूर्पकर्णाय नमः
\item ॐ हेरम्बाय नमः
\item ॐ स्कन्दपूर्वजाय नमः
\item ॐ सिद्धिविनायकाय नमः
\item ॐ विघ्नेश्वराय नमः
  \end{minipage}
\end{enumerate}
नानाविधपरिमलपत्रपुष्पाणि समर्पयामि॥\\
धूपमाघ्रापयामि।\\
अलङ्कारदीपं सन्दर्शयामि।\\
नैवेद्यम्।\\
ताम्बूलं समर्पयामि।\\
कर्पूरनीराजनं समर्पयामि।\\
कर्पूरनीराजनानन्तरमाचमनीयं समर्पयामि।\\
{वक्रतुण्डमहाकाय कोटिसूर्यसमप्रभ।}\\
{अविघ्नं कुरु मे देव सर्वकार्येषु सर्वदा॥}\\
प्रार्थनाः समर्पयामि।

अनन्तकोटिप्रदक्षिणनमस्कारान् समर्पयामि।\\
छत्त्रचामरादिसमस्तोपचारान् समर्पयामि।\\


\sect{प्रधान-पूजा — श्रीराम-पूजा}

\twolineshloka*
{शुक्लाम्बरधरं विष्णुं शशिवर्णं चतुर्भुजम्}
{प्रसन्नवदनं ध्यायेत् सर्वविघ्नोपशान्तये}
 
प्राणान्  आयम्य।  ॐ भूः + भूर्भुवः॒ सुव॒रोम्।

\dnsub{सङ्कल्पः}

ममोपात्त-समस्त-दुरित-क्षयद्वारा श्री-परमेश्वर-प्रीत्यर्थं शुभे शोभने मुहूर्ते अद्य ब्रह्मणः
द्वितीयपरार्धे श्वेतवराहकल्पे वैवस्वतमन्वन्तरे अष्टाविंशतितमे कलियुगे प्रथमे पादे
जम्बूद्वीपे भारतवर्षे भरतखण्डे मेरोः दक्षिणे पार्श्वे शकाब्दे अस्मिन् वर्तमाने व्यावहारिकाणां प्रभवादीनां षष्ट्याः संवत्सराणां मध्ये (	) नाम संवत्सरे उत्तरायणे वसन्त-ऋतौ  (मेष/मीन) मासे 
शुक्लपक्षे नवम्यां शुभतिथौ (इन्दु/भौम/बुध/गुरु/भृगु /स्थिर/भानु) वासरयुक्तायाम्
(आर्द्रा/पुनर्वसू/पुष्य) नक्षत्रयुक्तायां ()-योग ()-करण-युक्तायां च एवं गुण-विशेषण-विशिष्टायाम्
अस्याम् नवम्यां शुभतिथौ अस्माकं सहकुटुम्बानां क्षेमस्थैर्य-धैर्य-वीर्य-विजय-आयुरारोग्य-ऐश्वर्याभिवृद्ध्यर्थम्
 धर्मार्थकाममोक्ष\-चतुर्विधफलपुरुषार्थसिद्ध्यर्थं पुत्रपौत्राभिवृद्ध्यर्थम् इष्टकाम्यार्थसिद्ध्यर्थम्
मम इहजन्मनि पूर्वजन्मनि जन्मान्तरे च सम्पादितानां ज्ञानाज्ञानकृतमहा\-पातकचतुष्टय-व्यतिरिक्तानां रहस्यकृतानां प्रकाशकृतानां सर्वेषां पापानां सद्य अपनोदनद्वारा सकल-पापक्षयार्थं 
श्रीसीतालक्ष्मणभरतशत्रुघ्नहनुमत्समेत श्रीरामचन्द्रप्रीत्यर्थं
श्रीरामनवमीपुण्यकाले कल्पोक्तप्रकारेण यथाशक्ति श्रीरामचन्द्रपूजां
करिष्ये।
तदङ्गं कलशपूजां च करिष्ये।


श्रीविघ्नेश्वराय नमः यथास्थानं प्रतिष्ठापयामि।
(गणपति-प्रसादं शिरसा गृहीत्वा)

\dnsub{आसन-पूजा}
\centerline{पृथिव्या  मेरुपृष्ठ  ऋषिः।  सुतलं  छन्दः।  कूर्मो  देवता॥}
\twolineshloka*
{पृथ्वि  त्वया  धृता  लोका  देवि  त्वं  विष्णुना  धृता}
{त्वं  च  धारय  मां  देवि  पवित्रं  चाऽऽसनं  कुरु}


\dnsub{घण्टापूजा}
\twolineshloka*
{आगमार्थं तु देवानां गमनार्थं तु रक्षसाम्}
{घण्टारवं करोम्यादौ देवताऽऽह्वानकारणम्}


\dnsub{कलशपूजा}
ॐ कलशाय नमः दिव्यगन्धान् धारयामि।\\
ॐ गङ्गायै नमः। ॐ यमुनायै नमः। ॐ गोदावर्यै नमः।  ॐ सरस्वत्यै नमः। ॐ नर्मदायै नमः। ॐ सिन्धवे नमः। ॐ कावेर्यै नमः।\\
ॐ सप्तकोटिमहातीर्थान्यावाहयामि।\\[-0.25ex]

(अथ कलशं स्पृष्ट्वा जपं कुर्यात्) \\
आपो॒ वा इ॒द सर्वं॒ विश्वा॑ भू॒तान्याप॑ प्रा॒णा वा आप॑ प॒शव॒ आपो\-ऽन्न॒मापोऽमृ॑त॒माप॑ स॒म्राडापो॑ वि॒राडाप॑ स्व॒राडाप॒श्\-छन्दा॒स्यापो॒ ज्योती॒ष्यापो॒ यजू॒ष्याप॑ स॒त्यमाप॒ सर्वा॑ दे॒वता॒ आपो॒ भूर्भुव॒ सुव॒राप॒ ओम्॥\\

\twolineshloka* 
{कलशस्य मुखे विष्णुः कण्ठे रुद्रः समाश्रितः}
{मूले तत्र स्थितो ब्रह्मा मध्ये मातृगणाः स्मृताः}
\threelineshloka* 
{कुक्षौ तु सागराः सर्वे सप्तद्वीपा वसुन्धरा}
{ऋग्वेदोऽथ यजुर्वेदः सामवेदोऽप्यथर्वणः}
{अङ्गैश्च सहिताः सर्वे कलशाम्बुसमाश्रिताः}
\twolineshloka* 
{गङ्गे च यमुने चैव गोदावरि सरस्वति}
{नर्मदे सिन्धुकावेरि जलेऽस्मिन् सन्निधिं कुरु}
\twolineshloka*
{सर्वे समुद्राः सरितः तीर्थानि च ह्रदा नदाः}
{आयान्तु देवपूजार्थं दुरितक्षयकारकाः}

\centerline{ॐ भूर्भुवः॒ सुवो॒ भूर्भुवः॒ सुवो॒ भूर्भुवः॒ सुवः॑।}

(इति कलशजलेन सर्वोपकरणानि आत्मानं च प्रोक्ष्य।)


\dnsub{आत्म-पूजा}
ॐ आत्मने नमः, दिव्यगन्धान् धारयामि।
\begin{multicols}{2}
१. ॐ आत्मने नमः\\
२. ॐ अन्तरात्मने नमः\\
३. ॐ योगात्मने नमः\\
४. ॐ जीवात्मने नमः\\
५. ॐ परमात्मने नमः\\
६. ॐ ज्ञानात्मने नमः
\end{multicols}
समस्तोपचारान् समर्पयामि।

\twolineshloka*
{देहो देवालयः प्रोक्तो जीवो देवः सनातनः}
{त्यजेदज्ञाननिर्माल्यं सोऽहं भावेन पूजयेत्}


\begin{minipage}{\linewidth}
\dnsub{पीठ-पूजा}

\begin{multicols}{2}
\begin{enumerate}
\item ॐ आधारशक्त्यै नमः
\item ॐ मूलप्रकृत्यै नमः
\item ॐ आदिकूर्माय नमः 
\item ॐ आदिवराहाय नमः
\item ॐ अनन्ताय नमः
\item ॐ पृथिव्यै नमः
\item ॐ रत्नमण्डपाय नमः
\item ॐ रत्नवेदिकायै नमः
\item ॐ स्वर्णस्तम्भाय नमः
\item ॐ श्वेतच्छत्त्राय नमः
\item ॐ कल्पकवृक्षाय नमः
\item ॐ क्षीरसमुद्राय नमः 
\item ॐ सितचामराभ्यां नमः
\item ॐ योगपीठासनाय नमः
\end{enumerate}
\end{multicols}

\end{minipage}

\dnsub{गुरु ध्यानम्}

\twolineshloka*
{गुरुर्ब्रह्मा गुरुर्विष्णुर्गुरुर्देवो महेश्वरः}
{गुरुः साक्षात् परं ब्रह्म तस्मै श्री गुरवे नमः}

 
\begin{center}

\sect{षोडशोपचार-पूजा}
\renewcommand{\devAya}{सपरिवाराय श्रीरामाय नमः,}

\fourlineindentedshloka*
{वैदेही-सहितं सुर-द्रुम-तले हैमे महामण्डपे}
{मध्येपुष्पकमासने मणिमये वीरासने सुस्थितम्}
{अग्रे वाचयति प्रभञ्जन-सुते तत्त्वं मुनिभ्यः परं}
{व्याख्यान्तं भरतादिभिः परिवृतं रामं भजे श्यामलम्}

\fourlineindentedshloka*
{वामे भूमि-सुता पुरश्च हनुमान् पश्चात् सुमित्रा-सुतः}
{शत्रुघ्नो भरतश्च पार्श्व-दलयोर्-वाय्वादि-कोणेषु च}
{सुग्रीवश्च विभीषणश्च युवराट् तारा-सुतो जाम्बवान्}
{मध्ये नील-सरोज-कोमल-रुचिं रामं भजे श्यामलम्}

\textbf{श्री-सीता-लक्ष्मण-भरत-शत्रुघ्न-हनुमत्-समेत-श्री-रामचन्द्रं ध्यायामि।}

(अथ प्राणप्रतिष्ठा)


आवाहयामि विश्वेशं वैदेही-वल्लभं विभुम्।\\
कौसल्या-तनयं विष्णुं श्री-रामं प्रकृतेः परम्॥\\
\textbf{श्रीरामाय नमः – आवाहयामि।}

वामे सीताम् आवाहयामि।
पुरस्तात् हनुमन्तम् आवाहयामि ।
पश्चात् लक्ष्मणम् आवाहयामि।
उत्तरस्यां शत्रुघ्नमवाहयामि ।
दक्षिणस्यां दिशि भरतम् आवाहयामि ।
वायव्यायां सुग्रीवम् आवाहयामि ।
ऐशान्यां विभीषणम् आवाहयामि ।
आग्नेय्याम् अङ्गदम् आवाहयामि ।
नैर्ऋत्यां जाम्बवन्तम् आवाहयामि ॥

\twolineshloka*
{रत्न-सिंहासनारूढ सर्व-भूपाल-वन्दित}
{आसनं ते मया दत्तं प्रीतिं जनयतु प्रभो}
\textbf{\devAya{} आसनं समर्पयामि।\\}

\twolineshloka*
{पादाङ्गुष्ठ-समुद्भूत-गङ्गा-पावित-विष्टप}
{पाद्यार्थमुदकं राम ददामि परिगृह्यताम्}
\textbf{\devAya{} पाद्यं समर्पयामि।\\}

\twolineshloka*
{वालखिल्यादिभिर्-विप्रैस्-त्रिसन्ध्यं प्रयतात्मभिः}
{अर्घ्यैः आराधित विभो ममार्घ्यं राम गृह्यताम्}
\textbf{\devAya{} अर्घ्यं समर्पयामि।\\}

\twolineshloka*
{आचान्ताम्भोधिना राम मुनिना परिसेवित}
{मया दत्तेन तोयेन कुर्वाचमनमीश्वर}
\textbf{\devAya{} आचमनीयं समर्पयामि।\\}

\twolineshloka*
{नमः श्री-वासुदेवाय तत्त्व-ज्ञान-स्वरूपिणे }
{मधुपर्कं गृहाणेमं जानकीपतये नमः}
\textbf{\devAya{} मधुपर्कं समर्पयामि।\\}

\twolineshloka*
{कामधेनु-समुद्भूत-क्षीरेणेन्द्रेण राघव}
{अभिषिक्त अखिलार्थाप्त्यै स्नाहि मद्-दत्त-दुग्धतः}
\textbf{\devAya{} क्षीराभिषेकं समर्पयामि।\\}

\twolineshloka*
{हनूमता मधुवनोद्भूतेन मधुना प्रभो}
{प्रीत्याऽभिषेचित-तनो मधुना स्नाहि मेऽद्य भोः}
\textbf{\devAya{} मध्वभिषेकं समर्पयामि।\\}

\twolineshloka*
{त्रैलोक्य-ताप-हरण-नाम-कीर्तन राघव}
{मधूत्थ-ताप-शान्त्यर्थं स्नाहि क्षीरेण वै पुनः}
\textbf{\devAya{} मध्वभिषेकान्ते पुनः क्षीराभिषेकं समर्पयामि।\\}

\twolineshloka*
{नदी-नद-समुद्रादि-तोयैर्-मन्त्राभिसंस्कृतैः}
{पट्टाभिषिक्त राजेन्द्र स्नाहि शुद्ध-जलेन मे}
\textbf{\devAya{} शुद्धोदक-स्नानं समर्पयामि।\\}
स्नानोत्तरम् आचमनीयं समर्पयामि।\\

\twolineshloka*
{हित्वा पीताम्बरं चीर-कृष्णाजिन-धराच्युत}
{परिधत्स्वाद्य मे वस्त्रं स्वर्ण-सूत्र-विनिर्मितम्}
\textbf{\devAya{} वस्त्रं समर्पयामि।\\}

\twolineshloka*
{राजर्षि-वंश-तिलक रामचन्द्र नमोऽस्तु ते}
{यज्ञोपवीतं विधिना निर्मितं धत्स्व मे प्रभो}
\textbf{\devAya{} उपवीतं समर्पयामि।\\}

\twolineshloka*
{किरीटादीनि राजेन्द्र हंसकान्तानि राघव}
{विभूषणानि धृत्वाऽद्य शोभस्व सह सीतया}
\textbf{\devAya{} आभरणम् समर्पयामि।\\}

\twolineshloka*
{सन्ध्या-समान-रुचिना नीलाभ्र-सम-विग्रह}
{लिम्पामि तेऽङ्गकं राम चन्दनेन मुदा हृदि}
\textbf{\devAya{} गन्धान् धारयामि।\\}
गन्धस्योपरि हरिद्रा-कुङ्कुमं समर्पयमि।\\

\twolineshloka*
{अक्षतान् कुङ्कुमोन्मिश्रान् अक्षय्य-फल-दायक}
{अर्पये तव पादाब्जे शालि-तण्डुल-सम्भवान्}
\textbf{\devAya{} अक्षतान् समर्पयामि।\\}

\twolineshloka*
{चम्पकाशोक-पुन्नागैर्-जलजैस्-तुलसी-दलैः}
{पूजयामि रघूत्तंस पूज्यं त्वां सनकादिभिः}
\textbf{\devAya{} पुष्पाणि समर्पयामि।}

\dnsub{अङ्ग-गुण-पूजा}
\begin{supertabular}{p{0.525\linewidth}p{0.45\linewidth}}
अहल्या-उद्धारकाय नमः & पाद-रजः पूजयामि \\
शरणागत-रक्षकाय नमः & पाद-कान्तिं पूजयामि \\
गङ्गा-नदी-प्रवर्तन-पराय नमः & पाद-नखान् पूजयामि \\
सीता-संवाहित-पदाय नमः & पाद-तलं पूजयामि \\
दुन्दुभि-काय-विक्षेपकाय नमः & पादाङ्गुष्ठं पूजयामि \\
विनत-कल्प-द्रुमाय नमः & गुल्फौ पूजयामि \\
दण्डकारण्य-गमन-जङ्घालाय नमः & जङ्घे पूजयामि \\
जानु-न्यस्त-कराम्बुजाय नमः & जानुनी पूजयामि \\
वीरासन-अध्यासिने नमः & ऊरू पूजयामि \\
पीताम्बर-अलङ्कृताय नमः & कटिं पूजयामि \\
आकाश-मध्यगाय नमः & मध्यं पूजयामि \\
अरि-निग्रह-पराय नमः & कटि-लम्बितम् असिं पूजयामि \\
अब्धि-मेखला-पतये नमः & मध्य-लम्बित-मेखला-दाम पूजयामि \\
उदर-स्थित-ब्रह्माण्डाय नमः & उदरं पूजयामि \\
जगत्-त्रय-गुरवे नमः & वलि-त्रयं पूजयामि \\
सीतानुलेपित-काश्मीर-चन्दनाय नमः & वक्षः पूजयामि \\
अभय-प्रदान-शौण्डाय नमः & दक्षिण-बाहु-दण्डं पूजयामि \\
वितरण-जित-कल्पद्रुमाय नमः & दक्षिण-कर-तलं पूजयामि \\
आशर-निरसन-पराय नमः & दक्षिण-कर-स्थित-शरं पूजयामि \\
ज्ञान-विज्ञान-भासकाय नमः & चिन्मुद्रां पूजयामि \\
मुनि-सङ्घार्पित-दिव्य-पदाय नमः & वाम-भुज-दण्डं पूजयामि \\
दशानन-काल-रूपिणे नमः & वाम-हस्त-स्थित-कोदण्डं पूजयामि \\
शत-मख-दत्त-शत-पुष्कर-स्रजे नमः & अंसौ पूजयामि \\
कृत्त-दशानन-किरीट-कूटाय नमः & अंस-लम्बित-निषङ्ग-द्वयं पूजयामि \\
सीता-बाहु-लतालिङ्गिताय नमः & कण्ठं पूजयामि \\
स्मित-भाषिणे नमः & स्मितं पूजयामि \\
नित्य-प्रसन्नाय नमः & मुख-प्रसादं पूजयामि \\
सत्य-वाचे नमः & वाचं पूजयामि \\
कपालि-पूजिताय नमः & कपोलौ पूजयामि \\
चक्षुःश्रवः-प्रभु-पूजिताय नमः & श्रोत्रे पूजयामि \\
अनासादित-पाप-गन्धाय नमः & घ्राणं पूजयामि \\
पुण्डरीकाक्षाय नमः & अक्षिणी पूजयामि \\
अपाङ्ग-स्यन्दि-करुणाय नमः & अरुणापाङ्ग-द्वयं पूजयामि \\
विना-कृत-रुषे नमः & अनाथ-रक्षक-कटाक्षं पूजयामि \\
कस्तूरी-तिलकाङ्किताय नमः & फालं पूजयामि \\
राजाधिराज-वेषाय नमः & किरीटं पूजयामि \\
मुनि-मण्डल-पूजिताय नमः & जटा-मण्डलं पूजयामि \\
मोहित-मुनि-जनाय नमः & पुंसां मोहनं रूपं पूजयामि \\
जानकी-व्यजन-वीजिताय नमः & विद्युद्-विद्योतित-कालाभ्र-सदृश-कान्तिं पूजयामि \\
हनुमदर्पित-चूडामणये नमः & करुणारस-उद्वेल्लित-कटाक्ष-धारां पूजयामि \\
सुमन्त्रानुग्रह-पराय नमः & तेजोमयरूपं पूजयामि \\
कम्पिताम्भोधये नमः & आहार्य-कोपं पूजयामि \\
तिरस्कृत-लङ्केश्वराय नमः & धैर्यं पूजयामि \\
वन्दित-जनकाय नमः & विनयं पूजयामि \\
सम्मानित-त्रिजटाय नमः & अतिमानुष-सौलभ्यं पूजयामि \\
गन्धर्व-राज-प्रतिमाय नमः & लोकोत्तर-सौन्दर्यं पूजयामि \\
असहाय-हत-खर-दूषणादि-चतुर्दश-सहस्र-राक्षसाय नमः & पराक्रमं पूजयामि \\
आलिङ्गित-आञ्जनेयाय नमः & भक्त-वात्सल्यं पूजयामि \\
लब्ध-राज्य-परित्यक्त्रे नमः & धर्मं पूजयामि \\
दर्भ-शायिने नमः & लोकानुवर्तनं पूजयामि \\
सर्वेश्वराय नमः & सर्वाण्यङ्गानि सर्वांश्च गुणान् पूजयामि \\
\end{supertabular}

\begingroup
\setlength{\columnseprule}{1pt}
\let\chapt\sect
\centering
\input{../namavali-manjari/100/Rama_108.tex}
\input{../namavali-manjari/100/Sita_108.tex}
\input{../namavali-manjari/100/Anjaneya_108.tex}
\input{../namavali-manjari/100/Rama_Ramarahasya_108.tex}
\input{../namavali-manjari/100/Sita_Ramarahasya_108.tex}
\input{../namavali-manjari/100/Anjaneya_Ramarahasya_108.tex}
\endgroup

\dnsub{उत्तराङ्ग-पूजा}\markboth{उत्तराङ्ग-पूजा}{उत्तराङ्ग-पूजा}

\renewcommand{\devAya}{श्री-सीता-लक्ष्मण-भरत-शत्रुघ्न-हनुमत्-समेत-श्री-रामचन्द्र-परब्रह्मणे नमः}



\twolineshloka*
{वनस्पति-रसोद्भूतो गन्धाढ्यो धूप उत्तमः}
{रामचन्द्र महीपाल धूपोऽयं प्रतिगृह्यताम्}
\textbf{\devAya{} धूपम् आघ्रापयामि।}

\twolineshloka*
{ज्योतिषां पतये तुभ्यं नमो रामाय वेधसे}
{गृहाण मङ्गलं दीपं त्रैलोक्य-तिमिरापहम्}
\textbf{\devAya{} अलङ्कारदीपं सन्दर्शयामि।}

ओं भूर्भुवस्सुवः + ब्रह्मणे स्वाहा।
\twolineshloka*
{इदं दिव्यान्नम् अमृतं रसैः षड्भिः समन्वितम्}
{रामचन्द्रेश नैवेद्यं सीतेश प्रतिगृह्यताम्}

\textbf{\devAya{} नैवेद्यं निवेदयामि। मध्ये मध्ये पानीयं समर्पयामि। निवेदनोत्तरम् आचमनीयं समर्पयामि।}

\twolineshloka*
{नागवल्ली-दलैर्-युक्तं पूगी-फल-समन्वितम्}
{ताम्बूलं गृह्यतां राम कर्पूरादि-समन्वितम्}
\textbf{\devAya{} कर्पूरताम्बूलं समर्पयामि।}

\twolineshloka*
{मङ्गलार्थं महीपाल नीराजनमिदं हरे}
{सङ्गृहाण जगन्नाथ रामचन्द्र नमोऽस्तु ते}
\textbf{\devAya{} समस्त-अपराध-क्षमापणार्थंं समस्त-दुरित-उपशान्त्यर्थं समस्त-सन्मङ्गल-अवाप्त्यर्थं कर्पूर-नीराजनं दर्शयामि। रक्षां धारयामि।}


\twolineshloka*
{कल्पवृक्ष-समुद्भूतैः पुरुहूतादिभिः सुमैः}
{पुष्पाञ्जलिं ददाम्यद्य पूजिताय आशर-द्विषे}

यो॑ऽपां पुष्पं॒ वेद॑। पुष्प॑वान् प्र॒जावा᳚न् पशु॒मान् भ॑वति।\\
च॒न्द्रमा॒ वा अ॒पां पुष्पम्᳚। पुष्प॑वान् प्र॒जावा᳚न् पशु॒मान् भ॑वति।\\
य ए॒वं वेद॑। यो॑ऽपामा॒यत॑नं॒ वेद॑। आ॒यत॑नवान् भवति।\medskip

ओं᳚ तद्ब्र॒ह्म। ओं᳚ तद्वा॒युः। ओं᳚ तदा॒त्मा।\\ ओं᳚ तथ्स॒त्यम्‌।
ओं᳚ तथ्सर्वम्᳚‌। ओं᳚ तत्पुरो॒र्नमः॥\medskip

अन्तश्चरति॑ भूते॒षु॒ गुहायां वि॑श्वमू॒र्तिषु। \\
त्वं यज्ञस्त्वं वषट्कारस्त्वमिन्द्रस्त्वꣳ\\ रुद्रस्त्वं विष्णुस्त्वं ब्रह्म त्वं॑ प्रजा॒पतिः। \\
त्वं त॑दाप॒ आपो॒ ज्योती॒ रसो॒ऽमृतं॒ ब्रह्म॒ भूर्भुवः॒ सुव॒रोम्‌॥

\textbf{\devAya{} वेदोक्तमन्त्रपुष्पाञ्जलिं समर्पयामि।}

\twolineshloka*
{मन्दाकिनी-समुद्भूत-काञ्चनाब्ज-स्रजा विभो}
{सम्मानिताय शक्रेण स्वर्ण-पुष्पं ददामि ते}
\textbf{\devAya{} स्वर्ण-पुष्पम् समर्पयामि।}

\twolineshloka*
{चराचरं व्याप्नुवन्तम् अपि त्वां रघु-नन्दन}
{प्रदक्षिणं करोम्यद्य मदग्रे मूर्ति-संयुतम्}

\textbf{\devAya{} प्रदक्षिणं करोमि।}

\fourlineindentedshloka*
{ध्येयं सदा परिभव-घ्नम् अभीष्ट-दोहं}
{तीर्थास्पदं शिव-विरिञ्चि-नुतं शरण्यम्}
{भृत्यार्ति-हं प्रणत-पाल-भवाब्धि-पोतं}
{वन्दे महापुरुष ते चरणारविन्दम्}

\fourlineindentedshloka*     
{त्यक्त्वा सुदुस्त्यज-सुरेप्सित-राज्य-लक्ष्मीं}
{धर्मिष्ठ आर्य-वचसा यदगाद् अरण्यम्}
{माया-मृगं दयितयेप्सितम् अन्वधावत्}
{वन्दे महापुरुष ते चरणारविन्दम्}

\twolineshloka*
{साङ्गोपाङ्गाय साराय जगतां सनकादिभिः}
{वन्दिताय वरेण्याय राघवाय नमो नमः } 

\textbf{\devAya{} नमस्कारान् समर्पयामि।}

\sect{प्रार्थना}
\resetShloka
\threelineshloka
{त्वमक्षरोऽसि भगवन् व्यक्ताव्यक्त-स्वरूप-धृत्}
{यथा त्वं रावणं हत्वा यज्ञ-विघ्न-करं खलम्}
{लोकान् रक्षितवान् राम तथा मन्मानसाश्रयम्}

\twolineshloka
{रजस्तमश्च निर्हत्य त्वत्पूजालस्य-कारकम्}
{सत्त्वम् उद्रेचय विभो त्वत्पूजादर-सिद्धये}

\twolineshloka
{विभूतिं वर्षय गृहे पुत्रपौत्राभिवृद्धिकृत्}
{कल्याणं कुरु मे नित्यं कैवल्यं दिश चान्ततः}

\twolineshloka
{विधितोऽविधितो वाऽपि या पूजा क्रियते मया}
{तां त्वं सन्तुष्टहृदयो यथावद् विहितामिव}

\twolineshloka
{स्वीकृत्य परमेशान मात्रा मे सह सीतया}
{लक्ष्मणादिभिरप्यत्र प्रसादं कुरु मे सदा}

प्रार्थनाः समर्पयामि॥

\begin{center}
\input{../stotra-sangrahah/stotras/rama/SitaRamaStotram.tex}
\end{center}
\markboth{उत्तराङ्ग-पूजा}{उत्तराङ्ग-पूजा}

\twolineshloka*
{एकातपत्रच्छायायां शासिताशेषभूमिक}
{मम छत्रमिदं रत्नजालकं राम गृह्यताम्}
\hfill छत्रम् समर्पयामि।


\twolineshloka*
{रक्षोराजानुजाभ्यां ते कृतं चामरसेवया}
{वीजयेऽहं कराभ्यां ते चामरद्वयमादरात्}
\hfill चामरम् वीजयामि।


\twolineshloka*
{रामायणं साधु गीतं सुताभ्यां श्रुतवानसि}
{मयाऽपि गीयमानं ते स्तोत्रं चित्ताय रोचताम्}
\hfill गीतम् गायामि।


\twolineshloka*
{वीणावेणुमृदङ्गादिवाद्यैस्त्वां प्रीणयाम्यहम्}
{मददम्भाहङ्कृतीनां नाशको भव राघव}
\hfill वाद्यम् घोषयामि।


\twolineshloka*
{आरुह्य सीतया सार्धं दत्तामान्दोलिकां मया}
{विभाहि भूषितो राम मत्कृते पूजनोत्सवे}
\hfill आन्दोलिकां समर्पयामि।


\twolineshloka*
{मया कल्पितपल्याणं महान्तं मम घोटकम्}
{मदंसे चरणं न्यस्य मुदाऽऽरोह रघूत्तम}
\hfill अश्वान् आरोहयामि।


\twolineshloka*
{गजेन महताऽऽयान्तमाकाङ्क्षन्ति स्म नागराः}
{द्रष्टुं त्वां मगजे भाहि दृष्ट्वा नन्देयमप्यहम्}
\hfill गजान् आरोहयामि।

समस्तराजोपचारदेवोपचारपूजाः समर्पयामि।



\twolineshloka*
{मनसा वचसा कायेनागसां शतमन्वहम्}
{धियाऽधिया च रचये क्षमस्व सहजक्षम}

\twolineshloka*
{आवाहनं न जानामि न जानामि विसर्जनम्}
{पूजाविधिं न जानामि क्षमस्व पुरुषोत्तम}


\dnsub{अर्घ्य-प्रदानम्}

\twolineshloka*
{शुक्लाम्बरधरं विष्णुं शशिवर्णं चतुर्भुजम्}
{प्रसन्नवदनं ध्यायेत् सर्वविघ्नोपशान्तये}


प्राणान्  आयम्य।  ॐ भूः + भूर्भुवः॒ सुव॒रोम्।

ममोपात्त + प्रीत्यर्थम् अद्य पूर्वोक्त + शुभतिथौ श्रीरामचन्द्रपूजान्ते अर्घ्यप्रदानं करिष्ये (इति सङ्कल्प्य)।

\twolineshloka*
{राम रात्रिञ्चराराते क्षीरमध्वाज्यकल्पितम्}
{पूजान्तेऽर्घ्यं मया दत्तं स्वीकृत्य वरदो भव}

\devAya{} इदमर्घ्यं इदमर्घ्यं इदमर्घ्यम्॥

अनेनार्ध्यप्रदानेन श्री-सीता-लक्ष्मण-भरत-शत्रुघ्न-हनुमत्-समेत-श्री-रामचन्द्रः प्रीयताम्।

\twolineshloka*
{हिरण्यगर्भगर्भस्थं हेमबीजं विभावसोः}
{अनन्तपुण्यफलदम् अतः शान्तिं प्रयच्छ मे}

श्री-रामनवमी-पुण्यकाले अस्मिन् मया क्रियमाण श्रीरामपूजायां यद्देयमुपायनदानं तत्प्रतिनिधित्वेन हिरण्यं श्री-सीता-लक्ष्मण-भरत-शत्रुघ्न-हनुमत्-समेत-श्री-रामचन्द्र-प्रीतिं 
कामयमानः मनसोद्दिष्टाय ब्राह्मणाय सम्प्रददे नमः न मम। 

अनया पूजया श्री-सीता-लक्ष्मण-भरत-शत्रुघ्न-हनुमत्-समेत-श्री-रामचन्द्रः प्रीयताम्। 
 
\dnsub{विसर्जनम्}
\twolineshloka*
{यस्य स्मृत्या च नामोक्त्या तपः-पूजा-क्रियादिषु}
{न्यूनं सम्पूर्णतां याति सद्यो वन्दे तमच्युतम्} 

\twolineshloka*
{इदं व्रतं मया देव कृतं प्रीत्यै तव प्रभो}
{न्यूनं सम्पूर्णतां यातु त्वत्प्रसादाज्जनार्द्दन}

अस्मात् बिम्बात् श्री-सीता-लक्ष्मण-भरत-शत्रुघ्न-हनुमत्-समेत-श्री-रामचन्द्रं यथास्थानं प्रतिष्ठापयामि।\\
(अक्षतानर्पित्वा देवमुत्सर्जयेत्।)\\

\fourlineindentedshloka*
{कायेन वाचा मनसेन्द्रियैर्वा}
{बुद्‌ध्याऽऽत्मना वा प्रकृतेः स्वभावात्}
{करोमि यद्यत् सकलं परस्मै}
{नारायणायेति समर्पयामि}

अनया पूजया श्री-सीता-लक्ष्मण-भरत-शत्रुघ्न-हनुमत्-समेत-श्री-रामचन्द्रः प्रीयताम्। \\
ॐ तत्सद्ब्रह्मार्पणमस्तु।

\end{center}

\closesub

\sect{कथा}
\uvacha{अगस्त्य उवाच}
\twolineshloka
{रहस्यं कथयिष्यामि सुतीक्ष्ण मुनिसत्तम}
{चैत्रे नवम्यां प्राक्पक्षे दिवापुण्ये पुनर्वसौ}%॥ १ ॥

\twolineshloka
{उदये गुरुगौरांशे स्वोच्चस्थे ग्रहपञ्चके}
{मेष पूषणि सम्प्राप्ते लग्ने कर्कटकाह्वये}%॥ २ ॥

\twolineshloka
{आविरासीत्स कलया कौसल्यायां परः पुमान्}
{तस्मिन्दिने तु कर्तव्यमुपवासव्रतं सदा}%॥ ३ ॥

\twolineshloka
{तत्र जागरणं कुर्याद्रघुनाथपुरो भुवि}% भुवीतिखट्वादिव्यावृत्त्यर्थम्
{प्रतिमायां यथाशक्ति पूजा कार्या यथाविधि}%॥ ४ ॥

\twolineshloka
{प्रातर्दशम्यां स्नात्वैव कृत्वा सन्ध्यादिकाः क्रियाः}
{सम्पूज्य विधिवद् रामं भक्त्या वित्तानुसारतः}%॥ ५ ॥

\twolineshloka
{ब्राह्मणान् भोजयेत् सम्यक् दक्षिणाभिश्च तोषयेत्}
{गोभूतिलहिरण्याद्यैर्वस्त्रालङ्करणैस्तथा}%॥ ६ ॥

\twolineshloka
{रामभक्तान्प्रयत्नेन प्रीणयेत्परया मुदा}
{एवं यः कुरुते भक्त्या श्रीरामनवमीव्रतम्}%॥ ७ ॥

\twolineshloka
{अनेकजन्मसिद्धानि पापानि सुबहूनि च}
{भस्मीकृत्य व्रजत्येव तद्विष्णोः परमं पदम्}%॥ ८ ॥

\threelineshloka
{सर्वेषामप्ययं धर्मो भुक्तिमुक्त्येकसाधनः}
{अशुचिर्वाऽपि पापिष्ठः कृत्वेदं व्रतमुत्तमम्}
{पूज्यः स्यात्सर्वभूतानां यथा रामस्तथैव सः}%॥ ९ ॥

\twolineshloka
{यस्तु रामनवम्यां वै भुङ्क्ते स तु नराधमः}
{कुम्भीपाकेषु घोरेषु गच्छत्येव न संशयः}%॥ १० ॥

\twolineshloka
{अकृत्वा रामनवमीव्रतं सर्वव्रतोत्तमम्}
{व्रतान्यन्यानि कुरुते न तेषां फलभाग्भवेत्}%॥ ११ ॥

\twolineshloka
{रहस्यकृतपापानि प्रख्यातानि बहून्यपि}
{महान्ति च प्रणश्यन्ति श्रीरामनवमीव्रतात्}%॥ १२ ॥

\twolineshloka
{एकामपि नरो भक्त्या श्रीरामनवमीं मुने}
{उपोष्य कृतकृत्यः स्यात्सर्वपापैः प्रमुच्यते}%॥ १३ ॥

\twolineshloka
{नरो रामनवम्यां तु श्रीरामप्रतिमाप्रदः}
{विधानेन मुनिश्रेष्ठ स मुक्तो नात्र संशयः}%॥ १४ ॥

\uvacha{सुतीक्ष्ण उवाच}
\twolineshloka
{श्रीरामप्रतिमादानविधानं वा कथं मुने}
{कथय त्वं हि रामेऽपि भक्तस्य मम विस्तरात्}%॥ १५ ॥

\uvacha{अगस्त्य उवाच}
\onelineshloka{कथायिष्यामि तद्विद्वन् प्रतिमादानमुत्तमम्}%॥ १६ ॥

\twolineshloka
{विधानं चापि यत्नेन यतस्त्वं वैष्णवोत्तमः}
{अष्टम्यां चैत्रमासे तु शुक्लपक्षे जितेन्द्रियः}%॥ १७ ॥

\twolineshloka
{दन्तधावनपूर्वं तु प्रातः स्नायाद्यथाविधि}
{नद्यां तडागे कूपे वा ह्रदे प्रस्रवणेऽपि वा}%॥ १८ ॥

\twolineshloka
{ततः सन्ध्यादिका कार्याः संस्मरन् राघवं हृदि}
{गृहमासाद्य विप्रेन्द्र कुर्यादौपासनादिकम्}%॥ १९ ॥

\twolineshloka
{दान्तं कुटुम्बिनं विप्रं वेदशास्त्रपरं सदा}
{श्रीरामपूजानिरतं सुशीलं दम्भवर्जितम्}%॥ २० ॥

\twolineshloka
{विधिज्ञं राममन्त्राणां राममन्त्रैकसाधनम्}
{आहूय भक्त्या सम्पूज्य वृणुयात्प्रार्थयन्निति}%॥ २१ ॥

\twolineshloka
{श्रीरामप्रतिमादानं करिष्येऽहं द्विजोत्तम}
{तत्राचार्यो भव प्रीतः श्रीरामोऽसि त्वमेव च}%॥ २२ ॥

\twolineshloka
{इत्युक्त्वा पूज्य विप्रं तं स्नापयित्वा ततः परम्}
{तैलेनाभ्यज्य पयसा चिन्तयन्राघवं हृदि}%॥ २३ ॥

\twolineshloka
{श्वेताम्बरधरः श्वेतगन्धमाल्यानि धारयेत्}
{अर्चितो भूषितश्चैव कृतमाध्याह्निकक्रियः}%॥ २४ ॥

\twolineshloka
{आचार्यं भोजयेद् भक्त्या सात्त्विकान्नैः सुविस्तरम्}
{भुञ्जीत स्वयमप्येवं हृदि राममनुस्मरन्}%॥ २५ ॥

\twolineshloka
{एकभक्तव्रती तत्र सहाचार्यो जितेन्द्रियः}
{शृण्वन्रामकथां दिव्यामहःशेषं नयेन्मुने}%॥ २६ ॥

\twolineshloka
{सायं सन्ध्यादिकाः कुर्यात्क्रिया राममनुस्मरन्}
{आचार्यसहितो रात्रावधःशायी जितेन्द्रियः}%॥ २७ ॥

\twolineshloka
{वसेत्स्वयं न चैकान्ते श्रीरामार्पितमानसः}
{ततः प्रातः समुत्थाय स्नात्वा सन्ध्यां यथाविधि}%॥ २८ ॥

\twolineshloka
{प्रातः सर्वाणि कर्माणि शीघ्रमेव समापयेत्}
{ततः स्वस्थमना भूत्वा विद्वद्भिः सहितोऽनघ}%॥ २९ ॥

\twolineshloka
{स्वगृहे चोत्तरे देशे दानस्योज्ज्वलमण्डपम्}%स्वगृहे स्वगृहसमीपे॥
{चतुर्द्वारं पताकाढ्यं सवितानं सतोरणम्}%॥ ३० ॥

\twolineshloka
{मनोहरं महोत्सेधं पुष्पाद्यैः समलङ्कृतम्}
{शङ्खचक्रहनूमद्भिः प्राग्द्वारे समलङ्कृतम्}%॥ ३१ ॥

\twolineshloka
{गरुत्मच्छार्ङ्गबाणैश्च दक्षिणे समलङ्कृतम्}
{गदाखड्गाङ्गदैश्चैव पश्चिमे च विभूषितम्}%॥ ३२ ॥

\twolineshloka
{पद्मस्वस्तिकनीलैश्च कौबेर्यां समलङ्कृतम्}
{मध्यहस्तचतुष्काढ्यवेदिकायुक्तमायतम्}%॥ ३३ ॥

\twolineshloka
{प्रविश्य गीतनृत्यैश्च वाद्यैश्चापि समन्वितम्}
{पुण्याहं वाचयित्वा च विद्वद्भिः प्रीतमानसः}%॥ ३४ ॥

\twolineshloka
{ततः सङ्कल्पयेद्देवं राममेव स्मरन्मुने}
{अस्यां रामनवम्यां तु रामाराधनतत्परः}%॥ ३५ ॥

\twolineshloka
{उपोष्याष्टसु यामेषु पूजयित्वा यथाविधि}
{इमां स्वर्णमयीं रामप्रतिमां तु प्रयत्नतः}%॥ ३६ ॥

\twolineshloka
{श्रीरामप्रीतये दास्ये रामभक्ताय धीमते}
{प्रीतो रामो हरत्वाशु पापानि सुबहूनि मे}%॥ ३७ ॥

\twolineshloka
{अनेकजन्मसंसिद्धान्यभ्यस्तानि महान्ति च}
{विलिखेत्सर्वतोभद्रं वेदिकोपरि सुन्दरम्}%॥ ३८ ॥

\twolineshloka
{मध्ये तीर्थोदकैर्युक्तं पात्रं संस्थाप्य चार्चितम्}
{सौवर्णे राजते ताम्रे पात्रे षट्कोणमालिखेत्}%॥ ३९ ॥

\twolineshloka
{ततः स्वर्णमयीं रामप्रतिमां पलमात्रतः}
{निर्मितां द्विभुजां रम्यां वामाङ्कस्थितजानकीम्}%॥ ४० ॥

\twolineshloka
{बिभ्रतीं दक्षिणे हस्ते ज्ञानमुद्रां महामुने}
{वामेनाधःकरेणाराद्देवीमालिङ्ग्य संस्थिताम्}%॥ ४१ ॥

\twolineshloka
{सिंहासने राजते च पलद्वयविनिर्मिते}
{पञ्चामृतस्नानपूर्वं सम्पूज्य विधिवत्ततः}%॥ ४२ ॥

\twolineshloka
{मूलमन्त्रेण नियतो न्यासपूर्वमतन्द्रितः}
{दिवैवं विधिवत् कृत्वा रात्रौ जागरणं ततः}%॥ ४३ ॥

\twolineshloka
{दिव्यां रामकथां श्रुत्वा रामभक्तिसमन्वितः}
{गीतनृत्यादिभिश्चैव रामस्तोत्रैरनेकधा}%॥ ४४ ॥

\twolineshloka
{रामाष्टकैश्च संस्तुत्य गन्धपुष्पाक्षतादिभिः}
{कर्पूरागुरुकस्तूरीकह्लाराद्यैरनेकधा}%॥ ४५ ॥

\twolineshloka
{सम्पूज्य विधिवद् भक्त्या दिवारात्रं नयेद्बुधः}
{ततः प्रातः समुत्थाय स्नानसन्ध्यादिकाः क्रियाः}%॥ ४६ ॥

\twolineshloka
{समाप्य विधिवद्रामं पूजयेद्विधिवन्मुने}
{ततो होमं प्रकुर्वीत मूलमन्त्रेण मन्त्रवित्}%॥ ४७ ॥

\twolineshloka
{पूर्वोक्तपद्मकुण्डे वा स्थण्डिले वा समाहितः}
{लौकिकाग्नौ विधानेन शतमष्टोत्तरं मुने}%॥ ४८ ॥

\twolineshloka
{साज्येन पायसेनैव स्मरन्राममनन्यधीः}
{ततो भक्त्या सुसन्तोष्य आचार्यं पूजयेन्मुने}%॥ ४९ ॥

\twolineshloka
{कुण्डलाभ्यां सरत्नाभ्यामङ्गुलीयैरनेकधा}
{गन्धपुष्पाक्षतैर्वस्त्रैर्विचित्रैस्तु मनोहरैः}%॥ ५० ॥

\twolineshloka
{ततो रामं स्मरन्दद्यादिमं मन्त्रमुदीरयेत्}
{इमां स्वर्णमयीं रामप्रतिमां समलङ्कृताम्}%॥ ५१ ॥

\twolineshloka
{चित्रवस्त्रयुगच्छन्नरामोऽहं राघवाय ते}
{श्रीरामप्रीतये दास्ये तुष्टो भवतु राघवः}%॥ ५२ ॥

\twolineshloka
{इति दत्त्वा विधानेन दद्याद्वै दक्षिणां ध्रुवम्}
{अन्नेभ्यश्च यथाशक्त्या गोहिरण्यादि भक्तितः}%॥ ५३ ॥

\twolineshloka
{दद्याद्वासोयुगं धान्यं तथाऽलङ्करणानि च}
{एवं यः कुरुते रामप्रतिमादानमुत्तमम्}%॥ ५४ ॥

\twolineshloka
{ब्रह्महत्यादिपापेभ्यो मुच्यते नात्र संशयः}
{तुलापुरुषदानादिफलमाप्नोति सुव्रत}%॥ ५५ ॥

\twolineshloka
{अनेकजन्मसंसिद्धपापेभ्यो मुच्यते ध्रुवम्}
{बहुनाऽत्र किमुक्तेन मुक्तिस्तस्य करे स्थिता}%॥ ५६ ॥

\threelineshloka
{कुरुक्षेत्रे महापुण्ये सूर्यपर्वण्यशेषतः}
{तुलापुरुषदानाद्यैः कृतैर्यल्लभते फलम्}
{तत्फलं लभते मर्त्यो दानेनानेन सुव्रत}% ॥५७॥

\uvacha{सुतीक्ष्ण उवाच}
\twolineshloka
{प्रायेण हि नराः सर्वे दरिद्राः कृपणा मुने}
{कैः कर्तव्यं कथमिदं व्रतं ब्रूहि महामुने}%॥ ५८ ॥

\uvacha{अगस्त्य उवाच}
\onelineshloka
{दरिद्रश्च महाभाग स्वस्य वित्तानुसारतः}%॥ ५९ ॥

\twolineshloka
{पलार्धेन तदर्धेन तदर्धार्धेन वा पुनः}
{वित्तशाठ्यमकृत्वैव कुर्यादेवं व्रतं मुने}%॥ ६० ॥

\twolineshloka
{यदि घोरतरं दुष्टं पातकं नेहते क्वचित्}
{अकिञ्चनोऽपि यत्नेन उपोष्य नवमीदिने}%॥ ६१ ॥

\twolineshloka
{एकचित्तोऽपि विधिवत्सर्वपापैः प्रमुच्यते}
{प्रातःस्नानं च विधिवत्कृत्वा सन्ध्यादिकाः क्रियाः}%॥ ६२ ॥

\twolineshloka
{गोभूतिलहिरण्यादि दद्याद्वित्तानुसारतः}
{श्रीरामचन्द्रभक्तेभ्यो विद्वद्भ्यः श्रद्धयान्वितः}%॥ ६३ ॥

\twolineshloka
{पारणं त्वथ कुर्वीत ब्राह्मणैश्च स्वबन्धुभिः}
{एवं यः कुरुते भक्त्या सर्वपापैः प्रमुच्यते}%॥ ६४ ॥

\twolineshloka
{प्राप्ते श्रीरामनवमीदिने मर्त्यो विमूढधीः}
{उपोषणं न कुरुते कुम्भीपाकेषु पच्यते}%॥ ६५ ॥

\twolineshloka
{यत्किञ्चिद्राममुद्दिश्य क्रियते न स्वशक्तितः}
{रौरवे स तु मूढात्मा पच्यते नात्र संशयः}%॥ ६६ ॥

\uvacha{सुतीक्ष्ण उवाच}
\twolineshloka
{यामाष्टके तु पूजा वै तत्र चोक्ता महामुने}
{मूलमन्त्रेण संयुक्ता तां कथां वद सुव्रत}%॥ ६७ ॥

\uvacha{अगस्त्य उवाच}
\twolineshloka
{सर्वेषां राममन्त्राणां मन्त्रराज षडक्षरम्} %इदं तु स्कान्दे मोक्षखण्डे श्रीरामं प्रतिरुद्रगीतायां रुद्रवाक्यम्
{मुमूर्षोर्मणिकर्ण्यान्ते अर्धोदकनिवासिनः}%॥ ६८ ॥

\twolineshloka
{अहं दिशामि ते मन्त्रं तारकस्योपदेशतः}
{श्रीराम राम रामेति एतत्तारकमुच्यते}%॥ ६९ ॥

\twolineshloka
{अतस्त्वं जानकीनाथपरं ब्रह्माभिधीयसे}
{तारकं ब्रह्म चेत्युक्तं तेन पूजा प्रशस्यते}%॥ ७० ॥

\twolineshloka
{पीठाङ्गदेवतानां तु आवृत्तीनां तथैव च}
{आदावेव प्रकुर्वीत देवस्य प्रीतमानस}%॥ ७१ ॥

\twolineshloka
{उपचारैःषोडशभिः पूजा कार्या यथाविधि}
{आवाहनं स्थापनं च सम्मुखीकरणं तथा}%॥ ७२ ॥

\twolineshloka
{एवं मुद्रां प्रार्थनां च पूजामुद्रां प्रयत्नतः}
{शङ्खपूजां प्रकुर्वीत पूर्वोक्तविधिना ततः}%॥ ७३ ॥

\twolineshloka
{कलशं वामभागे च पूजाद्रव्याणि चादरात्}
{पीठे सम्पूज्य यत्नेन आत्मानं मन्त्रमुच्चरेत्}%॥ ७४ ॥

\twolineshloka
{पात्रासादनमप्येवं कुर्याद्यामेष्वतन्द्रितः}
{पीताम्बराणि देवाय प्रार्पयन्नर्चयेत्सुधीः}%॥ ७५ ॥

\twolineshloka
{स्वर्णयज्ञोपवीतानि दद्याद्देवाय भक्तितः}
{नानारत्नविचित्राणि दद्यादाभरणानि च}%॥ ७६ ॥

\twolineshloka
{हिमाम्बुघृष्टं रुचिरं घनसारमनोहरम्}
{क्रमात्तु मूलमन्त्रेण उपचारान्प्रकल्पयेत्}%॥ ७७ ॥

\twolineshloka
{कह्लारैः केतकैर्जात्यैः पुन्नागाद्यैः प्रपूजयेत्}
{चम्पकैः शतपत्रैश्च सुगन्धैः सुमनोहरैः}%॥ ७८ ॥

\twolineshloka
{पाद्यचन्दनधूपैश्च तत्तन्मन्त्रैः प्रपूजयेत्}
{भक्ष्यभोज्यादिकं भक्त्या देवाय विधिनाऽर्पयेत्}%॥ ७९ ॥

\twolineshloka
{येन सोपस्करं देवं दत्त्वा पापैः प्रमुच्यते}
{जन्मकोटिकृतैर्घोरैर्नानारूपैश्च दारुणः}%॥ ८० ॥

\twolineshloka
{विमुक्तः स्यात्क्षणादेव राम एव भवेन्मुने}
{श्रद्दधानस्य दातव्यं श्रीरामनवमीव्रतम्}%॥ ८१ ॥

\twolineshloka
{सर्वलोकहितायेदं पवित्रं पापनाशनम्}
{लोहेन निर्मितं वाऽपि शिलया दारुणाऽपि वा}%॥ ८२ ॥

\twolineshloka
{एकेनैव प्रकारेण यस्मै कस्मै च वा मुने}
{कृतं सर्वं प्रयत्नेन यत्किञ्चिदपि भक्तितः}%॥ ८३ ॥

\twolineshloka
{जपेदेकान्तमासीनो यावत्स दशमीदिनम्}
{अनेन स्यात्पुनः पूजा दशम्यां भोजयेद् द्विजान्}%॥ ८४ ॥

\twolineshloka
{भक्त्या भोज्यैर्बहुविधैर्दद्याद् भक्त्या च दक्षिणाम्}
{कृतकृत्यो भवेत्तेन सद्यो रामः प्रसीदति}%॥ ८५ ॥

\twolineshloka
{तूष्णीं तिष्ठन्नरो वाऽपि पुनरावृत्तिवर्जितः}
{द्वादशाब्दे कृतेनापि यत्पापं चापि मुच्यते}%॥ ८६ ॥
 
\twolineshloka
{विलयं याति तत्सर्वं श्रीरामनवमीव्रतम्}
{जपं च रामनन्त्राणां यो न जानाति तस्य वै}%॥ ८७ ॥

\twolineshloka
{उपोष्य संस्मरेद्रामं न्यासपूर्वमतन्द्रितः}
{गुरोर्लब्धमिमं मन्त्रं न्यसेन्न्यासपुरःसरम्}%॥ ८८ ॥

\threelineshloka
{यामे यामे च विधिना कुर्यात्पूजां समाहितः}
{मुमुक्षुश्च सदा कुर्याच्छ्रीरामनवमीत्रतम्}
{मुच्यते सर्वपापेभ्यो याति ब्रह्म सनातनम्}%॥ ८९ ॥

॥इति श्रीस्कान्दपुराणे अगस्त्यसंहितायामगस्तिसुतीक्ष्णसंवादे रामनवमी\-व्रत\-विधिः सम्पूर्णः॥


\closesection