% !TeX program = XeLaTeX
% !TeX root = ../pujavidhanam.tex
\twolineshloka*
{शुक्लाम्बरधरं विष्णुं शशिवर्णं चतुर्भुजम्}
{प्रसन्नवदनं ध्यायेत् सर्वविघ्नोपशान्तये}
 
प्राणान्  आयम्य।  ॐ भूः + भूर्भुवः॒ सुव॒रोम्।

ममोपात्तसमस्तदुरितक्षयद्वारा श्रीपरमेश्वरप्रीत्यर्थं शुभे शोभने मुहूर्ते अद्यब्रह्मणः
द्वितीयपरार्द्धे श्वेतवराहकल्पे वैवस्वतमन्वन्तरे अष्टाविंशतितमे कलियुगे प्रथमे पादे
जम्बूद्वीपे भारतवर्षे भरतखण्डे मेरोः दक्षिणेपार्श्वे शकाब्दे अस्मिन् वर्तमाने व्यावहारिकणां
 प्रभवादि षष्ट्याः संवत्सराणां मध्ये (  )\see{app:samvatsara_names} नाम संवत्सरे \textbf{उत्तरायणे} 
\textbf{शिशिर}-ऋतौ  \textbf{कुम्भ}-मासे \textbf{कृष्ण}-पक्षे त्र्योदश्यां/चतुर्दश्यां शुभतिथौ
(इन्दु / भौम / बुध / गुरु / भृगु / स्थिर / भानु) वासरयुक्तायाम्
(  )\see{app:nakshatra_names} नक्षत्र (  )\see{app:yoga_names} नाम  योग  (  ) करण युक्तायां च एवं गुण विशेषण विशिष्टायाम्
अस्याम् (त्र्योदश्यां/चतुर्दश्यां) शुभतिथौ 
अस्माकं सहकुटुम्बानां क्षेमस्थैर्य-धैर्य-वीर्य-विजय आयुरारोग्य ऐश्वर्याभिवृद्ध्यर्थम्
 धर्मार्थकाममोक्ष\-चतुर्विधफलपुरुषार्थसिद्ध्यर्थं पुत्रपौत्राभि\-वृद्ध्यर्थम् इष्टकाम्यार्थसिद्ध्यर्थम्
मम इहजन्मनि पूर्वजन्मनि जन्मान्तरे च सम्पादितानां ज्ञानाज्ञानकृतमहा\-पातकचतुष्टय
व्यतिरिक्तानां रहस्यकृतानां प्रकाशकृतानां सर्वेषां पापानां सद्य अपनोदनद्वारा सकल 
पापक्षयार्थं शिवरात्रौ श्री-साम्ब-परमेश्वर-प्रीत्यर्थं प्रथमयामपूजां करिष्ये।



श्रीविघ्नेश्वराय नमः यथास्थानं प्रतिष्ठापयामि।\\
(गणपति प्रसादं शिरसा गृहीत्वा)
\renewcommand{\devaName}{विष्णु}
\dnsub{आसन-पूजा}
\centerline{पृथिव्या  मेरुपृष्ठ  ऋषिः।  सुतलं  छन्दः।  कूर्मो  देवता॥}
\twolineshloka*
{पृथ्वि  त्वया  धृता  लोका  देवि  त्वं  विष्णुना  धृता}
{त्वं  च  धारय  मां  देवि  पवित्रं  चाऽऽसनं  कुरु}


\dnsub{घण्टापूजा}
\twolineshloka*
{आगमार्थं तु देवानां गमनार्थं तु रक्षसाम्}
{घण्टारवं करोम्यादौ देवताऽऽह्वानकारणम्}


\dnsub{कलशपूजा}
ॐ कलशाय नमः दिव्यगन्धान् धारयामि।\\
ॐ गङ्गायै नमः। ॐ यमुनायै नमः। ॐ गोदावर्यै नमः।  ॐ सरस्वत्यै नमः। ॐ नर्मदायै नमः। ॐ सिन्धवे नमः। ॐ कावेर्यै नमः।\\
ॐ सप्तकोटिमहातीर्थान्यावाहयामि।\\[-0.25ex]

(अथ कलशं स्पृष्ट्वा जपं कुर्यात्) \\
आपो॒ वा इ॒द सर्वं॒ विश्वा॑ भू॒तान्याप॑ प्रा॒णा वा आप॑ प॒शव॒ आपो\-ऽन्न॒मापोऽमृ॑त॒माप॑ स॒म्राडापो॑ वि॒राडाप॑ स्व॒राडाप॒श्\-छन्दा॒स्यापो॒ ज्योती॒ष्यापो॒ यजू॒ष्याप॑ स॒त्यमाप॒ सर्वा॑ दे॒वता॒ आपो॒ भूर्भुव॒ सुव॒राप॒ ओम्॥\\

\twolineshloka* 
{कलशस्य मुखे विष्णुः कण्ठे रुद्रः समाश्रितः}
{मूले तत्र स्थितो ब्रह्मा मध्ये मातृगणाः स्मृताः}
\threelineshloka* 
{कुक्षौ तु सागराः सर्वे सप्तद्वीपा वसुन्धरा}
{ऋग्वेदोऽथ यजुर्वेदः सामवेदोऽप्यथर्वणः}
{अङ्गैश्च सहिताः सर्वे कलशाम्बुसमाश्रिताः}
\twolineshloka* 
{गङ्गे च यमुने चैव गोदावरि सरस्वति}
{नर्मदे सिन्धुकावेरि जलेऽस्मिन् सन्निधिं कुरु}
\twolineshloka*
{सर्वे समुद्राः सरितः तीर्थानि च ह्रदा नदाः}
{आयान्तु देवपूजार्थं दुरितक्षयकारकाः}

\centerline{ॐ भूर्भुवः॒ सुवो॒ भूर्भुवः॒ सुवो॒ भूर्भुवः॒ सुवः॑।}

(इति कलशजलेन सर्वोपकरणानि आत्मानं च प्रोक्ष्य।)


\dnsub{आत्म-पूजा}
ॐ आत्मने नमः, दिव्यगन्धान् धारयामि।
\begin{multicols}{2}
१. ॐ आत्मने नमः\\
२. ॐ अन्तरात्मने नमः\\
३. ॐ योगात्मने नमः\\
४. ॐ जीवात्मने नमः\\
५. ॐ परमात्मने नमः\\
६. ॐ ज्ञानात्मने नमः
\end{multicols}
समस्तोपचारान् समर्पयामि।

\twolineshloka*
{देहो देवालयः प्रोक्तो जीवो देवः सनातनः}
{त्यजेदज्ञाननिर्माल्यं सोऽहं भावेन पूजयेत्}


\begin{minipage}{\linewidth}
\dnsub{पीठ-पूजा}

\begin{multicols}{2}
\begin{enumerate}
\item ॐ आधारशक्त्यै नमः
\item ॐ मूलप्रकृत्यै नमः
\item ॐ आदिकूर्माय नमः 
\item ॐ आदिवराहाय नमः
\item ॐ अनन्ताय नमः
\item ॐ पृथिव्यै नमः
\item ॐ रत्नमण्डपाय नमः
\item ॐ रत्नवेदिकायै नमः
\item ॐ स्वर्णस्तम्भाय नमः
\item ॐ श्वेतच्छत्त्राय नमः
\item ॐ कल्पकवृक्षाय नमः
\item ॐ क्षीरसमुद्राय नमः 
\item ॐ सितचामराभ्यां नमः
\item ॐ योगपीठासनाय नमः
\end{enumerate}
\end{multicols}

\end{minipage}

\dnsub{गुरु ध्यानम्}

\twolineshloka*
{गुरुर्ब्रह्मा गुरुर्विष्णुर्गुरुर्देवो महेश्वरः}
{गुरुः साक्षात् परं ब्रह्म तस्मै श्री गुरवे नमः}


\begin{center}
\sect{षोडशोपचारपूजा}

\fourlineindentedshloka*
{ध्यायेन्नित्यं महेशं रजतगिरिनिभं चारुचन्द्रावतंसम्}
{रत्नाकल्पोज्ज्वलाङ्गं परशुमृगवराभीतिहस्तं प्रसन्नम्}
{पद्मासीनं समन्तात् स्तुतममरगणैर्व्याघ्रकृत्तिं वसानम्}
{विश्वाद्यं विश्वबीजं निखिलभयहरं पञ्चवक्त्रं त्रिनेत्रम्}

अस्मिन् बिम्बे श्रीभूमिनीलासमेतं महाविष्णुं ध्यायामि। 
% प्राण-प्रतिष्ठा॥
% \medskip

% \twolineshloka*
% {स॒हस्र॑शीर्‌षा॒ पुरु॑षः। स॒ह॒स्रा॒क्षः स॒हस्र॑पात्}
% {स भूमिं॑ वि॒श्वतो॑ वृ॒त्वा। अत्य॑तिष्ठद्दशाङ्गु॒लम्}
% ॐ भूः पुरुषं साम्ब-सदाशिवमावाहयामि।
% ॐ भुवः पुरुषं साम्ब-सदाशिवमावाहयामि।
% ओꣳ सुवः पुरुषं साम्ब-सदाशिवमावाहयामि।
% ॐ भूर्भुवःसुवः साम्ब-सदाशिवमावाहयामि॥
% \medskip

% \twolineshloka*
% {स्वामिन् सर्वजगन्नाथ यावत्पूजावसानकम्}
% {तावत् त्वं प्रीतिभावेन लिङ्गेऽस्मिन् सन्निधो भव}
% इति पुष्पाञ्जलिं दद्यात्॥ 

नम॑स्ते रुद्र म॒न्यव॑ उ॒तो त॒ इष॑वे॒ नमः॑। नम॑स्ते अस्तु॒ धन्व॑ने बा॒हुभ्या॑मु॒त ते॒ नमः॑॥ ॐ ह्रीं न॒मः शि॒वाय॑। स॒द्योजा॒तं प्र॑पद्या॒मि॒।

या त॒ इषुः॑ शि॒वत॑मा शि॒वं ब॒भूव॑ ते॒ धनुः॑। शि॒वा श॑र॒व्या॑ या तव॒ तया॑ नो रुद्र मृडय॥ ॐ ह्रीं न॒मः शि॒वाय॑। स॒द्योजा॒ताय॒ वै नमो॒ नमः॑। आसनं समर्पयामि॥२॥

या ते॑ रुद्र शि॒वा त॒नूरघो॒राऽपा॑पकाशिनी। तया॑ नस्त॒नुवा॒ शन्त॑मया॒ गिरि॑शन्ता॒\-भिचा॑कशीहि॥ ॐ ह्रीं न॒मः शि॒वाय॑। भ॒वे भ॑वे॒ नाति॑ भवे भवस्व॒ माम्। पादयोः पाद्यं समर्पयामि॥३॥

यामिषुं॑ गिरिशन्त॒ हस्ते॒ बिभ॒र्ष्यस्त॑वे। शि॒वां गि॑रित्र॒ तां कु॑रु॒ मा हिꣳ॑सीः॒ पुरु॑षं॒ जग॑त्॥ ॐ ह्रीं न॒मः शि॒वाय॑। भ॒वोद्भ॑वाय॒ नमः॑॥ अर्घ्यं समर्पयामि॥४॥

शि॒वेन॒ वच॑सा त्वा॒ गिरि॒शाच्छा॑वदामसि। यथा॑ नः॒ सर्व॒मिज्जग॑दय॒क्ष्मꣳ सु॒मना॒ अस॑त्॥ ॐ ह्रीं न॒मः शि॒वाय॑। वा॒म॒दे॒वाय॒ नमः॑। आचमनीयं समर्पयामि॥५॥

अध्य॑वोचदधिव॒क्ता प्र॑थ॒मो दैव्यो॑ भि॒षक्। अहीꣴ॑श्च॒ सर्वा᳚ञ्ज॒म्भय॒न्थ्सर्वा᳚श्च यातुधा॒न्यः॑॥ ॐ ह्रीं न॒मः शि॒वाय॑। ज्ये॒ष्ठाय॒ नमः॑। मधुपर्कं समर्पयामि॥६॥

अ॒सौ यस्ता॒म्रो अ॑रु॒ण उ॒त ब॒भ्रुः सु॑म॒ङ्गलः॑। ये चे॒माꣳ रु॒द्रा अ॒भितो॑ दि॒क्षु श्रि॒ताः स॑हस्र॒शोऽवै॑षा॒ꣳ॒ हेड॑ ईमहे॥ ॐ ह्रीं न॒मः शि॒वाय॑। श्रे॒ष्ठाय॒ नमः॑। स्नानं समर्पयामि। 


रुद्रम्। चमकम्। पुरुषस्क्तम्॥


स्नानानन्तरम् आचमनीयं समर्पयामि॥७॥

साक्षतजलेन तर्पणं कार्यम्॥

ॐ भवं देवं तर्पयामि। ॐ शर्वं देवं तर्पयामि। ॐ ईशानं देवं तर्पयामि। ॐ पशुपतिं देवं तर्पयामि। ॐ रुद्रं देवं तर्पयामि। ॐ उग्रं देवं तर्पयामि। ॐ भीमं देवं तर्पयामि। ॐ महान्तं देवं तर्पयामि॥

ॐ भवस्य देवस्य पत्नीं तर्पयामि। ॐ शर्वस्य देवस्य पत्नीं तर्पयामि। ॐ ईशानस्य देवस्य पत्नीं तर्पयामि। ॐ पशुपतेर्देवस्य पत्नीं तर्पयामि। ॐ रुद्रस्य देवस्य पत्नीं तर्पयामि। ॐ उग्रस्य देवस्य पत्नीं तर्पयामि। ॐ भीमस्य देवस्य पत्नीं तर्पयामि। ॐ महतो देवस्य पत्नीं तर्पयामि॥


अ॒सौ यो॑ऽव॒सर्प॑ति॒ नील॑ग्रीवो॒ विलो॑हितः। उ॒तैनं॑ गो॒पा अ॑दृश॒न्न॒दृ॑शन्नुदहा॒र्यः॑। उ॒तैनं॒ विश्वा॑ भू॒तानि॒ स दृ॒ष्टो मृ॑डयाति नः॥ ॐ ह्रीं न॒मः शि॒वाय॑। रु॒द्राय॒ नमः॑। वस्त्रोत्तरीयं समर्पयामि॥८॥

नमो॑ अस्तु॒ नील॑ग्रीवाय सहस्रा॒क्षाय॑ मी॒ढुषे᳚। अथो॒ ये अ॑स्य॒ सत्वा॑नो॒ऽहं तेभ्यो॑ऽकरं॒ नमः॑॥ ॐ ह्रीं न॒मः शि॒वाय॑। काला॑य॒ नमः॑। यज्ञोपवीताभरणानि समर्पयामि॥९॥

प्र मु॑ञ्च॒ धन्व॑न॒स्त्वमु॒भयो॒रार्त्नि॑यो॒र्ज्याम्। याश्च॑ ते॒ हस्त॒ इष॑वः॒ परा॒ ता भ॑गवो वप॥ ॐ ह्रीं न॒मः शि॒वाय॑। कल॑विकरणाय॒ नमः॑। दिव्यपरिमलगन्धान् धारयामि। गन्धस्योपरि अक्षतान् समर्पयामि॥१०॥

अ॒व॒तत्य॒ धनु॒स्त्वꣳ सह॑स्राक्ष॒ शते॑षुधे। नि॒शीर्य॑ श॒ल्यानां॒ मुखा॑ शि॒वो नः॑ सु॒मना॑ भव॥ ॐ ह्रीं न॒मः शि॒वाय॑। बल॑विकरणाय॒ नमः॑। पुष्पैः पूजयामि॥११॥


ॐ भवाय देवाय नमः। ॐ शर्वाय देवाय नमः। \\
ॐ ईशानाय देवाय नमः। ॐ पशुपतये देवाय नमः।\\
ॐ रुद्राय देवाय नमः। ॐ उग्राय देवाय नमः।\\
ॐ भीमाय देवाय नमः। ॐ महते देवाय नमः॥\\
ॐ भवस्य देवस्य पत्न्यै नमः। ॐ शर्वस्य देवस्य पत्न्यै नमः।\\
ॐ ईशानस्य देवस्य पत्न्यै नमः। ॐ पशुपतेर्देवस्य पत्न्यै नमः।\\
ॐ रुद्रस्य देवस्य पत्न्यै नमः। ॐ उग्रस्य देवस्य पत्न्यै नमः।\\
ॐ भीमस्य देवस्य पत्न्यै नमः। ॐ महतो देवस्य पत्न्यै नमः॥ \\
\end{center}

\sect{शिवाष्टोत्तरशतनामावलिः}
\begin{multicols}{2}
\begin{flushleft}
ॐ शिवाय~नमः\\
ॐ महेश्वराय~नमः\\
ॐ शम्भवे~नमः\\
ॐ पिनाकिने~नमः\\
ॐ शशिशेखराय~नमः\\
ॐ वामदेवाय~नमः\\
ॐ विरूपाक्षाय~नमः\\
ॐ कपर्दिने~नमः\\
ॐ नीललोहिताय~नमः\\
ॐ शङ्कराय~नमः\hfill\devanumber{10}\\
ॐ शूलपाणिने~नमः\\
ॐ खट्वाङ्गिने~नमः\\
ॐ विष्णुवल्लभाय~नमः\\
ॐ शिपिविष्टाय~नमः\\
ॐ अम्बिकानाथाय~नमः\\
ॐ श्रीकण्ठाय~नमः\\
ॐ भक्तवत्सलाय~नमः\\
ॐ भवाय~नमः\\
ॐ शर्वाय~नमः\\
ॐ त्रिलोकेशाय~नमः\hfill\devanumber{20}\\
ॐ शितिकण्ठाय~नमः\\
ॐ शिवाप्रियाय~नमः\\
ॐ उग्राय~नमः\\
ॐ कपालिने~नमः\\
ॐ कामारये~नमः\\
ॐ अन्धकासुरसूदनाय~नमः\\
ॐ गङ्गाधराय~नमः\\
ॐ ललाटाक्षाय~नमः\\
ॐ कालकालाय~नमः\\
ॐ कृपानिधये~नमः\hfill\devanumber{30}\\
ॐ भीमाय~नमः\\
ॐ परशुहस्ताय~नमः\\
ॐ मृगपाणये~नमः\\
ॐ जटाधराय~नमः\\
ॐ कैलासवासिने~नमः\\
ॐ कवचिने~नमः\\
ॐ कठोराय~नमः\\
ॐ त्रिपुरान्तकाय~नमः\\
ॐ वृषाङ्काय~नमः\\
ॐ वृषभारूढाय~नमः\hfill\devanumber{40}\\
ॐ भस्मोद्धूलितविग्रहाय~नमः\\
ॐ सामप्रियाय~नमः\\
ॐ स्वरमयाय~नमः\\
ॐ त्रयीमूर्तये~नमः\\
ॐ अनीश्वराय~नमः\\
ॐ सर्वज्ञाय~नमः\\
ॐ परमात्मने~नमः\\
ॐ सोमसूर्याग्निलोचनाय~नमः\\
ॐ हविषे ~नमः\\
ॐ यज्ञमयाय~नमः\hfill\devanumber{50}\\
ॐ सोमाय~नमः\\
ॐ पञ्चवक्त्राय~नमः\\
ॐ सदाशिवाय~नमः\\
ॐ विश्वेश्वराय~नमः\\
ॐ वीरभद्राय~नमः\\
ॐ गणनाथाय~नमः\\
ॐ प्रजापतये~नमः\\
ॐ हिरण्यरेतसे~नमः\\
ॐ दुर्धर्षाय~नमः\\
ॐ गिरीशाय~नमः\hfill\devanumber{60}\\
ॐ गिरिशाय~नमः\\
ॐ अनघाय~नमः\\
ॐ भुजङ्गभूषणाय~नमः\\
ॐ भर्गाय~नमः\\
ॐ गिरिधन्वने~नमः\\
ॐ गिरिप्रियाय~नमः\\
ॐ कृत्तिवाससे~नमः\\
ॐ पुरारातये~नमः\\
ॐ भगवते~नमः\\
ॐ प्रमथाधिपाय~नमः\hfill\devanumber{70}\\
ॐ मृत्युञ्जयाय~नमः\\
ॐ सूक्ष्मतनवे~नमः\\
ॐ जगद्व्यापिने~नमः\\
ॐ जगद्गुरवे~नमः\\
ॐ व्योमकेशाय~नमः\\
ॐ महासेनजनकाय~नमः\\
ॐ चारुविक्रमाय~नमः\\
ॐ रुद्राय~नमः\\
ॐ भूतपतये~नमः\\
ॐ स्थाणवे~नमः\hfill\devanumber{80}\\
ॐ अहये बुध्न्याय~नमः\\
ॐ दिगम्बराय~नमः\\
ॐ अष्टमूर्तये~नमः\\
ॐ अनेकात्मने~नमः\\
ॐ सात्त्विकाय~नमः\\
ॐ शुद्धविग्रहाय~नमः\\
ॐ शाश्वताय~नमः\\
ॐ खण्डपरशवे~नमः\\
ॐ अजाय ~नमः\\
ॐ पाशविमोचकाय~नमः\hfill\devanumber{90}\\
ॐ मृडाय~नमः\\
ॐ पशुपतये~नमः\\
ॐ देवाय~नमः\\
ॐ महादेवाय~नमः\\
ॐ अव्ययाय~नमः\\
ॐ हरये~नमः\\
ॐ पूषदन्तभिदे~नमः\\
ॐ अव्यग्राय~नमः\\
ॐ दक्षाध्वरहराय~नमः\\
ॐ हराय~नमः\hfill\devanumber{100}\\
ॐ भगनेत्रभिदे~नमः\\
ॐ अव्यक्ताय~नमः\\
ॐ सहस्राक्षाय~नमः\\
ॐ सहस्रपदे~नमः\\
ॐ अपवर्गप्रदाय~नमः\\
ॐ अनन्ताय~नमः\\
ॐ तारकाय~नमः\\
ॐ परमेश्वराय~नमः\\
\end{flushleft}
\end{multicols}
॥इति शाक्तप्रमोदे श्री शिवाष्टोत्तरशतनामावलिः सम्पूर्णा॥

\sect{उत्तराङ्ग-पूजा}



विज्यं॒ धनुः॑ कप॒र्दिनो॒ विश॑ल्यो॒ बाण॑वाꣳ उ॒त। अने॑शन्न॒\-स्येष॑व आ॒भुर॑स्य निष॒ङ्गथिः॑॥ ॐ ह्रीं न॒मः शि॒वाय॑। बला॑य॒ नमः॑। धूपमाघ्रापयामि॥१२॥

या ते॑ हे॒तिर्मी॑ढुष्टम॒ हस्ते॑ ब॒भूव॑ ते॒ धनुः॑। तया॒ऽस्मान् वि॒श्वत॒स्त्वम॑य॒क्ष्मया॒ परि॑ब्भुज॥ ॐ ह्रीं न॒मः शि॒वाय॑। बल॑प्रमथनाय॒ नमः॑। अलङ्कारदीपं सन्दर्शयामि॥१३॥

ॐ भूर्भुवः॒ सुवः॑। + ब्र॒ह्मणे॒ स्वाहा᳚। नम॑स्ते अ॒स्त्वायु॑धा॒याना॑तताय धृ॒ष्णवे᳚। उ॒भाभ्या॑मु॒त ते॒ नमो॑ बा॒हुभ्यां॒ तव॒ धन्व॑ने॥ ॐ ह्रीं न॒मः शि॒वाय॑। सर्व॑भूतदमनाय॒ नमः॑। () निवेदयामि। मध्ये मध्ये अमृतपानीयं समर्पयामि। अमृतापिधानमसि।\\
हस्तप्रक्षालनं समर्पयामि। पादप्रक्षालनं समर्पयामि। निवेदनानन्तरम् आचमनीयं समर्पयामि॥१४॥

परि॑ ते॒ धन्व॑नो हे॒तिर॒स्मान्वृ॑णक्तु वि॒श्वतः॑। अथो॒ य इ॑षु॒धिस्तवा॒ऽ॒ऽ॒रे अ॒स्मन्नि धे॑हि॒ तम्॥ ॐ ह्रीं न॒मः शि॒वाय॑। म॒नोन्म॑नाय॒ नमः॑। कर्पूरताम्बूलं समर्पयामि॥१५॥

नम॑स्ते अस्तु भगवन् विश्वेश्व॒राय॑ महादे॒वाय॑ त्र्यम्ब॒काय॑ त्रिपुरान्त॒काय॑ त्रिकाग्निका॒लाय॑ कालाग्निरु॒द्राय॑ नीलक॒ण्ठाय॑ मृत्युञ्ज॒याय॑ सर्वेश्व॒राय॑ सदाशि॒वाय॑ श्रीमन्महादे॒वाय॒ नमः॑॥ कर्पूरनीराजनं दर्शयामि॥१६॥

\dnsub{रक्षा}
बृ॒हथ्साम॑ क्षत्र॒भृद्वृ॒द्ध वृ॑ष्णियं त्रि॒ष्टुभौजः॑ शुभि॒तमु॒ग्रवी॑रम्।
इन्द्र॒स्तोमे॑न पञ्चद॒शेन॒ मध्य॑मि॒दं वाते॑न॒ सग॑रेण रक्ष॥

रक्षां धारयामि॥

\dnsub{नमस्काराः}
ॐ भवाय देवाय नमः। 

ॐ शर्वाय देवाय नमः। 

ॐ ईशानाय देवाय नमः। 

ॐ पशुपतये देवाय नमः। 

ॐ रुद्राय देवाय नमः। 

ॐ उग्राय देवाय नमः। 

ॐ भीमाय देवाय नमः। 

ॐ महते देवाय नमः॥


ईशानः सर्व॑विद्या॒ना॒मीश्वरः सर्व॑भूता॒नां॒ ब्रह्माधि॑पति॒र्ब्रह्म॒णो\-ऽधि॑पति॒र्ब्रह्मा॑ शि॒वो मे॑ अस्तु सदाशि॒वोम्॥ ॐ ह्रीं न॒मः शि॒वाय॑। मन्त्रपुष्पाञ्जलिं समर्पयामि।   


शिवाय नमः। रुद्राय नमः। पशुपतये नमः। नीलकण्ठाय नमः। महेश्वराय नमः। हरिकेशाय नमः। विरूपाक्षाय नमः। पिनाकिने नमः। त्रिपुरान्तकाय नमः। शम्भव नमः। शूलिने नमः। महादेवाय नमः। इति द्वादशनामभिर्द्वादशपुष्पाञ्जलीन् दत्वा॥

प्रदक्षिणनमस्कारान् कृत्वा॥
