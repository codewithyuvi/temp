% !TeX program = XeLaTeX
% !TeX root = pUjA.tex

\setlength{\parindent}{0pt}

\section[एकादशीव्रतम्]{एकादशीव्रतम् — श्री-महाविष्णुपूजा}

\sect{पूर्वाङ्गविघ्नेश्वरपूजा}

(आचम्य)
\twolineshloka*
{शुक्लाम्बरधरं विष्णुं शशिवर्णं चतुर्भुजम्}
{प्रसन्नवदनं ध्यायेत् सर्वविघ्नोपशान्तये}
 
प्राणान्  आयम्य।  ॐ भूः + भूर्भुवः॒ सुव॒रोम्।
 
(अप उपस्पृश्य, पुष्पाक्षतान् गृहीत्वा)\\
ममोपात्तसमस्त दुरितक्षयद्वारा \\
श्रीपरमेश्वरप्रीत्यर्थं करिष्यमाणस्य कर्मणः\\
 निर्विघ्नेन परिसमाप्त्यर्थम् आदौ विघ्नेश्वरपूजां करिष्ये।

\twolineshloka*
{ॐ ग॒णानां᳚ त्वा ग॒णप॑तिꣳ हवामहे क॒विं क॑वी॒नामु॑प॒मश्र॑वस्तमम्}
{ज्ये॒ष्ठ॒राजं॒ ब्रह्म॑णां ब्रह्मणस्पत॒ आ नः॑ शृ॒ण्वन्नू॒तिभिः॑ सीद॒ साद॑नम्}
अस्मिन् हरिद्राबिम्बे महागणपतिं ध्यायामि, आवाहयामि।\\


ॐ महागणपतये नमः  आसनं समर्पयामि।\\
पादयोः पाद्यं समर्पयामि। हस्तयोरर्घ्यं समर्पयामि।\\
आचमनीयं समर्पयामि।\\
ॐ भूर्भुवस्सुवः। शुद्धोदकस्नानं समर्पयामि।\\
स्नानानन्तरमाचमनीयं समर्पयामि।\\
वस्त्रार्थमक्षतान् समर्पयामि।\\
यज्ञोपवीताभरणार्थे अक्षतान् समर्पयामि।\\
दिव्यपरिमलगन्धान् धारयामि।\\
गन्धस्योपरि हरिद्राकुङ्कुमं समर्पयामि। अक्षतान् समर्पयामि। \\
पुष्पमालिकां समर्पयामि। पुष्पैः पूजयामि।

\dnsub{अर्चना}
% \setenumerate{label=\devanumber.}
% \renewcommand{\labelenumi}{\devanumber\theenumi.}
\begin{enumerate}%[label=\devanumber\value{enumi}]
\begin{minipage}{0.475\linewidth}   
\item ॐ सुमुखाय नमः
\item ॐ एकदन्ताय नमः
\item ॐ कपिलाय नमः
\item ॐ गजकर्णकाय नमः
\item ॐ लम्बोदराय नमः
\item ॐ विकटाय नमः
\item ॐ विघ्नराजाय नमः
\item ॐ विनायकाय नमः
\item ॐ धूमकेतवे नमः
  \end{minipage}
  \begin{minipage}{0.525\linewidth}
\item ॐ गणाध्यक्षाय नमः
\item ॐ फालचन्द्राय नमः
\item ॐ गजाननाय नमः
\item ॐ वक्रतुण्डाय नमः
\item ॐ शूर्पकर्णाय नमः
\item ॐ हेरम्बाय नमः
\item ॐ स्कन्दपूर्वजाय नमः
\item ॐ सिद्धिविनायकाय नमः
\item ॐ विघ्नेश्वराय नमः
  \end{minipage}
\end{enumerate}
नानाविधपरिमलपत्रपुष्पाणि समर्पयामि॥\\
धूपमाघ्रापयामि।\\
अलङ्कारदीपं सन्दर्शयामि।\\
नैवेद्यम्।\\
ताम्बूलं समर्पयामि।\\
कर्पूरनीराजनं समर्पयामि।\\
कर्पूरनीराजनानन्तरमाचमनीयं समर्पयामि।\\
{वक्रतुण्डमहाकाय कोटिसूर्यसमप्रभ।}\\
{अविघ्नं कुरु मे देव सर्वकार्येषु सर्वदा॥}\\
प्रार्थनाः समर्पयामि।

अनन्तकोटिप्रदक्षिणनमस्कारान् समर्पयामि।\\
छत्त्रचामरादिसमस्तोपचारान् समर्पयामि।\\


\dnhead{प्रधान-पूजा - एकादशीपूजा}

\twolineshloka*
{शुक्लाम्बरधरं विष्णुं शशिवर्णं चतुर्भुजम्}
{प्रसन्नवदनं ध्यायेत् सर्वविघ्नोपशान्तये}
 
प्राणान्  आयम्य।  ॐ भूः + भूर्भुवः॒ सुव॒रोम्।

\dnsub{सङ्कल्पः}

ममोपात्त-समस्त-दुरित-क्षयद्वारा श्री-परमेश्वर-प्रीत्यर्थं शुभे शोभने मुहूर्ते अद्य ब्रह्मणः
द्वितीयपरार्द्धे श्वेतवराहकल्पे वैवस्वतमन्वन्तरे अष्टाविंशतितमे कलियुगे प्रथमे पादे
जम्बूद्वीपे भारतवर्षे भरतखण्डे मेरोः दक्षिणे पार्श्वे शकाब्दे अस्मिन् वर्तमाने व्यावहारिकाणां प्रभवादीनां षष्ट्याः संवत्सराणां मध्ये (	) नाम संवत्सरे उत्तरायणे/दक्षिणायने 
(ग्रीष्म/वर्ष/शरद्/हेमन्त/ शिशिर/वसन्त) ऋतौ  (मेष/वृषभ/मिथुन/
कर्कटक/सिंह/कन्या /तुला/वृश्चिक/धनुर्/मकर/कुम्भ/मीन) मासे 
(शुक्ल/कृष्ण) पक्षे (एकादश्यां/द्वादश्यां) शुभतिथौ
(इन्दु/भौम/बुध/गुरु/भृगु /स्थिर/भानु) वासरयुक्तायाम्
(अश्विनी/अपभरणी/कृत्तिका/रोहिणी/ मृगशीर्ष/आर्द्रा/पुनर्वसू/पुष्य/
आश्रेषा/मघा/पूर्वफल्गुनी/उत्तर\-फल्गुनी/हस्त/चित्रा/स्वाति/विशाखा/
अनूराधा/ज्येष्ठा/मूल/ पूर्वाषाढा/उत्तराषाढा/श्रवण/श्रविष्ठा/शतभिषङ्/
प्रोष्ठपदा/उत्तर\-प्रोष्ठपदा/रेवती) नक्षत्रयुक्तायां च एवं गुण विशेषण विशिष्टायाम्
अस्याम् (एकादश्यां/द्वादश्यां) शुभतिथौ 
अस्माकं सहकुटुम्बानां क्षेमस्थैर्य-धैर्य-वीर्य-विजय-आयुरारोग्य-ऐश्वर्याभिवृद्ध्यर्थम्
 धर्मार्थकाममोक्ष\-चतुर्विधफलपुरुषार्थसिद्ध्यर्थं पुत्रपौत्राभिवृद्ध्यर्थम् इष्टकाम्यार्थसिद्ध्यर्थम्
मम इहजन्मनि पूर्वजन्मनि जन्मान्तरे च सम्पादितानां ज्ञानाज्ञानकृतमहा\-पातकचतुष्टय-व्यतिरिक्तानां रहस्यकृतानां प्रकाशकृतानां सर्वेषां पापानां सद्य अपनोदनद्वारा सकल-पापक्षयार्थं श्रीभूमिनीलासमेतश्रीमहाविष्णुप्रीत्यर्थं यावच्छक्ति ध्यानावाहनादि 
षोडशोपचारपूजां करिष्ये तदङ्गं कलशपूजां च करिष्ये।


श्रीविघ्नेश्वराय नमः यथास्थानं प्रतिष्ठापयामि।
(गणपति-प्रसादं शिरसा गृहीत्वा)


\dnsub{घण्टापूजा}

\twolineshloka*
{आगमार्थं तु देवानां गमनार्थं तु रक्षसाम्}
{घण्टारवं करोम्यादौ देवताऽऽह्वानकारणम्}

\dnsub{कलशपूजा}
ॐ कलशाय नमः दिव्यगन्धान् धारयामि।\\
ॐ गङ्गायै नमः, ॐ यमुनायै नमः, ॐ गोदावर्यै नमः,  ॐ सरस्वत्यै नमः,\\ ॐ नर्मदायै नमः, ॐ सिन्धवे नमः, ॐ कावेर्यै नमः,\\
 ॐ सप्तकोटिमहातीर्थान्यावाहयामि। \\

(अथ कलशं स्पृष्ट्वा जपं कुर्यात्) \\
आपो॒ वा इ॒दꣳ सर्वं॒ विश्वा॑ भू॒तान्यापः॑ प्रा॒णा वा आपः॑ प॒शव॒ आपोऽन्न॒मापोऽमृ॑त॒मापः॑ स॒म्राडापो॑ वि॒राडापः॑ स्व॒राडाप॒श्छन्दा॒ꣴ॒स्यापो॒ ज्योती॒ꣴ॒ष्यापो॒ यजू॒ꣴ॒ष्यापः॑ स॒त्यमापः॒ सर्वा॑ दे॒वता॒ आपो॒ भूर्भुव॒स्सुव॒राप॒ ओम्॥\\
 
कलशस्य मुखे विष्णुः कण्ठे रुद्रः समाश्रितः।\\
मूले तत्र स्थितो ब्रह्मा मध्ये मातृगणाः स्मृताः॥\\
कुक्षौ तु सागराः सर्वे सप्तद्वीपा वसुन्धरा।\\
ऋग्वेदोऽथ यजुर्वेदः सामवेदोऽप्यथर्वणः॥\\
अङ्गैश्च सहिताः सर्वे कलशाम्बुसमाश्रिताः।\\
गङ्गे च यमुने चैव गोदावरि सरस्वति।\\
नर्मदे सिन्धुकावेरि जलेऽस्मिन् सन्निधिं कुरु॥\\
सर्वे समुद्राः सरितः तीर्थानि च ह्रदा नदाः।\\
आयान्तु विष्णुपूजार्थं दुरितक्षयकारकाः॥\\
ॐ भूर्भुवः॒ सुवो॒ भूर्भुवः॒ सुवो॒ भूर्भुवः॒ सुवः॑।\\

(इति कलशजलेन सर्वोपकरणानि आत्मानं च प्रोक्ष्य।)

\dnsub{आत्मपूजा}
ॐ आत्मने नमः, दिव्यगन्धान् धारयामि।
\begin{multicols}{2}
१. ॐ आत्मने नमः\\
२. ॐ अन्तरात्मने नमः\\
३. ॐ योगात्मने नमः\\
४. ॐ जीवात्मने नमः\\
५. ॐ परमात्मने नमः\\
६. ॐ ज्ञानात्मने नमः
\end{multicols}
समस्तोपचारान् समर्पयामि।\\

देहो देवालयः प्रोक्तो जीवो देवः सनातनः।\\
त्यजेदज्ञाननिर्माल्यं सोऽहं भावेन पूजयेत्॥\\

\dnsub{पीठपूजा}
\begin{multicols}{2}
\begin{enumerate}
\item ॐ आधारशक्त्यै नमः
\item ॐ मूलप्रकृत्यै नमः
\item ॐ आदिकूर्माय नमः 
\item ॐ आदिवराहाय नमः
\item ॐ अनन्ताय नमः
\item ॐ पृथिव्यै नमः
\item ॐ रत्नमण्डपाय नमः
\item ॐ रत्नवेदिकायै नमः
\item ॐ स्वर्णस्तम्भाय नमः
\item ॐ श्वेतच्छत्त्राय नमः
\item ॐ कल्पकवृक्षाय नमः
\item ॐ क्षीरसमुद्राय नमः 
\item ॐ सितचामराभ्यां नमः
\item ॐ योगपीठासनाय नमः
\end{enumerate}
\end{multicols}
 ध्यायेत् चतुर्भुजं देवं शङ्खचक्रगदाधरम्।\\
पीताम्बरयुगोपेतं लक्ष्मीयुक्तं विभूषितम्। \\
लसत्कौस्तुभशोभाढ्यं मेघश्यामं सुलोचनम्॥\\
- अस्मिन् बिम्बे श्रीभूमिनीलासमेतं महाविष्णुं ध्यायामि\\

आगच्छ देवदेवेश जगद्योने रमापते।\\
बिम्बेऽस्मिन् प्रतिष्ठाने सन्निधेहि कृपां कुरु॥\\
- अस्मिन् बिम्बे श्रीभूमिनीलासमेतं महाविष्णुम् आवाहयामि\\

नमोऽस्तु पद्मनाभाय नागाधिपतये नमः।\\
सहस्रशिरसे तुभ्यं आसनं प्रतिगृह्यताम्॥ - आसनं समर्पयामि\\

वराहरूपिणे तुभ्यं सुप्रसन्नाय ते नमः।\\
पाद्यं गृहाण सर्वज्ञ तीर्थपादाय ते नमः॥ - पाद्यं समर्पयामि\\
 
नमोऽस्तु देवदेवेश विश्वरूपिन् सनातन।\\
गृहाणार्घ्यं मया दत्तं नमस्ते धरणीधर॥ - अर्घ्यं समर्पयामि\\

उशीरवासितं तोयं शीतलं निर्मलं जलम्।\\
गृहाणाऽऽचमनीयं ते ददामि पुरुषोत्तम॥ - आचमनीयं समर्पयामि\\

मधुपर्कं मया देव दीयते वरसिद्धये।\\
गृहाण देवदेवेश परिपूर्ण नमोस्तुते॥ - मधुपर्कं समर्पयामि\\

पयोदधिघृतं चैव शर्करामधुसंयुतम्।\\
पञ्चामृतेन स्नपनं गृह्यतां पुरुषोत्तम॥ - पञ्चामृतस्नानं समर्पयामि\\

नानातीर्थसमुद्भूतैरानीतैः पुण्यवारिभिः। \\
स्नानमाचर देवेश सर्वपापक्षयङ्कर॥ - शुद्धोदकस्नानं समर्पयामि\\
- स्नानानन्तरं आचमनीयं समर्पयामि\\

दिव्याम्बरधरानन्द नरकार्णवतारण।\\
वस्त्रं गृहाण देवेश त्रैलोक्यव्यापक प्रभो॥ - वस्त्रं समर्पयामि\\

दामोदर नमस्तेऽस्तु त्राहि मां भवसागरात्।\\
ब्रह्मसूत्रं चोत्तरीयं गृहाण पुरुषोत्तम॥ - उपवीतं समर्पयामि\\

श्रीगन्धचन्दनैर्मिश्रं कर्पूरागरुसंयुतम्। \\
कस्तूरिकादिसंयुक्तं गन्धादिसुमनोहरम्॥ - गन्धं समर्पयामि\\
गन्धस्योपरि हरिद्राकुङ्कुमं समर्पयामि।\\

अक्षतान् धवलान् दिव्यान् शालेयान् तण्डुलान् शुभान्।\\
अक्षयार्थं प्रदास्यामि गृह्यतां मधुसूदन॥ - अक्षतान् समर्पयामि\\

मल्लिकादि सुगन्धीनि मालाद्यादीनि च प्रभो। \\
मयाहृतानि पूजार्थं पुष्पाणि प्रतिगृह्यताम्॥ - पुष्पाणि समर्पयामि

\dnsub{अङ्गपूजा}
\begin{tabular}{llll}
१.&	ॐ वराहाय नमः &-& पादौ पूजयामि	\\
२.&	सङ्कर्षणाय नमः &-& गुल्फौ पूजयामि\\
३.&	कालात्मने नमः &-& जानुनी पूजयामि	\\
४.&	विश्वरूपाय नमः &-& जङ्घे पूजयामि\\
\end{tabular}

\begin{tabular}{llll}
५.&	क्रोढाय नमः &-& ऊरू पूजयामि	\\
६.&	भोक्त्रे नमः &-& कटिं पूजयामि	\\
७.&	विष्णवे नमः &-& मेढ्रं पूजयामि		\\
८.&	हिरण्यगर्भाय नमः &-& नाभिं पूजयामि\\
९.&	श्रीवत्सधारिणे नमः &-& कुक्षिं पूजयामि	\\
१०.& परमात्मने नमः &-& हृदयं पूजयामि\\
११.& सर्वास्त्रधारिणे नमः &-& वक्षः पूजयामि	\\
१२.& वनमालिने नमः &-& कण्ठं पूजयामि\\
१३.& सर्वात्मने नमः &-& मुखं पूजयामि	\\
१४.&	 सहस्राक्षाय नमः &-& नेत्राणि पूजयामि\\
१५.& सुप्रभाय नमः &-& ललाटं पूजयामि	\\
१६.& चम्पकनासिकाय नमः &-& नासिकां पूजयामि	\\
१७.& सर्वेशाय नमः &-& कर्णौ पूजयामि	\\
१८.& सहस्रशिरसे नमः &-& शिरः पूजयामि\\
१९.& नीलमेघनिभाय नमः &-& केशान् पूजयामि	\\
२०.& महापुरुषाय नमः &-& सर्वाणि अङ्गानि पूजयामि	\\
\end{tabular}

%\newpage
\dnsub{चतुर्विंशति नामपूजा}\mbox{}\\[-3.5em]
\begin{multicols}{2}
\begin{enumerate}
\item ॐ केशवाय नमः
\item ॐ नारायणाय नमः
\item ॐ माधवाय नमः
\item ॐ गोविन्दाय नमः
\item ॐ विष्णवे नमः	
\item ॐ मधुसूदनाय नमः
\item ॐ त्रिविक्रमाय नमः
\item ॐ वामनाय नमः
\item ॐ श्रीधराय नमः
\item ॐ हृषीकेशाय नमः
\item ॐ पद्मनाभाय नमः
\item ॐ दामोदराय नमः
\item ॐ सङ्कर्षणाय नमः
\item ॐ वासुदेवाय नमः
\item ॐ प्रद्युम्नाय नमः
\item ॐ अनिरुद्धाय नमः
\item ॐ पुरुषोत्तमाय नमः
\item ॐ अधोक्षजाय नमः
\item ॐ नृसिंहाय नमः
\item ॐ अच्युताय नमः
\item ॐ जनार्दनाय नमः
\item ॐ उपेन्द्राय नमः 
\item ॐ हरये नमः
\item ॐ श्रीकृष्णाय नमः
\end{enumerate}
\end{multicols}

(तदनन्तरं विष्ण्वष्टोत्तरशतनामैरर्च्चयेत्)\\

\dnsub{श्री-कृष्णाष्टोत्तरशतनामावलिः}
\begin{multicols}{2}
\begin{enumerate}
\item ॐ श्रीकृष्णाय नमः
\item ॐ कमलानाथाय नमः
\item ॐ वासुदेवाय नमः
\item ॐ सनातनाय नमः
\item ॐ वसुदेवात्मजाय नमः
\item ॐ पुण्याय नमः
\item ॐ लीलामानुषविग्रहाय नमः
\item ॐ श्रीवत्सकौस्तुभधराय नमः
\item ॐ यशोदावत्सलाय नमः
\item ॐ हरये नमः
\item \mbox{ॐ चतुर्भुजात्तचक्रासिगदा-}\\ शङ्खाम्बुजायुधाय नमः
\item ॐ देवकीनन्दनाय नमः
\item ॐ श्रीशाय नमः
\item ॐ नन्दगोपप्रियात्मजाय नमः
\item ॐ यमुनावेगसंहारिणे नमः
\item ॐ बलभद्रप्रियानुजाय नमः
\item ॐ पूतनाजीवितहराय नमः
\item ॐ शकटासुरभञ्जनाय नमः
\item ॐ नन्दव्रजजनानन्दिने नमः
\item ॐ सच्चिदानन्दविग्रहाय नमः
\item ॐ नवनीतविलिप्ताङ्गाय नमः
\item ॐ नवनीतनटाय नमः
\item ॐ अनघाय नमः
\item ॐ नवनीतनवाहाराय नमः
\item ॐ मुचुकुन्दप्रसादकाय नमः
\item ॐ षोडशस्त्रीसहस्रेशाय नमः
\item ॐ त्रिभङ्गी-मधुराकृतये नमः
\item ॐ शुकवागमृताब्धीन्दवे नमः
\item ॐ गोविन्दाय  नमः
\item ॐ गोविदां पतये नमः
\item ॐ वत्सवाटचराय नमः
\item ॐ अनन्ताय नमः
\item ॐ धेनुकासुरमर्दनाय नमः
\item ॐ तृणीकृततृणावर्ताय नमः
\item ॐ यमलार्जुनभञ्जनाय नमः
\item ॐ उत्तालतालभेत्रे नमः
\item ॐ तमालश्यामलाकृतये नमः
\item ॐ गोपगोपीश्वराय नमः
\item ॐ योगिने नमः
\item ॐ सूर्यकोटिसमप्रभाय नमः
\item ॐ इलापतये नमः
\item ॐ परस्मै ज्योतिषे नमः
\item ॐ यादवेन्द्राय नमः
\item ॐ यदूद्वहाय नमः
\item ॐ वनमालिने नमः
\item ॐ पीतवाससे नमः
\item ॐ पारिजातापहारकाय नमः
\item ॐ गोवर्धनाचलोद्धर्त्रे नमः
\item ॐ गोपालाय नमः
\item ॐ सर्वपालकाय नमः
\item ॐ अजाय नमः
\item ॐ निरञ्जनाय नमः
\item ॐ कामजनकाय नमः
\item ॐ कञ्जलोचनाय नमः
\item ॐ मधुघ्ने नमः
\item ॐ मथुरानाथाय नमः
\item ॐ द्वारकानायकाय नमः
\item ॐ बलिने नमः
\item ॐ बृन्दावनान्तसञ्चारिणे नमः
\item ॐ तुलसीदामभूषणाय नमः
\item ॐ स्यमन्तकमणेर्हर्त्रे नमः
\item ॐ नरनारायणात्मकाय नमः
\item ॐ कुब्जाकृष्णाम्बरधराय नमः
\item ॐ मायिने नमः
\item ॐ परमपूरुषाय नमः
\item \mbox{ॐ~मुष्टिकासुरचाणूरमल्लयुद्ध-} विशारदाय नमः
\item ॐ संसारवैरिणे नमः
\item ॐ कंसारये नमः
\item ॐ मुरारये नमः
\item ॐ नरकान्तकाय नमः
\item ॐ अनादिब्रह्मचारिणे नमः
\item ॐ कृष्णाव्यसनकर्षकाय नमः
\item ॐ शिशुपालशिरश्छेत्रे नमः
\item ॐ  दुर्योधनकुलान्तकाय नमः
\item ॐ विदुराक्रूरवरदाय नमः
\item ॐ विश्वरूपप्रदर्शकाय नमः
\item ॐ सत्यवाचे नमः
\item ॐ सत्यसङ्कल्पाय नमः
\item ॐ सत्यभामारताय नमः
\item ॐ जयिने नमः
\item ॐ सुभद्रापूर्वजाय नमः
\item ॐ विष्णवे नमः
\item ॐ भीष्ममुक्तिप्रदायकाय नमः
\item ॐ जगद्गुरवे नमः
\item ॐ जगन्नाथाय नमः
\item ॐ वेणुनादविशारदाय नमः
\item ॐ वृषभासुरविध्वंसिने नमः
\item ॐ बाणासुरकरान्तकाय नमः
\item ॐ युधिष्ठिरप्रतिष्ठात्रे नमः
\item ॐ बर्हिबर्हावतंसकाय नमः
\item ॐ पार्थसारथये नमः
\item ॐ अव्यक्ताय नमः
\item ॐ गीतामृतमहोदधये नमः
\item \mbox{ॐ~कालीयफणिमाणिक्यरञ्जित-} श्रीपदाम्बुजाय नमः
\item ॐ दामोदराय नमः
\item ॐ यज्ञभोक्त्रे नमः
\item ॐ दानवेन्द्रविनाशनाय नमः
\item ॐ नारायणाय नमः
\item ॐ परब्रह्मणे नमः
\item ॐ पन्नगाशनवाहनाय नमः
\item \mbox{ॐ जलक्रीडासमासक्तगोपी-} वस्त्रापहारकाय नमः
\item ॐ पुण्यश्लोकाय नमः
\item ॐ तीर्थपादाय  नमः
\item ॐ वेदवेद्याय नमः
\item ॐ दयानिधये नमः
\item ॐ सर्वतीर्थात्मकाय नमः
\item ॐ सर्वग्रहरूपिणे नमः
\item ॐ परात्पराय नमः
\end{enumerate}
\end{multicols}
  
\dnsub{उत्तराङ्गपूजा}
दशाङ्गं गुग्गुलं धूपं सुगन्धं सुमनोहरम्।\\
धूपं गृहाण देवेश सर्वभूत मनोहर॥ \\
श्री-भूमीनीलासमेतमहाविष्णवे नमः धूपमाघ्रापयामि।\\
 
उद्दी᳚प्यस्व जातवेदोऽप॒घ्नन्निर्ऋ॑तिं॒ मम॑।\\
 प॒शूꣳश्च॒ मह्य॒माव॑ह॒ जीव॑नं च॒ दिशो॑ दिश॥ \\
मा नो॑ हिꣳसीज्जातवेदो॒ गामश्वं॒ पुरु॑षं॒ जग॑त्।\\
अबि॑भ्र॒दग्न॒ आग॑हि श्रि॒या मा॒ परि॑पातय॥ \\
%साज्यं त्रिवर्त्तिसंयुक्तं वह्निना योजितं मया।\\
%गृहाण मङ्गलं दीपं त्रैलोक्यतिमिरापह॥ \\
श्री-भूमीनीलासमेतमहाविष्णवे नमः अलङ्कारदीपं सन्दर्शयामि।\\

नैवेद्यम्\\
- श्रीभूमिनीलासमेत महाविष्णवे नमः (	) निवेदयामि, \\
अमृतापिधानमसि।निवेदनानन्तरम् आचमनीयं समर्पयामि।\\

पूगीफलसमायुक्तं नागवल्लीदलैर्युतम्।\\
कर्पूरचूर्णसंयुक्तं ताम्बूलं प्रतिगृह्यताम्॥\\
श्री-भूमीनीलासमेतमहाविष्णवे नमः कर्पूरताम्बूलं समर्पयामि।\\

यज्ज्योतिस्सर्वलोकानां तेजसां तेजसोत्तमम्।\\
आत्मज्योतिः परन्धाम नीराजनमिदं प्रभो॥\\
श्री-भूमीनीलासमेतमहाविष्णवे नमः समस्त अपराध क्षमापनार्थं कर्पूरनीराजनं दर्शयामि।\\
कर्पूरनीरजनानन्तरम् आचमनीयं समर्पयामि।\\

 यो॑ऽपां पुष्पं॒ वेद॑। पुष्प॑वान् प्र॒जावा᳚न् पशु॒मान् भ॑वति।\\
च॒न्द्रमा॒ वा अ॒पां पुष्पम्᳚। पुष्प॑वान् प्र॒जावा᳚न् पशु॒मान् भ॑वति।\\
य ए॒वं वेद॑। यो॑ऽपामा॒यत॑नं॒ वेद॑। आ॒यत॑नवान् भवति।\\

ओं᳚ तद्ब्र॒ह्म। ओं᳚ तद्वा॒युः। ओं᳚ तदा॒त्मा। ओं᳚ तथ्स॒त्यम्‌।\\
ओं᳚ तथ्सर्वम्᳚‌। ओं᳚ तत्पुरो॒र्नमः॥\\

अन्तश्चरति॑ भूते॒षु॒ गुहायां वि॑श्वमू॒र्तिषु। \\
त्वं यज्ञस्त्वं वषट्कारस्त्वमिन्द्रस्त्वꣳ रुद्रस्त्वं विष्णुस्त्वं ब्रह्म त्वं॑ प्रजा॒पतिः। \\
त्वं त॑दाप॒ आपो॒ ज्योती॒ रसो॒ऽमृतं॒ ब्रह्म॒ भूर्भुव॒स्सुव॒रोम्‌॥\\

यो वेदादौ स्व॑रः प्रो॒क्तो॒ वे॒दान्ते॑ च प्र॒तिष्ठि॑तः।\\
तस्य॑ प्र॒कृति॑लीन॒स्य॒ यः॒ परः॑ स म॒हेश्व॑रः॥\\

श्री-भूमीनीलासमेतमहाविष्णवे नमः वेदोक्तमन्त्रपुष्पाञ्जलिं समर्पयामि।\\

सुवर्णरजतैर्युक्तं चामीकरविनिर्मितम्।\\
स्वर्णपुष्पं प्रदास्यामि गृह्यतां मधुसूदन॥ - स्वर्णपुष्पं समर्पयामि\\
 
प्रदक्षिणं करोम्यद्य पापानि नुत माधव।\\
मयार्पितान्यशेषाणि परिगृह्य कृपां कुरु॥\\
 यानि कानि च पापानि जन्मान्तरकृतानि च।\\
तानि तानि विनश्यन्ति प्रदक्षिण-पदे पदे॥\\
प्रदक्षिणं कृत्वा।

 \\
नमस्ते देवदेवेश नमस्ते भक्तवत्सल।\\
नमस्ते पुण्डरीकाक्ष वासुदेवाय ते नमः॥\\
नमः सर्वहितार्त्थाय जगदाधाररूपिणे।\\
साष्टाङ्गोऽयं प्रणामोस्तु जगन्नाथ मया कृतः॥\\
- अनन्तकोटिप्रदक्षिणनमस्कारान् समर्पयामि\\
 - छत्त्रचामरादिसमस्तोपचारान् समर्पयामि\\

\dnsub{अर्घ्यप्रदानम्}
ममोपात्त-समस्त-दुरित-क्षयद्वारा श्रीपरमेश्वरप्रीत्यर्त्थम् एकादशीपुण्यकाले।\\
 महाविष्णुपूजान्ते क्षीरार्घ्यप्रदानं करिष्ये॥\\

एकादश्यामुपोष्यैव पारणात् पूर्वकालतः।\\
इदमर्घ्यं प्रदास्यामि गृहाण सुरवन्दित॥\\
	-महाविष्णवे नमः इदमर्घ्यमिदमर्घ्यमिदमर्घ्यम्॥\\
नमोऽस्तु केशवादिभ्यः सर्वलोकैकवन्दिताः।\\
इदमर्घ्यं प्रदास्यामि सुप्रीतो भव सर्वदा॥\\
	-केशवादिभ्यः इदमर्घ्यमिदमर्घ्यमिदमर्घ्यम्।\\
कूर्मरूपाय देवाय मत्स्यरूप नमोऽस्तुते। \\
नीलमेघस्वरूपाय अर्घ्यं दत्तं मया प्रभो॥\\
	-विष्णवे नमः इदमर्घ्यमिदमर्घ्यमिदमर्घ्यम्॥\\
 क्षीरोद्भवे महालक्ष्मि सुप्रसन्ने सुरेश्वरि।\\
सर्वप्रदे जगद्वन्द्ये गृह्णीदार्घ्यमिदं रमे॥ \\
	-महालक्ष्म्यै नमः इदमर्घ्यमिदमर्घ्यमिदमर्घ्यम्।\\
अनेन अर्घ्यप्रदानेन भगवान् सर्वात्मकः श्री-लक्ष्मीनारायणः प्रीयताम्।\\

हिरण्यगर्भगर्भस्थं हेमबीजं विभावसोः।\\
अनन्तपुण्यफलदम् अतः शान्तिं प्रयच्छ मे॥\\
 \\
एकादशीपुण्यकाले अस्मिन् मया क्रियमाण महाविष्णुपूजायां यद्देयमुपायनदानं तत्प्रतिनिधित्वेन हिरण्यं श्रीभूमिनीलासमेत श्री-महाविष्णुप्रीतिम् 
कामयमानः मनसोद्दिष्टाय ब्राह्मणाय सम्प्रददे नमः न मम। 
अनया पूजया श्रीभूमिनीलासमेतः श्रीमहाविष्णुः प्रीयताम्। 
 
\dnsub{विसर्जनम्}
 यस्य स्मृत्या च नामोक्त्या तपः-पूजा-क्रियादिषु।\\
न्यूनं सम्पूर्णतां याति सद्यो वन्दे तमच्युतम्॥ \\
इदं व्रतं मया देव कृतं प्रीत्यै तव प्रभो।\\
न्यूनं सम्पूर्णतां यातु त्वत्प्रसादाज्जनार्द्दन॥\\
 \\
अस्मात् बिम्बात् श्रीभूमिनीलासमेतश्रीमहाविष्णुं यथास्थानं प्रतिष्ठापयामि\\
(अक्षतानर्पित्वा देवमुत्सर्जयेत्।)\\
अनया पूजया श्रीभूमिनीलासमेतः श्रीमहाविष्णुः प्रीयताम्। \\
ॐ तत्सद्ब्रह्मार्पणमस्तु।
 
सालग्रामशिलावारि पापहारि शरीरिणाम्।\\
आजन्मकृतपापानां प्रायश्चित्तं दिने दिने॥\\
अकालमृत्युहरणं सर्वव्याधिनिवारणम्।\\
सर्वपापक्षयकरं विष्णुपादोदकं शुभम्॥ \\
 -इति तीर्थं पीत्वा शिरसि प्रसादं धारयेत्।


\dnsub{परेऽहनि पारणम्}

\twolineshloka*
{अज्ञानतिमिरान्धस्य व्रतेनानेन केशव}
{प्रसीद सुमुखो नाथ ज्ञानदृष्टिप्रदो भव}
