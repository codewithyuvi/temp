% !TeX program = XeLaTeX
% !TeX root = ../pujavidhanam.tex

\setlength{\parindent}{0pt}
\chapt{श्री-स्कन्द-षष्ठी-पूजा}

\sect{पूर्वाङ्गविघ्नेश्वरपूजा}

(आचम्य)
\twolineshloka*
{शुक्लाम्बरधरं विष्णुं शशिवर्णं चतुर्भुजम्}
{प्रसन्नवदनं ध्यायेत् सर्वविघ्नोपशान्तये}
 
प्राणान्  आयम्य।  ॐ भूः + भूर्भुवः॒ सुव॒रोम्।
 
(अप उपस्पृश्य, पुष्पाक्षतान् गृहीत्वा)\\
ममोपात्तसमस्त दुरितक्षयद्वारा \\
श्रीपरमेश्वरप्रीत्यर्थं करिष्यमाणस्य कर्मणः\\
 निर्विघ्नेन परिसमाप्त्यर्थम् आदौ विघ्नेश्वरपूजां करिष्ये।

\twolineshloka*
{ॐ ग॒णानां᳚ त्वा ग॒णप॑तिꣳ हवामहे क॒विं क॑वी॒नामु॑प॒मश्र॑वस्तमम्}
{ज्ये॒ष्ठ॒राजं॒ ब्रह्म॑णां ब्रह्मणस्पत॒ आ नः॑ शृ॒ण्वन्नू॒तिभिः॑ सीद॒ साद॑नम्}
अस्मिन् हरिद्राबिम्बे महागणपतिं ध्यायामि, आवाहयामि।\\


ॐ महागणपतये नमः  आसनं समर्पयामि।\\
पादयोः पाद्यं समर्पयामि। हस्तयोरर्घ्यं समर्पयामि।\\
आचमनीयं समर्पयामि।\\
ॐ भूर्भुवस्सुवः। शुद्धोदकस्नानं समर्पयामि।\\
स्नानानन्तरमाचमनीयं समर्पयामि।\\
वस्त्रार्थमक्षतान् समर्पयामि।\\
यज्ञोपवीताभरणार्थे अक्षतान् समर्पयामि।\\
दिव्यपरिमलगन्धान् धारयामि।\\
गन्धस्योपरि हरिद्राकुङ्कुमं समर्पयामि। अक्षतान् समर्पयामि। \\
पुष्पमालिकां समर्पयामि। पुष्पैः पूजयामि।

\dnsub{अर्चना}
% \setenumerate{label=\devanumber.}
% \renewcommand{\labelenumi}{\devanumber\theenumi.}
\begin{enumerate}%[label=\devanumber\value{enumi}]
\begin{minipage}{0.475\linewidth}   
\item ॐ सुमुखाय नमः
\item ॐ एकदन्ताय नमः
\item ॐ कपिलाय नमः
\item ॐ गजकर्णकाय नमः
\item ॐ लम्बोदराय नमः
\item ॐ विकटाय नमः
\item ॐ विघ्नराजाय नमः
\item ॐ विनायकाय नमः
\item ॐ धूमकेतवे नमः
  \end{minipage}
  \begin{minipage}{0.525\linewidth}
\item ॐ गणाध्यक्षाय नमः
\item ॐ फालचन्द्राय नमः
\item ॐ गजाननाय नमः
\item ॐ वक्रतुण्डाय नमः
\item ॐ शूर्पकर्णाय नमः
\item ॐ हेरम्बाय नमः
\item ॐ स्कन्दपूर्वजाय नमः
\item ॐ सिद्धिविनायकाय नमः
\item ॐ विघ्नेश्वराय नमः
  \end{minipage}
\end{enumerate}
नानाविधपरिमलपत्रपुष्पाणि समर्पयामि॥\\
धूपमाघ्रापयामि।\\
अलङ्कारदीपं सन्दर्शयामि।\\
नैवेद्यम्।\\
ताम्बूलं समर्पयामि।\\
कर्पूरनीराजनं समर्पयामि।\\
कर्पूरनीराजनानन्तरमाचमनीयं समर्पयामि।\\
{वक्रतुण्डमहाकाय कोटिसूर्यसमप्रभ।}\\
{अविघ्नं कुरु मे देव सर्वकार्येषु सर्वदा॥}\\
प्रार्थनाः समर्पयामि।

अनन्तकोटिप्रदक्षिणनमस्कारान् समर्पयामि।\\
छत्त्रचामरादिसमस्तोपचारान् समर्पयामि।\\


\sect{प्रधान-पूजा — स्कन्द-पूजा}

\twolineshloka*
{शुक्लाम्बरधरं विष्णुं शशिवर्णं चतुर्भुजम्}
{प्रसन्नवदनं ध्यायेत् सर्वविघ्नोपशान्तये}

प्राणान् आयम्य। ॐ भूः + भूर्भुवः॒ सुव॒रोम्।


\dnsub{सङ्कल्पः}

ममोपात्त-समस्त-दुरित-क्षयद्वारा श्री-परमेश्वर-प्रीत्यर्थं शुभे शोभने मुहूत्ते अद्य ब्रह्मणः
द्वितीयपरार्धे श्वेतवराहकल्पे वैवस्वतमन्वन्तरे अष्टाविंशतितमे कलियुगे प्रथमे पादे
जम्बूद्वीपे भारतवर्षे भरतखण्डे मेरोः दक्षिणे पार्श्वे शकाब्दे अस्मिन् वर्तमाने व्यावहारिकाणां प्रभवादीनां षष्ट्याः संवत्सराणां मध्ये \mbox{(~~~)}\see{app:samvatsara_names} नाम संवत्सरे दक्षिणायने 
शरद्-ऋतौ  तुला/वृश्चिक-मासे कार्तिक-शुक्लपक्षे षष्ठ्यां शुभतिथौ
(इन्दु / भौम / बुध / गुरु / भृगु / स्थिर / भानु) वासरयुक्तायाम्
\mbox{(~~~)}\see{app:nakshatra_names} नक्षत्र \mbox{(~~~)}\see{app:yoga_names} नाम  योग  (कौलव/तैतिल) करण युक्तायां च एवं गुण विशेषण विशिष्टायाम् अस्यां षष्ठ्यां शुभतिथौ 

श्री-वल्ली-देवसेना-समेत-सुब्रह्मण्य-प्रीत्यर्थं प्रसाद-सिद्ध्यर्थम्
अस्माकं सहकुटुम्बानां क्षेमस्थैर्य-धैर्य-वीर्य-विजय-आयुरारोग्य-ऐश्वर्याणाम् अभिवृद्ध्यर्थं
धर्मार्थ\-काम\-मोक्ष\-चतुर्विध\-फल\-पुरुषार्थ\-सिद्ध्यर्थं पुत्र\-पौत्रा\-भि\-वृद्ध्यर्थम् इष्ट\-काम्यार्थ\-सिद्ध्यर्थं
मम इहजन्मनि पूर्वजन्मनि जन्मान्तरे च सम्पादितानां ज्ञानाज्ञानकृतमहा\-पातकचतुष्टय-व्यतिरिक्तानां रहस्यकृतानां प्रकाशकृतानां सर्वेषां पापानां सद्य अपनोदनद्वारा 
सकल-पाप\-क्षयार्थं गो-भू-धन-धान्य-पुत्र-पौत्रादि अनविच्छिन्न-सन्तति स्थिर-लक्ष्मी-कीर्ति-लाभ शत्रु-पराजयादि सदभीष्ट-सिद्ध्यर्थं दिव्यज्ञान-सिद्ध्यर्थं

यावच्छक्ति-ध्यानावाहनादि षोडशोपचारैः कल्पोक्त-प्रकारेण श्री-वल्ली-देवसेना-समेत-सुब्रह्मण्य-पूजाराधनं करिष्ये। तदङ्गं कलश\-पूजां च करिष्ये।


श्रीविघ्नेश्वराय नमः, यथास्थानं प्रतिष्ठापयामि।\\
(गणपति-प्रसादं शिरसा गृहीत्वा)


\dnsub{घण्टापूजा}
\twolineshloka*
{आगमार्थं तु देवानां गमनार्थं तु रक्षसाम्}
{कुरु घण्टारवं तत्र देवताऽऽह्वानलाञ्चनम्}


\dnsub{कलशपूजा}
(कलशं गन्धपुष्पाक्षतैः अभ्यर्च्य)

गङ्गायै नमः। यमुनायै नमः। गोदावर्यै नमः। सरस्वत्यै नमः। नर्मदायै नमः। सिन्धवे नमः। कावेर्यै नमः।\\
सप्तकोटिमहातीर्थान्यावाहयामि। \\

(अथ कलशं स्पृष्ट्वा जपं कुर्यात्।)

\twolineshloka*
{कलशस्य मुखे विष्णुः कण्ठे रुद्रः समाश्रितः}
{मूले तत्र स्थितो ब्रह्मा मध्ये मातृगणाः स्मृताः}

\threelineshloka*
{कुक्षौ तु सागराः सर्वे सप्तद्वीपा वसुन्धरा}
{ऋग्वेदोऽथ यजुर्वेदः सामवेदोऽप्यथर्वणः}
{अङ्गैश्च सहिताः सर्वे कलशाम्बुसमाश्रिताः}

\twolineshloka*
{गङ्गे च यमुने चैव गोदावरि सरस्वति}
{नर्मदे सिन्धुकावेरि जलेऽस्मिन् सन्निधिं कुरु}

\twolineshloka*
{सर्वे समुद्राः सरितः तीर्थानि च ह्रदा नदाः}
{आयान्तु विष्णुपूजार्थं दुरितक्षयकारकाः}

% \centerline{ॐ भूर्भुवः॒ सुवो॒ भूर्भुवः॒ सुवो॒ भूर्भुवः॒ सुवः॑।}

(इति कलशजलेन सर्वोपकरणानि आत्मानं च प्रोक्ष्य।)

\dnsub{आत्मपूजा}
आत्मने नमः, दिव्यगन्धान् धारयामि। 


\dnsub{मण्टप-पूजा}

ॐ ह्रीं श्रीं मण्डूकादि-परतत्त्वात्म-पर्यन्त-पीठ-शक्ति-देवताभ्यो नमः।\\
ॐ ह्रीं श्रीं शं शकुन्यै नमः।\\
ॐ ह्रीं श्रीं रें रेवत्यै नमः।\\
ॐ ह्रीं श्रीं पूं पूताय नमः।\\
ॐ ह्रीं श्रीं मं महापूतायै नमः।\\
ॐ ह्रीं श्रीं निं निशीथिन्यै नमः।\\
ॐ ह्रीं श्रीं मां मालिन्यै नमः।\\
ॐ ह्रीं श्रीं शीं शीतलायै नमः।\\
ॐ ह्रीं श्रीं शुं शुद्धायै नमः।\\
ॐ ह्रीं श्रीं विं विश्वतोमुख्यै नमः।\\

% \dnsub{दीप-पूजा}

% \twolineshloka*
% {दीपदेवि महादेवि शुभं भवतु मे सदा}
% {यावत्पूजासमाप्तिः स्यात् तावत् प्रज्वल सुस्थिर}


\section{षोडशोपचार-पूजा}
\renewcommand{\devAya}{श्री-वल्ली-देवसेना-समेत-सुब्रह्मण्य-स्वामिने नमः,}

\begin{center}

\fourlineindentedshloka*
{सिन्धूरारुणमिन्दुकान्तिवदनं केयूरहारादिभिः}
{दिव्यैराभरणैर्विभूषिततनुं स्वर्गादिसौख्यप्रदम्}
{अम्भोजाभयशक्तिकुक्कुटधरं रक्ताङ्गराकोज्ज्वलं}
{सुब्रह्मण्यमुपास्महे प्रणमतां भीतिप्रणाशोद्यतम्}
\nobreak%\hfill{}
अस्मिन् कुम्भे \textbf{सपरिवारं\\
श्री-वल्ली-देवसेना-समेत-सुब्रह्मण्य-स्वामिनम्} ध्यायामि।

\fourlineindentedshloka*
{षड्वक्त्रं शिखिवाहनं त्रिनयनं चित्राम्बरालङ्कृतं}
{वज्रं शक्तिमसिं त्रिशूलमभयं खेटं धनुश्चक्रकम्}
{पाशं कुक्कुटमङ्कुशं च वरदं दोर्भिर्दधानं सदा}
{ध्यायेदीप्सितसिद्धिदं शिवसुतं स्कन्दं सुराराधितम्}

\nobreak%\hfill{}
अस्मिन् कुम्भे \textbf{सपरिवारं\\
श्री-वल्ली-देवसेना-समेत-सुब्रह्मण्य-स्वामिनम्} आवाहयामि। 

आवाहिता भव। संस्थापिता भव।\\
सन्निहिता भव। सन्निरुद्धा भव।\\
अवकुण्ठिता भव। सुप्रीता भव।\\
सुप्रसन्ना भव। वरदा भव।\\

\twolineshloka*
{स्वामिन् सर्वजगन्नाथ यावत्पूजावसानकम्}
{तावत् त्वं प्रीतिभावेन दीपेऽस्मिन् सन्निधिं कुरु}

\twolineshloka*
{देवदेव महाराज प्रियेश्वर प्रजापते}
{आसनं दिव्यमीशान दास्येयं परमेश्वर}
\nobreak%\hfill{}
\textbf{\devAya{} आसनं समर्पयामि।}

\twolineshloka*
{यद्भक्तिलेशसम्पर्कात् परमानन्दविग्रह}
{तस्मै ते शरणाब्जाय पाद्यं शुद्धाय कल्पये}
\nobreak%\hfill{}
\textbf{\devAya{} पाद्यं समर्पयामि।}

\twolineshloka*
{तापत्रयहरं दिव्यं परमानन्दलक्षणम्}
{तापत्रयविनिर्मुक्तं तवार्घ्यं कल्पयाम्यहम्}
\nobreak%\hfill{}
 \textbf{\devAya{} अर्घ्यं समर्पयामि।}

\twolineshloka*
{वेदानामपि वेद्याय देवानां देवतात्मने}
{आचामं कल्पयामीश शुद्धानां शुद्धिहेतवे}
\nobreak%\hfill{}
\textbf{\devAya{} आचमनीयं समर्पयामि।}

\twolineshloka*
{तरुपुष्पसमुद्भूतं सुस्वादु मधुरं मधु}
{तेजःपुष्टिकरं दिव्यं प्रतिगृह्णीष्व देवेश}
\nobreak%\hfill{}
\textbf{\devAya{} मधुपर्कं समर्पयामि।}

\twolineshloka*
{पयोदधिघृतं चैव मधु च शर्करायुतम्}
{पञ्चामृतं मयाऽऽनीतं स्नानार्थं प्रतिगृह्यताम्}
\nobreak%\hfill{}
\textbf{\devAya{} पञ्चामृत-स्नानं समर्पयामि।}

\twolineshloka*
{कामधेनुसमुत्पन्नं सर्वेषां जीवनं परम्}
{पावनं यज्ञहेतुश्च पयः स्नानार्थमर्पितम्}
\nobreak%\hfill{}
\textbf{\devAya{} क्षीरस्नानं समर्पयामि।}

\twolineshloka*
{भागीरथी यमुना चैव गौतमी च सरस्वती}
{तासां सुसलिलमादाय करोमि त्वामभिषेचनम्}
\nobreak%\hfill{}
\textbf{\devAya{} स्नानं समर्पयामि। स्नानान्तरमाचमनीयं समर्पयामि।}

\twolineshloka*
{सर्वभूषाधिके सौम्ये लोकलज्जानिवारणे}
{मयोपपादिते तुभ्यं वाससी प्रतिगृह्यताम्}
\nobreak%\hfill{}
\textbf{\devAya{} वस्त्रं समर्पयामि।}

\twolineshloka*
{नवभिस्तन्तुभिर्युक्तं त्रिगुणं देवतात्मकम्}
{उपवीतं प्रदास्यामि गृहाण परमेश्वर}
\nobreak%\hfill{}
\textbf{\devAya{} यज्ञोपवीतं समर्पयामि।}

\twolineshloka*
{मुक्ता-माणिक्य-वैडूर्य-रत्न-हेमादि-निर्मितम्}
{नानाभरणं दास्यामि स्वीकुरुष्व दयानिधे}
\nobreak%\hfill{}
\textbf{\devAya{} नवमणि-मकुटादि नानाभरणम् समर्पयामि।}

\twolineshloka*
{चन्दनागरुकर्पूरकस्तूरीकुङ्कुमान्वितम्}
{विलेपनं सुरश्रेष्ठ प्रीत्यर्थं प्रतिगृह्यताम्}
\nobreak%\hfill{}
\textbf{\devAya{} गन्धान् धारयामि। गन्धस्योपरि हरिद्रा-कुङ्कुमं समर्पयामि।}


\twolineshloka*
{अक्षतांश्च सुरश्रेष्ठ कुङ्कुमाक्ता सुशोभिताः}
{मया निवेदिता भक्त्या गृह्यतां परमेश्वर}
\nobreak%\hfill{}
\textbf{\devAya{} अक्षतान् समर्पयामि।}

\twolineshloka*
{मन्दार-पारिजाताब्ज-केतक्युत्पल-पाटलैः}
{मल्लिका-जाति-वकुलैः पुष्पैस्त्वां पूजयाम्यहम्}
\nobreak%\hfill{}
\textbf{\devAya{} मल्लिकादि-सर्वर्तु-पुष्पमालाः समर्पयामि।}

\dnsub{अङ्ग-पूजा}

\begin{longtable}{ll@{— }l}
१. & शरवणोद्भूताय नमः & पादौ पूजयामि।\\
२. & रौद्रेयाय नमः & जङ्घे पूजयामि।\\
३. & सहस्रपदे नमः & जानुनी पूजयामि।\\
४. & भयनाशनाय नमः & ऊरू पूजयामि।\\
५. & बालग्रहाच्छाटनाय नमः & मेढ्रं पूजयामि \\
६. & भक्तपालनाय नमः & गुह्यं पूजयामि।\\
७. & गुणनिधये नमः & कटिं पूजयामि।\\
८. & महनीयाय नमः & नाभिं पूजयामि।\\
९. & सर्वाभीष्टप्रदाय नमः & हृदयं पूजयामि।\\
१०. & विशालवक्षसे नमः & वक्षस्थलं पूजयामि।\\
११. & शक्तिधराय नमः & हस्तान् पूजयामि।\\
१२. & अभयप्रदानाय नमः & बाहून् पूजयामि।\\
१३. & नीलकण्ठ-तनयाय नमः & कण्ठान् पूजयामि।\\
१४. & पतित-पावनाय नमः & चुबुकानि पूजयामि।\\
१५. & पुरुष-श्रेष्ठाय नमः & नासिकानि पूजयामि\\
१६. & कमललोचनाय नमः & लोचनानि पूजयामि\\
१७. & पुण्यमूर्तये नमः & श्रोत्राणि पूजयामि\\
१८. & कस्तूरी-तिलकाञ्चित-फालाय नमः & ललाटानि पूजयामि\\
१९. & षडाननाय नमः & मुखानि पूजयामि\\
२०. & त्रिलोकगुरवे नमः & ओष्ठानि पूजयामि।\\
२२. & सहस्रशीर्ष्णे नमः & शिरांसि पूजयामि।\\
२३. &भस्मोद्धूलित-विग्रहाय नमः & सर्वाण्यङ्गानि पूजयामि। \\
\end{longtable}

\dnsub{षोडश-नामपूजा}
\begin{multicols}{2}
\begin{enumerate}
\item ॐ ज्ञानशक्त्यात्मने नमः
\item ॐ स्कन्दाय नमः
\item ॐ अग्निभुवे नमः
\item ॐ बाहुलेयाय नमः
\item ॐ गाङ्गेयाय नमः
\item ॐ शरवणोद्भवाय नमः
\item ॐ कार्त्तिकेयाय नमः
\item ॐ कुमाराय नमः
\item ॐ षण्मुखाय नमः
\item ॐ कुक्कुटध्वजाय नमः
\item ॐ शक्तिधराय नमः
\item ॐ गुहाय नमः
\item ॐ ब्रह्मचारिणे नमः
\item ॐ षण्मातुराय नमः
\item ॐ क्रौञ्चभित्रे नमः
\item ॐ शिखिवाहनाय नमः
\end{enumerate}
\end{multicols}

\begingroup
\setlength{\columnseprule}{1pt}
\let\chapt\sect
\input{../namavali-manjari/1000/Shanmukha_1000.tex}
\input{../namavali-manjari/100/Subrahmanya_108.tex}
\input{../namavali-manjari/100/Valli_108.tex}
\input{../namavali-manjari/100/Devasena_108.tex}
\endgroup

\textbf{\devAya{} नानाविध-परिमल-पत्र-पुष्पाणि समर्पयामि।}


\twolineshloka*
{दशाङ्गं च पटीरं च एला-कुङ्कुम-संयुतम्}
{धूपं गृहाण देवेश सुब्रह्मण्य नमोऽस्तु ते}
%\hfill{}
\textbf{\devAya{} धूपम् आघ्रापयामि।}

\twolineshloka*
{इन्द्वर्कवह्निनेत्राय देवसेनापतये नमः}
{घृतवर्तिसुसंयुक्तं दीपोऽयम् अवलोक्यताम्}
%\hfill{}
\textbf{\devAya{} दीपं दर्शयामि। धूप-दीपानन्तरम् आचमनीयं समर्पयामि।}

\twolineshloka*
{सत्पात्रसिद्धं सुहविर्विविधानेक-भक्षणम्}
{निवेदयामि देवेश सानुगाय गृहाण तत्}
%\hfill{}
\textbf{\devAya{} () महानैवेद्यं निवेदयामि। }
मध्ये मध्ये अमृतपानीयं समर्पयामि। हस्त-प्रक्षालनं समर्पयामि। गण्डूषं समर्पयामि। पुनः हस्त-प्रक्षालनं समर्पयामि।
 पाद-प्रक्षालनं समर्पयामि। आचमनीयं समर्पयामि।

\twolineshloka*
{पूगीफलसमायुक्तं नागवल्लीदलैर्युतम्}
{कर्पूरचूर्णसंयुक्तं ताम्बूलं प्रतिगृह्यताम्}
%\hfill{}
\textbf{\devAya{} ताम्बूलं समर्पयामि।}

\twolineshloka*
{नीराजनं देवदेव सूर्यकोटि-समप्रभ}
{अहं भक्त्या प्रदास्यामि स्वीकुरुष्व दयानिधे}
%\hfill{}
\textbf{\devAya{} कर्पूर-नीराजनं दर्शयामि। }
पुष्पाञ्जलिं समर्पयामि। आचमनीयं समर्पयामि। रक्षां धारयामि।

\twolineshloka*
{सर्व-पापौघ-विध्वंस साक्षाद्धर्मस्वरूपक}
{पुष्पाञ्जलिं प्रदास्यामि गृहाण भुवनेश्वर}
%\hfill{}
\textbf{\devAya{} मन्त्रपुष्पाञ्जलिं समर्पयामि। स्वर्णपुष्पं समर्पयामि।}

\twolineshloka*
{यानि कानि च पापानि जन्मान्तरकृतानि च}
{तानि तानि विनश्यन्ति प्रदक्षिण-पदे पदे}
\textbf{प्रदक्षिणं कृत्वा।}
\medskip

\twolineshloka*
{षण्मुखं पार्वतीपुत्रं क्रौञ्चशैलविमर्दनम्}
{देवसेनापतिं देवं स्कन्दं वन्दे शिवात्मजम्}
\twolineshloka*
{तारकासुर-हन्तारं मयूरोपरि संस्थितम्}
{शक्तिपाणिं च देवेशं स्कन्दं वन्दे शिवात्मजम्}
%\hfill{}
\textbf{\devAya{} प्रदक्षिण-नमस्कारान् समर्पयामि।}

\fourlineindentedshloka*
{नमः केकिने शक्तये चापि तुभ्यं}
{नमश्छाग तुभ्यं नमः कुक्कुटाय}
{नमः सिन्धवे सिन्धुदेशाय तुभ्यं}
{पुनः स्कन्दमूर्ते नमस्ते नमोऽस्तु}

\fourlineindentedshloka*
{जयाऽऽनन्दभूमन् जयापारधामन्}
{जयामोघकीर्ते जयाऽऽनन्दमूर्ते}
{जयाऽऽनन्दसिन्धो जयाशेषबन्धो}
{जय त्वं सदा मुक्तिदानेशसूनो}

\textbf{\devAya{} प्रार्थनाः समर्पयामि।}

\textbf{\devAya{} छत्रं समर्पयामि।}
चामरयुगलं वीजयामि।\\
दर्पणं दर्शयामि। गीतं श्रावयामि। \\
नृत्तं दर्शयामि। आन्दोलिकाम् आरोहयामि।\\
गजम् आरोहयामि। अश्वम् आरोहयामि।\\
रथम् आरोहयामि। समस्त-राजोपचार-देवोपचार-पूजाः समर्पयामि।


\dnsub{अर्घ्यप्रदानम्}
ममोपात्त-समस्त-दुरित-क्षयद्वारा श्रीपरमेश्वरप्रीत्यर्थम् स्कन्दषष्ठी-पुण्यकाले श्री-सुब्रह्मण्य-पूजान्ते अर्घ्यप्रदानं करिष्ये॥

\medskip

\twolineshloka*
{दत्त्वाऽर्घ्यं कार्तिकेयाय स्थित्वा वै \textbf{दक्षिणामुखः}}
{\textbf{दध्ना घृतोदकैः पुष्पै}र्मन्त्रेणानेन सुव्रत}

\threelineshloka*
{सप्तर्षिदारज स्कन्द स्वाहापतिसमुद्भव}
{रुद्रार्यमाग्निज विभो गङ्गागर्भ नमोऽस्तु ते}
{प्रीयतां देवसेनानीः सम्पादयतु हृद्गतम्}
\textbf{\devAya{} इदमर्घ्यमिदमर्घ्यमिदमर्घ्यम्॥\medskip}

\twolineshloka*
{दत्त्वा विप्राय चाऽऽत्मानं यच्चान्यदपि विद्यते}
{पश्चाद्भुङ्क्ते त्वसौ रात्रौ भूमिं कृत्वा तु भाजनम्}

[श्रीभविष्ये महापुराणे शतार्धसाहस्र्यां संहितायां ब्राह्मेपर्वणि पञ्चमीकल्पे षष्ठीकल्पवर्णनं नाम एकोनचत्वारिंशोऽध्यायः॥]


अनेन अर्घ्यप्रदानेन भगवान् सर्वात्मकः\\ \textbf{श्री-वल्ली-देवसेना-समेत-सुब्रह्मण्य-स्वामिनः} प्रीयन्ताम्।\medskip

\twolineshloka*
{हिरण्यगर्भगर्भस्थं हेमबीजं विभावसोः}
{अनन्तपुण्यफलदम् अतः शान्तिं प्रयच्छ मे}

स्कन्दषष्ठी-पुण्यकाले अस्मिन् मया क्रियमाण\\
श्री-सुब्रह्मण्यपूजायां यद्देयमुपायनदानं तत्प्रत्याम्नायार्थं हिरण्यं\\
\textbf{श्री-वल्ली-देवसेना-समेत-सुब्रह्मण्य-स्वामिनः} प्रीतिं कामयमानः\\
मनसोद्दिष्टाय ब्राह्मणाय सम्प्रददे नमः न मम।\\ 
अनया पूजया \textbf{श्री-वल्ली-देवसेना-समेत-सुब्रह्मण्य-स्वामिनः} प्रीयन्ताम्। 
 

\fourlineindentedshloka*
{कायेन वाचा मनसेन्द्रियैर्वा}
{बुद्‌ध्याऽऽत्मना वा प्रकृतेः स्वभावात्}
{करोमि यद्यत् सकलं परस्मै}
{नारायणायेति समर्पयामि}

ॐ तत्सद्ब्रह्मार्पणमस्तु।

\end{center}