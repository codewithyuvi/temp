% !TeX program = XeLaTeX
% !TeX root = pUjA.tex

\setlength{\parindent}{0pt}

\newcommand{\sep}{\hspace{-0.5ex}{\small$\circ$}\hspace{0.5ex}}
\newcommand{\yutithyadi}{\textbf{क्रोधि}-नाम-संवत्सरे    \textbf{दक्षिणायने} \textbf{वर्ष}-ऋतौ   \textbf{सिंह}-\textbf{श्रावण}-मासे  \textbf{शुक्ल}-पक्षे    \textbf{पौर्णमास्यां} शुभतिथौ   \textbf{इन्दु}\-वासर\-युक्तायां \textbf{श्रवण}-नक्षत्र (०८:०८; \textbf{श्रविष्ठा}-नक्षत्र)\-युक्तायां   \textbf{शोभन}-योग\-युक्तायां    \textbf{भद्रा}-करण (१३:३२; \textbf{बव}-करण)\-युक्तायाम् एवं-गुण-विशेषण-विशिष्टायाम्     अस्यां \textbf{पौर्णमास्यां}}
\newcommand{\instruct}[2]{#1}
\chapt{यजुर्वेद-उपाकर्म}
\begingroup

\sect{कामोऽकार्षीन्मन्त्र जपः}

आचमनम्। पवित्रपाणिः। दर्भेष्वासीनः। दर्भान् धारयमाणः।

शुक्लाम्बरधरं + शान्तये। प्राणायामः।

ममोपात्तसमस्तदुरितक्षयद्वारा श्रीपरमेश्वरप्रीत्यर्थं शुभे शोभने मुहूर्ते अद्यब्रह्मणः
द्वितीयपरार्द्धे श्वेतवराहकल्पे वैवस्वतमन्वन्तरे अष्टाविंशतितमे कलियुगे प्रथमे पादे
जम्बूद्वीपे भारतवर्षे भरतखण्डे मेरोः दक्षिणेपार्श्वे शकाब्दे अस्मिन् वर्तमाने व्यावहारिके
 प्रभवादीनां षष्ट्याः संवत्सराणां मध्ये
\yutithyadi

तैष्यां पौर्णमास्याम् अध्यायोत्सर्जन-अकरण- प्रायश्चित्तार्थम् अष्टोत्तर (शत/सहस्र) सङ्ख्यया कामोऽकार्षीन्मन्युरकार्षीदिति\footnote{विस्तृत-कामोऽकार्षीज्जपः---कामोऽकार्\mbox{}षी᳚न्नमो॒ नमः। 
 कामोऽकार्\mbox{}षीत्कामः करोति नाहं करोमि कामः कर्ता नाहं कर्ता कामः॑ कार॒यिता नाहं॑ कार॒यिता एष ते काम कामा॑य स्वा॒हा॥ मन्युरकार्\mbox{}षी᳚न्नमो॒ नमः। 
मन्युरकार्\mbox{}षीन्मन्युः करोति नाहं करोमि मन्युः कर्ता नाहं कर्ता मन्युः॑ कार॒यिता नाहं॑ कार॒यिता एष ते मन्यो मन्य॑वे स्वा॒हा॥} महा\-मन्त्र\-जपं करिष्ये।

\instruct{इति सङ्कल्प्य दर्भान्निरस्य अप उपस्पृश्य।}{என்று ஸங்கல்பம் செய்து கொண்டு தர்பைகளை கீழே போட்டுவிட்டு ஜலத்தை தொடவும்.}

\instruct{कामोऽकार्षीन्मन्युरकार्षीन्नमो॒  नमः  इति जपं कृत्वा पवित्रं विसृज्य आचामेत्।}{காமோகார்ஷீத் மன்யுரகார்ஷீன்னமோ நம: என்று ஜபித்து முடிவில் ப்ராணாயாமம் செய்து உபஸ்தானம் செய்யவும். பவித்ரத்தை விஸர்ஜனம் செய்து ஆசமனம் செய்யவும்.}


\sect{महा-सङ्कल्पः}
आचम्य, दर्भेषु आसीनः, दर्भान् धारयमाणः। शुक्लाम्बरधरं + शान्तये। प्राणायामः।

ममोपात्त समस्तदुरितक्षयद्वारा श्रीपरमेश्वरप्रीत्यर्थम्

    \twolineshloka*
      {तदेव लग्नं सुदिनं तदेव ताराबलं चन्द्रबलं तदेव}
      {विद्याबलं दैवबलं तदेव लक्ष्मीपते ते अङ्घ्रियुगं स्मरामि}
    \twolineshloka*
    {ॐ॥ अपवित्रः पवित्रो वा सर्वावस्थागतोऽपि वा}
    {यः स्मरेत्पुण्डरीकाक्षं स बाह्याभ्यन्तरः शुचिः}

     \twolineshloka*
  {मानसं वाचिकं पापं कर्मणा समुपार्जितम्}
{श्रीरामस्मरणेनैव व्यपोहति न संशयः}

श्रीराम राम राम। 
    \twolineshloka*
{तिथिर्विष्णुस्तथा वारो नक्षत्रं विष्णुरेव च}
{योगश्च करणं चैव सर्वं विष्णुमयं जगत्}

श्रीगोविन्द गोविन्द गोविन्द। 

अद्य श्री भगवतः आदिविष्णोः आदिनारायणस्य अचिन्त्यया अपरिमितया शक्त्या भ्रियमाणस्य महाजलौघस्य मध्ये
परिभ्रम\-माणानाम् अनेक\-कोटि\-ब्रह्माण्डानाम् एकतमे अव्यक्त-महदहङ्कार-पृथिव्यप्तेजो-वाय्वाकाशाद्यैः आवरणैः
आवृते\-ऽस्मिन् महति ब्रह्माण्डकरण्डमण्डले आधारशक्ति आदिकूर्मादि अनन्तादि अष्टदिग्गजोपरि
प्रतिष्ठितानाम् अतल-वितल-सुतल-तलातल-रसातल-महातल-पातालाख्यानां सप्त-लोकानाम् उपरितले
पुण्यकृताम् निवासभूते
भुवर्लोक-सुवर्लोक-महोलोक-जनोलोक-तपोलोक-सत्यलोकाख्य-लोकषट्कस्य अधोभागे महानालायमान\-फणि\-राज\-शेषस्य 
सहस्र\-फणामणि-मण्डल-मण्डिते दिग्दन्ति-शुण्डादण्ड-उत्तम्भिते 
%पञ्चाशत्कोटियोजन विस्तीर्णे लोकालोक अचलेन वलयिते 
लवणेक्षु-सुरासर्पि-दधि-क्षीर-शुद्धोदकार्णवैः परिवृते
जम्बू-प्लक्ष-शाल्मलि-कुश-क्रौञ्च-शाक-पुष्कराख्य-सप्तद्वीपानां मध्ये \textbf{जम्बूद्वीपे}
भारत-किम्पुरुष-हरि-इलावृत-भद्राश्व-केतुमाल-हिरण्मय-रमणक-कुरु-वर्षाख्य नववर्षाणां मध्ये \textbf{भारतवर्षे}
इन्द्र-कशेरु-ताम्र-गभस्ति-पुन्नाग-गन्धर्व-सौम्य-वरुण-भरत-खण्डानां मध्ये \textbf{भरतखण्डे}
सुमेरु-निषध-हेमकूट-हिमाचल-माल्यवत्-पारियात्रक-गन्धमादन-कैलास-विन्ध्याचलादि-महा\-शैल\-मध्ये
दण्डकारण्य-चम्पकारण्य-विन्ध्यारण्य-वीक्षारण्य-श्वेतारण्य-वेदारण्यादि अनेक\-पुण्या\-रण्यानां मध्ये
कर्मभूमौ दण्डकारण्ये समभूमिरेखायाः दक्षिणदिग्भागे श्रीशैलस्य आग्नेयदिग्भागे रामसेतोः उत्तर\-दिग्भागे
गङ्गा-यमुना-सरस्वती-भीमरथी-गौतमी-नर्मदा-गण्डकी-कृष्णवेणी-तुङ्गभद्रा-चन्द्रभागा-मलापहा-कावेरी-कपिला-ताम्रपर्णी-वेगवती-पिनाकिनी-क्षीरनद्यादि
अनेक-महानदी-विराजिते
इन्द्रप्रस्थ-यमप्रस्थ-अवन्तिका\-पुरी-हस्तिनापुरी-अयोध्या\-पुरी-मथुरा\-पुरी-मायापुरी-काशीपुरी-काञ्चीपुरी-द्वारकादि अनेक\-पुण्यपुरी-विराजिते
वाराणसी-चिदम्बर-श्रीशैल-अहोबिल-वेङ्कटाचल-रामसेतु-जम्बुकेश्वर-कुम्भघोण-हालास्य-गोकर्ण-अनन्तशयन-गया-प्रयागादि
अनेकपुण्यक्षेत्र-परिवृते सकलजगत्स्रष्टुः परार्धद्वयजीविनः ब्रह्मणः प्रथमे परार्धे पञ्चाशत्
अब्दात्मके अतीते द्वितीये परार्धे पञ्चाशद्-अब्दादौ प्रथमे वर्षे प्रथमे मासे प्रथमे पक्षे प्रथमे
दिवसे अह्नि द्वितीये यामे तृतीये मुहूर्ते
पार्थिव-कूर्म-प्रलयानन्त-श्वेतवराह-ब्राह्म-सावित्र्याख्य-सप्त-कल्पानां मध्ये \textbf{श्वेतवराहकल्पे}
स्वायम्भुव-स्वारोचिष-उत्तम-तामस-रैवत-चाक्षुषाख्येषु षट्सु मनुषु अतीतेषु सप्तमे \textbf{वैवस्वतमन्वन्तरे}
अष्टाविंशतितमे कलियुगे प्रथमे पादे युधिष्ठिर-विक्रम-शालिवाहन-विजय-अभिनन्दन-नागार्जुन-कलिभूपाख्य
शकपुरुष मध्यपरि\-गणितेन \textbf{शालिवाहनशके} बौद्धावतारे ब्राह्म-दैव-पित्र्य-प्राजापत्य-बार्हस्पत्य-सौर-चान्द्र
सावन-नक्षत्राख्य-नवमान-मध्य-परि\-गणितेन सौर-चान्द्रमान-द्वयेन प्रवर्तमाने प्रभवादीनां षष्ट्याः
संवत्सराणां मध्ये 
\yutithyadi

अनादि-अविद्या-वासनया प्रवर्तमाने अस्मिन् महति संसारचक्रे विचित्राभिः
कर्मगतिभिः विचित्रासु योनिषु पुनः पुनः अनेकधा जनित्वा केनापि पुण्यकर्मविशेषेण
इदानीन्तन-मानुष-द्विजजन्मविशेषं प्राप्त\-वतः मम जन्माभ्यासात् जन्मप्रभृति एतत्क्षण\-पर्यन्तं बाल्ये वयसि
कौमारे यौवने वार्धके च जाग्रत्-स्वप्न-सुषुप्ति-अवस्थासु मनो-वाक्-कायैः
कर्मेन्द्रिय-ज्ञानेन्द्रिय-व्यापारैश्च सम्भावितानां रहस्यकृतानां प्रकाशकृतानां ब्रह्महनन-सुरापान-स्वर्णस्तेय-गुरुदारगमन-तत्संसर्गाख्यानां महा\-पातकानां, महापातक अनुमन्तृत्वादीनाम् अति\-पातकानां,
सोमयागस्थ-क्षत्रिय-वैश्य\-वधादीनां सम\-पातकानां, गो\-वधादीनाम् उप\-पातकानां मार्जार\-वधादीनां सङ्कली\-करणानां,
कृमिकीट\-वधादीनां मलिनी\-करणानां, निन्दित धनादान उपजीवनादीनाम् अपात्री\-करणानां, मद्याघ्राणनादीनां जातिभ्रंशकराणां विहितकर्मत्यागादीनां
प्रकीर्णकानां, ज्ञानतः सकृत्कृतानां, अज्ञानतः असकृत्कृतानां, अत्यन्ताभ्यस्तानां निरन्तरा\-भ्यस्तानां
चिरकालाभ्यस्तानां नवानां नवविधानां बहूनां बहुविधानां सर्वेषां पापानां सद्यः अपनोदनार्थं
भास्कर\-क्षेत्रे  विनायकादि\-समस्त\-हरिहर\-देवतासन्निधौ ... पौर्णमास्याम् अध्यायोपा\-कर्म करिष्ये।  तदङ्गम्
अवगाह्य महानदीस्नानं करिष्ये।\\
\instruct{इति सङ्कल्प्य दर्भान्निरस्य अप उपस्पृश्य}{என்று ஸங்கல்பம் செய்து கொண்டு தர்பைகளை கீழே போட்டுவிட்டு ஜலத்தை தொடவும்.}

\twolineshloka*
{अतिक्रूर महाकाय कल्पान्त दहनोपम}%। 
{भैरवाय नमस्तुभ्यम् अनुज्ञां दातुम् अर्हसि}%॥\\
\twolineshloka*
{दुर्भोजन-दुरालाप-दुष्प्रतिग्रह-सम्भवम्}%। 
{पापं हर मम क्षिप्रं सह्यकन्ये नमोऽस्तु ते}%॥\\
\twolineshloka*
{त्रिरात्रं जाह्नवीतीरे पञ्चरात्रं तु यामुने}%। 
{सद्यः पुनातु कावेरी पापमामरणान्तिकम्}%॥\\
\twolineshloka*
{गङ्गा गङ्गेति यो ब्रूयाद्योजनानां शतैरपि}%। 
{मुच्यते सर्वपापेभ्यो विष्णुलोकं स गच्छति}%॥% २८॥॥

\instruct{स्नात्वा धौतवस्त्रं धृत्वा कुलाचारवत् पुण्ड्रधारणं च कृत्वा आचम्य यज्ञोपवीतं धारयेत्।}{ஸ்நானம் செய்து மடி வஸ்த்ரம் அணிந்து குலாசாரத்தின் படி புண்ட்ரதாரணம் செய்து பிறகு ஆசமனம் செய்து யஜ்ஞோபவீததாரணம் செய்ய வேண்டும்.}


\sect{यज्ञोपवीत-धारणम्}

शुक्लाम्बरधरं + शान्तये। प्राणायामः।

ममोपात्त समस्तदुरितक्षयद्वारा श्रीपरमेश्वरप्रीत्यर्थम् श्रावण्यां पौर्णमास्याम् अध्यायोपाकर्मणि 
श्रौत-स्मार्त-विहित-नित्यकर्मानुष्ठान-सदाचार-योग्यता-सिद्ध्यर्थं
ब्रह्मतेजो\-ऽभि\-वृद्ध्यर्थं यज्ञोपवीत-धारणं करिष्ये।

अस्य श्री यज्ञोपवीत-धारण-महामन्त्रस्य परब्रह्म ऋषिः, त्रिष्टुप् छन्दः, परमात्मा देवता।
यज्ञोपवीत-धारणे विनियोगः।

% य॒ज्ञो॒प॒वी॒तं प॑रमं + तेजः॥

य॒ज्ञो॒प॒वी॒तं \sep प॒र॒मं \sep प॒वित्रं॑ \sep प्र॒जाप॑तेः \sep यत् \sep स॒ह॒जं \sep पु॒रस्तात्। आ॒यु॒ष्यं॑ \sep अ॒ग्र्यं॑ \sep प्रति॑मुञ्च-शु॒भ्रं \sep य॒ज्ञो॒प॒वी॒तं \sep बल॑मस्तु \sep तेजः॑॥

इति यज्ञोपवीतं धृत्वा, ॐ। आचम्य।

उपवीतं भिन्नतन्तुं जीर्णं कश्मलदूषितम्। विसृजामि जले ब्रह्मन् वर्चो दीर्घायुरस्तु मे।

\instruct{इति जीर्णम् उपवीतं विसृज्य पुनराचमनं कुर्यात्।}{என்று கூறி பழைய பூணூலை விஸர்ஜனம் செய்து மீண்டும் ஆசமனம் செய்யவும்.}

\sect{काण्डऋषि-तर्पणम्}

आचम्य। शुक्लाम्बरधरं + शान्तये। प्राणायामः।

ममोपात्त समस्तदुरितक्षयद्वारा श्रीपरमेश्वरप्रीत्यर्थम्
अद्य\-पूर्वोक्त एवं गुण-विशेषण-विशिष्टायाम्
अस्यां श्रावण्यां पौर्णमास्याम् अध्यायोपाकर्माङ्गं काण्डऋषि-तर्पणं करिष्ये।

\instruct{निवीती। उपवीतमङ्गुष्ठयोः सक्तं कृत्वा। सतिलाक्षताभिः अद्भिः ऋषितीर्थेन त्रिस्त्रिः।}{பூணூலை மாலையாக தரித்துக் கொண்டு வலது கட்டை விரலில் பிடித்துகொண்டு எள் அக்ஷதை சேர்த்து சுண்டுவிரல் பக்கமாக மும்மூன்று முறை ஜலம் விடவும்.}

प्रजापतिं काण्डऋषिं तर्पयामि। 
सोमं काण्डऋषिं तर्पयामि। 
अग्निं काण्डऋषिं तर्पयामि। 
विश्वान् देवान् काण्डऋषीꣴ\-स्तर्पयामि।

साꣳहितीर्देवताः उपनिषद\-स्तर्पयामि। 
याज्ञिकीर्देवताः उपनिषद\-स्तर्पयामि। 
वारुणीर्देवताः उपनिषद\-स्तर्पयामि। 
ब्रह्माणं स्वयम्भुवं तर्पयामि। %
\instruct{ब्रह्मतीर्थेन।}{இந்த மந்த்ரத்திற்கு மட்டும் ஜலம் மூன்று முறை உள்ளம் கைகளிருந்து முழங்கைகள் வழியாக கீழே விழும்படி விட வெண்டும்.}\\
सदसस्पतिं तर्पयामि।

\instruct{उपवीती।}{பூணூலை உபவீதமாக போட்டுக் கொள்ளவும்.} %
आचम्य।

\sect{वेदारम्भः}



शुक्लाम्बरधरं + शान्तये। प्राणायामः।

ममोपात्त समस्तदुरितक्षयद्वारा श्रीपरमेश्वरप्रीत्यर्थम्
श्रावण्यां पौर्णमास्याम् अध्यायोपाकर्मणि वेदारम्भं करिष्ये।

\instruct{इति सङ्कल्प्य। अप उपस्पृश्य।}{என்று ஸங்கல்பம் செய்து கொண்டு ஜலத்தை தொடவும்.}

ॐ श्री॒गु॒रु॒भ्यो॒ न॒मः॒। हरिः ओ(४)म्।

इ॒षे त्वो॒र्जे त्वा॑ वा॒यवः॑ स्थोपा॒यवः॑ स्थ दे॒वो वः॑ सवि॒ता प्रार्प॑यतु॒ श्रेष्ठ॑तमाय॒ कर्म॑ण॒ आ
प्या॑यध्वमघ्निया देवभा॒गमूर्ज॑स्वतीः॒ पय॑स्वतीः प्र॒जाव॑तीरनमी॒वा अ॑य॒क्ष्मा मा वः॑ स्ते॒न ई॑शत॒
माऽघशसो रु॒द्रस्य॑ हे॒तिः परि॑ वो वृणक्तु ध्रु॒वा अ॒स्मिन्गोप॑तौ स्यात ब॒ह्वीर्यज॑मानस्य
प॒शून्पा॑हि॥ %हरिः॑ ओ(३)म्।

य॒ज्ञस्य॑ घो॒षद॑सि॒ प्रत्यु॑ष्ट॒ꣳ॒ रक्षः॒ प्रत्यु॑ष्टा॒ अरा॑तयः॒ प्रेयम॑गाद्धि॒षणा॑ ब॒र्॒हिरच्छ॒
मनु॑ना कृ॒ता स्व॒धया॒ वित॑ष्टा॒ त आव॑हन्ति क॒वयः॑ पु॒रस्ताद्दे॒वेभ्यो॒ जुष्ट॑मि॒ह ब॒र्॒हिरा॒सदे॑
दे॒वानां परिषू॒तम॑सि व॒र्॒षवृ॑द्धमसि॒ देव॑बर्हि॒र्मा त्वा॒ऽन्वङ्मा ति॒र्यक्पर्व॑ ते
राध्यासमाच्छे॒त्ता ते॒ मा रि॑ष॒न्देव॑बर्हिः श॒तव॑ल्शं॒ वि रो॑ह स॒हस्र॑वल्शाः। वि व॒यꣳ
रु॑हेम पृथि॒व्याः स॒म्पृचः॑ पाहि सुस॒म्भृता त्वा॒ सम्भ॑रा॒म्यदि॑त्यै॒ रास्ना॑ऽसीन्द्रा॒ण्यै
स॒न्नह॑नं पू॒षा ते ग्र॒न्थिं ग्र॑थ्नातु॒ स ते॒ माऽऽस्था॒दिन्द्र॑स्य त्वा बा॒हुभ्या॒मुद्य॑च्छे॒
बृह॒स्पतेर्मू॒र्ध्ना ह॑राम्यु॒र्व॑न्तरि॑क्ष॒मन्वि॑हि देवङ्ग॒मम॑सि॥

शुन्ध॑ध्वं॒ दैव्या॑य॒ कर्म॑णे देवय॒ज्यायै॑ मात॒रिश्व॑नो घ॒र्मो॑ऽसि॒ द्यौर॑सि पृथि॒व्य॑सि वि॒श्वधा॑या
असि पर॒मेण॒ धाम्ना॒ दृꣳह॑स्व॒ मा ह्वा॒र्वसू॑नां प॒वित्र॑मसि श॒तधा॑रं॒ वसू॑नां प॒वित्र॑मसि
स॒हस्र॑धारꣳ हु॒तः स्तो॒को हु॒तो द्र॒फ्सोऽग्नये॑ बृह॒ते नाका॑य॒ स्वाहा॒ द्यावा॑पृथि॒वीभ्या॒ꣳ॒ सा
वि॒श्वायुः॒ सा वि॒श्वव्य॑चाः॒ सा वि॒श्वक॑र्मा॒ सम्पृ॑च्यध्वमृतावरीरू॒र्मिणी॒र्मधु॑मत्तमा म॒न्द्रा
धन॑स्य सा॒तये॒ सोमे॑न॒ त्वाऽऽत॑न॒च्मीन्द्रा॑य॒ दधि॒ विष्णो॑ ह॒व्यꣳ र॑क्षस्व॥ 

कर्म॑णे वां दे॒वेभ्यः॑ शकेयं॒ वेषा॑य त्वा॒ प्रत्यु॑ष्ट॒ꣳ॒ रक्षः॒ प्रत्यु॑ष्टा॒ अरा॑तयो॒ धूर॑सि॒
धूर्व॒ धूर्व॑न्तं॒ धूर्व॒ तं योऽस्मान्धूर्व॑ति॒ तं धूर्व॒ यं व॒यं धूर्वा॑म॒स्त्वं दे॒वाना॑मसि॒
सस्नि॑तमं॒ पप्रि॑तमं॒ जुष्ट॑तमं॒ वह्नि॑तमं देव॒हूत॑म॒मह्रु॑तमसि हवि॒र्धानं॒ दृꣳह॑स्व॒ मा
ह्वार्मि॒त्रस्य॑ त्वा॒ चक्षु॑षा॒ प्रेक्षे॒ मा भेर्मा सं वि॑क्था॒ मा त्वा।
हि॒ꣳ॒सि॒ष॒मु॒रु वाता॑य दे॒वस्य॑ त्वा सवि॒तुः प्र॑स॒वेऽश्विनोर्बा॒हुभ्यां पू॒ष्णो
हस्ताभ्याम॒ग्नये॒ जुष्टं॒ निर्व॑पाम्य॒ग्नीषोमाभ्यामि॒दं दे॒वाना॑मि॒दमु॑ नः स॒ह स्फा॒त्यै त्वा॒
नारात्यै॒ सुव॑र॒भि वि ख्ये॑षं वैश्वान॒रं ज्योति॒र्दृꣳह॑न्ता॒न्दुर्या॒
द्यावा॑पृथि॒व्योरु॒र्व॑न्तरि॑क्ष॒मन्वि॒ह्यदि॑त्यास्त्वो॒पस्थे॑ सादया॒म्यग्ने॑ ह॒व्यꣳ र॑क्षस्व॥ ॐ॥

ॐ॥ ब्रह्म॒ सन्ध॑त्तं॒ तन्मे॑ जिन्वतम्।
क्ष॒त्रꣳ सन्ध॑त्तं॒ तन्मे॑ जिन्वतम्।
इष॒ꣳ॒ सन्ध॑त्तं॒ तां मे॑ जिन्वतम्।
ऊर्ज॒ꣳ॒ सन्ध॑त्तं॒ तां मे॑ जिन्वतम्।
र॒यिꣳ सन्ध॑त्तं॒ तां मे॑ जिन्वतम्।
पुष्टि॒ꣳ॒ सन्ध॑त्तं॒ तां मे॑ जिन्वतम्।
प्र॒जाꣳ सन्ध॑त्तं॒ तां मे॑ जिन्वतम्।
प॒शून्त्सन्ध॑त्तं॒ तान्मे॑ जिन्वतम्। ॐ॥

ॐ भ॒द्रं कर्णे॑भिः शृणु॒याम॑ देवाः। भ॒द्रं प॑श्येमा॒क्षभि॒र्यज॑त्राः। 
स्थि॒रैरङ्गै᳚स्तुष्टु॒वाꣳ स॑स्त॒नूभिः॑। व्यशे॑म दे॒वहि॑तं॒ यदायुः॑। 
स्व॒स्ति न॒ इन्द्रो॑ वृ॒द्धश्र॑वाः। स्व॒स्ति नः॑ पू॒षा वि॒श्ववे॑दाः। 
स्व॒स्ति न॒स्तार्क्ष्यो॒ अरि॑ष्टनेमिः। स्व॒स्ति नो॒ बृह॒स्पति॑र्दधातु॥ ॐ॥

ॐ॥ सं॒ज्ञानं॑ वि॒ज्ञानं॑ प्र॒ज्ञानं॑ जा॒नद॑भिजा॒नत्।
स॒ङ्कल्प॑मानं प्र॒कल्प॑मानमुप॒\-कल्प॑मान॒मुप॑कॢप्तं कॢ॒प्तम्।
श्रेयो॒ वसी॑य आ॒यत्सम्भू॑तं भू॒तम्।
चि॒त्रः के॒तुः प्र॒भाना॒भान्त्स॒म्भान्।
ज्योति॑ष्मा॒ꣴ॒स्तेज॑स्वाना॒तप॒ꣴ॒\-स्तप॑न्नभि॒\-तपन्॑।
रो॒च॒नो रोच॑मानः शोभ॒नः शोभ॑मानः क॒ल्याणः॑।
दर्\mbox{}शा॑ दृ॒ष्टा द॑र्\mbox{}श॒ता वि॒श्वरू॑पा सुदर्\mbox{}श॒ना।
आ॒प्याय॑माना॒ प्याय॑माना॒ प्याया॑ सू॒नृतेरा᳚।
आ॒पूर्य॑माणा॒ पूर्य॑माणा पू॒रय॑न्ती पू॒र्णा पौ᳚र्णमा॒सी। ॐ॥

ॐ॥ प्र॒सु॒ग्मन्ता॑ धि॒यसा॒नस्य॑ स॒क्षणि॑ व॒रेभि॑र्व॒राꣳ अ॒भि षु॒ प्रसी॑दत । अ॒स्माक॒मिन्द्र॑ उ॒भयं॑
जुजोषति॒ यथ्सौ॒म्यस्यान्ध॑सो॒ बुबो॑धति । अ॒नृ॒क्ष॒रा ऋ॒जवः॑ सन्तु॒ पन्था॒ येभिः॒ सखा॑यो॒ यन्ति॑ नो
वरे॒यम् । सम॑र्य॒मा सं भगो॑ नो निनीया॒थ्सञ्जास्प॒त्यꣳ सु॒यम॑मस्तु देवाः। ॐ॥

ॐ॥ अथातः दर्शपूर्णमासौ व्याख्यास्यामः। प्रातरग्निहोत्रं हुत्वा। अन्यमाहवनीयं प्रणीय।
अग्नीनन्वादधाति। नगतश्रियोऽन्यमग्निं प्रणयति॥ ॐ॥ 

ॐ॥ अथ कर्माणि -- आचारात् - यानि गृह्यन्ते॥ ॐ॥  

ॐ॥  अथातः - सामयाचारिकान् धर्मान् व्याख्यास्यामः॥ ॐ॥

ॐ॥ अथ शिक्षां प्रवक्ष्यामि - पाणिनीयं मतं यथा॥ ॐ॥

ॐ पञ्चसंवत्सरमयं युगाध्यक्षं प्रजापतिम्॥ ॐ॥

ॐ॥ मयरसतजभन लगसम्मितं भ्रमति वाङ्मयं जगति यस्य॥ ॐ॥

ॐ॥ गौः। ग्मा। ज्मा। क्ष्मा। क्षा। क्षमा। क्षोणी। क्षितिः। अवनिः। ॐ॥

ॐ॥ अ इ उ ण्। ऋ ऌ क्। ए ओ ङ्। ऐ औ च्। ह य व र ट्। ल ण्। ञ म ङ् ण न म्।
झ भ ञ्। घ ढ ध ष्। ज ब ग ड द श्। ख फ छ ठ थ च ट त व्।
क प य्। श ष स र्। ह ल्। इति माहेश्वराणि सूत्राणि॥ ॐ॥

ॐ॥ वृद्धिरादैच्। अदेङ्गुणः॥ ॐ॥  
ॐ॥ अथातश्छन्दसां विवृतिं व्याख्यास्यामः॥ ॐ॥ 

ॐ॥ गीर्नः श्रेयः। धेनवश्रीः। रुद्रस्तु नम्यः। भगो हि याज्यः।
धन्येयं नारी। धनवान् पुत्रः॥ ॐ॥

ॐ॥ अथ वर्णसमाम्नायः॥ ॐ॥
ॐ॥ अथातो धर्मजिज्ञासा॥ ॐ॥
ॐ॥ अथातो ब्रह्मजिज्ञासा॥ ॐ॥

ॐ॥ अ॒ग्निमी᳚ळे पु॒रोहि॑तं य॒ज्ञस्य॑ दे॒वमृ॒त्विजम्᳚। होता᳚रं रत्न॒-धात॑मम्॥ ॐ॥\\
ॐ॥ इ॒षेत्वो॒र्जे त्वा॑ वा॒यवः॑ स्थो पा॒यवः॑ स्थ दे॒वो वः॑ सवि॒ता प्रार्प॑यतु॒ श्रेष्ठ॑तमाय॒ कर्म॑णे॥ ॐ॥\\
ॐ॥ अग्न॒ आया॑हि वी॒तये॑ गृणा॒नो ह॒व्यदा॑तये। नि होता॑ सत्सि ब॒र्हिषि॑॥ ॐ॥\\
ॐ॥ शन्नो॑ दे॒वीर॒भिष्ट॑य॒ आपो॑ भवन्तु पी॒तये᳚। शं योर॒भिस्र॑वन्तु नः॥ हरिः॑ ॐ॥

ॐ॥ नमो॒ ब्रह्म॑णे॒ नमो॑ अस्त्व॒ग्नये॒ नमः॑ पृथि॒व्यै नम॒ ओष॑धीभ्यः।
नमो॑ वा॒चे नमो॑ वा॒चस्पत॑ये॒ नमो॒ विष्ण॑वे बृह॒ते क॑रोमि॥ (त्रिः)

ॐ तत्सत्॥

\sect{गायत्री-जपः}

आचमनम्। पवित्रपाणिः। दर्भेष्वासीनः। दर्भान् धारयमाणः।

शुक्लाम्बरधरं + शान्तये। प्राणायामः।

ममोपात्तसमस्तदुरितक्षयद्वारा श्रीपरमेश्वरप्रीत्यर्थं शुभे शोभने मुहूर्ते अद्यब्रह्मणः
द्वितीयपरार्द्धे श्वेतवराहकल्पे वैवस्वतमन्वन्तरे अष्टाविंशतितमे कलियुगे प्रथमे पादे
जम्बूद्वीपे भारतवर्षे भरतखण्डे मेरोः दक्षिणेपार्श्वे शकाब्दे अस्मिन् वर्तमाने व्यावहारिके
प्रभवादीनां षष्ट्याः संवत्सराणां मध्ये 
\textbf{क्रोधि}-नाम-संवत्सरे    \textbf{दक्षिणायने} \textbf{वर्ष}-ऋतौ   \textbf{सिंह}-\textbf{श्रावण}-मासे  \textbf{कृष्ण}-पक्षे    \textbf{प्रथमायां} शुभतिथौ  \textbf{भौम}\-वासर\-युक्तायां   \textbf{शतभिषङ्}-नक्षत्र\-युक्तायाम्    \textbf{अतिगण्ड}-योग\-युक्तायां \textbf{बालव}-करण (१०:१५; \textbf{कौलव}-करण)\-युक्तायाम्    एवं-गुण-विशेषण-विशिष्टायाम्     अस्यां \textbf{प्रथमायां}

मिथ्याधीत-दोष-प्रायश्चितार्थं दोषवत्सु अपतनीय-प्रायश्चितार्थं संवत्सर\-प्रायश्चित्तार्थं च
अष्टोत्तरसहस्रसङ्ख्यया सावित्रीं समिधम् आधास्ये (अथवा अष्टोत्तर\-सहस्र\-सङ्ख्यया गायत्री\-महा\-मन्त्र\-जपं करिष्ये)। %
\instruct{इति सङ्कल्प्य दर्भान्निरस्य अप उपस्पृश्य}{என்று ஸங்கல்பம் செய்து கொண்டு தர்பைகளை கீழே போட்டுவிட்டு ஜலத்தை தொடவும்.}

\instruct{प्रणवस्य ऋषिर्ब्रह्मा इत्याद्यारभ्य अष्टोत्तरसहस्रगायत्रीजपं कृत्वा प्राणायामं कृत्वा उपस्थानं कृत्वा पवित्रं
विसृज्य आचामेत्।}{ப்ரணவஸ்ய ரிஷி ப்ரஹ்மா என்று தொடங்கி காயத்ரி ஜபம் வரையில் செய்து உத்தமே ஶிக$^2$ரே என்ற மந்த்ரத்தை உபஸ்தானமாக கூறி விட்டு பவித்ரவிஸர்ஜனம் செய்து ஆசமனம் செய்யவும்.}

\instruct{यद्वा---सत्यां शक्तौ समिदाहुतिं कुर्यात्।}{முடிந்த வரையில் காயத்ரியை ஹோமமாக செய்ய வேண்டும்.} 

\instruct{श्रोत्रियागारादाहृतेऽग्नौ यथाविधिप्रतिष्ठापिते परिस्तीर्णे परिषिक्ते घृतेनाभ्युज्य एकैकशः प्रणवव्याहृतिपूर्वया गायत्र्या समित्सहस्रमादध्यात्। न स्वाहाकारः। परिषिच्य उपतिष्ठेत।}{லௌகிக அக்னியை பிரதிஷ்ட்டை செய்து பரிஸ்தரணம் அமைத்து பரிஷேசனம் செய்து ஒவ்வொரு ஸமித்தாக நெய்யில் தொட்டு ப்ரணவ வ்யாஹ்ரு'தி பூர்வமாக காயத்ரி மந்த்ரத்தினால்  ஸ்வஹா என்ற பதம் சொல்லாமல் ஹோமம் செய்யவும். பரிஷேசனம் செய்து உத்தமே ஶிக$^2$ரே என்ற மந்த்ரத்தை உபஸ்தானமாக கூறி விட்டு பவித்ரவிஸர்ஜனம் செய்து ஆசமனம் செய்யவும்.}

\closesection
\endgroup
