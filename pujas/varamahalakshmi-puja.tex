% !TeX program = XeLaTeX
% !TeX root = ../pujavidhanam.tex

\setlength{\parindent}{0pt}
\chapt{श्री-वरमहालक्ष्मी-पूजा}

\sect{पूर्वाङ्गविघ्नेश्वरपूजा}

(आचम्य)
\twolineshloka*
{शुक्लाम्बरधरं विष्णुं शशिवर्णं चतुर्भुजम्}
{प्रसन्नवदनं ध्यायेत् सर्वविघ्नोपशान्तये}
 
प्राणान्  आयम्य।  ॐ भूः + भूर्भुवः॒ सुव॒रोम्।
 
(अप उपस्पृश्य, पुष्पाक्षतान् गृहीत्वा)\\
ममोपात्तसमस्त दुरितक्षयद्वारा \\
श्रीपरमेश्वरप्रीत्यर्थं करिष्यमाणस्य कर्मणः\\
 निर्विघ्नेन परिसमाप्त्यर्थम् आदौ विघ्नेश्वरपूजां करिष्ये।

\twolineshloka*
{ॐ ग॒णानां᳚ त्वा ग॒णप॑तिꣳ हवामहे क॒विं क॑वी॒नामु॑प॒मश्र॑वस्तमम्}
{ज्ये॒ष्ठ॒राजं॒ ब्रह्म॑णां ब्रह्मणस्पत॒ आ नः॑ शृ॒ण्वन्नू॒तिभिः॑ सीद॒ साद॑नम्}
अस्मिन् हरिद्राबिम्बे महागणपतिं ध्यायामि, आवाहयामि।\\


ॐ महागणपतये नमः  आसनं समर्पयामि।\\
पादयोः पाद्यं समर्पयामि। हस्तयोरर्घ्यं समर्पयामि।\\
आचमनीयं समर्पयामि।\\
ॐ भूर्भुवस्सुवः। शुद्धोदकस्नानं समर्पयामि।\\
स्नानानन्तरमाचमनीयं समर्पयामि।\\
वस्त्रार्थमक्षतान् समर्पयामि।\\
यज्ञोपवीताभरणार्थे अक्षतान् समर्पयामि।\\
दिव्यपरिमलगन्धान् धारयामि।\\
गन्धस्योपरि हरिद्राकुङ्कुमं समर्पयामि। अक्षतान् समर्पयामि। \\
पुष्पमालिकां समर्पयामि। पुष्पैः पूजयामि।

\dnsub{अर्चना}
% \setenumerate{label=\devanumber.}
% \renewcommand{\labelenumi}{\devanumber\theenumi.}
\begin{enumerate}%[label=\devanumber\value{enumi}]
\begin{minipage}{0.475\linewidth}   
\item ॐ सुमुखाय नमः
\item ॐ एकदन्ताय नमः
\item ॐ कपिलाय नमः
\item ॐ गजकर्णकाय नमः
\item ॐ लम्बोदराय नमः
\item ॐ विकटाय नमः
\item ॐ विघ्नराजाय नमः
\item ॐ विनायकाय नमः
\item ॐ धूमकेतवे नमः
  \end{minipage}
  \begin{minipage}{0.525\linewidth}
\item ॐ गणाध्यक्षाय नमः
\item ॐ फालचन्द्राय नमः
\item ॐ गजाननाय नमः
\item ॐ वक्रतुण्डाय नमः
\item ॐ शूर्पकर्णाय नमः
\item ॐ हेरम्बाय नमः
\item ॐ स्कन्दपूर्वजाय नमः
\item ॐ सिद्धिविनायकाय नमः
\item ॐ विघ्नेश्वराय नमः
  \end{minipage}
\end{enumerate}
नानाविधपरिमलपत्रपुष्पाणि समर्पयामि॥\\
धूपमाघ्रापयामि।\\
अलङ्कारदीपं सन्दर्शयामि।\\
नैवेद्यम्।\\
ताम्बूलं समर्पयामि।\\
कर्पूरनीराजनं समर्पयामि।\\
कर्पूरनीराजनानन्तरमाचमनीयं समर्पयामि।\\
{वक्रतुण्डमहाकाय कोटिसूर्यसमप्रभ।}\\
{अविघ्नं कुरु मे देव सर्वकार्येषु सर्वदा॥}\\
प्रार्थनाः समर्पयामि।

अनन्तकोटिप्रदक्षिणनमस्कारान् समर्पयामि।\\
छत्त्रचामरादिसमस्तोपचारान् समर्पयामि।\\

 
\sect{प्रधान-पूजा — श्रीमहालक्ष्मी-पूजा}

\twolineshloka*
{शुक्लाम्बरधरं विष्णुं शशिवर्णं चतुर्भुजम्}
{प्रसन्नवदनं ध्यायेत् सर्वविघ्नोपशान्तये}
 
\dnsub{सङ्कल्पः}

ममोपात्त-समस्त-दुरित-क्षयद्वारा श्री-परमेश्वर-प्रीत्यर्थं शुभे शोभने मुहूर्ते अद्य ब्रह्मणः
द्वितीयपरार्द्धे श्वेतवराहकल्पे वैवस्वतमन्वन्तरे अष्टाविंशतितमे कलियुगे प्रथमे पादे
जम्बूद्वीपे भारतवर्षे भरतखण्डे मेरोः दक्षिणे पार्श्वे शकाब्दे अस्मिन् वर्तमाने व्यावहारिकाणां प्रभवादीनां षष्ट्याः संवत्सराणां मध्ये (  )\see{app:samvatsara_names} नाम संवत्सरे दक्षिणायने 
वर्ष-ऋतौ (कटक/सिंह)-श्रावण-मासे शुक्लपक्षे ( ) शुभतिथौ भृगुवासरयुक्तायाम्
(  )\see{app:nakshatra_names} नक्षत्र (  )\see{app:yoga_names} नाम  योग  ( )\see{app:karanam_names} करण युक्तायां च एवं गुण विशेषण विशिष्टायाम्
अस्याम् ( ) शुभतिथौ 
अस्माकं सहकुटुम्बानां क्षेमस्थैर्य-धैर्य-वीर्य-विजय-आयुरारोग्य-ऐश्वर्याभिवृद्ध्यर्थम्
 धर्मार्थकाममोक्ष\-चतुर्विधफलपुरुषार्थसिद्ध्यर्थं पुत्रपौत्राभि\-वृद्ध्यर्थम् इष्टकाम्यार्थसिद्ध्यर्थम्
मम इहजन्मनि पूर्वजन्मनि जन्मान्तरे च सम्पादितानां ज्ञानाज्ञानकृतमहा\-पातकचतुष्टय-व्यतिरिक्तानां रहस्यकृतानां प्रकाशकृतानां सर्वेषां पापानां सद्य अपनोदनद्वारा सकल-पापक्षयार्थं आयुष्मत्सत्सन्तानसमृद्ध्यर्थं दीर्घसौमङ्गल्यावाप्त्यर्थं
श्रीवरमहालक्ष्मी-प्रसादसिद्ध्यर्थं 
यथाशक्ति-ध्यानावाहनादिषोडशोपचारैः 
श्रीवरमहालक्ष्मी-पूजां करिष्ये। तदङ्गं कलशपूजां च करिष्ये। 


श्रीविघ्नेश्वराय नमः यथास्थानं प्रतिष्ठापयामि। शोभनार्थे क्षेमाय पुनरागमनाय च।\\
(गणपति-प्रसादं शिरसा गृहीत्वा)

\dnsub{आसन-पूजा}
\centerline{पृथिव्या  मेरुपृष्ठ  ऋषिः।  सुतलं  छन्दः।  कूर्मो  देवता॥}
\twolineshloka*
{पृथ्वि  त्वया  धृता  लोका  देवि  त्वं  विष्णुना  धृता}
{त्वं  च  धारय  मां  देवि  पवित्रं  चाऽऽसनं  कुरु}


\dnsub{घण्टापूजा}
\twolineshloka*
{आगमार्थं तु देवानां गमनार्थं तु रक्षसाम्}
{घण्टारवं करोम्यादौ देवताऽऽह्वानकारणम्}


\dnsub{कलशपूजा}
ॐ कलशाय नमः दिव्यगन्धान् धारयामि।\\
ॐ गङ्गायै नमः। ॐ यमुनायै नमः। ॐ गोदावर्यै नमः।  ॐ सरस्वत्यै नमः। ॐ नर्मदायै नमः। ॐ सिन्धवे नमः। ॐ कावेर्यै नमः।\\
ॐ सप्तकोटिमहातीर्थान्यावाहयामि।\\[-0.25ex]

(अथ कलशं स्पृष्ट्वा जपं कुर्यात्) \\
आपो॒ वा इ॒द सर्वं॒ विश्वा॑ भू॒तान्याप॑ प्रा॒णा वा आप॑ प॒शव॒ आपो\-ऽन्न॒मापोऽमृ॑त॒माप॑ स॒म्राडापो॑ वि॒राडाप॑ स्व॒राडाप॒श्\-छन्दा॒स्यापो॒ ज्योती॒ष्यापो॒ यजू॒ष्याप॑ स॒त्यमाप॒ सर्वा॑ दे॒वता॒ आपो॒ भूर्भुव॒ सुव॒राप॒ ओम्॥\\

\twolineshloka* 
{कलशस्य मुखे विष्णुः कण्ठे रुद्रः समाश्रितः}
{मूले तत्र स्थितो ब्रह्मा मध्ये मातृगणाः स्मृताः}
\threelineshloka* 
{कुक्षौ तु सागराः सर्वे सप्तद्वीपा वसुन्धरा}
{ऋग्वेदोऽथ यजुर्वेदः सामवेदोऽप्यथर्वणः}
{अङ्गैश्च सहिताः सर्वे कलशाम्बुसमाश्रिताः}
\twolineshloka* 
{गङ्गे च यमुने चैव गोदावरि सरस्वति}
{नर्मदे सिन्धुकावेरि जलेऽस्मिन् सन्निधिं कुरु}
\twolineshloka*
{सर्वे समुद्राः सरितः तीर्थानि च ह्रदा नदाः}
{आयान्तु देवपूजार्थं दुरितक्षयकारकाः}

\centerline{ॐ भूर्भुवः॒ सुवो॒ भूर्भुवः॒ सुवो॒ भूर्भुवः॒ सुवः॑।}

(इति कलशजलेन सर्वोपकरणानि आत्मानं च प्रोक्ष्य।)


\dnsub{आत्म-पूजा}
ॐ आत्मने नमः, दिव्यगन्धान् धारयामि।
\begin{multicols}{2}
१. ॐ आत्मने नमः\\
२. ॐ अन्तरात्मने नमः\\
३. ॐ योगात्मने नमः\\
४. ॐ जीवात्मने नमः\\
५. ॐ परमात्मने नमः\\
६. ॐ ज्ञानात्मने नमः
\end{multicols}
समस्तोपचारान् समर्पयामि।

\twolineshloka*
{देहो देवालयः प्रोक्तो जीवो देवः सनातनः}
{त्यजेदज्ञाननिर्माल्यं सोऽहं भावेन पूजयेत्}


\begin{minipage}{\linewidth}
\dnsub{पीठ-पूजा}

\begin{multicols}{2}
\begin{enumerate}
\item ॐ आधारशक्त्यै नमः
\item ॐ मूलप्रकृत्यै नमः
\item ॐ आदिकूर्माय नमः 
\item ॐ आदिवराहाय नमः
\item ॐ अनन्ताय नमः
\item ॐ पृथिव्यै नमः
\item ॐ रत्नमण्डपाय नमः
\item ॐ रत्नवेदिकायै नमः
\item ॐ स्वर्णस्तम्भाय नमः
\item ॐ श्वेतच्छत्त्राय नमः
\item ॐ कल्पकवृक्षाय नमः
\item ॐ क्षीरसमुद्राय नमः 
\item ॐ सितचामराभ्यां नमः
\item ॐ योगपीठासनाय नमः
\end{enumerate}
\end{multicols}

\end{minipage}

\dnsub{गुरु ध्यानम्}

\twolineshloka*
{गुरुर्ब्रह्मा गुरुर्विष्णुर्गुरुर्देवो महेश्वरः}
{गुरुः साक्षात् परं ब्रह्म तस्मै श्री गुरवे नमः}


\sect{षोडशोपचारपूजा}
\begin{center}

\twolineshloka*
{पद्मासनां पद्मकरां पद्ममालाविभूषिताम्}
{क्षीरसागरसम्भूतां हेमवर्णसमप्रभाम्}

\twolineshloka*
{क्षीरवर्गसमं वस्त्रं दधानां हरिवल्लभाम्}
{भावये भक्तियोगेन कलशेऽस्मिन् मनोहरे} 

\textbf{अस्मिन् कुम्भे (प्रतिमायां) श्री-वरलक्ष्मीं ध्यायामि।}
\medskip


\twolineshloka*
{बालभानुप्रतीकाशे पूर्णचन्द्रनिभानने} 
{सूत्रेऽस्मिन् सुस्थिरा भूत्वा प्रयच्छ बहुलान् वरान्} 
\textbf{इति दोरस्थापनम्।}\medskip

\twolineshloka*
{सर्वमङ्गलमाङ्गल्ये विष्णुवक्षस्थलालये} 
{आवाहयामि देवि त्वामभीष्टफलदा भव}
\textbf{श्री-वरलक्ष्मीमावाहयामि।}

\twolineshloka*
{अनेकरत्नखचितं क्षीरसागरसम्भवे}
{स्वर्णसिहासनं देवि स्वीकुरुष्व हरिप्रिये} 
\textbf{श्री-वरलक्ष्म्यै नमः आसनं समर्पयामि।}


\twolineshloka*
{गङ्गादिसरिदानीतं गन्धपुष्पसमन्वितम्}
{पाद्यं ददामि ते देवि प्रसीद परमेश्वरि}
\textbf{श्री-वरलक्ष्म्यै नमः पाद्यं समर्पयामि।}
\medskip

\twolineshloka*
{गङ्गानदीसमानीतं सुवर्णकलशस्थितम्}
{गृहाणार्घ्यं मया दत्तं पुत्रपौत्रफलप्रदे}
\textbf{श्री-वरलक्ष्म्यै नमः अर्घ्यं समर्पयामि।}
\medskip

\twolineshloka*
{प्रसन्नं शीतलं तोयं प्रसन्नमुखपङ्कजे}
{गृहाणाचमनार्थाय गरुडध्वजवल्लभे}
\textbf{श्री-वरलक्ष्म्यै नमः आचमनं समर्पयामि।}
\medskip

\twolineshloka*
{महालक्ष्मि महादेवि मध्वाज्यदधिसंयुतम्}
{मधुपर्कं गृहाणेमं मधुसूदनवल्ल्भे}
\textbf{श्री-वरलक्ष्म्यै नमः मधुपर्कं समर्पयामि।}
\medskip

\twolineshloka*
{पयोदधिघृतैर्युक्तं शर्करामधुसंयुतम्}
{पञ्चामृतं गृहाणेदं वरलक्ष्मि नमोऽस्तु ते}
\textbf{श्री-वरलक्ष्म्यै नमः पञ्चामृतं समर्पयामि।}
\medskip

\twolineshloka*
{हेमकुम्भस्थितं स्वच्छं गङ्गादिसरिदाहृतम्}
{स्नानार्थं सलिलं देवि गृह्यतां सागरात्मजे}
\textbf{श्री-वरलक्ष्म्यै नमः स्नानं समर्पयामि।}
\medskip

\twolineshloka*
{दिव्याम्बरयुगं सूक्ष्मं कञ्चुकं च मनोहरम्}
{वरलक्ष्मि महादेवि गृहाणेदं मयाऽर्पितम्}
\textbf{श्री-वरलक्ष्म्यै नमः वस्त्रं समर्पयामि।}
\medskip

\twolineshloka*
{माङ्गल्यमणिसंयुक्तं मुक्ता-विद्रुमसंयुतम्}
{दत्तं मङ्गलसूत्रं च गृहाण हरिवल्लभे}
\textbf{श्री-वरलक्ष्म्यै नमः कण्ठसूत्रं समर्पयामि।}
\medskip

\twolineshloka*
{रत्नताटङ्ककेयूरहारकङ्कणभूषिते} 
{भूषणानि महार्हाणि गृहाण करुणानिधे}
\textbf{श्री-वरलक्ष्म्यै नमः आभरणानि समर्पयामि।}
\medskip

\twolineshloka*
{कर्पूरचन्दनोपेतं कस्तूरीकुङ्कुमान्वितम्}
{सर्वगन्धं गृहाणाद्य सर्वमङ्गलदायिनि}
\textbf{श्री-वरलक्ष्म्यै नमः गन्धान् धारयामि।}
\medskip

\twolineshloka*
{शालिजातान् चन्द्रवर्णान् स्निग्धमौक्तिकसन्निभान्}
{अक्षतान् देवि गृह्णीष्व पङ्कजाक्षस्य वल्लभे}
\textbf{श्री-वरलक्ष्म्यै नमः अक्षतान् समर्पयामि।}
\medskip

\twolineshloka*
{मन्दारपारिजाताब्जैः केतक्युत्पलपाटलैः}
{मल्लिकाजातिवकुलैः पुष्पैस्त्वां पूजयाम्यहम्}
\textbf{श्री-वरलक्ष्म्यै नमः पुष्पमालां समर्पयामि। पुष्पैः पूजयामि।}
\medskip

 
\end{center}


\dnsub{अङ्गपूजा}
\begin{longtable}{ll@{— }l}
१. & वरलक्ष्म्यै नमः & पादौ पूजयामि\\
२. & महालक्ष्म्यै नमः & गुल्फौ पूजयामि\\
३. & इन्दिरायै नमः &  जङ्घे पूजयामि\\
४. & चण्डिकायै नमः & जानुनी पूजयामि\\
५. & क्षीराब्धितनयायै नमः & ऊरू  पूजयामि\\
६. & पीताम्बरधारिण्यै नमः & कटिं पूजयामि\\
७. & सागरसम्भवायै नमः & गुह्यं पूजयामि\\
८. & नारायणप्रियायै नमः & नाभिं पूजयामि\\
९. & जगत्कुक्ष्यै नमः & कुक्षिं पूजयामि\\
१०. & विश्वजनन्यै नमः & वक्षः पूजयामि\\
११. & सुस्तन्यै नमः & स्तनौ पूजयामि\\
१२. & कम्बुकण्ठ्यै नमः &  कण्ठं पूजयामि\\
१३. & सुन्दर्यै नमः & स्कन्धौ पूजयामि\\
१४. & पद्महस्तायै नमः & हस्तान् पूजयामि\\
१५. & बहुप्रदायै नमः & बाहून् पूजयामि\\
१६. & चन्द्रवदनायै नमः & वक्त्रं पूजयामि\\
१७. & चञ्चलायै नमः & चुबुकं पूजयामि\\
१८. & बिम्बोष्ठ्यै नमः & ओष्ठं पूजयामि\\
१९. & अनघायै नमः & अधरं पूजयामि\\
२०. & सुकपोलायै नमः & कपोलौ पूजयामि\\
२१. & फलप्रदायै नमः & फालम् पूजयामि\\
२२. & नीलालकायै नमः & अलकान् पूजयामि\\
२३. & शिवायै नमः & शिरः पूजयामि\\
२४. & सर्वमङ्गलायै नमः &  सर्वाण्यङ्गानि पूजयामि\\
\end{longtable}

\dnsub{अष्टलक्ष्मी-अर्चना}

\begin{multicols}{2}
\begin{enumerate}
\item आदिलक्ष्म्यै नमः
\item धान्यलक्ष्म्यै नमः
\item धैर्यलक्ष्म्यै नमः
\item गजलक्ष्म्यै नमः
\item सन्तानलक्ष्म्यै नमः
\item विजयलक्ष्म्यै नमः
\item विद्यालक्ष्म्यै नमः
\item धनलक्ष्म्यै नमः
\item वरलक्ष्म्यै नमः
\item महालक्ष्म्यै नमः 
\end{enumerate}
\end{multicols}



\begingroup
\centering
\setlength{\columnseprule}{1pt}
\let\chapt\sect
\input{../namavali-manjari/100/Lakshmi_108.tex}

\endgroup

\textbf{श्री-वरलक्ष्म्यै नमः नानाविध\-परिमल\-पत्रपुष्पाणि समर्पयामि।}

\sect{उत्तराङ्गपूजा}
\begin{center}
\twolineshloka*
{धूपं ददामि ते रम्यं गुग्गुल्वगरुसंयुतम्} 
{गृहाण त्वं महालक्ष्मि भक्तानामिष्टदायिनि}
\textbf{श्री-वरलक्ष्म्यै नमः धूपमाघ्रापयामि।}
\medskip

\twolineshloka*
{साज्यं त्रिवर्तिसंयुक्तं सर्वाभीष्टप्रदायिनि} 
{दीपं गृहाण कमले देहि मे सर्वमीप्सितम्}
\textbf{श्री-वरलक्ष्म्यै नमः अलङ्कार-दीपं सन्दर्शयामि। }
\medskip

\twolineshloka*
{नानाभक्ष्यसमायुक्तं नानाफलसमन्वितम्}
{नैवेद्यं गृह्यतां देवि नारायणकुटुम्बिनि} 
\textbf{श्री-वरलक्ष्म्यै नमः नैवेद्यं समर्पयामि। }
\medskip

\twolineshloka*
{उशीरवासितं तोयं शीतलं शशिसोदरि} 
{पानाय गृह्यतां देवि पारावारतनूभवे} 
\textbf{श्री-वरलक्ष्म्यै नमः पानीयं समर्पयामि। }
\medskip


\twolineshloka*
{पूगीफलं सकर्पूरं नागवल्लीदलानि च} 
{चूर्णं च चन्द्रसहजे गृह्यन्तां हरिवल्लभे}
\textbf{श्री-वरलक्ष्म्यै नमः ताम्बूलं समर्पयामि। }
\medskip


\twolineshloka*
{नीराजनं नीरजाक्षि नारायणविलासिनि }
{गृह्यतामर्पितं भक्त्या गरुडध्वजभामिनि}
\textbf{श्री-वरलक्ष्म्यै नमः कर्पूरनीराजनं  सन्दर्शयामि}
\textbf{कर्पूरनीराजनानन्तरम् आचमनीयं समर्पयामि।}
\medskip

\twolineshloka*
{पुष्पाञ्जलिं गृहाणेमं पुरुषोत्तमवल्लभे}
{वरलक्ष्मि नमस्तुभ्यं वरान्देहि ममाखिलान्}
\textbf{श्री-वरलक्ष्म्यै नमः मन्त्रपुष्पाञ्जलिं समर्पयामि।}
\medskip

\twolineshloka*
{सर्वमङ्गललाभाय सर्वपापनिवृत्तये}
{प्रदक्षिणं करोम्यद्य प्रसीद परमेश्वरि}
\textbf{श्री-वरलक्ष्म्यै नमः प्रदक्षिणं समर्पयामि।}
\medskip

\fourlineindentedshloka*
{नमोऽस्तु नालीकनिभाननायै} 
{नमोऽस्तु नारायणवल्लभायै }
{नमोऽस्तु रत्नाकरसम्भवायै} 
{नमोऽस्तु लक्ष्म्यै जगतां जनन्यै} 
\textbf{श्री-वरलक्ष्म्यै नमः नमस्कारान् समर्पयामि।}
\medskip

\twolineshloka*
{आयुरारोग्यमैश्वर्यं पुत्रपौत्रान् पशून् धनम्}
{शत्रुक्षयं महालक्ष्मि प्रयच्छ करुणानिधे}
\textbf{श्री-वरलक्ष्म्यै नमः प्रार्थनाः समर्पयामि।}
\medskip

\sect{दोरग्रन्थि-पूजा}

\begin{longtable}{l@{— }l}
कमलायै नमः & प्रथमग्रन्थिं पूजयामि।\\
रमायै नमः & द्वितीयग्रन्धिं पूजयामि।\\
लोकमात्रे नमः & तृतीयग्रन्थिं पूजयामि।\\
विश्वजनन्यै नमः & चतुर्थग्रन्थिं पूजयामि।\\
महालक्ष्म्यै नमः & पञ्चमग्रन्थिं पूजयामि।\\
क्षीराब्धितनयायै नमः & षष्ठग्रन्थिं पूजयामि।\\
विश्वसाक्षिण्यै नमः & सप्तमग्रन्थिं पूजयामि।\\
हरिवल्लभायै नमः & अष्टमग्रन्थिं पूजयामि।\\
चन्द्रसहोदर्यै नमः & नवमग्रन्थिं पूजयामि।\\
\end{longtable}

\twolineshloka*
{सर्वमङ्गलमाङ्गल्ये सर्वपापप्रणाशिनि} 
{दोरकं प्रतिगृह्णामि सुप्रीता भव सर्वदा} 
इति दोरकं हस्तेन गृहीत्वा।


\twolineshloka*
{करिष्यामि व्रतं देवि त्वद्भक्ता त्वत्परायणा}
{श्रियं देहि यशो देहि सौभाग्यं देहि मे शुभे}

\twolineshloka*
{नवतन्तुसमायुक्तं नवग्रन्थिसमन्वितम्}
{बध्नीयां दक्षिणे हस्ते दोरकं हरिवल्लभे}
इति दोरकं बध्नीयात्।

\dnsub{अर्घ्यम्}

ममोपात्त-समस्त-दुरित-क्षयद्वारा श्री-वरलक्ष्मी-प्रीत्यर्थं वरलक्ष्मीपूजान्ते क्षीरार्घ्यप्रदानं करिष्ये॥

\twolineshloka*
{गोक्षीरेण युतं देवि गन्धपुष्पसमन्वितम्}
{अर्घ्यं गृहाण वरदे वरलक्ष्मि नमोऽस्तु ते} 
\textbf{श्री-वरलक्ष्म्यै नमः इदमर्घ्यम् इदमर्घ्यम् इदमर्घ्यम्।}

\dnsub{उपायन-दानम्}

\twolineshloka*
{इन्दिरा प्रतिगृह्णाति इन्दिरा वै ददाति च}
{इन्दिरा तारयेद् द्वाभ्यां इन्दिरायै नमो नमः}

\twolineshloka*
{हिरण्यगर्भगर्भस्थं हेमबीजं विभावसोः}
{अनन्तपुण्यफलदम् अतः शान्तिं प्रयच्छ मे}

श्रीवरमहालक्ष्मीपूजाकाले अस्मिन् मया क्रियमाण\\
श्रीवरमहालक्ष्मीपूजायां यद्देयमुपायनदानं तत्प्रत्यायाम्नार्थं हिरण्यं\\
श्रीवरमहालक्ष्मीप्रीतिं कामयमानः\\
मनसोद्दिष्टाय ब्राह्मणाय तुभ्यमहं सम्प्रददे नमः न मम।\\ 
अनया पूजया श्रीवरमहालक्ष्मीः प्रीयताम्। 

इति उपायनं दत्त्वा सुवासिनीश्च सम्पूजयेत्॥


इति वरलक्ष्मीपूजाविधिः॥

\end{center}

\sect{प्रार्थना}




\twolineshloka*
{दामोदरि नमस्तेऽस्तु नमस्त्रैलोक्यमातृके}
{नमस्तेऽस्तु महालक्ष्मि त्राहि मां परमेश्वरि}

\twolineshloka*
{सर्वदा देहि मे द्रव्यं दानायापि च भुक्तये}
{धनधान्यं धरां हर्षं कीर्तिम् आयुश्च देहि मे}

\twolineshloka*
{यन्मया वाञ्छितं देवि तत्सर्वं सफलं कुरु}
{न बाधन्तां कुकर्माणि सङ्कटं मे निवारय}


\input{../stotra-sangrahah/stotras/lakshmi/KanakadharaStavam.tex}
\input{../stotra-sangrahah/stotras/lakshmi/Mahalakshmyashtakam.tex}

\sect{अपराध-क्षमापनम्}

\twolineshloka*
{न्यूनं वाऽप्यगुणं वाऽपि यन्मया मोहितं कृतम्}
{सर्वं तदस्तु सम्पूर्णं त्वत्प्रसादान्महेश्वरि}

\twolineshloka*
{लक्ष्मि त्वत्कृपया नित्यं कृता पूजा तवाऽऽज्ञया}
{स्थिरा भव गृहे ह्यस्मिन् मम सन्तानकर्मणि}


\fourlineindentedshloka*
{कायेन वाचा मनसेन्द्रियैर्वा}
{बुद्‌ध्याऽऽत्मना वा प्रकृतेः स्वभावात्}
{करोमि यद्यत् सकलं परस्मै}
{नारायणायेति समर्पयामि}

\centerline{सर्वं तत्सद्ब्रह्मार्पणमस्तु।}

\closesub

\sect{कथा}

\uvacha{सूत उवाच}

\twolineshloka
{कैलासशिखरे रम्ये सर्वदेवनिषेविते}
{गौर्या सह महादेवो दीव्यन्नक्षैवनोदतः}%॥१॥


\twolineshloka
{जितोऽसि त्वं मया चाऽऽह पार्वती परमेश्वरम्}
{सोऽपि त्वं च जितेत्याह सुविवादस्तयोरभूत्}%॥२॥


\twolineshloka
{चित्रनेमिस्तदा पृष्टो मृषावादमभाषत}
{तदा कोपसमाविष्टा गौरी शापं ददौ ततः}%॥३॥


\twolineshloka
{कुष्ठी भव मृषावादिन् चित्रनेमिर्हतप्रभः}
{नानृतेन समं पापं क्वाऽपि दृष्टं श्रुतं मया}%॥४॥


\twolineshloka
{चित्रनेमिर्महाप्राज्ञः सत्यं वदति नो मृषा}
{प्रसादः क्रियतां देवि देवीमाह वृषध्वजः}%॥५॥


\twolineshloka
{प्रसादसुमुखी तस्मै विशापं च जगाद सा}
{यदा सरोवरे रम्ये करिष्यन्ति शुचिव्रतम्}%॥६॥


\twolineshloka
{ततः स्वर्गणिकाः सर्वं यक्ष्यन्ति त्वां समाहिताः}
{तदा तव विशापः स्यादित्युक्तः स पपात ह}%॥७॥


\twolineshloka
{ततः कतिपयाहोभिश्चित्रनेमिः सरोवरे}
{कुष्ठी भूत्वा वसंस्तत्र ददर्श स्वर्विलासिनीः}%॥८॥


\twolineshloka
{देवतापूजनासक्ताः पप्रच्छ प्रणिपत्यताः}
{किमेतद्भो महाभागाः किं पूजा किं च वाञ्छितम्}%॥९॥


\twolineshloka
{किं मया च ह्यनुष्ठेयमिहामुत्र फलप्रदम्}
{इति व्रतं चित्रनेमिः पप्रच्छ स्वर्विलासिनीः}%॥१०॥


\twolineshloka
{येनाहे गिरिजाशापान्मोक्ष्यामि चिरदुःखतः}
{ता ऊचुः क्रियतामद्य त्वया चैतदनुत्तमम्}%॥११॥


\twolineshloka
{वरलक्ष्मीव्रतं दिव्यं सर्वकामसमृद्धिदम्}
{यदा रवौ कुलीरस्थे मासे च श्रावणे तथा}%॥१२॥


\twolineshloka
{गङ्गायमुनयोर्योगे तुङ्गभद्रासरित्तटे}
{तस्मिन्वै श्रावणे मासि शुक्लपक्षे भृगोर्दिने}%॥१३॥


\twolineshloka
{प्रारब्धव्यं व्रतं तत्र महालक्ष्म्या यतात्मभिः}
{सुवर्णप्रतिमां कुर्याच्चतुर्भुजसमन्विताम्}%॥१४॥


\twolineshloka
{पूर्व गृहमलङ्कृत्य तोरणै रङ्गवल्लिभिः}
{गृहस्य पूर्वदिग्भागे ईशान्यां च विशेषतः}%॥१५॥


\twolineshloka
{प्रस्थमितांस्तण्डुलांश्च भूमौ निक्षिप्य पद्मके}
{संस्थाप्य कलशं तत्र तीर्थतोयैः प्रपूरयेत्}%॥१६॥


\twolineshloka
{फलानि च विनिक्षिप्य सुवर्णं प्रक्षिपेत्ततः}
{पल्लवांश्च विनिक्षिप्य वस्त्रेणाच्छाद्य यत्नतः}%॥१७॥


\twolineshloka
{प्रतिमां स्थापयेत्तत्र पूजयेच्च यथाविधि}
{अग्न्युत्तारणपूर्वं तु शुद्धस्नानं यथाक्रमम्}%॥१८॥


\twolineshloka
{पञ्चामृतेन स्नपनं कारयेन्मन्त्रतः सुधीः}
{अभिषेकं ततः कृत्वा देवीसूक्तेन वै ततः}%॥१९॥


\twolineshloka
{अष्टगन्धैः समभ्यर्च्य पल्लवैश्च समर्चयेत्}
{अश्वत्थवटबिल्वाम्रमालतीदाडिमास्तथा}%॥२०॥


\twolineshloka
{एतेषां पत्राण्यादाय एकविंशतिसङ्स्ख्यया}
{नामाविधैस्तथा पुष्पैर्मालत्यादिसमुद्भवैः}%॥२१॥


\twolineshloka
{धूपदीपैर्महालक्ष्मीं पूजयेत् सर्वकामदाम्}
{पायसैर्भक्ष्यभोज्यैश्च नानाव्यञ्जनसंयुतैः}%॥२२॥


\twolineshloka
{एकविंशतिसङ्ख्याकैरपूपैः पूजयेच्छिवाम्}
{निवेद्य सर्वदेव्यै तु वरं स वृणुयात्ततः}%॥२३॥


\twolineshloka
{नृत्यगीतादिसहितो देवीं सम्प्रार्थयेच्छ्रियम्}
{रमां सरस्वतीं ध्यायेच्छचीं च प्रियवादिनीम्}%॥२४॥


\twolineshloka
{एवं व्रतविधिं तस्मै कथयित्वा विधानतः}
{पञ्चवायनकान् दत्त्वा कथां शृण्वीत यत्नतः}%॥२५॥


\twolineshloka
{तथा मौनं गृहीत्वा तु पञ्चार्तिक्येन पूजयेत्}
{व्रतं च कुर्वता गृह्य एकं पूगफलं तथा}%॥२६॥


\twolineshloka
{पर्णेकं चूर्णरहितं चर्वणीयं प्रयत्नतः}
{चैलखण्डे दृढं बद्ध्वा प्रातः पश्येद्विचक्षणः}%॥२७॥


\twolineshloka
{आरक्तं यदि जायेत कुर्याद्व्रतमनुतमम्}
{नोचेन्न तद्व्रतं कार्यं सर्वथा भूतिमिच्छता}%॥२८॥


\twolineshloka
{अनेनैव विधानेन व्रतं गृह्णीत यत्नतः}
{अप्सरोभिः कृतं सम्यग्व्रतं सर्वसमृद्धिदम्}%॥२९॥


\twolineshloka
{पूजावसानपर्यन्तं चित्रनेमिरलोकयत्}
{धूपधूमं समाघ्राय घृतदीपप्रभावतः}%॥३०॥


\twolineshloka
{गतकुष्ठः स्वर्णतेजाः शुचिस्तद्गतमानसः}
{अहं यत्नात् करिष्यामि व्रतं सर्वसमृद्धिदम्}%॥३१॥


\twolineshloka
{इत्युक्त्वा सर्वदेवीस्तु कारयामास तत्क्षणात्}
{सुवर्णनिर्मितां देवीं वस्त्रालङ्कारसंयुताम्}%॥३२॥


\twolineshloka
{पूर्वोक्तेन विधानेन पूजां कृत्वा प्रयत्नतः}
{ततो वैणवपात्राणि फलान्नैश्च सदक्षिणैः}%॥३३॥


\twolineshloka
{एकविंशतिपक्वान्नैः पूरितानि विधाय च}
{पञ्चवायनकान्येवं कृत्वादात्तु यथाक्रमम्}%॥३४॥


\twolineshloka
{विप्राय चाथ यतये देव्यै तु ब्रह्मचारिणे}
{सुवासिन्यै ततस्त्वेकमर्पितं चित्रनेमिना}%॥३५॥


\twolineshloka
{एवं सम्यक् क्रमेणैतद्दत्त्वा वायनपञ्चकम्}
{ततो गृहं गतः सोऽथ देवीं नत्वा यथाक्रमम्}%॥३६॥


\twolineshloka
{नागवल्लीदलं त्वेकं क्रमुकं चूर्णवजितम्}
{भक्षययित्वा तु चैलान्ते बद्ध्वा प्रातर्निरैक्षत}%॥३७॥


\twolineshloka
{आरक्ते च ततो जाते व्रतं चक्रे स भक्तितः}
{अद्याहं गतपापोऽस्मि देवीदर्शनयोगतः}%॥३८॥


\twolineshloka
{एतत्सम्यग्व्रतं चीर्णं भक्तिभावेन यन्मया}
{चित्रनेमिव्रतं कृत्वा कैलासं शङ्करालयम्}%॥३९॥


\twolineshloka
{गत्वा प्रणम्य देवेशं देवीमादरपूर्वकम्}
{पार्वती च तदा प्राह चित्रनेमे स्वपुत्रवत्}%॥४०॥


\twolineshloka
{पालनीयो मया त्वं च सत्यमित्यवधार्यताम्}
{चित्रनेमिस्तदा प्राह पार्वतीं हरवल्लभे}%॥४१॥


\twolineshloka
{तव पादाम्बुजं दृष्टं वरलक्ष्मीप्रसादतः}
{महादेवस्ततः प्राह चित्रनेमिं शुचिव्रतम्}%॥४२॥


\twolineshloka
{अद्यप्रभृति कैलासे भुङ्क्ष्व भोगान् यथेप्सितान्}
{पश्चाद्गन्तासि वैकुण्ठं वरस्यास्य प्रसादतः}%॥४३॥


\twolineshloka
{पार्वत्यापि कृतं पूर्वं पुत्रलाभार्थमेव च}
{लब्धश्च षण्मुखो देव्या व्रतराजप्रसादतः}%॥४४॥


\twolineshloka
{नन्दश्च विक्रमादित्यो राज्यं प्राप्तौ महाव्रतौ}
{नन्दश्च कान्तया हीनः कान्तां लेभे सुलक्षणाम्}%॥४५॥


\twolineshloka
{तया च तद् व्रतं कृत्स्नं कृतं वै पुत्रहेतवे}
{पुत्रं प्रसुषुवे सा च त्रैलोक्यभरणक्षमम्}%॥४६॥


\twolineshloka
{इह भुक्त्वा तु विपुलान्भोगान्वै सुमनोहरान्}
{तदाप्रभृति लोकेऽस्मिन् वरलक्ष्मीव्रतं शुभम्}%॥४७॥


\twolineshloka
{व्रतं करोति या नारी नरो वाऽपि शुचिव्रतः}
{भुक्त्वा भोगांश्च विपुलानन्ते शिवपुरं व्रजेत्}%॥४८॥


\twolineshloka
{इत्याख्यातं मया विप्रा वरलक्ष्मीव्रतं शुभम्}
{य इदं शृणुयान्नित्यं श्रावयेद्वा समाहितः}%॥४९।}

\onelineshloka
{धनं धान्यमवाप्नोति वरलक्ष्मीप्रसादतः}%॥५०॥

॥इति श्रीभविष्योत्तरपुराणे श्रावणशुक्रवारे वरलक्ष्मीव्रतं सम्पूर्णम्॥

\closesection
