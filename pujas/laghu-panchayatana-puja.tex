% !TeX program = XeLaTeX
% !TeX root = ../pujavidhanam.tex

\setlength{\parindent}{0pt}
\chapt{लघु-पञ्चायतन-पूजा}

% \sect{पूर्वाङ्गविघ्नेश्वरपूजा}

(आचम्य)
\twolineshloka*
{शुक्लाम्बरधरं विष्णुं शशिवर्णं चतुर्भुजम्}
{प्रसन्नवदनं ध्यायेत् सर्वविघ्नोपशान्तये}
 
प्राणान्  आयम्य।  ॐ भूः + भूर्भुवः॒ सुव॒रोम्।
 
(अप उपस्पृश्य, पुष्पाक्षतान् गृहीत्वा)\\
ममोपात्तसमस्त दुरितक्षयद्वारा \\
श्रीपरमेश्वरप्रीत्यर्थं करिष्यमाणस्य कर्मणः\\
 निर्विघ्नेन परिसमाप्त्यर्थम् आदौ विघ्नेश्वरपूजां करिष्ये।

\twolineshloka*
{ॐ ग॒णानां᳚ त्वा ग॒णप॑तिꣳ हवामहे क॒विं क॑वी॒नामु॑प॒मश्र॑वस्तमम्}
{ज्ये॒ष्ठ॒राजं॒ ब्रह्म॑णां ब्रह्मणस्पत॒ आ नः॑ शृ॒ण्वन्नू॒तिभिः॑ सीद॒ साद॑नम्}
अस्मिन् हरिद्राबिम्बे महागणपतिं ध्यायामि, आवाहयामि।\\


ॐ महागणपतये नमः  आसनं समर्पयामि।\\
पादयोः पाद्यं समर्पयामि। हस्तयोरर्घ्यं समर्पयामि।\\
आचमनीयं समर्पयामि।\\
ॐ भूर्भुवस्सुवः। शुद्धोदकस्नानं समर्पयामि।\\
स्नानानन्तरमाचमनीयं समर्पयामि।\\
वस्त्रार्थमक्षतान् समर्पयामि।\\
यज्ञोपवीताभरणार्थे अक्षतान् समर्पयामि।\\
दिव्यपरिमलगन्धान् धारयामि।\\
गन्धस्योपरि हरिद्राकुङ्कुमं समर्पयामि। अक्षतान् समर्पयामि। \\
पुष्पमालिकां समर्पयामि। पुष्पैः पूजयामि।

\dnsub{अर्चना}
% \setenumerate{label=\devanumber.}
% \renewcommand{\labelenumi}{\devanumber\theenumi.}
\begin{enumerate}%[label=\devanumber\value{enumi}]
\begin{minipage}{0.475\linewidth}   
\item ॐ सुमुखाय नमः
\item ॐ एकदन्ताय नमः
\item ॐ कपिलाय नमः
\item ॐ गजकर्णकाय नमः
\item ॐ लम्बोदराय नमः
\item ॐ विकटाय नमः
\item ॐ विघ्नराजाय नमः
\item ॐ विनायकाय नमः
\item ॐ धूमकेतवे नमः
  \end{minipage}
  \begin{minipage}{0.525\linewidth}
\item ॐ गणाध्यक्षाय नमः
\item ॐ फालचन्द्राय नमः
\item ॐ गजाननाय नमः
\item ॐ वक्रतुण्डाय नमः
\item ॐ शूर्पकर्णाय नमः
\item ॐ हेरम्बाय नमः
\item ॐ स्कन्दपूर्वजाय नमः
\item ॐ सिद्धिविनायकाय नमः
\item ॐ विघ्नेश्वराय नमः
  \end{minipage}
\end{enumerate}
नानाविधपरिमलपत्रपुष्पाणि समर्पयामि॥\\
धूपमाघ्रापयामि।\\
अलङ्कारदीपं सन्दर्शयामि।\\
नैवेद्यम्।\\
ताम्बूलं समर्पयामि।\\
कर्पूरनीराजनं समर्पयामि।\\
कर्पूरनीराजनानन्तरमाचमनीयं समर्पयामि।\\
{वक्रतुण्डमहाकाय कोटिसूर्यसमप्रभ।}\\
{अविघ्नं कुरु मे देव सर्वकार्येषु सर्वदा॥}\\
प्रार्थनाः समर्पयामि।

अनन्तकोटिप्रदक्षिणनमस्कारान् समर्पयामि।\\
छत्त्रचामरादिसमस्तोपचारान् समर्पयामि।\\


\sect{प्रधान-पूजा — पञ्चायतनपूजा}

\twolineshloka*
{शुक्लाम्बरधरं विष्णुं शशिवर्णं चतुर्भुजम्}
{प्रसन्नवदनं ध्यायेत् सर्वविघ्नोपशान्तये}

प्राणान् आयम्य। ॐ भूः + भूर्भुवस्सुवरोम्।

\dnsub{सङ्कल्पः}

ममोपात्त-समस्त-दुरित-क्षयद्वारा श्री-परमेश्वर-प्रीत्यर्थं शुभे शोभने मुहूर्ते अद्य ब्रह्मणः
द्वितीयपरार्धे श्वेतवराहकल्पे वैवस्वतमन्वन्तरे अष्टाविंशतितमे कलियुगे प्रथमे पादे
जम्बूद्वीपे भारतवर्षे भरतखण्डे मेरोः दक्षिणे पार्श्वे शकाब्दे अस्मिन् वर्तमाने व्यावहारिके
प्रभवादि षष्टिसंवत्सराणां मध्ये (	)\see{app:samvatsara_names} नाम संवत्सरे उत्तरायणे / दक्षिणायने 
(ग्रीष्म / वर्ष / शरद् / हेमन्त / शिशिर / वसन्त) ऋतौ (मेष / वृषभ / मिथुन / कर्कटक / सिंह / कन्या / तुला / 
वृश्चिक / धनुर् / मकर / कुम्भ / मीन) मासे (शुक्ल / कृष्ण) पक्षे (एकादश्यां / द्वादश्यां) शुभतिथौ
(इन्दु / भौम / बुध / गुरु / भृगु / स्थिर / भानु) वासरयुक्तायाम्
\mbox{(~~~)}\see{app:nakshatra_names} नक्षत्र \mbox{(~~~)}\see{app:yoga_names} नाम योग \mbox{(~~~)} करण युक्तायां च एवं गुण विशेषण विशिष्टायाम्
अस्याम् \mbox{(~~~)} शुभतिथौ 
अस्माकं सहकुटुम्बानां क्षेमस्थैर्य-धैर्य-वीर्य-विजय-आयुरारोग्य-ऐश्वर्याभिवृद्ध्यर्थम्
धर्मार्थकाममोक्ष\-चतुर्विधफलपुरुषार्थसिद्ध्यर्थं पुत्रपौत्राभि\-वृद्ध्यर्थम् इष्टकाम्यार्थसिद्ध्यर्थम्
मम इहजन्मनि पूर्वजन्मनि जन्मान्तरे च सम्पादितानां ज्ञानाज्ञानकृतमहा\-पातकचतुष्टय-व्यतिरिक्तानां रहस्यकृतानां प्रकाशकृतानां सर्वेषां पापानां सद्य अपनोदनद्वारा सकल पापक्षयार्थं
श्री-महागणपति-प्रीत्यर्थं श्री-छाया-सुवर्चलाम्बा-समेत-श्री-सूर्यनारायण-प्रीत्यर्थं श्री-लक्ष्मीनारायण-प्रीत्यर्थं 
श्री-महालक्ष्मीसमेतं श्री-सन्तानगोपाल-प्रीत्यर्थं श्री-गौरीदेवी-प्रीत्यर्थं श्री-साम्बपरमेश्वर-प्रीत्यर्थं श्री-नन्दिकेश्वर-प्रीत्यर्थं 
श्री-वल्लीदेवसेनासमेत-श्री-सुब्रह्मण्य-प्रीत्यर्थं श्री-सीता-लक्ष्मण-भरत-शत्रुघ्न-हनूमत्-समेत-श्री-रामचन्द्र-प्रीत्यर्थं 
यावच्छक्ति ध्यानावाहनादि षोडशोपचार-पूजां पञ्चायतनपूजां क्षीराभिषेकं च करिष्ये। तदङ्गं कलशपूजां च करिष्ये।


श्रीविघ्नेश्वराय नमः यथास्थानं प्रतिष्ठापयामि।\\
(गणपति-प्रसादं शिरसा गृहीत्वा)

\dnsub{आसन-पूजा}
\centerline{पृथिव्या  मेरुपृष्ठ  ऋषिः।  सुतलं  छन्दः।  कूर्मो  देवता॥}
\twolineshloka*
{पृथ्वि  त्वया  धृता  लोका  देवि  त्वं  विष्णुना  धृता}
{त्वं  च  धारय  मां  देवि  पवित्रं  चाऽऽसनं  कुरु}


\dnsub{घण्टापूजा}
\twolineshloka*
{आगमार्थं तु देवानां गमनार्थं तु रक्षसाम्}
{घण्टारवं करोम्यादौ देवताऽऽह्वानकारणम्}


\dnsub{कलशपूजा}
ॐ कलशाय नमः दिव्यगन्धान् धारयामि।\\
ॐ गङ्गायै नमः। ॐ यमुनायै नमः। ॐ गोदावर्यै नमः।  ॐ सरस्वत्यै नमः। ॐ नर्मदायै नमः। ॐ सिन्धवे नमः। ॐ कावेर्यै नमः।\\
ॐ सप्तकोटिमहातीर्थान्यावाहयामि।\\[-0.25ex]

(अथ कलशं स्पृष्ट्वा जपं कुर्यात्) \\
आपो॒ वा इ॒द सर्वं॒ विश्वा॑ भू॒तान्याप॑ प्रा॒णा वा आप॑ प॒शव॒ आपो\-ऽन्न॒मापोऽमृ॑त॒माप॑ स॒म्राडापो॑ वि॒राडाप॑ स्व॒राडाप॒श्\-छन्दा॒स्यापो॒ ज्योती॒ष्यापो॒ यजू॒ष्याप॑ स॒त्यमाप॒ सर्वा॑ दे॒वता॒ आपो॒ भूर्भुव॒ सुव॒राप॒ ओम्॥\\

\twolineshloka* 
{कलशस्य मुखे विष्णुः कण्ठे रुद्रः समाश्रितः}
{मूले तत्र स्थितो ब्रह्मा मध्ये मातृगणाः स्मृताः}
\threelineshloka* 
{कुक्षौ तु सागराः सर्वे सप्तद्वीपा वसुन्धरा}
{ऋग्वेदोऽथ यजुर्वेदः सामवेदोऽप्यथर्वणः}
{अङ्गैश्च सहिताः सर्वे कलशाम्बुसमाश्रिताः}
\twolineshloka* 
{गङ्गे च यमुने चैव गोदावरि सरस्वति}
{नर्मदे सिन्धुकावेरि जलेऽस्मिन् सन्निधिं कुरु}
\twolineshloka*
{सर्वे समुद्राः सरितः तीर्थानि च ह्रदा नदाः}
{आयान्तु देवपूजार्थं दुरितक्षयकारकाः}

\centerline{ॐ भूर्भुवः॒ सुवो॒ भूर्भुवः॒ सुवो॒ भूर्भुवः॒ सुवः॑।}

(इति कलशजलेन सर्वोपकरणानि आत्मानं च प्रोक्ष्य।)


\dnsub{आत्म-पूजा}
ॐ आत्मने नमः, दिव्यगन्धान् धारयामि।
\begin{multicols}{2}
१. ॐ आत्मने नमः\\
२. ॐ अन्तरात्मने नमः\\
३. ॐ योगात्मने नमः\\
४. ॐ जीवात्मने नमः\\
५. ॐ परमात्मने नमः\\
६. ॐ ज्ञानात्मने नमः
\end{multicols}
समस्तोपचारान् समर्पयामि।

\twolineshloka*
{देहो देवालयः प्रोक्तो जीवो देवः सनातनः}
{त्यजेदज्ञाननिर्माल्यं सोऽहं भावेन पूजयेत्}


\begin{minipage}{\linewidth}
\dnsub{पीठ-पूजा}

\begin{multicols}{2}
\begin{enumerate}
\item ॐ आधारशक्त्यै नमः
\item ॐ मूलप्रकृत्यै नमः
\item ॐ आदिकूर्माय नमः 
\item ॐ आदिवराहाय नमः
\item ॐ अनन्ताय नमः
\item ॐ पृथिव्यै नमः
\item ॐ रत्नमण्डपाय नमः
\item ॐ रत्नवेदिकायै नमः
\item ॐ स्वर्णस्तम्भाय नमः
\item ॐ श्वेतच्छत्त्राय नमः
\item ॐ कल्पकवृक्षाय नमः
\item ॐ क्षीरसमुद्राय नमः 
\item ॐ सितचामराभ्यां नमः
\item ॐ योगपीठासनाय नमः
\end{enumerate}
\end{multicols}

\end{minipage}

\dnsub{गुरु ध्यानम्}

\twolineshloka*
{गुरुर्ब्रह्मा गुरुर्विष्णुर्गुरुर्देवो महेश्वरः}
{गुरुः साक्षात् परं ब्रह्म तस्मै श्री गुरवे नमः}


\begin{center}

\sect{षोडशोपचार-पूजा}

ॐ ग॒णानां᳚ त्वा ग॒णप॑तिꣳ हवामहे क॒विं क॑वी॒नामु॑प॒मश्र॑\-वस्तमम्। 
ज्ये॒ष्ठ॒राजं॒ ब्रह्म॑णां ब्रह्मणस्पत॒ आ नः॑ शृ॒ण्वन्नू॒तिभिः॑ सीद॒ साद॑नम्॥ 
ॐ महागणपतये॒ नमः॥ 

अस्मिन् बिम्बे श्री-महागणपतिं ध्यायामि। आवाहयामि॥

आ स॒त्येन॒ रज॑सा॒ वर्त॑मानो निवे॒शय॑न्न॒मृतं॒ मर्त्यं॑ च। हि॒र॒ण्यये॑न सवि॒ता रथे॒नाऽदे॒वो या॑ति॒ भुव॑ना वि॒पश्यन्॑।

अस्मिन् बिम्बे श्री-छाया-सुवर्चलाम्बा-समेत-श्री-सूर्यनारायणं ध्यायामि। आवाहयामि॥


स॒हस्र॑शीर्‌षा॒ पुरु॑षः। स॒ह॒स्रा॒क्षः स॒हस्र॑पात्।\\
स भूमिं॑ वि॒श्वतो॑ वृ॒त्वा। अत्य॑तिष्ठद्दशाङ्गु॒लम्॥

हिर॑ण्यवर्णां॒ हरि॑णीं सुव॒र्णर॑जत॒स्रजाम्।\\
च॒न्द्रां॒ हि॒रण्म॑यीं ल॒क्ष्मीं॒ जात॑वेदो म॒ आव॑ह॥

अस्मिन् बिम्बे श्री-लक्ष्मीनारायणं ध्यायामि।
अस्मिन् बिम्बे श्री-महालक्ष्मीसमेतं श्री-सन्तानगोपालं ध्यायामि।

त्र्य॑म्बकं यजामहे सुग॒न्धिं पु॑ष्टि॒वर्ध॑नम्।\\
उ॒र्वा॒रु॒कमि॑व॒ बन्ध॑नान्मृ॒त्योर्मु॑क्षीय॒ माऽमृता᳚त्॥

गौ॒री मि॑माय सलि॒लानि॒ तक्ष॒ती। एक॑पदी द्वि॒पदी॒ सा चतु॑ष्पदी।\\
अ॒ष्टाप॑दी॒ नव॑पदी बभू॒वुषी। स॒हस्रा᳚क्षरा पर॒मे व्यो॑मन्।

अस्मिन् बिम्बे श्री-सपरिवार-साम्बपरमेश्वरं ध्यायामि। आवाहयामि॥\\
अस्मिन् बिम्बे श्री-गौरीदेवीं ध्यायामि। आवाहयामि॥\\
अस्मिन् बिम्बे श्री-साम्बपरमेश्वरं ध्यायामि। आवाहयामि॥

तत्पुरु॑षाय वि॒द्महे॑ चक्रतु॒ण्डाय॑ धीमहि। तन्नो॑ नन्दिः प्रचो॒दया᳚त्। 
अस्मिन् बिम्बे श्री-नन्दिकेश्वरं ध्यायामि। आवाहयामि॥\\

तत्पुरु॑षाय वि॒द्महे॑ महासे॒नाय॑ धीमहि। तन्नः॑ षण्मुखः प्रचो॒दया᳚त्॥

अस्मिन् बिम्बे श्री-वल्लीदेवसेनासमेत-श्री-सुब्रह्मण्यस्वामिनं ध्यायामि। 

\fourlineindentedshloka*
{कल्पद्रुमं प्रणमतां कमलारुणभम्}
{स्कन्दं भुजद्वयमनामयमेकवक्त्रम्}
{कात्यायनी-प्रियसुतं कटिबद्धवामम्}
{कौपीन-दण्डधर-दक्षिणहस्तमीडे}

आवाहयामि॥\\

\fourlineindentedshloka*
{वैदेहीसहितं सुरद्रुमतले हैमे महामण्डपे}
{मध्ये पुष्पकमासने मणिमये वीरासने सुस्थितम्}
{अग्रे वाचयति प्रभञ्जनसुते तत्त्वं मुनिभ्यः परम्}
{व्याख्यान्तं भरतादिभिः परिवृतं रामं भजे श्यामलम्}
अस्मिन् बिम्बे श्री-सीता-लक्ष्मण-भरत-शत्रुघ्न-हनूमत्-समेत-श्री-रामचन्द्रं ध्यायामि। 

\fourlineindentedshloka*
{वामे भूमिसुता पुरश्च हनुमान् पश्चात् सुमित्रासुतः}
{शत्रुघ्नो भरतश्च पार्श्वदलयोर्वाय्वादिकोणेषु च}
{सुग्रीवश्च विभीषणश्च युवराट् तारासुतो जाम्बवान्}
{मध्ये नीलसरोजकोमलरुचिं रामं भजे श्यामलम्}
आवाहयामि॥\\\medskip

आवाहिताभ्यः सर्वाभ्यो-देवताभ्यो नमः।

आसनं समर्पयामि।

पादयोः पाद्यं समर्पयामि।

अर्घ्यं समर्पयामि।

आचमनीयं समर्पयामि।

शुद्धोदकस्नानं समर्पयामि। स्नानानन्तरम् आचमनीयं समर्पयामि।


\dnsub{अभिषेकः}

ॐ ग॒णानां᳚ त्वा ग॒णप॑तिꣳ हवामहे क॒विं क॑वी॒नामु॑प॒मश्र॑\-वस्तमम्। 
ज्ये॒ष्ठ॒राजं॒ ब्रह्म॑णां ब्रह्मणस्पत॒ आ नः॑ शृ॒ण्वन्नू॒तिभिः॑ सीद॒ साद॑नम्॥ 
ॐ महागणपतये॒ नमः॥ 

ॐ आ स॒त्येन॒ रज॑सा॒ वर्त॑मानो निवे॒शय॑न्न॒मृतं॒ मर्त्यं॑ च। हि॒र॒ण्यये॑न सवि॒ता रथे॒नाऽदे॒वो या॑ति॒ भुव॑ना वि॒पश्यन्॑।

ॐ स॒हस्र॑शीर्‌षा॒ पुरु॑षः। स॒ह॒स्रा॒क्षः स॒हस्र॑पात्।\\
स भूमिं॑ वि॒श्वतो॑ वृ॒त्वा। अत्य॑तिष्ठद्दशाङ्गु॒लम्॥

ॐ हिर॑ण्यवर्णां॒ हरि॑णीं सुव॒र्णर॑जत॒स्रजाम्।\\
च॒न्द्रां॒ हि॒रण्म॑यीं ल॒क्ष्मीं॒ जात॑वेदो म॒ आव॑ह॥

गौ॒री मि॑माय सलि॒लानि॒ तक्ष॒ती। एक॑पदी द्वि॒पदी॒ सा चतु॑ष्पदी।\\
अ॒ष्टाप॑दी॒ नव॑पदी बभू॒वुषी। स॒हस्रा᳚क्षरा पर॒मे व्यो॑मन्।

तत्पुरु॑षाय वि॒द्महे॑ चक्रतु॒ण्डाय॑ धीमहि। तन्नो॑ नन्दिः प्रचो॒दया᳚त्। 

तत्पुरु॑षाय वि॒द्महे॑ महासे॒नाय॑ धीमहि। तन्नः॑ षण्मुखः प्रचो॒दया᳚त्॥


ॐ नमो भगवते॑ रुद्रा॒य॥\\
रुद्रप्रश्न-चमकप्रश्न-पुरुषषुक्तैः अभिषेकम् कृत्वा।

अभिषेकानन्तरं आचमनीयं समर्पयामि। वस्त्रार्थम् अक्षतान् समर्पयामि। यज्ञोपवीताभरणार्थे अक्षतान् समर्पयामि।

दिव्यपरिमलगन्धान् धारयामि। गन्धस्योपरि हरिद्राकुङ्कुमं समर्पयामि। अक्षतान् समर्पयामि।

पुष्पमालिकां समर्पयामि। पुष्पैः पूजयामि।

\clearpage
\dnsub{महागणपति-अर्चना}

\begin{enumerate}%[label=\devanumber\value{enumi}]
\begin{minipage}{0.475\linewidth} 
\item ॐ सुमुखाय नमः
\item ॐ एकदन्ताय नमः
\item ॐ कपिलाय नमः
\item ॐ गजकर्णकाय नमः
\item ॐ लम्बोदराय नमः
\item ॐ विकटाय नमः
\item ॐ विघ्नराजाय नमः
\item ॐ विनायकाय नमः
\item ॐ धूमकेतवे नमः
\end{minipage}
\begin{minipage}{0.525\linewidth}
\item ॐ गणाध्यक्षाय नमः
\item ॐ फालचन्द्राय नमः
\item ॐ गजाननाय नमः
\item ॐ वक्रतुण्डाय नमः
\item ॐ शूर्पकर्णाय नमः
\item ॐ हेरम्बाय नमः
\item ॐ स्कन्दपूर्वजाय नमः
\item ॐ सिद्धिविनायकाय नमः
\item ॐ विघ्नेश्वराय नमः
\end{minipage}
\end{enumerate}

ॐ श्री-महागणपतये नमः नानाविधपरिमलपत्रपुष्पाणि समर्पयामि। \medskip

\dnsub{आदित्य-अर्चना}
\begin{multicols}{2}
\begin{enumerate}
\item ॐ मित्राय नमः
\item ॐ रवये नमः
\item ॐ सूर्याय नमः
\item ॐ भानवे नमः
\item ॐ खगाय नमः
\item ॐ पूष्णे नमः
\item ॐ हिरण्यगर्भाय नमः
\item ॐ मरीचये नमः
\item ॐ आदित्याय नमः
\item ॐ सवित्रे नमः
\item ॐ अर्काय नमः
\item ॐ भास्कराय नमः

\end{enumerate}
\end{multicols}
ॐ श्री-छाया-सुवर्चलाम्बा-समेत-श्री-सूर्यनारायण-परब्रह्मणे नमः नानाविधपरिमलपत्रपुष्पाणि समर्पयामि। \medskip

\dnsub{लक्ष्मीनारायण-अर्चना}

\begin{multicols}{2}
\begin{enumerate}
\item ॐ केशवाय नमः
\item ॐ नारायणाय नमः
\item ॐ माधवाय नमः
\item ॐ गोविन्दाय नमः
\item ॐ विष्णवे नमः	
\item ॐ मधुसूदनाय नमः
\item ॐ त्रिविक्रमाय नमः
\item ॐ वामनाय नमः
\item ॐ श्रीधराय नमः
\item ॐ हृषीकेशाय नमः
\item ॐ पद्मनाभाय नमः
\item ॐ दामोदराय नमः
\item ॐ सङ्कर्षणाय नमः
\item ॐ वासुदेवाय नमः
\item ॐ प्रद्युम्नाय नमः
\item ॐ अनिरुद्धाय नमः
\item ॐ पुरुषोत्तमाय नमः
\item ॐ अधोक्षजाय नमः
\item ॐ नृसिंहाय नमः
\item ॐ अच्युताय नमः
\item ॐ जनार्दनाय नमः
\item ॐ उपेन्द्राय नमः 
\item ॐ हरये नमः
\item ॐ श्रीकृष्णाय नमः
\end{enumerate}
\end{multicols}


\begin{multicols}{2}
\begin{enumerate}
\item ॐ आदिलक्ष्म्यै नमः
\item ॐ धान्यलक्ष्म्यै नमः
\item ॐ धैर्यलक्ष्म्यै नमः
\item ॐ गजलक्ष्म्यै नमः
\item ॐ सन्तानलक्ष्म्यै नमः
\item ॐ विजयलक्ष्म्यै नमः
\item ॐ विद्यालक्ष्म्यै नमः
\item ॐ धनलक्ष्म्यै नमः
\item ॐ वरलक्ष्म्यै नमः
\item ॐ महालक्ष्म्यै नमः 
\end{enumerate}
\end{multicols}

ॐ श्री-लक्ष्मीनारायणाय नमः नानाविधपरिमलपत्रपुष्पाणि समर्पयामि। \medskip

\clearpage
\dnsub{महालक्ष्मी-समेत-श्री-सन्तानगोपाल-अर्चना}

\begin{multicols}{2}
\begin{enumerate}
\item ॐ केशवाय नमः
\item ॐ नारायणाय नमः
\item ॐ माधवाय नमः
\item ॐ गोविन्दाय नमः
\item ॐ विष्णवे नमः 
\item ॐ मधुसूदनाय नमः
\item ॐ त्रिविक्रमाय नमः
\item ॐ वामनाय नमः
\item ॐ श्रीधराय नमः
\item ॐ हृषीकेशाय नमः
\item ॐ पद्मनाभाय नमः
\item ॐ दामोदराय नमः
\item ॐ सङ्कर्षणाय नमः
\item ॐ वासुदेवाय नमः
\item ॐ प्रद्युम्नाय नमः
\item ॐ अनिरुद्धाय नमः
\item ॐ पुरुषोत्तमाय नमः
\item ॐ अधोक्षजाय नमः
\item ॐ नृसिंहाय नमः
\item ॐ अच्युताय नमः
\item ॐ जनार्दनाय नमः
\item ॐ उपेन्द्राय नमः 
\item ॐ हरये नमः
\item ॐ श्रीकृष्णाय नमः
\end{enumerate}
\end{multicols}


\begin{multicols}{2}
\begin{enumerate}
\item ॐ आदिलक्ष्म्यै नमः
\item ॐ धान्यलक्ष्म्यै नमः
\item ॐ धैर्यलक्ष्म्यै नमः
\item ॐ गजलक्ष्म्यै नमः
\item ॐ सन्तानलक्ष्म्यै नमः
\item ॐ विजयलक्ष्म्यै नमः
\item ॐ विद्यालक्ष्म्यै नमः
\item ॐ धनलक्ष्म्यै नमः
\item ॐ वरलक्ष्म्यै नमः
\item ॐ महालक्ष्म्यै नमः 
\end{enumerate}
\end{multicols}

ॐ श्री-महालक्ष्मीसमेत-श्री-सन्तानगोपाल-स्वामिने नमः नानाविधपरिमलपत्रपुष्पाणि समर्पयामि। \medskip

\clearpage
\dnsub{श्री-साम्बपरमेश्वर-अर्चना}

\vspace{-1em}
\begin{multicols}{2}
\begin{enumerate}
\item ॐ भवाय देवाय नमः
\item ॐ शर्वाय देवाय नमः
\item ॐ ईशानाय देवाय नमः
\item ॐ पशुपतये देवाय नमः
\item ॐ रुद्राय देवाय नमः
\item ॐ उग्राय देवाय नमः
\item ॐ भीमाय देवाय नमः
\item ॐ महते देवाय नमः
\end{enumerate}
\end{multicols}
\vspace{-1em}
ॐ श्री-साम्बपरमेश्वराय नमः नानाविधपरिमलपत्रपुष्पाणि समर्पयामि। \\
ॐ नन्दिकेश्वराय नमः।

\dnsub{गौरी-अर्चना}
\begin{enumerate}
\item ॐ भवस्य देवस्य पत्न्यै नमः
\item ॐ शर्वस्य देवस्य पत्न्यै नमः
\item ॐ ईशानस्य देवस्य पत्न्यै नमः
\item ॐ पशुपतेर्देवस्य पत्न्यै नमः
\item ॐ रुद्रस्य देवस्य पत्न्यै नमः
\item ॐ उग्रस्य देवस्य पत्न्यै नमः
\item ॐ भीमस्य देवस्य पत्न्यै नमः
\item ॐ महतो देवस्य पत्न्यै नमः
\end{enumerate}

ॐ श्री-गौरी-देव्यै नमः नानाविधपरिमलपत्रपुष्पाणि समर्पयामि। \medskip

\dnsub{श्री-वल्लीदेवसेनासमेत-सुब्रह्मण्यस्वामी-अर्चना}
\vspace{-1em}
\begin{multicols}{2}
\begin{enumerate}
\item ॐ ज्ञानशक्त्यात्मने नमः
\item ॐ स्कन्दाय नमः
\item ॐ अग्निभुवे नमः
\item ॐ बाहुलेयाय नमः
\item ॐ गाङ्गेयाय नमः
\item ॐ शरवणोद्भवाय नमः
\item ॐ कार्त्तिकेयाय नमः
\item ॐ कुमाराय नमः
\item ॐ षण्मुखाय नमः
\item ॐ कुक्कुटध्वजाय नमः
\item ॐ शक्तिधराय नमः
\item ॐ गुहाय नमः
\item ॐ ब्रह्मचारिणे नमः
\item ॐ षण्मातुराय नमः
\item ॐ क्रौञ्चभित्रे नमः
\item ॐ शिखिवाहनाय नमः
\end{enumerate}
\end{multicols}
\vspace{-1em}
ॐ श्री-वल्लीदेवसेनासमेत-श्री-सुब्रह्मण्यस्वामिने नमः नानाविधपरिमलपत्रपुष्पाणि समर्पयामि। \medskip


\dnsub{श्री-राम-अर्चना}
\begin{multicols}{2}
\begin{enumerate}
\item ॐ श्रीरामाय नमः
\item ॐ रामभद्राय नमः
\item ॐ रामचन्द्राय नमः
\item ॐ शाश्वताय नमः
\item ॐ राजीवलोचनाय नमः
\item ॐ श्रीमते नमः
\item ॐ राजेन्द्राय नमः
\item ॐ रघुपुङ्गवाय नमः
\item ॐ जानकीवल्लभाय नमः
\item ॐ जैत्राय नमः
\item ॐ जितामित्राय नमः
\item ॐ जनार्दनाय नमः
\item ॐ परमात्मने नमः
\item ॐ परस्मै ब्रह्मणे नमः
\item ॐ सच्चिदानन्दविग्रहाय
\item ॐ परस्मै ज्योतिषे नमः
\item ॐ परस्मै धाम्ने नमः
\item ॐ पराकाशाय नमः
\item ॐ परात्पराय नमः
\item ॐ परेशाय नमः
\item ॐ पारगाय नमः
\item ॐ पाराय नमः
\item ॐ सर्वदेवात्मकाय नमः
\item ॐ पराय नमः
\end{enumerate}
\end{multicols}

\dnsub{श्री-सीता-अर्चना}

\begin{multicols}{2}
\begin{enumerate}
\item ॐ श्रीसीतायै नमः
\item ॐ जानक्यै नमः
\item ॐ देव्यै नमः
\item ॐ वैदेह्यै नमः
\item ॐ राघवप्रियायै नमः
\item ॐ रमायै नमः
\item ॐ अवनिसुतायै नमः
\item ॐ रामायै नमः
\item ॐ राक्षसान्तप्रकारिण्यै
\item ॐ रत्नगुप्तायै नमः
\item ॐ मातुलुङ्ग्यै नमः
\item ॐ मैथिल्यै नमः
\end{enumerate}
\end{multicols}

\dnsub{श्री-हनूमद्-अर्चना}

\begin{enumerate}
\item ॐ हनुमते नमः
\item ॐ अञ्जनासूनवे नमः
\item ॐ वायुपुत्राय नमः
\item ॐ महाबलाय नमः
\item ॐ कपीन्द्राय नमः
\item ॐ पिङ्गलाक्षाय नमः
\item ॐ लङ्काद्वीपभयङ्कराय नमः
\item ॐ प्रभञ्जनसुताय नमः
\item ॐ वीराय नमः
\item ॐ सीताशोकविनाशकाय नमः
\item ॐ अक्षहन्त्रे नमः
\item ॐ रामसखाय नमः
\item ॐ रामकार्यधुरन्धराय नमः
\item ॐ महौषधगिरेर्धारिणे नमः
\item ॐ वानरप्राणदायकाय नमः
\item ॐ वारीशतारकाय नमः
\item ॐ मैनाकगिरिभञ्जनाय नमः
\item ॐ निरञ्जनाय नमः
\item ॐ जितक्रोधाय नमः
\item ॐ कदलीवनसंवृताय नमः
\item ॐ ऊर्ध्वरेतसे नमः
\item ॐ महासत्त्वाय नमः
\item ॐ सर्वमन्त्रप्रवर्तकाय नमः
\item ॐ महालिङ्गप्रतिष्ठात्रे नमः
\item ॐ बाष्पकृत् जपतान्तराय नमः
\item ॐ नित्यं शिवध्यानपराय नमः
\item ॐ शिवपूजापरायणाय नमः
\end{enumerate}

ॐ श्री-सीता-लक्ष्मण-भरत-शत्रुघ्न-हनूमत्-समेत-श्री-रामचन्द्र-परब्रह्मणे नमः नानाविधपरिमलपत्रपुष्पाणि समर्पयामि। \medskip

\newcommand{\swamine}{श्री~महागणपतये~नमः~श्री~छाया-सुवर्चलाम्बा-समेत\-श्री~सूर्यनारायण-परब्रह्मणे~नमः लक्ष्मी-नारायणाय नमः श्री-महालक्ष्मी-समेत-श्री-सन्तानगोपाल-स्वामिने नमः श्री-साम्बपरमेश्वराय नमः नन्दिकेश्वराय नमः श्री~वल्लीदेवसेनासमेत-श्री-सुब्रह्मण्यस्वामिने नमः श्री~सीता-लक्ष्मण-भरत-शत्रुघ्न-हनूमत्-समेत-श्री-रामचन्द्र परब्रह्मणे नमः}

\sect{उत्तराङ्ग-पूजा}
धूपमाघ्रापयामि।\medskip

पञ्च॑हूतो ह॒ वै नामै॒षः। तं वा ए॒तं पञ्च॑हूत॒ꣳ॒ सन्तम्।\\
पञ्च॑हो॒तेत्याच॑क्षते प॒रोक्षे॑ण। प॒रोक्ष॑प्रिया इव॒ हि दे॒वाः॥

पञ्चहारतीदीपं दर्शयामि। दीपानन्तरम् आचमनीयं समर्पयामि।\medskip

गायत्रीदीपं दर्शयामि। दीपानन्तरम् आचमनीयं समर्पयामि।\medskip

उद्दी᳚प्यस्व जातवेदोऽप॒घ्नन्निर्ऋ॑तिं॒ मम॑।\\
प॒शूꣴश्च॒ मह्य॒माव॑ह॒ जीव॑नं च॒ दिशो॑ दिश॥ \\
मा नो॑ हिꣳसीज्जातवेदो॒ गामश्वं॒ पुरु॑षं॒ जग॑त्।\\
अबि॑भ्र॒दग्न॒ आग॑हि श्रि॒या मा॒ परि॑पातय॥ \\
एकहारतीदीपं दर्शयामि।\\
दीपानन्तरम् आचमनीयं समर्पयामि।\medskip

नैवेद्यं कृत्वा।\\
\swamine{} \mbox{(~~~)} महानैवेद्यं निवेदयामि।\\
मध्ये मध्ये अमृतपानीयं समर्पयामि। अमृतापिधानमसि।\\
हस्तप्रक्षालनं समर्पयामि। पादप्रक्षालनं समर्पयामि।\\
निवेदनानन्तरम् आचमनीयं समर्पयामि।\medskip

\twolineshloka* 
{पूगीफलसमायुक्तं नागवल्लीदलैर्युतम्}
{कर्पूरचूर्णसंयुक्तं ताम्बूलं प्रतिगृह्यताम्}

कर्पूरताम्बूलं समर्पयामि।\medskip

सोमो॒ वा ए॒तस्य॑ रा॒ज्यमाद॑त्ते।
यो राजा॒ सन्रा॒ज्यो वा॒ सोमे॑न॒ यज॑ते।
दे॒व॒सु॒वामे॒तानि॑ ह॒वीꣳषि॑ भवन्ति।
ए॒ताव॑न्तो॒ वै दे॒वाना स॒वाः।
त ए॒वास्मै॑ स॒वान्प्रय॑च्छन्ति।
त ए॑नं॒ पुनः॑ सुवन्ते रा॒ज्याय॑।
दे॒व॒सू राजा॑ भवति॥

न तत्र सूर्यो भाति न चन्द्रतारकं नेमा विद्युतो भान्ति कुतोऽयमग्निः।
तमेव भान्तमनुभाति सर्वं तस्य भासा सर्वमिदं विभाति॥


\swamine{} समस्त-अपराध-क्षमापनार्थं कर्पूरनीराजनं दर्शयामि। कर्पूरनीरजनानन्तरम् आचमनीयं समर्पयामि।\medskip


यो॑ऽपां पुष्पं॒ वेद॑। पुष्प॑वान् प्र॒जावा᳚न् पशु॒मान् भ॑वति।\\
च॒न्द्रमा॒ वा अ॒पां पुष्पम्᳚। पुष्प॑वान् प्र॒जावा᳚न् पशु॒मान् भ॑वति।\\
य ए॒वं वेद॑। यो॑ऽपामा॒यत॑नं॒ वेद॑। आ॒यत॑नवान् भवति।\medskip

ओं᳚ तद्ब्र॒ह्म। ओं᳚ तद्वा॒युः। ओं᳚ तदा॒त्मा।\\ ओं᳚ तथ्स॒त्यम्‌।
ओं᳚ तथ्सर्वम्᳚‌। ओं᳚ तत्पुरो॒र्नमः॥\medskip

अन्तश्चरति॑ भूते॒षु॒ गुहायां वि॑श्वमू॒र्तिषु। \\
त्वं यज्ञस्त्वं वषट्कारस्त्वमिन्द्रस्त्वꣳ\\ रुद्रस्त्वं विष्णुस्त्वं ब्रह्म त्वं॑ प्रजा॒पतिः। \\
त्वं त॑दाप॒ आपो॒ ज्योती॒ रसो॒ऽमृतं॒ ब्रह्म॒ भूर्भुवः॒ सुव॒रोम्‌॥\medskip

\twolineshloka*
{यो वेदादौ स्व॑रः प्रो॒क्तो॒ वे॒दान्ते॑ च प्र॒तिष्ठि॑तः}
{तस्य॑ प्र॒कृति॑लीन॒स्य॒ यः॒ परः॑ स म॒हेश्व॑रः}
\medskip

आवाहिताभ्यः सर्वाभ्यो-देवताभ्यो नमः वेदोक्तमन्त्रपुष्पाञ्जलिं समर्पयामि।\medskip

स्वर्णपुष्पं समर्पयामि।\medskip


\twolineshloka*
{यानि कानि च पापानि जन्मान्तरकृतानि च}
{तानि तानि विनश्यन्ति प्रदक्षिण-पदे पदे}
\textbf{प्रदक्षिणं कृत्वा।}
\medskip


\twolineshloka*
{नमः शिवाय साम्बाय सगणाय ससूनवे}
{सनन्दिने सगङ्गाय सवृषाय नमो नमः}

\fourlineindentedshloka*
{नमः शिवाभ्यां नवयौवनाभ्याम्‌}
{परस्पराश्लिष्टवपुर्धराभ्याम्‌}
{नगेन्द्रकन्यावृषकेतनाभ्याम्‌}
{नमो नमः शङ्करपार्वतीभ्याम्‌}%१

\dnsub{नमस्कारमन्त्राः}
नमो हिरण्यबाहवे हिरण्यवर्णाय हिरण्यरूपाय हिरण्यपतये\-ऽम्बिकापतय उमापतये पशुपतये॑ नमो॒ नमः॥%३९॥

ऋ॒तꣳ स॒त्यं प॑रं ब्र॒ह्म॒ पु॒रुषं॑ कृष्ण॒पिङ्ग॑लम्।\\ 
ऊ॒र्ध्वरे॑तं वि॑रूपा॒क्षं॒ वि॒श्वरू॑पाय॒ वै नमो॒ नमः॑॥%४०॥

सर्वो॒ वै रु॒द्रस्तस्मै॑ रु॒द्राय॒ नमो॑ अस्तु।\\ 
पुरु॑षो॒ वै रु॒द्रः सन्म॒हो नमो॒ नमः॑।\\
विश्वं॑ भू॒तं भुव॑नं चि॒त्रं ब॑हु॒धा जा॒तं जाय॑मानं च॒ यत्।\\
सर्वो॒ ह्ये॑ष रु॒द्रस्तस्मै॑ रु॒द्राय॒ नमो॑ अस्तु॥%४१॥


%६.१७.१
कद्रु॒द्राय॒ प्रचे॑तसे मी॒ढुष्ट॑माय॒ तव्य॑से।\\ 
वो॒ चेम॒ शन्त॑मꣳ हृ॒दे। 
सर्वो॒ ह्ये॑ष रु॒द्रस्तस्मै॑ रु॒द्राय॒ नमो॑ अस्तु॥%४२॥


अनन्तकोटिप्रदक्षिणनमस्कारान् समर्पयामि। नमस्कारान् कृत्वा।\medskip

छत्त्र-चामर-नृत्त-गीत-वाद्य-समस्त-राजोपचारान् समर्पयामि।\medskip


\twolineshloka*
{बाण-रावण-चण्डेश-नन्दि-भृङ्गि-रिटादयः}
{महादेवप्रसादोऽयं सर्वे गृह्णन्तु शाम्भवाः}
नन्दिकेश्वराय नमः बलिं निवेदयामि। ॐ हर। ॐ हर। ॐ हर।\medskip

\twolineshloka*
{शङ्खमध्ये स्थितं तोयं भ्रामितं शङ्करोपरि}
{अङ्गलग्नं मनुष्याणां ब्रह्महत्यायुतं दहेत्}
शङ्खजलेन प्रोक्ष्य।

\twolineshloka* 
{सालग्रामशिलावारि पापहारि शरीरिणाम्}
{आजन्मकृतपापानां प्रायश्चित्तं दिने दिने}

\twolineshloka*
{अकालमृत्युहरणं सर्वव्याधिनिवारणम्}
{सर्वपापक्षयकरं शिवपादोदकं शुभम्}
इति अभिषेकतीर्थं प्राश्य।

\twolineshloka*
{साधु वाऽसाधु वा कर्म यद्यदाचरितं मया}
{तत्सर्वं कृपया देव गृहाणाऽऽराधनं मम}

\twolineshloka*
{हृद्पद्मकर्णिका-मध्यमुमया सह शङ्कर}
{प्रविश त्वं महादेव सर्वैरावरणैः सह}

\fourlineindentedshloka*
{सम्पूजकानां परिपालकानां}
{यतेन्द्रियानां च तपोधनानाम्}
{देशस्य राष्ट्रस्य कुलस्य राज्ञाम्}
{करोतु शान्तिं भगवान् कुलेशः}


अनया पूजया सपरिवार-साम्ब-परमेश्वरः प्रीयताम्। 


\dnsub{उद्वासनम्}

निर्याणमुद्रया पुष्पाण्यादाय आघ्राय हृदये स्थापयित्वा उद्वासयेत्। निर्माल्यं शिरसि धारयेत्।


\input{../stotra-sangrahah/stotras/vishnu/BrahmapaaraStotram.tex}


\fourlineindentedshloka*
{कायेन वाचा मनसेन्द्रियैर्वा}
{बुद्‌ध्याऽऽत्मना वा प्रकृतेः स्वभावात्}
{करोमि यद्यत् सकलं परस्मै}
{नारायणायेति समर्पयामि}


ॐ तत्सद्ब्रह्मार्पणमस्तु।\medskip

आचामेत्।

\end{center}
\closesection
\clearpage