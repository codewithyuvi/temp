% !TeX program = XeLaTeX
% !TeX root = ../pujavidhanam.tex

\setlength{\parindent}{0pt}
\chapt{शिवरत्रि-पूजा — याम-चतुष्टय-पूजा}

आचम्य।

\twolineshloka*
{शुक्लाम्बरधरं विष्णुं शशिवर्णं चतुर्भुजम्}
{प्रसन्नवदनं ध्यायेत् सर्वविघ्नोपशान्तये}
 
प्राणान्  आयम्य।  ॐ भूः + भूर्भुवः॒ सुव॒रोम्।

\sect{व्रत-सङ्कल्पः}

ममोपात्तसमस्तदुरितक्षयद्वारा श्रीपरमेश्वरप्रीत्यर्थं शुभे शोभने मुहूर्ते अद्यब्रह्मणः
द्वितीयपरार्द्धे श्वेतवराहकल्पे वैवस्वतमन्वन्तरे अष्टाविंशतितमे कलियुगे प्रथमे पादे
जम्बूद्वीपे भारतवर्षे भरतखण्डे मेरोः दक्षिणेपार्श्वे शकाब्दे अस्मिन् वर्तमाने व्यावहारिकणां
 प्रभवादि षष्ट्याः संवत्सराणां मध्ये (	)\see{app:samvatsara_names} नाम संवत्सरे \textbf{उत्तरायणे} 
\textbf{शिशिर}-ऋतौ  \textbf{कुम्भ}-मासे \textbf{कृष्ण}-पक्षे त्र्योदश्यां/चतुर्दश्यां शुभतिथौ
(इन्दु / भौम / बुध / गुरु / भृगु / स्थिर / भानु) वासरयुक्तायाम्
(  )\see{app:nakshatra_names} नक्षत्र (  )\see{app:yoga_names} नाम  योग  (  ) करण युक्तायां च एवं गुण विशेषण विशिष्टायाम्
अस्याम् (त्र्योदश्यां/चतुर्दश्यां) शुभतिथौ 
अस्माकं सहकुटुम्बानां क्षेमस्थैर्य-धैर्य-वीर्य-विजय आयुरारोग्य ऐश्वर्याभिवृद्ध्यर्थम्
 धर्मार्थकाममोक्ष\-चतुर्विधफलपुरुषार्थसिद्ध्यर्थं पुत्रपौत्राभि\-वृद्ध्यर्थम् इष्टकाम्यार्थसिद्ध्यर्थम्
मम इहजन्मनि पूर्वजन्मनि जन्मान्तरे च सम्पादितानां ज्ञानाज्ञानकृतमहा\-पातकचतुष्टय
व्यतिरिक्तानां रहस्यकृतानां प्रकाशकृतानां सर्वेषां पापानां सद्य अपनोदनद्वारा सकल 
पापक्षयार्थं श्री-साम्ब-परमेश्वर-प्रीत्यर्थं शिवरात्रि-व्रतं करिष्ये।

\twolineshloka*
{शिवरात्रिव्रतं ह्येतत्करिष्येऽहं महाफलम्}
{निर्विघ्नमस्तु मे चात्र त्वत्प्रसादाज्जगत्पते}

\twolineshloka*
{चतुर्दश्यां निराहारो भूत्वा शम्भो परेऽहनि}
{भोक्ष्येऽहं भुक्तिमुक्त्यर्थं शरणं मे भवेश्वर}






\sect{पूर्वाङ्गविघ्नेश्वरपूजा}

(आचम्य)
\twolineshloka*
{शुक्लाम्बरधरं विष्णुं शशिवर्णं चतुर्भुजम्}
{प्रसन्नवदनं ध्यायेत् सर्वविघ्नोपशान्तये}
 
प्राणान्  आयम्य।  ॐ भूः + भूर्भुवः॒ सुव॒रोम्।
 
(अप उपस्पृश्य, पुष्पाक्षतान् गृहीत्वा)\\
ममोपात्तसमस्त दुरितक्षयद्वारा \\
श्रीपरमेश्वरप्रीत्यर्थं करिष्यमाणस्य कर्मणः\\
 निर्विघ्नेन परिसमाप्त्यर्थम् आदौ विघ्नेश्वरपूजां करिष्ये।

\twolineshloka*
{ॐ ग॒णानां᳚ त्वा ग॒णप॑तिꣳ हवामहे क॒विं क॑वी॒नामु॑प॒मश्र॑वस्तमम्}
{ज्ये॒ष्ठ॒राजं॒ ब्रह्म॑णां ब्रह्मणस्पत॒ आ नः॑ शृ॒ण्वन्नू॒तिभिः॑ सीद॒ साद॑नम्}
अस्मिन् हरिद्राबिम्बे महागणपतिं ध्यायामि, आवाहयामि।\\


ॐ महागणपतये नमः  आसनं समर्पयामि।\\
पादयोः पाद्यं समर्पयामि। हस्तयोरर्घ्यं समर्पयामि।\\
आचमनीयं समर्पयामि।\\
ॐ भूर्भुवस्सुवः। शुद्धोदकस्नानं समर्पयामि।\\
स्नानानन्तरमाचमनीयं समर्पयामि।\\
वस्त्रार्थमक्षतान् समर्पयामि।\\
यज्ञोपवीताभरणार्थे अक्षतान् समर्पयामि।\\
दिव्यपरिमलगन्धान् धारयामि।\\
गन्धस्योपरि हरिद्राकुङ्कुमं समर्पयामि। अक्षतान् समर्पयामि। \\
पुष्पमालिकां समर्पयामि। पुष्पैः पूजयामि।

\dnsub{अर्चना}
% \setenumerate{label=\devanumber.}
% \renewcommand{\labelenumi}{\devanumber\theenumi.}
\begin{enumerate}%[label=\devanumber\value{enumi}]
\begin{minipage}{0.475\linewidth}   
\item ॐ सुमुखाय नमः
\item ॐ एकदन्ताय नमः
\item ॐ कपिलाय नमः
\item ॐ गजकर्णकाय नमः
\item ॐ लम्बोदराय नमः
\item ॐ विकटाय नमः
\item ॐ विघ्नराजाय नमः
\item ॐ विनायकाय नमः
\item ॐ धूमकेतवे नमः
  \end{minipage}
  \begin{minipage}{0.525\linewidth}
\item ॐ गणाध्यक्षाय नमः
\item ॐ फालचन्द्राय नमः
\item ॐ गजाननाय नमः
\item ॐ वक्रतुण्डाय नमः
\item ॐ शूर्पकर्णाय नमः
\item ॐ हेरम्बाय नमः
\item ॐ स्कन्दपूर्वजाय नमः
\item ॐ सिद्धिविनायकाय नमः
\item ॐ विघ्नेश्वराय नमः
  \end{minipage}
\end{enumerate}
नानाविधपरिमलपत्रपुष्पाणि समर्पयामि॥\\
धूपमाघ्रापयामि।\\
अलङ्कारदीपं सन्दर्शयामि।\\
नैवेद्यम्।\\
ताम्बूलं समर्पयामि।\\
कर्पूरनीराजनं समर्पयामि।\\
कर्पूरनीराजनानन्तरमाचमनीयं समर्पयामि।\\
{वक्रतुण्डमहाकाय कोटिसूर्यसमप्रभ।}\\
{अविघ्नं कुरु मे देव सर्वकार्येषु सर्वदा॥}\\
प्रार्थनाः समर्पयामि।

अनन्तकोटिप्रदक्षिणनमस्कारान् समर्पयामि।\\
छत्त्रचामरादिसमस्तोपचारान् समर्पयामि।\\


\sect{प्रधान पूजा - साम्ब-परमेश्वर पूजा (प्रथम-यामः)}

\input{pujas/shivaratri-yama-puja}

\dnsub{अर्घ्यप्रदानम्}
ममोपात्त समस्तदुरितक्षयद्वारा श्रीपरमेश्वरप्रीत्यर्थम् शिवरात्रौ प्रथम-याम-पूजान्ते क्षीरार्घ्यप्रदानं करिष्ये॥

\twolineshloka*
{शिवरात्रिवतं देव पूजाजपपरायणः}
{करोमि विधिवद्दत्ं गृहाणार्घ्यं नमोऽस्तुते}

इत्यर्घ्यं दत्वा।
\medskip

\twolineshloka*
{नमः शिवाय शान्ताय सर्वपापहराय च}
{शिवरात्रौ मया दत्तं गृहाणार्घ्यं मम प्रभो॥१}


\dnsub{पूजानिवेदनम्}
\twolineshloka*
{नमो यज्ञजगन्नाथ नमस्त्रिभुवनेश्वर}
{पूजां गृहाण मे दत्तां महेश प्रथमे पदे॥१}


\twolineshloka*
{यत्किञ्चित् कुर्महे देव सदा सुकृतदुष्कृतम्}
{तन्मे शिवपदस्थस्य भुक्ष्व क्षपय शङ्कर}

\twolineshloka*
{शिवो दाता शिवो भोक्ता शिवः सर्वमिदं जगत्}
{शिवो जयति सर्वत्र यः शिवः सोऽहमेव हि}

इतिप्रार्थ्य॥

 देवहृदयस्थं पादस्थं च पुष्पमादाय प्रणम्य देवेन दत्तमिति ध्यात्वा॥

\threelineshloka*
{प्रपन्नं पाहि मामीश भीतमृत्युमहार्णवात्}
{त्वयोपभुक्त-स्रग्-गन्ध-वासो-लङ्कार-चर्चिताः}
{उच्छिष्टभोजिनो दासास्तव मायां जयेम हि}

इति मूर्ध्नि धृत्वा॥

\twolineshloka*
{मन्त्रहीनं क्रियाहीनं भक्तिहीनं महेश्वर}
{यत्कृतं तु मया देव परिपूर्णं तदस्तु ते}


अनेन पूजनेन साम्बसदाशिवः प्रीयताम्। 

\sect{प्रधान पूजा - साम्ब-परमेश्वर पूजा (द्वितीय-यामः)}

\input{pujas/shivaratri-yama-puja}

\dnsub{अर्घ्यप्रदानम्}
ममोपात्त समस्तदुरितक्षयद्वारा श्रीपरमेश्वरप्रीत्यर्थम् शिवरात्रौ द्वितीय-याम-पूजान्ते क्षीरार्घ्यप्रदानं करिष्ये॥

\twolineshloka*
{शिवरात्रिवतं देव पूजाजपपरायणः}
{करोमि विधिवद्दत्ं गृहाणार्घ्यं नमोऽस्तुते}

इत्यर्घ्यं दत्वा।
\medskip

\twolineshloka*
{मया कृतान्यनेकानि पापानि हर शङ्कर}
{गृहाणार्घ्यमुमाकान्त शिवरात्रौ प्रसीद मे॥२}

\dnsub{पूजानिवेदनम्}
\threelineshloka*
{पूर्वे नन्दि महाकालौ गणभृङ्गी च दक्षिणे}
{वृषस्कन्दौ पश्चिमे ते देशकालौ तथोत्तरे}
{गङ्गा च यमुना पार्श्वे पूजां गृह्ण नमोऽस्तुते॥२}


\twolineshloka*
{यत्किञ्चित् कुर्महे देव सदा सुकृतदुष्कृतम्}
{तन्मे शिवपदस्थस्य भुक्ष्व क्षपय शङ्कर}

\twolineshloka*
{शिवो दाता शिवो भोक्ता शिवः सर्वमिदं जगत्}
{शिवो जयति सर्वत्र यः शिवः सोऽहमेव हि}

इतिप्रार्थ्य॥

 देवहृदयस्थं पादस्थं च पुष्पमादाय प्रणम्य देवेन दत्तमिति ध्यात्वा॥

\threelineshloka*
{प्रपन्नं पाहि मामीश भीतमृत्युमहार्णवात्}
{त्वयोपभुक्त-स्रग्-गन्ध-वासो-लङ्कार-चर्चिताः}
{उच्छिष्टभोजिनो दासास्तव मायां जयेम हि}

इति मूर्ध्नि धृत्वा॥

\twolineshloka*
{मन्त्रहीनं क्रियाहीनं भक्तिहीनं महेश्वर}
{यत्कृतं तु मया देव परिपूर्णं तदस्तु ते}


अनेन पूजनेन साम्बसदाशिवः प्रीयताम्। 

\sect{प्रधान पूजा - साम्ब-परमेश्वर पूजा (तृतीय-यामः)}

\input{pujas/shivaratri-yama-puja}

\dnsub{अर्घ्यप्रदानम्}
ममोपात्त समस्तदुरितक्षयद्वारा श्रीपरमेश्वरप्रीत्यर्थम् शिवरात्रौ तृतीय-याम-पूजान्ते क्षीरार्घ्यप्रदानं करिष्ये॥

\twolineshloka*
{शिवरात्रिवतं देव पूजाजपपरायणः}
{करोमि विधिवद्दत्ं गृहाणार्घ्यं नमोऽस्तुते}

इत्यर्घ्यं दत्वा।
\medskip

\twolineshloka*
{शिवरात्रिव्रतं देव पूजाजपपरायणः}
{करोमि विधिवद्दत्तं गृहाणार्घ्यं नमोऽस्तु ते॥निशीथे}

\twolineshloka*
{दुःखदारिद्र्यभारैश्च दग्धोऽहं पार्वतीपते}
{त्रायस्व मां महादेव गृहाणार्घ्यं नमोऽस्तु ते॥३} 


\dnsub{पूजानिवेदनम्}
\twolineshloka*
{नमोऽव्यक्ताय सूक्ष्माय नमस्ते त्रिपुरान्तक}
{पूजां गृहाण देवेश यथाशक्त्युपपादिताम्॥३}


\twolineshloka*
{यत्किञ्चित् कुर्महे देव सदा सुकृतदुष्कृतम्}
{तन्मे शिवपदस्थस्य भुक्ष्व क्षपय शङ्कर}

\twolineshloka*
{शिवो दाता शिवो भोक्ता शिवः सर्वमिदं जगत्}
{शिवो जयति सर्वत्र यः शिवः सोऽहमेव हि}

इतिप्रार्थ्य॥

 देवहृदयस्थं पादस्थं च पुष्पमादाय प्रणम्य देवेन दत्तमिति ध्यात्वा॥

\threelineshloka*
{प्रपन्नं पाहि मामीश भीतमृत्युमहार्णवात्}
{त्वयोपभुक्त-स्रग्-गन्ध-वासो-लङ्कार-चर्चिताः}
{उच्छिष्टभोजिनो दासास्तव मायां जयेम हि}

इति मूर्ध्नि धृत्वा॥

\twolineshloka*
{मन्त्रहीनं क्रियाहीनं भक्तिहीनं महेश्वर}
{यत्कृतं तु मया देव परिपूर्णं तदस्तु ते}


अनेन पूजनेन साम्बसदाशिवः प्रीयताम्। 

\sect{प्रधान पूजा - साम्ब-परमेश्वर पूजा (चतुर्थ-यामः)}

\input{pujas/shivaratri-yama-puja}

\dnsub{अर्घ्यप्रदानम्}
ममोपात्त समस्तदुरितक्षयद्वारा श्रीपरमेश्वरप्रीत्यर्थम् शिवरात्रौ चतुर्थ-याम-पूजान्ते क्षीरार्घ्यप्रदानं करिष्ये॥

\twolineshloka*
{शिवरात्रिवतं देव पूजाजपपरायणः}
{करोमि विधिवद्दत्ं गृहाणार्घ्यं नमोऽस्तुते}

इत्यर्घ्यं दत्वा।
\medskip


\twolineshloka*
{किं न जानासि देवेश त्वयि भक्तिं प्रयच्छ मे}
{स्वपादाग्रतले देव दास्यं देहि जगत्पते॥४}


\dnsub{पूजानिवेदनम्}
\twolineshloka*
{बद्धोऽहं विविधैः पाशैः संसारभयबन्धनैः}
{पतितं मोहजाले मां त्वं समुद्धर शङ्कर॥४}

\twolineshloka*
{यत्किञ्चित् कुर्महे देव सदा सुकृतदुष्कृतम्}
{तन्मे शिवपदस्थस्य भुक्ष्व क्षपय शङ्कर}

\twolineshloka*
{शिवो दाता शिवो भोक्ता शिवः सर्वमिदं जगत्}
{शिवो जयति सर्वत्र यः शिवः सोऽहमेव हि}

इतिप्रार्थ्य॥

 देवहृदयस्थं पादस्थं च पुष्पमादाय प्रणम्य देवेन दत्तमिति ध्यात्वा॥

\threelineshloka*
{प्रपन्नं पाहि मामीश भीतमृत्युमहार्णवात्}
{त्वयोपभुक्त-स्रग्-गन्ध-वासो-लङ्कार-चर्चिताः}
{उच्छिष्टभोजिनो दासास्तव मायां जयेम हि}

इति मूर्ध्नि धृत्वा॥

\twolineshloka*
{मन्त्रहीनं क्रियाहीनं भक्तिहीनं महेश्वर}
{यत्कृतं तु मया देव परिपूर्णं तदस्तु ते}


अनेन पूजनेन साम्बसदाशिवः प्रीयताम्। 




\fourlineindentedshloka*
{कायेन वाचा मनसेन्द्रियैर्वा}
{बुद्‌ध्याऽऽत्मना वा प्रकृतेः स्वभावात्}
{करोमि यद्यत् सकलं परस्मै}
{नारायणायेति समर्पयामि}


ॐ तत्सद्ब्रह्मार्पणमस्तु।\medskip


\dnsub{उत्तरस्मिन्   दिने पारणम्}

\twolineshloka*
{संसारक्लेशदग्धस्य व्रतेनानेन शङ्कर}
{प्रसीद सुमुखो नाथ ज्ञानदृष्टिप्रदो भव}

\closesection

