\sect{कथा}

\uvacha{सूत उवाच}

\twolineshloka
{कैलासशिखरे रम्ये सर्वदेवनिषेविते}
{गौर्या सह महादेवो दीव्यन्नक्षैवनोदतः}%॥१॥


\twolineshloka
{जितोऽसि त्वं मया चाऽऽह पार्वती परमेश्वरम्}
{सोऽपि त्वं च जितेत्याह सुविवादस्तयोरभूत्}%॥२॥


\twolineshloka
{चित्रनेमिस्तदा पृष्टो मृषावादमभाषत}
{तदा कोपसमाविष्टा गौरी शापं ददौ ततः}%॥३॥


\twolineshloka
{कुष्ठी भव मृषावादिन् चित्रनेमिर्हतप्रभः}
{नानृतेन समं पापं क्वापि दृष्टं श्रुतं मया}%॥४॥


\twolineshloka
{चित्रनेमिर्महाप्राज्ञः सत्यं वदति नो मृषा}
{प्रसादः क्रियतां देवि देवीमाह वृषध्वजः}%॥५॥


\twolineshloka
{प्रसादसुमुखी तस्मै विशापं च जगाद सा}
{यदा सरोवरे रम्ये करिष्यन्ति शुचिव्रतम्}%॥६॥


\twolineshloka
{ततः स्वर्गणिकाः सर्वं यक्ष्यन्ति त्वां समाहिताः}
{तदा तव विशापः स्यादित्युक्तः स पपात ह}%॥७॥


\twolineshloka
{ततः कतिपयाहोभिश्चित्रनेमिः सरोवरे}
{कुष्ठी भूत्वा वसंस्तत्र ददर्श स्वर्विलासिनीः}%॥८॥


\twolineshloka
{देवतापूजनासक्ताः पप्रच्छ प्रणिपत्यताः}
{किमेतद्भो महाभागाः किं पूजा किं च वाञ्छितम्}%॥९॥


\twolineshloka
{किं मया च ह्यनुष्ठेयमिहामुत्र फलप्रदम्}
{इति व्रतं चित्रनेमिः पप्रच्छ स्वर्विलासिनीः}%॥१०॥


\twolineshloka
{येनाहे गिरिजाशापान्मोक्ष्यामि चिरदुःखतः}
{ता ऊचुः क्रियतामद्य त्वया चैतदनुत्तमम्}%॥११॥


\twolineshloka
{वरलक्ष्मीव्रतं दिव्यं सर्वकामसमृद्धिदम्}
{यदा रवौ कुलीरस्थे मासे च श्रावणे तथा}%॥१२॥


\twolineshloka
{गङ्गायमुनयोर्योगे तुङ्गभद्रासरित्तटे}
{तस्मिन्वै श्रावणे मासि शुक्लपक्षे भृगोर्दिने}%॥१३॥


\twolineshloka
{प्रारब्धव्यं व्रतं तत्र महालक्ष्म्या यतात्मभिः}
{सुवर्णप्रतिमां कुर्याच्चतुर्भुजसमन्विताम्}%॥१४॥


\twolineshloka
{पूर्व गृहमलङ्कृत्य तोरणै रङ्गवल्लिभिः}
{गृहस्य पूर्वदिग्भागे ईशान्यां च विशेषतः}%॥१५॥


\twolineshloka
{प्रस्थमितांस्तण्डुलांश्च भूमौ निक्षिप्य पद्मके}
{संस्थाप्य कलशं तत्र तीर्थतोयैः प्रपूरयेत्}%॥१६॥


\twolineshloka
{फलानि च विनिक्षिप्य सुवर्णं प्रक्षिपेत्ततः}
{पल्लवांश्च विनिक्षिप्य वस्त्रेणाच्छाद्य यत्नतः}%॥१७॥


\twolineshloka
{प्रतिमां स्थापयेत्तत्र पूजयेच्च यथाविधि}
{अग्न्युत्तारणपूर्वं तु शुद्धस्नानं यथाक्रमम्}%॥१८॥


\twolineshloka
{पञ्चामृतेन स्नपनं कारयेन्मन्त्रतः सुधीः}
{अभिषेकं ततः कृत्वा देवीसूक्तेन वै ततः}%॥१९॥


\twolineshloka
{अष्टगन्धैः समभ्यर्च्य पल्लवैश्च समर्चयेत्}
{अश्वत्थवटबिल्वाम्रमालतीदाडिमास्तथा}%॥२०॥


\twolineshloka
{एतेषां पत्राण्यादाय एकविंशतिसंख्यया}
{नामाविधैस्तथा पुष्पैर्मालत्यादिसमुद्भवैः}%॥२१॥


\twolineshloka
{धूपदीपैर्महालक्ष्मीं पूजयेत् सर्वकामदाम्}
{पायसैर्भक्ष्यभोज्यैश्च नानाव्यञ्जनसंयुतैः}%॥२२॥


\twolineshloka
{एकविंशतिसङ्ख्याकैरपूपैः पूजयेच्छिवाम्}
{निवेद्य सर्वदेव्यै तु वरं स वृणुयात्ततः}%॥२३॥


\twolineshloka
{नृत्यगीतादिसहितो देवीं सम्प्रार्थयेच्छ्रियम्}
{रमां सरस्वतीं ध्यायेच्छचीं च प्रियवादिनीम्}%॥२४॥


\twolineshloka
{एवं व्रतविधिं तस्मै कथयित्वा विधानतः}
{पञ्चवायनकान् दत्त्वा कथां शृण्वीत यत्नतः}%॥२५॥


\twolineshloka
{तथा मौनं गृहीत्वा तु पञ्चार्तिक्येन पूजयेत्}
{व्रतं च कुर्वता गृह्य एकं पूगफलं तथा}%॥२६॥


\twolineshloka
{पर्णेकं चूर्णरहितं चर्वणीयं प्रयत्नतः}
{चैलखण्डे दृढं बद्ध्वा प्रातः पश्येद्विचक्षणः}%॥२७॥


\twolineshloka
{आरक्तं यदि जायेत कुर्याद्व्रतमनुतमम्}
{नोचेन्न तद्व्रतं कार्यं सर्वथा भूतिमिच्छता}%॥२८॥


\twolineshloka
{अनेनैव विधानेन व्रतं गृह्णीत यत्नतः}
{अप्सरोभिः कृतं सम्यग्व्रतं सर्वसमृद्धिदम्}%॥२९॥


\twolineshloka
{पूजावसानपर्यन्तं चित्रनेमिरलोकयत्}
{धूपधूमं समाघ्राय घृतदीपप्रभावतः}%॥३०॥


\twolineshloka
{गतकुष्ठः स्वर्णतेजाः शुचिस्तद्गतमानसः}
{अहं यत्नात् करिष्यामि व्रतं सर्वसमृद्धिदम्}%॥३१॥


\twolineshloka
{इत्युक्त्वा सर्वदेवीस्तु कारयामास तत्क्षणात्}
{सुवर्णनिर्मितां देवीं वस्त्रालङ्कारसंयुताम्}%॥३२॥


\twolineshloka
{पूर्वोक्तेन विधानेन पूजां कृत्वा प्रयत्नतः}
{ततो वैणवपात्राणि फलान्नैश्च सदक्षिणैः}%॥३३॥


\twolineshloka
{एकविंशतिपक्वान्नैः पूरितानि विधाय च}
{पञ्चवायनकान्येवं कृत्वादात्तु यथाक्रमम्}%॥३४॥


\twolineshloka
{विप्राय चाथ यतये देव्यै तु ब्रह्मचारिणे}
{सुवासिन्यै ततस्त्वेकमर्पितं चित्रनेमिना}%॥३५॥


\twolineshloka
{एवं सम्यक् क्रमेणैतद्दत्त्वा वायनपञ्चकम्}
{ततो गृहं गतः सोऽथ देवीं नत्वा यथाक्रमम्}%॥३६॥


\twolineshloka
{नागवल्लीदलं त्वेकं क्रमुकं चूर्णवजितम्}
{भक्षययित्वा तु चैलान्ते बद्ध्वा प्रातर्निरैक्षत}%॥३७॥


\twolineshloka
{आरक्ते च ततो जाते व्रतं चक्रे स भक्तितः}
{अद्याहं गतपापोऽस्मि देवीदर्शनयोगतः}%॥३८॥


\twolineshloka
{एतत्सम्यग्व्रतं चीर्णं भक्तिभावेन यन्मया}
{चित्रनेमिव्रतं कृत्वा कैलासं शङकरालयम्}%॥३९॥


\twolineshloka
{गत्वा प्रणम्य देवेशं देवीमादरपूर्वकम्}
{पार्वती च तदा प्राह चित्रनेमे स्वपुत्रवत्}%॥४०॥


\twolineshloka
{पालनीयो मया त्वं च सत्यमित्यवधार्यताम्}
{चित्रनेमिस्तदा प्राहः पार्वतीं हरवल्लभे}%॥४१॥


\twolineshloka
{तव पादाम्बुजं दृष्टं वरलक्ष्मीप्रसादतः}
{महादेवस्ततः प्राह चित्रनेमिं शुचिव्रतम्}%॥४२॥


\twolineshloka
{अद्यप्रभृति कैलासे भुङ्क्ष्व भोगान् यथेप्सितान्}
{पश्चाद्गन्तासि वैकुण्ठं वरस्यास्य प्रसादतः}%॥४३॥


\twolineshloka
{पार्वत्यापि कृतं पूर्वं पुत्रलाभार्थमेव च}
{लब्धश्च षण्मुखो देव्या व्रतराजप्रसादतः}%॥४४॥


\twolineshloka
{नन्दश्च विक्रमादित्यो राज्यं प्राप्तौ महाव्रतौ}
{नन्दश्च कान्तया हीनः कान्तां लेभे सुलक्षणाम्}%॥४५॥


\twolineshloka
{तयाचतद्व्रतं कृत्स्नं कृतं वै पुत्रहेतवे}
{पुत्रं प्रसुषुवे सा च त्रैलोक्यभरणक्षमम्}%॥४६॥


\twolineshloka
{इह भुक्त्वा तु विपुलान्भोगान्वै सुमनोहरान्}
{तदाप्रभृति लोकेऽस्मिन् वरलक्ष्मी व्रतं शुभम्}%॥४७॥


\twolineshloka
{व्रतं करोति या नारी नरो वापि शुचिव्रतः}
{भुक्त्वा भोगांश्च विपुलानन्ते शिवपुरं व्रजेत्}%॥४८॥


\twolineshloka
{इत्याख्यातं मया विप्रा वरलक्ष्मीव्रतं शुभम्}
{य इदं शृणुयान्नित्यं श्रावयेद्वा समाहितः}%॥४९।}

\onelineshloka
{धनं धान्यमवाप्नोति वरलक्ष्मीप्रसादतः}%॥५०॥

॥इति श्रीभविष्योत्तरपुराणे श्रावणशुक्रवारे वरलक्ष्मीव्रतं सम्पूर्णम्॥