\sect{श्रीमच्चिद्विलासीय-शङ्करविजयविलासे श्रीमत्-शङ्कर-भगवत्पाद-अवतार-घट्टः}

\dnsub{पञ्चमोऽध्यायः}
\addtocounter{shlokacount}{33}

\twolineshloka
{व्यराजत तदार्याम्बा शिवैकायत्तचेतना}
{दृष्ट्वा शिवगुरुर्यज्वा भार्यामार्यां च गर्भिणीम्} %॥३४॥

\twolineshloka
{वृषाचलेशं सततं स्मरन्नेकाग्रचेतसा}
{दयालुतां स्तुवन् शम्भोर्दीनेष्वपि महत्स्वपि} %॥३५॥

\twolineshloka
{ववृधे स पयोराशिः पूर्णेन्दोरिव दर्शनात्}
{ततः सा दशमे मासि सम्पूर्णशुभलक्षणे} %॥३६॥

\twolineshloka
{दिवसे माधवर्तौ च स्वोच्चस्थे ग्रहपञ्चके}
{मध्याह्ने चाभिजिन्नाममुहूर्ते चार्द्रया युते} %॥३७॥

\twolineshloka
{उदयाचलवेलेव भानुमन्तं महौजसम्}
{प्रासूत तनयं साध्वी गिरिजेव षडाननम्} %॥३८॥

\twolineshloka
{जयन्तमिव पौलोमी व्यासं सत्यवती यथा}
{तदैवाग्रे निरीक्ष्येयमनुभूयेव वेदनाम्} %॥३९॥

\twolineshloka
{चतुर्भुजमुदाराङ्गं त्रिणेत्रं चन्द्रशेखरम्}
{दुर्निरीक्ष्यैः स्वतेजोभिर्भासयन्तं दिशो दश} %॥४०॥

\twolineshloka
{दिवाकरकराकारैर्गौरैरीषद्विलोहितैः}
{एवमाकारमालोक्य विस्मिता विह्वला भिया} %॥४१॥

\twolineshloka
{किं किं किमिदमाश्चर्यमन्यदेव मदीप्सितम्}
{परं त्वन्यत् समुद्भूतमिति चिन्ताभृति स्वयम्} %॥४२॥

\twolineshloka
{उद्वीक्षन्त्यां प्रणमितुं तस्यां कुतुकतायुजि}
{ससृजुः पुष्पवर्षाणि देवा भुव्यन्तरिक्षगाः} %॥४३॥

\twolineshloka
{कह्लारकलिकागन्धबन्धुरो मरुदाववौ}
{दिशः प्रकाशिताकाशाः सा धरा सादरा बभौ} %॥४४॥

\twolineshloka
{प्रायः प्रदक्षिणज्वाला जज्वलुर्यज्ञपावकाः}
{प्रसन्नमभवच्चित्तं सतां प्रतपतामपि} %॥४५॥

\twolineshloka
{इत्थमन्यद्विलोक्यापि प्रश्रिता विनयान्विता}
{वृषाचलेशं निश्चित्य प्रादुर्भूतमतन्द्रिता} %॥४६॥

\twolineshloka
{स्वामिन् दर्शय मे लीला बालभावक्रमोचिताः}
{इत्थं सा प्रार्थयामास साध्वी भूयो महेश्वरम्} %॥४७॥

\twolineshloka
{ततः किशोरवत्सोऽपि किञ्चिद्विचलिताधरः}
{ताडयन् चरणौ हस्तौ रुरोदैव क्षणादसौ} %॥४८॥

\twolineshloka
{आर्या साऽपि तदैवासीन्मायामोहितमानसा}
{जगन्मोहकरी माया महेशितुरनीदृशी} %॥४९॥

\twolineshloka
{तत्रत्यास्तु जना नार्यो नाविन्दन् वृत्तमीदृशम्}
{बालकं मेनिरे प्रोद्यदिन्दुबिम्बमिवोज्ज्वलम्} %॥५०॥

\twolineshloka
{तत्रत्या वृद्धनार्योऽपि यथोचितमथाचरन्}
{ततः श्रुत्वा पिता सोऽपि निधिं प्राप्येव निर्धनः} %॥५१॥

\twolineshloka
{मुमुदे नितरां चित्ते वित्तेशं नाभ्यलक्षत}
{आविर्भावं तु जानाति शम्भोर्नाबोधयच्च सा} %॥५२॥

\twolineshloka
{स्नात्वा शिवगुरुर्यज्वा यज्वनामग्रणीस्ततः}
{विप्रानाकारयामास पुरन्ध्रीरपि सर्वतः} %॥५३॥

\twolineshloka
{तदोत्सवो महानासीत् पुरे सद्मनि सन्ततम्}
{धान्यराशिं मखिभ्योऽसौ विद्भ्यो भूयः प्रदत्तवान्} %॥५४॥

\twolineshloka
{धनानि भूरि विप्रेभ्यो वेदविद्भ्यो दिदेश सः}
{वासांसि भूयो दिव्यानि सफलानि प्रदत्तवान्} %॥५५॥

\twolineshloka
{पुरन्ध्रीणां च नीरन्ध्रं वस्तुजातान्यदादसौ}
{घटोघ्नीर्बहुशो गाश्च सालङ्काराः सदक्षिणाः} %॥५६॥

\twolineshloka
{वृषाचलेशः सततं प्रीयतामित्यसौ ददौ}
{ततः शिवगुरुर्यज्वा ब्राह्मणान् पूर्वतोऽधिकम्} %॥५७॥

\twolineshloka
{सन्तर्प्य बन्धुभिः सार्धं मुदितो न्यवसत् सुधीः}
{बालभावे विशालाक्षमतिविस्तृतवक्षसम्} %॥५८॥

\twolineshloka
{आजानुलम्बितभुजं सुविशालनिटालकम्}
{आरक्तोपान्तनयनविनिन्दितसरोरुहम्} %॥५९॥

\twolineshloka
{मुखकान्तिपराभूतराकाहिमकराकृतिम्}
{भासा गौर्या प्रसृतया प्रोद्यन्तमिव भास्करम्} %॥६०॥

\twolineshloka
{शङ्खचक्रध्वजाकाररेखाचिह्नपदाम्बुजम्}
{द्वात्रिंशल्लक्षणोपेतं विद्युदाभकलेवरम्} %॥६१॥

\twolineshloka
{प्रमोदं दृष्टमात्रेण दिशन्तं तं स्तनन्धयम्}
{पायम्पायं दृशा प्रेम्णा श्रीकृष्णमिव गोपिका} %॥६२॥


\threelineshloka
{प्रपेदे न क्षणं तृप्तिं चकोरीव सुधाकरम्}
{तादृशं बालकं दृष्ट्वा त्वार्याम्बा शुभलक्षणम्}
{तिष्ठति स्म सुखेनैव लालयन्ती तनूभवम्} %॥६३॥


॥इति श्रीचिद्विलासीयश्रीशङ्करविजयविलासे श्री\-शङ्कर\-भगवत्पादा\-चार्या\-णाम् अवतार\-घट्टः सम्पूर्णः॥