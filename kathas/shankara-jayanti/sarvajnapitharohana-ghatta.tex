\sect{काञ्च्यां सर्वज्ञपीठारोहण-घट्टः}

\dnsub{पञ्चविंशोऽध्यायः}
\addtocounter{shlokacount}{43}


\onelineshloka
{श्रीचक्रपश्चाद्भागे तु कामाक्षीं ज्ञानरूपिणीम्}% ॥४४॥

\twolineshloka
{प्रतिष्ठाप्य च पूजायै ब्राह्मणान् विनियुज्य च}
{एकाम्रेश्वरपूजार्थं विप्रानादिश्य भूयसः}% ॥४५॥

\twolineshloka
{श्रीमद्वरदराजस्य नमस्यायै नियुज्य च}
{सर्वज्ञपीठमारोढुमुत्सेहे देशिकोत्तमः}% ॥४६॥

\twolineshloka
{ततोऽशरीरिणी वाणी नभोमार्गाद् व्यजृम्भत}
{भो यतिन् भवता सर्वविद्यास्वपि विशेषतः}% ॥४७॥

\twolineshloka
{कृत्वा प्रसङ्गं विद्वद्भिः जित्वा तान् अखिलानपि}
{सर्वज्ञपीठमारोढुम् उचितं ननु भूतले}% ॥४८॥

\twolineshloka
{इति वाचं समाकर्ण्य किमेतदिति विस्मितः}
{किञ्चिदालोचयन्नास्त किं करोमीति मानसे}% ॥४९॥

\twolineshloka
{ताम्रपर्णीसरित्तीरवासिनो विबुधास्तदा}
{षड्दर्शिनीसुधावार्धिपारदृश्वगुणोन्नताः}% ॥५०॥

\twolineshloka
{आगत्य तं देशिकेन्द्रं प्रणिपत्येदमूचिरे}
{भिदा सत्यमिवाभाति त्वया त्वैक्यं निगद्यते}% ॥५१॥

\twolineshloka
{देवभेदो मूर्तिभेदः प्रत्यक्षेणात्र लक्ष्यते}
{स्वर्गादिफलभेदश्च सर्वशास्त्रविनिश्चितः}% ॥५२॥

\twolineshloka
{तत्प्रत्यक्षं च मिथ्येति कथयस्यधुना यते}
{इति ब्रुवत्सु विद्वत्सु शङ्कराचार्यदेशिकः}% ॥५३॥

\twolineshloka
{शृणुतात्रोत्तरं विप्राः ब्रह्मैकं तु सनातनम्}
{इन्द्रोपेन्द्रधनेन्द्राद्यास्तद्विभूतय एव हि}% ॥५४॥

\twolineshloka
{मृदि कुम्भो यथा भाति कनके कङ्कणं यथा}
{जले वीचिर्यथा भाति तथेदं च विभाव्यते}% ॥५५॥

\twolineshloka
{यां देवतां भजन्ते ये तत्सारूप्यं प्रयान्ति ते}
{ये वा पुण्यं चरन्तीह ते स्वर्गे फलभोगिनः}% ॥५६॥

\twolineshloka
{एको देव इति श्रुत्या जगत् सर्वं तदाकृतिः}
{तद्भिन्नमन्यन्नास्त्येव वेदान्तैकविनिश्चितम्}% ॥५७॥

\twolineshloka
{तस्मादखण्डमात्मानमद्वयानन्दलक्षणम्}
{ज्ञात्वा गुरुप्रसादेन मुक्ता भवत नान्यथा}% ॥५८॥

\twolineshloka
{श्रुतिस्मृतिपुराणोक्तैः वचनैरिति देशिकः}
{भेदवादरतान् विप्रान् आधायाद्वैतपारगान्}% ॥५९॥

\twolineshloka
{ततस्ततो विपश्चिद्भिः प्रणतश्चातिभक्तितः}
{गीतवादित्रनिर्घोषैः जयवादसमुज्ज्वलैः}% ॥६०॥

\twolineshloka
{आरुरोहाथ सर्वज्ञपीठं देशिकपुङ्गवः}
{पुष्पवृष्टिः पपाताथ ववुर्वाताः सुगन्धयः}% ॥६१॥

॥इति श्रीचिद्विलासीयश्रीशङ्करविजयविलासे श्री\-शङ्कर\-भगवत्पादा\-चार्या\-णां काञ्च्यां सर्वज्ञ\-पीठा\-रोहण\-घट्टः सम्पूर्णः॥ 
