\sect{श्री-शङ्कर-भगवत्पाद-प्रशस्ति-सङ्ग्रहः}

\newcommand{\creditline}[1]{\nobreak\hfill{—\normalsize #1}}

\section{देव-वन्दनम्}

\twolineshloka
{सदा बाल-रूपाऽपि विघ्नाद्रि-हन्त्री महा-दन्ति-वक्त्राऽपि पञ्चास्य-मान्या}
{विधीन्द्रादि-मृग्या गणेशाभिधा मे विधत्तां श्रियं काऽपि कल्याण-मूर्तिः}

\creditline{गणेशस्तुतिः - सुब्रह्मण्यभुजङ्गं शङ्करभगवत्पादकृतम् - 1}

\twolineshloka
{पुस्तक-जप-वट-हस्ते वरदाभय-चिह्न-चारु-बाहु-लते}
{कर्पूरामल-देहे वागीश्वरि शोधयाशु मम चेतः}

\creditline{प्रपञ्चसारः शङ्करभगवत्पादकृतः 8/70}

\section{गुरुपरम्परावन्दनम्}

\twolineshloka
{नारायणः पद्म-भवो वसिष्ठः शक्तिश्च तत्-पुत्र-पराशरश्च}
{व्यासः शुको गौड-पदो यतीन्द्रो गोविन्द-योगीति गुरु-क्रमोऽयम्}

\twolineshloka
{आद्यः श्री-शङ्कराचार्यो भगवत्पाद-संज्ञकः}
{अवतीर्णः शम्भुरिति प्रथितः कालटी-पदे}

\twolineshloka
{सुरेश्वरः पद्मपदो हस्तामलक-तोटकौ}
{सर्वज्ञश्चेति तच्छिष्याः प्रथिता गुरु-सन्निभाः}

\twolineshloka
{शङ्करः कामकोट्याख्यं पीठं काञ्च्यां व्यराजयत्}
{प्रत्यस्थापयदद्वैतं पीठे सर्वज्ञके स्थितः}

\twolineshloka
{आत्मानमनु सर्वज्ञं सुरेश्वर-मते स्थितम्}
{गोप्तारं कामकोट्याख्य-पीठस्य व्यदधाद् गुरुः}

\twolineshloka
{तदाद्येन्द्र-सरस्वत्याख्याऽविच्छिन्ना परम्परा}
{पाति नो गुरु-वर्याणां शारदा-मठ-सुस्थिता}

\twolineshloka
{श्री-शङ्करार्यमपरं श्री-शिवा-शिव-रूपिणम्}
{पूज्य-श्री-कामकोट्याख्य-पीठ-गं तं दया-निधिम्}

\twolineshloka
{अपार-करुणा-सिन्धुं ज्ञान-दं शान्त-रूपिणम्}
{श्री-चन्द्रशेखर-गुरुं प्रणमामि मुदाऽन्वहम्}

\twolineshloka
{देवे देहे च देशे च भक्त्यारोग्य-सुख-प्रदम्}
{बुध-पामर-सेव्यं तं श्री-जयेन्द्रं नमाम्यहम्}

\twolineshloka
{नमामः शङ्करान्वाख्य-विजयेन्द्र-सरस्वतीम्}
{श्री-गुरुं शिष्ट-मार्गानुनेतारं सन्मति-प्रदम्}

% ध्वनिमुद्रणम् 2

\section{गुरु-पादुका-पञ्चकम्}

\creditline{गोविन्द-भगवत्-पूज्यपाद-सन्निधौ शङ्कर-भगवत्पाद-कृतम्}

\twolineshloka
{जगज्जनि-स्थेम-लयालयाभ्याम् अगण्य-पुण्योदय-भाविताभ्याम्}
{त्रयी-शिरोजात-निवेदिताभ्यां नमो नमः श्री-गुरु-पादुकाभ्याम्}

\twolineshloka
{विपत्-तमः-स्तोम-विकर्तनाभ्यां विशिष्ट-सम्पत्ति-विवर्धनाभ्याम्}
{नमज्जनाशेष-विशेष-दाभ्यां नमो नमः श्री-गुरु-पादुकाभ्याम्}

\twolineshloka
{समस्त-दुस्तर्क-कलङ्क-पङ्कापनोदन-प्रौढ-जलाशयाभ्याम्}
{निराश्रयाभ्यां निखिलाश्रयाभ्यां नमो नमः श्री-गुरु-पादुकाभ्याम्}

\twolineshloka
{ताप-त्रयादित्य-करार्दितानां छाया-मयीभ्यामति-शीतलाभ्याम्}
{आपन्न-संरक्षण-दीक्षिताभ्यां नमो नमः श्री-गुरु-पादुकाभ्याम्}

\twolineshloka
{यतो गिरोऽप्राप्य धिया समस्ता ह्रिया निवृत्ताः सममेव नित्याः}
{ताभ्यामजेशाच्युत-भाविताभ्यां नमो नमः श्री-गुरु-पादुकाभ्याम्}

\twolineshloka
{ये पादुका-पञ्चकमादरेण पठन्ति नित्यं प्रयताः प्रभाते}
{तेषां गृहे नित्य-निवास-शीला श्री-देशिकेन्द्रस्य कटाक्ष-लक्ष्मीः}

% ध्वनिमुद्रणम् 3

\section{भगवत्पादकृतं गुरु-वन्दनम्}

\fourlineindentedshloka
{प्रज्ञा-वैशाख-वेध-क्षुभित-जल-निधेर्वेद-नाम्नोऽन्तर-स्थं}
{भूतान्यालोक्य मग्नान्यविरत-जनन-ग्राह-घोरे समुद्रे}
{कारुण्यादुद्दधारामृतमिदममरैर्दुर्लभं भूत-हेतोः}
{यस्तं पूज्याभिपूज्यं परम-गुरुममुं पाद-पातैर्नतोऽस्मि}

\fourlineindentedshloka
{यत्-प्रज्ञालोक-भासा प्रतिहतिमगमत् स्वान्त-मोहान्धकारो}
{मज्जोन्मज्जं च घोरे ह्यसकृदुपजनोदन्वति त्रासने मे}
{यत्-पादावाश्रितानां श्रुति-शम-विनय-प्राप्तिरग्र्या ह्यमोघा}
{तत्-पादौ पावनीयौ भव-भय-विनुदौ सर्व-भावैर्नमस्ये}

\creditline{माण्डूक्य-कारिका-भाष्यम्}

\twolineshloka
{यैरिमे गुरुभिः पूर्वं पद-वाक्य-प्रमाणतः}
{व्याख्याताः सर्व-वेदान्ताः तान् नित्यं प्रणतोऽस्म्यहम्}

\creditline{तैत्तिरीयोपनिषद्भाष्यम्}

\twolineshloka
{विमथ्य वेदोदधितः समुद्धृतं सुरैर्महाब्धेस्तु महात्मभिर्यथा}
{तथाऽमृतं ज्ञानमिदं हि यैः पुरा नमो गुरुभ्यः परमीक्षितं च यैः}

\creditline{उपदेशसाहस्र्याम्}

\twolineshloka
{सर्व-वेदान्त-सिद्धान्त-गोचरं तमगोचरम्}
{गोविन्दं परमानन्दं सद्गुरुं प्रणतोऽस्म्यहम्}

\twolineshloka
{अखण्डानन्द-सम्बोधो वन्दनाद् यस्य जायते}
{गोविन्दं तमहं वन्दे चिदानन्द-तनुं गुरुम्}

\fourlineindentedshloka
{नमो नमस्ते गुरवे महात्मने}
{विमुक्त-सङ्गाय सदुत्तमाय}
{नित्याद्वयानन्द-रस-स्वरूपिणे}
{भूम्ने सदाऽपार-दयाम्बु-धाम्ने}

\fourlineindentedshloka
{स्वाराज्य-साम्राज्य-विभूतिरेषा}
{भवत्-कृपा-श्री-महिम-प्रसादात्}
{प्राप्ता मया श्री-गुरवे महात्मने}
{नमो नमस्तेऽस्तु पुनर्नमोऽस्तु}

\fourlineindentedshloka
{स्वामिन् नमस्ते नत-लोक-बन्धो}
{कारुण्य-सिन्धो पतितं भवाब्धौ}
{मामुद्धरात्मीय-कटाक्ष-दृष्ट्या}
{ऋज्व्याऽतिकारुण्य-सुधाभिवृष्ट्या}

\creditline{विवेकचूडामणिः}

\fourlineindentedshloka
{वन्दे गुरूणां चरणारविन्दे}
{सन्दर्शित-स्वात्म-सुखावबोधे}
{जनस्य ये जाङ्गलिकायमाने}
{संसार-हालाहल-मोह-शान्त्यै}

\creditline{योगतारावलिः 1}

\twolineshloka
{श्री-गुरु-चरण-द्वन्द्वं वन्देऽहं मथित-दुस्सह-द्वन्द्वम्}
{भ्रान्ति-ग्रहोपशान्तिं पांसु-मयं यस्य भसितमातनुते}

\creditline{स्वात्मनिरूपणम् 1/1}

\section{वेदान्ताचार्यवन्दना}

\twolineshloka
{आदौ शिवस्ततो विष्णुः ततो ब्रह्मा ततः परम्}
{वसिष्ठश्च ततः शक्तिः ततः षष्ठः पराशरः}

\twolineshloka
{ततो व्यासः शुकः पश्चाद् गौडपादाभिधस्ततः}
{गोविन्दार्य-गुरुस्तस्माच्छङ्कराचार्य-संज्ञकः}

\twolineshloka
{पद्मपादः सुरेशश्च हस्तामलक-तोटकौ}
{वेदान्त-शिक्षा-गुरव आचार्याः पान्तु मां सदा}

\creditline{हुल्स्च्-कोशतः}

\twolineshloka
{सदाशिव-समारम्भां शङ्कराचार्य-मध्यमाम्}
{अस्मदाचार्य-पर्यन्तां वन्दे गुरु-परम्पराम्}

\fourlineindentedshloka
{नारायणं पद्म-भुवं वसिष्ठं}
{शक्तिञ्च तत्-पुत्र-पराशरं च}
{व्यासं शुकं गौड-पदं महान्तं}
{गोविन्द-योगीन्द्रमथास्य शिष्यम्}

\fourlineindentedshloka
{श्री-शङ्कराचार्यमथास्य पद्म-}
{पादं च हस्तामलकं च शिष्यम्}
{तं तोटकं वार्तिक-कारमन्यान्}
{अस्मद्-गुरून् सन्ततमानतोऽस्मि}

\creditline{साम्प्रदायिकश्लोकाः}

% ध्वनिमुद्रणम् 4

\section{मार्कण्डेय-संहितायां भगवत्पाद-प्रशंसा}

\twolineshloka
{श्री-शङ्कर-गुरु-चरण-स्मरणम् अभीष्टार्थ-करणमखिलानाम्}
{सम्भवतु सर्वदा मम सम-रस-सुख-भाग्य-दान-निपुणतरम्}

\twolineshloka
{श्री-शङ्कराचार्य-पदारविन्द-सेवा हि सर्वेप्सित-कल्प-वल्ली}
{लभ्येत जन्मान्तर-पुण्य-योगात् सुजन्मभिः शुद्ध-मनोभिषङ्गैः}

\twolineshloka
{शङ्कर-गुरु-चरणाम्बुजम् अखिल-जगन्मङ्गलं मनस्यनिशम्}
{कलयामि कलि-मलापहम् अमित-सुखाधायकं बुधेन्द्राणाम्}

\fourlineindentedshloka
{लोकानुग्रह-तत्परः पर-शिवः सम्प्रार्थितो ब्रह्मणा}
{चार्वाकादि-मत-प्रभेद-निपुणां बुद्धिं सदा धारयन्}
{कालट्याख्य-पुरोत्तमे शिव-गुरुर्विद्याधिनाथश्च यः}
{तत्-पत्न्यां शिव-तारके समुदितः श्री-शङ्कराख्यां वहन्}

\fourlineindentedshloka
{महात्रिपुरसुन्दरी-रमण-चन्द्रमौलीश्वर-}
{प्रसाद-परिलब्ध-वाङ्मय-विभूषिताशान्तरम्}
{निरन्तरमुपास्महे निरुपमात्म-विद्या-नदी-}
{नदी-नद-पति-प्रभं मनसि शङ्करार्यं गुरुम्}

% ध्वनिमुद्रणम् 5

\section{तोटकाष्टकम्}

\twolineshloka
{विदिताखिल-शास्त्र-सुधा-जलधे महितोपनिषत्-कथितार्थ-निधे}
{हृदये कलये विमलं चरणं भव शङ्कर देशिक मे शरणम्}

\twolineshloka
{करुणा-वरुणालय पालय मां भव-सागर-दुःख-विदून-हृदम्}
{रचयाखिल-दर्शन-तत्त्व-विदं भव शङ्कर देशिक मे शरणम्}

\twolineshloka
{भवता जनता सुहिता भविता निज-बोध-विचारण-चारु-मते}
{कलयेश्वर-जीव-विवेक-विदं भव शङ्कर देशिक मे शरणम्}

\twolineshloka
{भव एव भवानिति मे नितरां समजायत चेतसि कौतुकिता}
{मम वारय मोह-महा-जलधिं भव शङ्कर देशिक मे शरणम्}

\twolineshloka
{सुकृतेऽधिकृते बहुधा भवतो भविता सम-दर्शन-लालसता}
{अतिदीनमिमं परिपालय मां भव शङ्कर देशिक मे शरणम्}

\twolineshloka
{जगतीमवितुं कलिताकृतयो विचरन्ति महामहसश्छलतः}
{अहिमांशुरिवात्र विभासि पुरो* भव शङ्कर देशिक मे शरणम्}

\twolineshloka
{गुरु-पुङ्गव पुङ्गव-केतन ते समतामयतां न हि कोऽपि सुधीः}
{शरणागत-वत्सल तत्त्व-निधे भव शङ्कर देशिक मे शरणम्}

\twolineshloka
{विदिता न मया विशदैक-कला न च किञ्चन काञ्चनमस्ति गुरो}
{द्रुतमेव विधेहि कृपां सह-जां भव शङ्कर देशिक मे शरणम्}

{\normalsize[* गुरो इति पाठान्तरम्]}

% ध्वनिमुद्रणम् 6

\section{भगवत्पाद-शिष्यैः कृताः गुरु-स्तुतयः}

\fourlineindentedshloka
{येषां धी-सूर्य-दीप्त्या प्रतिहतिमगमन्नाशमेकान्ततो मे}
{ध्वान्तं स्वान्तस्य हेतुर्जनन-मरण-सन्तान-दोलाधिरूढेः}
{येषां पादौ प्रपन्नाः श्रुति-शम-विनयैर्भूषिताः शिष्य-सङ्घाः}
{सद्यो मुक्तौ स्थितास्तान् यति-वर-महितान् यावदायुर्नमामि}

\creditline{श्रुतिसारसमुद्धरणं तोटकाचार्यकृतम् 188}

\fourlineindentedshloka
{वेदान्तोदर-वर्ति भास्वदमलं ध्वान्त-च्छिदस्मद्-धियः}
{दिव्यं ज्ञानमतीन्द्रियेऽपि विषये व्याहन्यते न क्वचित्}
{यो नो न्याय-शलाकयैव निखिलं संसार-बीजं तमः}
{प्रोत्सार्याविरकार्षीद् गुरु-गुरुः पूज्याय तस्मै नमः}

\creditline{नैष्कर्म्यसिद्धिः - श्रीसुरेश्वराचार्यकृता 4.76-77}

\fourlineindentedshloka
{आ शैलादुदयात् तथाऽस्त-गिरितो भास्वद्-यशोराशिभिः}
{व्याप्तं विश्वमनन्धकारमभवद् यस्य स्म शिष्यैरिदम्}
{आराद् ज्ञान-गभस्तिभिः प्रतिहतश्चन्द्रायते भास्करः}
{तस्मै शङ्कर-भानवे तनु-मनोवाग्भिर्नमः स्यात् सदा}

\fourlineindentedshloka
{यत्-प्रज्ञोदधि-युक्ति-शब्दन-खज-श्रद्धैक-सन्नेत्रक-}
{स्थैर्य-स्तम्भ-मुमुक्षु-दुःखित-कृपा-यत्नोत्थ-बोधामृतम्}
{पीत्वा जन्म-मृति-प्रवाह-विधुरा मोक्षं ययुर्मोक्षिणः}
{तं वन्देऽत्रि-कुल-प्रसूतममलं वेधोभिधं मद्-गुरुम्}

\creditline{बृहदारण्यकभाष्यवार्त्तिकम् - सुरेश्वराचार्यकृतम्}

\twolineshloka
{नमाम्यभोगि-परिवार-सम्पदं निरस्त-भूतिमनुमार्ध-विग्रहम्}
{अनुग्रमुन्मृदित-काल-लाञ्छनं विना-विनायकमपूर्व-शङ्करम्}

\fourlineindentedshloka
{यद्-वक्त्र-मानस-सरः-प्रतिलब्ध-जन्म-}
{भाष्यारविन्द-मकरन्द-रसं पिबन्ति}
{प्रत्याशमुन्मुख-विनीत-विनेय-भृङ्गाः}
{तान् भाष्य-वित्तक-गुरून् प्रणमामि मूर्ध्ना}

\creditline{पञ्चपादिका पद्मपादाचार्यकृता}

\twolineshloka
{वक्तारमासाद्य यमेव नित्या सरस्वती स्वार्थ-समन्विताऽऽसीत्}
{निरस्त-दुस्तर्क-कलङ्क-पङ्का नमामि तं शङ्करमर्चिताङ्घ्रिम्}

\creditline{सङ्क्षेपशारीरकं श्रीसर्वज्ञात्मेन्द्रसरस्वतीश्रीचरणैः कृतम्}

% ध्वनिमुद्रणम् 7

\section{सदाशिवब्रह्मेन्द्रविरचितायां जगद्गुरुरत्नमालायां भगवत्पाद-चरितम्}

\twolineshloka
{यदबोध-वशादहं ममेदं तदिहेत्यादिरुदेति भूरि-भेदः}
{तदखण्डमनन्तमद्वितीयं परमानन्द-मयं पदं श्रयेयम्}

\twolineshloka
{कलिना बलिनाऽखिले खिलेऽपि स्खलिते श्रौत-पथेऽपथे प्रवृद्धे}
{जप-होम-तपस्सु नाम-शेषेष्वपि यातेषु सुभाषितेषु शोषम्}

\twolineshloka
{जगदीक्षण-विह्वलामृतान्धो-निगद-व्यक्त-कृपा-रसानुबन्धम्}
{प्रणिदिश्य गुहं पुरैव गन्तुं प्रणिबन्धुं च मखान् द्विषश्च यन्तुम्}

\twolineshloka
{अवतार्य सुरान् परांश्च पूर्वं विधि-विष्ण्विन्द्र-मुखान् विनोद-पूर्वम्}
{स्वयमप्यवतीर्य सुत्युरार्या-कमितुः श्री-शिव-शर्मणो विचार्य}

\twolineshloka
{उदभूत् सदने निटाल-दृग् यो मद-भाजां सुधियां प्रमाथ-योग्ये}
{शिशुरर्पयतान्मुमुक्षु-भाग्यं स शुभं शङ्कर-देशिकः सुभोग्यम्}

\twolineshloka
{प्रति-चन्द्र-भवं निवृत्ति-धर्मा श्रित-गोविन्द-मुनेरवाप्त-धर्मा}
{जयतात् कृत-सूत्र-भाष्य-कर्मा स्वयमन्ते-वसतां वितीर्ण-शर्मा}

\twolineshloka
{कुहनान्त्यज-विश्वनाथ-सृष्टो द्रुहिण-व्यास-वरोदितानुशिष्टः}
{ममतां मम तावदेष भिन्द्यान्नमतश्चोपरतिं ददात्वनिन्द्याम्}

\twolineshloka
{प्रविशन् बदरीमवाप्य सद्यः परमाचार्य-पदार्चनं क्रमाद् यः}
{धवलाचलमाप्य योऽप्यमाद्यच्छिव-लावण्यमुदीक्ष्य तं प्रपद्ये}

\twolineshloka
{प्रतिपादित-लिङ्ग-पञ्चकेऽमुं प्रणिवर्त्याशु तिरोहिते गिरीशे}
{विनिवृत्य स दिग्-जय-प्रवृत्तो विविधैः शिष्य-वरैर्विभातु चित्ते}

\twolineshloka
{अथ कान्यकुमार-सन्धि-सेतु-स्थलिनी-वैङ्कट-कालहस्ति-यातुः}
{यमि-नेतुरमुष्य काञ्चि-यात्रा शमिदानीं शम-दं क्रियाद् विचित्रा}

\twolineshloka
{श्रित-निर्मल-राजसेन-चोल-क्षिति-पालोद्धृत-विप्र-देव-शालः}
{वरदस्य तथाऽऽम्र-नायकस्याप्युरु-वेश्म-द्वय-कृज्जयाय मे स्यात्}

\twolineshloka
{प्रकृतिं च गुहाश्रयां महोग्रां स्व-कृते चक्र-वरे प्रवेश्य योऽग्रे}
{अकृताश्रित-सौम्य-मूर्तिमार्यां सुकृतं नः स चिनोतु शङ्करार्यः}

\twolineshloka
{उपयात्सु बुधेषु सर्व-दिग्भ्यः प्रदिशन्नाशु पराभवं य एभ्यः}
{विधृताखिल-वित्-पदश्च काञ्च्यामधृतार्तिः स दिशेच्छ्रियं च काञ्चित्}

\twolineshloka
{समतिष्ठिपदा-हिमाद्रि-सेव्यं क्रमशो धर्म-विचारणाय दिव्यम्}
{अधि-काञ्चि च शारदा-मठं योऽभ्यधिकं नः सुखमातनोतु सोऽयम्}

\twolineshloka
{परमन्तिक-सत्-सुरेश्वराद्यैः परमाद्वैत-मतं स्फुटं प्रवेद्य}
{परि-काञ्चिपुरं परे विलीनः परमायास्तु शिवाय सद्गुरुर्नः}

% ध्वनिमुद्रणम् 8

\section{कामकोटि-परम्परागतैः आचार्यैः कृताः स्तुतयः}

\twolineshloka
{नमस्तस्मै भगवते शङ्कराचार्य-रूपिणे}
{येन वेदान्त-विद्येयमुद्धृता वेद-सागरात्}

\creditline{विद्यारण्यमुनिविरचितायाम् अपरोक्षानुभूतिदीपिकायाम्}

\twolineshloka
{स्तुवन्मोह-तमः-स्तोम-भानु-भावमुपेयुषः}
{स्तुमस्तान् भगवत्पादान् भव-रोग-भिषग्-वरान्}

\creditline{सदाशिवेन्द्रसरस्वतीश्रीचरणैः कृता ब्रह्मसूत्रवृत्तिः}

\twolineshloka
{वेदान्तार्थाभिधानेन सर्वानुग्रह-कारिणम्}
{यति-रूप-धरं वन्दे शङ्करं लोक-शङ्करम्}

\creditline{नवपञ्चाशत्तमैः आचार्यैः श्रीभगवन्नामबोधेन्द्रसरस्वतीश्रीचरणैः कृतं हरिहराद्वैतभूषणम्}

\twolineshloka
{यमाश्रिता गिरां देवी नन्दयत्यात्म-संश्रितान्}
{तमाश्रये श्रिया जुष्टं शङ्करं करुणा-निधिम्}

\creditline{नवपञ्चाशत्तमैः आचार्यैः श्रीभगवन्नामबोधेन्द्रसरस्वतीश्रीचरणैः प्रणीतम् विवरणप्रमेयसङ्ग्रहात्मकम् अद्वैतभूषणम्}

\twolineshloka
{सर्व-तन्त्र-स्वतन्त्राय सदाऽऽत्माद्वैत-वेदिने}
{श्रीमते शङ्करार्याय वेदान्त-गुरवे नमः}

\twolineshloka
{अविप्लुत-ब्रह्मचर्यान् अन्वितेन्द्र-सरस्वतीन्}
{आत्त-मिथ्यावार-पथान् अद्वैताचार्य-सङ्कथान्}

\twolineshloka
{आ-सेतु-हिमवच्छैलं सदाचार-प्रवर्तकान्}
{जगद्-गुरून् स्तुमः काञ्ची-शारदा-मठ-संश्रयान्}

\creditline{पञ्चषष्टितमैः पीठाधिपतिभिः श्रीमत्सुदर्शनमहादेवेन्द्रसरस्वतीश्रीचरणैः प्रणीतः जगद्गुरु-परम्परा-स्तवः}

\twolineshloka
{गुरुर्नाम्ना महिम्ना च शङ्करो यो विराजते}
{तदीयाङ्घ्रि-गलद्-रेणु-गणायास्तु नमो मम}

\creditline{अष्टषष्टितमाचार्यैः श्रीचन्द्रशेखरेन्द्रसरस्वतीश्रीचरणैः प्रणीतम्}

\twolineshloka
{कामाक्षी-करुणा-रूपं कामकोटि-जगद्गुरुम्}
{चिन्मूर्तिं कलये चित्ते शङ्कराचार्यमव्ययम्}

\creditline{नवषष्टितमाचार्यैः श्रीजयेन्द्रसरस्वतीश्रीचरणैः प्रणीतम्}

\twolineshloka
{भजेऽहं भगवत्पादं भारतीय-शिखामणिम्}
{अद्वैत-मैत्री-सद्भाव-चेतनायाः प्रबोधकम्}

\creditline{सप्ततितमाचार्यैः श्रीशङ्करविजयेन्द्रसरस्वतीश्रीचरणैः प्रणीतम्}

% ध्वनिमुद्रणम् 9

\section{शङ्कर-चरित्र-ग्रन्थेषु}

\twolineshloka
{मेधावी निगम-पटुर्बहु-श्रुतो वा येनर्ते न कलयिता किलात्म-तत्त्वम्}
{तन्नेत्रं तमसि च दिव्य-दृष्टि-दायि श्रेयो नः प्रदिशतु धाम देशिकाख्यम्}

\creditline{पुण्यश्लोकमञ्जरी}

\twolineshloka
{नमामि शङ्कराचार्य-गुरु-पाद-सरोरुहम्}
{यस्य प्रसादान्मूढोऽपि सर्वज्ञोऽहं सदाऽस्म्यहम्}

\fourlineindentedshloka
{वेदे ब्रह्म-समस्तदङ्ग-निचये गर्गोपमस्तत्-कथा-}
{तात्पर्यार्थ-विवेचने गुरु-समस्तत्-कर्म-संवर्णने}
{आसीज्जैमिनिरेव तद्-वचन-ज-प्रोद्बोध-कन्दे समो}
{व्यासेनैव विभाति सद्गुरुरसौ श्री-शङ्कराख्यः क्षितौ}

\fourlineindentedshloka
{अद्वैतार्णव-पूर्ण-चन्द्रमभिदा-पद्माटवी-भास्करं}
{विद्वत्-कोटि-समर्चिताङ्घ्रि-युगलं प्रद्वेष-कक्षानलम्}
{हृद्याभेद्य-समस्त-वेद-जनित-प्रोद्यद्-विवेकाङ्कुरं}
{स्विद्यद्-वागमृतं परात्-पर-गुरुं श्री-शङ्करं तं भजे}

\creditline{आनन्दगिरीयशङ्करविजयः}

\fourlineindentedshloka
{गामाक्रम्य पदेऽधिकाञ्चि निबिडं स्कन्धैश्चतुर्भिस्तथा}
{व्यावृण्वन् भुवनान्तरं परिहरंस्तापं स-मोह-ज्वरम्}
{यः शाखी द्विज-संस्तुतः फलति तत् स्वाद्यं रसाख्यं फलं}
{तस्मै शङ्कर-पादपाय महते तन्मस्त्रि-सन्ध्यं नमः}

\creditline{व्यासाचलीयशङ्करविजयः}

\fourlineindentedshloka
{देशे कालडि-नाम्नि केरल-धरा-शोभा-करे सद्-द्विजे}
{जातः श्रीपति-मन्दिरस्य सविधे सर्वज्ञतां प्राप्तवान्}
{भूत्वा षोडश-वत्सरे यति-वरो गत्वा बदर्याश्रमं}
{कर्ता भाष्य-निबन्धनस्य सु-कविः श्री-शङ्करः पावनः}

\creditline{गोविन्दानन्दरचिते शङ्कराचार्यचरिते}

\fourlineindentedshloka
{अज्ञानान्तर्गहन-पतितान् आत्म-विद्योपदेशैः}
{त्रातुं लोकान् भव-दव-शिखा-ताप-पापच्यमानान्}
{मुक्त्वा मौनं वट-विटपिनो मूलतो निष्पतन्ती}
{शम्भोर्मूर्तिश्चरति भुवने शङ्कराचार्य-रूपा}

\creditline{नवकालिदासमाधवकविकृतः शङ्करदिग्विजयः}

\fourlineindentedshloka
{क्वेमे शङ्कर-सद्गुरोर्गुण-गणा दिग्-जाल-कूलङ्कषाः}
{कालोन्मीलित-मालती-परिमलावष्टम्भ-मुष्टिन्धयाः}
{क्वाहं हन्त तथापि सद्गुरु-कृपा-पीयूष-पारम्परी-}
{मग्नोन्मग्न-कटाक्ष-वीक्षण-बलादस्मि प्रशस्तोऽर्हताम्}

\creditline{सङ्क्षेपशङ्करविजये}

\fourlineindentedshloka
{श्रीमच्छङ्कर-सद्गुरोर्भगवतोऽगाधामसाधारणीं}
{वाणीं नः प्रतनीयसीं मुहुरिमां गाढुं समुत्कण्ठते}
{तन्मूर्तिः प्रभुरेव भक्त-जनता-वात्सल्य-वैपुल्य-भूः}
{अस्मै साधु ददातु शस्त-दयया हस्तावलम्बं हरः}

\creditline{वल्लीसहायकविकृतौ आचार्यदिग्विजये}

% ध्वनिमुद्रणम् 10

\section{अन्यैः वेदान्ताचार्यैः कृताः स्तुतयः}

\twolineshloka
{प्रचार्यं सर्व-लोकेषु सञ्चार्यं हृदयाम्बुजे}
{विचार्यं सर्व-वेदान्तैः आचार्यं शङ्करं भजे}

\creditline{नारायणीयोपनिषद्भाष्ये}

\twolineshloka
{भगवत्पाद-पादाब्ज-पांसवः सन्तु सन्ततम्}
{अपारासार-संसार-सागरोत्तार-सेतवः}

\creditline{चित्सुखाचार्याणां भाष्यभावप्रकाशिकायाम्}

\fourlineindentedshloka
{उद्धृत्य वेद-पयसः कमलामिवाब्धेः}
{आलिङ्गिताखिल-जगत्-प्रभवैक-मूर्तिम्}
{विद्यामशेष-जगतां सुख-दामदाद् यः}
{तं शङ्करं विमल-भाष्य-कृतं नमामि}

\creditline{विवरणाचार्याणां पञ्चपादिकाविवरणे}

\fourlineindentedshloka
{यद्-भाष्याम्बुज-जात-जात-मधुर-प्रेयोमधु-प्रार्थना-}
{सार्थ-व्यग्र-धियः समग्र-मरुतः स्वर्गेऽपि निर्वेदिनः}
{यस्मिन् मुक्ति-पथः पथीन-मुनिभिः सम्प्रार्थितः सम्बभौ}
{तस्मै भाष्य-कृते नमोऽस्तु भगवत्पादाभिधां बिभ्रते}

\creditline{आनन्दगिर्याचार्याणां सूत्रभाष्यव्याख्यायां - ६}

\twolineshloka
{श्री-गुरुं भगवत्पादं शरण्यं भक्त-वत्सलम्}
{शिवं शिव-करं शुद्धमप्रमेयं नमाम्यहम्}

\creditline{अद्वैतसभायाः ब्रह्मविद्यापत्रिकायां प्रकाशिते अज्ञातकर्तृके गुर्वष्टके}

\twolineshloka
{तं वन्दे शङ्कराचार्यं लोक-त्रितय-शङ्करम्}
{सत्-तर्क-नखरोद्गीर्ण-वावदूक-मतङ्गजम्}

\creditline{तत्त्वबोधभगवत्प्रणीते तत्त्वबोधे}

\twolineshloka
{आनन्द-घनमद्वन्द्वं निर्विकारं निरञ्जनम्}
{भजेऽहं भगवत्पादं भजतामभय-प्रदम्}

\creditline{मनीषा-पञ्चक-व्याख्याने}

\fourlineindentedshloka
{यद्-भाष्योक्तेर्लव-परिजुषश्छात्र-वर्गा महान्तः}
{निर्भिन्दन्ति प्रबल-मतयो वादि-शैलं समस्तम्}
{यैर्वेदाब्धेरमृतमिव सद्-भाष्यमापत् प्रकाशं}
{तत्-पादाब्जं स्फुरतु हृदये ह्युद्धृतं सर्वदा मे}

\creditline{मनीषापञ्चरत्नलघुविवरणे}

\fourlineindentedshloka
{महा-मोह-पङ्के विरिञ्चाचरान्तं}
{प्रजा-हस्तिनं मग्नमालोक्य भाष्यैः}
{जलैः क्षालयित्वाऽऽत्म-विद्या-दिवं यो}
{नयत्येकलं शङ्करं तं नमामि}

\creditline{ज्ञानामृतयतेः कृतौ विद्यासुरभिसंज्ञके नैष्कर्म्यसिद्धिविवरणे}

\fourlineindentedshloka
{संसार-सर्प-परिदष्ट-विनष्ट-जन्तु-}
{सञ्जीवनाय परया कृपयोपपन्नः}
{ब्रह्मावबोध-परमौषधमुद्वहन् यः}
{तं शङ्करं परतरं भिषजां भजामि}

\creditline{अद्वैतबोधामृतम्}

\fourlineindentedshloka
{वेदान्ताम्भोगभीरा नय-मकर-कुला ब्रह्म-विद्याब्ज-षण्डा}
{पाषण्डोत्तुङ्ग-वृक्ष-प्रमथन-निपुणा मान-वीची-तरङ्गा}
{यस्यास्योत्था सरस्वत्यखिल-भव-भय-ध्वंसिनी शङ्करस्य}
{गङ्गा शम्भोः कपर्दादिव निखिल-गुरोर्नौमि तत्-पाद-पद्मम्}

\creditline{ज्ञानघनपादानां तत्त्वशुद्धौ}

\fourlineindentedshloka
{सूत्र-प्रग्रह-वेद-वाजिनि महन्मीमांसक-स्यन्दने}
{तिष्ठन् भाष्य-पिनाकमुज्ज्वल-गुणं कृत्वाऽऽत्म-धी-सायकम्}
{आकृष्य प्रदहन्नशेष-विपदां मूलं पुराणां त्रयं}
{भूयान्नोऽभिनवः पुरारिरशुभस्योच्छित्तये शङ्करः}

\creditline{रामानन्दस्य ऋजु-विवरण-व्याख्यायाम्}

\fourlineindentedshloka
{यद्-भाष्य-सागर-ज-युक्ति-मणीन् प्रकीर्णान्}
{प्राप्याधुना कतिपयान् कवयो भवन्ति}
{तस्मै नमो जन-मनोब्ज-दिवाकराय}
{कृत्स्नागमार्थ-निलयाय यतीश्वराय}

\creditline{बोधनिधि-कृते उपदेश-प्रकरण-विवरणे}

\fourlineindentedshloka
{वेदान्तार्थं गभीरं ह्यति-सुगमतया बोधयामीति विष्णुः}
{व्यासात्माऽसूत्रयत् तद् दुरधिगममभूद् वादि-दुर्बुद्धि-भेदात्}
{भिन्दन् दुर्बुद्धि-भेदं य इह करुणयाऽभाष्ययद् भाष्यमेतत्}
{तं वन्दे सर्व-वन्द्यं त्रि-जगति भगवत्पाद-संज्ञं महेशम्}

\creditline{रामानन्दसरस्वतीकृतविवरणोपन्यासे}

\fourlineindentedshloka
{त्रि-वर्गेणाक्रान्ते जनन-मरणादि-व्रण-भुवा}
{जनेऽस्मिन् सर्वस्मिंस्तिमिर-परिणाहैक-शरणे}
{निषेक्तुं निध्यातोऽमृतमग-पतिः शङ्कर इति}
{स्व-नाम व्याख्यातुं जयति कुहना-भिक्षुरनिशम्}

\creditline{अभिनवद्राविडाचार्य-श्रीबालकृष्णानन्दसरस्वतीनां शारीरकमीमांसाभाष्यवार्तिके}

\fourlineindentedshloka
{श्री-सम्बन्धमुदीक्ष्य वाचक-पदे यान् शार्ङ्गिणं वैष्णवाः}
{चन्द्रोत्तंस-पदास्पदत्व-कलनाच्छम्भुं च शैवा विदुः}
{आनन्दाद्वय-शोभमान-परम-प्रेमास्पदं योगिनः}
{तान् पादाम्बुज-रेणु-धूत-तमसो वन्दे सदा श्री-गुरून्}

\creditline{गङ्गाधरसरस्वत्याख्यभिक्षुणा रचितायाम् आत्मसाम्राज्यसिद्धिव्याख्यायाम्}

\twolineshloka
{नमः श्री-शङ्कराचार्य-गुरवे शङ्करात्मने}
{शरीरिणां शङ्कराय शङ्कर-ज्ञान-हेतवे}

\creditline{नृसिंहाश्रमविरचितायां तत्वबोधिन्याख्यायां सङ्क्षेपशारीरकटीकायाम्}

\twolineshloka
{याऽनुभूतिः स्वयं-ज्योतिरादित्येशान-विग्रहा}
{शङ्कराख्या च तं नौमि सुरेश्वर-गुरुं परम्}

\creditline{नृसिंहप्रज्ञमुनिकृते बृहदारण्यकभाष्यवार्तिकन्यायतत्त्वविवरणे}

\twolineshloka
{संसाराब्धि-निषण्णाज्ञ-निकर-प्रोज्जिहीर्षया}
{कृत-संहननं वन्दे शङ्करं लोक-शङ्करम्}

\creditline{विज्ञानवासयतिरचितायां पञ्चपादिकाव्याख्यायाम्}

\twolineshloka
{वेदान्तार्थ-तदाभास-क्षीर-नीर-विवेकिनम्}
{नमामि भगवत्पादं पर-हंस-धुरन्धरम्}

\creditline{अमलानन्दसरस्वतीनां वेदान्तकल्पतरौ}

\fourlineindentedshloka
{नाना-भाष्यादृता सा सगुण-फल-गतिर्वैध-विद्या-विशेषैः}
{तत्-तद्-देशाप्ति-रम्या सरिदिव सकला यत्र यात्यंश-भूयम्}
{तस्मिन्नानन्द-सिन्धावतिमहति फले भाव-विश्रान्ति-मुद्रा}
{शास्त्रस्योद्घाटिता यैः प्रणमत हृदि तान् नित्यमाचार्य-पादान्}

\creditline{अप्पयदीक्षितानां न्यायरक्षामणौ}

\twolineshloka
{प्रचण्ड-पाखण्ड-विखण्डनोद्यतं त्रयी-शिरोर्थ-प्रतिपादने रतम्}
{बुधैर्नुतं योग-कलाभिरावृतं नमामि तं शङ्कर-देशिकं ततम्}

\creditline{सच्चिदानन्दसरस्वतीकृतायाम् आर्याव्याख्यायाम्}

\fourlineindentedshloka
{दृष्ट्वा यो दिव्य-दृष्टिः कलि-युग-समये “मन्द-भाग्या मनुष्याः}
{तस्मात् तन्त्र-प्रपञ्चः सुर-यजन-विधिर्मत्-कृतो निष्फलः स्यात्”}
{इत्याविर्भूय पृथ्व्यां पुनरपि कृतवांस्तन्त्र-सारं गिरीशः}
{तं वन्दे शङ्कराख्यं महिततम-मनः-प्रार्थनीयार्थ-रूपम्}

\creditline{प्रपञ्चसारसम्बन्धदीपिकायाम्}

\fourlineindentedshloka
{येनाद्वन्द्वमखण्डमक्षय-पदं प्रादर्शि तापापहं}
{भाष्य-ग्रन्थि-निबन्धनैः श्रुति-शिरोवाक्यार्थ-विद्योतिभिः}
{नित्यो यत्र समस्त-सद्-गुण-गणस्तं शङ्कराचार्य-गीर्}
{विख्यातं मुनि-मौलि-लालित-पद-द्वन्द्वं सदा संश्रये}

\creditline{रामतीर्थस्वामिरचितायाम् अन्वयार्थप्रकाशिकाख्यायां सङ्क्षेपशारीरकव्याख्यायाम्}

\fourlineindentedshloka
{वेदान्त-व्रात-नीरं शत-पथ-कथित-न्याय-रत्न-प्रपूरं}
{पारावारं सुतारं निगम-मुख-षडङ्गात्म-सद्-ग्राह-घोरम्}
{कारं-कारं सुगाहं श्रुत-मत-मथितैर्ब्रह्म-विद्यामृतं यः}
{प्रादादादाय तस्मादशरण-शरणं शङ्करं तं नमामः}

\creditline{आनन्दपूर्णरचितायां न्यायकल्पलतिकानाम्न्यां सुरेश्वरवार्तिकटीकायाम्}

\fourlineindentedshloka
{वेदान्तार्थ-विभासकाय गुरवे शान्ताय सन्न्यासिने}
{नाना-वादि-नगेन्द्र-सङ्घ-पवये योगीन्द्र-वन्द्याय च}
{मोह-ध्वान्त-दिवाकराय भगवत्पादाभिधां बिभ्रते}
{तस्मै भाष्य-कृते नमोऽस्तु सततं पूर्णाय बोधात्मने}

\creditline{तैत्तिरीयभाष्यटीकायाम्}

\fourlineindentedshloka
{ये वेदान्त-सुधोदधिं सुमनसां निःश्रेयसाय स्वयं}
{निर्मथ्योदहरन्निरूपण-गुणावृत्तेन चेतोमथा}
{अद्वैतामृतमासुरानुशयिनामास्वादनीयेतरत्}
{तानाऽऽस्माक-गुरोरुपैमि भगवत्पादादिमान् देशिकान्}

\creditline{कृष्णानन्दयतिकृतौ सिद्धान्तसिद्धाञ्जने}

\fourlineindentedshloka
{काले शिवः क्रम-वशात् कलि-दोष-दुष्टे}
{यः सम्प्रदाय-रहितं तदपेक्ष्य भूयः}
{क्षोण्यामवातरदशेष-जगद्धितार्थी}
{श्री-शङ्कराख्यममलं गुरुमाश्रये तम्}

\creditline{नारायणकृतौ प्रपञ्चसारार्थदीपे}

\fourlineindentedshloka
{वेदाद्यागम-दुग्ध-सिन्धु-मथनात् तन्मेय-मन्थाद्रिणा}
{दिव्याभोग-विचार-वासुकि-वशादाश्रित्य धैर्यं परम्}
{ब्रह्मोद्बोध-सुधां विधाय दयया मर्त्यानमर्त्यानमी}
{कुर्वन्तो गुरवो जयन्ति जगतां लक्ष्मीश-वद् रक्षकाः}

\creditline{वरदराजपण्डितकृतौ खण्डनमण्डने}

\fourlineindentedshloka
{यदीय-वाक्-सूर्य-रुचि-प्रणाशितः}
{हृदन्ध-कारो नमतामशेषतः}
{महात्मनः शिष्य-हिते सदा रतान्}
{नमामि तान् शङ्कर-पूज्य-देशिकान्}

\creditline{शङ्कुकविरचिते कैवल्यनवनीते}

\fourlineindentedshloka
{यद्-वक्त्राम्बुज-निस्सृतं परमकं श्री-सूत्र-भाष्यामृतं}
{पीत्वा मादृश-जीव-भङ्ग-निचया नन्दन्ति मोक्षाङ्गणे}
{नाना-वादि-मदेभ-भञ्जन-महा-व्यग्रोग्र-कण्ठीरवान्}
{वन्दे व्यास-मुनीन्द्र-शङ्कर-मुखान् सद्-देशिकांस्तानहम्}

\creditline{अमरेश्वरशास्त्रिरचिते अज्ञानध्वान्तचण्डभास्करे}

\fourlineindentedshloka
{यो लोकोपकृति-प्रविष्ट-हृदयो जित्वाऽतिबाह्यं मतं}
{श्रीमच्छङ्कर-शब्द-पूर्व-भगवत्पादाभिधानं गतम्}
{सद्-वेदान्त-रहस्य-वत् स्फुटितवान् गोप्यं रहो-मानवं}
{तं वन्दे भगवन्तमन्तक-रिपुं सर्वान्तराय-च्छिदम्}

\creditline{कामेश्वरसूरिकृतायाम् अरुणामोदिनीनाम्न्यां सौन्दर्यलहरीव्याख्यायाम्}

\twolineshloka
{अखिल-पर-हंस-देशिकमागम-गूढार्थ-दर्शकं प्राज्ञम्}
{स्वानन्द-पूर्ण-सागरमनिशमहं नौमि शङ्कराचार्यम्}

\creditline{नागनाथरचिते आत्मबोधप्रकरणे}

\twolineshloka
{विष्णवे व्यास-रूपाय ब्रह्म-सूत्र-कृते नमः}
{महेशाय च तद्-भाष्य-कृते शङ्कर-रूपिणे}

\creditline{अल्लालसूरिरचिते भामतीतिलके}

\twolineshloka
{हर-लीलावताराय शङ्कराय वरौजसे}
{कैवल्य-कलना-कल्प-तरवे गुरवे नमः}

\creditline{उमामहेश्वररचितायां तत्त्वचन्द्रिकायाम्}

\fourlineindentedshloka
{यत्-पादाब्ज-प्रभव-विमल-श्री-परागालि-भास्वान्}
{मत्-स्वान्त-स्थं प्रणुदति तमः-पुञ्जमत्यन्त-चण्डम्}
{यत्-कारुण्य-प्लव-परिजुषा तारितोऽनेन तूर्णं}
{संसाराब्धिः प्रणतिरनिशं स्याद् गुरूणां पदाब्जे}

\creditline{सीतारामसूरिरचिते वेदान्तकौस्तुमे}

\fourlineindentedshloka
{पाराशर्य-वचोविलास-मसृणैः सूत्रैः क्रमेणाततैः}
{अत्यस्तैः प्रकटीचकार भगवान् यो भाष्य-संज्ञं पटम्}
{अज्ञानोद्भव-जाड्य-नाश-करणं स्वानन्द-दं सेविनां}
{तं वन्देऽखिल-योगि-वन्द्य-चरणं श्री-शङ्करं शं-करम्}

\creditline{कमलाकरदेवकृतौ आनन्दविलासे}

\fourlineindentedshloka
{योऽयं दैवत-सार्वभौम-विभवो विश्वाधिको रुद्र इ}
{त्याद्यैराद्य-वचोभिरद्वयपरैरद्यापि संस्तूयते}
{अद्वैतात्म-विबोधनाय विदुषामिच्छा-समङ्गीकृत-}
{श्रीमच्छङ्कर-देशिकेन्द्र-वपुषं श्री-शङ्करं भावये}

\creditline{शङ्कराचार्याष्टके}

\section{शङ्कर-वाङ्महिमा}

% \fourlineindentedshloka
% {मेधा-याथात्म्य-सिद्ध्यै भवतु भगवती दग्ध-जीवच्छरीर-}
% {म्लेच्छालापाहि-मूर्च्छा-निमिषदुपनिषज्जीव-जीवातुरुक्तिः}
% {अद्वैत-ब्रह्म-राद्धान्तिक-सकल-पुराणेतिहासोपकॢप्ताम्}
% {आप्ता साम्राज्य-पीठीमनघ-जन-मनश्शङ्करी शाङ्करी नः}

% \creditline{वेङ्कटनाथस्य}

\fourlineindentedshloka
{अधिगत-भिदा पूर्वाचार्यानुपेत्य सहस्र-धा}
{सरिदिव मही-भागान् सम्प्राप्य शौरि-पदोद्गता}
{जयति भगवत्पाद-श्रीमन्मुखाम्बुज-निर्गता}
{जनन-हरणी सूक्तिर्ब्रह्माद्वयैक-परायणा}

\creditline{अप्पय्यदीक्षितानां सिद्धान्तलेशसङ्ग्रहे}

\fourlineindentedshloka
{संसाराध्वनि ताप-भानु-किरण-प्रोद्भूत-दाह-व्यथा-}
{खिन्नानां जल-काङ्क्षया मरु-भुवि श्रान्त्या परिभ्राम्यताम्}
{अत्यासन्न-सुखाम्बुधिं सुख-करं ब्रह्माद्वयं दर्शय-}
{न्त्येषा शङ्कर-भारती विजयते निर्वाण-सन्दायिनी}

\creditline{विवेकचूडामणौ}


\newcommand{\jayavijayibhava}{\hfill (स्वामिन् ! जय ! विजयी भव !)}

\section{जय-घोषः}

\twolineshloka
{श्री-शङ्कराचार्य-वर्य ब्रह्म-ज्ञान-प्रदायक}
{अज्ञान-तिमिरादित्य सुज्ञानाब्धि-सुधाकर}
\twolineshloka
{ज्ञान-मुद्राञ्चित-कर शिष्य-हृत्-ताप-हारक}
{कम्र-मुक्ति-गृह-द्वार-कवाट-घ्न-पदाम्बुज}
\jayavijayibhava

\twolineshloka
{षण्मत-स्थापनाचार्य त्रयी-मार्ग-प्रकाशक}
{प्रसन्न-वदनाम्भोज परमार्थ-प्रकाशक}
\twolineshloka
{ज्ञानात्मकैक-दण्डाढ्य कमण्डलु-लसत्-कर}
{काषाय-वसनोपेत भस्मोद्धूलित-विग्रह}
\jayavijayibhava

\twolineshloka
{श्रीमत्-कैलास-निलय-सच्छिवांशावतारक}
{कालटी-क्षेत्र-निवसदार्याम्बा-गर्भ-संश्रित}
\twolineshloka
{शिवादि-गुरु-वंशाम्बुनिधि-राकेश-सन्निभ}
{पितृ-दत्तान्वर्थ-भूत-शङ्कराख्या-समुज्ज्वल}
\jayavijayibhava

\twolineshloka
{अभ्यस्त-वेद-वेदाङ्ग निखिलागम-पारग}
{दरिद्र-ब्राह्मणी-दत्त-भिक्षामलक-तोषित}
\twolineshloka
{स्वर्णामलक-सद्वृष्टि-प्रसादानन्दित-द्विज}
{अष्ट-वर्ष-चतुर्-वेदिन् द्वादशाखिल-शास्त्र-ग}
\jayavijayibhava

\twolineshloka
{सरिद्-वर्त्मातप-श्रान्त-मातृ-दुःखापनोदक}
{नक्र-ग्रह-व्याज-मातृ-मत-पारमहंस्यक}
\twolineshloka
{चिन्तना-मात्र-सान्निध्य-करणाश्वासिताम्बक}
{सोमोद्भवा-तटी-कॢप्त-सौम्य-गोविन्द-सेवन}
\jayavijayibhava

\twolineshloka
{गोविन्दार्य-मुखावाप्त-महावाक्य-चतुष्टय}
{योग-सिद्धि-गृहीतेन्दुभवा-पूर-कमण्डलो}
\twolineshloka
{गुर्वनुज्ञात-विश्वेश-दिदृक्षा-गमनोत्सुक}
{चण्डालाकार-विश्वेश-प्रश्नानुप्रश्न-हर्षित}
\jayavijayibhava

\twolineshloka
{विश्वेशानुग्रहावाप्त-भाष्य-ग्रथन-नैपुण}
{भाष्य-स्फुट-श्रुतिशिरो-मत-तत्त्वाभिलापक}
\twolineshloka
{यदूद्वह-प्रोक्त-गीता-याथातथ्य-विवेचक}
{ब्रह्मसूत्रार्थ-संवाद-हृष्यत्-सत्यवती-सुत}
\jayavijayibhava

\twolineshloka
{मन्दाकिनी-झरी-रम्य-भाष्य-पावित-भूतल}
{भाष्य-सार-प्रकरण-कृत-जिज्ञासु-तोषण}
\twolineshloka
{सौन्दर्य-लहरी-मुख्य-बहु-स्तोत्र-विधायक}
{योगजाग्नि-कृत-स्वाम्बा-यज्ञ-स्थापित-सत्पथ}
\jayavijayibhava

\twolineshloka
{त्रिरधीतात्मीय-भाष्य-सनन्दन-समाश्रय}
{कुकूलानल-कूट-स्थ-कुमारिल-कृतानते}
\twolineshloka
{कर्मैक-पथिकोद्दण्ड-मण्डनान्त्याश्रम-प्रद}
{कञ्ज-योन्यवतार-श्री-सुरेश्वर-सुदेशिक}
\jayavijayibhava

\twolineshloka
{यथावत्-तत्त्व-विज्ञातृ-हस्तामलक-सद्गुरो}
{तोटकाभिव्यक्त-भक्ति-तत्त्व-ज्ञानाढ्य-शिष्यक}
\twolineshloka
{पृथ्वीधवादि-शिष्यौघ-शिरोधृत-पद-द्वय}
{शारदा-स्थापना-पूत-ऋश्यशृङ्ग-गिरि-स्थल}
\jayavijayibhava

\twolineshloka
{ककुब्जय-महायात्रा-पवित्रित-महीतल}
{रामेश्वरादि-मेर्वन्त-प्रतिष्ठापित-सन्मत}
\twolineshloka
{अद्वैत-स्थापनाचार्य भगवत्पाद-संज्ञक}
{वेद-वेदान्त-सम्प्रोक्त-रक्षार्थ-मठ-कल्पन}
\jayavijayibhava

\twolineshloka
{कैलास-यात्रा-सम्प्राप्त-चन्द्रमौलि-प्रपूजक}
{नेपाल-केदार-वर-सिद्धि-लिङ्ग-निधायक}
\twolineshloka
{चिदम्बर-सभा-न्यस्त-मोक्ष-लिङ्ग यतीश्वर}
{तुङ्गा-भद्रा-सङ्ग-भूमि-भोग-लिङ्ग-समर्चन}
\jayavijayibhava

\twolineshloka
{श्रीचक्रात्मक-ताटङ्क-पोषिताम्बा-मनोरथ}
{काञ्च्यां श्रीचक्र-राजाख्य-यन्त्र-स्थापन-दीक्षित}
\twolineshloka
{भेरी-पटह-वाद्यादि-राज-लक्षण-लक्षित}
{सर्वज्ञ-पीठाध्यारोह-लुप्त-सार्वज्ञ्य-संशय}
\jayavijayibhava

\twolineshloka
{वादार्थागत-सर्वज्ञ-बाल-सन्न्यास-दायक}
{शारदा-मठ-मेरु-श्री-योगलिङ्गाभिषेचन}
\twolineshloka
{सोपान-पञ्चकोद्घोष-कृत-शिष्यानुशासन}
{सत्यव्रत-समाख्यात-काञ्च्यन्तरित-विग्रह}
\jayavijayibhava

\twolineshloka
{काञ्चीपुराभरण-कामद-कामकोटि-पीठाभिषिक्त वर-देशिक-सार्वभौम}
{सार्वज्ञ्य-शक्त्यधिगताखिल-मन्त्र-तन्त्र-चक्र-प्रतिष्ठिति-विजृम्भित-चातुरीक}
\jayavijayibhava


\closesection


\centering

\vfill
\fbox{\parbox[]{0.8\linewidth}{\normalsize
  अयं पद्यसङ्ग्रहः

* श्रीमच्चन्द्रशेखरेन्द्रसरस्वतीश्रीपादानां षष्ट्यब्दपूर्त्यवसरे प्रकाशिते ब्रह्मसूत्रभाष्यपुस्तके

* श्रीशङ्करभक्तजनसभया प्रकाशिते अद्वैताक्षरमालिकायाः द्वितीयसंस्करणपुस्तके

* शिमिऴि-वेङ्कट-राधाकृष्णशास्त्रिभिः सङ्कलितायां श्रीशङ्करभगवत्पादप्रशस्तिमञ्जर्यां च

सङ्गृहीतानि आचार्यप्रशस्तिरूपाणि पद्यानि आधृत्य सङ्कलितः
}}
