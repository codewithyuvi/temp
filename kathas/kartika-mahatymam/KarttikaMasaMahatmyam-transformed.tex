\begingroup
\centering
\newcommand{\iti}[3]{\addtocounter{totalshlokas}{\value{shlokacount}}
{\centerline{\normalsize आदितः श्लोकाः — \devanumber{\value{totalshlokas}}}
॥इति श्रीस्कान्दे महापुराण एकाशीतिसाहस्र्यां संहितायां\\ द्वितीये वैष्णवखण्डे कार्तिकमासमाहात्म्ये\\#1\\ #2॥#3॥

\closesub
}
}

\chapt{कार्तिकमासमाहात्म्यम्}

\sect{अथ प्रथमोऽध्यायः}
\twolineshloka
{नारायणं नमस्कृत्य नरं चैव नरोत्तमम्}
{देवीं सरस्वतीं चैव ततो जयमुदीरयेत्} %१


\uvacha{ऋषय ऊचुः}
\twolineshloka
{सूत नः कथितं पुण्यं माहात्म्यमाश्विनस्य च}
{भूयोऽन्यच्छ्रोतुमिच्छामः कार्तिकस्य च वैभवम्} %२

\twolineshloka
{कलौ कलुषचित्तानां नराणां पापकर्मणाम्}
{संसाराब्धौ निमग्नानामनायासेन का गतिः} %३

\twolineshloka
{को धर्मः सर्वधर्माणामधिको मोक्षसाधकः}
{इहापि मुक्तिदो नृणामेतत्त्वं कथय प्रभो} %४


\uvacha{सूत उवाच}
\twolineshloka
{भवद्भिर्यदहं पृष्टस्तदेतत्पृष्टवान्मुनिः}
{नारदो ब्रह्मणः पुत्रो ब्रह्माणं तु जगद्गुरुम्} %५

\twolineshloka
{तथैव सत्यभामा च श्रीकृष्णं जगदीश्वरम्}
{अपृच्छत्कार्तिकस्यैव वैभवं श्रवणोत्सुका} %६

\twolineshloka
{वालखिल्यैश्च ऋषिभिर्यदुक्तमृषिसंसदि}
{श्रीसूर्यारुणसंवादरूपेणातिमनोहरम्} %७

\twolineshloka
{कैलासे शङ्करेणैव कार्तिकस्य च वैभवम्}
{वर्णितं षण्मुखस्याग्रे नानाख्यानसमन्वितम्} %८

\twolineshloka
{पृथुं प्रति नारदेन कथितं च महात्म्यकम्}
{कार्तिकस्य च विप्रेन्द्रा श्रुत्वा ब्रह्ममुखात्पुरा} %९

\twolineshloka
{एकदा नारदो योगी सत्यलोकमुपागतः}
{पप्रच्छ विनयेनैव सर्वलोकपितामहम्} %१०


\uvacha{श्रीनारद उवाच}
\twolineshloka
{पापेन्धनस्य घोरस्य शुष्कार्द्रस्य च भूरिशः}
{को वह्निर्दहते ब्रह्मंस्तद्भवान्वक्तुमर्हति} %११

\twolineshloka
{नाज्ञातं त्रिषु लोकेषु ब्रह्माण्डान्तर्गतस्य यत्}
{विद्यते तव देवेश त्रिविधस्य सुनिश्चितम्} %१२

\twolineshloka
{मासानां प्रवरो मासो देवानामुत्तमोत्तमः}
{तीर्थानि तद्विशेषेण कथयस्व पितामह} %१३


\uvacha{ब्रह्मोवाच}
\twolineshloka
{मासानां कार्तिकः श्रेष्ठो देवानां मधुसूदनः}
{तीर्थं नारायणाख्यं हि त्रितयं दुर्लभं कलौ} %१४


\uvacha{नारद उवाच}
\twolineshloka
{भगवंस्तव दासोऽस्मि भक्तोऽस्मि हरिवल्लभ}
{वैष्णवान्ब्रूहि मे धर्मान्सर्वज्ञोऽसि पितामह} %१५

\twolineshloka
{आदौ कार्तिकमाहात्म्यं वक्तुमर्हसि मे प्रभो}
{दीपदानस्य माहात्म्यं व्रतिनां नियमांस्तथा} %१६

\threelineshloka
{गोपीचन्दनमाहात्म्यं तुलस्याश्च तथा विभो}
{धात्र्याश्चैव च माहात्म्यं विधिं स्नानादिकस्य च}
{व्रतारम्भः कदा कार्यं उद्यापनविधिं तथा} %१७

\twolineshloka
{यत्किञ्चिद्वैष्णवं धर्मं तत्सर्वं वक्तुमर्हसि}
{येनाहं त्वत्प्रसादेन पदं यास्याम्यनामयम्} %१८


\uvacha{सूत उवाच}
\twolineshloka
{इति पुत्रवचः श्रुत्वा ब्रह्मा हर्षसमन्वितः}
{राधादामोदरं स्मृत्वा प्रोवाच तनुजं प्रति} %१९


\uvacha{ब्रह्मोवाच}
\twolineshloka
{साधु पृष्टं त्वया पुत्र लोकोद्धरणहेतवे}
{कथयामि न सन्देहः कार्तिकस्य च वैभवम्} %२०

\twolineshloka
{एकतः सर्वतीर्थानि सर्वे यज्ञाः सदक्षिणाः}
{कार्तिकस्य तु मासस्य कलां नार्हन्ति षोडशीम्} %२१

\twolineshloka
{एकतः पुष्करे वासः कुरुक्षेत्रे हिमालये}
{एकतः कार्तिकः पुत्र सर्वपुण्याधिको मतः} %२२

\twolineshloka
{स्वर्णानि मेरुतुल्यानि सर्वदानानि चैकतः}
{एकतः कार्तिको वत्स सर्वदा केशवप्रियः} %२३

\twolineshloka
{यत्किञ्चित्क्रियते पुण्यं विष्णुमुद्दिश्य कार्तिके}
{तस्य क्षयं न पश्यामि मयोक्तं तव नारद} %२४

\twolineshloka
{सोपानभूतं स्वर्गस्य मानुष्यं प्राप्य दुर्लभम्}
{तथात्मानं समादद्यान्न भ्रश्येत यथा पुनः} %२५

\twolineshloka
{दुष्प्राप्यं प्राप्य मानुष्यं कार्तिकोक्तं चरेन्न यः}
{धर्मं धर्मभृतां श्रेष्ठ स मातापितृघातकः} %२६

\twolineshloka
{कार्तिकः खलु वै मासः सर्वमासेषु चोत्तमः}
{पुण्यानां परमं पुण्यं पावनानां च पावनम्} %२७

\twolineshloka
{अस्मिन्मासे त्रयस्त्रिंशद्देवाः सन्निहिता मुने}
{अत्र स्नानानिदानानि भोजनानि व्रतानि च} %२८

\twolineshloka
{तिलधेनुं हिरण्यं च रजतं भूमिवाससी}
{गोप्रदानानि कुर्वन्ति सर्वभावेन नारद} %२९

\twolineshloka
{तानि दानानि दत्तानि गृह्णन्ति विधिवत्सुराः}
{यत्किञ्च दत्तं विप्रेन्द्र तपश्चैव तथा कृतम्} %३०

\twolineshloka
{तदक्षय्यफलं प्रोक्तं विष्णुना प्रभविष्णुना}
{पापानां मोक्षणं चैव कार्तिके मासि शस्यते} %३१

\twolineshloka
{तस्माद्यत्नेन विप्रेन्द्र कार्तिके मासि दीयते}
{यत्किञ्चित्कार्तिके दत्तं विष्णुमुद्दिश्य मानवैः} %३२

\twolineshloka
{तदक्षयं हि लभते अन्नदानं विशेषतः}
{यथा नदीनां विप्रेन्द्र शैलानां चैव नारद} %३३

\twolineshloka
{उदधीनां च विप्रर्षे क्षयो नैवोपपद्यते}
{दानं कार्तिकमासे तु यत्किञ्चिद्दीयते मुने} %३४

\twolineshloka
{न तस्यास्ति क्षयो विप्र पापं याति सहस्रधा}
{सम्प्राप्तं कार्तिकं दृष्ट्वा परान्नं यस्तु वर्जयेत्} %३५

\twolineshloka
{दिनेदिनेऽतिकृच्छ्रस्य फलं प्राप्नोत्ययत्नतः}
{न कार्तिकसमो मासो न कृतेन समं युगम्} %३६

\twolineshloka
{न वेदसदृशं शास्त्रं न तीर्थं गङ्गया समम्}
{न चान्नसदृशं दानं न सुखं भार्यया समम्} %३७

\twolineshloka
{न्यायेनोपार्जितं द्रव्यं दुर्लभं दानकारिणाम्}
{दुर्लभं मर्त्यधर्माणां तीर्थे च प्रतिपादनम्} %३८

\twolineshloka
{कार्तिके मुनिशार्दूल शालिग्रामशिलार्चनम्}
{स्मरणं वासुदेवस्य कर्तव्यं पापभीरुणा} %३९

\twolineshloka
{एतादृशं कार्तिकं च अकृतेनैव यो नयेत्}
{पूर्वं कृतस्य पुण्यस्य क्षयमाप्नोत्यसंशयम्} %४०


\uvacha{नारद उवाच}
\twolineshloka
{अशक्तेन कथं कार्यं कार्तिकव्रतमुत्तमम्}
{येन तत्फलमाप्नोति तन्मे वद पितामह} %४१


\uvacha{ब्रह्मोवाच}
\twolineshloka
{अशक्तस्तु यदा मर्त्यस्तदैवं व्रतमाचरेत्}
{अन्यस्मै द्रविणं दत्त्वा कारयेत्कार्तिकव्रतम्} %४२

\twolineshloka
{तस्मात्पुण्यं प्रगृह्णीत दानसङ्कल्पपूर्वकम्}
{द्रव्यदानेऽप्यशक्तश्चेद्यदा देवर्षिसत्तम} %४३

\twolineshloka
{तदा तेन प्रकर्तव्यं पानं तीर्थजलस्य च}
{तत्राप्यशक्तो यो मर्त्यस्तेन नित्यं हरेर्मुदा} %४४

\twolineshloka
{स्मरणं च प्रकर्तव्यं नाम्ना नियमपूर्वकम्}
{अखण्डितं तदा तेन कार्तिकव्रतजं फलम्} %४५

\twolineshloka
{विष्णोः शिवस्य वा कुर्यादालये हरिजागरम्}
{शिवविष्ण्वोर्गृहाभावे सर्वदेवालयेष्वपि} %४६

\twolineshloka
{दुर्गाटव्यां स्थितो वाऽथ यदि वाऽऽपद्गतो भवेत्}
{कुर्यादश्वत्थमूले तु तुलसीनां वनेष्वपि} %४७

\twolineshloka
{विष्णुनामप्रबन्धानां गायनं विष्णुसन्निधौ}
{गोसहस्रप्रदानस्य फलमाप्नोति मानवः} %४८

\twolineshloka
{वाद्यकृत्पुरुषश्चापि वाजपेयफलं लभेत्}
{सर्वतीर्थावगाहोत्थं नर्तकः फलमाप्नुयात्} %४९

\twolineshloka
{सर्वमेतल्लभेत्पुण्यमेतेषां द्रव्यदः पुमान्}
{श्रवणाद्दर्शनाद्वापि षडंशं फलमाप्नुयात्} %५०

\twolineshloka
{आपद्गतो यदाप्यम्भो न लभेत्कुत्रचिन्नरः}
{व्याधितो वाथवा कुर्याद्विष्णोर्नाम्नाऽपि मार्जनम्} %५१

\twolineshloka
{उद्यापनविधिं कर्तुमशक्तो यो व्रतस्थितः}
{ब्राह्मणान्भोजयेत्पश्चाद्व्रतसम्पूर्तिहेतवे} %५२

\twolineshloka
{अशक्तो दीपदानाय परदीपं प्रबोधयेत्}
{तस्य वा रक्षणं कुर्याद्वातादिभ्यः प्रयत्नतः} %५३

\threelineshloka
{श्रीविष्णोः पूजनाभावे तुलसीधात्रिपूजनम्}
{सर्वाभावे व्रती कुर्याद्ब्राह्मणानां गवामपि}
{तस्याप्यभावे मनसि विष्णोर्नामानुकीर्तनम्} %५४


\uvacha{नारद उवाच}

\onelineshloka
{ब्रह्मन्ब्रूहि विशेषेण धर्मान्कार्तिकसम्भवान्} %५५


\iti{कार्तिकव्रतप्रशंसावर्णनं}{नाम प्रथमोऽध्यायः}{१}

\sect{अथ द्वितीयोऽध्यायः}


\uvacha{ब्रह्मोवाच}
\twolineshloka
{अथ कार्तिकमासस्य धर्मान्वक्ष्यामि नारद}
{सम्प्राप्तं कार्तिकं दृष्ट्वा परान्नं यस्तु वर्जयेत्} %१

\threelineshloka
{स तु मोक्षमवाप्नोति नात्र कार्या विचारणा}
{सर्वेषामेव धर्माणां गुरुपूजा परा मता}
{गुरुशुश्रूषया सर्वं प्राप्नोति ऋषिसत्तम} %२

\twolineshloka
{गुरौ तुष्टे च तुष्टाः स्युर्देवाः सर्वे सवासवा}
{गुरौ रुष्टे च रुष्टाः स्युर्देवाः सर्वे सवासवाः} %३


\twolineshloka
{कार्तिकं मासि सम्प्राप्ते कृत्वा कर्माणि भूरिशः}
{अकृत्वा गुरुशुश्रूषां नरकानेव विन्दति} %४

\onelineshloka
{यत्किञ्चिद्वा समादिष्टो गुरुणा तत्समाचरेत्} %५

\twolineshloka
{आज्ञप्तो गुरुणा विप्र न तद्वाक्यं तु लङ्घयेत्}
{यदि दुःखादिकं प्राप्तं गुरुं तु शरणं व्रजेत्} %६

\twolineshloka
{मातृत्वे च पितृत्वे च गुरुमेव स्मरेद्बुधः}
{गुरौ न प्राप्यते यत्तन्नान्यत्रापि हि लभ्यते} %७

\threelineshloka
{गुरुप्रसादात्सर्वं तु प्राप्नोत्येव न संशयः}
{मेधावी कपिलश्चैव सुमतिश्च महातपाः}
{गौतमस्य गुरोः सम्यक्सेवयाऽमरतां गताः} %८

\twolineshloka
{तस्मात्सर्वप्रयत्नेन कार्तिके विष्णुतत्परः}
{गुरुसेवां प्रकुर्वीत ततो मोक्षमवाप्नुयात्} %९

\twolineshloka
{नरेभ्यो वैष्णवं धर्मं यो ददाति द्विजोत्तमः}
{ससागरमहीदाने तत्पुण्यं लभते हि सः} %१०

\twolineshloka
{तिलधेनुं हिरण्यं च रजतं भूमिवाससी}
{गोप्रदानानि दास्यन्ति सर्वभावेन सुव्रत} %११

\twolineshloka
{सर्वेषामेव दानानां कन्यादानं विशिष्यते}
{सहस्रमेव धेनूनां शतं चानडुहां समम्} %१२

\twolineshloka
{दशानडुत्समं यानं दशयानसमो हयः}
{हयदान सहस्रेभ्यो गजदानं विशिष्यते} %१३

\twolineshloka
{गजदानसहस्राणां स्वर्णदानं च तत्समम्}
{स्वर्णदानसहस्राणां विद्यादानं च तत्समम्} %१४

\twolineshloka
{विद्यादानात्कोटिगुणं भूमिदानं विशिष्यते}
{भूमिदानसहस्रेण गोप्रदानं विशिष्यते} %१५

\twolineshloka
{गोप्रदानसहस्रेभ्यो ह्यन्नदानं विशिष्यते}
{अन्नाधारमिदं प्रोक्तं तस्माद्देयं तु कार्तिके} %१६

\twolineshloka
{परान्नवर्जनादेव लभेच्चान्द्रायणं फलम्}
{दिनेदिनेऽतिकृच्छ्रस्य फलं प्राप्नोति मानवः} %१७

\twolineshloka
{कार्तिके वर्जयेन्मांसं सन्धानं च विशेषतः}
{राक्षसीं योनिमाप्नोति सकृन्मांसस्य भक्षणात्} %१८

\twolineshloka
{प्रवृत्तानां तु भक्ष्याणां कार्तिके नियमे कृते}
{अवश्यं विष्णुरूपत्वं प्राप्यते मोक्षदं पदम्} %१९

\twolineshloka
{ब्राह्मणेभ्यो महीं दत्त्वा ग्रहणे सूर्यचन्द्रयोः}
{यत्फलं लभते वत्स तत्फलं भूमिशायिनः} %२०

\twolineshloka
{भोजनं द्विजदम्पत्योः पूजनं च विलेपनैः}
{कम्बलानि च रत्नानि वासांसि विविधानि च} %२१

\twolineshloka
{तूलिकाश्च प्रदातव्याः प्रच्छादनपटैः सह}
{उपानहावातपत्रं कार्तिके देहि सुव्रत} %२२

\twolineshloka
{कार्तिके क्षितिशायी च हन्यात्पापं युगार्जितम्}
{जागरं कार्तिके मासि यः करोत्यरुणोदये} %२३

\twolineshloka
{दामोदराग्रे देवर्षे गोसहस्रफलं लभेत्}
{नदीस्नानं कथा विष्णोर्वैष्णवानां च दर्शनम्} %२४

\twolineshloka
{न भवेत्कार्तिके यस्य हरेत्पुण्यं दशाब्दिकम्}
{पुष्करं यः स्मरेत्प्राज्ञः कर्मणा मनसा गिरा} %२५

\twolineshloka
{कार्तिके मुनिशार्दूल लक्षकोटिगुणं भवेत्}
{प्रयागो माघमासे तु पुष्करं कार्तिके तथा} %२६

\twolineshloka
{अवन्ती माधवे मासि हन्यात्पापं युगार्जितम्}
{धन्यास्ते मानवा लोके कलिकाले विशेषतः} %२७

\twolineshloka
{ये कुर्वन्ति नरा नित्यं प्रीत्यर्थं हरिपूजनम्}
{तारितास्तैश्च पितरो नरकाच्च न संशयः} %२८

\twolineshloka
{क्षीरादिस्नपनं विष्णोः क्रियते पितृकारणात्}
{कल्पकोटिं दिवं प्राप्य वसन्ति त्रिदिवैः सह} %२९

\twolineshloka
{कार्तिके नार्चितो यैस्तु कृष्णस्तु कमलेक्षणः}
{जन्मकोटिषु विप्रेन्द्र न तेषां कमला गृहे} %३०

\twolineshloka
{अहो मुष्टा विनष्टास्ते पतिताः कलिकन्दरे}
{यैर्नार्चितो हरिर्भक्त्या कमलैरसितैः सितैः} %३१

\threelineshloka
{पद्मेनैकेन देवेशं योऽर्चयेत्कमलापतिम्}
{वर्षायुतसहस्रस्य पापस्य कुरुते क्षयम्}
{पुष्करार्चनयोगेन श्वेतो मुक्तिमवाप ह} %३२

\twolineshloka
{अपराधसहस्राणि तथा सप्तशतानि च}
{पद्मेनैकेन देवेशः क्षमते प्रणतोऽर्चितः} %३३

\twolineshloka
{तुलसीपत्रलक्षेण कार्तिके योऽर्चयेद्धरिम्}
{पत्रेपत्रे मुनिश्रेष्ठ मौक्तिकं लभते फलम्} %३४

\threelineshloka
{मुखे शिरसि देहे तु कृष्णोत्तीर्णां तु यो वहेत्}
{तुलसी कृष्णनिर्माल्यैर्यो गात्रं परिमार्जयेत्}
{सर्वरोगैस्तथा पापैर्मुक्तो भवति मानवः} %३५

\twolineshloka
{शङ्खोदकं हरेर्भक्तिर्निर्माल्यं पादयोर्जलम्}
{चन्दनं धूपशेषं च ब्रह्महत्यापहारकम्} %३६

\twolineshloka
{कार्तिके मासि विप्रेन्द्र प्रातःस्नानपरायणः}
{विप्रेभ्यश्चान्नदानं तु कुर्याच्छक्त्यनुसारतः} %३७

\twolineshloka
{सर्वेषामेव दानानामन्नदानं विशिष्यते}
{अन्नेन जायते लोको ह्यन्नेनैवाभिवर्द्धते} %३८

\twolineshloka
{अन्नं हि सर्वभूतानां प्राणभूतं परं विदुः}
{अन्नदः सर्वदो लोके सर्वयज्ञादिकृद्भवेत्} %३९

\twolineshloka
{तीर्थस्नानेन किं तस्य देवयात्रादिनाऽपि किम्}
{सर्वं सम्पाद्यते ब्रह्मन्नन्नदानान्न संशयः} %४०

\twolineshloka
{सत्यकेतुर्द्विजः पूर्वं चान्नदानेन केवलम्}
{सर्वपुण्यफलं प्राप्य मोक्षं प्राप सुदुर्लभम्} %४१

\twolineshloka
{कार्तिकव्रतनिष्ठस्तु कुर्याद्गोदानमुत्तमम्}
{व्रतं सम्पूर्णतां याति गोदानेन न संशयः} %४२

\twolineshloka
{गोदानात्परमं दानं संसारार्णवतारकम्}
{नास्ति नारद लोकेऽस्मिन्सुशर्मा ब्राह्मणो यथा} %४३

\twolineshloka
{कार्तिके मासि विप्रेन्द्र दत्त्वा दानान्यनेकशः}
{हरिस्मृतिविहीनश्चेन्न पुनन्ति कदाचन} %४४

\twolineshloka
{नामस्मरणमाहात्म्यं मया वक्तुं न शक्यते}
{पुष्करेण यथा पूर्वं नारकीयाश्च मोचिताः} %४५

\fourlineindentedshloka
{गोविन्द गोविन्द हरे मुरारे}
{गोविन्द गोविन्द् मुकुन्द कृष्ण}
{गोविन्द गोविन्द रथाङ्गपाणे}
{गोविन्द दामोदर माधवेति} %४६

\twolineshloka
{श्लोकार्द्धं श्लोकपादं वा नित्यं भागवतोद्भवम्}
{कार्तिके यः पठन्मर्त्यः श्रद्धाभक्तिसमन्वितः} %४७

\fourlineindentedshloka
{यैर्न श्रुतं भागवतं पुराणं}
{नाराधितो वै पुरुषः पुराणः}
{हुतं मुखे नैव धरामराणां}
{तेषां वृथा जन्म गतं नराणाम्} %४८

\twolineshloka
{कार्तिके मासि विप्रेन्द्र यस्तु गीतां पठेन्नरः}
{तस्य पुण्यफलं वक्तुं मम शक्तिर्न विद्यते} %४९

\twolineshloka
{गीतायास्तु समं शास्त्रं न भूतं न भविष्यति}
{सर्वपापहरा नित्यं गीतैका मोक्षदायिनी} %५०

\twolineshloka
{एकेनाध्यायपाठेन सर्वपापकृतोऽपि च}
{मुच्यन्ते नरकाद्घोराज्जडो वै ब्राह्मणो यथा} %५१

\twolineshloka
{शालिग्रामशिलादानं यः कुर्यात्कार्तिके मुने}
{तस्य पुण्यस्य विश्रान्तिर्विष्णुना न निरूपिता} %५२

\twolineshloka
{शालिग्रामं समभ्यर्च्य श्रोत्रियाय महामुने}
{दानं यः कुरुते विप्र तस्य पुण्यफलं शृणु} %५३

\twolineshloka
{सप्तसागरपर्यन्तं भूदानाद्यत्फलं भवेत्}
{शालिग्रामशिलादानात्तत्फलं समवाप्नुयात्} %५४

\twolineshloka
{शालिग्रामशिलादानात्कार्तिके ब्राह्मणी यथा}
{विधवा सधवा जाता विवाहे पञ्चमेऽहनि} %५५

\twolineshloka
{तस्मात्तु कार्तिके मासि स्नानदानपुरःसरम्}
{शालिग्रामशिलादानं कर्तव्यं नात्र संशयः} %५६


\iti{कार्तिकव्रतधर्मनिरूपणं}{नाम द्वितीयोऽध्यायः}{२}

\sect{अथ तृतीयोऽध्यायः}


\uvacha{ब्रह्मोवाच}
\twolineshloka
{भूयः शृणुष्व विप्रेन्द्र कार्तिकस्य च वैभवम्}
{दशमीदिनमारभ्य दशम्यां तु समापयेत्} %१

\twolineshloka
{पौर्णमासीं समारभ्य पौर्णमास्यां समापयेत्}
{आश्विनस्य हरिदिनीं समारभ्य तु भक्तिमान्} %२

\twolineshloka
{दामोदरं नमस्कृत्य कुर्यात्सङ्कल्पमादितः}
{दामोदर नमस्तेऽस्तु सर्वपापविनाशन} %३

\twolineshloka
{कार्तिकस्य व्रतं कर्तुमनुज्ञां दातुमर्हसि}
{निर्विघ्नं कुरु देवेश आमासं पुरुषोत्तम} %४

\threelineshloka
{इति सम्प्रार्थ्य विधिना कार्तिकव्रतमाचरेत्}
{अनूरुं वदता प्रोक्तं भास्करेण श्रुतं मया}
{कलौ च स्वर्गगमनकारणं श्रूयतां हि तत्} %५


\uvacha{सूर्य उवाच}

\onelineshloka
{द्वादशानां तु मासानां मार्गशीर्षोऽतिपुण्यदः} %६

\twolineshloka
{तस्मात्पुण्यफलः प्रोक्तो वैशाखो नर्मदातटे}
{ततो लक्षगुणः प्रोक्तः प्रयागे माघमासकः} %७

\twolineshloka
{तस्मान्महाफलः प्रोक्तः कार्तिको जलमात्रके}
{एकतः सर्वदानानि व्रतानि नियमास्तथा} %८

\twolineshloka
{एकतः कार्तिकस्नानं ब्रह्मणा तुलया धृतम्}
{सन्ततिश्चैव सम्पत्तिः कलौ येषां प्रजायते} %९

\twolineshloka
{अवश्यं तैः कृतं विद्धि कार्तिकस्नानमादरात्}
{स्नानं च दीपदानं च तुलसीवनपालनम्} %१०

\twolineshloka
{भूमिशय्या ब्रह्मचर्य्यं तथा द्विदलवर्जनम्}
{विष्णुसङ्कीर्तनं सत्यं पुराणश्रवणं तथा} %११

\twolineshloka
{कार्तिके मासि कुर्वन्ति जीवन्मुक्तास्त एव हि}
{न कार्तिकसमं धर्म्यमर्थ्यं नो कार्तिकात्परम्} %१२

\twolineshloka
{न कार्तिकसमं काम्यं मोक्षदानं न कार्तिकात्}
{युधिष्ठिरेण धर्मार्थमर्थार्थं च ध्रुवेण च} %१३

\twolineshloka
{श्रीकृष्णेन तु कामार्थं मोक्षार्थं नारदेन च}
{कृतमेतद्व्रतं तस्माच्छ्रेष्ठं कृष्णप्रियं च हि} %१४

\uvacha{अरुण उवाच}

\twolineshloka
{ब्रूहि भास्कर सर्वात्मन् कदाऽऽरभ्य व्रतं कृतम्}
{सफलं जायते सम्यक् का च पूज्याऽत्र देवता} %१५

\uvacha{भास्कर उवाच}

\twolineshloka
{अहं विष्णुश्च शर्वश्च देवी विघ्नेश्वरस्तथा}
{एकोऽहं पञ्चधा जातो नाट्ये सूत्रधरो यथा} %१६

\twolineshloka
{अस्माकं सर्व एवैते भेदा विद्धि खगेश्वर}
{तस्मात्सौरैश्च गाणेशैः शाक्तैः शैवैश्च वैष्णवैः} %१७

\twolineshloka
{कर्तव्यं कार्तिकस्नानं सर्वपापापनुत्तये}
{सूर्यस्य प्रीतये कार्यं तुलासंस्थे दिवाकरे} %१८

\twolineshloka
{इषपूर्णां समारभ्य यावत्कार्तिकपूर्णिमा}
{तावत्स्नानं विधातव्यं शिवसन्तुष्टये नरैः} %१९

\twolineshloka
{देवीपक्षं समारभ्य महारात्रिचतुर्दशी}
{तावत्स्नानं विधातव्यं देवी सम्प्रीयतामिति} %२०

\twolineshloka
{गणपक्षं समारभ्य कृष्णा या कार्तिके भवेत्}
{चतुर्थी तावदेव स्यात्स्नानं गणपतुष्टये} %२१

\threelineshloka
{एकादशीं समारभ्य आश्विनस्यासितेतराम्}
{एकादश्यां कार्तिकस्य शुक्लायां परिपूर्यते}
{कृतं येन तु तस्य स्यात्परितुष्टो जनार्दनः} %२२

\twolineshloka
{न कार्तिकसमो मासो न काशीसदृशी पुरी}
{न प्रयागसमं तीर्थं न देवः केशवात्परः} %२३

\twolineshloka
{प्रसङ्गाद्वा बलात्कारैर्ज्ञात्वाज्ञात्वा कृतं भवेत्}
{स्नानं कार्तिकमासस्य न पश्येद्यमयातनाम्} %२४

\twolineshloka
{स्नानार्थं चेन्न सामर्थ्यं दत्वान्यस्मै धनादिकम्}
{स्नातस्य तस्य हस्तस्य ग्रहणा पुण्यभाग्भवेत्} %२५

\twolineshloka
{अथवा कार्तिकस्नानं ये कुर्वन्ति द्विजातयः}
{तेषां प्रावरणं दत्त्वा स्नानजं फलमाप्नुयात्} %२६


\onelineshloka
{राधादामोदरः पूज्यः कार्तिके तु विशेषतः} %२७

\twolineshloka
{स्वर्णस्य वाथ रौप्यस्याप्यभावे शुल्बजामपि}
{मृज्जां वा चित्रजातां वाऽथ वा पिष्टविचित्रिताम्} %२८

\twolineshloka
{दामोदरस्य राधायास्तुलस्यधोऽर्चयन्ति ये}
{मूर्तिं ते तु नरा ज्ञेया जीवन्मुक्ता न संशयः} %२९

\twolineshloka
{अपि पापसहस्राढ्यः कार्तिकस्नानतो नरः}
{मुक्तोऽवश्यं स भवति नात्र कार्या विचारणा} %३०

\twolineshloka
{तुलस्यभावे कर्तव्या पूजा धात्रीतले खग}
{मुख्यपूजाविधानं तु कर्तव्यं सूर्यमण्डले} %३१

\twolineshloka
{अप्रत्यक्षाः सर्वदेवाः प्रत्यक्षो भगवानयम्}
{सर्वे देवाः कालवशाः कालकालो दिवाकरः} %३२

\twolineshloka
{एतदाराधनेऽशक्तः प्रतिमां पूजयेन्नरः}
{प्रतिमातोऽधिकं पुण्यं ब्राह्मणस्य तु पूजने} %३३

\twolineshloka
{दरिद्रो दानपात्रं स्याद्विद्यावांस्तु विशेषतः}
{विप्राभावे पूजनीया गावः कृष्णा मनोहराः} %३४

\threelineshloka
{विष्णोर्मूर्तिर्जङ्गमतः स्थावरा तु प्रशस्यते}
{शूद्रस्थापितमूर्तीनां नमस्कारं करोति यः}
{पितृभिर्निरयं याति दशपूर्वैर्दशापरैः} %३५


\onelineshloka
{शूद्रार्चितस्य संस्पर्शाद्दहेदासप्तमं कुलम्} %३६

\twolineshloka
{तस्माद्विचार्य्य विप्रैर्या स्थापिता तां समर्चयेत्}
{ततोऽपि या देवताभिः कृता सा भुक्तिमुक्तिदा} %३७

\twolineshloka
{मूर्त्यभावे पूजनीयोऽश्वत्थो वाऽथ वटोऽथ वा}
{अश्वत्थरूपी विष्णुः स्याद्वटरूपी शिवो यतः} %३८

\twolineshloka
{कार्तिके तुलसीशाकं ताम्बूलं वा नराधमः}
{अज्ञानाज्ज्ञानतो वाऽपि भुञ्जानो निरयं व्रजेत्} %३९

\twolineshloka
{शालिग्रामशिलाचक्रे नित्यं सन्निहितो हरिः}
{तस्मात्सर्वप्रयत्नेन शालिग्रामं प्रपूजयेत्} %४०

\twolineshloka
{रुद्रशापवशाद्गावो विष्ठाभक्षणतत्पराः}
{तथाऽपि ताः पूजनीया लोकद्वयफलप्रदाः} %४१

\twolineshloka
{ब्रह्मांशकसमुद्भूते पालाशे यस्तु भोजनम्}
{कुर्यात्कार्तिकमासेऽसौ विष्णुलोकं प्रयास्यति} %४२

\twolineshloka
{अश्वत्थरूपी भगवान्वटरूपी सदाशिवः}
{तस्मात्सर्वप्रयत्नेन कार्तिकेऽश्वत्थमर्चयेत्} %४३

\twolineshloka
{या नारी कार्तिके मासि लक्षं कुर्यात्प्रदक्षिणाः}
{राधादामोदरं पूज्य मन्दवारे च तत्तले} %४४

\twolineshloka
{दम्पती भोजयेद्राधादामोदरस्वरूपिणौ}
{भोजयित्वा सपत्नीकान्पश्चाद्भुञ्जीत वाग्यतः} %४५

\twolineshloka
{वन्ध्याऽपि लभते पुत्रमितरासां तु का कथा}
{सदा सन्निहितो विष्णुर्द्विपत्सु ब्राह्मणे यथा} %४६

\twolineshloka
{बोधिद्रुमे पादपेषु शालिग्रामे शिलासु च}
{तस्मादश्वत्थमूले वै कर्तव्यं विष्णुपूजनम्} %४७

\twolineshloka
{अश्वत्थपूजा स्पर्शेन कर्त्तव्या शनिवासरे}
{अन्यवारेऽश्वत्थसङ्गाद्दरिद्रो जायते नरः} %४८

\twolineshloka
{स्नानं जागरणं दीपं तुलसीवनपालनम्}
{कार्तिके मासि कुर्वन्ति ते नरा विष्णुमूर्तयः} %४९

\twolineshloka
{सम्मार्जनं विष्णुगृहे स्वस्तिकादिनिवेदनम्}
{विष्णोः पूजां च ये कुर्युर्जीवन्मुक्तास्तु ते नराः} %५०

\twolineshloka
{स्नानकालं प्रवक्ष्यामि तीर्थादिषु च यत्फलम्}
{स्नानधर्माश्च ये केचित्तान्सर्वान्मे निबोधत} %५१


\iti{कार्तिकवैभववर्णनं}{नाम तृतीयोऽध्यायः}{३}

\sect{अथ चतुर्थोऽध्यायः}



\uvacha{ब्रह्मोवाच}
\twolineshloka
{नाडीद्वयावशिष्टायां रात्र्यां गच्छेज्जलाशयम्}
{तुलसीमृत्तिकायुक्तः सवस्त्रकलशो मुने} %१

\twolineshloka
{आगत्य तोयनिकटे तीरे संस्थाप्य पात्रकम्}
{पादप्रक्षालनं कृत्वा देशकालादि चोच्चरेत्} %२

\twolineshloka
{स्मरेद्गङ्गादिका नद्यो विष्णुशर्वादिदेवताः}
{नाभिमात्रे जले स्थित्वा मन्त्रमेतमुदीरयेत्} %३

\twolineshloka
{कार्तिकेऽहं करिष्यामि प्रातःस्नानं जनार्दन}
{प्रीत्यर्थं तव देवेश दामोदर मया सह} %४

\twolineshloka
{नित्ये नैमित्तिके कृत्वा कार्तिके पापनाशन}
{स्नानं चार्घं प्रदास्यामि निर्विघ्नं कुरु केशव} %५

\twolineshloka
{तीर्थादिदेवताभ्यश्च क्रमादर्घ्यादि दापयेत्}
{गृहाणार्घ्यं मया दत्तं राधया सहितो हरे} %६

\twolineshloka
{नमः कमलनाभाय नमस्ते जलशायिने}
{नमस्तेऽस्तु हृषीकेश गृहाणार्घ्यं नमोऽस्तु ते} %७

\twolineshloka
{व्रतिनः कार्तिके मासि स्नातस्य विधिवन्मम}
{गृहाणार्घ्यं मया दत्तं दनुजेन्द्रनिषूदन} %८

\twolineshloka
{किरणा धूतपापा च पुण्यतोया सरस्वती}
{गङ्गा च यमुना चैव पञ्चनद्यः पुनन्तु माम्} %९

\twolineshloka
{अन्यासां च नदीनां च दद्यादर्घ्यं यथाविधि}
{जाह्नवीस्मरणं कुर्यात्सर्वतीर्थेषु मानवः} %१०

\twolineshloka
{नान्यत्तीर्थं तु जाह्नव्यां स्मरणीयं कदाचन}
{एतान्मन्त्रान्समुच्चार्य मलस्नानं समाचरेत्} %११

\twolineshloka
{मृत्स्नानं च पितृस्नानं गुरुस्नानं ततः परम्}
{ततस्तु पावमानीभिरभिषिञ्चेत्स्वमस्तकम्} %१२

\twolineshloka
{अघमर्षणकं कृत्वा स्नानाङ्गं तर्पणं तथा}
{ततः पुरुषसूक्तेन जलं शिरसि सिञ्चयेत्} %१३

\twolineshloka
{ततस्तु बहिरागत्य तीर्थं शिरसि निक्षिपेत्}
{तीर्थं पीत्वा त्रिवारं तु तुलसीं गृह्य पाणिना} %१४

\twolineshloka
{ततो जलाद्विनिष्क्रम्य चाञ्चलं पीडयेद्बहिः}
{यन्मया दूषितं तोयं शारीरमलसञ्चयैः} %१५

\twolineshloka
{तद्दोषपरिहारार्थं यक्ष्मणं तर्पयाम्यहम्}
{वस्त्रनिष्पीडनं कृत्वा कुर्याच्च तिलकादिकम्} %१६


\uvacha{सूत उवाच}
\twolineshloka
{शृणुध्वमृषयः सर्वे कार्तिकस्नानजं फलम्}
{अरुणं प्रति सूर्येण यदुक्तं च सविस्तरम्} %१७


\uvacha{अरुण उवाच}
\twolineshloka
{कस्मिंस्तीर्थे विशेषेण फलं कार्तिकसम्भवम्}
{क्षेत्रे वा एतदाऽऽख्याहि भगवन्स्नानयोगतः} %१८


\uvacha{सूर्य उवाच}
\twolineshloka
{यत्र कुत्रापि कर्तव्यं जले स्नानं तु कार्तिकम्}
{उष्णोदकेन कर्तव्यं स्नानं कुत्रापि कार्तिके} %१९

\twolineshloka
{ततो दशगुणं पुण्यं शीततोयनिमज्जनात्}
{ततः शतगुणं पुण्यं बहिः कूपोदके कृतम्} %२०

\twolineshloka
{कूपात्सहस्रगुणितं फलं वापीनिषेकतः}
{ततोऽयुतगुणं पुण्यं तडागस्नानतो भवेत्} %२१

\twolineshloka
{ततो दशगुणं पुण्यं निर्झरेषु निमज्जनात्}
{ततोऽधिकतरं पुण्यं नदीस्नानस्य कार्तिके} %२२

\twolineshloka
{नद्या दशगुणं प्रोक्तं तीर्थस्नानं खगोत्तम}
{ततो दशगुणं पुण्यं नद्योर्यत्र च सङ्गमः} %२३

\twolineshloka
{नदीत्रयस्य संयोगे पुण्यस्यान्तो न विद्यते}
{सिन्धुः कृष्णा च वेणी च यमुना च सरस्वती} %२४

\twolineshloka
{गोदावरी विपाशा च नर्मदा तमसा मही}
{कावेरी शरयूः शिप्रा तथा चर्मण्वती नदी} %२५

\twolineshloka
{वितस्ता वेदिका शोणो वेत्रवत्यपराजिता}
{गण्डकी गोमती पूर्णा ब्रह्मपुत्रा सरोवरम्} %२६

\twolineshloka
{वाग्मती च शतद्रुश्च तथा बदरिकाश्रमः}
{दुर्लभाः कार्तिके त्वेते तीर्थान्यथ निबोध मे} %२७

\twolineshloka
{सर्वेभ्यश्च स्थलेभ्यश्च आर्यावर्तं तु पुण्यदम्}
{कोल्हापुरी ततः श्रेष्ठा ततः काञ्चीद्वयं स्मृतम्} %२८

\twolineshloka
{अनन्तसेनवसतिर्वराहक्षेत्रमेव च}
{चक्रक्षेत्रं ततः पुण्यं मुक्तिक्षेत्रं ततोऽधिकम्} %२९

\twolineshloka
{अवन्तिका ततः श्रेष्ठा ततो बदरिकाश्रमः}
{अयोध्या च ततः श्रेष्ठा गङ्गाद्वारं ततोऽधिकम्} %३०

\twolineshloka
{ततः कनखलं तीर्थं ततो मधुपुरी वरा}
{एकोऽपि कार्तिको मासो मथुरायमुनाजले} %३१

\twolineshloka
{यैः स्नातस्ते तु वैकुण्ठे बहुकालं वसन्ति हि}
{राधादामोदरस्तत्र स्वयं स्नातस्तु कार्तिके} %३२


\onelineshloka
{अतो मधुपुरी श्रेष्ठा यमुना च विशेषतः} %३३

\twolineshloka
{द्वारावती ततः श्रेष्ठा प्रत्यहं स्नाति केशवः}
{षोडशस्त्रीसहस्रेण सार्द्धं यादवसंयुतः} %३४

\threelineshloka
{द्वारकायां मृत्तिकायास्तिलको येन मस्तके}
{धार्यतेऽसौ नरो ज्ञेयो जीवन्मुक्तो न संशयः}
{द्वारकास्नानमाहात्म्यं न वक्तुं शक्यते मया} %३५

\twolineshloka
{गोविन्दार्पितचित्तानां जायते पुण्यभास्करा}
{ततो भागीरथी श्रेष्ठा यत्र विन्ध्येन सङ्गता} %३६


\onelineshloka
{तस्माद्दशगुणं पुण्यं तीर्थराजेऽत्र जायते} %३७

\twolineshloka
{कलौ दशसहस्रान्ते विष्णुस्त्यक्ष्यति मेदिनीम्}
{तदर्द्धं जाह्नवीतोयं तदर्धं देवतागणाः} %३८

\twolineshloka
{यावत्तिष्ठति गङ्गाऽत्र तावत्तीर्थानि सन्ति च}
{स्वस्वस्थाने नृणां पापं तावदेव हरन्ति च} %३९

\twolineshloka
{यदैव गङ्गा नष्टा स्यात्को वा तत्पापमाहरेत्}
{विचार्यैवं सुतीर्थानि गमिष्यन्ति धरातले} %४०

\twolineshloka
{तस्मान्मुनीश्वराः सर्वे यावत्तिष्ठति जाह्नवी}
{तावच्च क्रियतां धर्मस्ततो भूमौ निलीयताम्} %४१

\twolineshloka
{समाधिं गृह्य सुदृढां यावत्कृतयुगं भवेत्}
{अन्यथा कलिकालेन भ्रंशनीयो भवेत्सुधीः} %४२

\twolineshloka
{ततः श्रेष्ठतरा काशी यस्या नाशो न जायते}
{यदाश्रयेण गङ्गापि सर्वपापं व्यपोहति} %४३

\threelineshloka
{काशिकाया नैव नाशो ब्रह्मण्यपि मृते सति}
{यद्दर्शनार्थं गङ्गाऽपि जाता चोत्तरवाहिनी}
{तस्यां पञ्चनदं तीर्थं त्रिषु लोकेषु विश्रुतम्} %४४

\twolineshloka
{आगते कार्तिके मासि रौरवं नरकं गताः}
{आक्रोशन्ते तु पितरो वंशेऽस्माकं भविष्यति} %४५

\twolineshloka
{कश्चिद्भाग्यवतां श्रेष्ठो गत्वा पञ्चनदे शुभे}
{अस्माकं तर्पणं कुर्यान्नरकार्णवतारकम्} %४६

\twolineshloka
{तीर्थराजादितीर्थानि प्राप्ते कार्तिकमासके}
{स्नानार्थं पञ्चगङ्गं तु समायान्ति न संशयः} %४७

\twolineshloka
{कृत्वा तु लक्षपापानि स्नात्वा पञ्चनदे शुभे}
{बिन्दुमाधवमभ्यर्च्य विलयं यान्ति तत्क्षणात्} %४८

\twolineshloka
{यैः स्नातं कार्तिके मासि सकृत्पञ्चनदे शुभे}
{सर्वतीर्थकृतास्नानात्फलं कोटिगुणं भवेत्} %४९


\uvacha{ब्रह्मोवाच}
\twolineshloka
{कार्तिके मासि कावेर्य्यां यः स्नानं कर्तुमिच्छति}
{तावता वै विमुक्ताऽघो विष्णुसायुज्यमाप्नुयात्} %५०

\twolineshloka
{कावेर्य्याश्चैव माहात्म्यं को वदेत्परमुत्तमम्}
{अत्र ते वर्णयिष्यामि इतिहासं पुरातनम्} %५१

\twolineshloka
{कावेर्या विषये ब्रह्मन्सावधानमनाः शृणु}
{गौतम्या उत्तरे तीरे विष्णुपादाब्जसम्भवा} %५२

\twolineshloka
{गङ्गा त्रैलोक्यपापघ्नी वर्तते लोकपूजिता}
{सा गङ्गा चिन्तयामास कदाचित्पापशङ्किता} %५३

\twolineshloka
{सर्वलोकाः समागत्य मयि पापं त्यजन्ति हि}
{तत्पापं तु कथं गच्छेदिति चिन्तापरा तदा} %५४

\twolineshloka
{प्रष्टुं जगाम कैलासं गिरिजावल्लभं भवम्}
{तत्र दृष्ट्वा महारुद्रं प्रोवाच हरिपादजा} %५५


\uvacha{गङ्गोवाच}
\twolineshloka
{महारुद्र नमस्तेऽस्तु त्वां प्रष्टुमहमागता}
{सर्वे लोकाः समागत्य मयि पापं त्यजन्ति हि} %५६

\twolineshloka
{तत्पापं तु मया सोढुं न शक्यं पार्वतीपते}
{येनोपायेन तत्पापं नाऽऽगच्छेन्मम तद्वद} %५७


\onelineshloka*
{एवं गङ्गावचः श्रुत्वा प्रत्याह परमेश्वरः}

\uvacha{रुद्र उवाच}

\onelineshloka
{पापनिर्हरणायादौ पद्मनाभाङ्घ्रिपङ्कजात्} %५८

\twolineshloka
{प्रादुर्भूताऽसि त्वं देवि किमर्थं तप्यते त्वया}
{पापप्रहाराऽऽधिपत्यं कल्पितं तव विष्णुना} %५९

\twolineshloka
{तथाऽपि पापनिर्हार उपायं ते ब्रवीम्यहम्}
{कवेश्च तनया देवी कावेरी सरितां वरा} %६०

\twolineshloka
{सर्वोत्कृष्टा च सर्वेषां हरेर्बलवशात्तु सा}
{सर्वपापप्रहरणे सामर्थ्यं तत्र वर्तते} %६१

\twolineshloka
{कार्तिके मासि कावेर्यां यः स्नानं कुरुते नरः}
{स तु पापविनिर्मुक्तो याति विष्णोः परं पदम्} %६२

\twolineshloka
{तस्मात्तां गच्छ देवि त्वं ततः पापाद्विमोक्ष्यसे}
{इत्युक्ता सा तदागच्छत्कावेरीं पापहारिणीम्} %६३

\twolineshloka
{तज्जलस्पर्शमात्रेण कार्तिके विष्णुपादजा}
{निर्धूतपातका गङ्गा जगाम स्वनिकेतनम्} %६४

\twolineshloka
{कार्तिके प्रतिवर्षं तु गङ्गा त्रैलोक्यपावनीम्}
{स्नातुं भक्त्या समायाति कावेरीं पापहारिणीम्} %६५

\twolineshloka
{तज्जलस्पर्शमात्रेण कार्तिके विष्णुपादजा}
{निर्धूतपातका गङ्गा जगाम स्वनिकेतनम्} %६६

\twolineshloka
{तस्माच्छस्तं तुलास्नानं कावेर्य्यां शस्यते बुधैः}
{यः कावेर्यां तुलास्नानं भक्त्या तु कुरुते मुने} %६७

\twolineshloka
{विमुक्तदुरितः सद्यस्ततो याति परां गतिम्}
{तस्मात्स्नानं तु कावेर्यां कार्तिके मासि शस्यते} %६८

\twolineshloka
{इतिहासमिमं श्रुत्वा कार्तिकव्रततत्परः}
{स कावेरी स्नानफलं प्राप्नोति च परां गतिम्} %६९

\twolineshloka
{रात्रिशेषे भवेत्स्नानमुत्तमं विष्णुतुष्टिकृत्}
{सूर्योदये मध्यमं स्याद्यावन्नाऽऽस्ता तु कृत्तिका} %७०

\twolineshloka
{तावदेव भवेत्स्नानमन्यथा तन्न कार्तिकम्}
{स्नानं स्त्रीभिर्विधातव्यं गृहीत्वाऽऽज्ञां धवस्य च} %७१

\twolineshloka
{अपृष्ट्वा यत्कृतं धर्म्यं भर्तारं तत्क्षयं नयेत्}
{स्त्रीणां नास्त्यपरो धर्मो भर्तारं प्रोज्झ्य कश्चन} %७२

\twolineshloka
{कुर्यात्सहस्रपापानि भर्त्राऽऽज्ञां या समाचरेत्}
{सैषा धर्मवती लोके न जायेत व्रतादिना} %७३

\twolineshloka
{दरिद्रः पतितो मूर्खो दीनोऽपि यदि चेत्पतिः}
{तादृशः शरणं स्त्रीणां तत्त्यागान्निरयं व्रजेत्} %७४

\twolineshloka
{कलौ वत्स मनुष्याणां शैथिल्यं स्नानकर्मणि}
{तथापि कथयिष्यामि स्नानं कार्तिकमाघयोः} %७५

\twolineshloka
{यस्य हस्तौ च पादौ च वाङ्मनश्च सुसंयतम्}
{विद्या तपश्च कीर्तिश्च स तीर्थफलभाङ्नरः} %७६

\twolineshloka
{अश्रद्दधानः पापात्मा नास्तिकश्छिन्नमानसः}
{हेतुवादी च पञ्चैते न तीर्थफलभागिनः} %७७

\twolineshloka
{प्रातरुत्थाय यो विप्रस्तीर्थस्नायी सदा भवेत्}
{सर्वपापविनिर्मुक्तः परं ब्रह्माऽधिगच्छति} %७८

\twolineshloka
{स्नानं चतुर्विधं प्रोक्तं स्नानविद्भिर्मनीषिभिः}
{वायव्यं वारुणं दिव्यं ब्राह्मं चेति तथा स्मृतम्} %७९

\twolineshloka
{वायव्यं गोरजःस्नानं वारुणं सागरादिषु}
{ब्राह्मं ब्राह्मणमन्त्रोक्तं दिव्यं मेघाम्बु भास्करम्} %८०

\twolineshloka
{स्नानानां चैव सर्वेषां विशिष्टं तत्र वारुणम्}
{ब्राह्मणः क्षत्रियो वैश्यो मन्त्रवत्स्नानमाचरेत्} %८१

\twolineshloka
{तूष्णीमेव हि शूद्रस्य स्त्रीणां चैव तथा स्मृतम्}
{बाला च तरुणी वृद्धा नरनारीनपुंसकाः} %८२

\twolineshloka
{पापैः सर्वैः प्रमुच्यन्ते स्नानात्कार्तिकमाघयोः}
{स्नाता वै कार्तिके लोकाः प्राप्नुवन्तीप्सितं फलम्} %८३

\twolineshloka
{पुष्करे तीर्थवर्ये तु नन्दायाः सङ्गमे पुरा}
{प्रभञ्जनश्च मुक्तोभूत्तदैव व्याघ्रजन्मतः} %८४

\twolineshloka
{नन्दाया वचनेनैव कार्तिके सा परं ययौ}
{एवं स्नानविधिः प्रोक्तः किं भूयः श्रोतुमिच्छसि} %८५


\iti{कार्तिकस्नानविधिनिरूपणं}{नाम चतुर्थोऽध्यायः}{४}

\sect{अथ पञ्चमोऽध्यायः}


\uvacha{नारद उवाच}
\twolineshloka
{कदा स्नानं प्रकर्तव्यं कथं स्थेयं दिनावधि}
{आह्निकं तत्समाचक्ष्व विशेषेण पितामह} %१

\uvacha{ब्रह्मोवाच}
\twolineshloka
{रात्र्यां तुर्यांशशेषायामुत्तिष्ठेत्सर्वदा व्रती}
{विष्णुं स्तुत्वा बहुस्तोत्रैर्दिनकार्यं विचारयेत्} %२

\twolineshloka
{ग्रामनैर्ऋत्यदिग्भागे मलोत्सर्गं यथाविधि}
{ब्रह्मसूत्रं दक्षकर्णे स्थाप्य तत्र उदङ्मुखः} %३

\twolineshloka
{अन्तर्धाय तृणं भूमौ शिरः प्रावृत्य वाससा}
{वक्त्रं नियम्य वस्त्रेणाऽसङ्गः सोदकभाजनः} %४

\twolineshloka
{कुर्यान्मूत्रपुरीषं तु रात्रौ चेद्दक्षिणामुखः}
{तत उत्थाय चाऽऽगच्छेत्समीपं कलशस्य हि} %५

\twolineshloka
{गन्धलेपक्षयकरं मृत्तिकाशौचमाचरेत्}
{एका लिङ्गे करे तिस्र उभयोर्मृद्द्वयं स्मृतम्} %६

\twolineshloka
{मूत्रशौचे त्विदं ज्ञेयं विष्ठाशौचमतः शृणु}
{पञ्चापानेऽथवा सप्त दश वामकरे तथा} %७

\twolineshloka
{उभयोः सप्त दातव्याः पादयोर्मृत्तिकात्रयम्}
{एतच्छौचं गृहस्थस्य द्विगुणं ब्रह्मचारिणः} %८

\twolineshloka
{वानप्रस्थस्य त्रिगुणं यतीनां च चतुर्गुणम्}
{एतच्छौचं दिवा प्रोक्तं रात्रावर्द्धं समाचरेत्} %९

\twolineshloka
{मार्गस्थस्य तदर्धं स्यात्स्त्रीशूद्राणां तदर्धकम्}
{शौचकर्मविहीनस्य समस्ता निष्फलाः क्रियाः} %१०

\twolineshloka
{दन्तजिह्वाविशुद्धिं च ततः कुर्यादतन्द्रितः}
{आयुर्बलं यशो वर्चः प्रजाः पशुवसूनि च} %११

\twolineshloka
{ब्रह्म प्रज्ञां च मेधां च त्वं नो देहि वनस्पते}
{दन्तकाष्ठं तु गृह्णीयाद्द्वादशाङ्गुलसम्मितम्} %१२

\twolineshloka
{क्षीरवृक्षस्य न ग्राह्यं कार्पासस्य तथैव च}
{कण्टकस्य च वृक्षस्य दग्धवृक्षस्य चैव हि} %१३


\onelineshloka
{सद्वासनं मृदुतरं दन्तधावनमादितः} %१४

\threelineshloka
{उपवासे नवम्यां च षष्ठ्यां श्राद्धदिने रवौ}
{ग्रहणे प्रतिपद्दर्शे न कुर्याद्दन्तधावनम्}
{कुर्याद्द्वादशगण्डूषाननुक्ते दन्तधावने} %१५

\twolineshloka
{दन्तान्विशोध्य विधिवन्मुखं सम्मार्ज्य वारिणा}
{ललाटे चोर्ध्वपुण्ड्रं तु धृत्वा चाचम्य वारिणा} %१६

\twolineshloka
{देवालये नदीतीरे राजमार्गे विशेषतः}
{दत्त्वा चाकाशदीपं तु तुलसीसन्निधावथ} %१७

\twolineshloka
{गृहीत्वार्चनसामग्रीमिष्टदेवगृहं व्रजेत्}
{ततो गायेत नृत्येत पूजां कृत्वा तु बुद्धिमान्} %१८


\onelineshloka*
{पठित्वा विष्णुनामानि कुर्य्यान्नीराजनं हरेः}

\onelineshloka
{नाडीद्वयावशिष्टायां रात्र्यां गच्छेज्जलाशयम्} %१९

\twolineshloka
{तत्रोक्तविधिना स्नानं कुर्याद्वै कार्तिकव्रती}
{वस्त्रनिष्पीडनं कृत्वा कुर्याच्च तिलकं तथा} %२०

\twolineshloka
{ततः सन्ध्यामुपासीत स्वसूत्रोक्तेन वर्त्मना}
{ततः कार्यो जपो देव्या यावदकोन्दयो भवेत्} %२१

\twolineshloka
{एतत्प्रोक्तं रात्रिशेषकृत्यं दैनमथोच्यते}
{यस्मिन्कृते कार्तिकोऽयं सकलः सफलो भवेत्} %२२

\twolineshloka
{विष्णोः सहस्रनामाऽऽद्यं सन्ध्यान्ते च पठेत्ततः}
{देवालये समागत्य पुनः पूजनमारभेत्} %२३

\twolineshloka
{नृत्यगानादिकार्येषु प्रहरं दिवसं नयेत्}
{ततः पुराणश्रवणं यामार्धं सम्यगाचरेत्} %२४

\twolineshloka
{पौराणिकस्य पूजां तु तुलसीपूजनं तथा}
{कृत्वा माध्याह्निकं कर्म भुञ्जीत द्विदलोज्झितम्} %२५

\twolineshloka
{बलिदानं वैश्वदेवमतिथीनां समर्पणम्}
{कृत्वा भुङ्क्ते तु यो मर्त्यः केवलं चाऽमृतं हि तत्} %२६

\twolineshloka
{यथाशक्ति द्विजा भोज्याः प्रत्यहं वाऽथ पर्वणि}
{हविष्यभोजनं कुर्यादामिषं परिवर्जयेत्} %२७

\twolineshloka
{भक्षयेत्तुलसीं वक्त्रशुद्ध्यर्थं तीर्थवारिणा}
{संसारव्यवहारेण दिनशेषं समापयेत्} %२८

\twolineshloka
{सायङ्काले पुनर्गच्छेद्विष्णोर्देवालयं प्रति}
{सन्ध्यां कृत्वा प्रयुञ्जीत तत्र दीपन्यथाबलम्} %२९

\twolineshloka
{विष्णुं प्रणम्य हरये कृत्वा नीराजनं शुभम्}
{स्तोत्रपाठादिकं कुर्वन्नाद्ययामे तु जागरम्} %३०

\twolineshloka
{यामे तु प्रथमेऽतीते निद्रां कुर्याद्विचक्षणः}
{ब्रह्मचर्यव्रतं कुर्याद्भार्यामीयादृतौ तथा} %३१

\twolineshloka
{तया कामयमानो वा भार्यां गच्छेन्न दोषभाक्}
{एवं प्रतिदिनं कुर्यादामासं तु यथाविधि} %३२

\twolineshloka
{एवं तु कार्तिके मासि यः कुर्यात्परमं व्रतम्}
{सर्वपापविनिर्मुक्तो याति विष्णोः सलोकताम्} %३३

\fourlineindentedshloka
{रोगापहं पातकनाशकृत्परं}
{सद्बुद्धिदं पुत्रधनादिसाधकम्}
{मुक्तेर्निदानं नहि कार्तिकव्रता-}
{द्विष्णुप्रियादन्यदिहाऽस्ति भूतले} %३४


\iti{नित्यकर्मकथनं}{नाम पञ्चमोऽध्यायः}{५}

\sect{अथ षष्ठोऽध्यायः}


\uvacha{ब्रह्मोवाच}
\twolineshloka
{शृणु नारद वक्ष्यामि कार्तिकस्य व्रतं महत्}
{यच्छुत्वा सर्वपापेभ्यो मुक्तो मोक्षमवाप्स्यसि} %१

\twolineshloka
{कार्तिके मासि सम्प्राप्ते निषिद्धानि च वर्जयेत्}
{तैलाभ्यङ्गं परान्नं च तथा वै तैलभोजनम्} %२

\twolineshloka
{फलानि बहुबीजानि धान्यानि द्विदलान्यपि}
{वर्जयेत्कार्तिके मासि नात्र कार्या विचारणा} %३

\twolineshloka
{अलाबुं गृञ्जरं चैव वृन्ताकं बृहतीफलम्}
{अन्नं पर्युषितं वाऽपि भिस्सटं च मसूरिकम्} %४

\twolineshloka
{पुनर्भोजनं माध्वं च परान्नं कांस्यभोजनम्}
{नखं चर्म च छत्राकं काञ्जि दुर्गन्धमेव च} %५

\twolineshloka
{गणान्नं गणिकान्नं च तथा वै ग्रामयाजिनः}
{शूद्रान्नं शद्रसम्पर्कं सूतकान्नं तथैव च} %६

\twolineshloka
{श्राद्धान्नमृतुशान्त्याश्च जातकं नामकं तथा}
{श्लेष्मातकफलं चैव वर्जयेत्कार्तिकव्रती} %७

\threelineshloka
{निषिद्धेषु च पत्रेषु भोजनं नैव कारयेत्}
{मधुपालाशकदलीजम्बूप्लक्षमकूटिकाः}
{एतत्पत्रेषु भोक्तव्यं पुष्करे न कदाचन} %८

\twolineshloka
{कार्तिके मासि सम्प्राप्ते यः कुर्याद्वनभोजनम्}
{स याति परमं लोकं विष्णोर्देवस्य चक्रिणः} %९

\twolineshloka
{प्रातःस्नानं तु कर्तव्यं तथैव हरिपूजनम्}
{कथायाः श्रवणं चैव कार्तिके शस्यते मुने} %१०

\twolineshloka
{गोपीचन्दनदानं तु गोदानं श्रोत्रियाय च}
{कर्तव्यं कार्तिके मासि तेन मोक्षमवाप्नुयात्} %११

\twolineshloka
{कदलीफलदानं तु दानं धात्रीफलस्य च}
{वस्त्रदानं तथा कुर्याच्छीतार्ताय द्विजन्मने} %१२

\twolineshloka
{शाकादिदानं कुर्वीत चान्नदानं विशेषतः}
{शालिग्रामस्य दानं च कर्तव्यं तु द्विजन्मने} %१३

\twolineshloka
{पौराणिकाय यो दद्यादामान्नं घृतपायसम्}
{स चैश्वर्यमवाप्नोति शतब्राह्मणभोजनात्} %१४

\twolineshloka
{कमलैः पूजयेद्यस्तु कार्तिके कमलाप्रियम्}
{स तु पुण्यमवाप्नोति नात्र कार्या विचारणा} %१५

\twolineshloka
{कार्तिके तुलसीपत्रं यो भक्त्या विष्णवेऽर्पयेत्}
{संसाराच्च विनिर्मुक्तो याति विष्णोः परं पदम्} %१६

\twolineshloka
{कार्तिके केतकीपुष्पैरर्चयेद्गरुडध्वजम्}
{पूजितो जन्मसाहस्रं नात्र कार्या विचारणा} %१७

\twolineshloka
{शङ्खदानं तु यः कुर्यात्तथा चक्राङ्कितस्य च}
{तस्य पापानि नश्यन्ति दानमात्रान्न संशयः} %१८

\twolineshloka
{गीतापाठं तु यः कुर्यात्कार्तिके विष्णुवल्लभे}
{तस्य पुण्यफलं वक्तुं नालं वर्षशतैरपि} %१९

\twolineshloka
{श्रीमद्रागवतस्याऽपि श्रवणं यः समाचरेत्}
{सर्वपापविनिर्मुक्तः परं निर्वाणमृच्छति} %२०

\twolineshloka
{एकादश्यां निराहारमुपवासं करोति यः}
{पूर्वजन्मकृतात्पापान्मुच्यते नात्र संशयः} %२१

\twolineshloka
{शालिग्रामस्य नैवेद्यं कोटियज्ञफलं लभेत्}
{अन्यदेवस्य नैवेद्यं भुक्त्वा चान्द्रायणं चरेत्} %२२

\twolineshloka
{पूजाकाले तु देवस्य घण्टानादं करोति यः}
{हरेस्तृप्तिं परां याति मनुजो नात्र संशयः} %२३

\twolineshloka
{परान्नं वर्जयेद्यस्तु कार्तिके विष्णुतुष्टये}
{दामोदरस्य प्रीतिं स सम्यक्प्राप्नोति मानवः} %२४

\twolineshloka
{अध्वगं तु परिश्रान्तं काले च गृहमागतम्}
{योऽतिथिं पूजयेद्भक्त्या जन्मसाहस्रनाशनम्} %२५

\twolineshloka
{निन्दां कुर्वन्ति ये मूढा वैष्णवानां महात्मनाम्}
{पतन्ति पितृभिः सार्द्धं महारौरव संज्ञके} %२६

\twolineshloka
{दृष्ट्वा भागवतान्विप्रान्सन्मुखो न च याति हि}
{न गृह्णाति हरिस्तस्य पूजां द्वादशवार्षिकीम्} %२७

\twolineshloka
{निन्दां भगवतः शृण्वंस्तत्परस्य जनस्य च}
{ततो नाऽपैति यः सोऽपि हरेः प्रियतमो नहि} %२८

\twolineshloka
{प्रदक्षिणां तु यः कुर्यात्कार्तिके केशवस्य हि}
{पदेपदेऽश्वमेधस्य फलं प्राप्नोत्यसंशयः} %२९

\twolineshloka
{दण्डप्रणामं यः कुर्यात्कार्तिके केशवाग्रतः}
{राजसूयाऽश्वमेधानां फलं प्राप्नोत्यसंशयः} %३०

\twolineshloka
{कुटुम्बभोजनं चैव कार्तिके भक्तिसंयुतः}
{कारयेद्विप्रशार्दूल तस्य पुण्यमनन्तकम्} %३१

\twolineshloka
{परस्त्रीसङ्गमं यस्तु कार्तिके कुरुते नरः}
{तस्य पापस्य विश्रान्तिर्यावद्वक्तुं न शक्यते} %३२

\twolineshloka
{तुलसीमृत्तिकापुण्ड्रं ललाटे यस्य दृश्यते}
{यमस्तं नेक्षितुं शक्तः किमु दूता भयङ्कराः} %३३

\twolineshloka
{शाकं वा लवणं वाऽपि यत्किञ्चिद्वा भविष्यति}
{तद्देयं कार्तिके मासि प्रीत्यर्थं शार्ङ्गधन्वनः} %३४

\twolineshloka
{इत्याद्या बहवो धर्माः कार्तिके विष्णुवल्लभाः}
{यथाशक्त्या प्रकुर्वीत धर्मं देवस्य तुष्टिदम्} %३५

\twolineshloka
{हरिसन्तुष्टये कार्यस्त्यागो वा स्वेष्टवस्तुनः}
{मासान्ते द्विजवर्याय दद्यात्तद्व्रतपूर्तये} %३६

\twolineshloka
{सर्वव्रतानि चैकत्र सत्यव्रतमथैकतः}
{तस्मात्सर्वप्रयत्नेन सत्यं भाषेत सर्वदा} %३७

\twolineshloka
{अन्यधर्मेष्वधिकृतिः कुलजातिविभागतः}
{अधिकारी कार्तिके तु सर्व एव जनो भवेत्} %३८

\twolineshloka
{गोग्रासः कार्तिके मासि विशेषाद्यैस्तु दीयते}
{तेषां पुण्यफलं वक्तुं न शक्नोति पितामहः} %३९

\twolineshloka
{विष्णुदेवालयं प्रातः सम्मार्जयति कार्तिके}
{तस्य वैकुण्ठभवने जायते सुदृढं गृहम्} %४०

\twolineshloka
{दद्यात्कार्तिकमासे तु धर्मकाष्ठानि भूरिशः}
{न तत्पुण्यस्य नाशोस्ति कल्पकोटिशतैरपि} %४१

\twolineshloka
{सुधादि लेपयेद्यस्तु कार्तिके विष्णुमन्दिरे}
{चित्रादिकं लिखेद्वाऽपि मोदते विष्णुसन्निधौ} %४२

\twolineshloka
{देवालये वा तीर्थे वा कृतो दुष्टैर्नृपैः करः}
{तं मोचयन्ति ये लोकास्तेषां धर्मः सनातनः} %४३

\twolineshloka
{कार्तिके मासि यो विप्रो गभस्तीश्वरसन्निधौ}
{शतरुद्रीजपं कुर्यान्मन्त्रसिद्धिः प्रजायते} %४४

\twolineshloka
{वाराणस्यां तु यैः स्थित्वा त्रिवर्षं कार्तिकव्रतम्}
{सोपाङ्गं साङ्गं यैर्मर्त्यैः कृतं भक्त्यैकतत्परैः} %४५

\twolineshloka
{इह लोके फलं तेषां प्रत्यक्षं जायते किल}
{सम्पत्त्या चैव सन्तत्या यशोभिर्धर्मबुद्धिभिः} %४६

\twolineshloka
{पलाण्डुं शृङ्गं मांसं च शय्यां सौवीरकं तथा}
{राजिकोन्मादिकं चापि चिपिटान्नं च वर्जयेत्} %४७

\twolineshloka
{धात्रीफलं भानुवारे परदेशागमं तथा}
{तीर्थं विना सदैवेह वर्जयेत्कार्तिकव्रती} %४८

\twolineshloka
{देववेदद्विजातीनां गुरुगोव्रतिनां तथा}
{स्त्रीराजमहतां निन्दां वर्जयेत्कार्तिकव्रती} %४९

\threelineshloka
{नरकस्य चतुर्दश्यां तैलाभ्यङ्गं च कारयेत्}
{अन्यत्र कार्तिके मासि तैलस्नानं विवर्जयेत्}
{नालिकां मूलकं चैव कूष्माण्डं च कपित्थकम्} %५०

\twolineshloka
{रजस्वलान्त्यज म्लेच्छ पतितऽव्रतिकैस्तथा}
{द्विजद्विड्वेदबाह्यैश्च न वदेत्सर्वदा व्रती} %५१

\twolineshloka
{एभिर्दृष्टं च काकैश्च सूतिकान्नं च यद्भवेत्}
{द्विःपाचितं च दग्धान्नं नैवाद्याद्वैष्णवव्रती} %५२

\twolineshloka
{क्रमात्कूष्माण्डबृहतीतरुणीमूलकं तथा}
{श्रीफलं च कलिङ्गं च फलं धात्रीभवं तथा} %५३

\twolineshloka
{नारिकेलमलाबुं च पटोलं बृहतीफलम्}
{चर्मवृन्ताकचवलीशाकं तुलसिजं तथा} %५४

\twolineshloka
{शाकान्येतानि वर्ज्यानि क्रमात्प्रतिपदादिषु}
{एवमेव हि माघेऽपि कुर्य्याच्च नियमान्व्रती} %५५

\twolineshloka
{कार्तिकव्रतिनः पुण्यं यथोक्तव्रतकारिणः}
{न समर्थो भवेद्वक्तुं ब्रह्माऽपीह चतुर्मुखः} %५६


\iti{कार्तिकव्रतनिरूपणं}{नाम षष्ठोऽध्यायः}{६}

\sect{अथ सप्तमोऽध्यायः}


\uvacha{नारद उवाच}
\twolineshloka
{भगवन्कृतकृत्योऽस्मि तव पादसमाश्रयात्}
{श्रोतव्यं नेह भूयो मे विद्यते देवसत्तम} %१

\twolineshloka
{तथापि भगवन्किञ्चित्प्रष्टव्यं मे हृदि स्थितम्}
{त्वद्वाक्यामृतपीतस्य न मे तृप्तिर्हि जायते} %२

\twolineshloka
{दीपदानस्य माहात्म्यं श्रोतुमिच्छामि ते प्रभो}
{येन चापि पुरा दत्तस्तद्वदस्व चतुर्मुख} %३


\uvacha{ब्रह्मोवाच}
\twolineshloka
{प्रातः स्नात्वा शुचिर्भूत्वा दीपं दद्यात्प्रयत्नतः}
{तेन पापानि नश्येयुस्तमांसीव भगोदये} %४

\twolineshloka
{आजन्म यत्कृतं पापं स्त्रिया वा पुरुषेण च}
{तत्सर्वं नाशमायाति कार्तिके दीपदानतः} %५

\twolineshloka
{अत्र ते वर्णयिष्यामि इतिहासं पुरातनम्}
{श्रवणात्सर्वपापघ्नं दीपदानफलप्रदम्} %६

\twolineshloka
{पुरा द्रविडदेशे तु ब्राह्मणो बुद्धनामकः}
{तस्य भार्याऽभवद्दुष्टा अनाचाररता मुने} %७

\twolineshloka
{तस्याः संसर्गदोषेण क्षीणाऽऽयुर्मृतिमाप्तवान्}
{पत्यौ मृतेऽपि सा पत्नी अनाचारे विशेषतः} %८

\twolineshloka
{रताभून्न हि तस्यास्तु लज्जा लोकापवादतः}
{सुतबन्धुविहीना सा सदाभिक्षान्नभोजना} %९

\twolineshloka
{न संस्कारान्नमल्पं वा भुक्त्वा पर्युषिताशिनी}
{परपाकरता नित्यं तीर्थयात्रादिवर्जिता} %१०

\twolineshloka
{कथायाः श्रवणं चैव न श्रुतं तु तया द्विज}
{एकदा ब्राह्मणः कश्चित्तीर्थयात्रापरायणः} %११

\threelineshloka
{तस्या गृहं समागच्छद्विद्वान्वै कुत्सनामकः}
{अनाचाररतां तां तु दृष्ट्वा ब्रह्मर्षिसत्तमः}
{कोपेन रक्तचक्षुः संस्तामुवाचाऽसतीं स्त्रियम्} %१२


\uvacha{कुत्स उवाच}

\onelineshloka
{वक्ष्यामि साम्प्रतं मूढे मद्वाक्यमवधारय} %१३

\twolineshloka
{दुःखहेतुमिमं देहं पूयशोणितपूरितम्}
{पञ्चभूतात्मकं चैव किं च पुष्णासि दूतिके} %१४

\twolineshloka
{जलबुद्बुदवद्देहो नाशमायाति निश्चितम्}
{अनित्यं देहमाश्रित्य नित्यं त्वं मन्यसे हृदि} %१५

\twolineshloka
{तस्मादन्तःस्थितं मोहं त्यज मूढे विचारतः}
{स्मर सर्वोत्तमं देवं कुरु श्रवणमादरात्} %१६

\twolineshloka
{कार्तिके मासि सम्प्राप्ते स्नानदानादिकं कुरु}
{दामोदरस्य प्रीत्यर्थं दीपदानं तथा कुरु} %१७

\twolineshloka
{लक्षवर्त्यादिकं चैव लक्षपद्मादिकं तथा}
{प्रदक्षिणां तु देवस्य नमस्कारं तथैव च} %१८

\twolineshloka
{धारणं पारणं चैव कुरु भक्त्या हि कार्तिके}
{विधवानां व्रतमिदं सधवानां तथैव च} %१९

\twolineshloka
{सर्वपापप्रशमनं सर्वोपद्रवनाशनम्}
{तत्रापि कार्तिके मासि दीयतां दीप उत्तमः} %२०

\twolineshloka
{दीपो हरेः प्रियकरः कार्तिके मासि निश्चितम्}
{महापातककृद्वापि दीपदानात्प्रमुच्यते} %२१

\twolineshloka
{पुरा कश्चिद्द्विजवरो नाम्ना हरिकरो ह्यभूत्}
{अधर्मविषयासक्तः शश्वद्वेश्यारतो द्विजः} %२२

\twolineshloka
{पितृवित्तक्षयकरो वंशच्छेदे कुठारकः}
{कदाचित्तेन विधवे द्यूते पितृधनं महत्} %२३

\twolineshloka
{हारितं दुष्टसंसर्गात्ततो दुःखी स चाभवत्}
{कदाचित्साधुसंसर्गात्तीर्थयात्राप्रसङ्गतः} %२४

\twolineshloka
{अयोध्यामागतो वत्से महापापकरो द्विजः}
{कार्तिके मासि सम्प्राप्तः श्रीमद्द्विजगृहे सदा} %२५

\twolineshloka
{द्यूतव्याजेन तेनाऽऽशु दीपो दत्तो हरेः पुरः}
{ततः कालान्तरे विप्रो मृतो मोक्षमवाप्तवान्} %२६

\twolineshloka
{महापातककृद्वाऽपि गतवानभयं हरिम्}
{तस्मात्त्वं कार्तिके मासि दीपदानं तथा कुरु} %२७

\twolineshloka
{तथाऽन्यान्यपि दानानि कुरु भक्तिसमन्विता}
{इत्यादिश्याथ तां कुत्सो जगामाऽन्यगृहं द्विजः} %२८

\twolineshloka
{साऽपि कुत्सवचः श्रुत्वा पश्चात्तापेन संयुता}
{व्रतं तु कार्तिके मासि करिष्यामीति निश्चिता} %२९

\twolineshloka
{पतङ्गोदयवेलायां कार्तिके स्नानमम्भसि}
{दीपदानं वत चैव मासमेकं चकार सा} %३०

\twolineshloka
{ततः कालान्तरे चैव गतायुर्मृतिमागता}
{दीपदानस्य माहात्म्यान्महापापकृदप्यसौ} %३१

\twolineshloka
{स्वर्गमार्गं गता सा स्त्री काले मोक्षमवाप ह}
{तस्मान्नारद माहात्म्यं दीपदानस्य को वदेत्} %३२

\twolineshloka
{कार्तिके दीपदानं तु महापुण्यफलप्रदम्}
{कार्तिकव्रतनिष्ठो यो दीपदानादिकृन्नरः} %३३


\onelineshloka
{दीपदानस्येतिहासं शृण्वन्वै मोक्षमाप्नुयात्} %३४

\twolineshloka
{दीपदानस्य माहात्म्यं वक्तुं केनेह शक्यते}
{परदीपप्रबोधस्य माहात्म्यं शृणु नारद} %३५

\twolineshloka
{स्वस्याऽपि शक्तिराहित्ये परस्याऽपि प्रबोधनम्}
{यः कुर्याल्लभते सोऽपि नात्र कार्या विचारणा} %३६

\twolineshloka
{दीपार्थं वर्तिकां तैलं पात्रं वा यो ददाति हि}
{सहायं वाऽथ कुरुते ददतां दीपमुत्तमम्} %३७

\twolineshloka
{स तु मोक्षमवाप्नोति नात्र कार्या विचारणा}
{कार्तिके दीपदानस्य माहात्म्यं को नु वर्णयेत्} %३८

\twolineshloka
{स्वस्यापि शक्तिराहित्ये परदीपं प्रबोधयेत्}
{सोऽपि तत्फलमाप्नोति नात्र कार्या विचारणा} %३९

\twolineshloka
{वेश्या चेन्दुमतीनाम तस्या गेहेऽथ मूषिका}
{परदीपप्रबोधेन मोक्षं प्राप सुदुर्लभम्} %४०

\twolineshloka
{तस्मात्सर्वप्रयत्नेन परदीपं प्रबोधयेत्}
{तेन मोक्षमवाप्नोति मूषिकावन्न संशयः} %४१

\twolineshloka
{परदीपप्रबबोधस्य फलमीदृग्विधं मुने}
{साक्षाद्दीपप्रदानस्य माहात्म्यं केन वर्ण्यते} %४२


\uvacha{नारद उवाच}
\threelineshloka
{कार्तिके दीपदानस्य माहात्म्यं च मया श्रुतम्}
{परदीप प्रबोधस्य माहात्म्यमपि वै श्रुतम्}
{इदानीं श्रोतुमिच्छामि व्योमदीपस्य वैभवम्} %४३


\uvacha{ब्रह्मोवाच}
\twolineshloka
{आकाशदीप माहात्म्यं शृणु पुत्र समाहितः}
{यस्य श्रवणमात्रेण दीपदाने मतिर्भवेत्} %४४

\twolineshloka
{सम्प्राप्ते कार्तिके मासि प्रातःस्नानपरायणः}
{आकाशदीपं यो दद्यात्तस्य पुण्यं वदाम्यहम्} %४५

\twolineshloka
{सर्वलोकाधिपो भूत्वा सर्वसम्पत्समन्वितः}
{इह लोके सुखं भुक्त्वा चान्ते मोक्षमवाप्नुयात्} %४६

\threelineshloka
{स्नानदानक्रियापूर्वं हरिमन्दिरमस्तके}
{आकाशदीपो दातव्यो मासमेकं तु कार्तिके}
{कार्तिके शुद्धपूर्णायां विधिनोत्सर्जयेच्च तम्} %४७

\twolineshloka
{यः करोति विधानेन कार्तिके व्योम्नि दीपकम्}
{न तस्य पुनरावृत्तिः कल्पकोटिशतैरपि} %४८

\twolineshloka
{अत्र ते वर्णयिष्यामि इतिहासं पुरातनम्}
{यस्य श्रवणमात्रेण व्योमदीपफलं लभेत्} %४९

\twolineshloka
{पुरा तु निष्ठुरोनाम लुब्धको लोककण्टकः}
{यमुनातीरवासी च कालमृत्युरिवाऽपरः} %५०

\twolineshloka
{वने चरन्मृगान्सर्वान्हत्वा वृत्तिमकल्पयत्}
{पथिकान्बाधते नित्यं चोरवृत्त्या धनुर्धरः} %५१

\twolineshloka
{कञ्चिद्ग्रामं जगामाशु चौर्यार्थं कार्तिके मुने}
{तस्मिन्विदर्भनगरे राजा सुकृतिनामकः} %५२

\twolineshloka
{चन्द्रशर्माख्यविप्रस्य वचनात्कार्तिके सुधीः}
{चकार व्योमदीपं तु हरिमन्दिरमस्तके} %५३

\twolineshloka
{दीपं दत्त्वा महाभक्त्या अशृणोच्च कथां निशि}
{एतस्मिन्नेव काले तु चौर्यार्थं समुपागतः} %५४

\twolineshloka
{राज्ञा दत्तं व्योमदीपं पश्यन्क्षणमतिष्ठत}
{तदानीं दैवयोगेन गृध्रो जवसमन्वितः} %५५

\twolineshloka
{शीघ्रमागत्य जग्राह तैलपात्रं सदीपकम्}
{स्वमुखेनैव सङ्गृह्य वृक्षाग्रं च समाश्रयत्} %५६

\twolineshloka
{तत्र पीत्वा तु तैलं च दीपं स्थाप्य स पक्षिराट्}
{वृक्षाग्रं तु समास्थाय क्षणमात्रमतिष्ठत} %५७

\twolineshloka
{तदानीं दैवयोगेन ग्रहीतुं पक्षिसत्तमम्}
{मार्जारोऽप्यारुहद्वृक्षं पक्षिणाऽधिष्ठितं तु तम्} %५८

\twolineshloka
{तदग्रे मुखदीपं च पश्यन्क्षणमतिष्ठत}
{आकाशदीपमाहात्म्यं कथितं चन्द्रशर्मणा} %५९

\twolineshloka
{राज्ञे सुकृतिनाम्ने च तौ वै शुश्रुवतुः क्षणम्}
{खगमार्जारकौ तत्र स्वस्वचामचल्यदोषतः} %६०

\twolineshloka
{मार्जारो जगृहे तत्र शाखामतरगतं खगम्}
{दैवेन चोदितौ वृक्षाच्छिलायां पतितौ तदा} %६१

\twolineshloka
{भग्नगात्रौ मृतौ तत्र पक्षिमार्जारकौ भुवि}
{दिव्यदेहसमायुक्तौ यानारूढौ दिवं गतौ} %६२

\twolineshloka
{तत्सर्वं लुब्धको दृष्ट्वा चौर्यार्थं समुपागतः}
{निवृत्तो दुष्टभावेन कथयन्तं कथां मुनिम्} %६३

\twolineshloka
{चन्द्रशर्माणमाभाष्य इदं वचनमब्रवीत्}
{चन्द्रशर्मन्मया दृष्टं चौर्यार्थं ह्यागतेन च} %६४

\twolineshloka
{राज्ञा सुकृतिना दत्तं व्योमदीपं मनोहरम्}
{तदानीं दैवयोगेन खगः पात्रं प्रगृह्य च} %६५

\twolineshloka
{तैलं पीत्वा तु तत्पात्रं सदीपं तु मनोहरम्}
{वृक्षाग्रे स्थापयित्वा च तत्र क्षणमतिष्ठत} %६६

\twolineshloka
{मार्जारोऽप्यागतस्तत्र ग्रहीतुं पक्षिपुङ्गवम्}
{दैवेन प्रेरितौ तौ च उभे शाखे समाश्रितौ} %६७

\twolineshloka
{त्वन्मुखात्कथ्यमानां हि कथां शुश्रुवतुः क्षणम्}
{पश्चाच्चाञ्चल्यदोषेण मार्जारो ह्यग्रहीत्खगम्} %६८

\twolineshloka
{तौ वृक्षात्पतितौ मृत्युं प्राप्तौ च क्षणमात्रतः}
{उभौ तौ दिव्यरूपौ च यानारूढौ दिवं गतौ} %६९

\twolineshloka
{तदाश्चर्यमहं दृष्ट्वा त्वां प्रष्टुं समुपागतः}
{तौ कौ पुरा च मार्जारखगौ तद्वद भो द्विज} %७०

\twolineshloka
{तिर्यग्योनिसमापन्नौ मुक्तौ केन च कर्मणा}
{इति लुब्धवचः श्रुत्वा चन्द्रशर्माऽब्रवीत्तदा} %७१

\twolineshloka
{शृणु लुब्ध प्रवक्ष्यामि तयोर्वृत्तान्तमञ्जसा}
{मार्जारोऽपि पुरा पापी तथा श्रीवत्सगोत्रजः} %७२

\twolineshloka
{देवशर्मा इति प्रोक्तो देवद्रव्याऽपहारकः}
{अहोबलनृसिंहस्य पूजाकर्तृत्वमाप सः} %७३

\twolineshloka
{तस्मिन्देवालये प्राप्तं तैलं द्रव्यादिकं तथा}
{अपहृत्य च तेनैव कुटुम्बं पोषयत्यसौ} %७४

\twolineshloka
{आयुर्नीत्वैवमेवाऽसौ ततः पञ्चत्वमागतः}
{तस्मात्पापात्कालसूत्रं महारौरवरौरवम्} %७५

\twolineshloka
{निरुच्छ्वासं तथा प्राप्य असिपत्रवनं क्रमात्}
{छिद्यमानो महाकायैर्यमदूतैर्भयङ्करैः} %७६

\twolineshloka
{अनुभूय च तान्सर्वान्ब्रह्मराक्षसतां गतः}
{ततस्तु श्वानयोनौ च चण्डालोऽभूत्कुकर्मतः} %७७

\threelineshloka
{एवं जन्मशतं प्राप्य भूमौ मार्जारतां गतः}
{आकाशदीपमाहात्म्यं श्रुत्वेदानीं तु दैवतः}
{निर्मुक्ताऽखिलपापस्तु अगमद्धरिमन्दिरम्} %७८

\twolineshloka
{गृध्रोऽयं तु पुरा विप्रो मिथिले वेदपारगः}
{शर्यातिरिति विख्यातौ नाम्ना लोके महाप्रभुः} %७९

\twolineshloka
{दासीसङ्गं चकारासौ वेश्यासङ्गं तथैव च}
{तेन दोषेण महता पञ्चत्वमगमत्तदा} %८०

\twolineshloka
{कुम्भीपाके महाघोरे स्थित्वा युगचतुष्टयम्}
{कर्मशेषेण भूमौ च गृध्रत्वमगमत्तदा} %८१


\onelineshloka
{दैवेन चोदितो गृध्रस्तैलपानार्थमागतः} %८२

\twolineshloka
{दत्त्वा चाकाशदीपं च श्रुत्वा चैव हरेः कथाम्}
{विध्वस्ताऽखिलपापस्तु जगाम हरिमन्दिरम्} %८३

\twolineshloka
{इत्येतत्सर्वमाख्यातं लुब्ध गच्छ यथासुखम्}
{व्याधोऽप्यस्य वचः श्रुत्वा गत्वा चैव स्वमन्दिरम्} %८४

\twolineshloka
{व्रतं चाकाशदीपस्य चकार विधिवन्मुने}
{आयुःशेषं तदा नीत्वा जगाम हरिमन्दिरम्} %८५

\twolineshloka
{सुनन्दोपि महाराज आश्चर्यं समुपागतः}
{चकार विधिना मासं चन्द्रशर्मोक्तमार्गतः} %८६

\twolineshloka
{प्रातः स्नात्वा शुचिर्भूत्वा कार्तिके मासि वै नृपः}
{कोमलैस्तुलसीपत्रैः समभ्यर्च्य जनार्दनम्} %८७


\onelineshloka
{रात्रौ दद्याद्व्योमदीपं मन्त्रेणानेन वै नृपः} %८८

\threelineshloka
{दामोदराय विश्वाय विश्वरूपधराय च}
{नमस्कृत्वा प्रदास्यामि व्योमदीपं हरिप्रियम्}
{निर्विघ्नं कुरु देवेश यावन्मासः समाप्यते} %८९

\twolineshloka
{व्रतेनानेन देवेश त्वयि भक्तिः प्रवर्द्धताम्}
{इति मन्त्रेण राजाऽसौ दीपदानं चकार ह} %९०

\twolineshloka
{ब्राह्मे मुहूर्ते च पुनर्व्योमदीपं ददाति हि}
{विष्णोः पूजा कृता प्रातः प्रातःस्नानं चकार ह} %९१

\twolineshloka
{उत्सर्गस्य विधिं कृत्वा व्योम्नि दीपं समाप्य च}
{ब्राह्मणान्भोजयित्वा च व्रतं विष्णोः समार्पयत्} %९२

\twolineshloka
{तेन पुण्यप्रभावेन स राजा मुनिसत्तम}
{शरदां शतसाहस्रमिह भोगान्मनोहरान्} %९३

\twolineshloka
{सुपुत्रपौत्रस्वजनैर्बुभुजे सह भार्यया}
{ततश्चान्ते द्विजवर विमानं सुमनोहरम्} %९४

\twolineshloka
{स्त्रीभिः सह समारुह्य मोक्षमार्गं गतो मुने}
{चतुर्भुजः पीतवासाः शङ्खचक्रगदाधरः} %९५

\twolineshloka
{विष्णुलोके विष्णुरिव प्रोच्यमानः सदाऽमरैः}
{क्रीडयामास राजाऽसौ यथाकामं महामनाः} %९६

\twolineshloka
{तस्मात्तु कार्तिके मासि मानुष्यं प्राप्य दुर्लभम्}
{आकाशदीपो दातव्यो विधानेन हरेः प्रियः} %९७

\fourlineindentedshloka
{दास्यन्ति ये कार्तिकमासि मर्त्या}
{व्योमप्रदीपं हरितुष्टयेऽत्र}
{पश्यन्ति ते नैव कदाऽपि देवं}
{यमं महाक्रूरमुखं मुनीन्द्र} %९८

\twolineshloka
{अथान्यच्च प्रवक्ष्यामि व्योमदीपस्य वैभवम्}
{वालखिल्यैः पुरा प्रोक्तं तच्छृणुष्व द्विजोत्तम} %९९


\uvacha{वालखिल्या ऊचुः}
\twolineshloka
{कृष्णादिमासक्रमतः कार्तिकस्यादिमासतः}
{आकाशादीपदानं तु कुर्वन्तु ऋषिसत्तमाः} %१००

\twolineshloka
{तुलायां तिलतैलेन सायं सन्ध्यासमागमे}
{आकाशदीपं यो दयान्मासमेकं निरन्तरम्} %१

\twolineshloka
{सश्रीकाय श्रीपतये श्रिया न स वियुज्यते}
{आकाशदीपवंशस्तु विंशद्धस्तोत्तमो भवेत्} %२

\twolineshloka
{मध्यमो नवहस्तः स्यात्कनिष्ठः पञ्चहस्तकः}
{यथा दूरस्थितैर्लोकैर्दृश्यते तत्तथाऽऽचरेत्} %३

\twolineshloka
{तथाऽभ्रादिकरण्डेषु दीपदानं विशिष्यते}
{वंशस्य नवमांशेन लम्बा कार्या पताकिका} %४

\twolineshloka
{मयूरपिच्छमुष्टिं वा कलशं चोपरि न्यसेत्}
{विष्णुप्रीतिकरो दीपः पित्रुद्धारस्य कारकः} %५

\twolineshloka
{एकादश्यास्तुलार्काद्वा दीपदानमतोऽपि वा}
{दामोदराय नभसि तुलायां लोलया सह} %६

\twolineshloka
{प्रदीपं ते प्रयच्छामि नमोऽनन्ताय वेधसे}
{आकाशदीपसदृशं पितुरुद्धारकं नहि} %७

\twolineshloka
{हेलिकस्य च द्वौ पुत्रौ तत्रैकस्तु पिशाचकः}
{व्योमदीपपुण्यदानान्मोक्षं प्राप सुदुर्लभम्} %८

\twolineshloka
{नमः पितृभ्यः प्रेतेभ्यो नमो धर्माय विष्णवे}
{नमो यमाय रुद्राय कान्तारपतये नमः} %९

\threelineshloka
{मन्त्रेणानेन ये मर्त्याः पितृभ्यः खे तु दीपकम्}
{प्रयच्छन्ति गता ये स्युर्नरके यान्ति तेऽपि वै}
{उत्तमां गतिमित्थं ते दीपदानं मयेरितम्} %११०


\onelineshloka
{लक्ष्मीसन्ततिसिद्ध्यर्थमारोग्याय प्रदीपयेत्} %११

\twolineshloka
{कार्तिके कृष्णपक्षे तु द्वादश्यादिषु पञ्चसु}
{तिथीषूक्तः पूर्वरात्रे नृणां नीराजनाविधिः} %१२

\twolineshloka
{ब्रह्मविष्णुशिवादीनां भवनेषु विशेषतः}
{कूटागारेषु चैत्येषु सभासु च नदीषु च} %१३

\twolineshloka
{प्राकारोद्यानवापीषु प्रतोलीनिष्कुटेषु च}
{मन्दुरासु विविक्तासु हस्तिशालासु चैव हि} %१४

\twolineshloka
{प्रदोषसमये दीपान्दद्यादेवं मनोहरान्}
{कृतं यैः कार्तिके मासि दीपदानं विधानतः} %१५

\twolineshloka
{दृश्यन्ते ये रत्नभाजस्तेऽत एव प्रकीर्तिताः}
{दीपदानासमर्थश्चेत्परदीपं तु रक्षयेत्} %१६

\twolineshloka
{यो वेदाभ्यासिने दद्याद्दीपार्थं तैलमादरात्}
{को वा तस्य फलं वक्तुं भुवि तिष्ठति मानवः} %१७

\twolineshloka
{दीपान्दद्याद्बहुविधान्कार्तिके विष्णुसन्निधौ}
{कार्तिके मासि सम्प्राप्ते गगने स्वच्छताके} %१८

\twolineshloka
{रात्रौ लक्ष्मीः समायाति द्रष्टुं भुवनकौतुकम्}
{यत्रयत्र च दीपान्सा पश्यत्यब्धिसमुद्भवा} %१९

\twolineshloka
{तत्रतत्र रतिं कुर्यान्नान्धकारे कदाचन}
{तस्माद्दीपः स्थापनीयः कार्तिके मासि वै सदा} %१२०

\twolineshloka
{लक्ष्मीरूपार्थिनां प्रोक्तं दीपदानं विशेषतः}
{देवालये नदीतीरे राजमार्गे विशेषतः} %२१

\twolineshloka
{निद्रास्थले दीपदाता तस्य श्रीः सर्वतोमुखी}
{दुर्बलस्याऽऽलयं वीक्ष्य दीपशून्यं तु यो ददेत्} %२२

\twolineshloka
{विप्रस्य वाऽन्यवर्णस्य विष्णुलोके महीयते}
{कीटकण्टकसङ्कीर्णे दुर्गमे विषमस्थले} %२३

\twolineshloka
{कुर्याद्यो दीपदानानि नरकं स न गच्छति}
{दद्याद्रात्रौ पञ्चनदे दीपं यो विधिपूर्वकम्} %२४

\twolineshloka
{तस्य वंशे प्रजायन्ते बालकाः कुलदीपकाः}
{पितृपक्षेऽन्नदानेन ज्येष्ठाषाढे च वारिणा} %२५

\twolineshloka
{कार्तिके तत्फलं तेषां परदीपप्रबोधनात्}
{बोधनात्परदीपस्य वैष्णवानां च सेवनात्} %२६

\twolineshloka
{कार्तिके फलमाप्नोति राजसूयाश्वमेधयोः}
{पुरा हरिकरोनाम द्विजः पापरतः सदा} %२७

\twolineshloka
{कृतं द्यूतप्रसङ्गेन दीपदानं हि कार्तिके}
{तेन पुण्यप्रभावेन स्वर्गं प्राप द्विजोत्तमः} %२८

\twolineshloka
{आकाशदीपदानेन पुरा वै धर्मनन्दनः}
{विमानवरमारुह्य विष्णुलोकं ययौ नृपः} %२९

\twolineshloka
{यः कुर्यात्कार्तिके विष्णोः पुरः कर्पूरदीपकम्}
{प्रबोधिन्यां विशेषेण तस्य पुण्यं वदाम्यहम्} %१३०

\twolineshloka
{कुले तस्य प्रसूता ये पुरुषास्ते हरिप्रियाः}
{क्रीडित्वा सुचिरं कालमन्ते मुक्तिं व्रजन्ति च} %३१

\twolineshloka
{दीपको ज्वलते यस्य दिवा रात्रौ हरेर्गृहे}
{एकादश्यां विशेषेण स याति हरिमन्दिरम्} %३२

\twolineshloka
{लुब्धकोऽपि चतुर्दश्यां दीपं दत्त्वा शिवालये}
{भक्त्या विना परे लिङ्गे शिवलोकं जगाम सः} %३३

\twolineshloka
{गोपः कश्चिदमावास्यां दीपं प्रज्वाल्य शार्ङ्गिणः}
{मुहुर्जयजयेत्युक्त्वा स च राजेश्वरोऽभवत्} %१३४


\iti{दीपदानमाहात्म्यवर्णनं}{नाम सप्तमोऽध्यायः}{७}

\sect{अथ नामाऽष्टमोऽध्यायः}


\uvacha{नारद उवाच}
\twolineshloka
{भूयः कथय तृप्तिर्हि नास्ति मे कमलासन}
{त्वद्वागमृतपानेन तृषा भूयः प्रवर्धते} %१


\uvacha{ब्रह्मोवाच}
\threelineshloka
{प्रातः स्नात्वा शुचिर्भूत्वा कार्तिके विष्णुतत्परः}
{देवं दामोदरं पूज्य कोमलैस्तुलसीदलैः}
{स तु मोक्षमवाप्नोति नात्र कार्या विचारणा} %२

\twolineshloka
{भक्त्या विरहितो यस्तु सुवर्णादिभिरर्चयेत्}
{तस्य पूजां न गृह्णाति नात्र कार्या विचारणा} %३

\twolineshloka
{सर्वेषामपि वर्णानां भक्तिरेषा परा स्मृता}
{भक्त्या विरहितं कर्म न विष्णोः प्रियकारणम्} %४

\twolineshloka
{भक्त्या सम्पूजितो नित्यं तुलस्यास्तु दलार्धतः}
{स्वयं प्रत्यक्षमायाति भगवान्हरिरीश्वरः} %५

\twolineshloka
{विष्णुदासः पुरा भक्त्या तुलसीपूजनेन च}
{विष्णुलोकं गतः शीघ्रं चोलो गौणत्वमागतः} %६

\twolineshloka
{तुलस्याः शृणु माहात्म्यं पापघ्नं पुण्यवर्द्धनम्}
{यत्पुरा विष्णुना प्रोक्तं रमायै तद्वदाम्यहम्} %७

\twolineshloka
{सम्प्राप्ते कार्तिके मासि तुलस्याः पूजनं हरेः}
{ये कुर्वन्ति नरा भक्त्या ते यान्ति परमं पदम्} %८

\twolineshloka
{तस्मात्सर्वप्रयत्नेन तुलस्याः कोमलैर्दलैः}
{पूजनीयो महाभक्त्या सर्वक्लेशविनाशनः} %९

\twolineshloka
{रोपिता तुलसी यावत्कुरुते मूलविस्तरम्}
{तावद्युगसहस्राणि ब्रह्मलोके महीयते} %१०

\twolineshloka
{तुलसीपत्रसंयुक्तजले स्नानं चरेद्यदि}
{सर्वपापविनिर्मुक्तो मोदते विष्णुमन्दिरे} %११

\twolineshloka
{वृन्दावनं च कुरुते रोपणार्थं महामुने}
{तावतैव विमुक्ताऽघो ब्रह्मभूयाय कल्पते} %१२

\twolineshloka
{तुलसीकाननं ब्रह्मन्गृहे यस्यावतिष्ठते}
{तद्गृहं तीर्थभूतं तु न यान्ति यमकिङ्कराः} %१३

\twolineshloka
{सर्वपापहरं पुण्यं कामदं तुलसीवनम्}
{रोपयन्ति नराः श्रेष्ठास्ते न पश्यन्ति भास्करिम्} %१४

\twolineshloka
{तुलसीकाष्ठसंयुक्तं गन्धं यो धारयेन्नरः}
{तद्देहं न स्पृशेत्पापं क्रियमाणं तथैव च} %१५

\twolineshloka
{तुलसीविपिनच्छाया यत्र चैव भवेद्द्विज}
{तत्र श्राद्धं प्रकर्तव्यं पितॄणां तृप्तिहेतवे} %१६

\twolineshloka
{यत्कृते तुलसीपत्रं कर्णे शिरसि दृश्यते}
{यमस्तं नेक्षितुं शक्तः किमु दूता भयङ्कराः} %१७

\twolineshloka
{तुलस्या महिमां यस्तु शृणुयान्नित्यमादृतः}
{सवपापविमुक्तात्मा ब्रह्मलोकं स गच्छति} %१८

\twolineshloka
{अत्रैवोदाहरन्तीममितिहासं पुरातनम्}
{तुलस्या विषये ब्रह्मञ्च्छ्रवणात्पापनाशनम्} %१९

\twolineshloka
{पुरा काश्मीरदेशे तु ब्राह्मणौ सम्बभूवतुः}
{हरिमेधःसुमेधाख्यौ विष्णुभक्तिपरायणौ} %२०

\twolineshloka
{सर्वभूतदयायुक्तौ सर्वतत्त्वार्थवेदिनौ}
{कदाचित्तौ द्विजवरौ तीर्थयात्रापरायणौ} %२१

\twolineshloka
{गच्छतावेकतो विप्रौ कान्तारे श्रमविह्वलौ}
{तुलसीकाननं तत्र ददर्शतुररिन्दमौ} %२२

\twolineshloka
{तयोः सुमेधास्तद्दृष्ट्वा तुलसीकाननं महत्}
{प्रदक्षिणीकृत्य तदा ववन्दे भक्तिसंयुतः} %२३

\twolineshloka
{दृष्ट्वैतद्धरिमेधास्तु उवाच परया मुदा}
{ज्ञातुं तुलस्या माहात्म्यं तत्फलं च पुनःपुनः} %२४

\uvacha{हरिमेधा उवाच}
\twolineshloka
{किमर्थं विप्र देवेषु तीर्थेषु च व्रतेषु च}
{स्थितेषु विप्रमुख्येषु प्रणामं कृतवानसि} %२५


\uvacha{सुमेधा उवाच}
\twolineshloka
{शृणु विप्र महाभाग साधु वाक्यमुदीरितम्}
{आतपो बाधते ह्यावां गत्वैतद्वटसन्निधौ} %२६

\twolineshloka
{तस्यच्छायां समाश्रित्य वक्ष्यामि ते यथार्थतः}
{एवमुक्तः सुमेधास्तु हरिमेधेन संयुतः} %२७

\twolineshloka
{वटं जगाम धर्मज्ञो महत्कोटरसंयुतम्}
{तत्र विश्राम्य विप्रोसौ हरिमेधमुवाच ह} %२८

\twolineshloka
{श्रूयतां विप्रशार्दूल तुलस्यास्तूत्तमां कथाम्}
{परमेशप्रसादेन सञ्जाता या पयोनिधौ} %२९

\twolineshloka
{पुरा दुर्वाससः शापाद्गतैश्वर्ये पुरन्दरे}
{ममन्थुः क्षीरजलधिं ब्रह्माद्याः ससुरासुराः} %३०

\twolineshloka
{ऐरावतः कल्पतरुश्चन्द्रमाः कमला तथा}
{उच्चैःश्रवा कौस्तुभश्च तथा धन्वन्तरिर्हरिः} %३१

\twolineshloka
{हरीतक्यादयश्चापि दिव्या ओषधयस्तथा}
{अजायन्त द्विजश्रेष्ठ लोकश्रेयोविधायकाः} %३२

\threelineshloka
{ततः पीयूषकलशमजरामरदायकम्}
{कराभ्यां कलशं विष्णुर्धारयन्सुतलं परम्}
{अवेक्ष्य मनसा सद्यः परां निर्वृतिमाप ह} %३३

\twolineshloka
{तस्मिन्पीयूषकलश आनन्दास्रोदबिन्दवः}
{व्यपतंस्तुलसी सद्यः समजायत मण्डला} %३४


\onelineshloka
{सर्व लक्षणसम्पन्ना सर्वाभरणभूषिता} %३५

\twolineshloka
{तत्रोत्पन्नां तथा लक्ष्मीं तुलसीं च ददुर्हरेः}
{देवा ब्रह्मादयस्ते हि जगृहे भगवान्हरिः} %३६


\onelineshloka
{ततोऽतीव प्रियकरा तुलसी जगतां पतेः} %३७

\twolineshloka
{सा तु देवगणैः सर्वैर्विष्णुवत्पूज्यते प्रिया}
{नारायणो जगत्त्राता तुलसी तस्य वल्लभा} %३८

\twolineshloka
{तस्मात्तस्या नमस्कारो मया विप्र कृतस्ततः}
{इत्येवं वदतस्तस्य सुमेधस्य महात्मनः} %३९

\twolineshloka
{आराददृश्यत महद्विमानं सूर्यवर्चसम्}
{तदानीं वटवृक्षस्तु पपात पुरतो मुने} %४०

\twolineshloka
{तथैव तस्माद्वृक्षाच्च पुरुषौ द्वौ विनिर्गतौ}
{द्योतयन्तौ दिशः सर्वास्तेजसा सूर्यसन्निभौ} %४१

\twolineshloka
{प्रणामं चक्रतुस्तौ हि हरिमेधसुमेधयोः}
{हरिमेधसुमेधौ तौ तौ दृष्ट्वा भयविह्वलौ} %४२


\onelineshloka
{ऊचतुर्विस्मयाविष्टौ तावुभौ देवसन्निभौ} %४३


\uvacha{हरिमेधसुमेधसावूचतुः}

\threelineshloka
{युवां कौ देवसङ्काशौ भवन्तौ सर्वमङ्गलौ}
{मन्दारमालां तरुणां धारयन्तौ तथाऽमरौ}
{नमस्कार्यौ तथाऽऽवाभ्यां पूज्यौ च सुररूपिणौ} %४४

\twolineshloka
{इत्युक्तौ ब्राह्मणाभ्यां तावूचतुर्वृक्षनिर्गतौ}
{युवामेव पिता माता आवयोश्च तथा गुरुः} %४५


\onelineshloka*
{बन्ध्वादयस्तथा चैव युवामेव न संशयः}

\uvacha{ज्येष्ठ उवाच}

\onelineshloka
{अहं तु देवलोकस्य आस्तीकोनाम नामतः} %४६

\twolineshloka
{अप्सरोगणसंवीतः कदाचिन्नन्दन वनम्}
{क्रीडार्थमगमं चाद्रौ विषयासक्त चेतनः} %४७

\twolineshloka
{रेमिरे देववनिता यथाकामं मया सह}
{मुक्तामल्लिकमाल्यानि निपेतुस्तानि योषिताम्} %४८

\twolineshloka
{तपतो रोमशस्यैव तद्दृष्ट्वा कुपितो मुनिः}
{योषितां नापराधोऽयं यासां वै परतन्त्रता} %४९

\twolineshloka
{अयमेव दुराचारः शापार्ह इति चाब्रवीत्}
{त्वं ब्रह्मराक्षसो भूत्वा वटवृक्षे चरेति माम्} %५०

\twolineshloka
{प्रसादितो मया सोऽथ विशापमपि दत्तवान्}
{तुलसीपत्रमाहात्म्यं विष्णोर्नाम तथा द्विजात्} %५१

\twolineshloka
{यदा शृणोषि सद्यस्त्वं विमुक्तिं यास्यसे पराम्}
{इति शप्तस्तु मुनिना चिरकालं सुदुःखितः} %५२

\twolineshloka
{वसाम्यत्र वटे दैवाद्भवद्दर्शनतो ध्रुवम्}
{मुक्तिर्जाता विप्रशापाद्द्वितीयस्य कथां शृणु} %५३

\twolineshloka
{अयं मुनिवरः पूर्वं गुरुशुश्रूषणे रतः}
{गुरोराज्ञामनादृत्य ब्रह्मराक्षसतां गतः} %५४

\twolineshloka
{युष्मत्प्रसादादधुना ब्रह्मशापाद्विमोचितः}
{तीर्थयात्राफलं चैव युवाभ्यामिह साधितम्} %५५

\twolineshloka
{उत्तरोत्तरपुण्यानि वर्धन्ते च दिनेदिने}
{इत्युक्त्वा तौ मुनिवरौ प्रणम्य च पुनःपुनः} %५६

\twolineshloka
{तावनुज्ञाप्य तौ धाम जग्मतुः परया मुदा}
{ततस्तौ तीर्थयात्रार्थं परमौ मुनिपुङ्गवौ} %५७

\twolineshloka
{शंसन्तौ तुलसीं पुण्यां जग्मतुर्मुनिपुङ्गव}
{एवं नारद माहात्म्यं तुलस्याः को नु वर्णयेत्} %५८

\twolineshloka
{तस्मान्नारद मासेऽस्मिन्कार्तिके हरितुष्टिदे}
{कर्तव्या तुलसीपूजा नात्र कार्या विचारणा} %५९

\twolineshloka
{एवमङ्ग व्रतान्येव प्रोक्तानि मुनिसत्तम}
{उपाङ्गानि प्रवक्ष्यामि वालखिल्योदितानि च} %६०


\iti{तुलसीमाहात्म्यवर्णनं}{नामाष्टमोऽध्यायः}{८}

\sect{अथ नवमोऽध्यायः}


\uvacha{वालखिल्या ऊचुः}
\twolineshloka
{कृष्णः प्रोवाच धर्माय द्वादशीं वत्ससंज्ञिताम्}
{गोधूलिकालसंयुक्ता द्वादशी वत्सपूजने} %१

\threelineshloka
{वत्सपूजा वटे चैव कर्तव्या प्रथमेऽहनि}
{सवत्सां तुल्यवर्णां च शीलिनीं गां पयस्विनीम्}
{चन्दनादिभिरालिप्य पुष्पमालाभिरर्चयेत्} %२

\twolineshloka
{तद्दिने तैलपक्वं च स्थालीपक्वं युधिष्ठिर}
{गोक्षीरं गोघृतं चैव दधिक्षीरं च वर्जयेत्} %३

\twolineshloka
{दिनान्ते सूर्यबिम्बार्धादुभयत्र घटीदलम्}
{ततो नीराजनं कार्य्यं निरीक्षेच्च शुभाशुभम्} %४

\twolineshloka
{नानादीपान्प्रकल्प्याऽऽदौ स्वर्णपात्रादि संस्थितान्}
{नीराजयेद्दीपपूर्वं निरीक्षेत शुभाशुभम्} %५

\twolineshloka
{लापयित्वा सर्वदीपानुत्तराभिमुखान्न्यसेत्}
{मुख्या दीपा नव प्रोक्ता अन्यानपि च कल्पयेत्} %६

\twolineshloka
{ज्वाला चेद्दक्षिणासंस्था सतेजस्का शिखान्विता}
{स्थिरा चेत्सौख्यदा प्रोक्ता विपरीता तु दुःखदा} %७

\twolineshloka
{कार्तिके कृष्णपत्रे तु द्वादश्यादिषु पञ्चसु}
{तिथिषूक्तः पूर्वरात्रे नृणां नीराजनाविधिः} %८

\threelineshloka
{पक्षं संसूचयत्यादिर्द्वितीयो मासमेव च}
{तृतीय ऋतुमेवेह चतुर्थस्त्वयनं तथा}
{वर्षं तु पञ्चमो दीपः शुभाशुभं विनिर्णयेत्} %९

\twolineshloka
{सूर्यांशसम्भवा दीपा अन्धकारविनाशकाः}
{त्रिकाले मां दीपयन्तु दिशन्तु च शुभाशुभम्} %१०


\onelineshloka
{अभिमन्त्र्य च मन्त्रेण ततो नीराजयेत्क्रमात्} %११

\twolineshloka
{आदौ देवांस्ततो विप्रान्हस्तिनश्च तुरङ्गमान्}
{ज्येष्ठाञ्च्छ्रेष्ठाञ्जघन्यांश्च मातृमुख्याश्च योषितः} %१२

\threelineshloka
{ततो नीराजितान्दीपान्स्वस्वस्थानेषु विन्यसेत्}
{रूक्षैर्लक्ष्मीविनाशः स्याच्छ्वेतैरन्नक्षयो भवेत्}
{अतिरक्तेषु युद्धानि मृत्युः कृष्णशिखेषु च} %१३

\twolineshloka
{एकाङ्गीनाम गोपाला तयैतच्च व्रतं कृतम्}
{धनधान्यसमायुक्ता जाता वर्षत्रयेण सा} %१४

\twolineshloka
{तस्माद्गोपूजनं कार्यं द्वादश्यां कार्तिकस्य तु}
{एतद्गोव्रतमाहात्म्यं श्रुत्वा कुर्वन्ति ये नराः} %१५

\twolineshloka
{ते गोव्रतप्रभावेन न गोभिर्विच्युता भुवि}
{गोऽपराधः कृतो यः स्यात्स व्रताद्विलयं व्रजेत्} %१६


\uvacha{वालखिल्या ऊचुः}
\twolineshloka
{कृष्णपक्षे चतुर्दश्यां मासि चाऽऽश्वयुजे तथा}
{दीपोत्सव समीपे तु व्रतमेतत्समाचरेत्} %१७

\twolineshloka
{प्रातः स्नात्वा त्रयोदश्यां कृत्वा वै दन्तधावनम्}
{त्रिरात्रनियमं कृत्वा गोविन्दे भक्तितत्परः} %१८

\twolineshloka
{कार्य एतद्व्रतस्यान्ते तथा गोवर्द्धनोत्सवः}
{त्रिमुहूर्ताधिका ग्राह्या परवेधो न दोषभाक्} %१९

\twolineshloka
{आश्विनस्यासिते पक्षे त्रयोदश्यां निशामुखे}
{यमदीपं बलिं दद्यादपमृत्युर्विनश्यति} %२०

\twolineshloka
{पुरा हेमनकस्यैव बालकश्चापमृत्युतः}
{मुक्तोभूदाश्विने कृष्णत्रयोदश्यां दयावशात्} %२१

\uvacha{दूता ऊचुः}
\twolineshloka
{यथा न जीविताद्भ्रश्येदीदृशे तु महोत्सवे}
{तथोपायं ब्रूहि यम कृपां कृत्वाऽस्मदग्रतः} %२२


\uvacha{यम उवाच}
\twolineshloka
{आश्विनस्यासिते पक्षे त्रयोदश्यां निशामुखे}
{प्रतिवर्षं तु यो दद्याद्गृहद्वारे सुदीपकम्} %२३

\twolineshloka
{मन्त्रेणानेन भो दूताः समानेयः स नोत्सवे}
{प्राप्तेऽपमृत्यावपि च शासनं क्रियतां मम} %२४

\twolineshloka
{मृत्युना पाशदण्डाभ्यां कालेन च मया सह}
{त्रयोदश्यां दीपदानात्सूर्यजः प्रीयतामिति} %२५

\twolineshloka
{मन्त्रेणानेन यो दीपं द्वारदेशे प्रयच्छति}
{उत्सवे चाऽपमृत्योश्च भयं तस्य न जायते} %२६


\uvacha{वालखिल्या ऊचुः}
\twolineshloka
{पूर्वविद्धचतुर्दश्यामाश्विनस्य सितेतरे}
{पक्षे प्रत्यूषसमये स्नानं कुर्यात्प्रयत्नतः} %२७

\twolineshloka
{अरुणोदयतोऽन्यत्र रिक्तायां स्नाति यो नरः}
{तस्याऽब्दिकभवो धर्मो नश्यत्येव न संशयः} %२८

\twolineshloka
{तथा कृष्णचतुर्दश्यामाश्विनेऽर्कोदये सुराः}
{यामिन्याः पश्चिमे यामे तैलाभ्यङ्गो विशिष्यते} %२९

\twolineshloka
{यदा चतुर्दशी न स्याद्द्विदिने चेद्विधूदये}
{दिनद्वये भवेच्चापि तदा पूर्वैव गृह्यते} %३०

\twolineshloka
{बलात्काराद्धठाद्वाऽपि शिष्टत्वान्न करोति चेत्}
{तैलाभ्यङ्गं चतुर्दश्यां रौरवं नरकं व्रजेत्} %३१

\twolineshloka
{तैले लक्ष्मीर्जले गङ्गा दीपावल्याश्चतुर्दशीम्}
{प्रातःस्नानं हि यः कुर्याद्यमलोकं न पश्यति} %३२

\twolineshloka
{अपामार्गमथो तुम्बीं प्रपुन्नाडमथापरम्}
{भ्रामयेत्स्नानमध्ये तु नरकस्य क्षयाय वै} %३३


\onelineshloka
{वारत्रयं त्रिवारं च पठित्वा मन्त्रमुत्तमम्} %३४

\threelineshloka
{सीतालोष्टसमायुक्त सकण्टकदलान्वित}
{हर पापमपामार्ग भ्राम्यमाणः पुनः पुनः}
{अपामार्गं प्रपुन्नाडं भ्रामयेच्छिरसोपरि} %३५

\threelineshloka
{स्नात्वार्द्रवाससा दद्याद्दीपकं मृत्युपुत्रयोः}
{शुनकौ श्यामशबलौ भ्रातरौ यमसेवकौ}
{तुष्टौ स्यातां चतुर्दश्यां दीपदानेन मृत्युजौ} %३६

\twolineshloka
{इष्टवन्धुजनैः सार्द्धमेतत्स्नानं समाचरेत्}
{स्नानाङ्गतर्पणं कृत्वा यमं सन्तर्पयेत्ततः} %३७

\twolineshloka
{यमाय धर्मराजाय मृत्यवे चान्तकाय च}
{वैवस्वताय कालाय सर्वभूतक्षयाय च} %३८

\twolineshloka
{औदुम्बराय दक्षाय नीलाय परमेष्ठिने}
{वृकोदराय चित्राय चित्रगुप्ताय ते नमः} %३९

\twolineshloka
{चतुर्दशैते मन्त्राः स्युः प्रत्येकं च नमोऽन्विताः}
{एकैकेन तिलैर्मिश्रान्दद्यात्त्रीनुदकाञ्जलीन्} %४०

\twolineshloka
{यज्ञोपवीतिना कार्यं प्राचीनावीतिनाऽथवा}
{देवत्वं च पितृत्वं च यमस्यास्ति द्विरूपता} %४१

\twolineshloka
{जीवत्पिताऽपि कुर्वीत तर्पणं यमभीष्मयोः}
{नरकाय प्रदातव्यो दीपः सम्पूज्य देवताः} %४२

\twolineshloka
{अत्रैव लक्ष्मीकामस्य विधिः स्नाने मयोच्यते}
{इषे भूते च दर्शे च कार्तिके प्रथमे दिने} %४३


\onelineshloka*
{यदा स्नाति तदाऽभ्यङ्गस्नानं कुर्याद्विधूदये}

\onelineshloka
{ऊर्ज्जशुक्लद्वितीयायां तिथौ च स्वातियुग्मगे} %४४

\twolineshloka
{मानवो मङ्गलस्नायी नैव लक्ष्म्या वियुज्यते}
{दीपैर्नीराजनादत्र सैषा दीपावलिः स्मृता} %४५

\twolineshloka
{इन्दुक्षयेऽपि सङ्क्रातौ रवौ पाते दिनक्षये}
{अत्राभ्यङ्गो न दोषाय प्रातः पापापनुत्तये} %४६

\twolineshloka
{माषपत्रस्य शाकं वै भुक्त्वा तस्मिन्दिने नरः}
{प्रेताख्यायां चतुर्दश्यां सर्वपापैः प्रमुच्यते} %४७

\twolineshloka
{इषासितचतुर्दश्यामिन्दुक्षयतिथावपि}
{दर्शादौ स्वातिसंयुक्ते तदा दीपावलिर्भवेत्} %४८

\twolineshloka
{कुर्यात्संलग्नमेतच्च दीपोत्सवदिनत्रयम्}
{महाराजो बलिः प्रोक्तस्तुष्टेन हरिणा तथा} %४९

\twolineshloka
{वरं याचस्व भद्रं ते यद्यन्मनसि वर्तते}
{इति विष्णुवचः श्रुत्वा बलिर्वचनमब्रवीत्} %५०

\twolineshloka
{आत्मार्थं किं याचनीयं सर्वं दत्तं मया तथा}
{लोकार्थं याचयिष्यामि शक्तश्चेद्देहि तच्च मे} %५१

\twolineshloka
{मयाऽद्य ते धरा दत्ता वामनच्छद्मरूपिणे}
{त्रिभिः पदैस्त्रिदिवसैः सा चाऽऽक्रान्ता यतस्त्वया} %५२


\onelineshloka
{तस्माद्भूमितले राज्यमस्तु घस्रत्रये हरे} %५३

\twolineshloka
{मद्राज्ये ये दीपदानं भुवि कुर्वन्ति मानवाः}
{तेषां गृहे तव स्त्रीयं सदा तिष्ठतु सुस्थिरा} %५४

\twolineshloka
{मम राज्ये गृहे यैषामन्धकारः पतिष्यति}
{लक्ष्मीसन्तानान्धकारः सदा पततु तद्गृहे} %५५

\twolineshloka
{चतुर्दश्यां च ये दीपान्नरकाय ददन्ति च}
{तेषां पितृगणाः सर्वे नरके न वसन्ति च} %५६

\twolineshloka
{बलिराज्यं समासाद्य यैर्न दीपावलिः कृता}
{तेषां गृहे कथं दीपाः प्रज्वलिष्यन्ति केशव} %५७

\twolineshloka
{बलिराज्ये तु ये लोकाः शोकाऽनुत्साहकारिणः}
{तेषां गृहे सदा शोकः पतेदिति न संशयः} %५८

\threelineshloka
{चतुर्दशीत्रये राज्यं बलेरस्त्विति याचयेत्}
{पुरा वामनरूपेण प्रार्थयित्वा धरामिमाम्}
{ददावतिथयेन्द्राय बलिं पातालवासिनम्} %५९

\twolineshloka
{दत्तं दैत्यपतेरित्थं हरिणा तद्दिनत्रयम्}
{तस्मान्महोत्सवं चात्र सर्वथैव हि कारयेत्} %६०

\twolineshloka
{महारात्रिः समुत्पन्ना चतुर्दश्यां मुनीश्वराः}
{अतस्तदुत्सवः कार्यः शक्तिपूजापरायणैः} %६१

\twolineshloka
{बलिराज्यं समासाद्य यक्षगन्धर्वकिन्नराः}
{औषध्यश्च पिशाचाश्च मन्त्राश्च मणयस्तथा} %६२

\twolineshloka
{सर्व एव प्रहृष्यन्ति नृत्यन्ति च निशामुखे}
{तत्तन्मन्त्राश्च सिद्ध्यन्ति बलिराज्ये न संशयः} %६३

\twolineshloka
{बलिराज्यं समासाद्य यथा लोकाः सुहर्षिताः}
{तथा तद्दिनमध्ये तु लोकाः स्युर्हर्षिता भृशम्} %६४

\twolineshloka
{तुलासंस्थे सहस्रांशौ प्रदोषे भूतदर्शयोः}
{उल्काहस्ता नराः कुर्युः पितॄणां मार्गदर्शनम्} %६५

\twolineshloka
{नरकस्थास्तु ये प्रेतास्ते मार्गं तु व्रतात्सदा}
{पश्यन्त्येव न सन्देहः कार्योऽत्र मुनिपुङ्गवैः} %६६

\twolineshloka
{आश्विने मासि भूतादितिथयः कीर्तितास्त्रयः}
{दीपदानादिकार्येषु ग्राह्या मध्याह्नकालिकाः} %६७

\twolineshloka
{यदि स्युः सङ्गवादर्वागेताश्च तिथयस्त्रयः}
{दीपदानादिकार्येषु कर्तव्याः पूर्वसंयुताः} %६८


\uvacha{ऋषय ऊचुः}
\twolineshloka
{कौमोदिन्यास्तु माहात्म्यं प्रष्टुमिच्छामहे द्विजाः}
{तस्मिन्दिने तु किं भोज्यं कस्य पूजां तु कारयेत्} %६९

\twolineshloka
{किमर्थं क्रियते सा तु तस्याः का देवता भवेत्}
{किं च तत्र भवेद्देयं किं न देयं विशेषतः} %७०

\twolineshloka
{प्रहर्षः कोऽत्र निर्दिष्टः क्रीडा काऽत्र प्रकीर्तिता}
{दीपावल्याः फलं सर्वं वदन्तु ऋषिसत्तमाः} %७१


\uvacha{वालखिल्या ऊचुः}
\twolineshloka
{ततः प्रभात समये त्वमायां तु मुनीश्वराः}
{स्नात्वा देवान्पितॄन्भक्त्या सम्पूज्याथ प्रणम्य च} %७२

\twolineshloka
{कृत्वा तु पार्वणश्राद्धं दधिक्षीरघृतादिभिः}
{दिवा तत्र न भोक्तव्यमृते बालातुराज्जनात्} %७३

\twolineshloka
{ततः प्रदोषसमये पूजयेदिन्दिरां शुभाम्}
{कुर्यान्नानाविधैर्वस्त्रैः स्वच्छं लक्ष्म्याश्च मण्डपम्} %७४

\twolineshloka
{नानापुष्पैः पल्लवैश्च चित्रैश्चापि विचित्रितम्}
{तत्र सम्पूजयेल्लक्ष्मीं देवांश्चापि प्रपूजयेत्} %७५

\twolineshloka
{सम्पूज्या देवनार्योऽपि वहुभिश्चोपचारकैः}
{पादसंवाहनं कुर्याल्लक्ष्म्यादीनां तु भक्तितः} %७६

\twolineshloka
{अस्मिन्नहनि सर्वेऽपि विष्णुना मोचिताः पुरा}
{बलिकारागृहाद्देवा लक्ष्मीश्चापि विमोचिता} %७७

\twolineshloka
{लक्ष्म्या सार्द्धं ततो देवा जग्मुः क्षीरोदधौ पुनः}
{प्रसुप्ता बहुकालं ते सुखं तस्मान्मुनीश्वराः} %७८

\twolineshloka
{रचनीयाः सूत्रगर्भाः पर्यङ्काश्च सुतूलिकाः}
{दुग्धफेनोपमैर्वस्त्रैरास्तृताश्च यथादिशम्} %७९

\twolineshloka
{स्थापयेत्तान्सुराँल्लक्ष्मीं वेदघोषसमन्वितः}
{लक्ष्मीर्दैत्यभयान्मुक्ता सुखं सुप्ताम्बुजोदरे} %८०

\twolineshloka
{अतोऽत्र विधिवत्कार्या तुष्ट्यै तु सुखसुप्तिका}
{तदह्नि पद्मशय्यां यः पद्मासौख्यविवृद्धये} %८१

\twolineshloka
{कुर्यात्तस्य गृहं मुक्त्वा तत्पद्मा क्वापि न व्रजेत्}
{न कुर्वन्ति नरा इत्थं लक्ष्म्या ये सुखसुप्तिकाम्} %८२

\twolineshloka
{धनचिन्ता विहीनास्ते कथं रात्रौ स्वपन्ति हि}
{तस्मात्सर्वप्रयत्नेन लक्ष्मीं सम्पूजयेन्नरः} %८३

\twolineshloka
{स तु दारिद्र्यनिर्मुक्तः स्वजातौ स्यात्प्रतिष्ठितः}
{जातिपत्रलंवगैलात्वक्कर्पूरसमन्वितम्} %८४

\twolineshloka
{पाचयित्वा गव्यदुग्धं सितां दत्त्वा यथोचिताम्}
{लड्डुकांस्तस्य कुर्वीत तांश्च लक्ष्म्यै समर्पयेत्} %८५

\twolineshloka
{अन्यच्चतुर्विधं भक्ष्यं दद्याच्छ्रीः प्रीयतामिति}
{अप्रबुद्धे हरौ पूर्वं स्त्रीभिर्लक्ष्मीं प्रबोधयेत्} %८६

\twolineshloka
{प्रबोधसमये लक्ष्मीं बोधयित्वा भुनक्ति या}
{पुमान्वा वत्सरं यावल्लक्ष्मीस्तं नैव मुञ्चति} %८७

\twolineshloka
{अभयं प्राप्य विप्रेभ्यो विष्णुभीताः सुरद्विषः}
{क्षीराब्धौ तुष्टुवुर्ज्ञात्वा सुप्तां पद्माश्रितां श्रियम्} %८८

\twolineshloka
{त्वं ज्योतिः श्रीरवीन्द्वग्निविद्युत्सौवर्णतारकाः}
{सर्वेषां ज्योतिषां ज्योतिर्दीपज्योतिःस्थिते नमः} %८९

\twolineshloka
{या लक्ष्मीर्दिवसे पुण्ये दीपावल्यां च भूतले}
{गवां गोष्ठे तु कार्तिक्यां सा लक्ष्मीर्वरदा मम} %९०

\twolineshloka
{दीपदानं ततः कुर्यात्प्रदोषे च तथोल्मुकम्}
{भ्रामयेत्स्वस्य शिरसि सर्वारिष्टनिवारणम्} %९१

\twolineshloka
{दीपवृक्षास्तथा कार्याः शक्त्या देवगृहादिषु}
{चतुष्पथे श्मशाने च नदीपर्वतवेश्मसु} %९२

\twolineshloka
{वृक्षमूलेषु गोष्ठेषु चत्वरेषु गृहेषु च}
{वस्त्रैः पुष्पैः शोभितव्या राजमार्गस्य भूमयः} %९३

\twolineshloka
{सर्वं पुरमलङ्कृत्य प्रदोषे तदनन्तरम्}
{ब्राह्मणान्भोजयित्वाऽऽदौ सम्भोज्य च बुभुक्षितान्} %९४

\twolineshloka
{अलङ्कृतेन भोक्तव्यं नववस्त्रोपशोभिना}
{ततोऽपराह्णसमये घोषयेन्नगरं नृपः} %९५

\twolineshloka
{अद्य राज्यं बलेर्लोका यथेच्छं क्रीड्यतामिति}
{यथेच्छं क्रीडतां बाला इत्याज्ञाप्य नृपेण तु} %९६

\twolineshloka
{तेभ्यो दद्यात्क्रीडनकं ततः पश्येच्छुभाशुभ्य}
{बलिराज्ये प्रकर्तव्यं यद्यन्मनसि वर्तते} %९७

\threelineshloka
{जीवहिंसा सुरापानमगम्यागमनं तथा}
{चौर्यं विश्वासघातश्च पञ्चैतानि मुनीश्वराः}
{बलिराज्ये तु नरकद्वाराण्युक्तानि सन्त्यजेत्} %९८

\threelineshloka
{ततोऽर्द्धरात्रसमये स्वयं राजा व्रजेत्पुरम्}
{अवलोकयितुं रम्यं पद्भ्यामेव शनैःशनैः}
{बलिराज्यप्रमोदं च दृष्ट्वा स्वगृहमाव्रजेत्} %९९

\threelineshloka
{एवं गते निशीथे च जने निद्रार्द्धलोचने}
{एवं नगरनारीभिः शूर्पडिण्डिमवादनैः}
{निष्कास्यते प्रदृष्टाभिरलक्ष्मीः स्वगृहाङ्गणात्} %१००

\twolineshloka
{दण्डैकरजनीयोगे दर्शः स्यात्तु परेऽहनि}
{तदा विहाय पूर्वेद्युः परेऽह्नि सुखरात्रिका} %१०१

\twolineshloka
{ये वैष्णवाऽवैष्णवाश्च बलिराज्योत्सवं नराः}
{न कुर्वन्ति वृथा तेषां धर्माः स्युर्नात्र संशयः} %१०२

\twolineshloka
{रात्रौ जागरणं कुर्यात्पुराणपठनादिभिः}
{द्यूतेन वा हरेरग्रे गीतया वा तथैव च} %१०३


\iti{वत्सद्वादशीयमत्रयोदशीनरकचतुर्दशीदीपावलीकृत्यवर्णनं}{नाम नवमोऽध्यायः}{९}

\sect{अथ दशमोऽध्यायः}


\uvacha{ब्रह्मोवाच}
\twolineshloka
{प्रतिपद्यथ चाऽभ्यङ्गं कृत्वा नीराजनं ततः}
{सुवेषः सत्कथागीतैर्दानैश्च दिवसं नयेत्} %१

\twolineshloka
{शङ्करस्तु पुरा द्यूतं ससर्ज सुमनोहरम्}
{कार्तिके शुक्लपक्षे तु प्रथमेऽहनि सत्यवत्} %२

\twolineshloka
{बलिराज्यदिनस्याऽपि माहात्म्यं शृणु तत्त्वतः}
{स्नातव्यं तिलतैलेन नरैर्नारीभिरेव च} %३

\twolineshloka
{यदि मोहान्न कुर्वीत स याति यमसादनम्}
{पुरा कृतयुगस्यादौ दानवेन्द्रो बलिर्महान्} %४

\twolineshloka
{तेन दत्ता वामनाय भूमिः स्वमस्तकान्विता}
{तदानीं भगवान्साक्षात्तुष्टो बलिमुवाच ह} %५

\twolineshloka
{कार्तिके मासि शुक्लायां प्रतिपद्यां यतो भवान्}
{भूमिं मे दत्तवान्भक्त्या तेन तुष्टोऽस्मि तेऽनघ} %६

\twolineshloka
{वरं ददामि ते राजन्नित्युक्त्वाऽदाद्वरं तदा}
{त्वन्नाम्नैव भवेद्राजन्कार्तिकी प्रतिपत्तिथिः} %७

\twolineshloka
{एतस्यां ये करिष्यन्ति तैलस्नानादिकार्चनम्}
{तदक्षयं भवेद्राजन्नात्र कार्या विचारणा} %८

\twolineshloka
{तदाप्रभृति लोकेऽस्मिन्प्रसिद्धा प्रतिपत्तिथिः}
{प्रतिपत्पूर्वविद्धा नो कर्तव्या तु कथञ्चन} %९

\twolineshloka
{तत्राभ्यङ्गं न कुर्वीत अन्यथा मृतिमाप्नुयात्}
{प्रतिपद्यां यदा दर्शो मुहूर्तप्रमितो भवेत्} %१०

\twolineshloka
{माङ्गल्यं तद्दिने चेत्स्याद्वित्तादिस्तस्य नश्यति}
{बलेश्च प्रतिपद्दर्शाद्यदि विद्धं भविष्यति} %११

\twolineshloka
{तस्यां यद्यथ चाऽऽर्तिक्यं नारी मोहात्करिष्यति}
{नारीणां तत्र वैधव्यं प्रजानां मरणं ध्रुवम्} %१२

\twolineshloka
{अविद्धा प्रतिपच्चेत्स्यान्मुहूर्तमपरेऽहनि}
{उत्सवादिककृत्येषु सैव प्रोक्ता मनीषिभिः} %१३

\twolineshloka
{प्रतिपत्स्वल्पमात्राऽपि यदि न स्यात्परेऽहनि}
{पूर्वविद्धा तदा कार्या कृता नो दोषभाग्भवेत्} %१४

\twolineshloka
{तद्दिने गृहमध्ये तु कुर्यान्मूर्तिं तदाङ्गणे}
{गोमयेन च तत्रापि दधि तत्पुरतः क्षिपेत्} %१५

\twolineshloka
{आर्तिक्यं तत्र संस्थाप्य एवं कुर्याद्विधानतः}
{अभ्यङ्गं ये न कुर्वन्ति तस्यां तु मुनिपुङ्गव} %१६

\twolineshloka
{न माङ्गल्यं भवेत्तेषां यावत्स्याद्वत्सरं ध्रुवम्}
{यो यादृशेन रूपेण तस्यां तिष्ठेच्छुभे दिने} %१७

\twolineshloka
{आवर्षं तद्भवेत्तस्य तस्मान्मङ्गलमाचरेत्}
{यदीच्छेत्स्वशुभान्भोगान्भोक्तुं दिव्यान्मनोहरान्} %१८

\twolineshloka
{कुरु दीपोत्सवं रम्यं त्रयोदश्यादिकेषु च}
{शङ्करश्च भवानी च क्रीडया द्यूतमास्थिते} %१९

\twolineshloka
{गौर्या जित्वा पुरा शम्भुर्नग्नो द्यूते विसर्जितः}
{अतोऽर्थं शङ्करो दुःखी गौरी नित्यं सुखस्थिता} %२०

\twolineshloka
{द्यूतं निषिद्धं सर्वत्र हित्वा प्रतिपदं बुधाः}
{प्रथमं विजयो यस्य तस्य संवत्सरं सुखम्} %२१

\twolineshloka
{भवान्याऽभ्यर्थिता लक्ष्मीर्धेनुरूपेण संस्थिता}
{प्रातर्गोवर्द्धनः पूज्यो द्यूतं रात्रौ समाचरेत्} %२२


\onelineshloka
{भूषणीयास्तदा गावो वर्ज्या वहनदोहनात्} %२३

\twolineshloka
{गोवर्द्धन धराधार गोकुलत्राणकारक}
{विष्णुबाहुकृतोच्छ्राय गवां कोटिप्रदो भव} %२४

\twolineshloka
{या लक्ष्मीर्लोकपालानां धेनुरूपेण संस्थिता}
{घृतं वहति यज्ञार्थे मम पापं व्यपोहतु} %२५

\twolineshloka
{अग्रतः सन्तु मे गावो गावो मे सन्तु पृष्ठतः}
{गावो मे हृदयं सन्तु गवां मध्ये वसाम्यहम्} %२६


॥इति गोवर्द्धनपूजा॥
\twolineshloka
{सद्भावेनैव सन्तोष्य देवान्सत्पुरुषान्नरान्}
{इतरेषामन्नपानैर्वाक्यदानेन पण्डितान्} %२७

\twolineshloka
{वस्त्रैस्ताम्बूलधूपैश्च पुष्पकर्पूरकुङ्कुमैः}
{भक्ष्यैरुच्चावचैर्भोज्यैरन्तःपुरनिवासिनः} %२८

\threelineshloka
{ग्राम्यान्वृषभदानैश्च सामन्तान्नृपतिर्धनैः}
{पदातिजनसङ्घांश्च ग्रैवेयैः कटकैः शुभैः}
{स्वनामाङ्कैश्च तान्राजा तोषयेत्सज्जनान्पृथक्} %२९

\twolineshloka
{यथार्थं तोषयित्वा तु ततो मल्लान्नरांस्तथा}
{वृषभान्महिषांश्चैव युध्यमानान्परैः सह} %३०

\twolineshloka
{राज्ञस्तथैव योधांश्च पदातीन्समलङ्कृतान्}
{मञ्चाऽऽरूढः स्वयं पश्येन्नटनर्तकचारणान्} %३१

\twolineshloka
{युद्धापयेद्वासयेच्च गोमहिष्यादिकं च यत्}
{वत्सानाकर्षयेद्गोभिरुक्तिप्रत्युक्तिवादनात्} %३२

\twolineshloka
{ततोऽपराह्नसमये पूर्वस्यां दिशि सुव्रत}
{मार्गपालीं प्रबध्नाति दुर्गस्तम्भेऽथ पादपे} %३३

\twolineshloka
{कुशकाशमयीं दिव्यां लम्बकैर्बहुभिः प्रिये}
{वीक्षयित्वा गजानश्वान्मार्गपाल्यास्तले नयेत्} %३४

\twolineshloka
{गावो वृषांश्च महिषान्महिषीर्घटकोत्कटान्}
{कृतहोमैर्द्विजेन्द्रैस्तु बध्नीयान्मार्गपालिकाम्} %३५

\threelineshloka
{नमस्कारं ततः कुर्यान्मन्त्रेणानेन सुव्रत}
{मार्गपालि नमस्तुभ्यं सर्वलोकसुखप्रदे}
{तले तव सुखेनाश्वा गजा गावश्च सन्तु मे} %३६

\twolineshloka
{मार्गपालीतले पुत्र यान्ति गावो महावृषाः}
{राजानो राजपुत्राश्च ब्राह्मणाश्च विशेषतः} %३७

\twolineshloka
{मार्गपाली समुल्लङ्घ्य नीरुजः सुखिनो हि ते}
{कृत्वैतत्सर्वमेवेह रात्रौ दैत्यपतेर्बलेः} %३८

\twolineshloka
{पूजां कुर्यात्ततः साक्षाद्भूमौ मण्डलके कृते}
{बलिमालिख्य दैत्येन्द्रं वर्णकैः पञ्चरङ्गकैः} %३९

\twolineshloka
{सर्वाभरणसम्पूर्णं विन्ध्यावलिसमन्वितम्}
{कूष्माण्डमयजम्भोरुमधुदानवसंवृतम्} %४०

\twolineshloka
{सम्पूर्णं कृष्टवदनं किरीटोत्कटकुण्डलम्}
{द्विभुजं दैत्यराजानं कारयित्वा स्वके पुनः} %४१

\twolineshloka
{गृहस्य मध्ये शालायां विशालायां ततोऽर्चयेत्}
{मातृभ्रातृजनैः सार्द्धं सन्तुष्टो बन्धुभिः सह} %४२

\twolineshloka
{कमलैः कुमुदैः पुष्पैः कह्लारै रक्तकोत्पलैः}
{गन्धपुष्पान्ननैवेद्यैः सक्षीरैर्गुडपायसैः} %४३

\threelineshloka
{मद्यमांससुरालेह्यचोष्यभक्ष्योपहारकैः}
{मन्त्रेणानेन राजेन्द्र समन्त्री सपुरोहितः}
{पूजां करिष्यते यो वै सौख्यं स्यात्तस्य वत्सरम्} %४४

\twolineshloka
{बलिराज नमतुभ्यं विरोचनसुत प्रभो}
{भविष्येन्द्र सुराराते पूजेयं प्रतिगृह्यताम्} %४५

\twolineshloka
{एवं पूजाविधानेन रात्रौ जागरणं ततः}
{कारयेद्वै क्षणं रात्रौ नटनृत्यकथानकैः} %४६

\twolineshloka
{लोकश्चापि गृहस्यान्ते सपर्यां शुक्लतन्दुलैः}
{संस्थाप्य बलिराजानं फलैः पुष्पैः प्रपूजयेत्} %४७

\twolineshloka
{बलिमुद्दिश्य वै तत्र कार्यं सर्व च सुव्रत}
{यानि यान्यक्षयाण्याहुर्मुनयस्तत्त्वदर्शिनः} %४८

\twolineshloka
{यदत्र दीयते दानं स्वल्पं वा यदि वा बहु}
{तदक्षयं भवेत्सर्वं विष्णोः प्रीतिकरं शुभम्} %४९

\twolineshloka
{रात्रौ ये न करिष्यन्ति तव पूजां बले नराः}
{तेषां च श्रोत्रियो धर्मः सर्वस्त्वामुपतिष्ठतु} %५०

\twolineshloka
{विष्णुना च स्वयं वत्स तुष्टेन बलये पुनः}
{उपकारकरं दत्तमसुराणां महोत्सवम्} %५१

\twolineshloka
{एकमेवमहोरात्रं वर्षेवर्षे च कार्तिके}
{दत्तं दानवराजस्य आदर्शमिव भूतले} %५२

\twolineshloka
{यः करोति नृपो राज्ये तस्य व्याधिभयं कुतः}
{सुभिक्षं क्षेममारोग्यं तस्य सम्पदनुत्तमा} %५३


\onelineshloka
{नीरुजश्च जनाः सर्वे सर्वोपद्रववर्जिताः} %५४

\threelineshloka
{कौमुदी क्रियते यस्माद्भावं कर्तुं महीतले}
{यो यादृशेन भावेन तिष्ठत्यस्यां च सुव्रत}
{हर्षदुःखादिभावेन तस्य वर्षं प्रयाति हि} %५५

\twolineshloka
{रुदिते रोदितं वर्षं प्रहृष्टे तु प्रहर्षितम्}
{भुक्तौ भोग्यं भवेद्वर्षं स्वस्थे स्वस्थं भविष्यति} %५६


\onelineshloka
{वैष्णवी दानवी चेयं तिथिः प्रोक्ता च कार्तिके} %५७

\fourlineindentedshloka
{दीपोत्सवं जनितसर्वजनप्रमोदं}
{कुर्वन्ति ये शुभतया बलिराजपूजाम्}
{दानोपभोगसुखबुद्धिमतां कुलानां}
{हर्षं प्रयाति सकलं प्रमुदा च वर्षम्} %५८


\onelineshloka
{बलिपूजां विधायैवं पश्चाद्गोक्रीडनं चरेत्} %५९

\twolineshloka
{गवां क्रीडादिने यत्र रात्रौ दृश्येत चन्द्रमाः}
{सोमो राजा पशून्हन्ति सुरभीपूजकांस्तथा} %६०

\twolineshloka
{प्रतिपद्दर्शसंयोगे क्रीडनं तु गवां मतम्}
{परविद्धासु यः कुर्यात्पुत्रदारधनक्षयः} %६१

\threelineshloka
{अलङ्कार्यास्तदा गावो गोग्रासादिभिरर्चिताः}
{गीतवादित्रनिर्घोषैर्नयेन्नगरबाह्यतः}
{आनीय च ततः पश्चात्कुर्यान्नीराजनाविधिम्} %६२

\twolineshloka
{अथ चेत्प्रतिपत्स्वल्पा नारी नीराजनं चरेत्}
{द्वितीयायां ततः कुर्यात्सायं मङ्गलमालिकाः} %६३

\twolineshloka
{एवं नीराजनं कृत्वा सर्वपापैः प्रमुच्यते}
{प्रतिपत्पूर्वविद्धैव यष्टिकाकर्षणे भवेत्} %६४

\twolineshloka
{कुशकाशमयीं कुर्याद्यष्टिकां सुदृढां नवाम्}
{देवद्वारे नृपद्वारेऽथवाऽऽनेया चतुष्पथे} %६५

\twolineshloka
{तामेकतो राजपुत्रा हीनवर्णास्तथैकतः}
{गृहीत्वा कर्षयेयुस्ते यथासारं मुहुर्मुहुः} %६६

\twolineshloka
{समसङ्ख्या द्वयोः कार्या सर्वेऽपि बलवत्तराः}
{जयोऽत्र हीनजातीनां जयो राज्ञस्तु वत्सरम्} %६७

\twolineshloka
{उभयोः पृष्ठतः कार्या रेखा तत्कर्षकोपरि}
{रेखान्ते यो नयेत्तस्य जयो भवति नान्यथा} %६८


\onelineshloka
{जयचिह्नमिदं राजा निदधीत प्रयत्नतः} %६९


\iti{कार्तिकशुक्लप्रतिपन्माहात्म्यवर्णनं}{नाम दशमोऽध्यायः}{१०}

\sect{अथ नामैकादशोऽध्यायः}


\uvacha{नारद उवाच}
\twolineshloka
{भगवन्प्रष्टुमिच्छामि त्वामहं विनयान्वितः}
{तद्व्रतं ब्रूहि मे मर्त्यो मृत्युं येन न पश्यति} %१


\uvacha{ब्रह्मोवाच}
\twolineshloka
{यदि पृच्छसि विप्रेन्द्र व्रतानामुत्तमं व्रतम्}
{व्रतं यमद्वितीयाख्यं शृणु त्वं मृत्युनाशनम्} %२

\twolineshloka
{कार्तिके मासि शुद्धायां द्वितीयायां मुनीश्वर}
{कर्तव्यं तद्विधानेन सर्वमृत्युनिवारणम्} %३

\twolineshloka
{ब्राह्मे मुहूर्ते चोप्थाय द्वितीयायां मुनीश्वर}
{मनसा चिन्तयेदात्महितं नैवाहितं स्मरेत्} %४


प्रातःस्नानं ततः कुर्याद्दन्तधावनपूर्वकम्॥
ततः शुक्लाम्बरधरः शुक्लमाल्यानुलेपनः॥५॥
\twolineshloka
{कृतनित्यक्रियो हृष्टः कुण्डलाङ्गदभूषितः}
{औदुम्बर तरुं गत्वा कृत्वा मण्डलमुत्तमम्} %६

\twolineshloka
{पद्ममष्टदलं कृत्वा तस्मिन्नौदुम्बरे शुभे}
{विधिं विष्णुं च रुद्रं च वरदां च सरस्वतीम्} %७

\twolineshloka
{वीणापुस्तकसंयुक्तां पूजयेत्स्वस्थमानसः}
{चन्दनागरुकस्तूरीकुङ्कुमैर्द्विजसत्तम} %८

\twolineshloka
{पुष्पैर्धूपैश्च नैवेद्यैर्नारिकेलफलादिभिः}
{ततो मृत्युविनाशाथ सालङ्कारां पयस्विनीम्} %९

\twolineshloka
{विप्राय वेदविदुषे गां दद्याच्च सवत्सकाम्}
{अपमृत्युविनाशार्थं संसारार्णवतारकाम्} %१०

\twolineshloka
{हे विप्र ते त्विमां सौम्यां धेनुं सम्प्रददाम्यहम्}
{इति मन्त्रेण गां दद्याद्विप्राय ब्रह्मवादिने} %११

\twolineshloka
{तदलाभे तु विप्राय भक्त्या दद्यादुपानहौ}
{ततः पूजां समाप्याथ भक्तिमान्पुरुषोत्तमे} %१२

\twolineshloka
{ज्ञातिश्रेष्ठान्वयोवृद्धान्सम्यग्भक्त्याऽभिवादयेत्}
{नानाविधैः फलै रम्यैस्तर्प्पयेत्स्वजनानपि} %१३

\twolineshloka
{ततः सोदरसम्पन्ना भगिनी या भवेन्मुने}
{तस्या गृहं समागत्य सम्यग्भक्त्याऽभिवादयेत्} %१४

\twolineshloka
{भगिनि सुभगे भद्रे त्वदङ्घ्रिसरसीरुहम्}
{श्रेयसेऽथ नमस्कर्तुमागतोऽस्मि तवालयम्} %१९

\twolineshloka
{इत्युक्त्वा भगिनीं तां तु विष्णुबुद्ध्याऽभिवादयेत्}
{तदा तु भगिनी श्रुत्वा भ्रातुर्वचनमुत्तमम्} %१६

\twolineshloka
{भगिन्या भ्रातरं वाक्यं वक्तव्यं प्रति नारद}
{अद्य भ्रातरहं जाता त्वत्तो धन्याऽस्मि मङ्गला} %१७

\twolineshloka
{भोक्तव्यं तेऽद्य मद्गेहे स्वायुषे कुलदीपक}
{कार्तिके शुक्लपक्षस्य द्वितीयायां सहोदर} %१८

\threelineshloka
{यमो यमुनया पूर्वं भोजितः स्वगृहेर्चितः}
{अस्मिन्दिने यमेनापि नारकीयाश्च मोचिताः}
{अपि बद्धाः कर्मपाशैः स्वेच्छया पर्यटन्ति ते} %१९

\fourlineindentedshloka
{स्वसुर्नरो वेश्मनि यो न भुङ्क्ते}
{यमद्वितीयादिनमत्र लब्ध्वा}
{तं पापिनं प्राप्य वयं सुहृष्टाः}
{प्रभक्षयामोऽद्य च भक्ष्यहीनाः} %२०

\twolineshloka
{इति पापा रटन्तीह ब्रह्महत्यादयस्तथा}
{तस्माद्भ्रातर्मद्गृहे तु भोजनं कुरु कार्तिके} %२१


\twolineshloka
{शुक्लायां तु द्वितीयायां विश्रुतायां जगत्त्रये}
{अस्यां निजगृहे पुत्र भुज्यते न बुधैरपि} %२२

\twolineshloka
{इत्युक्तः स तथेत्युक्ता भगिनीं पूजयेद्व्रती}
{प्रहर्षास्तुमहाभाग वस्त्रालङ्कारभूषणैः} %२३

\twolineshloka
{अग्रजामभिवन्द्याऽथ आशिषं च प्रगृह्य च}
{सर्वा भगिन्यः सन्तोष्या वस्त्रालङ्कारदानतः} %२४

\twolineshloka
{अभावे स्वस्य तु स्वसुः पितृव्याः स्वपितुः स्वसा}
{तस्या गृहं समागत्य कुर्याद्भोजनमादरात्} %२५

\twolineshloka
{एवं यः कुरुते पुत्र द्वितीयां यमनामिकाम्}
{अपमृत्युविनिर्मुक्तः पुत्रपौत्रादिभिर्वृतः} %२६

\twolineshloka
{इह भुक्त्वा तु विपुलान्भोगानन्यान्यथेप्सितान्}
{अन्ते मोक्षमवाप्नोति नान्यथा मद्वचो भवेत्} %२७

\twolineshloka
{व्रतान्येतानि सर्वाणि दानानि विविधानि च}
{गृहस्थस्यैव युज्यन्ते तस्माद्गार्हस्थ्यमाश्रयेत्} %२८

\twolineshloka
{कथां यमद्वितीयाया व्रतस्थः शृणुयान्नरः}
{तस्य सर्वाणि पापानि नश्यन्तीत्याह माधवः} %२९


\uvacha{सूत उवाच}
\twolineshloka
{कार्तिके च द्वितीयायां पूर्वाह्णे यममर्चयेत्}
{भानुजायां नरः स्नात्वा यमलोकं न पश्यति} %३०

\twolineshloka
{कार्तिके शुक्लपक्षे तु द्वितीयायां तु शौनक}
{यमो यमुनया पूर्वं भोजितः स्वगृहेऽर्चितः} %३१

\twolineshloka
{द्वितीयायां महोत्सर्गो नारकीयाश्च तर्पिताः}
{पापेभ्यो विप्रयुक्तास्ते मुक्ताः सर्वे निबन्धनात्} %३२

\twolineshloka
{अत्राऽऽशिताश्च सन्तुष्टाः स्थिताः सर्वे यदृच्छया}
{तेषां महोत्सवो वृत्तो यमराष्ट्रसुखावहः} %३३

\twolineshloka
{अतो यमद्वितीयेयं त्रिषु लोकेषु विश्रुता}
{तस्मान्निजगृहे विप्र न भोक्तव्यं ततो बुधैः} %३४

\twolineshloka
{स्नेहेन भगिनीहस्ताद्भोक्तव्यं बलवर्धनम्}
{ऊर्जे शुक्लद्वितीयायां पूजितस्तर्पितो यमः} %३९

\twolineshloka
{महिषासनमारूढो दण्डमुद्गरभृत्प्रभुः}
{वेष्टितः किङ्करैर्हृष्टैस्तस्मै याम्यात्मने नमः} %३६

\twolineshloka
{यैर्भगिन्यः सुवासिन्यो वस्त्रदानादितोषिताः}
{न तेषां वत्सरं यावत्कलहो न रिपोर्भयम्} %३७

\twolineshloka
{धन्यं यशस्यमायुष्यं धर्मकामार्थसाधनम्}
{व्याख्यातं सकलं पुत्र सरहस्यं मयाऽनघ} %३८

\fourlineindentedshloka
{यस्यां तिथौ यमुनया यमराजदेवः}
{सम्भोजितः प्रतितिथौ स्वसृसौहृदेन}
{तस्मात्स्वसुः करतलादिह यो भुनक्ति}
{प्राप्नोति वित्तशुभसम्पदमुत्तमां सः} %३९

\uvacha{सूत उवाच}
\twolineshloka
{विशेषश्चात्र सम्प्रोक्तो वालखिल्यैर्महर्षिभिः}
{तदहं सम्प्रवक्ष्यामि शृणुध्वं मुनिसत्तमाः} %४०


\uvacha{वालखिल्या ऊचुः}
\twolineshloka
{कार्तिकस्य सिते पक्षे द्वितीया यमसंज्ञिता}
{तत्राऽपराह्ने कर्तव्यं सर्वथैव यमार्चनम्} %४१

\twolineshloka
{प्रत्यहं यमुनाऽऽगत्य यमं सम्प्रार्थयत्पुरा}
{भ्रातर्मम गृहे याहि भोजनार्थं गणावृतः} %४२

\twolineshloka
{अद्यश्वो वा परश्वो वा प्रत्यहं वदते यमः}
{कार्यव्याकुलचित्तानामवकाशो न जायते} %४३

\twolineshloka
{तदैकदा यमुनया बलात्कारान्निमन्त्रितः}
{स गतः कार्तिके मासि द्वितीयायां मुनीश्वराः} %४४

\twolineshloka
{नारकीयजनान्मुक्त्वा गणैः सह रवेः सुतः}
{कृताऽऽतिथ्यो यमुनया नानापाकाः कृताः खग} %४५

\twolineshloka
{कृताभ्यङ्गो यमुनया तैलैर्गन्धमनोहरैः}
{उद्वर्तनं लापयित्वा स्नापितः सूर्यनन्दनः} %४६

\twolineshloka
{ततोऽलङ्कारकं दत्तं नानावस्त्राणि चन्दनम्}
{माल्यानि च प्रदत्तानि मञ्चोपरि उपाविशत्} %४७

\twolineshloka
{पक्वान्नानि विचित्राणि कृत्वा सा स्वर्णभाजने}
{यमायाभोजयद्देवी यमुना प्रीतमानसा} %४८

\threelineshloka
{भुक्त्वा यमोऽपि भगिनीमलङ्कारैः समर्चयत्}
{नानावस्त्रैस्ततः प्राह वरं वरय भामिनि}
{इति तद्वचनं श्रुत्वा यमुना वाक्यमब्रवीत्} %४९


\uvacha{यमुनोवाच}

\onelineshloka
{प्रतिवर्षं समागच्छ भोजनार्थं तु मद्गृहे} %५०

\threelineshloka
{अद्य सर्वे मोचनीयाः पापिनो नरकाद्यम}
{येऽद्यैव भगिनीहस्तात्करिष्यन्ति च भोजनम्}
{तेषां सौख्यं प्रदेहि त्वमेतदेव वृणोम्यहम्} %५१


\uvacha{यम उवाच}

\onelineshloka
{यमुनायां तु यः स्नात्वा सन्तर्प्य पितृदेवताः} %५२

\twolineshloka
{भुङ्क्ते च भगिनीगेहे भगिनीं पूजयेदपि}
{कदाचिदपि मद्वारं न स पश्यति भानुजे} %५३

\twolineshloka
{वीरेशैशानदिग्भागे यमतीर्थं प्रकीर्तितम्}
{तत्र स्नात्वा च विधिवत्सन्तर्प्य पितृदेवताः} %५४

\twolineshloka
{पठेदेतानि नामानि आमध्याह्नं नरोत्तमः}
{सूर्यस्याभिमुखो मौनी हृतचित्तः स्थिरासनः} %५५

\fourlineindentedshloka
{यमो निहन्ता पितृधर्मराजो}
{वैवस्वतो दण्डधरश्च कालः}
{भूताधिपो दत्तकृतानुसारी}
{कृतान्तमेतद्दशभिर्जपन्ति} %५६

\twolineshloka
{ततो यमेश्वरं पूज्य भगिनीगृहमाव्रजेत्}
{मन्त्रेणानेन च तया भोजितः पूर्वमादरात्} %५७

\twolineshloka
{भ्रातस्तवानुजाताऽहं भुङ्क्ष्व भक्तमिदं शुभम्}
{प्रीतये यमराजस्य यमुनाया विशेषतः} %५८

\twolineshloka
{ततः सन्तोष्य भगिनीं वस्त्रालङ्करणादिभिः}
{स्वप्नेऽपि यमलोकस्य भविष्यति न दर्शनम्} %५९

\twolineshloka
{नृपैः कारागृहे ये च स्थापिता मम वासरे}
{अवश्यं ते प्रेषणीया भोजनार्थं स्वसुर्गृहे} %६०

\twolineshloka
{विमोक्तव्या मया पापा नरकेभ्योऽद्य वासरे}
{येऽद्य बन्दी करिष्यन्ति ते ताड्या मम सर्वथा} %६१

\twolineshloka
{कनीयसी स्वसा नास्ति तदा ज्येष्ठागृहं व्रजेत्}
{तदभावे सपत्यायाः पितृव्यजागृहे ततः} %६२

\twolineshloka
{तदभावे मातृष्वसुर्मातुलस्याऽऽत्मजा तथा}
{सापत्नगोत्रसम्बन्धैः कल्पयेदथवा क्रमम्} %६३

\twolineshloka
{सर्वाभावे माननीया भगिनी काचिदेव हि}
{गोनद्याद्यथवा तस्या अभावे सति कारयेत्} %६४

\twolineshloka
{तदभावेप्यरण्यानीं कल्पयित्वा सहोदराम्}
{अस्यां निजगृहे देवि न भोक्तव्यं कदाचन} %६५

\twolineshloka
{ये भुञ्जते दुराचारा नरके ते पतन्ति च}
{एवमुक्त्वा धर्मराजो ययौ संयमिनीं ततः} %६६

\twolineshloka
{तस्मादृषिवराः सर्वे कार्तिकव्रतकारिणः}
{भुञ्जते भगिनीहस्तात्सत्यं सत्यं न संशयः} %६७

\twolineshloka
{यमद्वितीयां यः प्राप्य भगिनीगृहभोजनम्}
{न कुर्याद्वर्षजं पुण्यं नश्यतीति रवेः श्रुतिः} %६८

\twolineshloka
{या तु भोजयते नारी भ्रातरं भ्रातृके तिथौ}
{अर्चयेच्चापि ताम्बूलैर्न सा वैधव्यमाप्नुयात्} %६९

\twolineshloka
{भ्रातुरायुःक्षयो नूनं न भवेत्तत्र कर्हिचित्}
{अपराह्नव्यापिनी सा द्वितीया भ्रातृभोजने} %७०

\twolineshloka
{अज्ञानाद्यदि वा मोहान्न भुक्तं भगिनीगृहे}
{प्रवासिना ह्यभावाद्वा ज्वरितेनाथ बन्दिना} %७१

\twolineshloka
{एतदाख्यानकं श्रुत्वा भोजनस्य फलं भवेत्}
{कार्तिके तु विशेषेण धात्रीछायां समाश्रितः} %७२


\onelineshloka
{भोजनं कुरुते यस्तु स वैकुण्ठमवाप्नुयात्} %७३


\iti{यमद्वितीयामाहात्म्यवर्णनं}{नामैकादशोऽध्यायः}{११}

\sect{अथ द्वादशोऽध्यायः}


\uvacha{शौनक उवाच}
\twolineshloka
{कार्तिकस्य च माहात्म्यं महत्पुण्यफलप्रदम्}
{कदा धात्री समुत्पन्ना कथं सा ख्यातिमागता} %१

\twolineshloka
{कस्मादियं पवित्रा च कस्मात्पापप्रणाशिनी}
{आमर्दकी कृता केन कथयस्वात्र विस्तरात्} %२


\uvacha{सूत उवाच}
\twolineshloka
{कथयामि द्विजश्रेष्ठ यथा चेयं हि पुण्यदा}
{ऊर्जशुक्लचतुर्दश्यां धात्रीपूजां समाचरेत्} %३

\twolineshloka
{आमर्दकीमहावृक्षः सर्वपापप्रणाशनः}
{वैकुण्ठाख्यचतुर्दश्यां धात्रीछायां गतो नरः} %४

\twolineshloka
{पूजयेत्तत्र देवेशं राधया सहितं हरिम्}
{प्रदक्षिणां ततः कुर्याच्छतमष्टोत्तरं तथा} %५

\twolineshloka
{सुवर्णरजतैर्वापि फलैरामलकैस्तथा}
{शतमष्टोत्तरं कुर्यादेकैकेन प्रदक्षिणाम्} %६

\twolineshloka
{साष्टाङ्गं प्रणतो भूत्वा प्रार्थयेत्परमेश्वरम्}
{धात्रीछायां समाश्रित्य शृणुयाच्च कथामिमाम्} %७

\twolineshloka
{ब्राह्मणान्भोजयेत्पश्चाद्यथाशक्त्या च दक्षिणाम्}
{ब्राह्मणेषु च तुष्टेषु तुष्टो मोक्षप्रदो हरिः} %८

\twolineshloka
{अत्र ते कथयिष्यामि कथां पुण्यफलप्रदाम्}
{आमर्दकीफलं वक्तुं ब्रह्मा चापि न पार्यते} %९


\twolineshloka
{एकार्णवे पुरा जाते नष्टे स्थावरजङ्गमे}
{नष्टे देवासुरगणे प्रणष्टोरगराक्षसे} %१०

\twolineshloka
{तत्र देवाधिदेवेशः परमात्मा सनातनः}
{जजाप ब्रह्म परममात्मनः परमाव्ययम्} %११

\twolineshloka
{ततोऽस्य ब्रह्म जपतो निरगाच्छ्वसितं पुरः}
{तद्दर्शनानुरागेण नेत्राभ्यामगमज्जलम्} %१२

\twolineshloka
{प्रेमाश्रुभरनिर्भिन्नो भूमौ बिन्दुः पपात सः}
{तस्माद्बिन्दो समुत्पन्नः स्वयं धात्री नगो महान्} %१३

\twolineshloka
{शाखाप्रशाखाबहुलः फलभारेण पीडितः}
{सर्वेषामेव वृक्षाणामादिरोहः प्रकीर्तितः} %१४

\twolineshloka
{ब्रह्मा तमसृजत्पूर्वं तत्पश्चाच्चासृजत्प्रजाः}
{देवदानवगन्धर्वयक्षराक्षसपन्नगान्} %१५

\twolineshloka
{असृजद्भगवान्देवो मानुषांश्च तथामलान्}
{आजग्मुस्तत्र देवास्ते यत्र धात्री हरिप्रिया} %१६

\twolineshloka
{तां दृष्ट्वा ते महाभागाः परमं विस्मयं गताः}
{न जानीम इमं वृक्षं चिन्तयन्तो मुहुर्मुहुः} %१७

\twolineshloka
{एवं चिन्तयतां तेषां वागुवाचाशरीरिणी}
{आमर्दकी नगो ह्येष प्रवरो वैष्णवो यतः} %१८

\twolineshloka
{अस्य वै स्मरणादेव लभेद्गोदानजं फलम्}
{दर्शनाद्द्विगुणं पुण्यं त्रिगुणं भक्षणात्तथा} %१९

\twolineshloka
{तस्मात्सर्वप्रयत्नेन सेव्या आमर्दकी सदा}
{सर्वपापहरा प्रोक्ता वैष्णवी पापनाशिनी} %२०

\twolineshloka
{तस्या मूले स्थितो विष्णुस्तदूर्ध्वं च पितामहः}
{स्कन्धे च भगवान्रुद्रः संस्थितः परमेश्वरः} %२१

\twolineshloka
{शाखासु सवितारश्च प्रशाखासु च देवताः}
{पर्णेषु देवताः सन्ति पुष्पेषु मरुतस्तथा} %२२

\twolineshloka
{प्रजानां पतयः सर्वे फलेष्वेवं व्यवस्थिताः}
{सर्वदेवमयी ह्येषा धात्री वै कथिता मया} %२३

\threelineshloka
{अतः सा पूजनीया च सर्वकामार्थसिद्धये}
{एकदा नारदो योगी ब्रह्मणः पुरतः स्थितः}
{नमस्कृत्वा जगन्नाथं पप्रच्छातीव विस्मितः} %२४


\uvacha{श्रीनारद उवाच}
\twolineshloka
{यथा प्रियं सुतुलसीकाननं सर्वदा हरेः}
{तथा धात्रीवनं मासे कार्तिके श्रीहरिप्रियम्} %२५


\uvacha{ब्रह्मोवाच}
\twolineshloka
{धात्रीवने हरेः पूजा धात्री छायासु भोजनम्}
{कार्तिके मासि यः कुर्यात्तस्य पापं विनश्यति} %२६

\twolineshloka
{तीर्थानि मुनयो देवा यज्ञाः सर्वेऽपि कार्तिके}
{नित्यं धात्रीं समाश्रित्य तिष्ठन्त्यर्के तुलास्थिते} %२७

\twolineshloka
{यत्किञ्चित्कुरुते पुण्यं धात्रीछायासु मानवः}
{तत्कोटिगुणितं भूयान्नात्र कार्या विचारणा} %२८


\onelineshloka
{अत्रैवोदाहरन्तीममितिहासं पुरातनम्} %२९

\twolineshloka
{अयोध्यानगरे कश्चिद्वैश्यश्चासीद्विजोत्तम}
{पुत्रदारविहीनश्च दैवाद्दारिद्र्यपीडितः} %३०

\twolineshloka
{भिक्षया चोदराग्निं स शमयामास नारद}
{कदाचिद्वणिजो वैश्यो ययाचे क्षुत्प्रपीडितः} %३१

\twolineshloka
{भिक्षाप्तचणकान्गृह्य धात्रीछायामगात्किल}
{तत्र तान्भक्षयामास कार्तिके मासि नारद} %३२

\twolineshloka
{केचिदुर्वरितास्तेपु चणकास्तत्र नारद}
{वैश्येन तेन दत्ता हि क्षुत्क्षामाय द्विजातये} %३३

\twolineshloka
{तेन पुण्यप्रभावेन राजाऽऽसीद्धनिकः क्षितौ}
{तस्माद्दानं प्रकर्तव्यं कार्तिके मासि सर्वदा} %३४

\threelineshloka
{धात्रीवने मुनिश्रेष्ठ सर्वकामार्थसिद्धये}
{धात्रीछायां समाश्रित्य कार्तिके च हरेः कथाम्}
{यः शृणोति स पापेभ्यो मुच्यते द्विजसूनुवत्} %३५

\uvacha{नारद उवाच}
\twolineshloka
{कोऽभूद्द्विजसुतो ब्रह्मन्किं पापं कृतवान्पुरा}
{तस्य जाता कथं मुक्तिरेतद्विस्तरतो वद} %३६


\uvacha{ब्रह्मोवाच}

\onelineshloka
{पुरा द्विजवरश्चासीत्कावेर्या उत्तरे तटे} %३७

\twolineshloka
{देवशर्मेति विख्यातो वेदवेदाङ्गपारगः}
{तस्य पुत्रो दुराचारस्तमाह च पिता हितम्} %३८

\twolineshloka
{इदानीं कार्तिको मासो वर्तते हरिवल्लभः}
{तत्र स्नानं च दानं च व्रतानि नियमान्कुरु} %३९

\twolineshloka
{तुलसीपुष्पसहितां कुरुपूजां हरेः सुत}
{दीपदानं च विविधं नमस्कारं प्रदक्षिणाम्} %४०

\twolineshloka
{एवं पितुर्वचः श्रुत्वा पुत्रः क्रोधसमन्वितः}
{पितरं प्राह दुष्टात्मा चलदोष्ठो विनिन्दयन्} %४१


\uvacha{पुत्र उवाच}
\twolineshloka
{न करिष्याम्यहं तात कार्तिके पुण्यसङ्ग्रहम्}
{इति पुत्रवचः श्रुत्वा सक्रोधः प्राह तं सुतम्} %४२

\twolineshloka
{मूषको भव दुर्बुद्धे वने वृक्षस्य कोटरे}
{इति शापभयाद्भीतो नत्वा पितरमब्रवीत्} %४३

\twolineshloka
{दुर्योनेर्मम मुक्तिः स्यात्कथं तद्वद मे गुरो}
{इति प्रसादितो विप्रः प्राह निष्कृतिकारणम्} %४४

\twolineshloka
{यदोर्ज्जव्रतजं पुण्यं शृणोषि हरिवल्लभम्}
{तदा ते भविता मुक्तिस्तत्कथाश्रवणात्सुत} %४५

\twolineshloka
{स पित्रा चैवमुक्तस्तु तत्क्षणान्मूषकोऽभवत्}
{बहुवर्षसहस्राणि गह्वरे विपिने वसन्} %४६

\twolineshloka
{एकदा कार्तिके मासि विश्वामित्रः सशिष्यकः}
{स्नात्वा नद्यां हरिं चार्च्य धात्रीछायां समाश्रितः} %४७

\twolineshloka
{कथयामास माहात्म्यं शिष्येभ्यश्चोर्ज्जसम्भवम्}
{तदा कश्चिद्दुराचारो व्याधोऽगान्मृगयां चरन्} %४८

\twolineshloka
{दृष्ट्वा ऋषिगणान्हन्तुं कृतेच्छः प्राणिघातकः}
{तेषां दर्शनमात्रेण सुबुद्धिरभवत्तदा} %४९

\twolineshloka
{अथोवाच द्विजान्नत्वा भवद्भिः क्रियतेऽत्र किम्}
{तेनैवमुक्तो विप्रेन्द्रो विश्वामित्रस्तमब्रवीत्} %५०


\uvacha{विश्वामित्र उवाच}
\twolineshloka
{सर्वेषामेव मासानां कार्तिकः श्रेष्ठ उच्यते}
{तस्मिन्यत्क्रियते कर्म वर्धते वटबीजवत्} %५१

\twolineshloka
{कार्तिके मासि यः कुर्यात्स्नानं दानं च पूजनम्}
{विप्राणां भोजनं चैव तदक्षय्यफलं भवेत्} %५२

\twolineshloka
{व्याधप्रयुक्तमाकर्ण्य धर्मं च ऋषिणा द्विजः}
{मौषकं देहमुत्सृज्य दिव्यदेहोऽभवत्तदा} %५३

\twolineshloka
{विश्वामित्रं प्रणम्याथ स्ववृत्तान्तं निवेद्य च}
{अनुज्ञातोऽथ ऋषिणा विमानस्थो दिवं ययौ} %५४

\twolineshloka
{विस्मितो गाधिपुत्रस्तु व्याधश्चैव विशेषतः}
{व्याधोऽप्यूर्जव्रतं कृत्वा जगाम हरिमन्दिरम्} %५५

\twolineshloka
{तस्मात्सर्वप्रयत्नेन कार्तिके केशवाग्रतः}
{धात्रीछायां समाश्रित्य कथाश्रवणमाचरेत्} %५६

\twolineshloka
{मूषकोऽपि च दुर्योनेर्मुक्त ऊर्जकथाश्रुतेः}
{शृणुयाच्छ्रावयेद्यो वा मुक्तिभागी न संशयः} %५७

\threelineshloka
{धात्रीछायां समाश्रित्य वनभोजनमाचरेत्}
{आदौ कृत्वा तथा स्नानमुदके वनसंस्थिते}
{कृत्वा कर्माणि नित्यानि माधवं पूजयेत्ततः} %५८

\twolineshloka
{धात्रीछायां समाश्रित्य हरौ भक्तिसमन्वितः}
{शृणुयाच्च कथां दिव्यां मासमाहात्म्यशंसनीम्} %५९

\twolineshloka
{ततस्तु ब्राह्मणान्भक्त्या भोजयेद्ब्रह्मवित्तमान्}
{ततो भुञ्जीत विप्रेन्द्र स्वयं हरिमनुस्मरन्} %६०

\twolineshloka
{एवङ्कृते व्रते विप्र कार्तिके हरिवल्लभे}
{यत्पापं नश्यते पुत्र सावधानमना शृणु} %६१

\twolineshloka
{हरेर्नार्पितभोगाच्च भोजने सूर्यदर्शनात्}
{रजस्वलावाक्छ्रवणपापाद्भोजनके तथा} %६२

\twolineshloka
{भोजनावसरे चान्यस्पर्शदोषस्तु यद्भवेत्}
{निषिद्धभोजनात्तस्माद्भोजने चान्नदूषणात्} %६३

\twolineshloka
{शुद्धस्यापि तथा त्यागात्पुण्यकाले हरिप्रिये}
{एतैर्यत्साधितं पापं तत्सर्वं नश्यति धुवम्} %६४


\onelineshloka
{तस्मात्सर्वप्रयत्नेन धात्र्यां भोजनमाचरेत्} %६५

\twolineshloka
{कार्तिके मासि वै विप्रो धात्रीमालां तु यो वहेत्}
{तथैव तुलसीमालां तस्य पुण्यमनन्तकम्} %६६

\twolineshloka
{धात्रीछायां समाश्रित्य दीपमालार्पणं नरः}
{करिष्यति विशेषेण तस्य पुण्यमनन्तकम्} %६७

\twolineshloka
{राधादामोदरौ पूज्यौ तुलस्यधो विशेषतः}
{तुलस्यभावे कर्तव्या पूजा धात्रीतले शुभा} %६८

\twolineshloka
{धात्रीछायातले येन सकृद्भुक्तं तु कार्तिके}
{दम्पत्योर्भोजनं दत्तमन्नदोषात्प्रमुच्यते} %६९

\threelineshloka
{सम्पूर्णे कार्तिके यस्तु सम्पूज्यामलकीं शुभाम्}
{राधादामोदरप्रीत्यै भोजयित्वा च दम्पती}
{पश्चात्स्वयं तु भुञ्जीत न श्रीस्तस्य क्षयं व्रजेत्} %७०

\twolineshloka
{यः कश्चिद्वैष्णवो लोके धत्ते धात्रीफलं मुने}
{प्रियो भवति देवानां मनुष्याणां च का कथा} %७१

\twolineshloka
{धात्रीफलविलिप्ताङ्गो धात्रीफलसमन्वितः}
{धात्रीफलकृताहारो नरो नारायणो भवेत्} %७२

\twolineshloka
{धात्रीफलानि यो नित्यं वहते करसम्पुटे}
{तस्य नारायणो देवो वरमिष्टं प्रयच्छति} %७३

\twolineshloka
{श्रीकामः सर्वदा स्नानं कुर्यादामलकैर्नरः}
{तुष्यत्यामलकैर्विष्णुरेकादश्यां विशेषतः} %७४

\twolineshloka
{नवम्यां दर्शे सप्तम्यां सङ्क्रान्तौ रविवासरे}
{चन्द्रसूर्योपरागे च स्नानमामलकैस्त्यजेत्} %७५

\twolineshloka
{धात्रीछायां समाश्रित्य कुर्य्यात्पिण्डं त यो नरः}
{प्रयान्ति पितरो मुक्तिं प्रसादान्माधवस्य तु} %७६

\twolineshloka
{मूर्ध्नि पाणौ मुखे चैव बाह्वोः कण्ठे तु यो नरः}
{धत्ते धात्रीफलं वत्स धात्रीफलविभूषितः} %७७

\twolineshloka
{यावल्लुठति कण्ठस्था धात्रीमाला नरस्य हि}
{तावत्तस्य शरीरे तु प्रीत्या लुण्ठति केशवः} %७८

\twolineshloka
{धात्रीफलं च तुलसी मृत्तिका द्वारकोद्भवा}
{सफलं जीवितं तस्य त्रितयं यस्य वेश्मनि} %७९


\twolineshloka
{यावद्दिनानि वहते धात्रीमालां कलौ नरः}
{तावद्युगसहस्राणि वैकुण्ठे वसतिर्भवेत्} %८०

\twolineshloka
{मालायुग्मं वहेद्यस्तु धात्रीतुलसिसम्भवम्}
{यो नरः कण्ठदेशे तु कल्पकोटिं दिवं वसेत्} %८१

\twolineshloka
{धात्रीछायां गतो यस्तु द्वादश्यां पूजयेद्धरिम्}
{तत्रैव भोजनं यस्तु ब्राह्मणानां च कारयेत्} %८२

\twolineshloka
{स्वयं च तत्र भुङ्क्ते यः सूपभक्षादिकं तथा}
{न तस्य पुनरावृत्तिः कल्पकोटिशतैरपि} %८३


\onelineshloka
{तुलस्याश्चैव धात्र्याश्च फलैः पत्रैर्हरिं यजेत्} %८४

\twolineshloka
{तुलसी धात्रीयुक्ता हि सिक्ते सति च कार्तिके}
{विलयं यान्ति पापानि ब्रह्महत्यादिकानि च} %८५


\onelineshloka
{धर्मदत्तो द्विजः पूर्वं यथा मुक्तिमवाप ह} %८६


\uvacha{नारद उवाच}
\threelineshloka
{कार्तिके मासि सा सेव्या पूजनीया सदा नरैः}
{चातुर्मास्ये न सेव्या सा इत्युक्तं भवता पुरा}
{तस्मात्सर्वमशेषेण कथयस्व ममाग्रतः} %८७


\uvacha{ब्रह्मोवाच}
\threelineshloka
{कार्तिके मासि विप्रर्षे शुक्ला या दशमी शुभा}
{तद्दिनाऽऽरभ्य सा सेव्या दैवे पित्र्ये च कर्मणि}
{दशम्यारभ्य तत्पत्रैः फलकैर्मधुसूदनम्} %८८

\twolineshloka
{पूजयन्ति नरा ये वै ते वै वैकुण्ठगामिनः}
{समाप्ते कार्तिकव्रते वनभोजनमाचरेत्} %८९

\twolineshloka
{दशम्यां वाऽथ द्वादश्यां पौर्णमास्यामथापि वा}
{पञ्चम्यां वा महाभाग वनभोजनमाचरेत्} %९०

\twolineshloka
{सर्वोपस्करसंयुक्तो वृद्धबालैश्च संयुतः}
{वनं प्रवेशयेद्धीमान्धात्रीवृक्षैः सुशोभितम्} %९१

\twolineshloka
{चूतैर्बकैस्तथाऽश्वत्थैः पिचुमन्दैः कदम्बकैः}
{न्यग्रोधतिन्तिणीवृक्षैः समन्तात्परिशोभितम्} %९२

\twolineshloka
{तत्र गत्वा महाप्राज्ञ पुण्याहं कारयेत्पुरा}
{वास्तुपीठं तथा पूज्यं धात्रीमूले तु कारयेत्} %९३

\twolineshloka
{वेदिकां चतुरस्रां च हस्तमात्रायतां शुभाम्}
{तथोपवेदिकां कृत्वा वेदिकाग्रे महामते} %९४

\twolineshloka
{उपवेशाय देवस्य ह्यलं कार्यं तु धातुभिः}
{वेदिकापश्चिमेभागे कारयेत्कुण्डमण्डपम्} %९५

\twolineshloka
{मेखलात्रयसंयुक्तं पिप्पलच्छदसंयुतम्}
{हस्तमात्रायतं सौम्य एवं कुण्डं तु कारयेत्} %९६

\twolineshloka
{पश्चात्स्नात्वा ततो जप्त्वा देवपूजां समाचरेत्}
{पश्चादग्निं समाधाय होमं कुर्याद्यथाविधि} %९७

\twolineshloka
{पायसाऽऽज्यगुडसूपपालाशसमिधा तथा}
{ग्रहाणां वास्तुदेवेभ्यश्चरुं कृत्वा प्रयत्नतः} %९८

\twolineshloka
{धात्री शान्तिस्तथा कान्तिर्माया प्रकृतिरेव च}
{विष्णुपत्नी महालक्ष्मी रमा मा कमला तथा} %९९

\twolineshloka
{इन्दिरा लोकमाता च कल्याणी कमला तथा}
{सावित्री च जगद्धात्री गायत्री सुधृतिस्तथा} %१००

\twolineshloka
{अन्तज्ञा विश्वरूपा च सुकृपा ह्यब्धिसम्भवा}
{प्रधानदेवताभिस्तु रक्षाहोमं समारभेत्} %१

\twolineshloka
{संसृष्टेति च मन्त्रेण ऋषभं मेति मन्त्रतः}
{अपूपं गुडसूपाभ्यां संयुतं जुहुयाद्धविः} %२

\twolineshloka
{अष्टोत्तरशतं हुत्वा मूलमन्त्रेण पायसम्}
{ततो ग्रहादि देवांस्तु यथासङ्ख्येन होमयेत्} %३

\twolineshloka
{धात्रीहोमे महाप्राज्ञ रक्षाहोमे तु पायसम्}
{ततः स्विष्टकृतं हुत्वा बलिदानं समाचरेत्} %४

\twolineshloka
{इन्द्रादि लोकपालांश्च रक्षा पूज्या प्रयत्नतः}
{धात्रीवृक्षस्य सर्वत्र वेदिकासंयुतस्य च} %५

\twolineshloka
{सूपेन गुडमिश्रेण बलिं पश्चान्निवेदयेत्}
{देवि धात्रि नमस्तुभ्यं गृहाण बलिमुत्तमम्} %६

\twolineshloka
{मिश्रितं गुडसूपाभ्यां सर्वमङ्गलदायिनि}
{पुत्रान्देहि महाप्राज्ञान्यशो देहि शुभप्रदम्} %७

\twolineshloka
{प्रज्ञां मेधां च सौभाग्यं विष्णुभक्तिं च देहि मे}
{नीरोगं कुरु मे नित्यं निष्पापं कुरु सर्वदा} %८

\twolineshloka
{वर्चस्कं कुरु मां देवि धनवन्तं तथा कुरु}
{इति तां प्रार्थयेद्देवीं प्रादक्षिण्याद्बलिं न्यसेत्} %९

\twolineshloka
{बलिप्रदानकाले तु ये कुर्वन्ति प्रदक्षिणम्}
{ते यान्ति विष्णुसालोक्यं पितृभिः सार्द्धमेव च} %११०


\onelineshloka
{ततः पूर्णाहुतिं कृत्वा होमशेषं समापयेत्} %११

\twolineshloka
{धात्रीवृक्षस्य मूलस्थं मन्दस्मितरमापतिम्}
{ते यान्ति विष्णुसायुज्यं ये पश्यन्तीह चक्षुषा} %१२

\twolineshloka
{वैश्वदेवं ततः कृत्वा पूजयेद्वनदेवताः}
{गन्धाक्षतांस्ततो दत्त्वा विप्रेभ्यः पद्मसम्भव} %१३

\twolineshloka
{ब्राह्मणान्भोजयेत्पश्चात्स्वयं भुञ्जीत बन्धुभिः}
{गृहं प्रवेशयेत्पश्चाद्वृद्धान्बालादिकैः सह} %१४

\twolineshloka
{ब्रह्मचारी भवेद्रात्रौ क्षितिशायी भवेत्ततः}
{ग्रामस्थैश्च मिलित्वा च स्वयं वा कारयेद्बुधः} %१५

\twolineshloka
{सर्वपापविमुक्त्यर्थं वनभोजनमुत्तमम्}
{कृत्वैवं सकलं कर्म कृष्णाय च समर्पयेत्} %१६

\twolineshloka
{अश्वमेधसहस्रस्य राजसूयशतस्य च}
{यत्फलं समवाप्नोति तत्फलं वनभोजने} %१७

\twolineshloka
{अतो धात्री महाभाग पवित्रा पापनाशनी}
{धात्री चैव नृणां धात्री धात्रीवत्कुरुते क्रियाम्} %१८

\threelineshloka
{ददात्यायुः पयःपानात्स्नानाद्वै धर्मसञ्चयम्}
{अलक्ष्मीनाशनं स्नानमात्रैर्निर्वाणमाप्नुयात्}
{विघ्नानि नैव जायन्ते धात्रीस्नानेन वै नृणाम्} %१९

\twolineshloka
{तस्मात्त्वं कुरु विप्रेन्द्र धात्रीस्नानं हि यत्नतः}
{प्रयास्यसि हरेर्द्धाम देवत्वं प्राप्य नारद} %१२०

\twolineshloka
{यत्रयत्र मुनिश्रेष्ठ धात्रीस्नानं समाचरेत्}
{तीर्थे वाऽपि गृहे वाऽपि तत्रतत्र हरिः स्थितः} %२१

\twolineshloka
{धात्रीस्नानेन विप्रर्षे यस्यास्थीनि कलेवरे}
{प्रक्षाल्यन्ते मुनिश्रेष्ठ न स गर्भगृहं वसेत्} %२२

\twolineshloka
{धात्रीजलेन विप्रेन्द्र येषां केशाश्च रञ्जिताः}
{ते नराः केशवं यान्ति नाशयित्वा कलेर्मलम्} %२३

\twolineshloka
{धात्रीफलं महापुण्यं स्नानं पुण्यतमं स्मृतम्}
{पुण्यात्पुण्यतरं वत्स भक्षणे मुनिसत्तम} %२४

\twolineshloka
{न गङ्गा न गया काशी न वेणी न च पुष्करम्}
{एकैव हि यथा पुण्या धात्री माधववासरे} %२५

\twolineshloka
{धात्रीस्नानं हरेर्नाम तथैवैकादशी सुत}
{गयाश्राद्धं तथा वत्स समानि मुनयो विदुः} %२६

\twolineshloka
{संस्पृशन्यस्तु वै धात्रीमहन्यहनि मानवः}
{मुच्यते पातकैः सर्वैर्मनोवाक्कायसम्भवैः} %२७

\twolineshloka
{धात्रीफलैरमावास्यासप्तमीनवमीषु च}
{रविवारे च सङ्क्रान्तौ न स्नायान्मुनिसत्तम} %२८

\twolineshloka
{यस्मिन्गृहेमुनिवर धात्री तिष्ठति सर्वदा}
{तस्मिन्गृहे न गच्छन्ति प्रेतकूष्माण्डराक्षसाः} %२९

\twolineshloka
{धात्रीफलकृतां मालां कण्ठस्थां यो वहेन्नहि}
{स वैष्णवो न विज्ञेयो विष्णोर्भक्तिपरो यदि} %१३०

\twolineshloka
{न त्याज्या तुलसीमाला धात्रीमाला विशेषतः}
{तथा पद्माक्षमालाऽपि धर्मकामार्थमीप्सुभिः} %३१

\twolineshloka
{यावद्दिनानि वहते धात्रीमालां कलौ नरः}
{तावद्युगसहस्राणि वैकुण्ठे वसतिर्भवेत्} %३२

\twolineshloka
{सर्वदेवमयी धात्री वासुदेवमनःप्रिया}
{आरोपणीया सेव्या च पूजनीया सदा नरैः} %३३

\twolineshloka
{एतत्ते सर्वमाख्यातं धात्रीमाहात्म्यमुत्तमम्}
{श्रोतव्यं च सदा भक्तैश्चतुर्वर्गफलप्रदम्} %३४

\twolineshloka
{धात्रीछायां समाश्रित्य कार्तिकेऽन्नं भुनक्ति यः}
{अन्नसंसर्गजं पापमावर्षं तस्य नश्यति} %१३५


\iti{धात्रीमाहात्म्यवर्णनं}{नाम द्वादशोऽध्यायः}{१२}

\sect{अथ त्रयोदशोऽध्यायः}


\uvacha{सूत उवाच}
\twolineshloka
{श्रियः पतिमथामन्त्र्य गते देवर्षिसत्तमे}
{हर्षोत्फुल्लानना सत्या वासुदेवमथाऽब्रवीत्} %१


\uvacha{सत्यभामोवाच}
\twolineshloka
{धन्यास्मि कृतकृत्याऽस्मि सफलं जीवितं मम}
{दानं व्रतं तपो वाऽपि किं नु पूर्वं कृतं मया} %२

\threelineshloka
{येनाहं मर्त्यजा देव तवाङ्गार्द्धहराऽभवम्}
{भवान्तरे च किंशीला काचाऽहं कस्य कन्यका}
{तवाहं वल्लभा जाता तद्वदस्व ममाखिलम्} %३


\uvacha{श्रीकृष्ण उवाच}

\onelineshloka
{शृणुष्वैकमना कान्ते यथा त्वं पूर्वजन्मनि} %४

\twolineshloka
{पुण्यव्रतं कृतवती तत्सर्वं कथयामि ते}
{आसीत्कृतयुगस्यान्ते मायापुर्यां द्विजोत्तमः} %५

\twolineshloka
{आत्रेयो देवशर्मेति वेदवेदाङ्गपारगः}
{तस्यातिवयसश्चाऽऽसीन्नाम्ना गुणवती सुता} %६

\twolineshloka
{अपुत्रः स स्वशिष्याय चन्द्रनाम्ने ददौ सुताम्}
{तमेव पुत्रवन्मेने स च तं पितृवद्वशी} %७

\twolineshloka
{तौ कदाचिद्वनं यातौ कुशेध्माहरणार्थिनौ}
{निहतौ रक्षसा तौ च कृतान्तसमरूपिणा} %८

\twolineshloka
{स्वस्वपुण्य प्रभावेन विष्णुलोकं गतावुभौ}
{ततो गुणवती श्रुत्वा रक्षसा निहतावुभौ} %९

\twolineshloka
{पितृभर्तृजदुःखार्ता कारुण्यं पर्यदेवयत्}
{सा गृहोपस्करा न्सर्वान्विक्रीयाशु च कर्मं तत्} %१०

\twolineshloka
{तयोश्चक्रे यथाशक्ति पारलौकीं ततः क्रियाम्}
{तस्मिन्नेव पुरे चक्रे वासं सा मृतजीविनी} %११

\twolineshloka
{व्रतद्वयं तया सम्यगाजन्ममरणात्कृतम्}
{एकादशीव्रतं सम्यक्सेवनं कार्तिकस्य च} %१२

\twolineshloka
{इत्थं गुणवती सम्यक्प्रत्यब्द व्रतिनी ह्यभूत्}
{कदाचित्सरुजा साऽथ कृशाङ्गी ज्वरपीडिता} %१३

\twolineshloka
{स्नातुं गङ्गां गता कान्ते कथञ्चिच्छनकैस्तदा}
{यावज्जलान्तरगता कम्पिता शीतपीडिता} %१४

\twolineshloka
{तावत्सा विह्वलाऽपश्यद्विमानं यातमम्बरात्}
{अथ सा तद्विमानस्था वैकुण्ठभुवनं ययौ} %१५

\twolineshloka
{कार्तिकव्रतपुण्येन मत्सान्निध्यं गताऽभवत्}
{अथ ब्रह्मादिदेवानां यदा प्रार्थनया भुवम्} %१६

\twolineshloka
{आगतोऽहं गणाः सर्वे यातास्तेऽपि मया सह}
{एते हि यादवाः सर्वे मद्गणा एव भामिनि} %१७

\twolineshloka
{पिता ते देवशर्माऽभूत्सत्राजिदभिधो ह्ययम्}
{यश्चन्द्रनामाऽसोऽक्रूरस्त्वं सा गुणवती शुभा} %१८

\twolineshloka
{कार्तिकव्रतपुण्येन बहु मत्प्रीतिदायिनी}
{मद्द्वारि यत्त्वया पूर्वं तुलसीवाटिका कृता} %१९

\twolineshloka
{तस्मादयं कल्पवृक्षस्तवाङ्गणगतः शुभे}
{आजन्ममरणात्पूर्वं यत्कृतं कार्तिकव्रतम्} %२०


\onelineshloka*
{कदाचिदपि तेन त्वं मद्वियोगं न यास्यसि}

\uvacha{सत्योवाच}

\onelineshloka
{मासानां तु कथं नाम स मासः कार्तिको वरः} %२१


\onelineshloka*
{प्रियस्ते देवदेवेश कारणं तत्र कथ्यताम्}

\uvacha{श्रीकृष्ण उवाच}

\onelineshloka
{साधु पृष्टं त्वया कान्ते शृणुष्वैकाग्रमानसा} %२२

\twolineshloka
{पृथोर्वैन्यस्य संवादं महर्षेर्नारदस्य च}
{एवमेव पुरा पृष्टो नारदः पृथुनाब्रवीत्} %२३


\uvacha{नारद उवाच}
\twolineshloka
{शङ्खनामाऽभवत्पूर्वमसुरः सागरात्मजः}
{इन्द्रादिलोकपालानामधिकाराञ्जहार ह} %२४

\twolineshloka
{सुवर्णाद्रिगुहादुर्गसंस्थितास्त्रिदशादयः}
{तद्वीक्षयाम्बभूवुस्ते तदा दैत्यो व्यचारयत्} %२५

\twolineshloka
{हृताधिकारास्त्रिदशा मया यद्यपि निर्जिताः}
{लक्ष्यन्ते बलयुक्तास्ते करणीयं मयाऽत्र किम्} %२६

\twolineshloka
{ज्ञातं तत्तु मया देवा वेदमन्त्रबलान्विताः}
{तान्हरिष्ये ततः सर्वे बलहीना भवन्ति वै} %२७

\twolineshloka
{इति मत्वा ततो दैत्यो विष्णुमालक्ष्य निद्रितम्}
{सत्यलोकाज्जहाराशु वेदानादिस्वयम्भुवः} %२८

\twolineshloka
{नीतास्तु तेन ते वेदास्तद्भयात्ते निराक्रमन्}
{तोयानि विविशुर्यज्ञमन्त्रबीजसमन्विताः} %२९

\threelineshloka
{तान्मार्गमाणः शङ्खोऽपि समुद्रान्तर्गतो भ्रमन्}
{न ददर्श तदा दैत्यः क्वचिदेकत्र संस्थितान्}
{अथ देवैः स्तुतो विष्णुर्बोधितस्तानुवाच ह} %३०


\uvacha{विष्णुरुवाच}

\onelineshloka
{वरदोऽहं सुरगणा गीतवाद्यादिमङ्गलैः} %३१

\twolineshloka
{ऊर्जस्य शुक्लैकादश्यां भवद्भिः प्रतिबोधितः}
{अतश्चैषा तिथिर्मान्या साऽतीव प्रीतिदा मम} %३२

\twolineshloka
{वेदाः शङ्खहृताः सर्वे तिष्ठन्त्युदकसंस्थिताः}
{तानानयाम्यहं देवा हत्वा सागरनन्दनम्} %३३

\twolineshloka
{अद्यप्रभृति वेदास्तु मन्त्रबीजसमन्विताः}
{प्रत्यब्दं कार्तिके मासि विश्रमन्त्वप्सु सर्वदा} %३४

\twolineshloka
{कालेऽस्मिन्ये प्रकुर्वन्ति प्रातःस्नानं नरोत्तमाः}
{ते सर्वे यज्ञावभृथैः सुस्नाताः स्युर्न संशयः} %३५

\twolineshloka
{अद्यप्रभृत्यहमपि भवामि जलमध्यगः}
{भवन्तोऽपि मया सार्द्धमायान्तु समुनीश्वराः} %३६

\threelineshloka
{कातिकव्रतिनां चेन्द्र रक्षा कार्या त्वया सदा}
{इत्युक्त्वा भगवान्विष्णुः शफरीतुल्यरूपधृक्}
{खात्पपात जले विन्ध्यवासिनः कस्य पश्यतः} %३७

\twolineshloka
{हत्वा शङ्खासुरं विष्णुर्बदरीवनमागमत्}
{तत्राऽऽहूय ऋषीन्सर्वानिदमाज्ञापयत्प्रभुः} %३८


\uvacha{विष्णुरुवाच}
\threelineshloka
{जलान्तरविशीर्णांस्तान्यूयं वेदान्प्रमार्गथ}
{आनयध्वं च त्वरिताः सागरस्य जलान्तरात्}
{तावत्प्रयागं तिष्ठामि देवतागणसंयुतः} %३९


\uvacha{नारद उवाच}

\onelineshloka
{ततस्तैस्सर्वमुनिभिस्तपोबलसमन्वितैः} %४०

\twolineshloka
{उद्धृताश्च सबीजास्ते वेदा यज्ञसमन्विताः}
{तेषु यावन्मितं येन लब्धं तावद्धि तस्य तत्} %४१

\twolineshloka
{स स एव ऋषिर्जातस्तत्तत्प्रभृति पार्थिव}
{अथ सर्वेऽपि सङ्गम्य प्रयागं मुनयो ययुः} %४२

\twolineshloka
{विष्णवे सविधात्रे ते लब्धान्वेदान्न्यवेदयन्}
{लब्ध्वा वेदान्समग्रांस्तु ब्रह्मा हर्षसमन्वितः} %४३

\twolineshloka
{अयजद्वाजिमेधेन देवर्षिगणसंयुतः}
{यज्ञान्ते देवताः सर्वे विज्ञप्तिं चक्रुरञ्जसा} %४४


\uvacha{देवा ऊचुः}
\twolineshloka
{देवदेव जगन्नाथ विज्ञप्तिं शृणु नः प्रभो}
{हर्षकालोऽयमस्माकं तस्मात्त्वं वरदो भव} %४५

\twolineshloka
{स्थानेऽस्मिन्द्रुहिणो वेदान्नष्टान्प्राप पुनस्त्वयम्}
{यज्ञभागान्वयं प्राप्तास्त्वत्प्रसादाद्रमापते} %४६

\twolineshloka
{स्थानमेतद्धि नः श्रेष्ठं पृथिव्यां पुण्यवर्धनम्}
{भुक्तिमुक्तिप्रदं चाऽस्तु प्रसादाद्भवतः सदा} %४७

\twolineshloka
{कालोऽप्ययं महापुण्यो ब्रह्मघ्नाऽऽदिविशुद्धिकृत्}
{दत्ताऽक्षयकरं चाऽस्तु वरमेवं ददस्व नः} %४८


\uvacha{विष्णुरुवाच}
\twolineshloka
{ममाप्येतद्वृतं देवा यद्भवद्भिरुदाहृतम्}
{तथास्तु सुलभं त्वेतद्ब्रह्मक्षेत्रमितिप्रथम्} %४९

\twolineshloka
{सूर्यवंशोद्भवो राजा गङ्गामत्रानयिष्यति}
{सा सूर्यकन्यया चात्र कालिन्द्या योगमेष्यति} %५०

\twolineshloka
{यूयं च सर्वे ब्रह्माद्या निवसन्तु मया सह}
{तीर्थराजेति विख्यातं तीर्थमेतद्भविष्यति} %५१

\twolineshloka
{सर्वपापानि नश्यन्ति तीर्थराजस्य दर्शनात्}
{सूर्ये मकरगे प्राप्ते स्नायिनां पापनाशनः} %५२

\twolineshloka
{कालोऽप्येष महापुण्यफलदोऽस्तु सदा नृणाम्}
{सालोक्यादिफलं स्नानैर्माघे मकरगे रवौ} %५३

\uvacha{नारद उवाच}
\fourlineindentedshloka
{एवं देवान्देवदेवस्तदुक्त्वा}
{तत्रैवान्तर्धानमागात्सवेधाः}
{देवाः सर्वेऽप्यंशकैस्तेऽप्यतिष्ठं-}
{श्चान्तर्धानं प्रापुरिन्द्रादयस्ते} %५४

\twolineshloka
{कार्तिके तुलसीमूले योऽर्चयेद्धरिमीश्वरम्}
{भुक्त्वेह निखिलान्भोगानन्ते विष्णुपुरं व्रजेत्} %५५


\iti{सत्यभामापूर्वजन्मवृत्तान्तकथनपूर्वक प्रयागतीर्थप्रशंसाप्रसङ्गवर्णनं}{नाम त्रयोदशोऽध्यायः}{१३}

\sect{अथ चतुर्दशोऽध्यायः}


\uvacha{पृथुरुवाच}
\twolineshloka
{यत्त्वया कथितं ब्रह्मन्व्रतमूर्जस्य विस्तरात्}
{तत्र या तुलसीमूले विष्णोः पूजा त्वयोदिता} %१

\twolineshloka
{तेनाहं प्रष्टुमिच्छामि माहात्म्यं तुलसीभवम्}
{कथं साऽतिप्रिया तस्य देवदेवस्य शार्ङ्गिणः} %२

\twolineshloka
{कथमेषा समुत्पन्ना कस्मिन्स्थाने च नारद}
{एवं ब्रूहि समासेन सर्वज्ञोऽसि मतो मम} %३


\uvacha{नारद उवाच}
\twolineshloka
{शृणु राजन्नवहितो माहात्म्यं तुलसीभवम्}
{सेतिहासं पुरावृत्तं तत्सर्वं कथयामि ते} %४

\twolineshloka
{पुरा शक्रः शिवं द्रष्टुमगात्कैलासपर्वतम्}
{सर्वदेवैः परिवृतो ह्यप्सरोगणसेवितः} %५

\twolineshloka
{यावद्गतः शिवगृहं तावत्तत्र स दृष्टवान्}
{पुरुषं भीमकर्माणं दंष्ट्राननविभीषणम्} %६

\twolineshloka
{स पृष्टस्तेन कस्त्वं भोः क्व गतो जगदीश्वरः}
{एवं पुनः पुनः पृष्टः स तदा नोक्तवान्नृप} %७

\twolineshloka
{ततः कुद्धो वज्रपाणिस्तं निर्भर्त्स्य वचोऽब्रवीत्}
{रे मया पृच्छयमानोऽपि नोत्तरं दत्तवानसि} %८

\twolineshloka
{अतस्त्वां हन्मि वज्रेण कस्ते त्राताऽस्ति दुर्मते}
{इत्युदीर्य ततो वज्री वज्रेणाभ्यहनद्दृढम्} %९

\twolineshloka
{तेनास्य कण्ठो नीलत्वमगाद्वज्रं च भस्मताम्}
{ततो रुद्रः प्रजज्वाल तेजसा प्रदहन्निव} %१०

\twolineshloka
{दृष्ट्वा बृहस्पतिस्तूर्णं कृताञ्जलिपुटोऽभवत्}
{इन्द्रं च दण्डवद्भूमौ कृत्वा स्तोतुं प्रचक्रमे} %११


\uvacha{बृहस्पतिरुवाच}
\twolineshloka
{नमो देवाधिपतये त्र्यम्बकाय कपर्दिने}
{त्रिपुरघ्नाय शर्वाय नमोऽधङ्कनिषूदिने} %१२

\twolineshloka
{विरूपायातिरूपाय बहुरूपाय शम्भवे}
{यज्ञविध्वंसकर्त्रे च यज्ञानां फलदायिने} %१३

\twolineshloka
{कालान्तकाय कालाय कालभोगिधराय च}
{नमो ब्रह्मशिरोहन्त्रे ब्राह्मणाय नमो नमः} %१४


\uvacha{नारद उवाच}
\twolineshloka
{एवं स्तुतस्तदा शम्भुर्धिषणेन जगाद तम्}
{संहरन्नयनज्वालां त्रिलोकीदहन क्षमाम्} %१५

\twolineshloka
{वरं वरय भो ब्रह्मन्प्रीतः स्तुत्याऽनया तव}
{इन्द्रस्य जीवदानेन जीवेति त्वं प्रथां वज्र} %१६


\uvacha{बृहस्पतिरुवाच}
\twolineshloka
{यदि तुष्टोऽसि देव त्वं पाहीन्द्रं शरणागतम्}
{अग्निरेष शमं यातु भालनेत्रसमुद्भवः} %१७


\uvacha{ईश्वर उवाच}
\twolineshloka
{पुनः प्रवेशमायाति भालनेत्रे कथं शिखी}
{एनं त्यक्ष्याम्यहं दूरे यथेन्द्रं नैव पीडयेत्} %१८


\uvacha{नारद उवाच}
\twolineshloka
{इत्युक्त्वा तं करे धृत्वा प्राक्षिपल्लवणार्णवे}
{सोऽपतत्सिन्धुगङ्गायाः सागरस्य च सङ्गमे} %१९

\twolineshloka
{तावत्स बालरूपत्वमगात्तत्र रुरोद च}
{रुदतस्तस्य शब्देन प्राकम्पद्धरणी मुहुः} %२०

\twolineshloka
{स्वर्गाद्याः सत्यलोकान्तास्तत्स्वनाद्बधिरीकृताः}
{श्रुत्वा ब्रह्मा ययौ तत्र किमेतदिति विस्मितः} %२१

\twolineshloka
{तावत्समुद्रस्योत्सङ्गे तं बालं स ददर्श ह}
{दृष्ट्वा ब्रह्माणमायातं समुद्रोऽपि कृताञ्जलिः} %२२

\threelineshloka
{प्रणम्य शिरसा बालं तस्योत्सङ्गे न्यवेशयत्}
{भो ब्रह्मन्सिन्धुगङ्गायां जातोऽयं मम पुत्रकः}
{जातकर्मादिसंस्कारान्कुरुष्वाद्य जगद्गुरो} %२३


\uvacha{नारद उवाच}

\onelineshloka
{इत्थं वदति पाथोधौ स बालः सागरात्मजः} %२४

\threelineshloka
{ब्रह्माणमग्रहीत्कूर्चे विधुन्वंस्तं मुहुर्मुहुः}
{धुन्वतस्तस्य कूर्चे तु नेत्राभ्यामगमज्जलम्}
{कथञ्चिन्मुक्तकूर्चोऽथ ब्रह्मा प्रोवाच सागरम्} %२५


\uvacha{ब्रह्मोवाच}
\twolineshloka
{नेत्राभ्यां विधृतं यस्मादनेनैतज्जलं मम}
{तस्माज्जलन्धर इति ख्यातो नाम्ना भविष्यति} %२६

\twolineshloka
{अनेनैवैष तरुणः सर्वशस्त्रास्त्रपारगः}
{अवध्यः सर्वभूतानां विना रुद्रं भविष्यति} %२७


\onelineshloka
{यत एष समुद्भूतस्तत्रैवान्तं गमिष्यति} %२८


\uvacha{नारद उवाच}
\twolineshloka
{इत्युक्त्वा शुक्रमाहूय राज्ये तं चाभ्यषेचयत्}
{आमन्त्र्य सरितां नाथं ब्रह्मान्तर्धानमागमत्} %२९

\twolineshloka
{अथ तद्दर्शनोत्फुल्लनयनः सागरस्तदा}
{कालनेमिसुतां वृन्दां तद्भार्यार्थमयाचत} %३०

\fourlineindentedshloka
{ते कालनेमिप्रमुखास्ततोऽसुरा-}
{स्तस्मै सुतां तां प्रददुः प्रहर्षिताः}
{स चापि तां प्राप्य सुहृद्वरां वशां}
{शशास गां शुक्रसहायवान्बली} %३१


\iti{जलन्धरोत्पत्तिवर्णनं}{नाम चतुर्दशोऽध्यायः}{१४}

\sect{अथ पञ्चदशोऽध्यायः}


\uvacha{नारद उवाच}
\twolineshloka
{ये देवैर्निर्जिताः पूर्वं दैत्याः पातालसंस्थिताः}
{तेऽपि भूमण्डलं याता निर्भयास्तमुपाश्रिताः} %१

\twolineshloka
{कदाचिच्छिन्नशिरसं राहुं दृष्ट्वा स दैत्यराट्}
{पप्रच्छ भार्गवं तत्र तच्छिरश्छेदकारणम्} %२

\twolineshloka
{स शशंस समुद्रस्य मथनं देवकारितम्}
{रत्नापहरणं चैव दैत्यानां च पराभवम्} %३

\twolineshloka
{स श्रुत्वा क्रोधरक्ताक्षः स्वपितुर्मथनं तदा}
{दूतं सम्प्रेषयामास घस्मरं शक्रसन्निधौ} %४

\twolineshloka
{दूतस्त्रिविष्टपं गत्वा सुधर्मां प्राविशद्वराम्}
{जगादाखर्वमौलिस्तु देवेन्द्रं वाक्यमद्भुतम्} %५


\uvacha{घस्मर उवाच}
\twolineshloka
{जलन्धरोऽब्धितनयः सर्वदैत्यजनेश्वरः}
{दूतोऽहम्प्रेषितस्तेन स यदाह शृणुष्व तत्} %६

\twolineshloka
{कस्मात्त्वया मम पिता मथितः सागरोऽद्रिणा}
{नीतानि सर्वरत्नानि तानि शीघ्रं प्रयच्छ मे} %७

\twolineshloka
{इति दूतवचः श्रुत्वा विस्मितस्त्रिदशाधिपः}
{उवाच घस्मरं रौद्रं भयरोषसमन्वितः} %८


\uvacha{इन्द्र उवाच}
\twolineshloka
{शृणु दूत मया पूर्वं मथितः सागरो यथा}
{अद्रयो मद्भयात्त्रस्ताः स्वकुक्षिस्थाः कृतास्तथा} %९

\twolineshloka
{अन्येऽपि मद्द्विषस्तेन रक्षिता दितिजाः पुरा}
{तस्माद्यत्तत्प्रजातं तु मयाऽप्यपहृतं किल} %१०

\twolineshloka
{शङ्खोऽप्येवं पुरा देवानद्विषत्सागरात्मजः}
{ममानुजेन निहतः प्रविष्टः सागरोदरम्} %११


\onelineshloka*
{तद्गच्छ कथयस्वास्य सर्वं मथनकारणम्}

\uvacha{नारद उवाच}

\onelineshloka
{इत्थं विसर्जितो दूतस्तदेन्द्रेणागमद्भुवम्} %१२

\twolineshloka
{तदिदं वचनं सर्वं दैत्यायाकथयत्तदा}
{तन्निशम्य तदा दैत्यो रोषात्प्रस्फुरिताधरः} %१३

\twolineshloka
{दैत्यसेनासमायुक्तो ययौ योद्धुं त्रिविष्टपम्}
{ततो युद्धे महाञ्जातो देवदानवसङ्क्षयः} %१४

\twolineshloka
{तत्र युद्धे मृतान्दैत्यान्भार्गवस्तूदतिष्ठपत्}
{विद्यया मृतजीविन्या मन्त्रितैस्तोयबिन्दुभिः} %१५

\twolineshloka
{देवानपि तथा युद्धे तत्राजीवयदङ्गिराः}
{दिव्यौषधी समानीय द्रोणाद्रेः स पुनःपुनः} %१६

\twolineshloka
{दृष्ट्वा देवांस्तथा युद्धे पुनरेव समुत्थितान्}
{जलन्धरः क्रोधवशो भार्गवं वाक्यमब्रवीत्} %१७


\uvacha{जलन्धर उवाच}
\twolineshloka
{मया युद्धे हता देवा उत्तिष्ठन्ति कथं पुनः}
{तव सञ्जीविनीविद्या न वाऽन्यत्रेति विश्रुतम्} %१८


\uvacha{शुक्र उवाच}
\twolineshloka
{दिव्यौषधीः समानीय द्रोणाद्रेरङ्गिराः सुरान्}
{जीवयत्येव तच्छीघ्रं द्रोणाद्रिं त्वमपाहर} %१९


\uvacha{नारद उवाच}
\twolineshloka
{इत्युक्तः स तु दैत्येन्द्रो नीत्वा द्रोणाचलं तदा}
{प्राक्षिपत्सागरे तूर्णं पुनरागान्महाहवम्} %२०

\twolineshloka
{अथ देवान्हतान्दृष्ट्वा द्रोणाद्रिमगमद्गुरुः}
{तावत्तत्र गिरीन्द्रं तु न ददर्श सुरार्चितः} %२१

\twolineshloka
{ज्ञात्वा दैत्यहृतं द्रोणं धिषणो भयविह्वलः}
{आगत्य दूराद्व्याजह्रे श्वासाऽऽकुलितविग्रहः} %२२

\twolineshloka
{पलायध्वं हवाद्देवा नायं जेतुं क्षमो यतः}
{रुद्रांशसम्भवो ह्येष स्मरध्वं शक्रचेष्टितम्} %२३

\twolineshloka
{श्रुत्वा तद्वचनं देवा भयविह्वलितास्तदा}
{दैत्येन वध्यमानास्ते पलायन्ते दिशो दश} %२४

\twolineshloka
{देवान्विद्रावितान्दृष्ट्वा दैत्यैः सागरनन्दनः}
{शङ्खभेरीजयरवैः प्रविवेशामरावतीम्} %२५

\twolineshloka
{प्रविष्टे नगरीं दैत्ये देवाः शक्रपुरोगमाः}
{सुवर्णाद्रिगुहां प्राप्ता न्यवसन्दैत्यतापिताः} %२६

\twolineshloka
{ततश्च सर्वेष्वसुरोऽधिकारेष्विन्द्रादिकानां विनिवेशयत्तदा}
{शुम्भादिकान्दैत्यवरान्पृथक्पृथक्स्वयं सुवर्णाद्रिगुहामगात्पुनः} %२७


\iti{जलन्धरविजयप्राप्ति}{नाम पञ्चदशोऽध्यायः}{१५}

\sect{अथ षोडशोऽध्यायः}


\uvacha{नारद उवाच}
\twolineshloka
{पुनर्दैत्यं समायान्तं दृष्ट्वा देवाः सवासवाः}
{भयप्रकम्पिताः सर्वे विष्णुं स्तोतुं प्रचक्रमुः} %१

\fourlineindentedshloka
{नमो मत्स्यकूर्मादिनानास्वरूपैः}
{सदाभक्तकार्योद्यतायार्तिहन्त्रे}
{विधात्रादिसर्गस्थितिध्वंसकर्त्रे}
{गदाशङ्खपद्मारिहस्ताय तेस्तु} %२

\fourlineindentedshloka
{रमा वल्लभायासुराणां निहन्त्रे}
{भुजङ्गारियानाय पीताम्बराय}
{मखादिक्रियापाककर्त्रे विकर्त्रे}
{शरण्याय तस्मै नताः स्मो नताः स्मः} %३

\fourlineindentedshloka
{नमो दैत्यसन्तापितामर्त्यदुःखा-}
{चलध्वंसदम्भोलये विष्णवे ते}
{भुजङ्गेशतल्पेशयायार्कचन्द्र-}
{द्विनेत्राय तस्मै नताः स्मो नताः स्मः} %४


\uvacha{नारद उवाच}
\twolineshloka
{सङ्कष्टनाशनं नाम स्तोत्रमेतत्पठेन्नरः}
{स कदाचिन्न सङ्कष्टैः पीड्यते कृपया हरेः} %५

\twolineshloka
{इति देवाः स्तुतिं यावत्कुर्वन्ति दनुजद्विषः}
{तावत्सुराणामापत्तिर्विज्ञाता विष्णुना तदा} %६

\twolineshloka
{सहसोत्थाय दैत्यारिः सक्रोधः खिन्नमानसः}
{आरूढो गरुडं वेगाल्लक्ष्मीं वचनमब्रवीत्} %७


\uvacha{श्रीभगवानुवाच}
\twolineshloka
{जलन्धरेण ते भ्रात्रा देवानां कदनं कृतम्}
{तैराहूतो गमिष्यामि युद्धायाद्य त्वरान्वितः} %८


\uvacha{श्रीरुवाच}
\twolineshloka
{अहं ते वल्लभा नाथ भक्त्या च यदि सर्वदा}
{तत्कथं ते मम भ्राता युद्धे वध्यः कृपानिधे} %९


\uvacha{श्रीभगवानुवाच}
\twolineshloka
{रुद्रांशसम्भवत्वाच्च ब्रह्मणो वचनादपि}
{प्रीत्या च तव नैवायं मम वध्यो जलन्धरः} %१०


\uvacha{नारद उवाच}
\twolineshloka
{इत्युक्त्वा गरुडारूढः शङ्खचक्रगदासिभृत्}
{विष्णुर्वेगाद्ययौ योद्धुं यत्र देवाः स्तुवन्ति ते} %११

\twolineshloka
{अथाऽरुणानुजात्युग्रपक्षवातप्रपीडिताः}
{वात्या विमर्दिता दैत्या बभ्रमुः खे यथा घनाः} %१२

\twolineshloka
{ततो जलन्धरो दृष्ट्वा दैत्यान्वात्याप्रपीडितान्}
{उद्वृत्तनयनः क्रोधात्ततोविष्णुं समभ्ययात्} %१३

\twolineshloka
{ततः समभवद्युद्धं विष्णुदैत्येन्द्रयोर्महत्}
{आकाशं कुर्वतोर्बाणैस्तदा निरवकाशवत्} %१४

\twolineshloka
{विष्णुर्दैत्यस्य बाणौघैर्ध्वजं छत्रं धनुर्हयान्}
{चिच्छेद तं च हृदये बाणेनैकेन ताडयत्} %१५

\twolineshloka
{ततो दैत्यः समुत्पत्य गदापाणिस्त्वरान्वितः}
{आहत्य गरुडं मूर्ध्नि पातयामास भूतले} %१६

\twolineshloka
{विष्णुर्गदां स्वखङ्गेन चिच्छेद प्रहसन्निव}
{तावत्स हृदये विष्णुं जघान दृढमुष्टिना} %१७

\twolineshloka
{ततस्तौ बाहुयुद्धेन युयुधाते महाबलौ}
{बाहुभिर्मुष्टिभिश्चैव जानुभिर्नादयन्महीम्} %१८

\twolineshloka
{एवं तौ सुचिरं युद्धं कृत्वा विष्णुः प्रतापवान्}
{उवाच दैत्यराजानं मेघगम्भीरनिस्वनः} %१९


\uvacha{विष्णुरुवाच}
\twolineshloka
{वरं वरय दैत्येन्द्र प्रीतोऽस्मि तव विक्रमात्}
{अदेयमपि ते दद्मि यत्ते मनसि वर्तते} %२०

\uvacha{जलन्धर उवाच}
\twolineshloka
{यदि भावुक तुष्टोऽसि वरमेनं ददस्व मे}
{मद्भगिन्या सहाऽद्य त्वं मद्गृहे सगणो वस} %२१


\uvacha{नारद उवाच}
\twolineshloka
{तथेत्युक्ता स भगवान्सर्वदेवगणैः सह}
{तदा जलन्धरपुरमगमद्रमया सह} %२२

\twolineshloka
{जलन्धरस्तु देवानामधिकारेषु दानवान्}
{स्थापयित्वा महाबाहुः पुनरागान्महीतलम्} %२३

\twolineshloka
{देवगन्धर्वसिद्धेषु यत्किञ्चिद्रत्नसंयुतम्}
{तदात्मवशगं कृत्वाऽतिष्ठत्सागरनन्दनः} %२४

\twolineshloka
{पातालभुवने दैत्यं निशुम्भं स महाबलम्}
{स्थापयित्वा स शेषादीनानयद्भूतलं बली} %२५

\twolineshloka
{देवगन्धर्वसिद्धाऽऽद्यान्सर्पराक्षसमानुषान्}
{स्वपुरे नागरान्कृत्वा शशास भुवनत्रयम्} %२६

\twolineshloka
{एवं जलन्धरः कृत्वा देवान्स्ववशवर्तिनः}
{धर्मेण पालयामास प्रजाः पुत्रानिवौरसान्} %२७

\twolineshloka
{न कश्चिद्व्याधितो नैव दुःखी नैव कृतस्तथा}
{न दीनो दृश्यते तस्मिन्धर्माद्राज्यं प्रशासति} %२८

\twolineshloka
{एवं महीं शासति दानवेन्द्रे धर्मेण सम्यक्च दिदृक्षयाऽहम्}
{कदाचिदागामथ तस्य लक्ष्मीं विलोकितुं श्रीरमणं च सेवितुम्} %२९


\iti{जलन्धरसभायां नारदागमनं}{नाम षोडशोऽध्यायः}{१६}

\sect{अथ सप्तदशोऽध्यायः}


\uvacha{नारद उवाच}
\twolineshloka
{स मां प्रोवाच विधिवत्सम्पूज्यातीव भक्तिमान्}
{सम्प्रहस्य तदा वाक्यं स्नेहपूर्वं च वै नृप} %१

\twolineshloka
{कुत आगम्यते ब्रह्मन्किचिद्दृष्टं त्वया प्रभो}
{यदर्थमिह चाऽऽयातस्तदाऽऽज्ञापय मां मुने} %२


\uvacha{नारद उवाच}
\twolineshloka
{गतः कैलासशिखरं दैत्येन्द्राहं यदृच्छया}
{तत्रोमया समासीनं दृष्टवानस्मि शङ्करम्} %३

\twolineshloka
{योजनायुतविस्तीर्णे कल्पवृक्षमहावने}
{कामधेनुशताकीर्णे चिन्तामणिसुदीपिते} %४

\twolineshloka
{तद्दृष्ट्वा महदाश्चर्यं विस्मयो मेऽभवत्तदा}
{क्वाऽपीदृशी भवेदृद्धिस्त्रैलोक्ये वा न वेति च} %५

\twolineshloka
{तदा तवाऽपि दैत्येन्द्र समृद्धिः संस्मृता मया}
{तद्विलोकनकामोऽस्मि त्वत्सान्निध्यमिहाऽऽगतः} %६

\twolineshloka
{त्वत्समृद्धिमिमां पश्यन्स्त्रीरत्नरहितां धुवम्}
{तर्कयामि शिवादन्यस्त्रिलोक्यां न समृद्धिमान्} %७

\twolineshloka
{अप्सरोनागकन्याद्या यद्यपि त्वद्वशे स्थिताः}
{तथाऽपि ता न पार्वत्या रूपेण सदृशा ध्रुवम्} %८

\twolineshloka
{यस्या लावण्यजलधौ निमग्नश्चतुराननः}
{स्वधैर्यममुचत्पूर्वं तया काऽन्योपमीयते} %९

\twolineshloka
{वीतरागोऽपि हि यथा मदनारिः स्वलीलया}
{सौन्दर्यगहनेऽभ्रामि शफरीरूपया पुरा} %१०

\twolineshloka
{यस्याः पुनः पुनः पश्यन्रूपं धाताऽपि सर्जने}
{ससर्जाऽप्सरसस्तासां तत्समैकाऽपि नाभवत्} %११

\twolineshloka
{अतः स्त्रीरत्नसम्भोक्तुः समृद्धिस्तस्य सा वरा}
{तथा न तव दैत्येन्द्र सर्वरत्नाऽधिपस्य च} %१२

\twolineshloka
{एवमुक्त्वा तमामन्त्र्य गते सति स दैत्यराट्}
{तद्रूप श्रवणादासीदनङ्गज्वरपीडितः} %१३

\twolineshloka
{अथ सम्प्रेषयामास स दूतं सिंहिकासुतम्}
{त्र्यम्बकायाऽपि च तदा विष्णुमायाविमोहितः} %१४

\twolineshloka
{कैलासमगमद्राहुः कुर्वञ्छुक्लेन्दुवर्चसम्}
{कार्ष्ण्येन कृष्णपक्षेन्दुवर्चसं स्वाङ्गजेन तम्} %१५

\twolineshloka
{निवेदितस्तदेशाय नन्दिना प्रविवेश सः}
{त्र्यम्बकभ्रूलतासंज्ञा प्रेरितो वाक्यमब्रवीत्} %१६


\uvacha{राहुरुवाच}
\twolineshloka
{देवपन्नगसेव्यस्य त्रैलोक्याधिपतेः प्रभोः}
{सर्वरत्नेश्वरस्य त्वमाज्ञां शृणु वृषध्वज} %१७

\twolineshloka
{स्मशानवासिनो नित्यमस्थिभारवहस्य च}
{दिगम्बरस्य ते भार्या कथं हैमवती शुभा} %१८

\twolineshloka
{अहं रत्नाधिनाथोऽस्मि सा च स्त्रीरत्नसंज्ञिका}
{तस्मान्ममैव सा योग्या नैव भिक्षाशिनस्तव} %१९


\uvacha{नारद उवाच}
\twolineshloka
{वदत्येवं तदा राहौ भ्रूमध्याच्छूलपाणिनः}
{अभवत्पुरुषो रौद्रस्तीव्राशनिसमस्वनः} %२०

\twolineshloka
{सिंहास्यः प्रललज्जिह्वः स ज्वलन्नयनो महान्}
{ऊर्ध्वकेशः शुष्कतनुर्नृसिंह इव चाऽपरः} %२१

\twolineshloka
{स तं खादितुमायान्तं दृष्ट्वा राहुर्भयातुरः}
{अधावत स वेगेन बहिः स च दधार तम्} %२२

\twolineshloka
{स च राहुर्महाबाहो मेघगम्भीरया गिरा}
{उवाच देवदेव त्वं पाहि मां शरणागतम्} %२३

\twolineshloka
{ब्राह्मणं मां महादेव खादितुं समुपागतः}
{महादेवो वचः श्रुत्वा ब्राह्मणस्य तदाऽब्रवीत्} %२४

\twolineshloka
{नैवाऽसौ वध्यतामेति दूतोऽयं परवान्यतः}
{मुञ्चेति पुरुषः श्रुत्वा राहुं तत्याज्य सोऽम्बरे} %२५


\onelineshloka*
{राहुं त्यक्त्वाऽथ पुरुषस्तदा रुद्रं व्यजिज्ञपत्}

\uvacha{पुरुष उवाच}

\onelineshloka
{क्षुधा मां वाधतेऽत्यन्तं क्षुत्क्षामश्चाऽस्मि सर्वथा} %२६


\onelineshloka*
{किं भक्षयामि देवेश तदाज्ञापय मां प्रभो}

\uvacha{ईश्वर उवाच}

\onelineshloka
{भक्षयस्वात्मनः शीघ्रं मांसं त्वं हस्तपादयोः} %२७


\uvacha{नारद उवाच}
\twolineshloka
{स शिवेनैवमाज्ञप्तश्चखाद पुरुषः स्वकम्}
{हस्तपादोद्भवं मांसं शिरःशेषो यथाऽभवत्} %२८

\twolineshloka
{दृष्ट्वा शिरोऽवशेषं तं सुप्रसन्नस्तदा शिवः}
{उवाच भीमकर्माणं पुरुषं जातविस्मयः} %२९


\uvacha{ईश्वर उवाच}
\twolineshloka
{त्वं कीर्तिमुखसंज्ञो हि भव मद्द्वारिगः सदा}
{त्वदर्चां ये न कुर्वन्ति नैव ते मे प्रियङ्कराः} %३०


\uvacha{नारद उवाच}
\twolineshloka
{तदा प्रभृति देवस्य द्वारि कीर्तिमुखः स्थितः}
{नार्चयन्तीह ये पूर्वं तेषामर्चा वृथा भवेत्} %३१

\twolineshloka
{राहुर्विमुक्तो यस्तेन सोऽपि तद्बर्बरे स्थले}
{अतः स बर्बरोद्भूत इति भूमौ प्रथां गतः} %३२

\twolineshloka
{ततः स राहुः पुनरेव जातमात्मानमस्मिन्निति मन्यमानः}
{समेत्य सर्वं कथयाम्बभूव जलन्धरायैव विचेष्टितं तत्} %३३


\iti{जलन्धरोपाख्याने दूतवाक्यकथनं}{नाम सप्तदशोऽध्यायः}{१७}

\sect{अथ रुद्रसेनापराभवोनामाऽष्टादशोऽध्यायः}


\uvacha{नारद उवाच}
\twolineshloka
{जलन्धरस्तु तच्छुत्वा कोपाकुलितविग्रहः}
{निर्जगामाऽऽशु दैत्यानां कोटिभिः परिवारितः} %१

\twolineshloka
{गच्छतोऽस्याग्रतः शुक्रो राहुर्दृष्टिपथेऽभवत्}
{मुकुटश्चापतद्भूमौ वेगात्प्रस्खलितस्तदा} %२

\twolineshloka
{दैत्यसैन्यावृतैस्तस्य विमानानां शतैस्तदा}
{व्यराजत नभःपूर्णं प्रावृषीव यथा घनैः} %३

\twolineshloka
{तस्योद्योगं तदा दृष्ट्वा देवाः शक्रपुरोगमाः}
{अलक्षितास्तदा जग्मुः शूलिनं तं व्यजिज्ञपुः} %४


\uvacha{देवा ऊचुः}
\twolineshloka
{न जानासि कथं स्वामिन्देवापत्तिमिमां विभो}
{तदस्मद्रक्षणार्थाय जहि सागरनन्दनम्} %५


\uvacha{नारद उवाच}
\twolineshloka
{इति देववचः श्रुत्वा प्रहस्य वृषभध्वजः}
{महाविष्णुं समाहूय वचनं चेदमब्रवीत्} %६


\uvacha{ईश्वर उवाच}
\twolineshloka
{जलन्धरः कथं विष्णो न हतः सङ्गरे त्वया}
{तद्गृहं चापि यातोऽसि त्यक्त्वा वैकुण्ठमात्मनः} %७


\uvacha{विष्णुरुवाच}
\twolineshloka
{तवांशसम्भवत्वाच्च भ्रातृत्वाच्च तथा श्रियः}
{न मया निहतः सङ्ख्ये त्वमेनं जहि दानवम्} %८


\uvacha{ईश्वर उवाच}
\twolineshloka
{नायमेभिर्महातेजाः शस्त्रास्त्रैर्वध्यते मया}
{देवैः सह स्वतेजोंशं शस्त्रार्थं दीयतां मम} %९


\uvacha{नारद उवाच}
\twolineshloka
{अथ विष्णुमुखा देवाः स्वतेजांसि ददुस्तदा}
{तान्यैक्यमागतानीशो दृष्ट्वा स्वं चामुचन्महः} %१०

\twolineshloka
{तेनाकरोन्महादेवो महसा शस्त्रमुत्तमम्}
{चक्रं सुदर्शनं नाम ज्वालामालातिभीषणम्} %११

\twolineshloka
{ततः शेषेण च तदा वज्रं च कृतवान्हरिः}
{तावज्जलन्धरो दृष्टः कैलासतलभूमिषु} %१२

\twolineshloka
{हस्त्यश्वरथपत्तीनां कोटिभिः परिवारितः}
{तं दृष्ट्वा लक्षिता जग्मुर्देवाः सर्वे यथागताः} %१३

\twolineshloka
{गणाश्च समसज्जन्त युद्धायाऽतित्वरान्विताः}
{नन्दीभवक्त्रसेनानीमुखाः सर्वे शिवाज्ञया} %१४

\twolineshloka
{अवतेरुर्गणा वेगात्कैलासाद्युद्धदुर्मदाः}
{ततः समभवद्युद्धं कैलासोपत्यकाभुवि} %१५

\twolineshloka
{प्रमथाधिपदैत्यानां घोरशस्त्रास्त्रसङ्कुलम्}
{भेरीमृदङ्गशङ्खौघ निःस्वनैर्वीरहर्षणैः} %१६

\twolineshloka
{गजाश्वरथशब्दैश्च नादिता भूर्व्यकम्पत}
{शक्तितोमरबाणौघमुसलप्रासपट्टिशैः} %१७

\twolineshloka
{व्यराजत नभः पूर्णमुल्काभिरिव संवृतम्}
{निहतैरथनागाश्वपत्तिभिर्भूर्व्यराजत} %१८

\twolineshloka
{वज्राहताचलशिरःशकलैरिव संवृता}
{प्रमथाहतदैत्यौघैर्दैत्याहतगणैस्तथा} %१९

\twolineshloka
{वसासृङ्मांसपङ्काढ्या भूरगम्याऽभवत्तदा}
{प्रमथाहतदैत्यौघान्भार्गवः समजीवयत्} %२०

\threelineshloka
{युद्धे पुनः पुनस्तत्र मृतसञ्जीविनीबलात्}
{तं दृष्ट्वा व्याकुलीभूता गणाः सर्वे भयान्विताः}
{शशंसुर्देवदेवाय तत्सर्वं शुक्रचेष्टितम्} %२१

\twolineshloka
{अथ रुद्रमुखात्कृत्या बभूवातीवभीषणा}
{तालजङ्घा दरीवक्त्रा स्तनापीडितभूरुहा} %२२

\twolineshloka
{सा युद्धभूमिमासाद्य भक्षयन्ती महासुरान्}
{भार्गवं स्वभगे धृत्वा जगामान्तर्हिता नभः} %२३

\twolineshloka
{विधृतं भार्गवं दृष्ट्वा दैत्यसैन्यं गणास्तदा}
{अम्लानवदना हर्षान्निजघ्नुर्युद्धदुर्मदाः} %२४

\twolineshloka
{अथाभज्यत दैत्यानां सेना गणभयार्दिता}
{वायुवेगेनाहतेव प्रकीर्णा तृणसन्ततिः} %२५

\twolineshloka
{भग्नां गणभयात्सेनां दृष्ट्वामर्षयुता ययुः}
{निशुम्भशुम्भौ सेनान्यौ कालनेमिश्च वीर्यवान्} %२६


\twolineshloka
{त्रयस्ते वारयामासुर्गणसेनां महाबलाः}
{मुञ्चतः शरवर्षाणि प्रावृषीव बलाहकाः} %२७

\twolineshloka
{ततो दैत्यशरौघास्ते शलभानामिव व्रजाः}
{रुरुधुः खं दिशः सर्वा गणसेनामकम्पयन्} %२८

\twolineshloka
{गणाः शरशतैर्भिन्ना रुधिरासारवर्षिणः}
{वसन्ते किंशुकाभासा न प्राज्ञायत किञ्चन} %२९

\twolineshloka
{पतिताः पात्यमानाश्च भिन्नाश्छिन्नास्तदा गणाः}
{त्यक्त्वा सङ्ग्रामभूमिं ते सर्वेऽपि विमुखाऽभवन्} %३०

\fourlineindentedshloka
{ततः प्रभग्नं स्वबलं विलोक्य}
{शैलादिलम्बोदरकार्तिकेयाः}
{त्वरान्विता दैत्यवरान्प्रसह्य}
{निवारयामासुरमर्षिणस्ते} %३१


\iti{जलन्धरोपाख्याने रुद्रसेनापराभ}{नामाष्टादशोऽध्यायः}{१८}

\sect{अथ वीरभद्रपतननामैकोनविंशोऽध्यायः}


\uvacha{नारद उवाच}
\twolineshloka
{ते गणाधिपतीन्दृष्ट्वा नन्दीभमुखषण्मुखान्}
{अमर्षादभ्यधावन्त द्वन्द्वयुद्धाय दानवाः} %१

\twolineshloka
{नन्दिनं कालनेमिश्च शुम्भो लम्बोदरं तथा}
{निशुम्भः षण्मुखं वेगादभ्यधावत दंशितः} %२

\twolineshloka
{निशुम्भः कार्तिकेयस्य मयूरं पञ्चभिः शरैः}
{हृदि विव्याध वेगेन मूर्च्छितः स पपात च} %३

\twolineshloka
{ततः शक्तिधरः शक्तिं यावज्जग्राह रोषितः}
{तावन्निशुम्भो वेगेन स्वशक्त्या तमपातयत्} %४

\twolineshloka
{नन्दीश्वरः शरव्रातैः कालनेमिमवध्यत}
{सप्तभिश्च हयान्केतुं त्रिभिः सारथिमच्छिनत्} %५

\twolineshloka
{कालनेमिस्तु सङ्क्रुद्धो धनुश्चिच्छेद नन्दिनः}
{तदपास्य स शूलेन तं वक्षस्यहनद्बली} %६

\twolineshloka
{स शूलभिन्नहृदयो हताश्वो हतसारथिः}
{अद्रेः शिखरमामुच्य शैलादिं सोऽप्यपातयत्} %७

\twolineshloka
{अथ शुम्भो गणेशश्च रथमूषकवाहनौ}
{युध्यमानौ शरव्रातैः परस्परमविध्यताम्} %८

\twolineshloka
{गणेशस्तु तदा शुम्भं हृदि विव्याध पत्रिणा}
{सारथिं च त्रिभिर्बाणैः पातयामास भूतले} %९

\twolineshloka
{ततोऽतिक्रुद्धः शुम्भोऽपि बाणषष्ट्या गणाधिपम्}
{मूषकं च त्रिभिर्विद्ध्वा ननाद जलदस्वनः} %१०

\twolineshloka
{मूषकः शरभिन्नाङ्गश्चचाल दृढवेदनः}
{लम्बोदरश्च पतितः पदातिरभवन्नृप} %११

\twolineshloka
{ततो लम्बोदरः शुम्भं हत्वा परशुना हृदि}
{अपातयत्तदा भूमौ मूषकं चारुहत्पुनः} %१२

\twolineshloka
{कालनेमिर्निशुम्भश्चाप्युभौ लम्बोदरं शरैः}
{युगपज्जघ्नतुः क्रोधात्तोत्रैरिव महाद्विपम्} %१३

\twolineshloka
{तं पीडयमानमालोक्य वीरभद्रो महाबलः}
{अभ्यधावत वेगेन भूतकोटियुतस्तदा} %१४

\twolineshloka
{कूष्माण्डभैरवाश्चापि वेताला योगिनीगणाः}
{पिशाचयोगिनीसङ्घा गणाश्चापि तमन्वयुः} %१५

\twolineshloka
{ततः किलकिलाशब्दैः सिंहनादैः सुघर्घरैः}
{भेरीतालमृदङ्गैश्च पृथिवी समकम्पत} %१६

\twolineshloka
{ततो भूतान्यधावन्त भक्षयन्तिस्म दानवान्}
{उत्पतन्त्यापतन्ति स्म ननृतुश्च रणाङ्गणे} %१७

\twolineshloka
{नन्दी च कार्तिकेयश्च समाश्वस्य त्वरत्वितौ}
{निजघ्नतू रणे दैत्यान्निरन्तरशरव्रजैः} %१८

\twolineshloka
{छिन्नभिन्ना हतैर्देत्यैः पतितैर्भक्षितैस्तदा}
{व्याकुला साऽभवत्सेना विषण्णवदना तदा} %१९

\twolineshloka
{प्रविध्वस्तां तदा सेनां दृष्ट्वा सागरनन्दनः}
{रथेनातिपताकेन गणानभिययौ बली} %२०

\twolineshloka
{हस्त्यश्वरथसंह्रादाः शङ्खभेरीस्वनास्तथा}
{अभवन्सिंहनादाश्च सेनयोरुभयोस्तदा} %२१

\twolineshloka
{जलन्धरशरव्रातैर्नीहारपटलैरिव}
{द्यावापृथिव्योराच्छिन्नमन्तरं समपद्यत} %२२

\twolineshloka
{गणेशं पञ्चभिर्विद्ध्वा शैलादिं नवभिः शरैः}
{वीरभद्रं च विंशत्या ननाद जलदस्वनः} %२३

\twolineshloka
{कार्तिकेयस्तदा दैत्यं शक्त्या विव्याध सत्वरः}
{युयुधे शक्तिनिर्भिन्नः किञ्चिद्व्याकुलमानसः} %२४

\twolineshloka
{ततः क्रोधपरीताक्षः कार्तिकेयं जलन्धरः}
{गदया ताडयामास स च भूमितलेऽपतत्} %२५

\twolineshloka
{तथैव नन्दिनं वेगादपातयत भूतले}
{ततो गणेश्वरः क्रुद्धो गदां परशुनाऽहनत्} %२६

\twolineshloka
{वीरभद्रस्त्रिभिर्बाणैर्हृदि विव्याध दानवम्}
{सप्तभिश्च हयान्केतुं धनुश्छत्रं च चिच्छिदे} %२७

\twolineshloka
{ततोऽतिक्रुद्धो दैत्येन्द्रः शक्तिमुद्यम्य दारुणाम्}
{गणेशं पातयामास रथं चान्यमथाऽऽरुहत्} %२८

\twolineshloka
{अभ्ययादथ वेगेन वीरभद्रं रुषान्वितः}
{ततस्तौ सूर्यसङ्काशौ युयुधाते परस्परम्} %२९

\twolineshloka
{वीरभद्रः पुनस्तस्य हयान्बाणैरपातयत्}
{धनुश्चिच्छेद दैत्येन्द्रः पुप्लुवे परिघायुधः} %३०

\twolineshloka
{स वीरभद्रं त्वरयाऽभिगम्य जघान दैत्यः परिघेण मूर्ध्नि}
{स चापि वीरः प्रविभिन्नमूर्द्धा पपात भूमौ रुधिरं समुद्गिरन्} %३१


\iti{जलन्धरोपाख्याने वीरभद्रपत}{नामैकोनविंशोऽध्यायः}{१९}

\sect{अथ विंशोऽध्यायः}


\uvacha{नारद उवाच}
\twolineshloka
{पतितं वीरभद्रं तु दृष्ट्वा रुद्रगणा भयात्}
{अगमंस्ते रणं हित्वा क्रोशमाना महेश्वरम्} %१

\twolineshloka
{अथ कोलाहलं श्रुत्वा गणानां चन्द्रशेखरः}
{अभ्ययाद्वृषभारूढः सङ्ग्रामं प्रहसन्निव} %२

\twolineshloka
{रुद्रमायान्तमालोक्य सिंहनादैर्गणाः पुनः}
{निवृत्ताः सङ्गरे दैत्यान्निर्जघ्नुः शरवृष्टिभिः} %३

\twolineshloka
{दैत्याश्च भीषणं दृष्ट्वा सर्वे चैव विदुद्रुवुः}
{कार्तिकव्रतिनं दृष्ट्वा पातकानीव तद्भयात्} %४

\twolineshloka
{जलन्धरोथ तान्दैत्यान्निवृत्तान्प्रेक्ष्य सङ्गरे}
{रोषादधावच्चण्डीशं मुञ्चन्बाणान्सहस्रशः} %५

\twolineshloka
{शुम्भो निशुम्भोऽश्वमुखः कालनेमिर्बलाहकः}
{खड्गरोमा प्रचण्डश्च घस्मराद्याः शिवं ययुः} %६

\twolineshloka
{बाणान्धकारसञ्छन्नं दृष्ट्वा गणबलं शिवः}
{बाणजालमवाच्छिद्य स्वबाणैरावृणोन्नभः} %७

\twolineshloka
{दैत्यांश्च बाणवात्याभिः पीडितानकरोत्तदा}
{प्रचण्डबाणजालौघैरपातयत भूतले} %८

\twolineshloka
{खड्गरोम्णः शिरः कायात्तदा परशुनाऽच्छिनत्}
{बलाहकस्य च शिरः खट्वाङ्गेनाऽकरोद्द्विधा} %९

\twolineshloka
{बद्ध्वा च घस्मरं दैत्यं पाशेनाभ्यहनद्भुवि}
{वृषभेण हताः केचित्केचिद्बाणैर्निपातिताः} %१०

\twolineshloka
{न शेकुरसुराः स्थातुं गजाः सिंहार्दिता इव}
{ततः क्रोधपरीतात्मा वेगाद्रुद्रं जलन्धरः} %११


\onelineshloka*
{आह्वयामास समरे तीव्राशनिसमस्वनः}

\uvacha{जलन्धर उवाच}

\onelineshloka
{युध्यस्व च मया सार्द्धं किमेभिर्निहितैस्तव} %१२

\twolineshloka
{यच्च किञ्चिद्बलं तेऽस्ति तद्दर्शय जटाधर}
{इत्युक्त्वा बाणसप्तत्या जघान वृषभध्वजम्} %१३

\twolineshloka
{तान्प्राप्तान्निशितैर्बाणैश्चिच्छेद प्रहसन्निव}
{ततो हयान्ध्वजं छत्रं धनुश्चिच्छेद शक्तिभिः} %१४

\twolineshloka
{स च्छिन्नधन्वा विरथो गदामुद्यम्य वेगवान्}
{अभ्यधावच्छिवस्तावद्गदां बाणैर्विधाऽच्छिनत्} %१५

\twolineshloka
{तथाऽपि मुष्टिमुद्यम्य ययौ रुद्रं जिघांसया}
{तावच्छिवेन बाणौघैः क्रोशमात्रमपाकृतः} %१६

\twolineshloka
{ततो जलन्धरो दैत्यो मत्वा रुद्रं बलाधिकम्}
{ससर्ज मायां गान्धर्वीमद्भुतां रुद्रमोहिनीम्} %१७

\twolineshloka
{ततो जगुश्च ननृतुर्गन्धर्वाप्सरसां गणाः}
{तालवेणुमृदङ्गाद्यान्वादयन्ति स्म चापरे} %१८

\twolineshloka
{तद्दृष्ट्वा महदाश्चर्यं रुद्रो नादविमोहितः}
{पतितान्यपि शस्त्राणि करेभ्यो न विवेद सः} %१९

\twolineshloka
{एकाग्रीभूतमालोक्य रुद्रं दैत्यो जलन्धरः}
{कामार्तः स जगामाशु यत्र गौरी स्थिताऽभवत्} %२०

\twolineshloka
{युद्धे शुम्भनिशुम्भाख्यौ स्थापयित्वा महाबली}
{दशदोर्दण्डपचास्यस्त्रिनेत्रश्च जटाधरः} %२१

\twolineshloka
{महावृषभमारूढः स बभूव जलन्धरः}
{अथो रुद्रं समायान्तमालोक्य भववल्लभा} %२२

\twolineshloka
{अभ्याययौ सखीमध्यात्तद्दर्शनपथेऽभवत्}
{यावद्ददर्श चार्वङ्गी पार्वती दनुजेश्वरः} %२३

\twolineshloka
{तावत्स्ववीर्यं मुमुचे जडाङ्गश्चाभवत्तदा}
{अथ ज्ञात्वा तदा गौरी दानवं भयविह्वला} %२४

\twolineshloka
{जगामान्तर्हिता वेगात्सा तदोत्तरमानसे}
{तामदृष्ट्वा ततो दैत्यः क्षणाद्विद्युल्लतामिव} %२५

\twolineshloka
{जवेनाऽऽगात्पुनर्युद्धं यत्र देवो वृषध्वजः}
{पार्वत्यपि भयाद्विष्णुं सस्मार मनसा तदा} %२६


\onelineshloka*
{तावद्ददर्श तं देवं सूपविष्टं समीपगम्}

\uvacha{पार्वत्युवाच}

विष्णो जलन्धरो दैत्यः कृतवान्परमाद्भुतम्॥२७॥

\onelineshloka*
{तत्किं न विदितं तेऽस्ति चेष्टितं तस्य दुर्मतेः}

\uvacha{विष्णुरुवाच}

\onelineshloka
{तेनैव दर्शितः पन्था वयमप्यन्वयामहे} %२८


\onelineshloka*
{नान्यथा स भवेद्वध्यः पातिव्रत्यसुरक्षित्ः}

\uvacha{नारद उवाच}

\onelineshloka
{जगाम विष्णुरित्युक्त्वा पुनर्जालन्धरं पुरम्} %२९

\twolineshloka
{अथ रुद्रश्च गन्धर्वानुगतः सङ्गरे स्थितः}
{अन्तर्धानं गतां मायां दृष्ट्वा स बुबुधे तदा} %३०

\fourlineindentedshloka
{ततो भवो विस्मितमानसः पुन-}
{र्जगाम युद्धाय जलन्धरं रुषा}
{स चापि दैत्यः पुनरागतं शिवं}
{दृष्ट्वा शरौघैः समवाकिरद्रणे} %३१


\iti{जलन्धरोपाख्याने शिवजलन्धरयुद्धवर्णनं}{नाम विंशोऽध्यायः}{२०}

\sect{अथ नामैकविंशोऽध्यायः}


\uvacha{नारद उवाच}
\twolineshloka
{विष्णुर्जलन्धरं गत्वा तद्दैत्यपुटभेदनम्}
{पातिव्रत्यस्य भङ्गाय वृन्दायाश्चाऽकरोन्मतिम्} %१

\twolineshloka
{अथ वृन्दारका देवी स्वप्नमध्ये ददर्श ह}
{भर्तारं महिषारूढं तैलाभ्यक्तं दिगम्म्बरम्} %२

\twolineshloka
{कृष्णप्रसूनभूषाढ्यं क्रव्यादगणसेवितम्}
{दक्षिणाशागतं मुण्डं तमसाप्यावृतं तदा} %३

\twolineshloka
{स्वपुरं सागरे मग्नं सहसैवाऽऽत्मना सह}
{ततः प्रबुद्धा सा बाला तत्स्वप्नं प्रविचिन्वती} %४

\twolineshloka
{ददर्शोदितमादित्यं सच्छिद्रं निष्प्रभं मुहुः}
{तदनिष्टमिति ज्ञात्वा रुदती भयविह्वला} %५

\twolineshloka
{कुत्रचिन्नालभच्छर्म गोपुराट्टालभूमिषु}
{ततः सखीद्वययुता नगरोद्यानमागमत्} %६

\twolineshloka
{तत्रापि साऽभ्रमद्बाला नालभत्कुत्रचित्सुखम्}
{वनाद्वनान्तरं याता नैव वेदात्मनस्तदा} %७

\twolineshloka
{ततः सा भ्रमती बाला ददर्शातीवभीषणौ}
{राक्षसौ सिंहवदनौ दंष्ट्राननविभीषणौ} %८

\twolineshloka
{तौ दृष्ट्वा विह्वलाऽतीव पलायनपराऽभवत्}
{ददर्श तापसं शान्तं सशिष्यं मौनमास्थितम्} %९

\twolineshloka
{ततस्तत्कण्ठमावृत्य निजां बाहुलतां भयात्}
{मुने मां रक्ष शरणमागताऽस्मीत्यभाषत} %१०

\twolineshloka
{मुनिस्तां विह्वलां दृष्ट्वा राक्षसानुगतां तदा}
{हुङ्कारेणैव तौ घोरौ चकार विमुखौ रुषा} %११

\twolineshloka
{तौ हुङ्कारभयत्रस्तौ दृष्ट्वा च विमुखौ गतौ}
{प्रणम्य दण्डवद्भूमौ वृन्दा वचनमब्रवीत्} %१२


\uvacha{वृन्दोवाच}
\twolineshloka
{रक्षिताहं त्वया घोराद्भयादस्मात्कृपानिधे}
{किञ्चिद्विज्ञस्तुमिच्छामि कृपया तन्निशामय} %१३

\twolineshloka
{जलन्धरो हि मद्भर्ता रुद्रं योद्धुं गतः प्रभो}
{स तत्राऽऽस्ते कथं युद्धे तन्मे कथय सुव्रत} %१४


\uvacha{नारद उवाच}
\twolineshloka
{मुनिस्तद्वाक्यमाकर्ण्य कृपयोर्ध्वमवैक्षत}
{तावत्कपी समायातौ प्रणम्य चाग्रतः स्थितौ} %१५

\threelineshloka
{ततस्तद्भ्रूलतासंज्ञानियुक्तौ गगनं गतौ}
{गत्वा क्षणार्द्धादागत्य प्रणतावग्रतः स्थितौ}
{शिरःकबन्धे हस्तौ च गृहीत्वा समुपस्थितौ} %१६

\twolineshloka
{शिरःकबन्धे हस्तौ च दृष्ट्वाऽब्धितनयस्य सा}
{पपात मूर्छिता भूमौ भर्तृव्यसनदुःखिता} %१७

\twolineshloka
{कमण्डलूदकैः सिक्त्वा मुनिनाऽऽश्वासिता तदा}
{स्वभर्तृभाले सा भालं कृत्वा दीना रुरोद ह} %१८


\uvacha{वृन्दोवाच}
\twolineshloka
{यः पुरा सुखसंवादे विनोदयसि मां प्रभो}
{स कथं न वदस्यद्य वल्लभा मामनागसम्} %१९

\twolineshloka
{येन देवाः सगन्धर्वा निर्जिता विष्णुना सह}
{स कथं तापसेनाद्य त्रैलोक्यविजयी हतः} %२०


\uvacha{नारद उवाच}

\onelineshloka*
{रुदित्वेति तदा वृन्दा तं मुनिं वाक्यमब्रवीत्}

\uvacha{वृन्दोवाच}

\onelineshloka
{कृपानिधे मुनिश्रेष्ठ जीवयैनं मम प्रियम्} %२१


त्वमेवास्य मुने शक्तो जीवनाय मतो मम

\uvacha{नारद उवाच}

\onelineshloka
{इति तद्वाक्यमाकर्ण्य प्रहसन्मुनिरब्रवीत्} %२२


\uvacha{मुनिरुवाच}
\twolineshloka
{नायं जीवयितुं शक्तो रुद्रेण निहतो युधि}
{तथाऽपि त्वत्कृपाविष्ट एनं सञ्जीवयाम्यहम्} %२३


\uvacha{नारद उवाच}
\twolineshloka
{इत्युक्त्वान्तर्दधे विप्रस्तावत्सागरनन्दनः}
{वृन्दामालिङ्ग्य तद्वक्त्रं चुचुम्ब प्रीतमानसः} %२४

\twolineshloka
{अथ वृन्दाऽपि भर्तारं दृष्ट्वा हर्षितमानसा}
{रेमे तद्वनमध्यस्था तद्युक्ता बहुवासरम्} %२५

\twolineshloka
{कदाचित्सुरतस्यान्ते दृष्ट्वा विष्णुं तमेव च}
{निर्भर्त्स्य क्रोधसंयुक्ता वृन्दा वचनमब्रवीत्} %२६


\uvacha{वृन्दोवाच}
\twolineshloka
{धिक्त्वदीयं हरे शीलं परदाराभिगामिनः}
{ज्ञातोऽसि त्वं मया सम्यङ्मायाप्रच्छन्नतापसः} %२७

\twolineshloka
{यौ त्वया मायया द्वाःस्थौ स्वकीयौ दर्शितौ मम}
{तावेव राक्षसौ भूत्वा भार्यां तव हरिष्यतः} %२८

\twolineshloka
{त्वं चापि भार्यादुःखार्तो वने कपिसहायवान्}
{भ्रम सर्पेश्वरेणाऽयं यस्ते शिष्यत्वमागतः} %२९

\twolineshloka
{इत्युक्त्वा सा तदा वृन्दा प्राविशद्धव्यवाहनम्}
{विष्णुना वार्यमाणाऽपि तस्यामासक्तचेतसा} %३०

\fourlineindentedshloka
{ततो हरिस्तामनुसंस्मरन्मुहु-}
{र्वृन्दान्वितोभस्मरजोवगुण्ठितः}
{तत्रैव तस्थौ सुरसिद्धसङ्घैः}
{प्रबोध्यमानोऽपि ययौ न शान्तिम्} %३१


\iti{जलन्धरोपाख्याने वृन्दाग्निप्रवेशवर्णनं}{नामैकशोऽध्यायः}{२१}

\sect{अथ द्वाविंशोऽयायः}


\uvacha{नारद उवाच}
\twolineshloka
{ततो जलन्धरो दृष्ट्वा रुद्रमद्भुतविक्रमम्}
{चकार मायया गौरीं त्र्यम्बकं मोहयन्निव} %१

\twolineshloka
{रथोपरि च तां बद्धां रुदन्तीं पार्वतीं शिवः}
{निशुम्भप्रमुखाद्यैश्च वध्यमानां ददर्श सः} %२

\twolineshloka
{गौरीं तथाविधां दृष्ट्वा शिवोऽप्युद्विग्नमानसः}
{अवाङ्मुखः स्थितस्तूष्णीं विस्मृत्य स्वपराक्रमम्} %३

\twolineshloka
{ततो जलन्धरो वेगात्त्रिभिर्विव्याध सायकैः}
{आपुङ्खमग्रैस्तं रुद्रं शिरस्युरसि चोदरे} %४

\twolineshloka
{ततो जज्ञे स तां मायां विष्णुना च प्रबोधितः}
{रौद्ररूपधरो जातो ज्वालामालाऽतिभीषणः} %५

\twolineshloka
{तस्यातीव महारौद्रं रूपं दृष्ट्वा महासुराः}
{न शेकुः सम्मुखे स्थातुं भेजिरे ते दिशो दश} %६

\twolineshloka
{ततः शापं ददौ रुद्रस्तयोः शुम्भनिशुम्भयोः}
{मम युद्धादपक्रान्तौ गौर्या बध्यौ भविष्यथः} %७

\twolineshloka
{पुनर्जलन्धरो वेगाद्ववर्ष निशितैः शरैः}
{बाणान्धकारैः सञ्छन्नं तदा भूमितलं महत्} %८

\twolineshloka
{यावद्रुद्रश्च चिच्छेद तस्य बाणगणं जवात्}
{तावत्स परिघेणाऽऽशु जघान वृषभं बली} %९

\twolineshloka
{वृषस्तेन प्रहारेण परावृत्तो रणाङ्गणात्}
{रुद्रेणाऽऽष्यमाणोऽपि न तस्थौ रणभूमिषु} %१०

\twolineshloka
{ततः परमसङ्कुद्धो रुद्रो रौद्रवपुर्धरः}
{चक्रं सुदर्शनं वेगाच्चिक्षेऽदित्यवर्चसम्} %११

\twolineshloka
{प्रदहद्रोदसी वेगात्पपात वसुधातले}
{जहार तच्छिरः कायान्महदायतलोचनम्} %१२

\twolineshloka
{रथात्कायः पपातास्य नादयन्वसुधातलम्}
{तेजश्च निर्गतं देहात्तद्रुद्रे लयमागमत्} %१३

\twolineshloka
{वृन्दादेहोद्भवं तेजस्तद्गौर्यां विलयं गतम्}
{अथ ब्रह्मादयो देवा हर्षादुत्फुल्ललोचनाः} %१४


\onelineshloka*
{प्रणम्य शिरसा रुद्रं शशंसुर्विष्णुचेष्टितम्}

\uvacha{देवा ऊचुः}

\onelineshloka
{महादेव त्वया देवा रक्षिताः शत्रुजाद्भयात्} %१५

\twolineshloka
{किञ्चिदन्यत्समुद्भूतं तत्र किं करवामहे}
{वृन्दालावण्यसम्भ्रान्तो विष्णुस्तिष्ठति मोहितः} %१६


\uvacha{ईश्वर उवाच}
\twolineshloka
{गच्छध्वं शरणं देवा विष्णोर्मोहापनुत्तये}
{शरण्यां मोहिनीं मायां सा वः कार्यं करिष्यति} %१७


\uvacha{नारद उवाच}
\twolineshloka
{इत्युक्त्वान्तर्दधे देवः सर्वभूतगणैस्तदा}
{देवाश्च तुष्टुवुर्मूलप्रकृतिं भक्तवत्सलाम्} %१८


\uvacha{देवा ऊचुः}
\fourlineindentedshloka
{यदुद्भवाः सत्त्वरजस्तमोगुणाः}
{सर्गस्थितिध्वंसनिदानकारिणः}
{यदिच्छया विश्वमिदं भवाभवौ}
{तनोति मूलप्रकृतिं नताः स्म ताम्} %१९

\twolineshloka
{या हि त्रयोविंशतिभेदशब्दिता जगत्यशेषे समधिष्ठिता परा}
{यद्रूपकर्माणि जडास्त्रयोऽपि देवा न विद्युः प्रकृतिं नताः स्म ताम्} %२०

\fourlineindentedshloka
{यद्भक्तियुक्ताः पुरुषास्तु नित्यं}
{दारिद्र्यभीमोहपराभवादीन्}
{न प्राप्नुवन्त्येव हि भक्तवत्सलां}
{सदैव मूलप्रकृतिं नताः स्म ताम्} %२१


\uvacha{नारद उवाच}
\twolineshloka
{स्तोत्रमेतत्त्रिसन्ध्यं यः पठेदेकाग्रमानसः}
{दारिद्यमोहदुःखानि न कदाचित्स्पृशन्ति तम्} %२२

\twolineshloka
{इत्थं स्तुवन्तस्ते देवास्तेजोमण्डलमास्थितम्}
{ददृशुर्गगनं तत्र ज्वालाव्याप्तदिगन्तरम्} %२३


\onelineshloka*
{तन्मध्याद्भारतीं सर्वे शुश्रुवुर्व्योमचारिणीम्}

\uvacha{शक्तिरुवाच}

\onelineshloka
{अहमेव त्रिधा भिन्ना तिष्ठामि त्रिविधैर्गुणैः} %२४

\twolineshloka
{गौरी लक्ष्मी स्वरा चेति रजःसत्त्वतमोगुणैः}
{तत्र गच्छत ताः कार्यं विधास्यति च वः सुराः} %२५


\uvacha{नारद उवाच}
\twolineshloka
{शृण्वतामिति तां वाचमन्तर्धानमगान्महः}
{देवानां विस्मयोत्फुल्लनेत्राणां तत्तदा नृप} %२६

\twolineshloka
{ततः सर्वेऽपि ते देवा गत्वा तद्वाक्यनोदिताः}
{गौरीं लक्ष्मीं स्वरां चैव प्रणेमुर्भक्तितत्पराः} %२७

\twolineshloka
{ततस्तास्तान्सुरान्दृष्ट्वा प्रणतान्भक्तवत्सलाः}
{बीजानि प्रददुस्तेभ्यो वाक्यान्यूचुश्च भूमिप} %२८


\uvacha{देव्य ऊचुः}
\twolineshloka
{इमानि तत्र बीजानि विष्णुर्यत्रावतिष्ठते}
{निर्वपध्वं ततः कार्यं भवतां सिद्धिमेष्यति} %२९

\uvacha{नारद उवाच}
\fourlineindentedshloka
{ततस्तु हृष्टाः सुरसिद्धसङ्घाः}
{प्रगृह्य बीजानि विचिक्षिपुस्ते}
{वृन्दान्वितो भूमितले स यत्र}
{विष्णुः सदा तिष्ठति सौख्यहीनः} %३०


\iti{जलन्धरमुक्तिकथनं}{नाम द्वाविंशोऽध्यायः}{२२}

\sect{अथ त्रयोविंशोऽध्यायः}


\uvacha{नारद उवाच}
\twolineshloka
{क्षिप्तेभ्यस्तत्र बीजेभ्यो वनस्पत्यस्त्रयोऽभवन्}
{धात्री च मालती चैव तुलसी च नृपोत्तम} %१

\twolineshloka
{धात्र्युद्भवा स्मृता धात्री माभवा मालती स्मृता}
{गौरीभवा च तुलसी तमःसत्त्वरजोगुणाः} %२

\twolineshloka
{स्त्रीरूपिण्यौ वनस्पत्यौ दृष्ट्वा विष्णुस्तदा नृप}
{उत्तस्थौ सम्भ्रमाद्वृन्दा रूपातिशयविभ्रमः} %३

\twolineshloka
{दृष्ट्वा च याचते मोहात्कामासक्तेन चेतसा}
{तं चापि तुलसीधात्र्यौ रागेणैव व्यलोकताम्} %४

\twolineshloka
{यच्च लक्ष्म्या पुरा बीजमीर्ष्ययैव समर्पितम्}
{तस्मात्तदुद्भवा नारी तस्मिन्नीर्ष्यापराऽभवत्} %५

\twolineshloka
{अतः सा बर्बरीत्याख्यामवापाध विगर्हिताम्}
{धात्रीतुलस्यौ तद्रागात्तस्य प्रीतिप्रदे सदा} %६

\twolineshloka
{ततो विस्तदुःखोऽसौ विष्णुस्ताभ्यां सहैव तु}
{वैकुण्ठमगमद्धृष्टः सर्वदेव नमस्कृतः} %७

\twolineshloka
{कार्तिकोद्यापने विष्णोस्तस्मात्पूजा विधीयते}
{तुलसीमूलदेशेऽस्य प्रीतिदा सा यतः स्मृता} %८

\twolineshloka
{तुलसीकाननं राजन्गृहे स्यावतिष्ठते}
{तद्गृहं तीर्थरूपं तु नाऽऽयान्ति यमकिङ्कराः} %९

\twolineshloka
{सर्वपापहरं नित्यं कामदं तुलसीवनम्}
{रोपयन्ति नराः श्रेष्ठास्ते न पश्यन्ति भास्करिम्} %१०

\twolineshloka
{दर्शनं नर्मदायास्तु गङ्गास्नानं तथैव च}
{तुलसीवनसंसर्गः सममेव त्रयं स्मृतम्} %११

\twolineshloka
{रोपणात्पालनात्सेकाद्दर्शनात्स्पर्शनान्नृणाम्}
{तुलसी दहते पापं वाङ्मनःकायसञ्चितम्} %१२

\twolineshloka
{तुलसीमञ्जरीभिर्यः कुर्याद्धरिहरार्चनम्}
{न स गर्भगृहं याति मुक्तिभागी न संशयः} %१३

\twolineshloka
{पुष्कराद्यानि तीर्थानि गङ्गाद्याः सरितस्तथा}
{वासुदेवादयो देवास्तिष्ठन्ति तुलसीदले} %१४

\twolineshloka
{तुलसीमञ्जरीयुक्तो यस्तु प्राणान्विमुञ्चति}
{यमोऽपि नेक्षितुं शक्तो युक्तं पापशतैरपि} %१५

\twolineshloka
{विष्णोः सायुज्यमाप्नोति सत्यं सत्यं नृपोत्तम}
{तुलसीकाष्ठजं यस्तु चन्दनं धारयेन्नरः} %१६

\twolineshloka
{तद्देहं न स्पृशेत्पापं क्रियमाणमपीह यत्}
{तुलसीविपिनच्छाया यत्रयत्र भवेन्नृप} %१७

\twolineshloka
{तत्र श्राद्धं प्रकर्तव्यं पितॄणां दत्तमक्षयम्}
{धात्रीफलविमिश्रैश्च तुलसीपत्रमिश्रितैः} %१८

\twolineshloka
{जलैः स्नाति नरस्तस्य गङ्गास्नानफलं स्मृतम्}
{देवार्चनं नरः कुर्याद्धात्रीपत्रैः फलैस्तथा} %१९

\twolineshloka
{सुवर्णमणिमुक्तौघैरर्चनस्याप्नुयात्फलम्}
{तीर्थानि मुनयो देवा यज्ञाः सर्वेऽपि कार्तिके} %२०

\twolineshloka
{नित्यं धात्रीं समाश्रित्य तिष्ठन्त्यर्के तुलास्थिते}
{द्वादश्यां तुलसीपत्रं धात्रीपत्रं तु कार्तिके} %२१

\threelineshloka
{लुनाति स नरो गच्छेन्निरयानतिगर्हितान्}
{धात्रीतुलस्योर्माहात्म्यमपि देवश्चतुर्मुखः}
{न समर्थो भवेद्वक्तुं यथा देवस्य शार्ङ्गिणः} %२२

\fourlineindentedshloka
{धात्रीतुलस्युद्भवकारणं यः}
{शृणोति यः श्रावयते च भक्त्या}
{विधूतपाप्मा सह पूर्वजैः स्वैः}
{स्वर्गं व्रजत्यग्र्यविमानसंस्थैः} %२३


\iti{धात्रीतुलस्युत्पत्तिवर्णनं}{नाम त्रयोविंशोऽध्यायः}{२३}

\sect{अथ चतुर्विंशोऽध्यायः}


\uvacha{पृथुरुवाच}
\twolineshloka
{यदूर्जव्रतिनः पुंसः फलं महदुदादृतम्}
{तत्पुनर्ब्रूहि माहात्म्यं केन चीर्णमिदं शुभम्} %१


\uvacha{नारद उवाच}
\twolineshloka
{आसीत्सह्याद्रिविषये करवीरपुरे पुरा}
{ब्राह्मणो धर्मवित्कश्चिद्धर्मदत्तेति विश्रुतः} %२

\twolineshloka
{विष्णुव्रतकरः सम्यग्विष्णुपूजारतः सदा}
{कदाचित्कार्तिके मासि हरिजागरणाय सः} %३

\twolineshloka
{रात्र्यां तुर्यावशेषायां जगाम हरिमदिरम्}
{हरिपूजोपकरणान्प्रगृह्य व्रजता तदा} %४

\twolineshloka
{तेन दृष्टा समायाता राक्षसी भीमदर्शना}
{तां दृष्ट्वा भयवित्रस्तः कम्पितावयवस्तदा} %५

\threelineshloka
{पूजोपकरणैः सर्वैः पयोभिश्चाहनद्भयात्}
{संस्मृत्य तद्धरेर्नाम तुलसीयुक्तवारिणा}
{तेन वै हतमात्रे तु पापं तस्या ह्यगाल्लयम्} %६

\twolineshloka
{अथ संस्मृत्य सा पूर्वजन्मकर्मविपाकजाम्}
{स्वां दशामब्रवीद्विप्रं दण्डवच्च प्रणम्य वै} %७


\uvacha{कलहोवाच}
\twolineshloka
{पूर्वकर्मविपाकेन दशामेतां गतास्म्यहम्}
{तत्कथं नु पुनर्विप्र प्रयास्याम्युत्तमां गतिम्} %८


\uvacha{नारद उवाच}
\twolineshloka
{तां दृष्ट्वा प्रणतां सम्यग्वदमानां स्वकर्म तत्}
{अतीव विस्मितो विप्रस्तदा वचनमब्रवीत्} %९

\uvacha{धर्मदत्त उवाच}
\twolineshloka
{केन कर्मविपाकेन त्वं दशामीदृशीं गता}
{कुत्रत्या का च किं शीला तत्सर्वं कथयस्व मे} %१०


\uvacha{कलहोवाच}
\twolineshloka
{सौराष्ट्रनगरे ब्रह्मन् भिक्षुर्नामाभवद्द्विजः}
{तस्याहं गृहिणी पूर्वं कलहाख्याऽतिनिष्ठुरा} %११

\twolineshloka
{न कदाचिन्मया भर्तुर्वचसाऽपि शुभं कृतम्}
{नार्पितं तस्य मिष्टान्नं भर्तुर्वचनशीलया} %१२

\twolineshloka
{कलहप्रियया नित्यं मयोद्विग्नमना यदा}
{परिणेतुं यदाऽन्यां स मतिं चक्रे पतिर्मम} %१३

\twolineshloka
{ततो गरं समादाय प्राणास्त्यक्ता मया द्विज}
{अथ बद्ध्वा बध्यमानां मां निन्युर्यमकिङ्कराः} %१४


\onelineshloka
{यमश्च मां तदा दृष्ट्वा चित्रगुप्तमपृच्छत} %१५


\uvacha{यम उवाच}
\twolineshloka
{अनया किं कृतं कर्म चित्रगुप्त विलोकय}
{प्राप्नोत्वेषा च तत्कर्म शुभं वा यदि वाऽशुभम्} %१६


\uvacha{कलहोवाच}

\onelineshloka*
{चित्रगुप्तस्तदा वाक्यं भर्त्सयन्मामुवाच सः}

\uvacha{चित्रगुप्त उवाच}

\onelineshloka
{अनया तु कृतं कर्म शुभं किञ्चिन्न विद्यते} %१७

\twolineshloka
{मिष्टान्नं भुञ्जमानेयं न भर्तरि तदर्पितम्}
{अतश्च वल्गुलीयोन्यां विष्ठादाऽवतिष्ठतु} %१८

\twolineshloka
{भर्तुर्द्वेषात्तदाप्येषा नित्यं कलहकारिणी}
{विष्ठादां सूकरीं योनिं तस्मात्तिष्ठत्वियं हरे} %१९

\twolineshloka
{पाकभाण्डे सदा भुङ्क्ते भुङ्क्ते चैका यतस्ततः}
{तस्मादेषा बिडाल्यस्तु स्वजाताऽपत्यभक्षिणी} %२०

\twolineshloka
{भर्तारमपि चोद्दिश्य ह्यात्मघातः कृतोऽनया}
{तस्मात्प्रेतशरीरेऽपि तिष्ठत्वेकाऽतिनिन्दिता} %२१

\twolineshloka
{अतश्चैषा मरुद्देशं प्रापितव्या भटैरियम्}
{तत्र प्रेतशरीरस्था चिरं तिष्ठत्वियं ततः} %२२


\onelineshloka
{ऊर्ध्वं योनित्रयं चैषा भुनक्त्वशुभकारिणी} %२३


\uvacha{कलहोवाच}
\threelineshloka
{साहं पञ्चशताब्दानि प्रेतदेहे स्थिता किल}
{क्षुतृड्भ्यां पीडिताऽऽविश्य शरीरं वणिजस्य च}
{आयाता दक्षिणं देशं कृष्णावेण्योश्च सङ्गमम्} %२४

\twolineshloka
{तत्तीरं संश्रिता यावत्तावत्तस्य शरीरतः}
{शिवविष्णुगणैर्दूरमपकृष्टा बलादहम्} %२५

\twolineshloka
{ततः क्षुत्क्षामया दृष्टो मया हि त्वं द्विजोत्तम}
{त्वद्धस्ततुलसीवारिसंसर्गगतपापया} %२६

\twolineshloka
{तत्कृत्यं कुरु विप्रेन्द्र कथं मुक्तिमियाम्यहम्}
{योनित्रयादग्रभवादस्माच्च प्रेतदेहतः} %२७

\twolineshloka
{इत्थं विचिन्त्य कलहावचनं द्विजाग्र्यस्तत्कर्मपाकभयविस्मयदुःखयुक्तः}
{तद्ग्लानिदर्शनकृपाचलचित्तवृत्ति ध्यात्वा चिरं स वचनं निजगाद दुःखात्} %२८


\iti{धर्मदत्तेतिहासकथनं}{नाम चतुर्विंशोऽध्यायः}{२४}

\sect{अथ पञ्चविंशोऽध्यायः}


\uvacha{धर्मदत्त उवाच}
\twolineshloka
{विलयं यान्ति पापानि तीर्थे दानव्रतादिभिः}
{प्रेतदेहस्थितायास्ते तेषु नैवाऽधिकारिता} %१

\twolineshloka
{त्वद्ग्लानिदर्शनादस्मात्खिन्नं च मम मानसम्}
{न वै निर्वृतिमायाति त्वामनुद्धृत्य दुःखिताम्} %२

\twolineshloka
{तस्मादाजन्मचरितं यन्मया कार्तिकव्रतम्}
{तत्पुण्यस्यार्द्धभागेन सद्गतिं त्वमवाप्नुहि} %३


\uvacha{नारद उवाच}
\twolineshloka
{इत्युक्त्वा धर्मदत्तोऽसौ यावत्तामभ्यषेचयत्}
{तुलसीमिश्रतोयेन श्रावयन्द्वादशाक्षरम्} %४

\twolineshloka
{तावत्प्रेतत्वनिर्मुक्ता ज्वलदग्निशिखोपमा}
{दिव्यरूपधरा जाता लावण्येन यथेन्दिरा} %५

\twolineshloka
{ततः सा दण्डवद्भूमौ प्रणनामाथ तं द्विजम्}
{उवाच सा तदा वाक्यैर्हर्षगद्गदभाषिणी} %६


\uvacha{कलहोवाच}
\twolineshloka
{त्वत्प्रसादाद्द्विजश्रेष्ठ विमुक्ता निरयादहम्}
{पापाब्धौ मज्जमानायास्त्वं नौभूतोऽसि मे ध्रुवम्} %७


\uvacha{नारद उवाच}
\twolineshloka
{इत्थं वदन्ती सा विप्रं ददर्शाऽऽयातमम्बरात्}
{विमानं भास्वरं युक्तं विष्णुरूपधरैर्गणैः} %८

\twolineshloka
{अथ सा तद्विमानाग्र्यं द्वाःस्थाभ्यामवरोपिता}
{पुण्यशीलसुशीलाभ्यामप्सरोगणसेविता} %९

\twolineshloka
{तद्विमानं तदाऽपश्यद्धर्मदत्तः सविस्मयः}
{पपात दण्डवद्भूमौ दृष्ट्वा तौ विष्णुरूपिणौ} %१०

\twolineshloka
{पुण्यशीलसुशीलौ च तमुत्थाप्याऽऽनतं द्विजम्}
{अभिनन्द्य ततो वाक्यमूचतुर्धर्मसंयुतम्} %११


\uvacha{गणावूचतुः}
\twolineshloka
{साधुसाधु द्विजश्रेष्ठ यस्त्वं विष्णुरतः सदा}
{दीनानुकम्पी सर्वज्ञो विष्णुव्रतपरायणः} %१२


\twolineshloka
{आ बालत्वाच्छुभं त्वेतद्यत्त्वया कार्तिकव्रतम्}
{कृतं तस्यार्द्धदानेन पुण्यं द्वैगुण्यमागमत्} %१३

\twolineshloka
{जन्मान्तरशतोद्भूतं पापं तद्विलयं गतम्}
{स्नानैरेव गतं पापं यदस्याः पूर्वकर्मजम्} %१४

\twolineshloka
{हरिजागरणाद्यैश्च विमानमिदमास्थिता}
{वैकुण्ठे नीयते साधो नानाभोगयुता त्वियम्} %१५

\threelineshloka
{दीपदानभवैः पुण्यैस्तेजःसारूप्यमास्थिता}
{तुलसीपूजनाद्यैश्च कार्तिकव्रतकैः शुभैः}
{विष्णुसान्निध्यगा जाता त्वया दत्तैः कृपानिधे} %१६

\twolineshloka
{त्वमप्यस्य भवस्यान्ते भार्याभ्यां सह यास्यसि}
{वैकुण्ठ भुवनं विष्णोः सान्निध्यं च सरूपताम्} %१७

\twolineshloka
{ते धन्याः कृतकृत्यास्ते तेषां च सफलो भवः}
{यैर्भक्त्याराधितो विष्णुर्धर्मदत्त यथा त्वया} %१८

\twolineshloka
{सम्यगाराधितो विष्णुः किं न यच्छति देहिनाम्}
{औत्तानचरणिर्येन ध्रुवत्वे स्थापितः पुरा} %१९


\onelineshloka
{यन्नामस्मरणादेव देहिनो यान्ति सद्गतिम्} %२०

\twolineshloka
{ग्राहग्रस्तो हि नागेन्द्रो यन्नामस्मरणात्पुरा}
{विमुक्तः सन्निधिं प्राप्तो जातोऽयं जयसंज्ञकः} %२१

\twolineshloka
{यतस्त्वयाऽर्चितो विष्णुस्तत्सान्निध्यं प्रयास्यसि}
{बहून्यब्दसहस्राणि भार्याद्वययुतः किल} %२२

\twolineshloka
{ततः पुण्यक्षयेजाते यदा यास्यसि भूतलम्}
{सूर्यवंशोद्भवो राजा विख्यातस्त्वं भविष्यसि} %२३

\twolineshloka
{नाम्ना दशरथस्तत्र भार्याद्वययुतः पुनः}
{तृतीययाऽनया चापि या ते पुण्यार्द्धभागिनी} %२४

\twolineshloka
{तत्रापि तव सान्निध्यं विष्णुर्यास्यति भूतले}
{आत्मानं तव पुत्रत्वे प्रकल्प्याऽमरकार्यकृत्} %२५

\twolineshloka
{तव जन्मव्रतादस्माद्विष्णुसन्तुष्टिकारकात्}
{न यज्ञा न च दानानि न तीर्थान्यधिकानि वै} %२६

\twolineshloka
{धन्योऽसि विप्राग्र्य यतस्त्वयैतद्व्रतं कृतं तुष्टिकरं जगद्गुरोः}
{यदर्धभागात्सफला मुरारेः प्रणीयतेऽस्माभिरियं सलोकताम्} %२७


\iti{धर्मदत्तोपाख्याने कलहामोक्षकथनं}{नाम पञ्चविंशोऽध्यायः}{२५}

\sect{अथ षड्विंशोऽध्यायः}


\uvacha{नारद उवाच}
\twolineshloka
{इत्थं तद्वचनं श्रुत्वा धर्मदत्तः सविस्मयः}
{प्रणम्य दण्डवद्भूमौ वाक्यमेतदुवाच ह} %१

\uvacha{धर्मदत्त उवाच}
\twolineshloka
{आराधयन्ति सर्वेऽपि विष्णुं भक्तार्तिनाशनम्}
{यज्ञैर्दानैर्व्रतैस्तीर्थैस्तपोभिश्च यथाविषि} %२

\twolineshloka
{विष्णुप्रीतिकरं तेषां किञ्चित्सान्निध्यकारकम्}
{यत्कृत्वा तानि चीर्णानि सर्वाण्यपि भवन्ति हि} %३


\uvacha{गणावूचतुः}
\twolineshloka
{साधु पृष्टं त्वया विप्र शृणुष्वैकाग्रमानसः}
{सेतिहासकथां पुण्यां कथ्यमानां पुराभवाम्} %४


\twolineshloka
{काञ्चिपुर्यां पुरा चोलश्चक्रवर्ती नृपोऽभवत्}
{यस्याख्ययैव ते देशाश्चोला इति प्रथां गताः} %५

\twolineshloka
{यस्मिञ्छासति भूचक्रं दरिद्रो वाऽपि दुःखितः}
{पापबुद्धिः सरुग्वापि नैव कश्चिदभून्नरः} %६

\twolineshloka
{यस्याप्युन्नतयज्ञस्य ताम्रपर्ण्यास्तटावुभौ}
{सुवर्णयूपैः शोभाढ्यावास्तां चैत्ररथोपमौ} %७

\twolineshloka
{स कदाचिदगाद्राजा ह्यनन्तशयनं द्विज}
{यत्रासौ जगतां नाथो योगनिद्रामुपाश्रितः} %६

\twolineshloka
{तत्र श्रीरमणं देवं सम्पूज्य विधिवन्नृपः}
{मणिमुक्ताफलैर्दिव्यैः स्वर्णपुष्पैश्च शोभनैः} %९

\twolineshloka
{प्रणम्य दण्डवद्भूमावुपविष्टः स तत्र वै}
{तावद्ब्राह्मणमायातमपश्यद्देवसन्निधौ} %१०

\twolineshloka
{देवार्चनार्थं पाणौ तु तुलस्युदकधारिणम्}
{स्वपुरीवासिनं तत्र विष्णुदासाह्वय द्विजम्} %११

\twolineshloka
{स तत्राभ्येत्य विप्रर्षिर्देवदेवमपूजयत्}
{विष्णुसूक्तेन संस्नाप्य तुलसीमञ्जरीदलैः} %१२

\twolineshloka
{तुलसीपूजया तस्य रत्नपूजां पुरा कृताम्}
{आच्छादितां समालोक्य राजा क्रुद्धोऽब्रवीदिदम्} %१३


\uvacha{चोल उवाच}
\twolineshloka
{माणिक्यस्वर्णपूजाऽत्र शोभाढ्या या कृता मया}
{विष्णुदास कथं सेयमाच्छन्ना तुलसीदलैः} %१४

\twolineshloka
{विष्णुभक्तिं न जानासि वराकोऽसि मतो मम}
{यस्त्विमामतिशोभाढ्यां पूजामाच्छादयस्यहो} %१५

\twolineshloka
{इति तद्वचनं श्रुत्वा सक्रोधः स द्विजोत्तमः}
{राज्ञो गौरवमुल्लङ्घ्य जगाद वचनं तदा} %१६


\uvacha{विष्णुदास उवाच}
\twolineshloka
{राजन्भक्तिं न जानासि गर्वितोऽसि नृपश्रिया}
{कियद्विष्णुव्रतं पूर्वं त्वया चीर्णं वदस्व तत्} %१७


\uvacha{गणावूचतुः}
\twolineshloka
{तद्ब्राह्मणवचः श्रुत्वा प्रहस्य स नृपोत्तमः}
{विष्णुदासं तदा गर्वादुवाच वचनं द्विजम्} %१८


\uvacha{राजोवाच}
\twolineshloka
{इत्थं चेद्वदसे विप्र विष्णुभक्त्याऽतिगर्वितः}
{भक्तिस्ते कियती विष्णोर्दरिद्रस्याऽधनस्य च} %१९


\twolineshloka
{यज्ञदानादिकं नैव विष्णोस्तुष्टिकरं कृतम्}
{नापि देवालयं पूर्वं कृतं विप्र त्वया क्वचित्} %२०

\twolineshloka
{ईदृशस्यापि ते गर्व एष तिष्ठति भक्तितः}
{तच्छृण्वन्तु वचो मेऽद्य सर्वेऽप्येते द्विजातयः} %२१

\twolineshloka
{साक्षात्कारमहं विष्णोरेष वादो गमिष्यति}
{पश्यन्तु सर्वेऽपि ततो भक्तिं ज्ञास्यन्ति चावयोः} %२२


\uvacha{गणावूचतुः}
\twolineshloka
{इत्युक्त्वा स नृपोऽगच्छन्निजराजगृहं तदा}
{आरभद्वैष्णवं सत्रं कृत्वाऽऽचार्यं तु मुद्गलम्} %२३


\twolineshloka
{ऋषिसङ्घसमाजुष्टं बह्वन्नं बहुदक्षिणम्}
{यच्च ब्रह्मकृतं पूर्वं गयाक्षेत्रे समृद्धिमत्} %२४

\twolineshloka
{विष्णुदासोऽपि तत्रैव तस्थौ देवालये व्रती}
{यथोक्तनियमान्कुर्वन्विष्णोस्तुष्टिकरान्सदा} %२५

\twolineshloka
{माघोर्जयोर्व्रतं सम्यक्तुलसीवनपालनम्}
{एकादश्यां हरेर्जाप्यं द्वादशाक्षरविद्यया} %२६

\twolineshloka
{उपचारैः षोडशभिर्नृत्यगीतादिमङ्गलैः}
{नित्यं विष्णोस्तथा पूजां व्रतान्येतानि सोऽकरोत्} %२७

\twolineshloka
{नित्यं संस्मरणं विष्णोर्गच्छन्भुवि स्वपन्नपि}
{सर्वभूतस्थितं विष्णुमपश्यत्समदर्शनः} %२८

\twolineshloka
{माघकार्तिकयोर्नित्यं विशेषनियमानपि}
{अकरोद्विष्णुतुष्ट्यर्थं सोद्यापनविधिं तथा} %२९

\fourlineindentedshloka
{एवं समाराधयतोः श्रियः पतिं}
{तयोश्च चोलेश्वरविष्णुदासयोः}
{अगाद्धि कालः सुमहान्व्रतस्थयो-}
{स्तन्निष्ठसर्वेन्द्रियकर्मणोस्तदा} %३०


\iti{चोलराजविष्णुदासब्राह्मणविवादकथनं}{नाम षड्विंशोऽध्यायः}{२६}

\sect{अथ सप्तविंशोऽध्यायः}


\uvacha{नारद उवाच}
\twolineshloka
{कदाचिद्विष्णुदासोऽथ कृत्वा नित्यविधिं द्विज}
{स पाकमकरोत्तावदहरत्कोऽप्यलक्षितः} %१

\twolineshloka
{तमदृष्ट्वाऽप्यसौ पाकं पुनर्नैवाकरोत्तदा}
{सायङ्कालार्चनस्यासौ व्रतभङ्गभयाद्द्विजः} %२

\twolineshloka
{द्वितीयेऽह्नि पुनः पाकं कृत्वा यावत्स विष्णवे}
{उपहारार्पणं कर्तुं गतः कोऽप्यहरत्पुनः} %३

\twolineshloka
{एवं सप्तदिनं तस्य पाकं कोऽप्यहरन्नृप}
{ततः सविस्मयश्चाथ मनस्येवमधारयत्} %४

\twolineshloka
{अहो नित्यं समभ्येत्य कः पाकं हरते मम}
{क्षेत्रसन्न्यासिनः स्थानं न त्याज्यं मम सर्वथा} %५

\twolineshloka
{पुनः पाकं विधायात्र भुज्यते यदि चेन्मया}
{सायङ्कालार्चनं चैव परित्याज्यं कथं भवेत्} %६

\twolineshloka
{यदि पाकं विधायैवं भोक्तव्यं तु मया न तत्}
{अनिवेद्य हरौ सर्वं वैष्णवैर्नैव भुज्यते} %७

\twolineshloka
{उपोषितोऽहं सप्ताहं तिष्ठाम्यत्र व्रतस्थितः}
{अद्यसंरक्षणं सम्यक्पाकस्यात्र करोम्यहम्} %८

\twolineshloka
{इति पाकं विधायासौ तत्रैवालक्षितः स्थितः}
{तावद्ददर्श चण्डालं पाकान्नहरणे स्थितम्} %९

\twolineshloka
{क्षुत्क्षामं दीनवदनमस्थिचर्मावशेषितम्}
{तमालोक्य द्विजाग्र्योऽभूत्कृपयाऽन्वितमानसः} %१०

\twolineshloka
{विलोक्यान्नहरं विप्रस्तिष्ठतिष्ठेत्यभाषत}
{कथमश्नासि तद्रूक्षं घृतमेतद्गृहाण भोः} %११

\twolineshloka
{इत्थं वदन्तं विप्राग्र्यमायान्तं स विलोक्य च}
{वेगादधावत्तद्भीत्या मृर्च्छितश्च पपात ह} %१२

\twolineshloka
{भीतं सम्मूर्च्छितं दृष्ट्वा चण्डालं स द्विजाग्रणीः}
{वेगादभ्येत्य कृपया स्ववस्त्रान्तैरवीजयत्} %१३

\twolineshloka
{अथोत्थितं तमेवासौ विष्णुदासो व्यलोकयत्}
{साक्षान्नारायणं देवं शङ्खचक्रगदाधरम्} %१४

\twolineshloka
{तं दृष्ट्वा सात्त्विकैर्भावैरावृतो द्विजसत्तमः}
{स्तोतुं चैव नमस्कर्तुं तदा नालं बभूव सः} %१५

\twolineshloka
{अथ शक्रादयो देवास्तत्रैवाभ्याययुस्तदा}
{गन्धर्वाप्सरसश्चापि जगुश्च ननृतुर्मुदा} %१६

\twolineshloka
{विमानशतसङ्कीर्णं देवर्षिशतसङ्कुलम्}
{गीतवादित्रनिर्घोषं स्थानं तदभवत्तदा} %१७

\twolineshloka
{ततो विष्णुः समालिङ्ग्य स्वभक्तं सात्त्विकव्रतम्}
{सारूप्यमात्मनो दत्त्वाऽनयद्वैकुण्ठमन्दिरम्} %१८

\twolineshloka
{विमानवरसंस्थं तं गच्छन्तं विष्णुसन्निधिम्}
{दीक्षितश्चोलनृपतिर्विष्णुदासं ददर्श सः} %१९

\twolineshloka
{वैकुण्ठभुवनं यान्तं विष्णुदासं विलोक्य सः}
{स्वगुरुं मुद्गलं वेगादाहूयेत्थं वचोऽब्रवीत्} %२०


\uvacha{चोल उवाच}
\twolineshloka
{यत्स्पर्द्धया मया चैव यज्ञदानादिकं कृतम्}
{स विष्णुरूपधृग्विप्रो याति वैकुण्ठमन्दिरम्} %२१

\twolineshloka
{दीक्षितेन मया सम्यक्सत्रेऽस्मिन्वैष्णवे त्वया}
{हुतमग्नौ कृता विप्रा दानाद्यैः पूर्णमानसाः} %२२

\twolineshloka
{नैवाद्यापि स मे देवः प्रसन्नो जायते ध्रुवम्}
{विष्णुदासस्य भक्त्यैव साक्षात्कारं ददौ हरिः} %२३

\twolineshloka
{तस्माद्दानैश्च यज्ञैश्च नैव विष्णुः प्रसीदति}
{भक्तिरेव परं तस्य निदानं दर्शने विभोः} %२४


\uvacha{गणावूचतुः}
\twolineshloka
{इत्युक्त्वा भागिनेयं स्वमभ्यषिञ्चन्नृपासने}
{आबाल्याद्दीक्षितो यज्ञे ह्यपुत्रत्वमगाद्यतः} %२५


\twolineshloka
{तस्मादद्यापि तद्देशे सदा राज्यांशभागिनः}
{स्वस्रेया एव जायन्ते तत्कृतावधिवर्तिनः} %२६

\twolineshloka
{यज्ञवाटं ततोऽभ्येत्य यज्ञकुण्डाग्रतः स्थितः}
{त्रिरुच्चैर्व्याजहाराऽऽशु विष्णुं सम्बोधयंस्तदा} %२७

\twolineshloka
{विष्णो भक्तिं स्थिरां देहि मनोवाक्काय कर्मभिः}
{इत्युक्त्वा सोऽपतद्वह्नौ सर्वेषामेव पश्यताम्} %२८

\twolineshloka
{मुद्गलस्तु तदा क्रोधाच्छिखामुत्पाटयत्स्वकाम्}
{ततस्त्वद्यापि तद्गोत्रे मुद्गला विशिखा बभुः} %२९

\twolineshloka
{तावदाविरभूद्विष्णुः कुण्डाग्नौ भक्तवत्सलः}
{तमालिङ्ग्य विमानाग्र्यं समारोहयदच्युतः} %३०

\twolineshloka
{तमालिङ्ग्याऽऽत्मसारूप्यं दत्त्वा वैकुण्ठमन्दिरम्}
{तेनैव सह देवेशो जगाम त्रिदशैर्वृतः} %३१


\uvacha{नारद उवाच}
\fourlineindentedshloka
{यो विष्णुदासः स तु पुण्यशीलो}
{यश्चोलभूपः स सुशीलनामा}
{एतावुभौ तत्समरूपभाजौ}
{द्वाःस्थौ कृतौ तेन रमाप्रियेण} %३२


\iti{चोलविष्णुदासमुक्तिकथनं}{नाम सप्तविंशोऽध्यायः}{२७}

\sect{अथ नामाऽष्टाविंशोऽध्यायः}


\uvacha{धर्मदत्त उवाच}
\twolineshloka
{जयश्च विजयश्चैव विष्णोर्द्वास्थौ श्रुतौ मया}
{किं नु ताभ्यां पुरा चीर्णं तस्मात्तद्रूपधारिणौ} %१


\uvacha{गणावूचतुः}
\twolineshloka
{तृणबिन्दोस्तु कन्यायां देवहूत्यां पुरा द्विज}
{कर्दमस्य तु दृष्ट्यैव पुत्रौ द्वौ सम्बभूवतुः} %२


\twolineshloka
{ज्येष्ठो जयः कनिष्ठोऽभूद्विजयश्चैव नामतः}
{तस्यामेवाऽभवत्पश्चात्कपिलो योगधर्मवित्} %३

\twolineshloka
{जयश्च विजयश्चैव विष्णुभक्तिरतौ सदा}
{तौ तन्निष्ठेन्द्रियग्रामौ धर्मशीलौ बभूवतुः} %४

\twolineshloka
{नित्यमष्टाक्षरीजाप्यौ विष्णुव्रतकरावुभौ}
{साक्षात्कारं ददौ विष्णुस्तयोर्नित्यार्चने सदा} %५

\twolineshloka
{मरुत्तेन कदाचित्तावाहूतौ यज्ञकर्मणि}
{जग्मतुर्यज्ञकुशलौ देवर्षिगणपूजितौ} %६

\twolineshloka
{जयस्तत्राभवद्ब्रह्मायाजको विजयोऽभवत्}
{ततो यज्ञविधिं कृत्स्नं परिपूर्णं च चक्रतुः} %७

\twolineshloka
{मरुत्तोऽवभृथस्नातस्ताभ्यां वित्तं ददौ बहु}
{तत्समादाय तौ वित्तं जग्मतुः स्वाश्रमं प्रति} %८

\twolineshloka
{यजनाय पृथग्विष्णोस्तुष्ट्यर्थं तौ ततो मुनी}
{तद्धनं विभजन्तौ तु पस्पर्धाते परस्परम्} %९

\twolineshloka
{जयोऽब्रवीत्समो भागः क्रियतामिति तत्र सः}
{विजयश्चाब्रवीन्नैतद्यल्लब्धं येन तस्य तत्} %१०

\twolineshloka
{ततोऽशपज्जयः क्रोधाद्विजयं लुब्धमानसम्}
{गृहीत्वा न ददास्येतत्तस्माद्ग्राहो भवेति तम्} %११

\twolineshloka
{विजयस्तस्य तं शापं श्रुत्वा सोप्यशपच्च तम्}
{मदभ्रान्तोऽशपस्त्वं मां तस्मान्मातङ्गतां व्रज} %१२

\twolineshloka
{तत्तदाचख्यतुर्विष्णुं दृष्ट्वा नित्यार्चने विभुम्}
{शापयोश्च निवृत्तिं तौ ययाचाते रमापतिम्} %१३


\uvacha{जयविजयावूचतुः}
\twolineshloka
{भक्तावावां कथं देव ग्राहमातङ्गयोनिगौ}
{भविष्यावः कृपासिन्धो तच्छापो विनिवर्त्यताम्} %१४


\uvacha{श्रीभगवानुवाच}
\twolineshloka
{मद्भक्तयोर्वचोऽसत्यं न कदाचिद्भविष्यति}
{मयाऽपि नान्यथा कर्तुं शक्यते तत्कदाचन} %१५

\twolineshloka
{प्रह्लादवचसा स्तम्भेऽप्याविर्भूतो ह्यहं पुरा}
{तथाऽम्बरीषवाक्येन जातो गर्भे स्वयं किल} %१६

\twolineshloka
{तस्माद्युवामिमौ शापावनुभूय स्वयङ्कृतौ}
{लभेथां मत्पदं नित्यमित्युक्त्वाऽन्तर्दधे हरिः} %१७

\uvacha{गणावूचतुः}
\twolineshloka
{ततस्तौ ग्राहमातङ्गावभूतां गण्डकीतटे}
{जातिस्मरौ तु तद्योन्यामपि विष्णुव्रते स्थितौ} %१८


\twolineshloka
{कदाचित्स गजः स्नातुं कार्तिके गण्डकीं गतः}
{तावज्जग्राह तं ग्राहः संस्मरञ्च्छापकारणम्} %१९

\twolineshloka
{ग्राहग्रस्तो ह्यसौ नागः सस्मार श्रीपतिं तदा}
{तावदाविरभूद्विष्णुश्चक्रशङ्खगदाधरः} %२०

\twolineshloka
{ततस्तौ ग्राहमातङ्गौ चक्रं क्षिप्त्वा समुद्धृतौ}
{दत्त्वैव निजसारूप्यं वैकुण्ठमनयद्विभुः} %२१

\twolineshloka
{ततःप्रभृति तत्स्थानं हरिक्षेत्रमितिस्मृतम्}
{चक्रसङ्घर्षणाद्यस्मिन्ग्रावाणोऽपि हि लाञ्छिताः} %२२

\twolineshloka
{तावुभौ विश्रुतौ लोके जयश्च विजयस्तथा}
{नित्यं विष्णुप्रियौ द्वाःस्थौ पृष्टौ यौ हि त्वया द्विज} %२३

\twolineshloka
{अतस्त्वमपि धर्मज्ञ नित्यं विष्णुव्रते स्थितः}
{त्यक्तमात्सर्यदम्भोऽपि भवस्व समदर्शनः} %२४

\twolineshloka
{तुलामकरमेषेषु प्रातःस्नायी सदा भव}
{एकादशीव्रते तिष्ठ तुलसीवनपालकः} %२५

\twolineshloka
{ब्राह्मणानथ गाश्चापि वैष्णवांश्च सदा भज}
{मसूरिकामारनालं वृन्ताकान्यपि खाद मा} %२६

\twolineshloka
{एवं त्वमपि देहान्ते तद्विष्णोः परमं पदम्}
{प्राप्नोषि धर्मदत्त त्वं तद्भक्त्यैव यथा वयम्} %२७

\twolineshloka
{तावज्जन्म व्रतादस्माद्विष्णुसन्तुष्टिकारकात्}
{न यज्ञा न च दानानि न तीर्थान्यधिकानि वै} %२८

\twolineshloka
{धन्योऽसि विप्राग्र्य यतस्त्वयैतद्व्रतं कृतं तुष्टिकरं जगद्गुरोः}
{यदर्धभागाप्तफला मुरारेः प्रणीयतेऽस्माभिरियं सलोकताम्} %२९


\uvacha{नारद उवाच}
\twolineshloka
{इत्थं तौ धर्मदत्तं तमुपदिश्य विमानगौ}
{तया कलहया सार्द्धं वैकुण्ठभवनं गतौ} %३०

\twolineshloka
{धर्मदत्तो ह्यसौ जातप्रत्ययस्तद्व्रते स्थितः}
{देहान्ते तद्विभोः स्थानं भार्याभ्यां संयुतोऽभ्ययात्} %३१

\twolineshloka
{इतिहासमिमं पुराभवं शृणुते श्रावयते च यः पुमान्}
{हरिसन्निधिकारणीं मतिं लभतेऽसौ कृपया जगद्गुरोः} %३२


\iti{धर्मदत्तमोक्षप्राप्तिकथनं}{नामाष्टाविंशोऽध्यायः}{२८}

\sect{अथ नामैकोनत्रिंशोऽध्यायः}


\uvacha{श्रीकृष्ण उवाच}
\twolineshloka
{इति तद्वचनं श्रुत्वा पृथुर्विस्मितमानसः}
{सम्पूज्य नारदं सम्यग्विससर्ज तदा प्रिये} %१

\twolineshloka
{पुराऽवन्तीपुरे कश्चिद्विप्र आसीद्धनेश्वरः}
{ब्रह्मकर्मपरिभ्रष्टः पापकर्मा सुदुर्मतिः} %२

\twolineshloka
{देशाद्देशातरं गच्छन्क्रयविक्रयकारणात्}
{माहिष्मतीपुरीमागात्कदाचित्स धनेश्वरः} %३

\twolineshloka
{महिषेण कृता पूर्वं तस्मान्माहिष्मतीति सा}
{यस्या वप्रगता भाति नर्मदा पापनाशिनी} %४

\twolineshloka
{कार्तिकव्रतिनस्तत्र नानादेशागतान्नरान्}
{स दृष्ट्वा विक्रयं कुर्वन्मासमेकमुवास सः} %५

\twolineshloka
{स नित्यं नर्मदातीरे भ्रमन्विक्रयकारणात्}
{ददर्श ब्राह्मणान्स्नानजपदेवार्चने स्थितान्} %६

\twolineshloka
{कांश्चित्पुराणं पठतः कांश्चिच्च श्रवणे रतान्}
{नृत्यगायनवादित्रविष्णुश्रवणतत्परान्} %७

\twolineshloka
{उद्यापनविधौ सक्तान्कांश्चिज्जागरणे रतान्}
{विप्रगोपूजनरतान्दीपदानरतांस्तथा} %८

\twolineshloka
{ददर्श कौतुकाविष्टस्तत्रतत्र धनेश्वरः}
{नित्यं परिभ्रमंस्तत्र दर्शनस्पर्शभाषणात्} %९

\twolineshloka
{वैष्णवानां तथा विष्णोर्नामश्रावादि सोऽलभत्}
{एवं मासं स्थितस्तस्या नर्मदायास्तटे द्विजः} %१०

\twolineshloka
{तावत्कृष्णाहिना दष्टो विह्वलः स पपात ह}
{अथ देहपरित्यक्तं तं वद्ध्वा यमकिङ्कराः} %११

\twolineshloka
{यमाज्ञया कुम्भिपाके चिक्षिपुस्तं धनेश्वरम्}
{यावत्क्षिप्तश्च तत्रासौ तावच्छीतलतां ययौ} %१२

\twolineshloka
{कुम्भीपाको यथा वह्निः प्रह्लादक्षेपणात्पुरा}
{यमस्तु कौतुकं दृष्ट्वा पप्रच्छानीय तं ततः} %१३


\onelineshloka*
{तावदभ्यागतस्तत्र नारदः प्राह सत्वरम्}

\uvacha{नारद उवाच}

\onelineshloka
{नैवायं निरयान्भोक्तुमर्हो ह्यरुणनन्दन} %१४

\twolineshloka
{यस्मादन्तेऽस्य सञ्जातं कर्म यन्निरयापहम्}
{यः पुण्यकर्मिणां कुर्याद्दर्शनस्पर्शभाषणम्} %१५

\twolineshloka
{ततः षडंशमाप्नोति पुण्यस्य नियतं नरः}
{सख्यं तु तैस्तु संसर्गं कृतवान्वै धनेश्वरः} %१६


\onelineshloka
{कार्तिकव्रतिभिर्मासं तेषां पुण्यांशभागयम्} %१७

\twolineshloka
{तस्मादकामपुण्यो हि यक्षयोनिस्थितो ह्ययम्}
{विलोक्य निरयान्सर्वान्पापभोगप्रदर्शकान्} %१८


\uvacha{श्रीकृष्ण उवाच}
\fourlineindentedshloka
{इत्युक्त्वा गतवति नारदे स सौरि-}
{स्तद्वाक्यश्रवणविबुद्धतत्सुकर्मा}
{तं विप्रं पुनरनयत्स्वकिङ्करेण}
{तान्सर्वान्निरयगणान्प्रदर्शयिष्यन्} %१९


\twolineshloka
{ततो धनेश्वरं नीत्वा निरयान्प्रेतपोऽब्रवीत्}
{दर्शयिष्यंस्तु तान्सर्वान्यमानुज्ञाकरस्तदा} %२०


\uvacha{प्रेतप उवाच}
\twolineshloka
{पश्येमान्निरयान्घोरान्धनेश्वर महाभयान्}
{एषु पापकरा नित्यं पच्यन्ते यमकिङ्करैः} %२१

\twolineshloka
{अकामात्पातकं शुष्कं कामादार्द्रमुदाहृतम्}
{आर्द्रशुष्कादिभिः पापैर्द्विप्रकारानवस्थितान्} %२२

\twolineshloka
{चतुराशीतिसङ्ख्याकैः पृथग्भेदैरवस्थितान्}
{यत्प्रकीर्णमपाङ्क्तेयं मलिनीकरणं तथा} %२३

\twolineshloka
{जातिभ्रंशकरं तद्वदुपपातकसंज्ञकम्}
{अतिपापं महापापं सप्तधा पातकं स्मृतम्} %२४

\twolineshloka
{एभिः सप्तसु पच्यन्ते निरयेषु यथाक्रमम्}
{कार्तिकव्रतिभिर्यस्मात्संसर्गो ह्यभवत्तव} %२५


\onelineshloka*
{तत्पुण्योपचयादेते निर्हृता निरयाः खलु}

\uvacha{श्रीकृष्ण उवाच}

\onelineshloka
{दर्शयित्वेति निरयान्प्रेतपस्तमथाहरत्} %२६

\twolineshloka
{धनेश्वरं यक्षलोकं यक्षश्चाभूत्स तत्र हि}
{धनदस्यानुगः सोऽयं धनयक्षेति विश्रुतः} %२७


\uvacha{सूत उवाच}
\twolineshloka
{इत्युक्त्वा वासुदेवोऽसौ सत्यभामामतिप्रियम्}
{सायंसन्ध्याविधिं कर्तुं जगाम जननीगृहम्} %२८


\uvacha{ब्रह्मोवाच}
\twolineshloka
{एवम्प्रभावः खलु कार्तिकोऽयं मुक्तिप्रदो भुक्तिकरश्च यस्मात्}
{प्रयान्त्यनेकाऽर्जितपातकानि व्रतस्य सन्दर्शयतोऽपि मुक्तिम्} %२९


\iti{धनेश्वरयक्षजन्मप्राप्तिवर्णनं}{नामैकोनत्रिंशोऽध्यायः}{२९}

\sect{अथ त्रिंशोऽध्यायः}


\uvacha{नारद उवाच}
\twolineshloka
{अद्भुतोऽयं त्वया प्रोक्तो महिमा कार्तिकस्य तु}
{स्वस्य कर्तुमसामर्थ्यं कथमेतत्कृतं भवेत्} %१


\uvacha{ब्रह्मोवाच}
\twolineshloka
{नास्ति कर्तुं स्वसामर्थ्यमुपायात्प्राप्यते फलम्}
{द्रव्यं दत्त्वा ब्राह्मणाय गृह्णीयात्फलमुत्तमम्} %२

\twolineshloka
{शिष्याद्वा भृत्यवर्गाद्वा स्त्रीभ्यो वाऽऽप्ताच्च कारयेत्}
{तस्मादपि फलं गृह्णन्फलभाग्जायते नरः} %३


\uvacha{नारद उवाच}
\twolineshloka
{अदत्तान्यपि पुण्यानि प्राप्यन्ते केनचित्क्वचित्}
{एतदिच्छाम्यहं श्रोतुं कौतुकं मम वर्तते} %४


\uvacha{ब्रह्मोवाच}
\twolineshloka
{अदत्तान्यपि पुण्यानि लभन्ते पातकान्यपि}
{येनोपायेन तद्वच्मि शृणुष्वैकमना द्विज} %५

\twolineshloka
{सुकृतं वा दुष्कृतं वा कृतमेकेन यत्कृते}
{जायते तस्य तद्राष्ट्रे त्रेतायां तु पुरो भवेत्} %६

\twolineshloka
{द्वापरे वंशमध्ये तु कलौ कर्तैव केवलम्}
{आज्ञानाद्यत्कृतं कर्म बाल्ये स्वप्ने तु तत्फलम्} %७

\twolineshloka
{अज्ञानाद्यच्च तारुण्ये बाल्ये तस्य फलं भवेत्}
{ज्ञानपूर्वं कृतं कर्म आजन्मान्तं च तत्फलम्} %८

\twolineshloka
{षण्मासं पापिसङ्गेन नरः पापी प्रजायते}
{पापिनां वा धर्मिणां वा संसर्गाद्दशमासिकम्} %९

\twolineshloka
{भोजनादेकपङ्क्तौ च विंशांशः पुण्यपापयोः}
{एकासने द्वयोर्वासात्सहस्रांशेन लिप्यते} %१०

\twolineshloka
{यो वै यस्यान्नमश्नाति स भुङ्क्ते तस्य किल्बिषम्}
{जपादौ पापिसंसर्गात्षोडशांशो विनश्यति} %११

\twolineshloka
{परस्य स्तवनाद्यानादेकपात्रस्थभोजनात्}
{एकशय्याप्रावरणात्षष्ठांशः पुण्यपापयोः} %१२

\twolineshloka
{पुरुषो हरते सर्वं भार्याया औरसस्य च}
{अर्द्धं शिष्याच्चतुर्थांशं पापं पुण्यं तथैव च} %१३

\twolineshloka
{भर्तुराज्ञाकरी नारी भर्तुरर्द्धं वृषं हरेत्}
{यद्धस्तपक्वं भुजीयाद्दशांशं तदघं हरेत्} %१४

\twolineshloka
{वर्षाऽशनं तु यो दत्ते तदर्द्धाघस्य भागयम्}
{वर्षाशनार्द्धं पुण्यं तु भुङ्क्ते वर्षाशनी नरः} %१५

\twolineshloka
{पुरोहितस्य षष्ठांशं पापं वा पुण्यमेव वा}
{यजमानो भुनक्त्येव तद्दशांशं पुरोहितः} %१६

\twolineshloka
{उद्योगी चानुमन्ता च यश्चोपकरणप्रदः}
{षष्ठांशं पुण्यपापानामुपद्रष्टा दशांशकम्} %१७

\twolineshloka
{यद्धस्तात्कार्यते कर्म नान्नमस्मै प्रयच्छति}
{विना भृतकशिष्याभ्यां षष्ठांशं पुण्यमाहरेत्} %१८

\twolineshloka
{व्यवहारात्तथा प्रीत्या नित्यं सम्भाषणादिभिः}
{दशांशं पुण्यपापानां लभते नात्र संशयः} %१९

\twolineshloka
{संसर्गपुण्ययोगेन एकदन्तो द्विजाधमः}
{नरकान्विविधान्दृष्ट्वा स्वर्गं प्राप तदैव हि} %२०


\uvacha{नारद उवाच}
\twolineshloka
{ईदृशं कार्तिकव्रतमल्पायासं महत्फलम्}
{न कुर्वन्ति जनाः केचित्किमर्थं वै पितामह} %२१


\uvacha{ब्रह्मोवाच}
\twolineshloka
{स्वसृष्टिवृद्धये वेधा धर्माधर्मौ ससर्ज ह}
{धर्ममेवानुतिष्ठन्तः प्राप्नुवन्ति शुभां गतिम्} %२२

\twolineshloka
{अधर्ममनुतिष्ठन्तो यान्ति तेऽधोगतिं नराः}
{पुण्यकर्मफलं नाको नरकस्तद्विपर्ययः} %२३

\twolineshloka
{तयोः पालनकर्तारौ द्वावेव विधिना कृतौ}
{शतक्रतुयमौ तौ च पुण्यपापानुसारिणौ} %२४

\twolineshloka
{गुरुतल्पादयः पुत्राः कामस्य प्रथिता भुवि}
{क्रोधस्य पितृघाताद्या लोभस्य तनयाञ्छृणु} %२५

\twolineshloka
{ब्रह्मस्वहरणाद्याश्च एते नरकनायकाः}
{कृता यमेन तैर्व्याप्ता मनुजा न हि कुर्वते} %२६


\onelineshloka
{व्रतादिधर्मकृत्यं यैस्तैर्मुक्तास्ते हि कुर्वते} %२७

\twolineshloka
{श्रद्धा मेधा विघातिन्यौ वर्तेते भुवि सर्वदा}
{ताभ्यां व्याप्तस्तु मनुजः श्रीविष्णोः श्रवणादिकम्} %२८

\twolineshloka
{न करोति सुदुर्मेधा येनान्धं याति वै तमः}
{कृष्णेन सत्यभामायै यदुक्तं तद्वदामि ते} %२९

\twolineshloka
{अध्यापनाद्याजनाद्वाऽप्येकपङ्क्त्यशनादपि}
{तुर्यांश पुण्यपापानां परोक्षं लभते नरः} %३०

\twolineshloka
{एकासनादेकयानान्निश्वासस्याङ्गसङ्गतः}
{षडंशं फलभागी स्यान्नियतं पुण्यपापयोः} %३१

\twolineshloka
{स्पर्शनाद्भाषणाद्वाऽपि परस्य स्तवनादपि}
{दशांशं पुण्यपापानां नित्यं प्राप्नोति मानवः} %३२

\twolineshloka
{दर्शनश्रवणाभ्यां च मनोध्यानात्तथैव च}
{परस्य पुण्यपापानां शतांशं प्राप्नुयान्नरः} %३३

\twolineshloka
{परस्य निन्दां पैशुन्यं धिक्कारं च करोति यः}
{तत्कृतं पातकं प्राप्य स्वपुण्यं प्रददाति सः} %३४

\twolineshloka
{कुर्वतः पुण्यकर्माणि सेवां यः कुरुते नरः}
{पत्नीभृतकशिष्येभ्यो यदन्यः कोऽपि मानवः} %३५

\twolineshloka
{तस्य सेवानुरूपं च द्रव्यं किञ्चिन्न दीयते}
{सोऽपि सेवानुरूपेण तत्पुण्यफलभाग्भवेत्} %३६

\twolineshloka
{एकपङ्क्तिस्थितं यस्तु लङ्घयेत्परिवेषणम्}
{तत्पुण्यस्य षडंशं च लभेद्यस्तु विलङ्घितः} %३७

\twolineshloka
{स्नानसन्ध्यादिकं कुर्वन्यः स्पृशेद्वाऽथ भाषते}
{स कर्मपुण्यषष्ठांशं दद्यात्तस्मै विनिश्चितम्} %३८

\twolineshloka
{धर्मोद्देशेन यो द्रव्यमपरं याचते नरः}
{तत्पुण्यकर्मजं तस्य धनदस्त्वाप्नुयात्फलम्} %३९

\twolineshloka
{अपहृत्य परद्रव्यं पुण्यकर्म करोति यः}
{कर्मकृत्पापभाक्तत्र धनिनस्तद्भवं फलम्} %४०

\twolineshloka
{नापकृत्य ऋणं यस्तु परस्य म्रियते नरः}
{धनी तत्पुण्यमादत्ते तद्धनस्यानुरूपतः} %४१

\twolineshloka
{बुद्धिदाताऽनुमन्ता च यश्चोपकरणप्रदः}
{बलकृच्चापि षष्ठांशं प्राप्नुयात्पुण्यपापयोः} %४२

\twolineshloka
{प्रजाभ्यः पुण्यपापानां राजा षष्ठांशमुद्धरेत्}
{शिष्याद्गुरुः स्त्रियो भर्ता पिता पुत्रात्तथैव च} %४३

\twolineshloka
{स्वपतेरपि पुण्यस्य योषिदर्धमवाप्नुयात्}
{चित्तस्यानुव्रता शश्वद्वर्तते तुष्टिकारिणी} %४४

\twolineshloka
{परहस्तेन दानादि कुर्वतः पुण्यकर्मणः}
{विना भृतकपुत्राभ्यां कर्ता षष्ठांशमुद्धरेत्} %४५

\twolineshloka
{वृत्तिदो वृत्तिसम्भोक्तुः पुण्यं षष्ठांशमुद्धरेत्}
{आत्मनो वा परस्यापि यदि सेवां न कारयेत्} %४६

\twolineshloka
{इत्थं ह्यदत्तान्यपि पुण्यपापान्यायान्ति नित्यं परसञ्चितानि}
{कलौ त्वयं वै नियमो न कार्यः कर्तैव भोक्ता खलु पुण्यपापयोः} %४७

\twolineshloka
{कलौ ज्ञानं दृढं नास्ति कलौ गर्वेण सत्क्रिया}
{कलौ दम्भान्वितो योगो नश्यत्येव न संशयः} %४८

\twolineshloka
{तपोनिष्ठः पुरा दम्भी सती शुद्धप्रभावतः}
{पित्रोः पूजादर्शनेन चोर्जसेवी परं गतः} %४९


\uvacha{नारद उवाच}
\twolineshloka
{भगवञ्छ्रोतुमिच्छामि व्रतानामुत्तमं व्रतम्}
{विधिं मासोपवासस्य फलं चास्य यथोचितम्} %५०


\uvacha{ब्रह्मोवाच}
\twolineshloka
{साधु नारद सर्वं ते यत्पृष्टं प्रब्रुवेऽनघ}
{भक्त्या मतिमतां श्रेष्ठ शृणुष्व गदतो मम} %५१

\twolineshloka
{सुराणां च यथा विष्णुस्तपतां च यथा रविः}
{मेरुः शिखरिणां यद्वद्वैनतेयश्च पक्षिणाम्} %५२

\twolineshloka
{श्रेष्ठं सर्वव्रतानां तु तद्वन्मासोपवासनम्}
{सर्वव्रतेषु यत्पुण्यं सर्वतीर्थेषु चैव हि} %५३

\twolineshloka
{सर्वदानोद्भवं चैव यज्ञैश्च भूरिदक्षिणैः}
{न तत्पुण्यमवाप्नोति यन्मासपरिलङ्घनात्} %५४

\twolineshloka
{गुरोराज्ञां ततो लब्ध्वा कुर्यान्मासोपवासनम्}
{अतिकृच्छ्रं च पाराकं कृत्वा चान्द्रायणं ततः} %५५

\twolineshloka
{मासोपवासं कुर्वीत ज्ञात्वा देहबलाबलम्}
{वानप्रस्थो यतिर्वापि नारी वा विधवा मुने} %५६

\twolineshloka
{मासोपवासं कुर्वीत गुरोर्विप्राज्ञया ततः}
{आश्विनस्यामले पक्षे एकादश्यामुपोषितः} %५७

\twolineshloka
{व्रतमेतत्तु गृह्णीयाद्यावत्त्रिंशद्दिनानि तु}
{अच्युतस्याऽऽलये भक्त्या त्रिकालं पूजयेद्धरिम्} %५८

\twolineshloka
{नैवेद्यधूपदीपाद्यैः पुष्पैर्नानाविधैरपि}
{मनसा कर्मणा वाचा पूजयेद्गरुडध्वजम्} %५९

\twolineshloka
{नरः स्वधर्मनिरतः सधवा च जितेन्द्रिया}
{नारी वा विधवा साध्वी वासुदेवं समर्चयेत्} %६०

\twolineshloka
{वस्त्वालोकनगन्धादिस्वादितं परिकीर्तितम्}
{अन्यस्य वर्जयेद्वासं ग्रासानां सम्प्रमोक्षणम्} %६१

\twolineshloka
{गात्राभ्यङ्गं शिरोभ्यङ्गं ताम्बूलं सविलेपनम्}
{व्रतस्थो वर्जयेत्सर्वं यच्चान्यच्च निराकृतम्} %६२

\twolineshloka
{न व्रतस्थः स्पृशेत्कञ्चिद्विकर्मस्थं न चालपेत्}
{देवतायतने तिष्ठन्गृहस्थश्चाऽऽचरेद्व्रतम्} %६३

\twolineshloka
{कृत्वा मासोपवासं तु यथोक्तविधिना नरः}
{अन्यूनाधिकमेवं तु व्रतं त्रिंशद्दिनैरिति} %६४

\twolineshloka
{ततोऽर्चयेदेव पुण्यं द्वादश्यां गरुडध्वजम्}
{वस्त्रदानादिभिश्चैव भोजयित्वा द्विजोत्तमान्} %६५

\twolineshloka
{दद्याच्च दक्षिणां तेभ्यः प्रणिपत्य क्षमापयेत्}
{विप्रान्क्षमापयित्वा तुविसृज्याभ्यर्च्य पूज्य च} %६६

\twolineshloka
{एवं मासोपवासान्ते वृत्वा विप्रांस्त्रयोदश}
{कारयेद्वैष्णवं यज्ञमेकादश्यामुपोषितः} %६७

\twolineshloka
{ततोऽनुभोजयेद्विप्रान्नमस्कारपुरःसरम्}
{ताम्बूलवस्त्रयुग्मानि भोजनाच्छादनानि च} %६८

\twolineshloka
{योगपट्टानि सूत्राणि शय्यां सोपस्करां तथा}
{दत्त्वा चैव द्विजाग्र्येभ्यः पूजयित्वा विसर्जयेत्} %६९

\twolineshloka
{विधिर्मासोपवासस्य यथावत्परिकीर्तितः}
{अतःपरं प्रवक्ष्यामि नवम्यादितिथौ विधि} %७०


\onelineshloka
{ऋषिभ्यो वालखिल्यैश्च प्रोक्तं तं शृणु नारद} %७१


\iti{दत्तपुण्यपापफलप्राप्तिवर्णनपूर्वक मासोपवासव्रतविधिकथनं}{नाम त्रिंशोऽध्यायः}{३०}

\sect{अथ नामैकत्रिंशोऽध्यायः}


\uvacha{वालखिल्या ऊचुः}
\twolineshloka
{कार्तिके शुक्लनवमी तत्राऽभूद्द्वापरं युगम्}
{पूर्वापराह्णगा ग्राह्या क्रमाद्दानोपवासयोः} %१

\twolineshloka
{अत्र कूष्माण्डकोनाम हतो दैत्यस्तु विष्णुना}
{तद्रोमभिः समुद्भूता वल्ल्यः कूष्माण्डसम्भवाः} %२

\twolineshloka
{तस्मात्कूष्माण्डदानेन फलमाप्नोति निश्चितम्}
{अस्यामेव नवम्यां तु कुर्यात्कृष्णोत्सवं नरः} %३

\twolineshloka
{स्वशाखोक्तेन विधिना तुलस्याः करपीडनम्}
{कन्यादानफलं तस्य जायते नात्र संशयः} %४

\twolineshloka
{कार्तिके शुक्लनवमीमवाप्य विजितेन्द्रियः}
{हरिं विधाय सौवर्णं तुलस्या सहितं शुभम्} %५

\twolineshloka
{पूजयेद्विधिवद्भक्त्या व्रती तत्र दिनत्रयम्}
{एवं यथोक्तविधिना कुर्याद्वैवाहिकं विधिम्} %६

\twolineshloka
{ग्राह्यं त्रिरात्रमत्रैव नवम्याद्यनुरोधतः}
{मध्याह्नव्यापिनी ग्राह्या नवमी पूर्ववेधिता} %७

\twolineshloka
{धात्र्यश्वत्थौ य एकत्र पालयित्वा समुद्वहेत्}
{न नश्यते तस्य पुण्यं कल्पकोटिशतैरपि} %८

\twolineshloka
{कनकस्य सुता पूर्वमेकादश्यां किशोरिका}
{चकार भक्तितः सायं तुलस्युद्वाहजं विधिम्} %९

\twolineshloka
{तेन वैधव्यदोषेण निर्मुक्ताऽऽसीत्सुलोचना}
{तस्मात्सायं प्रकर्तव्यस्तुलस्तुद्वाहजो विधिः} %१०

\twolineshloka
{अवश्यमेव कर्तव्यः प्रतिवर्षं तु वैष्णवैः}
{विधिं तस्य प्रवक्ष्यामि यथा साङ्गा क्रिया भवेत} %११

\twolineshloka
{विष्णोस्तु प्रतिमां कुर्यात्पलस्य स्वर्णजां शुभाम्}
{तदर्द्धार्द्धं तदर्द्धार्द्धं यथाशक्त्या प्रकल्पयेत्} %१२

\twolineshloka
{प्राणप्रतिष्ठां कृत्वैव तुलसीविष्णुरूपयोः}
{तत उत्थापयेद्देवं पूर्वोक्तैश्च स्तवादिभिः} %१३

\twolineshloka
{उपचारैः षोडशभिः पूजयेत्पुरुषोक्तिभिः}
{देशकालौ ततः स्मृत्वा गणेशं तत्र पूजयेत्} %१४

\twolineshloka
{पुण्याहं वाचयित्वाथ नान्दीश्राद्धं समाचरेत्}
{वेदवाद्यादिनिर्घोषैर्विष्णुमूर्तिं समानयेत्} %१५

\twolineshloka
{तुलसीनिकटे सा तु स्थाप्या चान्तर्हिता पटैः}
{आगच्छ भगवन्देव अर्चयिष्यामि केशव} %१६

\twolineshloka
{तुभ्यं दास्यामि तुलसीं सर्वकामप्रदो भव}
{दद्यात्त्रिवारमर्घ्यं च पाद्यं विष्टरमेव च} %१७

\twolineshloka
{तत आचमनीयं च त्रिरुक्त्वा च प्रदापयेत्}
{ततो दधि घृतं क्षीरं कांस्यपात्रपुटीकृतम्} %१८

\twolineshloka
{मधुपर्कं गृहाण त्वं वासुदेव नमोस्तु ते}
{हरिद्रालेपानाभ्यङ्गकार्यं सर्वं विधाय च} %१९

\twolineshloka
{गोधूलिसमये पूज्यौ तुलसीकेशवौ पुनः}
{पृथक्पृथक्तथा कार्यौ सम्मुखौ मङ्गलं पठेत्} %२०

\twolineshloka
{ईषद्दृश्ये भास्करे तु सङ्कल्पं तु समुच्चरेत्}
{स्वगोत्रप्रवरानुक्त्वा तथा त्रिपुरुषादिकम्} %२१

\twolineshloka
{अनादिमध्यनिधन त्रैलोक्यप्रतिपालक}
{इमां गृहाण तुलसीं विवाहविधिनेश्वर} %२२

\twolineshloka
{पार्वतीबीजसम्भूतां वृन्दाभस्मनि संस्थिताम्}
{अनादिमध्यनिधनां वल्लभां ते ददाम्यहम्} %२३

\twolineshloka
{पयोघटैश्च सेवाभिः कन्यावद्वर्धिता मया}
{त्वत्प्रियां तुलसीं तुभ्यं ददामि त्वं गृहाण भोः} %२४

\twolineshloka
{एवं दत्त्वा च तुलसीं पश्चात्तौ पूजयेत्ततः}
{रात्रौ जागरणं कुर्याद्विवाहोत्सवपूर्वकम्} %२५

\twolineshloka
{ततः प्रभातसमये तुलसीं विष्णुमर्चयेत्}
{वह्निसंस्थापनं कृत्वा द्वादशाक्षरविद्यया} %२६

\threelineshloka
{पायसाऽऽज्यक्षौद्रतिलैर्जुह्यादष्टोत्तरं शतम्}
{ततः स्विष्टकृतं हुत्वा दद्यात्पूर्णाहुतिं ततः}
{आचार्यं च समभ्यर्च्य होमशेषं समापयेत्} %२७

\twolineshloka
{चतुरो वार्षिकान्मासान्नियमो येन यः कृतः}
{कथयित्वा द्विजेभ्यस्तत्तथाऽन्यत्परिपूरयेत्} %२८

\twolineshloka
{इदं व्रतं मया देव कृतं प्रीत्यै तव प्रभो}
{न्यूनं सम्पूर्णतां यातु त्वत्प्रसादाज्जनार्दन} %२९

\twolineshloka
{रेवतीतुर्यचरणे द्वादशीसंयुते नरः}
{न कुर्यात्पारणं कुर्वन्व्रतं निष्फलतां नयेत्} %३०

\threelineshloka
{ततो येषां पदार्थानां वर्जनं तु कृतं भवेत्}
{चातुर्मास्येऽथवा चोर्जे ब्राह्मणेभ्यः समर्पयेत्}
{ततः सर्वं समश्नीयाद्यद्यत्त्यक्तं व्रते स्थितम्} %३१


\onelineshloka
{दम्पतिभ्यां सहैवात्र भोक्तव्यं च द्विजैः सह} %३२

\twolineshloka
{ततो भुक्त्युत्तरं यानि गलितानि दलानि च}
{तानि भुक्त्वा तुलस्याश्च स्वयं पापैः प्रमुच्यते} %३३

\twolineshloka
{इक्षुदण्डं तथा धात्रीफलं कोलिफलं तथा}
{भुक्त्वा तु भोजनस्यान्ते तस्योच्छिष्टं विनश्यति} %३४

\twolineshloka
{एषु त्रिषु न भुक्तं चेदेकैकमपि येन तु}
{ज्ञेय उच्छिष्ट आवर्षं नरोऽसौ नात्र संशयः} %३५

\twolineshloka
{ततः सायं पुनः पूज्याविक्षुदण्डैश्च शोभितैः}
{तुलसीवासुदेवौ च कृतकृत्यो भवेत्ततः} %३६

\threelineshloka
{ततो विसर्जनं कृत्वा दत्त्वा दायादिकं हरेः}
{वैकुण्ठं गच्छ भगवँस्तुलसीसहितः प्रभो}
{मत्कृतं पूजनं गृह्य सन्तुष्टो भव सर्वदा} %३७

\twolineshloka
{गच्छगच्छ सुरश्रेष्ठ स्वस्थाने परमेश्वर}
{यत्र ब्रह्मादयो देवास्तत्र गच्छ जनार्दन} %३८

\twolineshloka
{एवं विसृज्य देवेशमाचार्याय प्रदापयेत्}
{मूर्त्यादिकं सर्वमेव कृतकृत्यो भवेन्नरः} %३९

\threelineshloka
{प्रतिवर्षं तु यः कुर्यात्तुलसीकरपीडनम्}
{भक्तिमान्धनधान्यैः स युक्तो भवति निश्चितम्}
{इहलोके परत्रापि विपुलं च यशो लभेत्} %४०


\iti{कूष्माण्डनवमीतुलसीविवाहविधिवर्णनं}{नामैकत्रिंशोऽध्यायः}{३१}

\sect{अथ द्वात्रिंशोऽध्यायः}


\uvacha{वालखिल्या ऊचुः}
\twolineshloka
{कार्तिकस्यामले पक्षे स्नात्वा सम्यग्यतव्रतः}
{एकादश्यां तु गृह्णीयाद्व्रतं पञ्चदिनात्मकम्} %१

\threelineshloka
{शरपञ्जरसुप्तेन भीष्मेण तु महात्मना}
{राजधर्मा मोक्षधर्मा दानधर्मास्ततः परम्}
{कथिता पाण्डुदायादैः कृष्णेनापि श्रुतास्तदा} %२

\twolineshloka
{ततः प्रीतेन मनसा वासुदेवेन भाषितम्}
{धन्यधन्योऽसि भीष्म त्वं धर्माः संश्रावितास्त्वया} %३

\twolineshloka
{एकादश्यां कार्तिकस्य याचितं च जलं त्वया}
{अर्जुनेन समानीतं गाङ्गं बाणस्य वेगतः} %४

\twolineshloka
{तुष्टानि तव गात्राणि तस्मादद्यदिनावधि}
{पूर्णान्तं सर्वलोकास्त्वां तर्पयन्त्वर्घ्यदानतः} %५

\twolineshloka
{तस्मात्सर्वप्रयत्नेन मम सन्तुष्टिकारकम्}
{एतद्व्रतं प्रकुर्वन्तु भीष्मपञ्चकसंज्ञितम्} %६

\twolineshloka
{कार्तिकस्य व्रतं कृत्वा न कुर्याद्भीष्मपञ्चकम्}
{समग्रं कार्तिकव्रतं वृथा तस्य भविष्यति} %७

\twolineshloka
{अशक्तश्चेन्नरो भूयादसमर्थश्च कार्तिके}
{भीष्मस्य पञ्चकं कृत्वा कार्तिकस्य फलं लभेत्} %८

\twolineshloka
{सत्यव्रताय शुचये गाङ्गेयाय महात्मने}
{भीष्मायैतद्ददाम्यर्घ्यमाजन्मब्रह्मचारिणे} %९


\onelineshloka
{सव्येनानेन मन्त्रेण तर्पणं सार्ववर्णिकम्} %१०

\twolineshloka
{व्रताङ्गत्वात्पूर्णिमायां प्रदेयः पापपूरुषः}
{अपुत्रेण प्रकर्तव्यं सर्वथा भीष्मपञ्चकम्} %११

\twolineshloka
{यः पुत्रार्थं व्रतं कुर्यात्सस्त्रीको भीष्मपञ्चकम्}
{प्रदत्त्वा पापपुरुषं वर्षमध्ये सुतं लभेत्} %१२

\twolineshloka
{अवश्यमेव कर्तव्यं तस्माद्भीष्मस्य पञ्चकम्}
{विष्णुप्रीतिकरं प्रोक्तं मया भीष्मस्य पञ्चकम्} %१३


\uvacha{सूत उवाच}
\twolineshloka
{शृण्वन्तु ऋषयः सर्वे विशेषो भीष्मपञ्चके}
{कार्तिकेयाय रुद्रेण पुरा प्रोक्तः सविस्तरात्} %१४


\uvacha{ईश्वर उवाच}
\twolineshloka
{प्रवक्ष्यामि महापुण्यं व्रतं व्रतवतां वर}
{भीष्मेणैतद्यतः प्राप्तं व्रतं पञ्चदिनात्मकम्} %१५

\twolineshloka
{सकाशाद्वासुदेवस्य तेनोक्तं भीष्मपञ्चकम्}
{व्रतस्यास्य गुणान्वक्तुं कः शक्तः केशवादृते} %१६

\twolineshloka
{कार्तिके शुक्लपक्षे तु शृणु धर्मं पुरातनम्}
{वसिष्ठभृगुगर्गाद्यैश्चीर्णं कृतयुगादिषु} %१७

\twolineshloka
{अम्बरीषेण भोगाद्यैश्चीर्णं त्रेतायुगादिषु}
{ब्राह्मणैर्ब्रह्मचर्येण जपहोमक्रियादिभिः} %१८

\twolineshloka
{क्षत्रियैश्च तथा वैश्यैः सत्यशौचपरायणैः}
{दुष्करं सत्यहीनानामशक्यं बालचेतसाम्} %१९

\twolineshloka
{दुष्करं भीष्ममित्याहुर्न शक्यं प्राकृतैर्नरैः}
{यस्मात्करोति विप्रेन्द्र तेन सर्वं कृतं भवेत्} %२०

\twolineshloka
{व्रतं चैतन्महापुण्यं महापातकनाशनम्}
{अतो नरैः प्रयत्नेन कर्तव्यं भीष्मपञ्चकम्} %२१

\twolineshloka
{कार्तिकस्यामले पक्षे स्नात्वा सम्यग्विधानतः}
{एकादश्यां तु गृह्णीयाद्व्रतं पञ्चदिनात्मकम्} %२२

\twolineshloka
{प्रातः स्नात्वा विशेषेण मध्याह्ने च तथा व्रती}
{नद्यां निर्झरतोये वा समालभ्य च गोमयम्} %२३

\twolineshloka
{यवव्रीहितिलैः सम्यक्पितॄन्सन्तर्पयेत्क्रमात्}
{स्नात्वा मौनं नरः कृत्वा धौतवासा दृढव्रतः} %२४

\twolineshloka
{भीष्मायोदकदानं च अर्घ्यं चैव प्रयत्नतः}
{पूजा भीष्मस्य कर्तव्या दानं दद्यात्प्रयत्नतः} %२५

\twolineshloka
{पञ्चरत्नं विशेषेण दत्त्वा विप्राय यत्नतः}
{वासुदेवोऽपि सम्पूज्यो लक्ष्मीयुक्तः सदा प्रभुः} %२६


\onelineshloka
{पञ्चके पूजयित्वा तु कोटिजन्मानि तुष्यति} %२७

\twolineshloka
{यत्किञ्चिद्ददते मर्त्यः पञ्चधातुप्रकल्पितम्}
{संवत्सरव्रतानां स लभते सकलं फलम्} %२८

\twolineshloka
{कृत्वा तूदकदानं तु तथाऽर्घ्यस्य च दापनम्}
{मन्त्रेणानेन यः कुर्यान्मुक्तिभागी भवेन्नरः} %२९

\twolineshloka
{वैयाघ्रपादगोत्राय साङ्कृत्य प्रवराय च}
{अनपत्याय भीष्माय उदकं भीष्मवर्मणे} %३०

\twolineshloka
{वसूनामवताराय शन्तनोरात्मजाय च}
{अर्घ्यं ददामि भीष्माय आजन्मब्रह्मचारिणे} %३१


॥इत्यर्घ्यमन्त्रः॥
\twolineshloka
{अनेन विधिना यस्तु पञ्चकं तु समापयेत्}
{अश्वमेधसमं पुण्यं प्राप्नोत्यत्र न संशयः} %३२

\twolineshloka
{पञ्चाहमपि कर्तव्यं नियमं च प्रयत्नतः}
{नियमेन विना यत्र न भाव्यं वरवर्णिना} %३३

\twolineshloka
{उत्तरायणहीनाय भीष्माय प्रददौ हरिः}
{उत्तरायणहीनेऽपि शुद्धलग्नं सुतोषितः} %३४

\twolineshloka
{ततः सम्पूजयेद्देवं सर्वपापहरं हरिम्}
{अनन्तरं प्रयत्नेन कर्तव्यं भीष्मपञ्चकम्} %३५

\twolineshloka
{स्नापयेत जलैर्भक्त्या मधुक्षीरघृतेन च}
{तथैव पञ्चगव्येन गन्धचन्दनवारिणा} %३६

\twolineshloka
{चन्दनेन सुगन्धेन कुमकुमेनाथ केशवम्}
{कर्पूरोशीरमिश्रेण लेपयेद्गरुडध्वजम्} %३७

\twolineshloka
{अर्चयेद्रुचिरैः पुष्पैर्गन्धधूपसमन्वितैः}
{गुग्गुलुं घृतसंयुक्तं ददेत्कृष्णाय भक्तिमान्} %३८

\twolineshloka
{दीपकं तु दिवा रात्रौ दद्यात्पञ्च दिनानि तु}
{नैवेद्यं देवदेवस्य परमान्नं निवेदयेत्} %३९

\twolineshloka
{एवमभ्यर्चयेद्देवं संस्मृत्य च प्रणम्य च}
{ॐ नमो वासुदेवायेति जपेदष्टोत्तरं शतम्} %४०

\twolineshloka
{जुहुयाच्च घृताभ्यक्तैस्तिलव्रीहियवादिभिः}
{षडक्षरेण मन्त्रेण स्वाहाकाराऽन्वितेन च} %४१

\twolineshloka
{उपास्य पश्चिमां सन्ध्यां प्रणम्य गरुडध्वजम्}
{जपित्वा पूर्ववन्मन्त्रं क्षितिशायी भवेत्सदा} %४२

\twolineshloka
{सर्वमेतद्विधानं तु कार्यं पञ्च दिनानि तु}
{विशेषोऽत्र व्रते ह्यस्मिन्यदन्यूनं शृणुष्व तत्} %४३

\twolineshloka
{प्रथमेऽह्नि हरेः पादौ पूजयेत्कमलैर्व्रती}
{द्वितीये बिल्वपत्रेण जानुदेशं समर्चयेत्} %४४

\twolineshloka
{ततोऽनुपूजयेच्छीर्षं मालत्या चक्रपाणिनः}
{कार्तिक्यां देवदेवस्य भक्त्या तद्गतमानसः} %४५

\twolineshloka
{अर्चित्वा तं हृषीकेशमेकादश्यां समासतः}
{निःप्राश्य गोमयं सम्यगेकादश्यामुपावसेत्} %४६

\twolineshloka
{गोमूत्रं मन्त्रवद्भूमौ द्वादश्यां प्राशयेद्व्रती}
{क्षीरं चैव त्रयोदश्यां चतुर्दश्यां तथा दधि} %४७

\threelineshloka
{सम्प्राश्य कायशुद्ध्यर्थं लङ्घयित्वा चतुर्दिनम्}
{पञ्चमे दिवसे स्नात्वा विधिवत्पूज्य केशवम्}
{भोजयेद्ब्राह्मणान्भक्त्या तेभ्यो दद्याच्च दक्षिणाम्} %४८

\twolineshloka
{पापबुद्धिं परित्यज्य ब्रह्मचर्येण धीमता}
{मद्यं मांसं परित्याज्यं मैथुनं पापकारणम्} %४९

\twolineshloka
{शाकाहारेण मुन्यन्नैः कृष्णार्चनपरो नरः}
{ततो नक्तं समश्नीयात्पञ्चगव्यपुरःसरम्} %५०


\onelineshloka
{एवं सम्यक्समाप्यं स्याद्यथोक्तं फलमाप्नुयात्} %५१

\twolineshloka
{मद्यपो यः पिबेन्मद्यं जन्मनो मरणान्तिकम्}
{एतद्भीष्मव्रतं कृत्वा प्राप्नोति परमं पदम्} %५२

\twolineshloka
{स्त्रीभिर्वा भर्तृवाक्येन कर्तव्यं धर्मवर्धनम्}
{विधवाभिश्च कर्तव्यं मोक्षसौख्यातिवृद्धये} %५३

\threelineshloka
{अयोध्यायां पुरा कश्चिदतिथिर्नाम वै नृपः}
{वसिष्ठवचनात्कृत्वा व्रतमेतत्सुदुर्लभम्}
{भुक्त्वेह निखिलान्भोगानन्ते विष्णुपुरं ययौ} %५४

\threelineshloka
{इत्थं कुर्याद्व्रतं नित्यं पञ्चकं भीष्मसंज्ञितम्}
{नियमेनोपवासेन पञ्चगव्येन वा पुनः}
{पयोमूलफलाहारैर्हविष्यैर्व्रततत्परः} %५५

\twolineshloka
{पौर्णमासीदिने प्राप्ते पूजां कृत्वा तु पूर्ववत्}
{ब्राह्मणान्भोजयेद्भक्त्या गां च दद्यात्सवत्सकाम्} %५६

\fourlineindentedshloka
{यद्भीष्मपञ्चकमिति प्रथितं पृथिव्या-}
{मेकादशीप्रभृति पञ्चदशीनिरुद्धम्}
{उक्तं न भोजनपरस्य तदा निषेध-}
{स्तस्मिन्व्रते शुभफलं प्रददाति विष्णुः} %५७


\iti{भीष्म पञ्चकव्रतमाहात्म्यवर्णनं}{नाम द्वात्रिंशोऽध्यायः}{३२}

\sect{अथ त्रयस्त्रिंशोऽध्यायः}


\uvacha{ईश्वर उवाच}
\twolineshloka
{प्रबोधिन्याश्च माहात्म्यं पापघ्नं पुण्यवर्धनम्}
{मुक्तिदं तत्त्वबुद्धीनां शृणुष्व सुरसत्तम} %१

\twolineshloka
{तावद्गर्जति सेनानीर्गङ्गा भागीरथी क्षितौ}
{यावत्प्रयाति पापघ्नी कार्तिके हरिबोधिनी} %२

\twolineshloka
{तावद्गर्जन्ति तीर्थानि आसमुद्रसरांसि वै}
{यावत्प्रबोधिनी विष्णोस्तिथिर्नाऽऽयाति कार्तिके} %३

\twolineshloka
{अश्वमेधसहस्राणि राजसूयशतानि च}
{एकेनैवोपवासेन प्रबोधिन्या यथाऽभवत्} %४

\twolineshloka
{दुर्लभं चैव दुष्प्राप्यं त्रैलोक्ये सचराचरे}
{तदपि प्रार्थितं विप्र ददाति प्रतिबोधिनी} %५

\twolineshloka
{ऐश्वर्यं सन्ततिं ज्ञानं राज्यं च सुखसम्पदः}
{ददात्युपोषिता विप्र हेलया हरिबोधिनी} %६

\twolineshloka
{मेरुमन्दरतुल्यानि पापान्युपार्जितानि च}
{एकेनैवोपवासेन दहते हरिबोधिनी} %७

\twolineshloka
{उपवासं प्रबोधिन्यां यः करोति स्वभावतः}
{विधिना नरशार्दूल यथोक्तं लभते फलम्} %८

\twolineshloka
{पूर्वजन्मसहस्रेषु पापं यत्समुपार्जितम्}
{जागरेण प्रबोधिन्यां दह्यते तृलराशिवत्} %९

\twolineshloka
{शृणु षण्मुख वक्ष्यामि जागरस्य च लक्षणम्}
{तस्य विज्ञानमात्रेण दुर्लभो न जनार्दनः} %१०

\twolineshloka
{गीतं वाद्यं च नृत्यं च पुराणपठनं तथा}
{धूपं दीपं च नैवेद्यं पुष्पगन्धाऽनुलेपनम्} %११

\twolineshloka
{फलमर्घ्यं च श्रद्धा च दानमिन्द्रियसंयमम्}
{सत्याऽन्वितं विनिन्दं च मुदा युक्तं क्रियान्वितम्} %१२

\twolineshloka
{साश्चर्यं चैव प्रोत्साहमालस्यादिविवर्जितम्}
{प्रदक्षिणादिसंयुक्तं नमस्कारपुरःसरम्} %१३

\twolineshloka
{नीराजनसमायुक्तमनिर्विण्णेन चेतसा}
{यामेयामे महाभाग कुर्वन्नीराजनं हरेः} %१४

\twolineshloka
{एतैर्गुणैः समायुक्तं कुर्याज्जागरणं विभोः}
{एकाग्रमनसा यस्तु न पुनर्जायते भुवि} %१५

\twolineshloka
{य एवं कुरुते भक्त्या वित्तशाठ्यविवर्जितः}
{जागरं वासरे विष्णोर्लीयते परमात्मनि} %१६

\twolineshloka
{पुरुषसूक्तेन यो नित्यं कार्तिकेऽथार्चयेद्धरिम्}
{वर्षकोटिसहस्राणि पूजितस्तेन केशवः} %१७

\twolineshloka
{यथोक्तेन विधानेन पञ्चरात्रोदितेन वै}
{कार्तिके त्वर्चयेन्नित्यं मुक्तिभागी भवेन्नरः} %१८

\twolineshloka
{नमो नारायणायेति कार्तिके योर्चयेद्धरिम्}
{स मुक्तो नारकैर्दुःखैः पदं गच्छत्यनामयम्} %१९

\twolineshloka
{हरेर्नामसहस्रं च गजराजस्य मोक्षणम्}
{कार्तिके पठते यस्तु पुनर्जन्म न विन्दति} %२०

\twolineshloka
{युगकोटिसहस्राणि मन्वन्तरशतानि च}
{द्वादश्यां कार्तिके मासि जागरी वसते दिवि} %२१

\twolineshloka
{कुले तस्य च ये जाताः शतशोथ सहस्रशः}
{प्राप्नुवन्ति पदं विष्णोस्तस्मात्कुर्वीत जागरम्} %२२

\twolineshloka
{कार्तिके पश्चिमे यामे स्तवं गानं करोति यः}
{श्वेतद्वीपे तु वसते पितृभिः सह सुव्रत} %२३

\twolineshloka
{नैवेद्यदानं हरये कार्तिके दिनसङ्क्षये}
{युगानि वसते स्वर्गे तावन्ति मुनिसत्तमाः} %२४

\twolineshloka
{अक्षयं मुनिशार्दूल मालतीकमलार्चनम्}
{अर्चयेद्देवदेवेशं स याति परमं पदम्} %२५

\twolineshloka
{कार्तिके शुक्लपक्षे तु कृत्वा ह्येकादशीं नरः}
{प्रातर्दत्त्वा शुभान्कुम्भान्स याति मम मन्दिरम्} %२६

\twolineshloka
{अत्रैव तु प्रकर्तव्यः प्रबोधस्तु हरेः खग}
{हतः शङ्खासुरो दैत्यो नभसः शुक्लपक्षके} %२७

\twolineshloka
{एकादश्यां ततो विष्णुश्चातुर्मास्ये प्रसुप्तवान्}
{क्षीराम्भोधौ जागृतोऽसावेकादश्यां तु कार्तिके} %२८

\threelineshloka
{अतः प्रबोधनं कार्यमेकादश्यां तु वैष्णवैः}
{उत्तिष्ठोत्तिष्ठ गोविन्द उत्तिष्ठ गरुडध्वज}
{उत्तिष्ठ कमलाकान्त त्रैलोक्यं मङ्गलं कुरु} %२९

\twolineshloka
{इत्युक्त्वा शङ्खभेर्यादि प्रातःकाले तु वादयेत्}
{वीणावेणुमृदङ्गादि नृत्यगीतादि कारयेत्} %३०

\twolineshloka
{उत्थापयित्वा देवेशं पूजां तस्य विधाय च}
{सायङ्काले प्रकर्तव्यस्तुलस्युद्वाहजो विधिः} %३१

\twolineshloka
{सर्वदैकादशी पुण्या विशेषात्कार्तिकी स्मृता}
{यानि कानि च पापानि ब्रह्महत्यादिकानि च} %३२

\twolineshloka
{अन्नमाश्रित्य तिष्ठन्ति सम्प्राप्ते हरिवासरे}
{स केवलमघं भुङ्क्ते यो भुङ्क्ते हरिवासरे} %३३

\twolineshloka
{तस्मात्सर्वप्रयत्नेन कुर्यादेकादशीव्रतम्}
{न कुर्याद्यदि मोहेन उपवासं नराधमः} %३४

\twolineshloka
{नरके नियतं वासः पितृभिः सह तस्य वै}
{सूतके मृतके वाऽपि नोपवासं त्यजेद्बुधः} %३५

\twolineshloka
{दशमीवेधसंयुक्ता त्याज्या चैकादशी व्रते}
{गान्धार्याऽपि पुरा तस्यामुपवासः कृतो गुह} %३६

\twolineshloka
{तस्याः पुत्रशतं नष्टं तस्मात्तां वधजां त्यजेत्}
{एकादशीमुपवसेत्स्नानदानपुरःसरम्} %३७

\twolineshloka
{रुक्माङ्गदोऽपि राजर्षिर्मोहिन्याः सङ्गमेन च}
{इह लोके सुखं भुक्त्वा चामते विष्णुपुरं ययौ} %३८


॥इति प्रबोधोत्सवः॥

॥अथ द्वादशीमाहात्म्यम्॥
\twolineshloka
{द्वादशी पुण्यदा प्रोक्ता सर्वाघौघविनाशिनी}
{किं दानैः कि तपोभिश्च किमु पोष्यैर्व्रतैश्च किम्} %३९

\twolineshloka
{किमिष्टैश्चैव पुत्रैश्च द्वादशी येन सेविता}
{गङ्गायां चैव दुर्भिक्षे प्रत्यहं कोटिभोजनात्} %४०

\twolineshloka
{यत्फलं तदवाप्नोति द्वादश्यामेकभोजनात्}
{यद्दत्तं चार्हते दानं द्वादश्यां तु सिते शुभे} %४१

\twolineshloka
{सिक्थेसिक्थे च वैकस्य कतिब्राह्मणभोजनम्}
{तदहं नैव जानामि महिमानं हि सुव्रत} %४२

\threelineshloka
{शालिग्रामशिलादानं यः कुर्याद्द्वादशीदिने}
{सप्तद्वीपवतीं भूमिं गङ्गायां च रविग्रहे}
{दत्त्वा यत्फलमाप्नोति तत्फलं लभते नरः} %४३

\twolineshloka
{पञ्चामृतैस्तु यो विष्णुं भक्त्या संस्नापयेद्द्विज}
{स सर्वकुलमुद्धृत्य विष्णुलोके महीयते} %४४

\threelineshloka
{शुक्ले कार्तिकमासस्य द्वादश्यां परमोत्सवे}
{प्रातरारभ्य यः कुर्यात्स्नानदानादिकं तथा}
{स तु मोक्षमवाप्नोति नात्र कार्या विचारणा} %४५

\twolineshloka
{द्वादश्यां कार्तिके मासि स्नानसन्ध्यादिकर्म च}
{कृत्वा दामोदरं पूज्य भक्तिश्रद्धासमन्वितः} %४६

\twolineshloka
{यस्तस्यां सूपनैवेद्यं न ददाति नराधमः}
{नरके नियतं वासो भवतीत्यनुशुश्रुम} %४७

\twolineshloka
{तस्मात्सूपस्य नैवेद्यं द्वादश्यां कार्तिके शुभे}
{दद्याद्भक्तियुतो ब्रह्मंश्चान्यथा नरकं व्रजेत्} %४८

\twolineshloka
{यस्तस्यां दम्पतीनां तु भोजनं कुरुते नरः}
{न तस्य फलविश्रान्तिर्मया वक्तुं तु शक्यते} %४९

\twolineshloka
{धात्रीच्छायां गतो यस्तु द्वादश्यां पूजयेद्धरिम्}
{तत्रैव भोजनं यस्तु ब्राह्मणानां तु कारयेत्} %५०

\twolineshloka
{स्वयं च तत्र भुङ्क्ते यः सूपभक्ष्यादिकं तथा}
{न तस्य पुनरावृत्तिः कल्पकोटिशतैरपि} %५१

\twolineshloka
{एवं प्रातर्विधायाथ पूजां दामोदरस्य हि}
{रात्रौ पुनः प्रकर्तव्यं पूजाकर्म हरेर्द्विज} %५२

\twolineshloka
{तुलसीसन्निधौ कृत्वा पताकाध्वजशोभितम्}
{पुष्पमालासमाकीर्णं नानारत्नोपशोभितम्} %५३

\twolineshloka
{मुक्तादामभिराच्छन्नं कृत्वा मण्डपमुत्तमम्}
{पूजयेद्विष्णुमव्यग्रस्तद्गतैकाग्रमानसः} %५४

\twolineshloka
{पञ्चरात्रोक्तमार्गेण गन्धपुष्पाक्षतादिभिः}
{नवनीतं दधिक्षीरं तथैव च घनं घृतम्} %५५

\twolineshloka
{विविधैः खाद्यनैवेद्यैर्जलेन च सुगन्धिना}
{युक्तं निवेदयेद्विष्णोस्ताम्बूलं सलवङ्गकम्} %५६

\twolineshloka
{पुष्पाणि च विचित्राणि सुगन्धीनि बहूनि च}
{प्रोक्षयित्वा च विधिवदर्पयित्वा दलैः शुभैः} %५७

\twolineshloka
{तुलस्याश्चापि धात्र्याश्च फलैश्चापि प्रपूजयेत्}
{नीराजनं ततः कृत्वा मन्त्रपुष्पं समर्पयेत्} %५८

\twolineshloka
{अभिषेकं विना सर्वपूजां कृत्वा विधानतः}
{विष्णोः पूजां समाप्याथ ब्राह्मणानां प्रपूजनम्} %५९

\twolineshloka
{कुर्याद्भक्तियुतो विप्र दद्याच्चैव फलादिकम्}
{ताम्बूलं च ततो दत्त्वा दक्षिणां शक्तितोऽर्पयेत्} %६०

\twolineshloka
{ततो वृद्धान्पितॄन्मातॄन् पूजयित्वा विधानतः}
{ततः स्वयं स्वभार्याभिर्नैवेद्यं भक्षयेत्सुधीः} %६१

\twolineshloka
{इत्येवं तु विधानेन यः कुर्याद्द्वादशीव्रतम्}
{न तस्य लोकाः क्षीयन्ते कल्पकोटिशतैरपि} %६२

\twolineshloka
{पुत्रपौत्रैः परिवृतो भुक्त्वा भोगान्मनोहरान्}
{भोगान्ते च व्रजेन्मोक्षमतीतकुलसप्तकैः} %६३

\twolineshloka
{तस्मान्नारद माहात्म्यं द्वादश्याः कार्तिकस्य च}
{न मया शक्यते वक्तुं किमन्यैर्मनुजैरपि} %६४

\twolineshloka
{द्वादश्या ह्युत्तमं पुण्यं माहात्म्यं यः पठेन्नरः}
{शृणुयाद्वा मुनिश्रेष्ठ स याति परमां गतिम्} %६५

\twolineshloka
{राजर्षिरम्बरीषोऽपि चकारैतद्व्रतं शुभम्}
{यथाविधि तपोनिष्ठस्तेन मोक्षमवाप्तवान्} %६६


\iti{प्रबोधोत्सवद्वादशीतिथिकृत्यवर्णनं}{नाम त्रयस्त्रिंशोऽध्यायः}{३३}

\sect{अथ चतुस्त्रिंशोऽध्यायः}


\uvacha{नारद उवाच}
\twolineshloka
{व्रतानामपि सर्वेषां ब्रह्मन्नुद्यापनं श्रुतम्}
{अभावे तूद्यापनस्य फलं नैवाऽऽप्नुयात्क्वचित्} %१

\twolineshloka
{कृतव्रतफलाप्त्यर्थं कुर्यादुद्यापनं बुधः}
{अन्यथा निष्फलं याति कृतं व्रतमनुत्तमम्} %२

\twolineshloka
{कार्तिकेऽपि कृतं देव व्रतानामुत्तमं व्रतम्}
{न तस्योद्यापनाभावे व्रतोक्तफलमाप्नुयात्} %३

\twolineshloka
{तस्मात्कार्तिकमासस्य चोद्यापनविधिं प्रभो}
{वद मे शिष्यवर्याय प्रपन्नायानुवर्तिने} %४


\uvacha{ब्रह्मोवाच}
\twolineshloka
{अथोर्जोद्यापनं वक्ष्ये सर्वपापप्रणाशनम्}
{तच्छृणुष्व महाभक्त्या सविधानं समासतः} %५

\twolineshloka
{ऊर्जे शुक्लचतुर्दश्यां कुर्यादुद्यापनं व्रती}
{व्रतसम्पूरणार्थाय विष्णुप्रीत्यर्थहेतवे} %६

\twolineshloka
{तुलस्या उपरिष्टात्तु कुर्यान्मण्डपिकां शुभाम्}
{कदलीस्तम्भसंयुक्तां नानाधातुविचित्रिताम्} %७

\twolineshloka
{दीपमाला चतुर्दिक्षु कार्या तत्र सुशोभना}
{सुतोरणाश्चतुर्द्वारः पुष्पचामरशोभिताः} %८

\twolineshloka
{द्वारेषु द्वारपालाश्च पूजयेन्मृन्मयान्पृथक्}
{जयश्च विजयश्चैव चण्डश्चैव प्रचण्डकः} %९

\twolineshloka
{नन्दश्चैव सुनन्दश्च कुमुदः कुमुदाक्षकः}
{एतांश्चतुर्षु द्वारेषु पूजयेद्भक्तिसंयुतः} %१०

\twolineshloka
{तुलसीमूलदेशे तु सर्वतोभद्रसंज्ञितम्}
{चतुर्भिर्वर्णकैः सम्यक्छोभाढ्यं समलङ्कृतम्} %११

\twolineshloka
{तस्योपरिष्टात्कलशं पूर्णरत्नसमन्वितम्}
{तत्र सम्पूजयेद्देवं शङ्खचक्रगदाधरम्} %१२

\twolineshloka
{कौशेयपीतवसनं लक्ष्म्या युक्तं प्रपूजयेत्}
{इन्द्रादिलोकपालांश्च मण्डपे पूजयेद्व्रती} %१३

\twolineshloka
{तस्यामुपवसेद्भक्त्या शान्तः प्रणतमानसः}
{रात्रौ जागरणं कुर्याद्गीतवाद्यादिमङ्गलैः} %१४

\twolineshloka
{गीतं कुर्वन्ति ये भक्त्या जागरे चक्रपाणिनः}
{जन्मान्तरशतोद्भूतैस्ते मुक्ताः पापसञ्चयैः} %१५

\twolineshloka
{ततस्तु पूर्णिमायां तु सपत्नीकान्द्विजोत्तमान्}
{त्रिंशन्मितानथैकं वा ब्राह्मणांश्च निमन्त्रयेत्} %१६

\twolineshloka
{प्रातःस्नानं ततः कृत्वा देवपूजां तथैव च}
{स्थण्डिले च ततः कृत्वा समाधायाग्निमत्र हि} %१७

\twolineshloka
{अतो देवेति मन्त्रेण जुहुयात्तिलपायसम्}
{प्रीत्यर्थं देवदेवस्य देवानां च पृथक्पृथक्} %१८

\twolineshloka
{होमशेषं समाप्याथ ब्राह्मणान्पूज्य भक्तितः}
{ब्राह्मणेभ्यो यथाशक्त्या प्रदद्याद्दक्षिणां नरः} %१९

\twolineshloka
{ततो गां कपिलां तत्र पूजयेद्विधिवद्व्रती}
{सवत्सां गां तथा दद्याद्विप्राय च कुटुम्बिने} %२०

\twolineshloka
{गुरुं व्रतोपदेष्टारं वस्त्रालङ्कारभूषणैः}
{सपत्नीकं समभ्यर्च्य तांश्च विप्रान्क्षमापयेत्} %२१

\twolineshloka
{युष्मत्प्रसादाद्देवेशः प्रसन्नोऽस्तु सदा मम}
{व्रतादस्माच्च यत्पापं सप्तजन्मकृतं मया} %२२

\twolineshloka
{तत्सर्वं नाशमायातु स्थिरा मे चास्तु सन्ततिः}
{मनोरथास्तु सफलाः सन्तु भक्तिर्हरौ भवेत्} %२३

\twolineshloka
{सतां समागमो भूयान्मम जन्मनिजन्मनि}
{इति क्षमाप्य तान्विप्रान्प्रसाद्य च विसर्जयेत्} %२४

\twolineshloka
{प्रतिमां तां गुरोर्दद्यात्सवस्त्रां मुनिपुङ्गव}
{ततः सुहृद्गुरुयुतः स्वयं भुञ्जीत भक्तिमान्} %२५

\twolineshloka
{द्वादश्यां प्रतिबुद्धोऽसौ त्रयोदश्यां युतः सुरैः}
{दृष्टोर्च्चितश्चतुर्दश्यां तस्मात्पूज्यस्तिथाविह} %२६

\twolineshloka
{पूजयेद्देवदेवेशं सौवर्णं गुर्वनुज्ञया}
{पराऽत्र पौर्णमास्यां तु यात्रा स्यात्पुष्करस्य तु} %२७

\twolineshloka
{वरान्दत्त्वा यतो विष्णुर्मत्स्यरूपोऽभवत्ततः}
{तस्यां दत्तं हुतं जप्तं तदक्षय्यफलं भवेत्} %२८

\twolineshloka
{कार्तिके मासि कर्तव्यो विधिरेष हि नारद}
{एवं यः कुरुते सम्यक्कार्तिकस्य व्रतं नरः} %२९

\twolineshloka
{यत्फलं तदवाप्नोति व्रतं कृत्वा तु कार्तिके}
{ते धन्यास्ते सदा पूज्यास्तेषां वै सफलोदयः} %३०

\twolineshloka
{विष्णुभक्तिरता ये स्युः कार्तिके व्रतचारिणः}
{देहस्थितानि पापानि विलयं यान्ति तत्क्षणात्} %३१

\twolineshloka
{क्व यामोऽद्य भवत्येष यदूर्जव्रतकृन्नरः}
{इति सर्वाणि पापानि रटन्तीह पुनःपुनः} %३२

\twolineshloka
{तस्मात्कार्तिकमासस्य सदृशं नहि विद्यते}
{सर्वपापस्य दहने अग्नेः सदृश उच्यते} %३३

\twolineshloka
{ऊर्जोद्यापनमाहात्म्यं शृणुयाच्छ्रद्धयान्वितः}
{श्रावयेद्वा पुमान्यस्तु विष्णुसायुज्यमाप्नुयात्} %३४


\uvacha{नारद उवाच}
\twolineshloka
{ऊर्जे व्रतोद्यापनादावशक्तः सिद्धिभाक्कथम्}
{कथं विमुच्यते जन्तुर्दुःखसंसारसागरात्} %३५


\uvacha{ब्रह्मोवाच}
\twolineshloka
{शृणुयादूर्जमाहात्म्यं नियमेन शुचिः पुमान्}
{उद्यापनफलं प्राप्य विष्णुलोके वसेच्च सः} %३६


\iti{व्रतोद्यापनविधिकथनं}{नाम चतुस्त्रिंशोऽध्यायः}{३४}

\sect{अथ पञ्चत्रिंशोऽध्यायः}


\uvacha{ब्रह्मोवाच}
\twolineshloka
{वैकुण्ठाख्यचतुर्दश्या माहात्म्यं ते वदाम्यहम्}
{वालखिल्यैः पुरा प्रोक्तं सङ्क्षेपेण शृणुष्व तत्} %१


\uvacha{वालखिल्या ऊचुः}
\twolineshloka
{कार्तिकस्य सिते पक्षे चतुर्दश्यां समागमत्}
{वैकुण्ठेशस्तु वैकुण्ठाद्वाराणस्यां कृते युगे} %२

\twolineshloka
{रात्र्यां तुर्यांशशेषायां स्नात्वाऽसौ मणिकर्णिके}
{गृहीत्वा हेमपद्मानां सहस्रं वै ततोऽव्रजत्} %३

\twolineshloka
{अतिभक्त्या पूजयितुं शिवया सहितं शिवम्}
{विधाय पूजां वैश्वेशीं ततः पद्मैरपूजयत्} %४

\twolineshloka
{सहस्रसङ्ख्यां कृत्वादावेकनाम्ना ततः परम्}
{आरब्धं पूजनं तेन शिवस्तद्भक्तिमैक्षत} %५

\twolineshloka
{एकं पद्मं पद्ममध्यान्निलीयाऽऽत्तं हरेण तु}
{ततः पूजितवान्विष्णुरेकोनं कमलं त्वभूत्} %६

\twolineshloka
{इतस्ततस्तेन दृष्टं पद्मं तिष्ठति न क्वचित्}
{कमलेषु भ्रमो जातोऽथवा नामसु मे भ्रमः} %७

\twolineshloka
{क्षणं विचार्य स हरिर्न मे नामभ्रमोऽभवत्}
{पद्मे चैव भ्रमो जातो विचार्यैवं पुनःपुनः} %८

\twolineshloka
{सहस्रपद्मसङ्कल्पः पूजार्थं तु कृतो मया}
{अर्च्यः कथं महादेव एकोनकमलैर्मया} %९

\twolineshloka
{यद्यानेतुं गमिष्यामि भङ्गः स्यादासनस्य तु}
{अतः परं किं विधेयं चिन्तोद्विग्नो हरिस्तदा} %१०

\twolineshloka
{एकः प्रकार उत्पन्नो हृदयेऽस्य मुनीश्वराः}
{पुण्डरीकाक्ष इत्येवं मां वदन्ति मुनीश्वराः} %११

\twolineshloka
{नेत्रं मे पद्मसदृशे पद्मार्थे त्वर्पयाम्यहम्}
{इति निश्चित्य मनसा दत्त्वा तर्जनिकां स तु} %१२

\twolineshloka
{नेत्रमध्यात्तदुत्पाट्य महादेवस्तु पूजितः}
{ततो महेश्वरस्तुष्टो वाक्यमेतदुवाच ह} %१३


\uvacha{महादेव उवाच}
\twolineshloka
{त्वत्समो नास्ति मद्भक्तस्त्रैलोक्ये सचराचरे}
{राज्यं दत्तं त्रिलोक्यास्ते भव त्वं लोकपालकः} %१४

\twolineshloka
{अन्यं वरय भद्रं ते वरं यन्मनसेप्सितम्}
{अवश्यमेव दास्यामि नात्र कार्या विचारणा} %१५

\twolineshloka
{मद्भक्तिं तु समालम्ब्य ये द्विषन्ति जनार्दनम्}
{ते मद्द्वेष्या नरा विष्णो व्रजेयुर्नरकं ध्रुवम्} %१६


\uvacha{विष्णुरुवाच}
\twolineshloka
{त्रैलोक्यरक्षाकरणं ममादिष्टं महेश्वर}
{दुमर्दाश्च महासत्त्वा दैत्या मार्याः कथं मया} %१७

\uvacha{शिव उवाच}
\twolineshloka
{एतत्सुदर्शनं चक्रं महादैत्यनिकृन्तनम्}
{गृहाण भगवन्विष्णो मया तुभ्य निवेदितम्} %१८

\twolineshloka
{अनेन सर्वदैत्यानां भगवन्कदनं कुरु}
{एवं चक्रं हरेर्दत्त्वा ततो वचनमब्रवीत्} %१९


\uvacha{शिव उवाच}
\twolineshloka
{वर्षे च हेमलम्बाख्ये मासे श्रीमति कार्तिके}
{शुक्लपक्षे चतुर्दश्यामरुणाभ्युदयं प्रति} %२०

\twolineshloka
{महादेवतिथौ ब्राह्मे मुहूर्ते मणिकर्णिके}
{स्नात्वा वैश्वेश्वरं लिङ्गं वैकुण्ठादेत्य पूजितम्} %२१

\twolineshloka
{सहस्रकमलैस्तस्माद्भविष्यति मम प्रिया}
{विख्याता सर्वलोकेषु वैकुण्ठाख्या चतुर्दशी} %२२

\twolineshloka
{अन्यं वरं प्रयच्छामि शृणु विष्णो वचो मम}
{पूर्वरात्रेषु ते पूजा कर्तव्या सर्वजातिभिः} %२३

\twolineshloka
{उपवासं दिवा कुर्यात्सायङ्काले तवार्चनम्}
{पश्चान्ममार्चनं कार्यमन्यथा निष्फलं भवेत्} %२४

\twolineshloka
{ग्राह्या तु हरिपूजायां रात्रिव्याप्ता चतुर्दशी}
{अरुणोदयवेलायां शिवपूजां समाचरेत्} %२५

\twolineshloka
{सहस्रकमलैर्विष्णुरादौ यैः पूजितो नरैः}
{पश्चाच्छिवः पूजितश्चेज्जीवन्मुक्तास्त एव हि} %२६

\twolineshloka
{सायं स्नात्वा पञ्चनदे बिन्दुमाधवमर्चयेत्}
{स्नात्वा यो विष्णुकाञ्च्यां वाऽनन्तसेनं समर्चयेत्} %२७

\twolineshloka
{रुद्रकाञ्च्यां ततः स्नात्वा प्रणवेशं समर्चयेत्}
{आदौ स्नात्वा वह्नितीर्थे यजेन्नारायणं ततः} %२८

\twolineshloka
{रेतोदके ततः स्नात्वा केदारेशं समर्चयेत्}
{आदौ स्नात्वा सूर्यपुत्र्यां वेणीमाधवमर्चयेत्} %२९

\twolineshloka
{जाह्नव्यां च ततः स्नात्वा सङ्गमेशं प्रपूजयेत्}
{सर्वाः श्रियस्तस्य वश्याः सत्यं विष्णो मयोदितम्} %३०

\twolineshloka
{एवं तस्मै वरान्दत्त्वा ह्यन्तर्धानं ययौ शिवः}
{तस्मात्सर्वप्रयत्नेन पूज्यौ हरिहरावुभौ} %३१

\twolineshloka
{कलौ दशसहस्राणि विष्णुस्त्यजति मेदिनीम्}
{तदर्द्धं जाह्नवीतोयं तदर्द्धं ग्रामदेवताः} %३२

\twolineshloka
{कार्तिक्यां पूर्णिमायां तु कुर्यात्त्रैपुरमुत्सवम्}
{दीपो देयोऽवश्यमेव सायङ्काले शिवालये} %३३

\twolineshloka
{त्रिपुरो नाम दैत्येन्द्रः प्रयागे तप आस्थितः}
{तपसा तस्य सन्तुष्टो ददौ ब्रह्मा वरं परम्} %३४

\twolineshloka
{देवासुरमनुष्येभ्यो न ते मृत्युर्भविष्यति}
{इति लब्धवरो दैत्यो विश्वकर्मविनिर्मितम्} %३५

\twolineshloka
{त्रिपुराख्यं विमानं तमारुह्य भुवनत्रयम्}
{यदा वै पीडयामास तदा देवैः स्तुतो हरः} %३६

\twolineshloka
{त्रिपुरं घातयामास बाणेनैकेन शत्रुहा}
{कार्तिक्यां पूर्णिमायां तु सर्वे देवाः प्रतुष्टुवुः} %३७

\twolineshloka
{तस्मिन्दिने सर्वदेवैर्दीपा दत्ता हराय च}
{सर्वथैव प्रदेयाश्च दीपास्तु हरतुष्टये} %३८

\twolineshloka
{विंशतिः सप्तशतकाः सहिता दीपवर्तयः}
{ददद्दीपं पूर्णिमायां सर्वपापैः प्रमुच्यते} %३९

\twolineshloka
{पौर्णमास्यां तु सन्ध्यायां कर्तव्यस्त्रिपुरोत्सवः}
{दद्यादनेन मन्त्रेण प्रदीपांश्च सुरालये} %४०

\twolineshloka
{कीटाः पतङ्गा मशकाश्च वृक्षा जले स्थले ये विचरन्ति जीवाः}
{दृष्ट्वा प्रदीपं न च जन्मभागिनो भवन्तु नित्यं श्वपचा हि विप्राः} %४१

\twolineshloka
{कार्यस्तस्मात्पौर्णमास्यां त्रिपुराय महोत्सवः}
{कार्तिक्यां कृत्तिकायोगे यः कुर्यात्स्वामिदर्शनम्} %४२

\twolineshloka
{सप्त जन्म भवेद्विप्रो धनाढ्यो वेदपारगः}
{अत्र कृत्वा वृषोत्सर्गं नक्ताच्छैवपुरं व्रजेत्} %४३


\iti{वैकुण्ठचतुर्दशी त्रिपुरीपूर्णिमाव्रतविधानकथनं}{नाम पञ्चत्रिंशोऽध्यायः}{३५}

\sect{अथ षट्त्रिंशोऽध्यायः}


\uvacha{ब्रह्मोवाच}
\twolineshloka
{यास्तिस्रस्तिथयः पुण्या अन्तिके शुक्लपक्षके}
{कार्तिके मासि विप्रेन्द्र पूर्णिमान्ताः शुभवहाः} %१

\twolineshloka
{अन्तिपुष्करिणी संज्ञा सर्वपापक्षयावहा}
{कार्त्तिके मासि सम्पूर्णं यो वै स्नानं करोति ह} %२

\twolineshloka
{तिथिष्वेतासु सः स्नानात्पूर्णमेव फलं लभेत्}
{सर्वे वेदास्त्रयोदश्यां गत्वा जन्तून्पुनन्ति हि} %३

\twolineshloka
{चतुर्दश्यां सयज्ञाश्च देवा जन्तून्पुनन्ति हि}
{पूर्णिमायां सुतीर्थानि विष्णुना संस्थितानि हि} %४

\twolineshloka
{ब्रह्मघ्नान्वा सुरापान्वा सर्वाञ्जन्तून्पुनन्ति हि}
{उष्णोदकेन यः स्नायात्कार्त्तिक्यादिदिनत्रये} %५

\twolineshloka
{रौरवं नरकं याति यावदिन्द्राश्चतुर्दश}
{आमासनियमाशक्तः कुर्यादेतद्दिनत्रये} %६

\twolineshloka
{तेन पूर्णफलं प्राप्य मोदते विष्णुमन्दिरे}
{यो वै देवान्पितॄन्विष्णुं गुरुमुद्दिश्य मानवः} %७

\twolineshloka
{न स्नानादि करोत्यद्धा स याति नरकं ध्रुवम्}
{कुटुम्बभोजन यस्तु गृहस्थस्तु दिनत्रये} %८

\twolineshloka
{सर्वान्पितॄन्समुद्धृत्य स याति परमं पदम्}
{गीतापाठं तु यः कुर्यादन्तिमे च दिनत्रये} %९

\twolineshloka
{दिनेदिनेऽश्वमेधानां फलमेति न संशयः}
{सहस्रनामपठनं यः कुर्यात्तु दिनत्रये} %१०

\twolineshloka
{न पापैर्लिप्यते क्वापि पद्मपत्रमिवाम्भसा}
{देवत्वं मनुजैः कैश्चित्कैश्चित्सिद्धत्वमेव च} %११

\twolineshloka
{तस्य पुण्यफलं वक्तुं कः शक्तो दिवि वा भुवि}
{यो वै भागवतं शास्त्रं शृणोति च दिनत्रयम्} %१२

\twolineshloka
{कैश्चित्प्राप्तो ब्रह्मभावो दिनत्रयनिषेवणात्}
{ब्रह्मज्ञानेन वा मुक्तिः प्रयागमरणेन वा} %१३

\twolineshloka
{अथ वा कार्त्तिके मासि दिनत्रयनिषेवणात्}
{कार्त्तिके हरिपूजां तु यः करोति दिनत्रये} %१४

\twolineshloka
{न तस्य पुनरावृत्तिः कल्पकोटिशतैरपि}
{कार्त्तिके मासि विप्रेन्द्र सर्वमन्त्यदिनत्रये} %१५

\twolineshloka
{पुण्यं तत्रापि वैशेष्यं राकायां वर्ततेऽनघ}
{प्रातःकाले समुत्थाय शौचं स्नानादिकं चरेत्} %१६

\twolineshloka
{समाप्य सर्वकर्माणि विष्णुपूजां समाचरेत्}
{उद्याने वा गृहे वाऽपि कार्त्तिक्यां विष्णुतत्परः} %१७

\twolineshloka
{मण्डपं तत्र कुर्वीत कदलीस्तम्भमण्डितम्}
{चूतपल्लवसंवीतमिक्षुदण्डैः सुमण्डितम्} %१८

\twolineshloka
{चित्रवस्त्रैः स्वलङ्कृत्य तत्र देवं प्रपूजयेत्}
{चूतपल्लवपुष्पाढ्यैः फलाद्यैः पूजयेद्धरिम्} %१९

\twolineshloka
{शृणुयादूर्जमाहात्म्यं नियमेन शुचिः पुमान्}
{सम्पूर्णमथ वाऽध्यायमेकश्लोकमथापि वा} %२०

\twolineshloka
{मुहूर्तं वाऽपि शृणुयात्कथां पुण्यां दिनेदिने}
{यदि प्रतिदिनं श्रोतुमशक्तः स्यात्तु मानवः} %२१

\twolineshloka
{पुण्यमासेऽथवा पुण्यतिथौ संशृणुयादपि}
{तेन पुण्यप्रभावेन पापान्मुक्तो भवेन्नरः} %२२

\twolineshloka
{पुराणज्ञः शुचिर्दक्षः शान्तो विगतमत्सरः}
{साधुः कारुणिको वाग्ग्मी वदेत्पुण्यां कथां सुधीः} %२३

\twolineshloka
{व्यासासनं समारूढो यदा पौराणिको भवेत्}
{आसमाप्तेः प्रसङ्गस्य नमस्कुर्यान्न कस्यचित्} %२४

\twolineshloka
{न दुर्जनसमाकीर्णे न शूद्रश्वापदावृते}
{देशे न द्यूतसदने वदेत्पुण्यकथां सुधीः} %२५

\twolineshloka
{श्रद्धाभक्तिसमायुक्ता नाऽन्यकार्येषु लालसा}
{वाग्यताः शुचयो दक्षाः श्रोतारः पुण्यभागिनः} %२६

\twolineshloka
{अभक्ता ये कथां पुण्यां शृण्वन्ति मनुजाधमाः}
{तेषां पुण्यफलं नास्ति दुःखं स्याजन्मजन्मनि} %२७

\twolineshloka
{पौराणिकं च मासान्ते पूजयेद्भक्तितत्परः}
{गन्धमाल्यैस्तथा वस्त्रैरलङ्कारैर्धनेन च} %२८


\onelineshloka
{शृण्वन्ति च कथां भक्त्या न दरिद्रा न पापिनः} %२९

\twolineshloka
{कथायां कीर्त्यमानायां ये गच्छन्त्यन्यतो नराः}
{भोगान्तरे प्रणश्यन्ति तेषां दाराश्च सम्पदः} %३०

\twolineshloka
{उच्चासनसमारूढो न नरः प्रणतो भवेत्}
{विषवृक्षस्तथा स्वापे वने चाजगरो भवेत्} %३१

\twolineshloka
{कथायां कीर्त्यमानायां विघ्नं कुर्वन्ति ये नराः}
{कोट्यब्दनरकान्भुक्त्वा भवन्ति ग्रामसूकराः} %३२

\twolineshloka
{ये श्रावयन्ति मनुजाः कथां पौराणिकीं शुभाम्}
{कल्पकोटिशतं साग्रं तिष्ठन्ति ब्रह्मणः पदे} %३३

\twolineshloka
{आसनार्थे प्रयच्छति पुराणज्ञस्य ये नराः}
{कम्बलाजिनवासांसि मञ्चं फलकमेव वा} %३४

\twolineshloka
{परिधानीयवस्त्राणि प्रयच्छन्ति च ये नराः}
{भूषणादि प्रयच्छन्ति वसेयुर्ब्रह्मसद्मनि} %३५

\threelineshloka
{वाचके परितुष्टे तु तुष्टाः स्युः सर्वदेवताः}
{अतः सन्तोषयेद्भक्त्या भक्तिश्रद्धान्वितः पुमान्}
{तस्य पुण्यफलं पूर्णं भवत्येव न संशयः} %३६

\twolineshloka
{यत्फलं सर्वयज्ञेषु सर्वदानेषु यत्फलम्}
{सकृत्पुराणश्रवणात्तत्फलं विन्दते नरः} %३७

\threelineshloka
{कलौ युगे विशेषेण पुराणश्रवणादृते}
{नास्ति धर्मः परः पुंसां नास्ति मुक्तिपथः परः}
{पुराणश्रवणाद्विष्णोर्नास्ति सङ्कीर्तनात्परम्} %३८

\twolineshloka
{य एतदूर्जमाहात्म्यं शृणुयाच्छ्रावयेदपि}
{स तीर्थराज बदरीगमनस्य फलं लभेत्} %३९


\onelineshloka
{सर्वरोगापहं सर्वपापनाशकरं शुभम्} %४०

\twolineshloka
{श्रुत्वा चैकपदे यो वै अगम्यागमने रतः}
{कन्यास्वस्रोर्विक्रयिणमुभयं तु विमोचयेत्} %४१

\twolineshloka
{माहात्म्यमेतदाकर्ण्य पूजयेद्यस्तु पाठकम्}
{गोभूहिरण्यवस्त्रैश्च विष्णुतुल्यो यतो हि सः} %४२

\threelineshloka
{धर्मशास्त्रं पुराणं च वेदविद्यादिकं च यत्}
{पुस्तकं वाचकायैव दातव्यं धर्ममिच्छता}
{पुराणविद्यादातारो ह्यनन्तफलभोगिनः} %४३

\twolineshloka
{इदं यः पठते भक्त्या श्रुत्वा चैवावधारयेत्}
{मुच्यते सर्वपापेभ्यो विष्णुलोकं स गच्छति} %४४


\onelineshloka
{न कस्यापीदमाख्येयं श्रद्धाहीनाय दुर्मतेः} %४५

\fourlineindentedshloka
{अपूजयित्वा गुरुमग्रबुद्ध्या}
{धर्मप्रवक्तारमनन्यबुद्धिः}
{भुक्त्वा तु भोगान्नरकेषु चैव}
{ततो हि जन्मान्तर दुःखभोगी} %४६

\twolineshloka
{तस्मात्सम्पूजयेद्भक्त्या गुरुं तत्त्वावबोधकम्}
{माहात्म्यस्य च लेशोऽयं तव चोक्तो मयाऽनघ} %४७

\twolineshloka
{न शक्यते हि सम्पूर्णं वक्तुं वर्षशतैरपि}
{पुरा कैलासशिखरे पार्वत्यै प्रोक्तवाञ्च्छिवः} %४८

\twolineshloka
{कार्तिकस्य तु माहात्म्यं यावद्वर्षशतं वदन्}
{तथापि नान्तमगमदशक्तो विरराम ह} %४९

\twolineshloka
{पुत्रार्थी च धनार्थी च राज्यार्थी स्वफलं लभेत्}
{किमत्र बहुनोक्तेन मोक्षार्थी मोक्षमाप्नुयात्} %५०


\uvacha{सूत उवाच}
\twolineshloka
{इत्युक्तो ब्रह्मणा चैव नारदः प्रेमनिर्भरः}
{भूयोभूयो नमस्कृत्य ययौ यादृच्छिको मुनिः} %५१

\twolineshloka
{कथितं शङ्करेणापि पुत्राय हितकाम्यया}
{पितुस्तद्वाक्यमाकर्ण्य षण्मुखो हर्षनिर्भरः} %५२

\twolineshloka
{कृष्णेन सत्यभामायै कार्तिकस्य च वैभवः}
{कथितस्तेन सन्तुष्टा सत्या व्रतमथाकरोत्} %५३

\twolineshloka
{ऋषयो वालखिल्येभ्यः श्रुत्वा माहात्म्यमुत्तमम्}
{ऊर्जव्रतपरा जातास्तस्मादूर्जोऽतिवल्लभः} %५४

\twolineshloka
{अधीत्य सर्वशास्त्राणि पयःसारमिवोद्धृतम्}
{नानेन सदृशं शास्त्रं विष्णुप्रीतिकरं शुभम्} %५५


\uvacha{व्यास उवाच}
\twolineshloka
{इत्युक्त्वा तानृषीन्सर्वान्सूतो वै धर्मवित्तमः}
{विरराम ततस्ते तु पूजां चक्रुस्तदास्य च} %५६

\twolineshloka
{ते पुनः स्वाश्रमं गत्वा हृष्टास्ते परमर्षयः}
{यथा सूतेनोपदिष्टं तथा चक्रुर्व्रतं शुभम्} %५७

\twolineshloka
{अनेन विधिना ये वै कुर्वन्ति कार्तिकव्रतम्}
{ते सर्वपापनिर्मुक्ता गच्छन्ति विष्णुमन्दिरम्} %५८


\iti{पुष्करिणीसंज्ञिकान्तिमतिथित्रयमाहात्म्यकथनपूर्वकपुराणश्रवणमहिमवर्णनं}{नाम षट्त्रिंशोऽध्यायः}{३६}

\endgroup
