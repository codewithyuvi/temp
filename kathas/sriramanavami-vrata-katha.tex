\sect{कथा}
\uvacha{अगस्त्य उवाच}
\twolineshloka
{रहस्यं कथयिष्यामि सुतीक्ष्ण मुनिसत्तम}
{चैत्रे नवम्यां प्राक्पक्षे दिवापुण्ये पुनर्वसौ}%॥ १ ॥

\twolineshloka
{उदये गुरुगौरांशे स्वोच्चस्थे ग्रहपञ्चके}
{मेष पूषणि सम्प्राप्ते लग्ने कर्कटकाह्वये}%॥ २ ॥

\twolineshloka
{आविरासीत्स कलया कौसल्यायां परः पुमान्}
{तस्मिन्दिने तु कर्तव्यमुपवासव्रतं सदा}%॥ ३ ॥

\twolineshloka
{तत्र जागरणं कुर्याद्रघुनाथपुरो भुवि}% भुवीतिखट्वादिव्यावृत्त्यर्थम्
{प्रतिमायां यथाशक्ति पूजा कार्या यथाविधि}%॥ ४ ॥

\twolineshloka
{प्रातर्दशम्यांस्नात्वैव कृत्वा सन्ध्यादिकाः क्रियाः}
{सम्पूज्य विधिवद् रामं भक्त्या वित्तानुसारतः}%॥ ५ ॥

\twolineshloka
{ब्राह्मणान् भोजयेत् सम्यक् दक्षिणाभिश्च तोषयेत्}
{गोभूतिलहिरण्याद्यैर्वस्त्रालङ्करणैस्तथा}%॥ ६ ॥

\twolineshloka
{रामभक्तान्प्रयत्नेन प्रीणयेत्परया मुदा}
{एवं यः कुरुते भक्त्या श्रीरामनवमीव्रतम्}%॥ ७ ॥

\twolineshloka
{अनेकजन्मासिद्धानि पापानि सुबहूनि च}
{भस्मीकृत्य व्रजत्येव तद्विष्णोः परमं पदम्}%॥ ८ ॥

\threelineshloka
{सर्वेषामप्ययं धर्मो भुक्तिमुक्त्येकसाधनः}
{अशुचिर्वाऽपि पापिष्ठः कृत्वेदं व्रतमुत्तमम्}
{पूज्यः स्यात्सर्वभूतानां यथा रामस्तथैव सः}%॥ ९ ॥

\twolineshloka
{यस्तु रामनवम्यां वै भुङ्क्ते स तु नराधमः}
{कुम्भीपाकेषु घोरेषु गच्छत्येव न संशयः}%॥ १० ॥

\twolineshloka
{अकृत्वा रामनवमीव्रतं सर्वव्रतोत्तमम्}
{व्रतान्यन्यानि कुरुते न तेषां फलभाग्भवेत्}%॥ ११ ॥

\twolineshloka
{रहस्यकृतपापानि प्रख्यातानि बहून्यपि}
{महान्ति च प्रणश्यन्ति श्रीरामनवमीव्रतात्}%॥ १२ ॥

\twolineshloka
{एकामपि नरो भक्त्या श्रीरामनवमीं मुने}
{उपोष्य कृतकृत्यः स्यात्सर्वपापैः प्रमुच्यते}%॥ १३ ॥

\twolineshloka
{नरो रामनवम्यां तु श्रीरामप्रतिमाप्रदः}
{विधानेन मुनिश्रेष्ठ स मुक्तो नात्र संशयः}%॥ १४ ॥

\uvacha{सुतीक्ष्ण उवाच}
\twolineshloka
{श्रीरामप्रतिमादानविधानं वा कथं मुने}
{कथय त्वं हि रामेऽपि भक्तस्य मम विस्तरात्}%॥ १५ ॥

\uvacha{अगस्त्य उवाच}
\onelineshloka{कथायिष्यामि तद्विद्वन् प्रतिमादानमुत्तमम्}%॥ १६ ॥

\twolineshloka
{विधानं चापि यत्नेन यतस्त्वं वैष्णवोत्तमः}
{अष्टम्यां चैत्रमासे तु शुक्लपक्षे जितेन्द्रियः}%॥ १७ ॥

\twolineshloka
{दन्तधावनपूर्वं तु प्रातः स्नायाद्यथाविधि}
{नद्यां तडागे कूपे वा ह्रदे प्रस्रवणेऽपि वा}%॥ १८ ॥

\twolineshloka
{ततः सन्ध्यादिका कार्याः संस्मरन् राघवं हृदि}
{गृहमासाद्य विप्रेन्द्र कुर्यादौपासनादिकम्}%॥ १९ ॥

\twolineshloka
{दान्तं कुटुम्बिनं विप्रं वेदशास्त्रपरं सदा}
{श्रीरामपूजानिरतं सुशीलं दम्भवर्जितम्}%॥ २० ॥

\twolineshloka
{विधिज्ञं राममन्त्राणां राममन्त्रैकसाधनम्}
{आहूय भक्त्या सम्पूज्य वृणुयात्प्रार्थयन्निति}%॥ २१ ॥

\twolineshloka
{श्रीरामप्रतिमादानं करिष्येऽहं द्विजोत्तम}
{तत्राचार्यो भव प्रीतः श्रीरामोऽसि त्वमेव च}%॥ २२ ॥

\twolineshloka
{इत्युक्त्वा पूज्य विप्रं तं स्नापयित्वा ततः परम्}
{तैलेनाभ्यज्य पयसा चिन्तयन्राघवं हृदि}%॥ २३ ॥

\twolineshloka
{श्वेताम्बरधरः श्वेतगन्धमाल्यानि धारयेत्}
{अर्चितो भूषितश्चैव कृतमाध्याह्निकक्रियः}%॥ २४ ॥

\twolineshloka
{आचार्यं भोजयेद् भक्त्या सात्त्विकान्नैः सुविस्तरम्}
{भुञ्जीत स्वयमप्येवं हृदि राममनुस्मरन्}%॥ २५ ॥

\twolineshloka
{एकभक्तव्रती तत्र सहाचार्यो जितेन्द्रियः}
{शृण्वन्रामकथां दिव्यामहःशेषं नयेन्मुने}%॥ २६ ॥

\twolineshloka
{सायं सन्ध्यादिकाः कुर्यात्क्रिया राममनुस्मरन्}
{आचार्यसहितो रात्रावधःशायी जितेन्द्रियः}%॥ २७ ॥

\twolineshloka
{वसेत्स्वयं न चैकान्ते श्रीरामार्पितमानसः}
{ततः प्रातः समुत्थाय स्नात्वा सन्ध्यां यथाविधि}%॥ २८ ॥

\twolineshloka
{प्रातः सर्वाणि कर्माणि शीघ्रमेव समापयेत्}
{ततः स्वस्थमना भूत्वा विद्वद्भिः सहितोऽनघ}%॥ २९ ॥

\twolineshloka
{स्वगृहे चोत्तरे देशे दानस्योज्ज्वलमण्डपम्}%स्वगृहे स्वगृहसमीपे॥
{चतुर्द्वारं पताकाढ्यं सवितानं सतोरणम्}%॥ ३० ॥

\twolineshloka
{मनोहरं महोत्सेधं पुष्पाद्यैः समलङ्कृतम्}
{शङ्खचक्रहनूमद्भिः प्रारद्वारे समलङ्कृतम्}%॥ ३१ ॥

\twolineshloka
{गरुत्मच्छार्ङ्गबाणैश्च दक्षिणे समलकृतम्}
{गदाखड्गाङ्गदैश्चैव पश्चिमे च विभूषितम्}%॥ ३२ ॥

\twolineshloka
{पद्मस्वस्तिकनीलैश्च कौबेर्यां समलङ्कृतम्}
{मध्यहस्तचतुष्काढ्यवेदिकायुक्तमायतम्}%॥ ३३ ॥

\twolineshloka
{प्रविश्य गीतनृत्यैश्च वाद्यैश्चापि समन्वितम्}
{पुण्याहं वाचयित्वा च विद्वद्भिः प्रीतमानसः}%॥ ३४ ॥

\twolineshloka
{ततः सङ्कल्पयेद्देवं राममेव स्मरन्मुने}
{अस्यां रामनवम्यां तु रामाराधनतत्परः}%॥ ३५ ॥

\twolineshloka
{उपोष्याष्टसु यामेषु पूजयित्वा यथाविधि}
{इमां स्वर्णमयीं रामप्रतिमां तु प्रयत्नतः}%॥ ३६ ॥

\twolineshloka
{श्रीरामप्रीतये दास्ये रामभक्ताय धीमते}
{प्रीतो रामो हरत्वाशु पापानि सुबहूनि मे}%॥ ३७ ॥

\twolineshloka
{अनेकजन्मसंसिद्धान्यभ्यस्तानि महान्ति च}
{विलिखेत्सर्वतोभद्रं वेदिकोपरि सुन्दरम्}%॥ ३८ ॥

\twolineshloka
{मध्ये तीर्थोदकैर्युक्तं पात्र संस्थाप्य चार्चितम्}
{सौवर्णे राजते ताम्रे पात्रे षट्कोणमालिखेत्}%॥ ३९ ॥

\twolineshloka
{ततः स्वर्णमयीं रामप्रतिमां पलमात्रतः}
{निर्मितां द्विभुजां रम्यां वामाङ्कस्थितजानकीम्}%॥ ४० ॥

\twolineshloka
{बिभ्रतीं दक्षिणे हस्ते ज्ञानमुद्रां महामुने}
{वामेनाधःकरेणाराद्देवीमालिंङ्ग्य संस्थिताम्}%॥ ४१ ॥

\twolineshloka
{सिंहासने राजते च पलद्वयविनिर्मिते}
{पञ्चामृतस्नानपूर्वं सम्पूज्य विधिवत्ततः}%॥ ४२ ॥

\twolineshloka
{मूलमन्त्रेण नियतो न्यासपूर्वमतन्द्रितः}
{दिवैवं विधिवत् कृत्वा रात्रौ जागरणं ततः}%॥ ४३ ॥

\twolineshloka
{दिव्यां रामकथां श्रुत्वा रामभक्तिसमन्वितः}
{गीतनृत्यादिभिश्चैव रामस्तोत्रैरनेकधा}%॥ ४४ ॥

\twolineshloka
{रामाष्टकैश्च संस्तुत्य गन्धपुष्पाक्षतादिभिः}
{कर्पूरागुरुकस्तूरीकह्लाराद्यैरनेकधा}%॥ ४५ ॥

\twolineshloka
{सम्पूज्य विधिवद् भक्त्या दिवारात्रं नयेद्बुधः}
{ततः प्रातः समुत्थाय स्नानसन्ध्यादिकाः क्रियाः}%॥ ४६ ॥

\twolineshloka
{समाप्य विधिवद्रामं पूजयेद्विधिवन्मुने}
{ततो होमं प्रकुर्वीत मूलमन्त्रेण मन्त्रवित्}%॥ ४७ ॥

\twolineshloka
{पूर्वोक्तपद्मकुण्डे वा स्थण्डिले वा समाहितः}
{लौकिकाग्नौ विधानेन शतमष्टोत्तरं मुने}%॥ ४८ ॥

\twolineshloka
{साज्येन पायसेनैव स्मरन्राममनन्यधीः}
{ततो भक्त्या सुसन्तोष्य आचार्यं पूजयेन्मुने}%॥ ४९ ॥

\twolineshloka
{कुण्डलाभ्यां सरत्नाभ्यामङ्गुलीयैरनेकधा}
{गन्धपुष्पाक्षतैर्वस्त्रैर्विचित्रैस्तु मनोहरैः}%॥ ५० ॥

\twolineshloka
{ततो रामं स्मरन्दद्यादिमं मन्त्रमुदीरयेत्}
{इमां स्वर्णमयीं रामप्रतिमां समलङ्कृताम्}%॥ ५१ ॥

\twolineshloka
{चित्रवस्त्रयुगच्छन्नरामोऽहं राघवाय ते}
{श्रीरामप्रीतये दास्ये तुष्टो भवतु राघवः}%॥ ५२ ॥

\twolineshloka
{इति दत्त्वा विधानेन दद्याद्वै दक्षिणां ध्रुवम्}
{अन्नेभ्यश्च यथाशक्त्या गोहिरण्यादि भक्तितः}%॥ ५३ ॥

\twolineshloka
{दद्याद्वासोयुगं धान्यं तथाऽलङ्करणानि च}
{एवं यः कुरुते रामप्रतिमादानमुत्तमम्}%॥ ५४ ॥

\twolineshloka
{ब्रह्महत्यादिपापेभ्यो मुच्यते नात्र संशयः}
{तुलापुरुषदानादिफलमाप्नोति सुव्रत}%॥ ५५ ॥

\twolineshloka
{अनेकजन्मसंसिद्धपापेभ्यो मुच्यते ध्रुवम्}
{बहुनाऽत्र किमुक्तेन मुक्तिस्तस्य करे स्थिता}%॥ ५६ ॥

\threelineshloka
{कुरुक्षेत्रे महापुण्ये सूर्यपर्वण्यशेषतः}
{तुलापुरुषदानाद्यैः कृतैर्यल्लभते फलम्}
{तत्फलं लभते मर्त्यो दानेनानेन सुव्रत}% ॥५७॥

\uvacha{सुतीक्ष्ण उवाच}
\twolineshloka
{प्रायेण हि नराः सर्वे दरिद्राः कृपणा मुने}
{कैः कर्तव्यं कथमिदं व्रतं ब्रूहि महामुने}%॥ ५८ ॥

\uvacha{अगस्त्य उवाच}
\onelineshloka
{दरिद्रश्च महाभाग स्वस्य वित्तानुसारतः}%॥ ५९ ॥

\twolineshloka
{पलार्धेन तदर्धेन तदर्धार्धेन वा पुनः}
{वित्तशाठ्यमकृत्वैव कुर्यादेवं व्रतं मुने}%॥ ६० ॥

\twolineshloka
{यदि घोरतरं दुष्टं पातकं नेहते क्वचित्}
{अकिञ्चनोऽपि यत्नेन उपोष्य नवमीदिने}%॥ ६१ ॥

\twolineshloka
{एकचित्तोऽपि विधिवत्सर्वपापैः प्रमुच्यते}
{प्रातःस्नानं च विधिवत्कृत्वा सन्ध्यादिकाः क्रियाः}%॥ ६२ ॥

\twolineshloka
{गोभूतिलहिरण्यादि दद्याद्वित्तानुसारतः}
{श्रीरामचन्द्रभक्तेभ्यो विद्वद्भयः श्रद्धयान्वितः}%॥ ६३ ॥

\twolineshloka
{पारणं त्वथ कुर्वीत ब्राह्मणैश्च स्वबन्धुभिः}
{एवं यः कुरुते भक्त्या सर्वपापैः प्रमुच्यते}%॥ ६४ ॥

\twolineshloka
{प्राप्ते श्रीरामनवमीदिने मर्त्यो विमूढधीः}
{उपोषणं न कुरुते कुम्भीपाकेषु पच्यते}%॥ ६५ ॥

\twolineshloka
{यत्किञ्चिद्राममुद्दिश्य क्रियते न स्वशक्तितः}
{रौरवे स तु मूढात्मा पच्यते नात्र संशयः}%॥ ६६ ॥

\uvacha{सुतीक्ष्ण उवाच}
\twolineshloka
{यामाष्टके तु पूजा वै तत्र चोक्ता महामुने}
{मूलमन्त्रेणं संयुक्ता तां कथां वद सुव्रत}%॥ ६७ ॥

\uvacha{अगस्त्य उवाच}
\twolineshloka
{सर्वेषां राममन्त्राणां मन्त्रराज षडक्षरम्} %इदं तु स्कान्दे मोक्षखण्डे श्रीरामं प्रतिरुद्रगीतायां रुद्रवाक्यम्
{मुमूर्षोर्मणिकर्ण्यान्ते अर्धोदकनिवासिनः}%॥ ६८ ॥

\twolineshloka
{अहं दिशामि ते मन्त्रं तारकस्योपदेशतः}
{श्रीराम राम रामेति एतत्तारकमुच्यते}%॥ ६९ ॥

\twolineshloka
{अतस्त्वं जानकीनाथपरं ब्रह्माभिधीयसे}
{तारकं ब्रह्म चेत्युक्तं तेन पूजा प्रशस्यते}%॥ ७० ॥

\twolineshloka
{पीठाङ्गदेवतानां तु आवृत्तीनां तथैव च}
{आदावेव प्रकुर्वीत देवस्य प्रीतमानस}%॥ ७१ ॥

\twolineshloka
{उपचारैःषोडशभिः पूजा कार्या यथाविधि}
{आवाहनं स्थापनं च सम्मुखीकरणं तथा}%॥ ७२ ॥

\twolineshloka
{एवं मुद्रां प्रार्थनां च पूजामुद्रां प्रयत्नतः}
{शङ्खपूजां प्रकुर्वीत पूर्वोक्तविधिना ततः}%॥ ७३ ॥

\twolineshloka
{कलशं वामभागे च पूजाद्रव्याणि चादरात्}
{पीठे सम्पूज्य यत्नेन आत्मानं मन्त्रमुच्चरेत्}%॥ ७४ ॥

\twolineshloka
{पात्रासादनमप्येवं कुर्याद्यामेष्वतन्द्रितः}
{पीताम्बराणि देवाय प्रार्पयन्नर्चयेत्सुधीः}%॥ ७५ ॥

\twolineshloka
{स्वर्णयज्ञोपवीतानि दद्याद्देवाय भक्तितः}
{नानारत्नविचित्राणि दद्यादाभरणानि च}%॥ ७६ ॥

\twolineshloka
{हिमाम्बुघृष्टं रुचिरं घनसारमनोहरम्}
{क्रमात्तु मूलमन्त्रेण उपचारान्प्रकल्पयेत्}%॥ ७७ ॥

\twolineshloka
{कह्लारैः केतकैर्जात्यैः पुन्नागाद्यैः प्रपूजयेत्}
{चम्पकैः शतपत्रैश्च सुगन्धैः सुमनोहरैः}%॥ ७८ ॥

\twolineshloka
{पाद्यचन्दनधूपैश्च तत्तन्मन्त्रैः प्रपूजयेत्}
{भक्ष्यभोज्यादिकं भक्त्या देवाय विधिनाऽर्पयेत्}%॥ ७९ ॥

\twolineshloka
{येन सोपस्करं देवं दत्त्वा पापैः प्रमुच्यते}
{जन्मकोटिकृतैर्घोरैर्नानारूपैश्च दारुणः}%॥ ८० ॥

\twolineshloka
{विमुक्तः स्यात्क्षणादेव राम एव भवेन्मुने}
{श्रद्दधानस्य दातव्यं श्रीरामनवमीव्रतम्}%॥ ८१ ॥

\twolineshloka
{सर्वलोकहितायेदं पवित्रं पापनाशनम्}
{लोहेन निर्मितं वाऽपि शिलया दारुणाऽपि वा}%॥ ८२ ॥

\twolineshloka
{एकेनैव प्रकारेण यस्मै कस्मै च वा मुने}
{कृतं सर्वं प्रयत्नेन यत्किञ्चिदपि भक्तितः}%॥ ८३ ॥

\twolineshloka
{जपेदेकान्तमासीनो यावत्स दशमीदिनम्}
{अनेन स्यात्पुनः पूजा दशम्यां भोजयेद् द्विजान्}%॥ ८४ ॥

\twolineshloka
{भक्त्या भोज्यैर्बहुविधैर्दद्याद् भक्त्या च दक्षिणाम्}
{कृतकृत्यो भवेत्तेन सद्यो रामः प्रसीदति}%॥ ८५ ॥

\twolineshloka
{तूष्णीं तिष्ठन्नरो वाऽपि पुनरावृत्तिवर्जितः}
{द्वादशाब्दे कृतेनापि यत्पापं चापि मुच्यते}%॥ ८६ ॥
 
\twolineshloka
{विलयं याति तत्सर्वं श्रीरामनवमीव्रतम्}
{जपं च रामनन्त्राणां यो न जानाति तस्य वै}%॥ ८७ ॥

\twolineshloka
{उपोष्य संस्मरेद्रामं न्यासपूर्वमतन्द्रितः}
{गुरोर्लब्धमिमं मन्त्रं न्यसेन्न्यासपुरःसरम्}%॥ ८८ ॥

\threelineshloka
{यामे यामे च विधिना कुर्यात्पूजां समाहितः}
{मुमुक्षुश्च सदा कुर्याच्छ्रीरामनवमीत्रतम्}
{मुच्यते सर्वपापेभ्यो याति ब्रह्म सनातनम्}%॥ ८९ ॥

॥इति श्रीस्कान्दपुराणे अगस्त्यसंहितायामगस्तिसुतीक्ष्णसंवादे रामनवमी\-व्रत\-विधिः सम्पूर्णः॥
