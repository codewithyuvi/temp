\sect{नृसिंह-जयन्ती-व्रत-कथा}

\uvacha{सूत उवाच}

\twolineshloka
{हिरण्यकशिपुं हत्वा देवदेवं जगद्गुरुम्}
{सुखासीनं च नृहरि शान्तकोपं रमापतिम्} %॥१॥

\twolineshloka
{प्रह्लादो ज्ञानिनां श्रेष्ठः पालयन् राज्यमुत्तमम्}
{एकाकी च तदुत्सङ्गे प्रियं वचनमब्रवीत्} %॥२॥

\uvacha{प्रह्लाद उवाच}

\twolineshloka
{नमस्ते भगवन्विष्णो नृसिंह-रूपिणे नमः}
{त्वद्भक्तोऽहं सुरेशैकं त्वां पृच्छामि तु तत्वतः} %॥३॥

\twolineshloka
{स्वामिस्त्वार्य ममाभिन्ना भक्तिर्जाता त्वनेकधा}
{कथं च ते प्रियो जातः कारणं मे वद प्रभो} %॥४॥

\uvacha{नृसिंह उवाच}

\twolineshloka
{कथयामि महाप्राज्ञ शृणुष्वैकापमानसः}
{भक्तेर्यकारणं वत्स प्रियत्वस्य च कारणम्} %॥५॥

\twolineshloka
{पुरा काले ह्यभूद्विप्रः किञ्चित्त्वं नाप्यधीतवान्}
{नाना त्वं वासुदेवो हि वेश्यासंसक्तमानसः} %॥६॥


\twolineshloka
{तस्मिातु न चैव त्वं चकर्थ सुकृतं कियत्}
{कृतवान्मद्वतं चैकं वेश्यासङ्गतिलालसः} %॥७॥


{मद्रतस्य प्रभावेण भक्तिर्जाता तवानघ।}

\uvacha{प्रह्लाद उवाच}

\onelineshloka
{श्रीनृसिंहोच्यतां तावत्कस्य पुत्रश्च किं व्रतम्} %॥८॥

\twolineshloka
{वेश्यायां वर्तमानेन कथं तच्च कृतं मया}
{येन त्वत्प्रीतिमापनो वक्तुमर्हसि सांप्रतम्} %॥९॥

\uvacha{नृसिंह उवाच}

\twolineshloka
{पुराऽवन्तीपुरे ह्यासीद्राह्मणो वेदपारगः}
{तस्य नाम मुशमति बहुलोकयु विश्रुतः} %॥१०॥

\twolineshloka
{नित्यहोमक्रियां चव विदधाति द्विजोत्तमः}
{ब्राह्मक्रियासु नियतं सर्वासु किल तत्परः} %॥११॥

\twolineshloka
{अग्निष्टोमादिभिर्यजरिष्टाः सर्वे सुरोत्तमाः}
{तस्य भार्या सुशीलाभूद्विख्याता भुवनत्रये} %॥१२॥

\twolineshloka
{पतिव्रता सदाचारा पतिभक्तिपरायगा}
{जजिरेस्या सुताः पञ्च तस्माविजवरात्तथा} %॥१३॥

\twolineshloka
{सदाचारेषु विद्वांसः पितृभक्तिपरायणाः}
{तेषां मध्ये कनिष्ठस्त्वं वेश्यासङ्गतितत्परः} %॥१४॥

\twolineshloka
{तया निषेध्यमानेन सुरापानं त्वया कृतम्}
{सुवर्ण चाप्यपहृतं चौरैः सार्ध त्वया बहु} %॥१५॥

\twolineshloka
{विलासिन्या समं चैव त्वया चीर्णमघं बहु}
{एकदा सद्गृहे चासीन्म किलिस्त्वया सह} %॥१६॥

\twolineshloka
{तेन कलभावेन व्रतमेतत्त्वया कृतम्}
{अज्ञानान्मतं जातं व्रतानामुत्तमं हि तत्} %॥१७॥

\twolineshloka
{तस्यां विहारयोगेन रात्रौ जागरणं कृतम्}
{वेश्याया वर्लंभं किंचित्प्रजातं न त्वया सह} %॥१८॥

\twolineshloka
{रात्री जागरणं चीर्ण त्यक्तं भोग्यमनेकधा}
{व्रतेनानेन चीणेन मोदन्ति दिवि देवताः} %॥१९॥

\twolineshloka
{सृष्टयर्थे च पुरा ब्रह्मा चक्रे ह्येत-दनुत्तमम्}
{मद्रतस्य प्रभावेण निर्मितं सचराचरम्} %॥२०॥

\twolineshloka
{ईश्वरेण पुरा चीर्ण वधार्थ त्रिपु-रस्य च}
{माहात्म्येन व्रतस्याशु त्रिपुरस्तु निपातितः} %॥२१॥

\twolineshloka
{अन्यश्च बहुभिर्देवैर्ऋषिभिश्च पुरानघ}
{राजभिश्च महाप्राज्ञैर्विदितं व्रतमुत्तमम्} %॥२२॥

\twolineshloka
{एतद्वतप्रभावेण सर्वे सिद्धिमुपामताः}
{वेश्यापि मत्प्रिया जाता त्रैलोक्ये सुखचारिणी} %॥२३॥

\twolineshloka
{ईदृशं मद्वतं वत्स त्रैलोक्ये तु सुविश्रुतम्}
{कलहेन विलासिन्या व्रतमेतदुपस्थितम्} %॥२४॥

\twolineshloka
{प्रहाद तेन ते भक्तिर्मथि जाता ह्यनुत्तमा}
{धूर्तया च विलासिन्या ज्ञात्वा व्रतदिनं मम} %॥२५॥

\twolineshloka
{कलहश्च कृतो येन मद्वतं च कृतं भवेत्}
{सा वेश्या त्वप्सरा जाता भुक्त्वा भोगाननेकशः} %॥२६॥

\twolineshloka
{मुक्ता कर्मविलीना तु त्वं प्रसाद विशस्व माम्}
{कार्यार्थ च भवानास्ते मच्छरीरपृथक्तया} %॥२७॥

\twolineshloka
{विधाय सर्वकार्याणि शीघ्रं चैव गामिष्यसि}
{इदं व्रतमवश्यं ये प्रकरिष्यन्ति मानवाः} %॥२८॥

\twolineshloka
{न तेषां पुनरावृत्तिर्मत्तः कल्पशतैरपि}
{अपुत्रो लभते पुत्रान्मद्भक्तश्च सुवर्चसः} %॥२९॥

\twolineshloka
{दरिद्रो लभते लक्ष्मी धनदस्य च यादृशी}
{तेजाकामो लभत्तेजो राज्येच्छू राज्यमुत्तमम्} %॥३०॥

\twolineshloka
{आयुःकामो लभेदायुर्यादृशं च शिवस्य हि}
{स्त्रीणां व्रतमिदं साधुपुत्रदं भाग्यदं तथा} %॥३१॥

\twolineshloka
{अवैधव्यकरं तासां पुत्रशोक-विनाशनम्}
{धनधान्यकरं चैव जातिश्रेष्ठयकरं शुभम्} %॥३२॥

\twolineshloka
{सार्वभौमसुखं तासां दिव्यं सौख्यं भवेत्ततः}
{स्त्रियो वा पुरुषाश्चापि कुर्वन्ति व्रतमुत्तमम्} %॥३३॥

\twolineshloka
{तेभ्योऽहं प्रददे सौख्यं भुक्तिमुक्ति-समन्वितम्}
{बहुनोक्तेन किं वत्स व्रतस्यास्य फलं महत्} %॥३४॥

\twolineshloka
{मद्रतस्य फलं वक्तुं नाहं शक्तो न शङ्करः}
{ब्रह्मा चतुर्भिर्ववैश्च न लभेन्महिमावधिम्} %॥३५॥

\uvacha{प्रह्लाद उवाच}
\twolineshloka
{भगवंस्त्वत्प्रसादेन श्रुतं व्रतमनुत्तमम्}
{व्रतस्यास्य फलं साधु त्वयि मे भक्तिकारणम्} %॥३६॥

\twolineshloka
{स्वामिञातं विशेषण त्वत्तः पापनिकृन्तनम्}
{अधुना श्रोतुमिच्छामि व्रतस्यास्य विधि परम्} %॥३७॥

\twolineshloka
{कस्मिन्मासे भवेदेतत्कस्मिन्वा तिथिवासरे}
{एतद्विस्तरतो देव वक्तुमर्हसि सांप्रतम्} %॥३८॥

\twolineshloka
{विधिना येन वै स्वामिन् समप्रफलभुग्भवेत्}
{ममोपरि कृपां कृत्वा ब्रूहि त्वं सकलं प्रभो} %॥३९॥

\uvacha{नृसिंह उवाच}

\twolineshloka
{साधुसाधु महाभाग व्रतस्यास्य विधिं परम्}
{सर्व कथयतो मेऽद्य त्वमेकासमनाः भ्रूण} %॥४०॥

\twolineshloka
{वैशाखशुक्लपक्षे तु चतुर्दश्यां समाचरेत्}
{मजन्मसंभवं पुण्यं व्रतं पापप्रणाशनम्} %॥४१॥


\twolineshloka
{वर्षेवर्षे तु कर्तव्यं मम संतुष्टिकारकम्}
{महापुण्यमिदं श्रेष्ठं मानुषैर्भवभीरुभिः} %॥४२॥

\twolineshloka
{तेनैव क्रियमाणेन सहस्रद्वादशीफलम्}
{जायते तद्वते वच्मि मानुषाणां महात्मनाम्} %॥४३॥

\twolineshloka
{स्वाती नक्षत्रयोगेन शनिवारेण संयुते}
{सिद्धियोगस्य संयोगे वणिजे करणे तथा} %॥४४॥

\twolineshloka
{पुण्य-सौभाग्ययोगेन लभ्यते दैवयोगतः}
{सर्वैरेतैस्तु संयुक्तं हत्याकोटिविनाशनम्} %॥४५॥

\twolineshloka
{एत-दन्यतरे योगे तदिनं पापनाशनम्}
{केवलेपि च कर्तव्यं मदिने व्रतमुत्तमम्} %॥४६॥

\twolineshloka
{अन्यथा नरकं याति यावच्चन्द्रदिवाकरौ}
{यथा यथा प्रवृत्तिः स्यात्पातकस्य कलौ युगे} %॥४७॥

\twolineshloka
{तथा तथा प्रणश्यन्ति सर्वे धर्मा न संशयः}
{एतद्वतप्रभावेण मद्भक्तिः स्यादुरात्मनाम्} %॥१८॥


\twolineshloka
{विचार्येत्थं प्रकर्तव्यं माधवे मासि तद्वतम्}
{नियमश्च प्रकर्तव्यो दन्तधावनपूर्वकम्} %॥४९॥


\twolineshloka
{श्रीनृसिंह महोग्रस्त्वं दयां कृत्वा ममोपरि}
{अद्याहं ते विधास्यामि व्रतं निर्विव्रतां नय} %॥५०॥


\twolineshloka
{इति नियममन्त्रावतस्थेन न कर्तव्या सङ्गतिः पापिभिः सह}
{मिथ्यालापो न कर्तव्यः समग्र-फलकांक्षिणा} %॥५१॥

\twolineshloka
{स्त्रीभिर्दुष्टैश्च आलापान्त्रतस्थो नैव कारयेत्}
{स्मर्तव्यं च महारूपं महिने सकलं शुभे} %॥५२॥

\twolineshloka
{ततो मध्याह्नवेलायां नद्यादौ विमले जले}
{गृहे वा देवखाते वा तडागे विमले शुभे} %॥५३॥

\twolineshloka
{वैदिकेन च मंत्रेण स्नानं कृत्वा विचक्षणः}
{मृत्तिकागोमयेनैव तथा धात्रीफलेन च} %॥५४॥

\twolineshloka
{तिलैश्च सर्वपापन्नः स्नानं कृत्वा महात्मभिः}
{परिधाय शुचिर्वासो नित्यकर्म समाचरेत्} %॥५५॥

\twolineshloka
{ततो गृहं समागत्य स्मरन् मां भक्तियोगतः}
{गोमयेन प्रलिप्याथ कुर्यादष्टदलं शुभम्} %॥५६॥

\twolineshloka
{कलशं तत्र संस्थाप्य तानं रत्नसमन्वितम्}
{तस्योपरि न्यसेत् पात्रं वंशजं व्रीहिपूरितम्} %॥५७॥

\twolineshloka
{हैमी तत्र च मन्मूर्तिः स्थाप्या लक्षम्यास्तथैव च}
{पलेन वा तदर्धेन तदर्धाधन वा पुनः} %॥५८॥

\twolineshloka
{यथाशक्त्याथवा कार्या वित्तशाठयविवर्जितैः}
{पञ्चामृतेन संस्नाप्य पूजनं तु समाचरेत्} %॥५९॥

\twolineshloka
{ततो ब्राह्मणमाहूय तमाचार्यमलोलुपम्}
{सदाचारसमायुक्तं शान्तं दान्तं जितेन्द्रियम्} %॥६०॥

\twolineshloka
{आचार्यवचनाद्धीमान् पूजां कुर्याद्यथाविधि}
{मण्डपं कारयेत्तत्र पुष्पस्तबकशोभितम्} %॥६१॥

\threelineshloka
{ऋतुकालोद्भवैः पुष्पैः पूजयेत्स्वस्थमानसः}
{उपचारः षोडेशभिमत्रैर्वेदोद्भवैस्तथा} %॥६२॥
{शुभैः पौराणिकर्मन्त्रैः पूजनीयो यथाविधि}

\twolineshloka
{चन्दनं शीतलं दिव्यं चन्द्रकुङ्कुममिश्रितम्}
{ददामि तव तुष्टयर्य नृसिंह परमेश्वर} %॥६३॥

चन्दनम्।
\twolineshloka
{कालोद्भवानि पुष्पा,ण तुलस्यादीनि वै प्रमो}
{सम्यक गृहाण देवेश लक्ष्म्या सह नमोस्तु ते} %॥६४॥

पुष्पाणि।
\twolineshloka
{कृष्णागुरुमयं धूपं श्रीनृसिंह जगत्पते}
{तब तुष्टचे प्रदास्यामि सर्वदेव नमोस्तु ते} %॥६५॥

धूपम्।
\twolineshloka
{सर्वतेजोद्भव तेजस्तस्मादी ददामि ते}
{श्रीनृसिंह महाबाहो तिमिरं मे विनाशय} %॥६६॥

दीपम्।
\twolineshloka
{नैवेद्यं सौख्यदं चारु भक्ष्यभोज्यसमन्वितम्}
{ददामि ते रमाकान्त सर्वपापक्षयं कुरु} %॥६७॥

नैवेद्यम्।
\twolineshloka
{नृसिंहाच्युत देवेश लक्ष्मीकान्त जगत्पते}
{अनेनार्यप्रदानेन सफलाः स्युर्मनोरथाः} %॥६८॥

अर्घ्यम्।
\twolineshloka
{पीताम्बर महाबाहो प्रह्लादभयनाशन}
{यथाभूतेनार्चनेन यथोक्तफलदो भव} %॥६९॥

इति प्रार्थना॥
\twolineshloka
{रात्री जागरणं कार्य गीतवादित्रनिःस्वनैः}
{पुराणश्रवणायैश्च श्रोतव्याश्च कथाः शुभाः} %॥७०॥

\twolineshloka
{ततः प्रभातसमये स्नान कृत्वा जितन्द्रियः}
{पूर्वोक्तेन विधानेन पूजयेन्मां प्रयत्नतः} %॥७१॥

\twolineshloka
{वैष्णवान्प्रजपेन्मंत्रान् मदने स्वस्थमानसः}
{ततो दानानि देयानि वक्ष्यमाणानि चानघ} %॥७२॥

\twolineshloka
{पात्रेभ्यस्तु द्विजेभ्यो हि लोकद्वयजिगीषया}
{सिंहः स्वर्णमयो देयो मम सन्तोषकारकः} %॥७३॥

\twolineshloka
{गोभूतिलहिरण्यानि दयानि च फलेप्सुभिः}
{शय्या सतूलिका देया सप्तधान्यसन्वितः} %॥७४॥

\twolineshloka
{अन्यानि च यथाशक्त्या देयानि मम तुष्टये}
{वित्तशाठयं न कुर्वीत यथोक्तफलकांक्षया} %॥७५॥

\twolineshloka
{ब्राह्मणाभोजयेद्भक्त्या तेभ्यो दद्याञ्च दक्षिणाम्}
{निर्धनेनापि कर्तव्यं देय शक्त्यनुसारतः} %॥७६॥


\twolineshloka
{सर्वेषामेव वर्णानामधिकारोऽस्ति मदते}
{मद्भक्तैस्तु विशेषेण कर्तव्य मत्परायणैः} %॥७७॥


\twolineshloka
{तवशे न भवेःदुःखं न दोषो मत्प्रसादतः}
{मद्वंशे ये नरा जाता ये निष्पत्तिपरायणाः} %॥७८॥


\twolineshloka
{तान् समुद्धर देवेश दुस्तराद्भवसागरात्}
{पातकार्णवमग्नस्य व्याधिदुःखाम्बुवासिभिः} %॥७९॥


\twolineshloka
{जीवैस्तु परिभूतस्य मोहदुःखगतस्य मे}
{करावलम्बनं देहि शेषशायिञ्जगत्पते} %॥८०॥


\twolineshloka
{श्रीनृसिंह रमाकान्त भक्तानां भयनाशन}
{क्षीराम्बुनिधिवासिंस्त्वं चक्रपाणे जनार्दन} %॥८१॥


\twolineshloka
{व्रतेनानेन देवेश भुक्तिमुक्तिप्रदो भव}
{एवं प्रार्थ्य ततो देवं विसृज्य च यथाविधि} %॥८२॥

\threelineshloka
{उपहारादिकं सर्वमाचार्याय निवेदयेत्}
{दक्षिणाभिस्तु संतोष्य ब्राह्मणांस्तु विसजयेत्} %॥८३॥
{मध्याद्दे तु सुसंयत्तो भुञ्जीत सह बन्धुभिः॥}

\twolineshloka
{य इदं शृणुयाद्भक्त्या व्रतं पापप्रणाशनम्}
{तस्य श्रवणमात्रेण ब्रह्महत्या व्यपोहति} %॥८४॥

\twolineshloka
{पवित्रं परमं गुह्यं कीर्तयेद्यस्तु मानवः}
{सर्वान् कामानवाप्नोति व्रतस्यास्य फलं लभेत्}

इति हेमाद्रौ नृसिंहपुराणे नृसिंहचतुर्दशीव्रतकथा समाप्ता॥

