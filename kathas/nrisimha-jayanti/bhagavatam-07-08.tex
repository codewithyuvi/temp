\sect{श्रीमद्भागवते महापुराणे सप्तमस्कन्धे अष्टमोऽध्यायः}

\uvacha{श्रीनारद उवाच}


\twolineshloka
{अथ दैत्यसुताः सर्वे श्रुत्वा तदनुवर्णितम्}
{जगृहुर्निरवद्यत्वान्नैव गुर्वनुशिक्षितम्} %1

\twolineshloka
{अथाचार्यसुतस्तेषां बुद्धिमेकान्तसंस्थिताम्}
{आलक्ष्य भीतस्त्वरितो राज्ञ आवेदयद्यथा} %2

\twolineshloka
{श्रुत्वा तदप्रियं दैत्यो दुःसहं तनयानयम्}
{कोपावेशचलद्गात्रः पुत्रं हन्तुं मनो दधे} %3

\twolineshloka
{क्षिप्त्वा परुषया वाचा प्रह्रादमतदर्हणम्}
{आहेक्षमाणः पापेन तिरश्चीनेन चक्षुषा} %4

\twolineshloka
{प्रश्रयावनतं दान्तं बद्धाञ्जलिमवस्थितम्}
{सर्पः पदाहत इव श्वसन्प्रकृतिदारुणः} %5

\uvacha{श्रीहिरण्यकशिपुरुवाच}


\twolineshloka
{हे दुर्विनीत मन्दात्मन्कुलभेदकराधम}
{स्तब्धं मच्छासनोद्वृत्तं नेष्ये त्वाद्य यमक्षयम्} %6

\twolineshloka
{क्रुद्धस्य यस्य कम्पन्ते त्रयो लोकाः सहेश्वराः}
{तस्य मेऽभीतवन्मूढ शासनं किं बलोऽत्यगाः} %7

\uvacha{श्रीप्रह्राद उवाच}


\twolineshloka
{न केवलं मे भवतश्च राजन्स वै बलं बलिनां चापरेषाम्}
{परेऽवरेऽमी स्थिरजङ्गमा ये ब्रह्मादयो येन वशं प्रणीताः} %8

\twolineshloka
{स ईश्वरः काल उरुक्रमोऽसावोजः सहः सत्त्वबलेन्द्रियात्मा}
{स एव विश्वं परमः स्वशक्तिभिः सृजत्यवत्यत्ति गुणत्रयेशः} %9

\twolineshloka
{जह्यासुरं भावमिमं त्वमात्मनः समं मनो धत्स्व न सन्ति विद्विषः}
{ऋतेऽजितादात्मन उत्पथे स्थितात्तद्धि ह्यनन्तस्य महत्समर्हणम्} %10

\twolineshloka
{दस्यून्पुरा षण्न विजित्य लुम्पतो मन्यन्त एके स्वजिता दिशो दश}
{जितात्मनो ज्ञस्य समस्य देहिनां साधोः स्वमोहप्रभवाः कुतः परे} %11

\uvacha{श्रीहिरण्यकशिपुरुवाच}


\twolineshloka
{व्यक्तं त्वं मर्तुकामोऽसि योऽतिमात्रं विकत्थसे}
{मुमूर्षूणां हि मन्दात्मन्ननु स्युर्विक्लवा गिरः} %12

\twolineshloka
{यस्त्वया मन्दभाग्योक्तो मदन्यो जगदीश्वरः}
{क्वासौ यदि स सर्वत्र कस्मात्स्तम्भे न दृश्यते} %13

\twolineshloka
{सोऽहं विकत्थमानस्य शिरः कायाद्धरामि ते}
{गोपायेत हरिस्त्वाद्य यस्ते शरणमीप्सितम्} %14

\twolineshloka
{एवं दुरुक्तैर्मुहुरर्दयन्रुषा सुतं महाभागवतं महासुरः}
{खड्गं प्रगृह्योत्पतितो वरासनात्स्तम्भं तताडातिबलः स्वमुष्टिना} %15

\twolineshloka
{तदैव तस्मिन्निनदोऽतिभीषणो बभूव येनाण्डकटाहमस्फुटत्}
{यं वै स्वधिष्ण्योपगतं त्वजादयः श्रुत्वा स्वधामात्ययमङ्ग मेनिरे} %16

\twolineshloka
{स विक्रमन्पुत्रवधेप्सुरोजसा निशम्य निर्ह्रादमपूर्वमद्भुतम्}
{अन्तःसभायां न ददर्श तत्पदं वितत्रसुर्येन सुरारियूथपाः} %17

\twolineshloka
{सत्यं विधातुं निजभृत्यभाषितं व्याप्तिं च भूतेष्वखिलेषु चात्मनः}
{अदृश्यतात्यद्भुतरूपमुद्वहन्स्तम्भे सभायां न मृगं न मानुषम्} %18

\twolineshloka
{स सत्त्वमेनं परितो विपश्यन्स्तम्भस्य मध्यादनुनिर्जिहानम्}
{नायं मृगो नापि नरो विचित्रमहो किमेतन्नृमृगेन्द्ररूपम्} %19

\twolineshloka
{मीमांसमानस्य समुत्थितोऽग्रतो नृसिंहरूपस्तदलं भयानकम्}
{प्रतप्तचामीकरचण्डलोचनं स्फुरत्सटाकेशरजृम्भिताननम्} %20

\twolineshloka
{करालदंष्ट्रं करवालचञ्चल क्षुरान्तजिह्वं भ्रुकुटीमुखोल्बणम्}
{स्तब्धोर्ध्वकर्णं गिरिकन्दराद्भुत व्यात्तास्यनासं हनुभेदभीषणम्} %21

\twolineshloka
{दिविस्पृशत्कायमदीर्घपीवर ग्रीवोरुवक्षःस्थलमल्पमध्यमम्}
{चन्द्रांशुगौरैश्छुरितं तनूरुहैर्विष्वग्भुजानीकशतं नखायुधम्} %22

\twolineshloka
{दुरासदं सर्वनिजेतरायुध प्रवेकविद्रावितदैत्यदानवम्}
{प्रायेण मेऽयं हरिणोरुमायिना वधः स्मृतोऽनेन समुद्यतेन किम्} %23

\twolineshloka
{एवं ब्रुवंस्त्वभ्यपतद्गदायुधो नदन्नृसिंहं प्रति दैत्यकुञ्जरः}
{अलक्षितोऽग्नौ पतितः पतङ्गमो यथा नृसिंहौजसि सोऽसुरस्तदा} %24

\twolineshloka
{न तद्विचित्रं खलु सत्त्वधामनि स्वतेजसा यो नु पुरापिबत्तमः}
{ततोऽभिपद्याभ्यहनन्महासुरो रुषा नृसिंहं गदयोरुवेगया} %25

\twolineshloka
{तं विक्रमन्तं सगदं गदाधरो महोरगं तार्क्ष्यसुतो यथाग्रहीत्}
{स तस्य हस्तोत्कलितस्तदासुरो विक्रीडतो यद्वदहिर्गरुत्मतः} %26

\threelineshloka
{असाध्वमन्यन्त हृतौकसोऽमरा घनच्छदा भारत सर्वधिष्ण्यपाः}
{तं मन्यमानो निजवीर्यशङ्कितं यद्धस्तमुक्तो नृहरिं महासुरः}
{पुनस्तमासज्जत खड्गचर्मणी प्रगृह्य वेगेन गतश्रमो मृधे} % ॥२७॥\\

\twolineshloka
{तं श्येनवेगं शतचन्द्रवर्त्मभिश्चरन्तमच्छिद्रमुपर्यधो हरिः}
{कृत्वाट्टहासं खरमुत्स्वनोल्बणं निमीलिताक्षं जगृहे महाजवः} %28

\twolineshloka
{विष्वक्स्फुरन्तं ग्रहणातुरं हरिर्व्यालो यथाखुं कुलिशाक्षतत्वचम्}
{द्वार्यूरुमापत्य ददार लीलया नखैर्यथाहिं गरुडो महाविषम्} %29

\twolineshloka
{संरम्भदुष्प्रेक्ष्यकराललोचनो व्यात्ताननान्तं विलिहन्स्वजिह्वया}
{असृग्लवाक्तारुणकेशराननो यथान्त्रमाली द्विपहत्यया हरिः} %30

\twolineshloka
{नखाङ्कुरोत्पाटितहृत्सरोरुहं विसृज्य तस्यानुचरानुदायुधान्}
{अहन्समस्तान्नखशस्त्रपाणिभिर्दोर्दण्डयूथोऽनुपथान्सहस्रशः} %31

\twolineshloka
{सटावधूता जलदाः परापतन्ग्रहाश्च तद्दृष्टिविमुष्टरोचिषः}
{अम्भोधयः श्वासहता विचुक्षुभुर्निर्ह्रादभीता दिगिभा विचुक्रुशुः} %32

\twolineshloka
{द्यौस्तत्सटोत्क्षिप्तविमानसङ्कुला प्रोत्सर्पत क्ष्मा च पदाभिपीडिता}
{शैलाः समुत्पेतुरमुष्य रंहसा तत्तेजसा खं ककुभो न रेजिरे} %33

\twolineshloka
{ततः सभायामुपविष्टमुत्तमे नृपासने सम्भृततेजसं विभुम्}
{अलक्षितद्वैरथमत्यमर्षणं प्रचण्डवक्त्रं न बभाज कश्चन} %34

\twolineshloka
{निशाम्य लोकत्रयमस्तकज्वरं तमादिदैत्यं हरिणा हतं मृधे}
{प्रहर्षवेगोत्कलितानना मुहुः प्रसूनवर्षैर्ववृषुः सुरस्त्रियः} %35

\twolineshloka
{तदा विमानावलिभिर्नभस्तलं दिदृक्षतां सङ्कुलमास नाकिनाम्}
{सुरानका दुन्दुभयोऽथ जघ्निरे गन्धर्वमुख्या ननृतुर्जगुः स्त्रियः} %36

\twolineshloka
{तत्रोपव्रज्य विबुधा ब्रह्मेन्द्रगिरिशादयः}
{ऋषयः पितरः सिद्धा विद्याधरमहोरगाः} %37

\twolineshloka
{मनवः प्रजानां पतयो गन्धर्वाप्सरचारणाः}
{यक्षाः किम्पुरुषास्तात वेतालाः सहकिन्नराः} %38

\threelineshloka
{ते विष्णुपार्षदाः सर्वे सुनन्दकुमुदादयः}
{मूर्ध्नि बद्धाञ्जलिपुटा आसीनं तीव्रतेजसम्}
{ईडिरे नरशार्दुलं नातिदूरचराः पृथक्} % 39


\uvacha{श्रीब्रह्मोवाच}


\twolineshloka
{नतोऽस्म्यनन्ताय दुरन्तशक्तये विचित्रवीर्याय पवित्रकर्मणे}
{विश्वस्य सर्गस्थितिसंयमान्गुणैः स्वलीलया सन्दधतेऽव्ययात्मने} %40

\uvacha{श्रीरुद्र उवाच}


\twolineshloka
{कोपकालो युगान्तस्ते हतोऽयमसुरोऽल्पकः}
{तत्सुतं पाह्युपसृतं भक्तं ते भक्तवत्सल} %41

\uvacha{श्रीइन्द्र उवाच}


\fourlineindentedshloka
{प्रत्यानीताः परम भवता त्रायता नः स्वभागा}
{दैत्याक्रान्तं हृदयकमलं तद्गृहं प्रत्यबोधि}
{कालग्रस्तं कियदिदमहो नाथ शुश्रूषतां ते}
{मुक्तिस्तेषां न हि बहुमता नारसिंहापरैः किम्} %42

\uvacha{श्रीऋषय ऊचुः}


\fourlineindentedshloka
{त्वं नस्तपः परममात्थ यदात्मतेजो}
{येनेदमादिपुरुषात्मगतं ससर्क्थ}
{तद्विप्रलुप्तममुनाद्य शरण्यपाल}
{रक्षागृहीतवपुषा पुनरन्वमंस्थाः} %43

\uvacha{श्रीपितर ऊचुः}


\fourlineindentedshloka
{श्राद्धानि नोऽधिबुभुजे प्रसभं तनूजैर्}
{दत्तानि तीर्थसमयेऽप्यपिबत्तिलाम्बु}
{तस्योदरान्नखविदीर्णवपाद्य आर्च्छत्}
{तस्मै नमो नृहरयेऽखिलधर्मगोप्त्रे} %44

\uvacha{श्रीसिद्धा ऊचुः}


\twolineshloka
{यो नो गतिं योगसिद्धामसाधुरहार्षीद्योगतपोबलेन}
{नाना दर्पं तं नखैर्विददार तस्मै तुभ्यं प्रणताः स्मो नृसिंह} %45

\uvacha{श्रीविद्याधरा ऊचुः}


\twolineshloka
{विद्यां पृथग्धारणयानुराद्धां न्यषेधदज्ञो बलवीर्यदृप्तः}
{स येन सङ्ख्ये पशुवद्धतस्तं मायानृसिंहं प्रणताः स्म नित्यम्} %46

\uvacha{श्रीनागा ऊचुः}


\twolineshloka
{येन पापेन रत्नानि स्त्रीरत्नानि हृतानि नः}
{तद्वक्षःपाटनेनासां दत्तानन्द नमोऽस्तु ते} %47

\uvacha{श्रीमनव ऊचुः}


\twolineshloka
{मनवो वयं तव निदेशकारिणो दितिजेन देव परिभूतसेतवः}
{भवता खलः स उपसंहृतः प्रभो करवाम ते किमनुशाधि किङ्करान्} %48

\uvacha{श्रीप्रजापतय ऊचुः}


\twolineshloka
{प्रजेशा वयं ते परेशाभिसृष्टा न येन प्रजा वै सृजामो निषिद्धाः}
{स एष त्वया भिन्नवक्षा नु शेते जगन्मङ्गलं सत्त्वमूर्तेऽवतारः} %49

\uvacha{श्रीगन्धर्वा ऊचुः}


\twolineshloka
{वयं विभो ते नटनाट्यगायका येनात्मसाद्वीर्यबलौजसा कृताः}
{स एष नीतो भवता दशामिमां किमुत्पथस्थः कुशलाय कल्पते} %50

\uvacha{श्रीचारणा ऊचुः}


\twolineshloka
{हरे तवाङ्घ्रिपङ्कजं भवापवर्गमाश्रिताः}
{यदेष साधुहृच्छयस्त्वयासुरः समापितः} %51

\uvacha{श्रीयक्षा ऊचुः}


\fourlineindentedshloka
{वयमनुचरमुख्याः कर्मभिस्ते मनोज्ञैस्}
{त इह दितिसुतेन प्रापिता वाहकत्वम्}
{स तु जनपरितापं तत्कृतं जानता ते}
{नरहर उपनीतः पञ्चतां पञ्चविंश} %52

\uvacha{श्रीकिम्पुरुषा ऊचुः}


\twolineshloka
{वयं किम्पुरुषास्त्वं तु महापुरुष ईश्वरः}
{अयं कुपुरुषो नष्टो धिक्कृतः साधुभिर्यदा} %53

\uvacha{श्रीवैतालिका ऊचुः}


\twolineshloka
{सभासु सत्रेषु तवामलं यशो गीत्वा सपर्यां महतीं लभामहे}
{यस्तामनैषीद्वशमेष दुर्जनो द्विष्ट्या हतस्ते भगवन्यथामयः} %54

\uvacha{श्रीकिन्नरा ऊचुः}


\twolineshloka
{वयमीश किन्नरगणास्तवानुगा दितिजेन विष्टिममुनानुकारिताः}
{भवता हरे स वृजिनोऽवसादितो नरसिंह नाथ विभवाय नो भव} %55

\uvacha{श्रीविष्णुपार्षदा ऊचुः}

\twolineshloka
{अद्यैतद्धरिनररूपमद्भुतं ते दृष्टं नः शरणद सर्वलोकशर्म}
{सोऽयं ते विधिकर ईश विप्रशप्तस्तस्येदं निधनमनुग्रहाय विद्मः} %॥५६॥\\


॥इति श्रीमद्भागवते महापुराणे पारमहंस्यां संहितायां सप्तमस्कन्धे अष्टमोऽध्यायः॥

