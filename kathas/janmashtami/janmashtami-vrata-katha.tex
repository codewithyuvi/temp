% !TeX program = XeLaTeX
% !TeX root = ../../pujavidhanam.tex
\begingroup
\setlength{\parindent}{0pt}
\sect{जन्माष्टमी-व्रत-कथा}
\uvacha{युधिष्ठिर उवाच}

\twolineshloka
{जन्माष्टमीव्रतं ब्रूहि विस्तरेण ममाच्युत}
{कस्मिन्काले समुत्पन्नं किं पुण्यं को विधिः स्मृतः}%॥१॥

\uvacha{श्रीकृष्ण उवाच}
\twolineshloka
{मल्लयुद्धे परावृत्ते शमिते कुकुरान्धके}
{स्वजनैर्बन्धुभिः स्त्रीभिः समः स्निग्धैः समावृते}%॥२॥

\twolineshloka
{हते कंसासुरे दुष्टे मथुरायां युधिष्ठिर}
{देवकी मां परिष्वज्य कृत्वोत्सङ्गे रुरोद ह}%॥३॥

\twolineshloka
{वसुदेवोऽपि तत्रैव वात्सल्यात्प्ररुरोद ह}
{समालिङ्ग्याश्रुवदनः पुत्र पुत्रेत्युवाच ह}%॥४॥

\twolineshloka
{सगद्गदस्वरो दीनो बाष्पपर्याकुलेक्षणः}
{बलभद्रं च मां चैव परिष्वज्य मुदा पुनः}%॥५॥

\twolineshloka
{अद्य मे सफलं जन्म जीवितं च सुजीवितम्}
{उभाभ्यामद्य पुत्राभ्यां समुद्भूतः समागमः}%॥६॥

\twolineshloka
{एवं हर्षेण दाम्पत्यं हृष्टं पुष्टं तदा ह्यभूत्}
{प्रणिपत्य जनाः सर्वे बभूवुस्ते प्रहर्षिताः}%॥७॥

\onelineshloka*
{एवं महोत्सवं दृष्ट्वा मामूचुर्मधुसूदनम्}
\uvacha{जना ऊचुः}
\onelineshloka
{प्रसादः क्रियतामस्य लोकस्याऽऽर्तस्य दुःखहन्}%॥८॥

\twolineshloka
{यस्मिन्दिने च प्रासूत देवकी त्वां जनार्दन}
{तद्दिनं देहि वैकुण्ठ कुर्मस्तत्र महोत्सवम्}%॥९॥

\twolineshloka
{एवं स्तुतो जनौघेन वासुदेवो मयेक्षितः}
{विलोक्य बलभद्रं च मां च हृष्टतनूरुहः}%॥१०॥

\twolineshloka
{उवाच स ममादेशाल्लोकाञ्जन्माष्टमीव्रतम्}
{मथुरायां ततः पश्चात् पार्थ सम्यक् प्रकाशितम्}%॥११॥

\twolineshloka
{कुर्वन्तु ब्राह्मणाः सर्वे व्रतं जन्माष्टमीदिने}
{क्षत्रिया वैश्यजातीयाः शूद्रा येऽन्येऽपि धर्मिणः}%॥१२॥

\uvacha{युधिष्ठिर उवाच}
\twolineshloka
{कीदृशं तद्व्रतं देवदेव सर्वैरनुष्ठितम्}
{जन्माष्टमीति संज्ञं च पवित्रं पापनाशनम्}%॥१३॥

\twolineshloka
{येन त्वं तुष्टिमायासि कार्त्स्न्येन प्रभवाव्यय}
{एतन्मे तत्त्वतो ब्रूहि सविधानं सविस्तरम्}%॥१४॥

\uvacha{श्रीकृष्ण उवाच}
\twolineshloka
{मासि भाद्रपदेष्टम्यां निशीथे कृष्णपक्षके}
{शशाङ्के वृषराशिस्थे ऋक्षे रोहिणीसंज्ञके}%॥१५॥

\twolineshloka
{योगेऽस्मिन्वसुदेवाद्धि देवकी मामजीजनत्}
{भगवत्याश्च तत्रैव क्रियते सुमहोत्सवः}%॥१६॥

\twolineshloka
{योगेऽस्मिन्कथितेऽष्टम्यां सिंहराशिगते रवौ}
{सप्तम्यां लघुभुक् कुर्याद्दन्तधावनपूर्वकम्}%॥१७॥

\twolineshloka
{उपवासस्य नियमं रात्रौ स्वप्याज्जितेन्द्रियः}
{केवलेनोपवासेन तस्मिञ्जन्मदिने मम}%॥१८॥

\twolineshloka
{सप्तजन्मकृतात्पापान्मुच्यते नात्र संशयः}
{उपावृत्तस्य पापेभ्यो यस्तु वासोगुणैः सह}%॥१९॥

\twolineshloka
{उपवासः स विज्ञेयः सर्वभोगविवर्जितः}
{ततोऽष्टम्यां तिलैः स्नात्वा नद्यादौ विमले जले}%॥२०॥

\twolineshloka
{सुदेशे शोभनं कुर्याद्देवक्याः सूतिकागृहम्}
{सितपीतैस्तथा रक्तैः कर्बुरैरितरैरपि}%॥२१॥

\twolineshloka
{वासोभिः शोभितं कृत्वा समन्तात्कलशैर्नवैः}
{पुष्पैः फलैरनेकैश्च दीपालिभिरितस्ततः}%॥२२॥

\twolineshloka
{पुष्पमालाविचित्रं च चन्दनागुरुधूपितम्}
{अतिरम्यमनौपम्यं रक्षामणिविभूषितम्}%॥२३॥

\twolineshloka
{हरिवंशस्य चरितं गोकुलं च विलेखयेत्}
{ततो वादिबनिनदेर्वीणावेणुरवाकुलम्}%॥२४॥

\twolineshloka
{नृत्यगीतक्रमोपेतं मङ्गलैश्च समन्ततः}
{वेष्टकारीं लोहखड्गं कृष्णछागं च यत्नतः}%॥२५॥

\twolineshloka
{द्वारे विन्यस्य मुसलं रक्षितं रक्षपालकैः}
{षष्ठया देव्याधिष्ठितं च तद्गृहं चोत्सवैस्तथा}%॥२६॥

\twolineshloka
{एवंविभवसारेण कृत्वा तत्सूतिकागृहम्}
{तन्मध्ये प्रतिमा स्थाप्या सा चाप्यष्टविधा स्मृता}%॥२७॥

\twolineshloka
{काञ्चनी राजती ताम्री पैत्तली मृन्मयी तथा}
{वार्क्षी मणिमयी चैव वर्णकैर्लिखिता तथा}%॥२८॥

\twolineshloka
{सर्वलक्षणसम्पूर्णा पर्यङ्के चाष्टशल्यके}
{प्रतप्तकाञ्चनाभासां महार्हां सुतपस्विनीम्}%॥२९॥

\twolineshloka
{प्रसूतां च प्रसुप्तां च स्थापयन्मञ्चकोपरि}
{मां तत्र बालकं सुप्तं पर्यङ्के स्तनपायिनम्}%॥३०॥

\twolineshloka
{श्रीवत्सवक्षसं शान्तं नीलोत्पलदलच्छविम्}
{यशोदा तत्र चैकस्मिन् प्रदेशे सूतिकागृहे}%॥३१॥

\twolineshloka
{तद्वच्च कल्पयेत् पार्थ प्रसूतां वरकन्यकाम्}
{तथैव मम पार्श्वस्थाः कृताञ्जलिपुटा नृप}%॥३२॥

\twolineshloka
{देवा ग्रहास्तथा नागा यक्षविद्याधराभराः}
{प्रणताः पुष्पमालाग्रचारुहस्ताः सुरासुराः}%॥३३॥

\twolineshloka
{सञ्चरन्त इवाऽऽकाशे प्रहारैरुदितोदितैः}
{वसुदेवोऽपि तत्रैव खड्गचर्मधरः स्थितः}%॥३४॥

\twolineshloka
{कश्यपो वसुदेवोऽयमदितिश्चैव देवकी}
{शेगे वै बलदेवोऽयं यशोदादितिरन्वभूत्}%॥३५॥

\twolineshloka
{नन्दः प्रजापतिर्दक्षो गर्गश्चापि चतुर्मुखः}
{गोप्यश्चाप्सरसश्चैव गोपाश्चापि दिवौकसः}%॥३६॥

\twolineshloka
{एषोऽवतारो राजेन्द्र कंसोऽयं कालनेमिजः}
{तत्र कंसनियुक्ताश्च मोहिता योगनिद्रया}%॥३७॥

\twolineshloka
{गोधेनुकुञ्जराश्चैव दानवाः शस्त्रपाणयः}
{नृत्यतश्चाप्सरोभिस्ते गन्धर्वा गीततत्पराः}%॥३८॥

\twolineshloka
{लेखनीयश्च तत्रैव कालियो यमुनाह्रदे}
{इत्येवमादि यत्किञ्चिद्विद्यते चरितं मम}%॥३९॥

\twolineshloka
{लेखयित्वा प्रयत्नेन पूजयेद्भक्तितत्परः}
{रम्यमेवं बीजपूरैः पुष्पमालादिशोभितम्}%॥४०॥

\threelineshloka
{कालदेशोद्भवैः पुष्पैः फलैश्चापि युधिष्ठिर}
{पाद्यार्घ्यैः पूजयेद्भक्त्या गन्धपुष्पाक्षतैः सह}
{मन्त्रेणानेन कौन्तेय देवकीं पूजयेन्नरः}%॥४१॥

\fourlineindentedshloka
{गायद्भिः किन्नराद्यैः सततपरिवृता वेणुवीणानिनादैः}
{भृङ्गारादर्शकुम्भप्रवरवृतकरैः किङ्करैः सेव्यमाना}
{पर्यङ्के स्वास्तृते यामुदिततरमुखी पुत्रिणी सम्यगास्ते}
{सा देवी देवमाता जयतु च ससुता देवकी कान्तरूपा}%॥४२॥


\twolineshloka
{पादावभ्यञ्जयन्ती श्रीदेवक्याश्चरणान्तिके}
{निषण्णा पङ्कजे पूज्या दिव्यगन्धातुलेपनैः}%॥४३॥


\twolineshloka
{पङ्कजैः पूजयेद्देवीं नमो देव्यै श्रिया इति}
{देववत्से नमस्तेऽस्तु कृष्णोत्पादनतत्परा}%॥४४॥


\twolineshloka
{पापक्षयकरा देवी तुष्टिं यातु मयाऽर्चिता}
{प्रणवादिनमोऽन्तं च पृथङ्नामानुकीर्तनम्}%॥४५॥


\twolineshloka
{कुर्यात्पूजा विधिज्ञश्च सर्वपापापनुत्तये}
{देवक्यै वसुदेवाय वासुदेवाय चैव हि}%॥४६॥


\twolineshloka
{बलदेवाय नन्दाय यशोदायै पृथक् पृथकू}
{क्षीरादिस्नपनं कृत्वा चन्दनेनानुलेपयेत्}%॥४७॥


\twolineshloka
{विध्यन्तरमपीच्छन्ति केचिदत्रैव सूरयः}
{चन्द्रोदये शशाङ्काय अर्घ्यं दत्त्वा हरिं स्मरन्}%॥४८॥


\twolineshloka
{अनघं वामनं शौरिं वैकुण्ठं पुरुषोत्तमम्}
{वासुदेवं हृषीकेशं माधवं मधुसूदनम्}%॥४९॥


\twolineshloka
{वराहं पुण्डरीकाक्षं नृसिंहं ब्रह्मणः प्रियम्}
{समस्तस्यापि जगतः सृष्टिस्थित्यन्तकारकम्}%॥५०॥


\twolineshloka
{अनादिनिधनं विष्णुः त्रैलोक्येशं त्रिविक्रमम्}
{नारायणं चतुर्बाहुं शङ्खचक्रगदाधरम्}%॥५१॥


\twolineshloka
{पीताम्बरधरं नित्यं वनमालाविभूषितम्}
{श्रीवत्साक्षं जगत्सेतुं श्रीपतिं श्रीधरं हरिम्}%॥५२॥


\twolineshloka
{योगेश्वराय देवाय योगिनां पतये नमः}
{योगोद्भवाय नित्याय गोविन्दाय नमो नमः}%॥५३॥


\twolineshloka
{यज्ञेश्वराय देवाय तथा यज्ञोद्भवाय च}
{यज्ञानां पतये नाथ गोविन्दाय नमो नमः}%॥५४॥


\twolineshloka
{विश्वेश्वराय विश्वाय तथा विश्वोद्भवाय च}
{विश्वस्य पतये तुभ्यं गोविन्दाय नमो नमः}%॥५५॥


\twolineshloka
{जगन्नाथ नमस्तुभ्यं संसारभयनाशन}
{जगदीशाय देवाय भूतानां पतये नमः}%॥५६॥


\twolineshloka
{धर्मेश्वराय धर्माय सम्भवाय जगत्पते}
{धर्मज्ञाय च देवाय गोविन्दाय नमो नमः}%॥५७॥


\twolineshloka
{एताभ्यां चैव मन्त्राभ्यां नैवेद्यं शयनं तथा}
{चन्द्रायार्घ्यं च मन्त्रेण अनेनैवाथ दापयेत्}%॥५८॥


\twolineshloka
{क्षीरोदार्णवसम्भूत अत्रिगोत्रसमुद्भव}
{गृहाणार्घ्यं शशाङ्केश रोहिण्या सहितो मम}%॥५९॥


\twolineshloka
{ज्योस्नापते नमस्तुभ्यं ज्योतिषां पतये नमः}
{नमस्ते रोहिणीकान्त अर्घ्यं नः प्रतिगृह्यताम्}%॥६०॥


\twolineshloka
{स्थण्डिले स्थापयेद्देवं शशाङ्कं रोहिणीयुतम्}
{देवक्या वसुदेवं च नन्दं चैव यशोदया}%॥६१॥


\twolineshloka
{बलदेवं मया सार्धं भक्त्या परमया नृप}
{सम्पूज्य विधिवद्देहि किं नाऽऽप्नोत्यतिदुर्लभम्}%॥६२॥


\twolineshloka
{एकादशीनां विंशत्यः कोटयो याः प्रकीर्तिताः}
{ताभिः कृष्णाष्टमी तुल्या ततोऽनन्तचतुर्दशी}%॥६३॥


\twolineshloka
{अर्धरात्रे वसोर्धारां पातयेद् द्रव्यसर्पिषा}
{ततो वर्धापयेन्नालं षष्ठीनामादिकं मम}%॥६४॥


\twolineshloka
{कर्तव्यं तत्क्षणाद्रात्रौ प्रभाते नवमीदिने}
{यथा मम तथा कार्यो भगवत्या महोत्सवः}%॥६५॥


\twolineshloka
{ब्राह्मणान् भोजयेद्भक्त्या तेभ्यो दद्याच्च दक्षिणाम्}
{हिरण्यं मेदिनीं गावो वासांसि कुसुमानि च}%॥६६॥


\twolineshloka
{यद्यदिष्टतमं तत्तत्कृष्णो मे प्रीयतामिति}
{यं देवं देवकी देवीं वसुदेवादजीजनत्}%॥६७॥


\twolineshloka
{भौमस्य ब्रह्मणो गुप्त्यै तस्मै ब्रह्मात्मने नमः}
{नमस्ते वासुदेवाय गोब्राह्मणहिताय च}%॥६८॥


\twolineshloka
{शान्तिरस्तु शिवं चास्तु इत्युक्त्वा मां विसर्जयेत्}
{ततो बन्धुजनौघं च दीनानाथांश्च भोजयेत्}%॥६९॥


\twolineshloka
{भोजयित्वा सुशान्तांस्तान् स्वयं भुञ्जीत वाग्यतः}
{एवं यः कुरुते देव्या देवक्याः सुमहोत्सवम्}%॥७०॥


\twolineshloka
{प्रतिवर्षं विधानेन मद्भक्तो धर्मनन्दन}
{नरो वा यदि वा नारी यथोक्तं लभते फलम्}%॥७१॥


\twolineshloka
{पुत्रसन्तानमारोग्यं सौभाग्यमतुलं लभेत्}
{इह धर्मरतिर्भूत्वा मृतो वैकुण्ठमाप्नुयात्}%॥७२॥


\twolineshloka
{तत्र देवविमानेन वर्षलक्षं युधिष्ठिर}
{भोगान्नानाविधान् भुक्त्वा पुण्यशेषादिहागतः}%॥७३॥


\twolineshloka
{सर्वकामसमृद्धे च सर्वाशुभविवर्जिते}
{कुले नृपतिशीलानां जायते हृच्छयोपमः}%॥७४॥


\twolineshloka
{यस्मिन सदैव देशे तु लिखितं तु पटार्पितम्}
{मम जन्मदिनं भक्त्या सर्वालङ्कारभूषितम्}%॥७५॥


\twolineshloka
{पूज्यते पाण्डवश्रेष्ठ जनैरुत्सवसंयुतैः}
{परचक्रभयं तत्र न कदाऽपि भवेत्पुनः}%॥७६॥


\twolineshloka
{पर्जन्यः कामवर्षी स्यादीतिभ्यो न भयं भवेत्}
{गृहे वा पूज्यते यत्र देवक्याश्चरितं मम}%॥७७॥


\twolineshloka
{तत्र सर्वं समृद्धं स्यान्नोपसर्गादिकं भवेत्}
{पशुभ्यो नकुलाद्व्यालात्पापरोगाच्च पातकात्}%॥७८॥

\threelineshloka
{राजतश्चोरतो वाऽपि न कदाचिद्भयं भवेत्}
{संसर्गेणापि यो भक्त्या व्रतं पश्येदनाकुलम्}
{सोऽपि पापविनिर्मुक्तः प्रयाति हरिमन्दिरम्}%॥७९॥

\fourlineindentedshloka
{जन्माष्टमीं जनमनोनयनाभिरामा}
{पापापहां सपदि नन्दितनन्दगोपाम्}
{यो देवकी सुतयुतां च भजेद्धि भक्त्या}
{पुत्रानवाप्य समुपैति पदं स विष्णोः}%॥८०॥

\centerline{॥इति भविष्योत्तरे जन्माष्टमीव्रतकथा॥}


\sect{शिष्टाचारप्राप्ता जन्माष्टमीव्रतकथा}

\uvacha{व्यास उवाच}

\twolineshloka
{निवृत्ते भारते युद्धे कृतशौचो युधिष्ठिरः}
{उवाच वाक्यं धर्मात्मा कृष्णं देवकिनन्दनम्}%॥१॥

\uvacha{युधिष्ठिर उवाच}
\twolineshloka
{त्वत्प्रसादात्तु गोविन्द निहताः शत्रवो रणे}
{कर्णश्च निहतः सैन्ये त्वत्प्रसादात्किरीटिना}%॥२॥

\twolineshloka
{जेता को युधि भीष्मस्य यस्य मृत्युर्न विद्यते}
{अजेयोऽपि जितः सोऽपि त्वत्प्रसादाज्जनार्दन}%॥३॥

\twolineshloka
{प्राप्तं निष्कण्टकं राज्यं कृत्वा कर्म सुदुष्करम्}
{आचारो दण्डनीतिश्च राजधर्माः क्रियान्विताः}%॥४॥

\twolineshloka
{अधुना श्रोतुमिच्छामि शुभं जन्माष्टमीव्रतम्}
{जन्माष्टमी व्रतं ब्रूहि विस्तरेण ममाच्युत}%॥५॥

\onelineshloka*
{कुतः काले समुत्पन्नं किं पुण्यं को विधिः स्मृतः}
\uvacha{श्रीकृष्ण उवाच}
\onelineshloka
{शृणु राजन्प्रवक्ष्यामि व्रतानामुत्तमं व्रतम्}%॥६॥

\twolineshloka
{यतः प्रभृति विख्यातं फलेन विधिनान्वितम्}
{राजवंशसमुत्पन्नैर्दैत्यानीकैः सुपीडिता}%॥७॥

\twolineshloka
{धरा भारसमाक्रान्ता ब्रह्माणं शरणं ययौ}
{ज्ञात्वा तदा प्रभुर्ब्रह्मा भूमेर्भारं समाहितः}%॥८॥

\twolineshloka
{श्वेतदीपं समागत्य सर्वदेवसमन्वितः}
{समाहितमतिर्ब्रह्मा मां तुष्टाव विशां पते}%॥९॥

\twolineshloka
{स्तुत्या तयाऽहं सम्प्रीतस्तेषां दृग्गोचरोऽभवम्}
{दृष्ट्वा मां प्रणिपत्याऽऽशु भक्तिभावसमन्विताः}%॥१०॥

\twolineshloka
{ब्रह्माणमग्रतः कृत्वा तुष्टाः सर्वे दिवौकसः}
{विजिज्ञपुर्महराज भूमिभारापनुत्तये}%॥११॥

\twolineshloka
{उपधार्य तदा तेषां वचनं चान्वचिन्तयम्}
{केनोपायन हन्तव्या दानवाः क्षत्रियोद्भवाः}%॥१२॥

\twolineshloka
{स्वधर्मनिरताः सर्वे महाबलपराक्रमाः}
{ततो निश्चित्य मनसा ब्रह्माणमहमब्रुवम्}%॥१३॥

\twolineshloka
{वसुदेवो देवकी च प्रजाकामौ पुरा नृप}
{भक्त्या मां भजमानौ तौ तप्तवन्तौ महत्तपः}%॥१४॥

\twolineshloka
{तयोः प्रसन्नः सुप्रीतो याचतं वरमुत्तमम्}
{अब्रुवं तावपि ततो वरयामासतुः किल}%॥१५॥

\twolineshloka
{यदि देव प्रसन्नोऽसि त्वादृशौ नौ भवेत्सुतः}
{तथेति च मया ताभ्यामुक्तं प्रीतेन चेतसा}%॥१६॥

\twolineshloka
{तत्कामपूरणार्थाय सम्भविष्याम्यहं तयोः}
{दिवौकसोऽपि स्वांशेन सम्भवन्तु सुरस्त्रियः}%॥१७॥

\twolineshloka
{योगमाया च नन्दस्य यशोदायां भविष्यति}
{देवक्या जठरे गर्भमनन्तं धाम मामकम्}%॥१८॥

\twolineshloka
{सन्निकृष्य च सा तूर्णं रोहिण्या जठरं नयेत्}
{इति सन्दिश्य तान् सर्वानहमन्तर्हितोऽभवम्}%॥१९॥

\twolineshloka
{ततो देवैः समं ब्रह्मा तां दिशं प्रणिपत्य च}
{आश्वास्य च महीं देवीं वरधाम्नि जगाम ह}%॥२०॥

\threelineshloka
{ततोऽहं देवकीगर्भमाविशं स्वेन तेजसा}
{हतेषु षट्सु बालेषु देवक्या औग्रसेनिना}
{कारागृहस्थितायाश्च वसुदेवेन वै सह}%॥२१॥

\twolineshloka
{गतेऽर्धरात्रसमये सुप्ते सर्वजने निशि}
{भाद्रे मास्यसिते पक्षेऽष्टम्यां ब्रह्मर्क्षसंयुजि}%॥२२॥

\twolineshloka
{सर्वग्रहशुभे काले प्रसन्नहृदयाशये}
{आविरासं निजेनैव रुपेण ह्यवनीपते}%॥२३॥

\twolineshloka
{वसुदेवोऽपि मां दृष्ट्वा हर्षशोकसमन्वितः}
{भीतः कंसादतितरां तुष्टाव च कृताञ्जलिः}%॥२४॥

\onelineshloka*
{पुनः पुनः प्रणम्याथ प्रार्थयामास सादरम्}
\uvacha{वसुदेव उवाच}
\onelineshloka
{अलौकिकमिदं रूपं दुर्दर्शं योगिनामपि}%॥२५॥

\twolineshloka
{यत्तेजसाऽरिष्टगृहमभवत्सम्प्रकाशितम्}
{उद्धिजे भगवत्कंसाद्यो मे बालानघातयत्}%॥२६॥

\twolineshloka
{उपसंहर तस्माच्च एतद्रूपमलौकिकम्}
{शङ्खचक्रगदापद्मलसत्कौस्तुभमालिनम्}%॥२७॥

\twolineshloka
{किरीटहारमुकुटकेयूरवलयाङ्कितम्}
{तडिद्वसनसंवीतक्वणत्काञ्चनमेखलम्}%॥२८॥

\twolineshloka
{स्फुरद्राजीवताम्राक्षं स्निग्धाञ्जनसमप्रभम्}
{महामरकतस्वच्छं कोटिसूर्यसमप्रभम्}%॥२९॥

\uvacha{कृष्ण उवाच}
\twolineshloka
{एवं सम्प्रार्थितो राजन्वसुदेवेन वै तदा}
{तेनैव निजरूपेण भूत्वाऽहं प्राकृतः शिशुः}%॥३०॥

\twolineshloka
{नय मां गोकुलमिति वसुदेवमचोदयम्}
{समादायागमत्सोऽपि नन्दगोकुलमञ्जसा}%॥३१॥

\twolineshloka
{द्वाराण्यपाकृतान्यासन्मत्प्रभावात्स्वयं प्रभो}
{ददौ मार्गं च कालिन्दीजलकल्लोलमालिनी}%॥३२॥

\twolineshloka
{ततो यशोदाशयने न्यस्य माऽऽनकदुन्दुभिः}
{तत्पर्यङ्के स्थितां गृह्य दारिकामगमत्पुनः}%॥३३॥

\twolineshloka
{द्वाराणि पिहितान्यासन् पूर्ववन्निगडं ततः}
{विन्यस्य पादयोरास्ते शयने न्यस्य दारिकाम्}%॥३४॥

\twolineshloka
{ततो रुरोद महता स्वरेणाऽऽपूर्य सा दिशः}
{तस्या रुदितशब्देन उत्थिता रक्षका गृहात्}%॥३५॥

\twolineshloka
{कंसायाऽऽगत्य चाचख्युः प्रसूता देवकीति च}
{सोऽपि तल्पात्समुत्थाय भयेनातीव विह्वलः}%॥३६॥

\twolineshloka
{जगाम सूतिकागेहं देवक्याः प्रस्खलन्पथि}
{दारिकां शयनाद्गृह्य रुदत्याश्चैव स्वस्वसुः}%॥३७॥

\twolineshloka
{अपोथयच्छिलापृष्ठे साऽपि तस्य कराच्च्युता}
{उवाच कंसमाभाष्य देवी ह्याकाशगा सती}%॥३८॥

\twolineshloka
{किं मया हतया मन्द जातः कुत्रापि ते रिपुः}
{इत्युक्तः सोऽप्यभूत्कंसः परमोद्विग्नमानसः}%॥३९॥

\twolineshloka
{आज्ञापयामास ततो बालानां कदनाय वै}
{दानवा अपि बालानां कदनं चक्रुरुद्यताः}%॥४०॥

\twolineshloka
{वनेषूपवने चैव पुरग्रामव्रजेष्वपि}
{अहं च गोकुले स्थित्वा पूतनां बालघातिनीम्}%॥४१॥

\twolineshloka
{स्तनं दातुं प्रवृत्तां च प्राणैः सममशोषयम्}
{तृणावर्तबकारिष्टान् धेनुकं केशिनं तथा}%॥४२॥

\twolineshloka
{अन्यानपि खलान् हत्वा स्वप्रभावमदर्शयम्}
{ततश्च मथुरां गत्वा हत्वा कंसादिदानवान्}%॥४३॥

\twolineshloka
{ज्ञातीनां परमं हर्षं कृतवानस्मि सादरम्}
{देवकीवसुदेवौ च परिष्वज्य मुदा मम}%॥४४॥

\twolineshloka
{आनन्दर्जैर्जलैर्मूर्ध्नि सेचयामासतुर्नृप}
{तस्मिन् रङ्गवरे मल्लान् हत्वा चाणूरमुख्यकान्}%॥४५॥

\twolineshloka
{गजं कुवलयापीडं कंसभ्रातॄननेकशः}
{एवं हतेऽसुरे कंसे सर्वलोकैककण्टके}%॥४६॥

\twolineshloka
{अन्येषु दुष्टदैत्येषु सर्वलोकभयङ्करम्}
{लोकाः समुत्सुकाः सर्वे मांसमेत्योचुरादृताः}%॥४७॥

\twolineshloka
{कृष्ण कृष्ण महायोगिन् भक्तानामभयप्रद}
{प्रलयात्पाहि नो देव शरणागतवत्सलः}%॥४८॥

\twolineshloka
{अनाथनाथ सर्वज्ञ सर्वभूतहिते रत}
{किञ्चिद् विज्ञाप्यतेऽस्माभिस्तन्नो वक्तुं त्वमर्हसि}%॥४९॥

\twolineshloka
{तव जन्मदिनं लोके न ज्ञातं केनचित्क्वचित्}
{ज्ञात्वा च तत्त्वतः सर्वे कुर्मो वर्धापनोत्सवम्}%॥५०॥

\twolineshloka
{तेषां दृष्ट्वा तु तां भक्तिं श्रद्धामपि च सौहृदम्}
{मया जन्मदिनं तेभ्यः ख्यातं निर्मलचेतसा}%॥५१॥

\twolineshloka
{श्रुत्वा तेऽपि तथा चक्रुर्विधिना येन तच्छृणु}
{पार्थ तद्दिवसे प्राप्ते दन्तधावनपूर्वकम्}%॥५२॥

\twolineshloka
{स्नात्वा पुण्यजले शुद्धे वाससी परिधाय च}
{निर्वर्त्यावश्यकं कर्म व्रतसङ्कल्पमाचरेत्}%॥५३॥

\twolineshloka
{अद्य स्थित्वा निराहारः श्वोभूते तु परेऽहनि}
{भोक्ष्यामि पुण्डरीकाक्ष शरणं मे भवाव्यय}%॥५४॥

\twolineshloka
{गृहीत्वा नियमं चैव सम्पाद्यार्चनसाधनम्}
{मण्डपं शोभनं कृत्वा फलपुष्पादिभिर्युतम्}%॥५५॥

\twolineshloka
{तस्मिन्मां पूजयेद्भक्त्या गन्धपुष्पादिभिः क्रमात्}
{उपचारैः षोडशभिर्द्वादशाक्षरविद्यया}%॥५६॥

\twolineshloka
{सद्यःप्रसूतां जननीं वसुदेवं च मारिषः}
{बलदेवसमायुक्तां रोहिणीं गुणशोभिनीम्}%॥५७॥

\twolineshloka
{नन्दं यशोदां गोपीश्च गोपान् गाश्चैव सर्वशः}
{गोकुलं यमुनां चैव योगमायां च दारिकाम्}%॥५८॥

\twolineshloka
{यशोदाशयने सुप्तां सद्योजातां वरप्रभाम्}
{एवं सम्पूजयेत्सम्यङ्नाममन्त्रैः पृथक्पृथक्}%॥५९॥

\twolineshloka
{सुवर्णरौप्यताम्रारमृदादिभिरलङ्कृताः}
{काष्ठपाषाणरचिताश्चित्रमय्योऽथ लेखिताः}%॥६॥

\twolineshloka
{प्रतिमा विविधाः प्रोक्तास्तासु चान्यतमां यजेत्}
{रात्रौ जागरणं कुर्याद्गीतनृत्यादिभिः सह}%॥६॥

\twolineshloka
{पुराणैः स्तोत्रपाठैश्च जातनामादिसूत्सवैः}
{श्वभूते पारणं कुर्याद् द्विजान् सम्भोज्य यत्नतः}%॥६॥

\twolineshloka
{एवं कृते महाराज व्रतानामुत्तमे व्रते}
{सर्वान्कामानवाप्नोति विष्णुलोके महीयते}%॥६३॥

\twolineshloka
{मोहान्न कुरुते यस्तु याति संसारगह्वरे}
{तस्मात्कुर्वन्प्रयत्नेन निष्पापो जायते नरः}%॥६४॥

\twolineshloka
{अत्रैवोदाहरन्तीममितिहासं पुरातनम्}
{अङ्गदेशोद्भवो राजा मित्रजिन्नाम नामतः}%॥६५॥

\twolineshloka
{तस्य पुत्रो महातेजः सत्यजित्सत्पथे स्थितः}
{पालयामास धर्मज्ञो विधिवद्रञ्जयन्प्रजाः}%॥६६॥

\twolineshloka
{तस्यैवं वर्तमानस्य कदाचिद्दैवयोगतः}
{पापण्डैः सह संवासो बभूव बहुवासरम्}%॥६७॥

\twolineshloka
{तत्संसर्गात्स नृपतिरधर्मनिरतोऽभवत्}
{वेदशास्त्रपुराणानि विनिन्द्य बहुशो नृप}%॥६८॥

\twolineshloka
{ब्राह्मणेषु तथा धर्मे विद्वेषं परमं गतः}
{एवं बहुतिथे काले गते भरतसत्तम}%॥६९॥

\twolineshloka
{कालेन निधनप्राप्तो यमदूतवशं गतः}
{बद्धा पाशैर्नीयमानो यमदूतैर्यमान्तिकम्}%॥७०॥

\twolineshloka
{पीडितस्ताड्यमानोऽसौ दुष्टसङ्गवशं गतः}
{नरके पतितः पापो यातनां बहुवत्सरम्}%॥७१॥

\twolineshloka
{भुक्त्वा पापस्य शेषेण पैशाची योनिमास्थितः}
{तृषाक्षुधासमाक्रान्तो भ्रमन्स मरुधन्वसु}%॥७२॥

\twolineshloka
{कस्यचित्त्वथ वैश्यस्य देहमाविश्य संस्थितः}
{सह तेनैव सम्प्राप्तो मथुरा पुण्यदां पुरीम्}%॥७३॥

\twolineshloka
{तत्रत्यैरक्षकैः सोऽथ तद्देहात्तु बहिष्कृतः}
{बभ्राम विपिने सोऽपि ऋषीणामाश्रमेष्वपि}%॥७४॥

\twolineshloka
{कदाचिद् दैवयोगेन मम जन्माष्टमीदिने}
{क्रियमाणां महापूजां व्रतिभिर्मुनिभिर्द्विजैः}%॥७५॥

\twolineshloka
{रात्रौ जागरणं चैव नामसङ्कीर्तनादिभिः}
{ददर्श सर्वं विधिवच्छुश्राव च हरेः कथाः}%॥७६॥

\twolineshloka
{निष्पापस्तत्क्षणादेव शुद्धनिर्मलमानसः}
{प्रेतदेहं समुत्सृज्य विष्णुलोकं विमानतः}%॥७७॥

\twolineshloka
{मम दूतैः समानीतो दिव्यभोगसमन्वितः}
{मम सान्निध्यमापन्नो व्रतस्यास्य प्रभावतः}%॥७८॥

\twolineshloka
{नित्यमेव व्रतं चैतत् पुराणे सार्वकालिकम्}
{गीयते विधिवत्सम्यङ्मुनिभिस्तत्त्वदर्शिभिः}%॥७९॥

\threelineshloka
{सार्वकालिकमेवैतत्कृत्वा कामानवाप्नुयात्}
{एतत्ते सर्वमाख्यातं व्रतानामुत्तमं व्रतम्}
{मम सान्निध्यकृद्राजन्किं भूयः श्रोतुमिच्छसि}%॥८०॥

\centerline{॥इति भविष्ये जन्माष्टमीव्रतकथा}


\sect{व्रतोद्यापनम्}

\uvacha{युधिष्ठिर उवाच}
\twolineshloka
{उद्यापनविधिं ब्रूहि सर्वदेव दयानिधे}
{येन सम्पूर्णतां याति व्रतमेतदनुत्तमम्}

\uvacha{श्रीकृष्ण उवाच}
\twolineshloka
{पूर्णां तिथिमनुप्राप्य वित्तचित्तादिसंयुतः}
{पूर्वेद्युरेकभक्ताशी स्वपेन्मां संस्मरन्हदि}

\twolineshloka
{प्रातरुत्थाय संस्मृत्य पुण्यश्लोकान् समाहितः}
{निर्वर्त्यावश्यकं कर्म ब्राह्मणान्स्वस्ति वाचयेत्}

\twolineshloka
{गुरुमानीय धर्मज्ञं वेदवेदाङ्गपारगम्}
{वृणुयादृत्विजश्चैव वस्त्रालङ्करणादिभिः}

\twolineshloka
{पलेन वा तदर्धेन तदर्धार्धेन वा पुनः}
{शक्त्या वाऽपि नृपश्रेष्ठ वित्तशाठ्यविवर्जितः}

\twolineshloka
{सौवर्णीं प्रतिमां कुर्यात्पाद्यार्घ्याचमनीयकम्}
{पात्रं सम्पाद्य विधिवत्पूजोपकरणं तथा}

\twolineshloka
{गोचर्ममात्रं संलिप्य मध्ये मण्डलमाचरेत्}
{ब्रह्माद्या देवतास्तत्र स्थापयित्वा प्रपूजयेत्}

\twolineshloka
{मण्डपं रचयेत्तत्र कदलीस्तम्भमण्डितम्}
{चतुर्द्वारसमोपेतं फलपुष्पादिशोभितम्}

\twolineshloka
{वितानं तत्र बध्नीयाद्विचित्रं चैव शोभनम्}
{मण्डले स्थापयेत्कुम्भं ताम्रं वा मृन्मयं शुचिम्}

\twolineshloka
{तस्योपरि न्यसेत्पात्रं राजतं वैष्णवं तु वा}
{वाससाऽऽच्छाद्य कौन्तेय पूजयेत्तत्र मां बुधः}

\twolineshloka
{उपचारैः षोडशभिर्मन्त्रैरेतैः समाहितः}
{ध्यात्वाऽऽवाह्यामृतीकृत्य स्वागतादिभिरादरात्}

\threelineshloka
{ध्यायेच्चतुर्भुजं देवं शङ्खचक्रगदाधरम्}
{पीताम्बरयुगोपेतं लक्ष्मीयुक्तं विभूषितम्}
{लसत्कौस्तुभशोभाढ्यं मेघश्यामं सुलोचनम्}
ध्यानम्॥

\twolineshloka
{आगच्छ देवदेवेश जगद्योने रमापते}
{शुद्धे ह्यस्मिन्नधिष्ठाने सन्निधेहि कृपां कुरु}
आवाहनम्॥

\twolineshloka
{देवदेव जगन्नाथ गरुडासनसंस्थित}
{गृहाण चाऽऽसनं दिव्यं जगद्धातर्नमोऽस्तु ते}
आसनम्॥

\twolineshloka
{नानातीर्थाहृतं तोयं निर्मलं पुष्षमिश्रितम्}
{पाद्यं गृहाण देवेश विश्वरूप नमोऽस्तु ते}
पाद्यम्॥

\twolineshloka
{गङ्गादिसर्वतीर्थेभ्यो भक्त्याऽऽनीतं सुशीतलम्}
{गन्धपुष्पाक्षतोपेतं गृहाणार्घ्यं नमोऽस्तु ते}
अर्घ्यम्॥

\twolineshloka
{कृष्णावेणीसमुद्भूतं कालिन्दीजलसंयुतम्}
{गृहाणाऽऽचमनं देव विश्वकाय नमोऽस्तु ते}
आचमनम्॥

\twolineshloka
{दधि क्षौद्रं घृतं शुद्धं कपिलायाः सुगन्धि यत्}
{सुस्वादु मधुरं शौरे मधुपर्कं गृहाण मे}
मधुपर्कम्॥
पुनराचमनम्॥

\twolineshloka
{पञ्चामृतेन स्नपनं करिष्यामि सुरोत्तम}
{क्षीरोदधिनिवासाय लक्ष्मीकान्ताय ते नमः}
पञ्चामृतस्नानम्॥

\twolineshloka
{मन्दाकिनी गौतमी च यमुना च सरस्वती}
{ताभ्यः स्नानार्थमानीतं गृहाण शिशिरं जलम्}
स्नानम्॥
पुनराचमनम्॥

\twolineshloka
{शुद्धजाम्बूनदप्रख्ये तडिद्भासुररोचिषी}
{मयोपपादिते तुभ्यं वाससी प्रतिगृह्यताम्}
वस्त्रयुग्मम्॥

यज्ञोपवीतमिति यज्ञोपवीतम्॥

\twolineshloka
{किरीटकुण्डलादीनि काञ्चीवलययुग्मकम्}
{कौस्तुभं वनमालां च भूषणानि भजस्व मे}
भूषणानि॥

\twolineshloka
{मलयाचलसम्भूतं घनसारं मनोहरम्}
{हृदयानन्दनं चारु चन्दनं प्रतिगृह्यताम्}
चन्दनम्॥

अक्षताश्च सुरश्रेष्ठेति कुङ्कुमाक्षतान्।

\twolineshloka
{मालतीचम्पकादीनि यूथिकाबकुलानि च}
{तुलसीपत्रमिश्राणि गृहाण सुरसत्तम}
पुष्पाणि॥


अथाङ्गपूजा-

अघनाशनाय नमः — पादौ पूजयामि।\\
वामनाय नमः — गुल्फौ पूजयामि।\\
शौरये नमः — जङ्घे पूजयामि।\\
वैकुण्ठवासिने नमः — ऊरू पूजयामि।\\
पुरुषोत्तमाय नमः — मेढ्रं पूजयामि।\\
वासुदेवाय नमः — कटिं पूजयामि।\\
हृषीकेशाय नमः — नाभिं पूजयामि।\\
माधवाय नमः — हृदयं पूजयामि।\\
मधुसूदनाय नमः — कण्ठं पूजयामि।\\
वराहाय नमः — बाहून् पूजयामि।\\
नृसिंहाय नमः — हस्तौ पूजयामि।\\
दैत्यसूदनाय नमः — मुखं पूजयामि।\\
दामोदराय नमः — नासिकां पूजयामि।\\
पुण्डरीकाक्षाय नमः — नेत्रे पूजयामि।\\
गरुडध्वजाय नमः — श्रोत्रे पूजयामि।\\
गोविन्दाय नमः — ललाटं पूजयामि।\\
अच्युताय नमः — शिरः पूजयामि।\\
कृष्णाय नमः — सर्वाङ्गं पूजयामि॥\\


अथ परिवारदेवतापूजा—

\twolineshloka
{देवकीं वसुदेवं च रोहिणीं सबलां तथा}
{सात्यकिं चोद्धवाक्रूरोग्रसेनादियादवान्}

\twolineshloka
{नन्दं यशोदां तत्कालप्रसूतां गोपगोपिकाः}
{कालिन्दीं कालियं चैव पूजयेन्नाममन्त्रतः}

\twolineshloka
{वनस्पतिरसोद्भतं कालागुरुसमन्वितम्}
{धूपं गृहाण गोविन्द गुणसागर गोपते}
धूपम्॥

साज्यं त्रिवर्तिसंयुक्तम्॥\\
दीपम्॥

\twolineshloka
{शाल्योदनं पायसं च सिताघृतविमिश्रितम्}
{नानापक्कासंयुक्तं नैवेद्यं प्रतिगृह्यताम्}
नैवेद्यम्।
उत्तरापोशनम्॥

इदं फलमिति फलम्॥

पूगीफलमिति ताम्बूलम्॥

हिरण्यगर्भेति दक्षिणाम्॥

\twolineshloka
{नीराजयेत्ततो भक्त्या मङ्गलं समुदीरयन्}
{जयमङ्गलनिर्घोषैर्देवदेवं समर्चयेत्}
नीराजनम्॥


\twolineshloka
{दत्त्वा पुष्पाञ्जलिं चैव प्रदक्षिणपुरःसरम्}
{प्रणमेद्दण्डवद् भूभौ भक्तिप्रह्वः पुनः पुनः}

स्तुत्वा नानाविधैः स्तोत्रैः प्रार्थयेत जगत्पतिम्॥

\twolineshloka
{नमस्तुभ्यं जगन्नाथ देवकीतनय प्रभो}
{वसुदेवात्मजानन्त यशोदानन्दवर्धन}

\twolineshloka
{गोविन्द गोकुलाधार गोपीकान्त नमोऽस्तु ते}
{ततस्तु दापयेदर्घ्यमिन्दोरुदयतः शुचिः}

\twolineshloka
{कृष्णाय प्रथमं दद्याद् देवकीसहिताय च}
{नालिकेरेण शुद्धेन मुक्तमर्घ्यं विचक्षण}

कृष्णाय परया भक्त्या शङ्खे कृत्वा विधानतः॥

\twolineshloka
{जातः कंसवधार्थाय भूभारोत्तारणाय च}
{कौरवाणां विनाशाय दैत्यानां निधनाय च}

\twolineshloka
{पाण्डवानां हितार्थाय धर्मसंस्थापनाय च}
{गृहाणाऱ्या मया दत्तं देवकीसहितो हरे}
कृष्णार्घ्यमन्त्रः॥

\twolineshloka
{शङ्खे कृत्वा ततस्तोयं सपुष्पफलचन्दनम्}
{जानुभ्यामवनिं गत्वा चन्द्रायार्घ्यं निवेदयेत्}

\twolineshloka
{क्षीरोदार्णवसम्भूत अत्रिगोत्रसमुद्भव}
{गृहाणार्घ्यं मया दत्तं रोहिण्या सहित प्रभो}

\twolineshloka
{ज्योत्स्नापते नमस्तुभ्यं नमस्ते ज्योतिषां पते}
{नमस्ते रोहिणीकान्त गृहाणार्घ्यं नमोऽस्तु ते}
चन्द्रार्घ्यमन्त्रः॥

\twolineshloka
{इत्थं सम्पूज्य देवेशं रात्रौ जागरणं चरेत्}
{गीतनृत्यादिना चैव पुराणश्रवणादिभिः}

\twolineshloka
{प्रत्यूषे विमले स्नात्वा पूजयित्वा जगद्गुरुम्}
{पायसेन तिलाज्यैश्च मूलमन्त्रेण भक्तितः}

\twolineshloka
{अष्टोत्तरशतं हुत्वा ततः पुरुषसूक्ततः}
{इदं विष्णुरिति प्रोक्त्वा जुहुयाद्वै घृताहुतीः}

\twolineshloka
{होमशेषं समाप्याथ पूर्णाहुतिपुरःसरम्}
{आचार्यं पूजयेद्भक्त्या भूषणाच्छादनादिभिः}

\twolineshloka
{गामेकां कपिलां दद्याद् व्रतसम्पूर्तिहेतवे}
{पयस्विनीं सुशीलां च सवत्सां सगुणां तथा}

\twolineshloka
{स्वर्णशृङ्गीं रौप्यखुरांं कांस्यदोहनिकायुताम्}
{रत्नपुच्छां ताम्रपृष्ठीं स्वर्णघण्टासमन्विताम्}

\twolineshloka
{वस्त्रच्छन्नां दक्षिणाढ्यामेवं सम्पूर्णतां व्रजेत्}
{कपिलाया अभावे तु गौरन्याऽपि प्रदीयते}

\twolineshloka
{ततो दद्याच्च ऋत्विग्भ्योऽन्येभ्यश्चैव यथाविधि}
{शय्यां सोपस्करां दद्याद् व्रतसम्पूर्तिहेतवे}

\twolineshloka
{ब्राह्मणान्भोजयेत्पश्चादष्टौ तेभ्यश्च दक्षिणाम्}
{कलशानन्नसम्पूर्णान्दद्याच्चैव समाहितः}

\twolineshloka
{दीनान्धकृपणांश्चैव यथार्हं प्रतिपूजयेत्}
{प्राप्यानुज्ञां तथा तेभ्यो भुञ्जीत सह बन्धुभिः}

\twolineshloka
{एवं कृते महाराज व्रतोद्यापनकर्मणि}
{निष्पापस्तत्क्षणादेव जायते विबुधोपमः}

\twolineshloka
{पुत्रपौत्रसमायुक्तो धनधान्यसमन्वितः}
{भुक्त्वा भोगांश्चिरं कालमन्ते मम पुरं व्रजेत्}

॥इति श्रीभविष्यपुराणे कृष्णयुधिष्ठिरसंवादे जन्माष्टमीव्रतोद्यापनं सम्पूर्णम्॥

\endgroup