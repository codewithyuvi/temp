% !TeX program = XeLaTeX
% !TeX root = ../../pujavidhanam.tex
\sect{जन्माष्टमी-व्रत-कथा}

\uvacha{युधिष्ठिर उवाच}

\twolineshloka
{जन्माष्टमीव्रतं ब्रूहि विस्तरेण ममाच्युत}
{कस्मिन्काले समुत्पन्नं किं पुण्यं को विधिः स्मृतः}%॥१॥

\uvacha{श्रीकृष्ण उवाच}
\twolineshloka
{मल्लमुद्धे परावृत्ते शमिते कुकुरान्धके}
{स्वजनैर्बन्धुभिः स्त्रीभिः समः स्निग्धैः समावृते}%॥२॥

\twolineshloka
{हते कंसासुरे दुष्टे मथुरायां युधिष्ठिर}
{देवकी मां परिष्वज्य कृत्वोत्सङ्गे रुरोद ह}%॥३॥

\twolineshloka
{वसुदेवोऽपि तत्रैव वात्सल्यात्प्ररुरोद ह}
{समालिङ्गयात्रुवदनः पुत्रपुत्रेत्युवाच ह}%॥४॥

\twolineshloka
{सगद्दस्वरो दीनो बाप्पपर्याकुलक्षणः}
{बलभद्रं च मां चैव परिष्वज्य मुदा पुनः}%॥५॥

\twolineshloka
{अद्य मे सफलं जन्म जीवितं च बुजीवितम्}
{उभाभ्यामद्य पुत्राभ्यां समुद्भूतः समागमः}%॥६॥

\twolineshloka
{एवं हर्षेण दाम्पत्यं हृष्टं पुष्ट तदा ह्यभूत्}
{प्रणिपत्य जनाः सर्वे बभूवुस्ते प्रहर्षिताः}%॥७॥

\onelineshloka*
{एवं महोत्सवं दृष्ट्वा मामूचुर्मधुसूदनम्}
\uvacha{जना ऊचुः}
\onelineshloka
{प्रसादः क्रियतामस्य लोकस्यार्तस्य दुःखहन्}%॥८॥

\twolineshloka
{यस्मिन्दिने च प्रासूत देवकी त्वां जनार्दन}
{तदिनं देहि वैकुण्ठ कुर्मस्तत्र होइन्}%॥९॥

\twolineshloka
{एवं स्तुतो जनौधेन वासुदेवो मयेक्षितः}
{विलोक्य बलभद्रं च मां च हृष्टतनूरुहः}%॥१०॥

\twolineshloka
{उवाच स ममादेशाल्लोकाञ्जन्माष्टमीव्रतम्}
{मथुरायां ततः पश्चात् पार्थ सम्यक् प्रकाशितम्}%॥११॥

\twolineshloka
{कुर्वन्तु ब्राह्मणाः सर्वे व्रतं जन्माष्टमीदिने}
{क्षित्रिया वैश्यजातीयाः शूद्रा येऽन्येऽपि धर्मिणः}%॥१२॥

\uvacha{युधिष्ठिर उवाच}
\twolineshloka
{कीदृशं तद्वतं देवदेव सर्वैरनुष्ठितम्}
{जन्माष्टमीति संज्ञं च पवित्रं पापनाशनम्}%॥१३॥

\twolineshloka
{येन त्वं तुष्टिमायासि कात्स्न्येन प्रभवाव्यय}
{एतन्मे तत्त्वतो ब्रूहि सविधान सविस्तरम्}%॥१४॥

\uvacha{श्रीकृष्ण उवाच}
\twolineshloka
{मासि भाद्रपदेष्टम्यां निशीथे कृष्णपक्षके}
{शशाङ्के वृषराशिस्थे ऋक्षे रोहिणीसंज्ञके}%॥१५॥

\twolineshloka
{योगेऽस्मिन्वसुदेवाद्धि देवकी मामजीजनत्}
{भगवत्याश्च तत्रैव क्रियते सुमहोत्सवः}%॥१६॥

\twolineshloka
{योगेऽस्मिन्कथितेऽष्टम्यां सिंहराशिगते रवौ}
{सप्तम्यां लघुभुक् कुर्यादन्तधावनपूर्वकम्}%॥१७॥

\twolineshloka
{उपवासस्य नियम रात्रौ स्वप्याजितेन्द्रियः}
{केवलेनोपवासेन तस्मिञ्जन्मदिने मम}%॥१८॥

\twolineshloka
{सप्तजन्मकृतात्पापान्मुच्यते नात्र संशयः}
{उपावृत्तस्य पापेभ्यो यस्तु वासोगुणैः सह}%॥१९॥

\twolineshloka
{उपवासः स विज्ञेयः सर्वभोगविवर्जितः}
{ततोऽष्टम्यां तिलैः स्नात्वा नद्यादौ विमले जले}%॥२०॥

\twolineshloka
{सुदेशे शोभनं कुर्यादेवक्याः सूतिकागृहम्}
{सितपीतैस्तथा रक्तैः कङ्घरैरितरैरपि}%॥२१॥

\twolineshloka
{वासोभिः शोभितं कृत्वा समन्तात्कलशेनवैः}
{पुष्पैः फलैरनेकैश्च दीपालिभिरितस्ततः}%॥२२॥

\twolineshloka
{पुष्पमालाविचित्रं च चन्दनागुरुधूपितम्}
{अतिरम्यमनौपम्यं रक्षामणिविभूषितम्}%॥२३॥

\twolineshloka
{हरिवंशस्य चरितं गोकुलं च विलेखयेत}
{ततो वादिबनिनदेर्वीणावेणुरवाकुलम्}%॥२४॥

\twolineshloka
{नृत्यगीतक्रमोपेतं मङ्गलैश्च समन्ततः}
{वेष्टकारों लोहखङ्गं कृष्णछांग च यत्नतः}%॥२५॥

\twolineshloka
{द्वारे विन्यस्य मुसलं रक्षितं रक्षपालकैः}
{षष्ठया देव्याधिष्ठितं च तद्गृहं चोत्सवैस्तथा}%॥२६॥

\twolineshloka
{एवंविभवसारेण कृत्वा तत्सूतिकागृहम्}
{तन्मध्ये प्रतिमा स्थाप्या सा चाप्यष्टविधा स्मृता}%॥२७॥

\twolineshloka
{काञ्चनी राजती ताम्री पैतली मृन्मयी तथा}
{वाी मणिमयी चव वर्णकैलिविता तथा}%॥२८॥

\twolineshloka
{सर्वलक्षणसम्पूर्णा पर्यढ़े चाष्टशल्यके}
{प्रतप्तकाञ्चनाभासां महाही सुतपस्विनीम्}%॥२९॥

\twolineshloka
{प्रसूतां च प्रसुतां च स्थापयन्मश्चकोपरि}
{मां तत्र बालकं सुप्तं पर्यः स्तनपायिनम्}%॥३०॥

\twolineshloka
{श्रीवत्सवक्षसं शान्तं नीलोत्पलदलच्छविम्}
{यशोदा तत्र चैकस्मिन् प्रदेशे सूतिकागृहे}%॥३१॥

\twolineshloka
{तद्वञ्च कल्पयेत् पार्थ प्रसूतां वरकन्यकाम}
{तथैव मम पार्श्वस्थाः कृताञ्जलिपुटा नृप}%॥३२॥

\twolineshloka
{देवा ग्रहास्तथा नागा अभियान}
{प्रणताः पुष्पमालामचारुहस्ताः सुरासुराः}%॥३३॥

\twolineshloka
{सञ्चरन्त इवाकाशे प्रहाररुदितोदितः}
{डेमोऽपि तत्रैव खड्गचर्मधरः स्थितः}%॥३४॥

\twolineshloka
{कश्यपो वसुदेवोऽयमदितिश्चैव देवकी}
{शेगे वै बलदेवोऽयं यशोदादितिरन्वभूत्}%॥३५॥

\twolineshloka
{नन्दः प्रतापतिदक्षोगर्गश्चापि चतुर्मुखः}
{गोप्यश्चाप्सरसश्चैव गोपाश्चापि दिवौकसः}%॥३६॥

\twolineshloka
{एषोऽवतारो राजेन्द्र कंसोऽयं कालोनिः}
{नत्र कंसनियुक्ताश्च मोहिता योगनिद्रया}%॥३७॥

\twolineshloka
{गोधेनुकुञ्जराश्चैव दानवाः शस्त्रपाणयः}
{नृत्यत वासरोभिस्ते गन्धर्वा गीततत्पराः}%॥३८॥

\twolineshloka
{लेखनीयश्च तत्रैव कालियो यमुनाहूदे}
{इत्येत्रमादि यत्किचिद्विद्यते चरितं मम}%॥३९॥

\twolineshloka
{लेखयित्वा प्रयत्नेन पूजयेद्भक्तितत्परः}
{रम्यमेवं वीजपूरैः पुष्पमालादिशोभितम्}%॥४०॥

\threelineshloka
{कालदेशोद्भवैः पुष्पैः फलैचापि युधिष्ठिर}
{पाद्याध्यैः पूजयेद्भक्त्या गन्धपुष्पाक्षतैः सह}
{मन्त्रेणानेन कौन्तेय देवकी पूजयेन्नरः}%॥४१॥

\fourlineindentedshloka
{गायद्भिः किन्नराद्यैः सततपरिवृता वेणुवीणानिनादैः}
{भृङ्गारादर्शकुम्भप्रवरवृतकरैः किङ्करैः सेव्यमाना}
{पर्यङ्के स्वास्तृते यामुदिततरमुखी पुत्रिणी सम्यगास्ते}
{सा देवी देवमाता जयतु च ससुता देवकी कान्तरूपा}%॥४२॥


\twolineshloka
{पादावभ्यञ्जयन्ती श्रीदेवक्याश्चरणान्तिके}
{निषण्णा पङ्कजे पूज्या दिव्यगन्धातुलेपनः}%॥४३॥


\twolineshloka
{पङ्कजैः पूजयेद्देवी नमो देव्यै श्रिया इति}
{देववत्से नमस्तेऽस्तु कृष्णोत्पादनतत्परा}%॥४४॥


\twolineshloka
{पापक्षयकरा देवी तुष्टिं यातु मयार्चिता}
{प्रणवादिनमोऽन्तं च पृथङ्नामानुकीर्तनम्}%॥४५॥


\twolineshloka
{कुर्यात्पूजा विधिज्ञश्च सर्वपापापतुत्तये}
{देवक्यै वसुदेवाय वासुदेवाय चैव हि}%॥४६॥


\twolineshloka
{बलदेवाय नन्दाय यशोदायै पृथक् पृथकू}
{क्षीरादिस्नपनं कृत्वा चन्दनेनानुलेपयेत्}%॥४७॥


\twolineshloka
{विध्यन्तरमपीच्छन्ति केचिदत्रैव सूरयः}
{चन्द्रोदये शशाङ्काय अयं दत्त्वा हरिं स्मरन्}%॥४८॥


\twolineshloka
{अन वामनं शौरि वैकुण्ठं पुरुषोत्तमम्}
{वासुदेवं हृषीकेशं माधवं मधुसूदनम्}%॥४९॥


\twolineshloka
{वराहं पुण्डरीकाक्षं नृसिंहं ब्रह्मणः प्रियम्}
{समस्तस्यापि जगतः सृष्टिस्थित्यन्तकारकम्}%॥५०॥


\twolineshloka
{अनादिनिधनं विष्णुः त्रैलोक्येशं त्रिविक्रमम्}
{नारायणं चतुर्बाहुं शङ्खचक्रगदाधरम्}%॥५१॥


\twolineshloka
{पीताम्बरधरं नित्यं वनमालाविभूषितम्}
{श्रीवत्साक्षं जगत्सेतुं श्रीपतिं श्रीधरं हरिम्}%॥५२॥


\twolineshloka
{योगेश्वराय देवाय योगिनां पतये नमः}
{योगोद्भवाय नित्याय गोविन्दाय नमो नमः}%॥५३॥


\twolineshloka
{यज्ञेश्वराय देवाय तथा यज्ञोद्भवाय च}
{यज्ञानां पतये नाथ गोविन्दाय नमो नमः}%॥५४॥


\twolineshloka
{विश्वेश्वराय विश्वाय तथा विश्वोद्भवाय च}
{विश्वस्य पतये तुभ्यं गोविन्दाय नमो नमः}%॥५५॥


\twolineshloka
{जगन्नाथ नमस्तुभ्यं संसारभयनाशन}
{जगदीशाय देवाय भूतानां पतये नमः}%॥५६॥


\twolineshloka
{धर्मेश्वराय धर्माय संभवाय जगत्पते}
{धर्मज्ञाय च देवाय गोविन्दाय नमो नमः}%॥५७॥


\twolineshloka
{एताभ्यां चैव मन्त्राभ्यां नैवेद्यं शयनं तथा}
{चन्द्रायाय॑ च मन्त्रेण अनेनैवाथ दापयेत्}%॥५८॥


\twolineshloka
{क्षीरोदार्णवसंभूत अत्रिगोत्रसमुद्भव}
{गृहाणार्य शशाङ्केश रोहिण्या सहितो मम}%॥५९॥


\twolineshloka
{ज्योस्नापते नमस्तुभ्यं ज्योतिषां पतये नमः}
{नमस्ते रोहिणीकान्त अयं नः प्रतिगृह्यताम्}%॥६०॥


\twolineshloka
{स्थण्डिले स्थापयेदेवं शशांक रोहिणीयुतम्}
{देवक्या वसुदेवं च नन्दं चैव यशोदया}%॥६१॥


\twolineshloka
{बलदेवं मया साध भक्त्या परमया नृप}
{संपूज्य विधिवदेहि किं नाप्नोत्यतिदुर्लभम्}%॥६२॥


\twolineshloka
{एकादशीनां विंशत्यः कोटयो याः प्रकीर्तिताः}
{ताभिः कृष्णाष्टमी तुल्या ततोऽनन्तचतुर्दशी}%॥६३॥


\twolineshloka
{अर्धरात्रे वसोर्धारां पातयेद्रव्यसर्पिषा}
{ततो वर्धापयेत्रालं षष्ठीनामादिकं मम}%॥६४॥


\twolineshloka
{कर्तव्यं तत्क्षणाद्रात्रौ प्रभाते नवमीदिने}
{यथा मम तथा कार्यों भगवत्या महोत्सवः}%॥६५॥


\twolineshloka
{ब्राह्मणान् भोजयेद्भक्त्या तेभ्यो दद्याच्च दक्षिणाम्}
{हिरण्यं मेदिनी गावो वासांसि कुसुमानिच}%॥६६॥


\twolineshloka
{यद्यदिष्टतमं तत्तत्कृष्णो मे प्रीयतामिति}
{यं देवं देवकी देवी वसुदेवादजीजनत्}%॥६७॥


\twolineshloka
{भौमस्य ब्रह्मणो गुप्त्यै तस्मै ब्रह्मात्मने नमः}
{नमस्ते वासुदेवाय मेवाणहिताय च}%॥६८॥


\twolineshloka
{शान्तिरस्तु शिवं चास्तु इत्युक्त्वा मां विसर्जयेत्}
{ततो बन्धुजनौघं च दीनानाथांश्च भोजयेत्}%॥६९॥


\twolineshloka
{भोजयित्वा सुशान्तांस्तान् स्वयं भुनीत वाग्यतः}
{एवं यः कुरुते देव्या देवक्याः सुमहोत्सवम्}%॥७०॥


\twolineshloka
{प्रतिवर्ष विधानेन मद्भक्तो धर्मनन्दन}
{नरो वा यदि वा नारी यथोक्तं लभते फलम्}%॥७१॥


\twolineshloka
{पुत्रसन्तानमारोग्यं सौभाग्यमतुलं लभेत्}
{इह धर्मरतिभूत्वा मृतो वैकुण्ठमाप्नुयात्}%॥७२॥


\twolineshloka
{तत्र देवविमानेन वर्षलक्षं युधिष्ठिर}
{भोगान्नानाविधान् भुक्त्वा पुण्यशेषादिहागतः}%॥७३॥


\twolineshloka
{सर्वकामसमृद्धे च सर्वाशुभविवर्जिते}
{कुले नृपतिशीलानां जायते हृच्छयोपमः}%॥७४॥


\twolineshloka
{यस्मिन सदैव देशे तु लिखितं तु पटार्पितम्}
{मम जन्मदिनं भक्त्या सर्वालंकारभूषितम्}%॥७५॥


\twolineshloka
{पूज्यते पाण्डवश्रेष्ठ जनरुत्सवसंयुतैः}
{परचक्रभयं तत्र न कदापि भवेत्पुनः}%॥७६॥


\twolineshloka
{पर्जन्यः कामवर्षी स्यादीतिभ्यो न भयं भवेत्}
{गृहे वा पूज्यते यत्र देवक्याश्चरितं मम}%॥७७॥


\twolineshloka
{तत्र सर्व समृद्धं स्यानोपसर्गादिकं भवेत्}
{पशुभ्यो नकुलाब्यालात्पापरोगाच्च पातकात्}%॥७८॥

\threelineshloka
{राजतश्चोरतो वापि न कदाचिद्भयं भवेत्}
{संसर्गणापि यो भक्त्या व्रतं पश्येदनाकुलम्}
{सोऽपि पापविनिर्मुक्तः प्रयाति हरिमन्दिरम्}%॥७९॥

\fourlineindentedshloka
{जन्माष्टमी जनमनोनयनाभिरामा}
{पापापहां सपदि नन्दितनन्दगोपाम्}
{यो देवकी सुतयुतां च भजेद्धि भक्त्या}
{पुत्रानवाप्य समुपैति पदं स विष्णोः}%॥८०॥

॥इति भविष्योत्तरे जन्माष्टमीव्रतकथा॥




\sect{शिष्टाचारप्राप्ता जन्माष्टमीव्रतकथा}

\uvacha{व्यास उवाच}

\twolineshloka
{निवृत्ते भारते युद्धे कृतशोचो युधिष्ठिरः}
{उवाच वाक्यं धर्मात्मा कृष्ण देवकिनन्दनम्}%॥१॥

\uvacha{युधिष्ठिर उवाच}
\twolineshloka
{त्वत्प्रसादातु गोविन्द निहताः शत्रवो रणे}
{कर्णश्च निहतः सैन्ये त्वत्प्रसादाकिरीटिना}%॥२॥

\twolineshloka
{जेता को युधि भीष्मस्य यस्य मृत्युन विद्यते}
{अजेयोऽपि जितः सोऽपि त्वत्प्रसादाजनार्दन}%॥३॥

\twolineshloka
{प्राप्त निष्कण्टकं राज्यं कृत्वा कर्म करम्}
{आचारो दण्डनीतिश्च राजधर्माः क्रियान्विताः}%॥४॥

\twolineshloka
{अधुना श्रोतुमिच्छामि शुभं जन्माष्टमीवतन्}
{जन्माष्टमी व्रतं ब्रूहि विस्तरेण ममाच्युत}%॥५॥

\onelineshloka*
{कुतः काले समुत्पन्नं किंपुण्यं को विधिः स्मृतः}
\uvacha{श्रीकृष्ण उवाच}
\onelineshloka
{शृणु राजन्प्रवक्ष्यामि व्रतानामुत्तमं व्रतम्}%॥६॥

\twolineshloka
{यतः प्रभृति विख्यातं फलेन विधिनान्वितम्}
{राजवंशसमुत्पनैर्दैत्यानीकैः सुपीडिता}%॥७॥

\twolineshloka
{धरा भारसमाक्रान्ता ब्रह्माणं शरणं ययौ}
{ज्ञात्वा तदा प्रभुर्ब्रह्मा भूमेरिं समाहितः}%॥८॥

\twolineshloka
{श्वेतदीपं समागत्य सर्वदेवसमन्वितः}
{समाहितमतिर्ब्रह्मा मां तुष्टाव विशांपते}%॥९॥

\twolineshloka
{स्तुत्या तयाहं संप्रीतस्तेषां दृग्गोचरोऽभवम्}
{दृष्ट्वा मां प्रणिपत्याशु भक्तिभावसमन्विताः}%॥१०॥

\twolineshloka
{ब्रह्माणमग्रतः कृत्वा तुष्टाः सर्वे दिवौकसः}
{विजिनपुर्महराज अभिमानहत्तरे}%॥११॥

\twolineshloka
{उपधार्य तदा तेषां वचनं चान्वचिन्तयम्}
{केनोपायन हन्तव्या दानवाः क्षत्रियोद्भवाः}%॥१२॥

\twolineshloka
{स्वधर्मनिरताः सर्वे महाबलपराक्रमाः}
{ततो निश्चित्य मनसा ब्रह्माणमहमब्रुवम्}%॥१३॥

\twolineshloka
{वसुदेवो देवकी च प्रजाकामौ पुरा नृप}
{भक्त्या मां भजमानौ तौ तप्तवन्तौ महत्तपः}%॥१४॥

\twolineshloka
{तयोः प्रसन्नः सुप्रीतो याचतं वरमुत्तमम्}
{अत्रुवं तावपि ततो वरयामासतुः किल}%॥१५॥

\twolineshloka
{यदि देव प्रसन्नोऽसि त्वादशौ नौ भवेत्सुतः}
{तथेति च मया ताभ्यामुक्तं प्रीतेन चेतसा}%॥१६॥

\twolineshloka
{तत्कामपूरणार्थाय संभविष्याम्यहं तयोः}
{दिवौकसोऽपि स्वांशेन संभवन्तु तुरस्त्रियः}%॥१७॥

\twolineshloka
{योगमाया च नन्दस्य यशोदायां भविष्यति}
{देवक्या जठरे गर्भमनन्तं धाम मामकम्}%॥१८॥

\twolineshloka
{सन्निकृष्य च सा तूर्ण रोहिण्या जठरं नयेत्}
{इति सन्दिश्य तान् सर्वानहमन्तर्हितोऽभवम्}%॥१९॥

\twolineshloka
{ततो देवैः समं ब्रह्मा तां दिशं प्रणिपत्य च}
{आश्वास्य च महीं देवीं वरधान्त्रि जगामह}%॥२०॥

\threelineshloka
{ततोऽहं देवकीगर्भमाविशं स्वेन तेजसा}
{हतेषु षट्सु बालेषु देवक्या औग्रसेनिना}
{कारागृहस्थितायाश्च वसुदेवेन वै सह}%॥२१॥

\twolineshloka
{गतेर्धरात्रसमये सुप्ते सर्वजन निशि}
{भाद्रे मास्यसिते पक्षेऽष्टम्यां ब्रह्मक्षसंयुजि}%॥२२॥

\twolineshloka
{सर्वग्रहशुभे काले प्रसन्नहृदयाशये}
{आविरासं निजेनैव रुपेण यवनीपते}%॥२३॥

\twolineshloka
{वसुदेवोऽपि मां दृष्ट्वा हर्षशोकसमन्वितः}
{भीतः कंसादतितरां तुष्टाव च कृताञ्जलिः}%॥२४॥

\onelineshloka*
{पुनः पुनः प्रणम्याथ प्रार्थयामास सादरम्}
\uvacha{वसुदेव उवाच}
\onelineshloka
{अलौकिकमिदं रूपं दुर्दर्श योगिनामपि}%॥२५॥

\twolineshloka
{यत्तेजसारिष्टगृहमनवलंत्रकाशितम्}
{उद्धिजे भगवत्कंसाद्यो मे बालानधातयत्}%॥२६॥

\twolineshloka
{उपसंहर तस्माच एतद्पमलौकिकम्}
{शङ्खचक्रगदापालसत्कौस्तुभमालिनम्}%॥२७॥

\twolineshloka
{किरीटहारमुकुटकेयूरवलयाङ्कितम्}
{तडिद्वसनसंवीतक्वणकाञ्चनमेखलम्}%॥२८॥

\twolineshloka
{स्फुरद्राजीवताम्राक्षं स्निग्धाञ्जनसमप्रभम्}
{महामरकतस्वच्छ कोटिसूर्यसमप्रभम्}%॥२९॥

\uvacha{कृष्ण उवाच}
\twolineshloka
{एवं संप्रार्थितो राजन्वसुदेवेन वै तदा}
{तेनैव निजरूपेण भूत्वाहं प्राकृतः शिशुः}%॥३०॥

\twolineshloka
{नय मां गोकुलमिति वसुदेवमचोदयम्}
{समादायागमत्सोपि नन्दगोकुलमञ्जसा}%॥३१॥

\twolineshloka
{द्वारण्यपाकृतान्यासन्मत्प्रभावात्स्वयं प्रभो}
{ददौ मार्ग च कालिन्दीजलकल्लोलमालिनी}%॥३२॥

\twolineshloka
{ततो यशोदाशयने न्यस्य माऽनकदुन्दुभिः}
{तत्पर्य स्थितां गृह्य दारिकामगमत्पुनः}%॥३३॥

\twolineshloka
{द्वाराणि पिहितान्यासन् पूर्ववन्निगडं ततः}
{विन्यस्य पादयोरास्ते शयने न्यस्य दारिकाम्}%॥३४॥

\twolineshloka
{ततो रुरोद महता स्वरेणापूर्य सा दिशः}
{तस्या रुदितशब्देन उत्थिता रक्षका गृहात्}%॥३५॥

\twolineshloka
{कंसायागत्य चाचख्युः प्रसूता देवकीति च}
{सोऽपि तल्पात्समुत्थाय भयेनातीव विह्वलः}%॥३६॥

\twolineshloka
{जगाम सूतिकागेह देवक्याः प्रस्खलन्पथि}
{दारिका शयनादूगृह्य रुदत्याश्चैव स्वस्वस्तुः}%॥३७॥

\twolineshloka
{अपोथयच्छिलापृष्ठे सापि तस्य कराच्च्युता}
{उवाच कंसमाभाष्य देवी याकाशगा सती}%॥३८॥

\twolineshloka
{किं मया हतया मन्द जातः कुत्रापि ते रिपुः}
{इत्युक्तः सोऽप्यभूत्कंसः परमोद्विग्नमानसः}%॥३९॥

\twolineshloka
{आज्ञापयामास ततो बालानां कदनाय वै}
{दानवा अपि बालानां कदनं चक्रुरुद्यताः}%॥४०॥

\twolineshloka
{वनेषूपवने चैव पुरग्रामव्रजेष्वपि}
{अहं च गोकुले स्थित्वा पूतनां बालघातिनीम्}%॥४१॥

\twolineshloka
{स्तनं दातुं प्रवृत्तां च प्राणैः सममशोषयम्}
{तृणावर्तवकारिष्टान् धेनुकं केशिनं तथा}%॥४२॥

\twolineshloka
{अन्यानपि खलान् हत्वा स्वप्रभावमदर्शयम्}
{ततश्च मथुरां गत्वा हत्वा कंसादिदानवान्}%॥४३॥

\twolineshloka
{ज्ञातीनां परमं हर्ष कृतवानस्मि सादरम्}
{देवकीवसुदेवौ च परिष्वज्य मुदा ममं}%॥४४॥

\twolineshloka
{आनन्दर्जलै निसेचयामासतुप}
{तस्मिन् रङ्गबरे मल्लान् हत्वा चाणूरमुख्यकान्}%॥४५॥

\twolineshloka
{गज कुवलयापीडं कसभ्रातृननेकशः}
{एवं हतेऽसुरे कसे सर्वलोकैककण्टके}%॥४६॥

\twolineshloka
{अन्येषु दुष्टदैत्येषु सर्वलोका भयंकरम्}
{लोकाः समुत्सुकाः सर्वे मांसमेत्योचुराहताः}%॥४७॥

\twolineshloka
{कृष्ण कृष्ण महायोगिन् भक्तानामभयप्रद}
{प्रलयात्पाहि नो देव शरणागतवत्सलः}%॥४८॥

\twolineshloka
{अनाथनाथ सर्वज्ञ माहिने रत}
{किंचिद्विज्ञाप्यतेऽस्माभिस्तन्नो वक्तुं त्वमर्हसि}%॥४९॥

\twolineshloka
{तव जन्मदिनं लोके न ज्ञातं केनचित्क्वचित्}
{ज्ञात्वा च तत्त्वतः सर्वे कुर्मों वर्धापनोत्सवम्}%॥५०॥

\twolineshloka
{तेषां दृष्ट्रा तु तां भक्तिं श्रद्धामपि च सौहृदम्}
{मया जन्मदिनं तेभ्यः ख्यातं निर्मलचेतसा}%॥५१॥

\twolineshloka
{श्रुत्वा तेऽपि तथा चर्विधिना येन तच्छृणु}
{पार्थ तद्दिवसे प्राप्ते दत्तधावनपूर्वकम्}%॥५२॥

\twolineshloka
{स्नात्वा पुण्यजले शुद्ध वाससी परिधाय च}
{निर्वावश्यकं कर्म व्रतसङ्कल्पमाचरेत्}%॥५३॥

\twolineshloka
{अद्य स्थित्वा निराहारः श्वोभूते तु परेऽहनि}
{भोक्ष्यामि पुण्डरीकाक्ष शरणं मे भवाव्यय}%॥५४॥

\twolineshloka
{गृहीत्वा नियमं चैव संपाद्यार्चनसाधनम्}
{मण्डपं शोभनं कृत्वा फलपुष्पादिभिर्युतम्}%॥५५॥

\twolineshloka
{तस्मिन्मां पूजयेद्भक्त्या गन्धपुष्पादिभिःक्रमात्}
{उपचारैः षोडशभिादशाक्षरविद्यया}%॥५६॥

\twolineshloka
{सद्यःप्रसूतां जननी वसुदेवं च मारिषः}
{बलदेवसमायुक्तां रोहिणी गुणशोभिनीम्}%॥५७॥

\twolineshloka
{नन्दं यशोदां गोपीश्च गोपान् गाश्चैव सर्वशः}
{गोकुलं यमुनां चैव योगमायां च दारिकाम्}%॥५८॥

\twolineshloka
{यशोदाशयने सुप्तां सद्योजातां वरप्रभाम्}
{एवं संपूजयेत्सम्यङ्नाममन्त्रैः पृथक्पृथक्}%॥५९॥

\twolineshloka
{तुवर्णरौ शताबारम्बादिमिरलंकृताः}
{काष्ठपाषाणरचिताश्चित्रमय्योथ लेखिताः}%॥६॥

\twolineshloka
{प्रतिमा विविधाः प्रोक्तास्तासु चान्यतनां यजेत्}
{रात्रौ जागरणं कुर्याद्गीतनृत्यादिभिः सह}%॥६॥

\twolineshloka
{पुराणेः लोत्रपाठश्च जातनामादिसूत्सवैः}
{श्वभूते पारणं कुर्याद्विजान् संभोज्य यत्नतः}%॥६॥

\twolineshloka
{एवं कृते महाराज बतानामुत्तमे व्रते}
{सर्वान्कामानवाप्नोति विष्णुलोके महीयते}%॥६३॥

\twolineshloka
{मोहान्न कुरुते यस्तु याति संसारगह्वरे}
{तस्मात्कुर्वन्प्रयत्नेन निष्पापो जायते नरः}%॥६४॥

\twolineshloka
{अत्रैवोदाहरन्तीममितिहासं पुरातनम्}
{अङ्गदेशोद्भवो राजा मित्रजिनाम नामतः}%॥६५॥

\twolineshloka
{तस्य पुत्रो महातेजः सत्यजित्सत्पथे स्थितः}
{पालयामास धर्मज्ञो विधिवद्रञ्जयन्प्रजाः}%॥६६॥

\twolineshloka
{तस्यैवं वर्तमानस्य कदाचिदैवयोगतः}
{पापण्डैः सहसंवालो बभूव बहुवासरम्}%॥६७॥

\twolineshloka
{तत्संसर्गात्स नृपतिरधर्मनिरतोऽभवत्}
{वेदशास्त्रपुराणानि विनिन्द्य बहुशो नृप}%॥६८॥

\twolineshloka
{ब्राह्मणेषु तथा धर्मे विद्वेषं परमं गतः}
{एवं बहुतिथे काले गते भरतसत्तम}%॥६९॥

\twolineshloka
{कालेन निधनप्राप्तो यमदूनवशं गतः}
{बद्धा पाशैनीयमानो यमदूतैर्यमान्तिकम्}%॥७०॥

\twolineshloka
{पीडितस्ताड्यमानोऽसौ दुष्टसङ्गवशं गतः}
{नरके पतितः पापो यातनां बहुवत्सरम्}%॥७१॥

\twolineshloka
{भुक्त्वा पापस्य शेषेण पैशाची योनिमास्थितः}
{तृषाक्षुधासमाक्रान्तो भ्रमन्स मरुधन्वसु}%॥७२॥

\twolineshloka
{कस्यचित्वथ वैश्यस्य देहमाविश्य संस्थितः}
{सह तेनैव संप्राप्तो मथुरा पुण्यदां पुरीम्}%॥७३॥

\twolineshloka
{तत्रत्यरक्षकः सोऽथ तदेहात्तु बहिष्कृतः}
{बनाम विपिने सोऽपि ऋषीणामाश्रमेष्वपि}%॥७४॥

\twolineshloka
{कदाचिदेवयोगेन मम जन्माष्टमीदिने}
{क्रियमाणां महापूजां व्रतिभिर्मुनिभिर्द्विजैः}%॥७५॥

\twolineshloka
{रात्रौ जागरणं चैव नामसंकीर्तनादिभिः}
{ददर्श सर्व विधिवच्छुश्राव च हरेः कथाः}%॥७६॥

\twolineshloka
{निष्पापस्तत्क्षणादेव शुद्धनिर्मलमानसः}
{प्रेतदेहं समुत्सृज्य विष्णु लोकं विमानतः}%॥७७॥

\twolineshloka
{मम दूतैः समानीतो दिव्यभोगसमन्वितः}
{मम सांनिध्यमापन्नो व्रतस्यास्य प्रभावतः}%॥७८॥

\twolineshloka
{नित्यमेव व्रतं चैतत् पुराणे सार्वकालिकम्}
{गीयते विधिवत्सम्यनुनिभिस्तत्वदर्शिभिः}%॥७९॥

\threelineshloka
{सार्वकालिकमेवैतत्कृत्वा कामानवाप्नुयात्}
{एतत्ते सर्वमाख्यातं व्रतानामुत्तमं व्रतम्}
{मम सान्निध्यकृद्राजन्कि भूयः श्रोतुमिच्छसि}%॥८०॥

\centerline{॥इति शिष्टाचारप्राप्ता जन्माष्टमीव्रतकथा}


\sect{व्रतोद्यापनम्}


\uvacha{युधिष्ठिर उवाच}
\twolineshloka
{उद्यापनविधि ब्रूहि सर्वदेव दयानिधे}
{येन संपूर्णतां याति व्रतमे तदनुत्तमम्}

\uvacha{श्रीकृष्ण उवाच}
\twolineshloka
{पूर्ण तिथिमनुप्राप्य वित्तचित्तादिसंथुतः}
{पूर्वेधुरेकभक्ताशी स्वपेन्मां संस्मरन्हदि}

\twolineshloka
{प्रातरुत्थाय संस्मृत्य पुण्यश्लोकान् समाहितः}
{निर्वावश्यकं कर्म ब्राह्मणास्वस्ति वाचयेत्}

\twolineshloka
{गुरुमानीय धर्मज्ञं वेदवेदाङ्गपारगम्}
{वृणुयादृत्विजश्चैव वस्त्रालङ्करणादेभिः}

\twolineshloka
{पलेन वा तदर्धन तदर्धार्धन वा पुनः}
{शक्त्या वापि नृपश्रेष्ठ वित्तशाम्यविवर्जितः}

\twolineshloka
{सौवर्णी प्रतिमां कुर्यात्पाद्यार्थ्याचमनीयकम्}
{पात्रं संपाद्य विधिवत्पूजोपकरणं तथा}

\twolineshloka
{गोचर्ममात्र संलिप्य मध्ये मण्डलमाचरेत्}
{ब्रह्माद्या देवतास्तत्र स्थापयित्वा प्रपूजयेत्}

\twolineshloka
{मण्डपं रचयत्तत्र कदलीस्तम्भमण्डितम्}
{चतुरिसमोपेतं फलपुष्पादिशोभितम्}

\twolineshloka
{वितानं तत्र बनीयाद्विचित्रं चैव शोभनम्}
{मण्डले स्थापयेत्कुम्भं तानं वा मृन्मयं शुचिम्}

\twolineshloka
{तस्योपरि न्यसेत्पात्रं राजतं वैष्णवं तु वा}
{वाससाच्छाद्य कौन्तेय पूजयेत्तत्र मां बुधः}

\twolineshloka
{उपचारैः षोडशभिर्मन्त्रैरेतैः समाहितः}
{ध्यात्वावाह्यामृतीकृत्य स्वागतादिभिरादरावा}

\threelineshloka
{ध्यायेच्चतुर्भुज देवं शङ्खचक्रगदाधरम्}
{पीताम्बरयुगोपेतं लक्ष्मीयुक्तं विभूषितम्}
{लसत्कौस्तुभशोभाब्यं मेघश्यामं सुलोचनम्}
ध्यानम्॥

आगच्छ देवदेवेश जगद्योने रमापते॥
शुद्ध ह्यस्मिन्नधिष्ठाने संनिधेहि कृपां कुरु॥
आवाह०॥

देवदेव जगन्नाथ गरुडासनसंस्थित॥
गृहाण चासनं दिव्यं जगद्धातर्नमोऽस्तु ते॥
आसनम्॥

नानातीर्थाहतं तोयं निर्मलं पुष्षमिश्रितम्॥
पाद्यं गृहाण देवेश विश्वरूप नमोस्तु ते॥
पाद्यम्॥

गङ्गादिसर्वतीर्थेभ्यो भक्त्यानीतं सुशीतलम्॥
गन्धपुष्पाक्षतोपेतं गृहाणाय नमोऽस्तु ते॥
अर्घ्यम्॥

कृष्णावेणीसमद्भतं कालिन्दीजलसंयुतम्॥
गृहाणाचमनं देव विश्वकाय नमोऽस्तु ते॥
आचमनम्॥

दधि क्षौद्रं वृतं शुद्ध कपिलायाः सुगन्धि यत्॥
सुस्वादु मधुरं शौरे मधुपर्क गृहाण मे॥
मधुपर्कम्॥
पुनराचमनम्॥

पञ्चामृतेन स्नपनं करिष्यामि सुरोत्तम॥
क्षीरोदधिनिवासाय लक्ष्मीकान्ताय ते नमः॥
पञ्चामृत०॥

मन्दाकिनी गौतमी च यमुना च सरस्वती॥
ताभ्यः स्नानार्थमानीतं गृहाण शिशिरं जलम्॥
स्नानम्॥
पुनराचमनम्॥

शुद्धजाम्बूनदप्रख्ये तडिद्भासुररोचिषी॥
मयोपपादिते तुभ्यं वाससी प्रतिगृह्यताम्॥
वस्त्रयुग्मम्॥

यज्ञोपवीतमिति यज्ञोपवीतम्॥
किरीटकुण्डलादीनि काञ्चीवलययुग्मकम्॥

कौस्तुभं वनमालां च भूषणानि भजस्व मे॥
भूषणानि॥

मलयाचलसंभूतं घनसारं मनोहरम्॥
हृदयानन्दनं चारु चन्दनं प्रतिगृह्यताम्॥
चन्दनम्॥

अक्षताश्च सुरश्रेष्ठति कुंकुमाक्षतान्॥
मालतीचम्पकादीनि यूथिकाबकुलानि च॥
तुलसीपत्रमिश्राणि गृहाण सुरसत्तम॥
पुष्पाणि॥



अथाङ्गपूजा-
अघनाशनाय०पादौ पू०
वामनाय गुल्फो० पू०।
शौरये० जंघे पूछ।
वैकुण्ठवासिने० ऊरू पू०।
पुरुषोत्तमाय मे पू०।
वासुदेवाय० कटी पू।
हृषीकेशाय नाभिं पू०।
माधवाय० हृदयं पू०।
मधुसूदनाय० कण्ठं पू०।
वराहाय० बाहू पू०।
नृसिंहाय० हस्तौ पू०।
दैत्यसूदनाय० मुखं पू०।
दामोदरायः नासिकां पू०।
पुण्डरीकाक्षाय नेत्रे पू०।
गरुडध्वजाय० श्रोत्रे पू०।
गोविन्दाय० ललाटं पू०।
अच्युताय० शिरः पू०।
कृष्णाय सर्वाङ्गं पू०॥


अथ परिवारदेवतापूजा

देवकी वसुदेवं च रोहिणी सबलां तथा॥
सात्यार्क चोरालेवाडियादवान्॥
नन्दं यशोदां तत्कालप्रसूतां गोपगोपिकाः॥
कालिन्दी कालियं चैव पूजयेन्नाममन्त्रतः॥
वनस्पतिरसोद्भतं कालागुरुसमन्वितम्॥

धूपं गृहाण गोविन्द गुणसागर गोपते॥
धूपम्॥

साज्यं च वर्तिसंयुक्तम्॥
दीपम्॥

शाल्योदनं पायसं च सिताघृतविमिश्रितम्॥
नानापक्कासंयुक्तं नैवेद्यं प्रतिगृह्यताम्॥
नैवेद्यम्।
उत्तरापोशनम्॥

इदं फल मिति फलम्॥

पूगीफलमिति तांबूलम्॥

हिरण्यगर्भेति दक्षिणाम्॥

नीराजयेत्ततो भक्त्या मङ्गले समुदीरयन्॥
जयमङ्गलनिघोषैर्देवदेवं समर्चयेत्॥
नीराजनम्॥


दत्त्वा पुष्पाञ्जलिं चैव प्रदक्षिणपुरःसरम्॥
प्रणमेदण्डवद्भौ भक्तिप्रहः पुनःपुनः

स्तुत्वा नानाविधेः स्तोत्रैः प्रार्थयेत जगत्पतिम्॥

नमस्तुभ्यं जगन्नाथ देवकीतनय प्रभो॥
वसुदेवात्मजानन्त यशोदानन्दवर्द्धन॥

गोविन्द गोकुलाधार गोपीकान्त नमोऽस्तु ते॥
ततस्तु दापयेदय॑मिन्दोरुदयतः शुचिः॥

कृष्णाय प्रथमं दद्यादेवकीसहिताय च॥
नालिकेरेण शुद्धेन मुक्तमध्ये विचक्षण॥
कृष्णाय परया भक्त्या शङ्के कृत्वा विधानतः॥

जातः कंसवधार्थाय भूभारोत्तारणाय च॥
कौरवाणां विनाशाय दैत्यानां निधनाय च॥

पाण्डवानां हितार्थाय धर्मसंस्थापनाय च॥
गृहाणाऱ्या मया दत्तं देवकीसहितो हरे॥
कृष्णार्घ्यमन्त्रः॥

शङ्खे कृत्वा ततस्तोयं सपुष्पफलचन्दनम्॥
जानुभ्यामवनिं गत्वा चन्द्रायाय निवेदयेत्॥

क्षीरोदार्णवसंभूत अत्रिगोत्रसमुद्भव॥
गृहाणार्य मया दत्तं रोहिण्या सहित प्रभो॥

ज्योत्स्नापते नमस्तुभ्यं नमस्ते ज्योतिषांपते॥
नमस्ते रोहिणीकान्त गृहाणाय नमोस्तु ते॥

चन्द्रायमन्त्रः॥

\twolineshloka
{इत्थं संपूज्य देवेशं रात्रौ जागरणं चरेत्}
{गीतनृत्यदिना चैव पुराणश्रवणादिभिः}

\twolineshloka
{प्रत्यूषे विमले स्नात्वा पूजयित्वा जगद्गुरुम्}
{पायसेन तिलाज्यश्च मूलमन्त्रेण भक्तितः}

\twolineshloka
{अष्टोत्तरशतं हुत्वा ततः पुरुषसूक्ततः}
{इदं विष्णुरिति प्रोक्त्वा जुहुयाद्वै घृताहुतीः}

\twolineshloka
{होम-शेष समाप्याथ पूर्णाहुतिपुरःसरम्}
{आचार्य पूजयेद्भक्त्या भूषणाच्छादनादिभिः}

\twolineshloka
{गामैकां कपिलां दद्याद्वतसंपूतिहेतवे}
{पयस्विनी सुशीला च सवत्सां सगुणां तथा}

\twolineshloka
{स्वर्णशृङ्गी रौप्यखुरां कांस्यदोहनिकायुताम्}
{रत्नपुच्छां ताम्रपृष्ठीं स्वर्णघण्टासमन्विताम्}

\twolineshloka
{वस्त्रच्छन्नी दक्षिणाढ्यामेवं सम्पूर्णतां व्रजेत्}
{कपिलाया अभावे तु गौरन्यापि प्रदीयते}

\twolineshloka
{ततो दद्याञ्च ऋत्वि ग्भ्योऽन्येभ्यश्चैव यथाविधि}
{शय्यां सोपस्करां दद्याद्रतसम्पूर्तिहेतवे}

\twolineshloka
{ब्राह्मणान्भोजयेत्पश्चादष्टौ तेभ्यश्च दक्षिणाम्}
{कलशानन्नसम्पूर्णान्दद्याच्चैव समाहितः}

\twolineshloka
{दीनान्धकृपणांश्चैव यहि प्रतिपूजयेत्}
{प्राप्यानुज्ञा तथा तेभ्यो भुञ्चीत सह बन्धुभिः}

\twolineshloka
{एवंकृते महाराज व्रतोद्यापनकर्मणि}
{निष्पापस्तत्क्षणादेव जायते विबुधोपमः}

\twolineshloka
{पुत्रपौत्रसमायुक्तो धनधान्यसमन्वितः}
{भुक्त्वा भोगांश्चिरं कालमन्ते मम पुरं व्रजेत्}

इति श्रीभविष्यपुराणे कृष्णयुधिष्ठिरसंवादे जन्माष्टमीव्रतोद्यापनं सम्पूर्णम्॥
