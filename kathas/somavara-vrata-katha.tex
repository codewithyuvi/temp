\sect{कथा}

\uvacha{सूत उवाच}
\twolineshloka
{नित्यानन्दमयं शान्तं निर्विकल्पं निरामयम्}
{शिवतत्त्वमनाद्यन्तं ये विदुस्ते परं गताः} %॥१॥

\twolineshloka
{विरक्ताः कामभोगेभ्यो ये प्रकुर्वन्त्यहैतुकीम्}
{भक्तिं परां शिवे धीरास्तेषां मुक्तिर्न संसृतिः} %॥२॥

\twolineshloka
{विषयानभिसन्धाय ये कुर्वन्ति शिवे रतिम्}
{विषयैर्नाभिभूयन्ते भुञ्जानास्तत्फलान्यपि} %॥३॥

\twolineshloka
{येन केनापि भावेन शिवभक्तियुतो नरः}
{न विनश्यति कालेन स याति परमां गतिम्} %॥४॥

\twolineshloka
{आरुरुक्षुः परं स्थानं विषयासक्तमानसः}
{पूजयेत्कर्मणा शम्भुं भोगान्ते शिवमाप्नुयात्} %॥५॥

\twolineshloka
{अशक्तः कश्चिदुत्स्रष्टुं प्रायो विषयवासनाम्}
{अतः कर्ममयी पूजा कामधेनुः शरीरिणाम्} %॥६॥

\twolineshloka
{मायामयेऽपि संसारे ये विहृत्य चिरं सुखम्}
{मुक्तिमिच्छन्ति देहान्ते तेषां धर्मोऽयमीरितः} %॥७॥

\twolineshloka
{शिवपूजा सदा लोके हेतुः स्वर्गापवर्गयोः}
{सोमवारे विशेषेण प्रदोषादिगुणान्विते} %॥८॥

\twolineshloka
{केवलेनापि ये कुर्युः सोमवारे शिवार्चनम्}
{न तेषां विद्यते किञ्चिदिहामुत्र च दुर्लभम्} %॥९॥

\twolineshloka
{उपोषितः शुचिर्भूत्वा सोमवारे जितेन्द्रियः}
{वैदिकैर्लौकिकैर्वाऽपि विधिवत्पूजयेच्छिवम्} %॥१०॥

\twolineshloka
{ब्रह्मचारी गृहस्थो वा कन्या वाऽपि सभर्तृका}
{विभर्तृका वा सम्पूज्य लभते वरमीप्सितम्} %॥११॥

\twolineshloka
{अत्राहं कथयिष्यामि कथां श्रोतृमनोहराम्}
{श्रुत्वा मुक्तिं प्रयान्त्येव भक्तिर्भवति शाम्भवी} %॥१२॥

\twolineshloka
{आर्यावर्ते नृपः कश्चिदासीद्धर्मभृतां वरः}
{चित्रवर्मेति विख्यातो धर्मराजो दुरात्मनाम्॥} %॥१३॥

\twolineshloka
{स गोप्ता धर्मसेतूनां शास्ता दुष्पथगामिनाम्}
{यष्टा समस्तयज्ञानां त्राता शरणमिच्छताम्} %॥१४॥

\twolineshloka
{कर्ता सकलपुण्यानां दाता सकलसम्पदाम्}
{जेता सपत्नवृन्दानां भक्तः शिवमुकुन्दयोः} %॥१५॥

\twolineshloka
{सोनुकूलासु पत्नीषु लब्ध्वा पुत्रान्महौजसः}
{चिरेण प्रार्थितां लेभे कन्यामेकां वराननाम्} %॥१६॥

\twolineshloka
{स लब्ध्वा तनयां दिष्ट्या हिमवानिव पार्वतीम्}
{आत्मानं देवसदृशं मेने पूर्णमनोरथम्} %॥१७॥

\fourlineindentedshloka
{स एकदा जातकलक्षणज्ञान्}{आहूय साधून्द्विजमुख्यवृन्दान्}
{कुतूहलेनाभिनिविष्टचेताः}{पप्रच्छ कन्याजनने फलानि} %॥१८॥

\twolineshloka
{अथ तत्राब्रवीदेको बहुज्ञो द्विजसत्तमः}
{एषा सीमन्तिनी नाम्ना कन्या तव महीपते} %॥१९॥

\twolineshloka
{उमेव माङ्गल्यवती दमयन्तीव रूपिणी}
{भारतीव कलाभिज्ञा लक्ष्मीरिव महागुणा॥} %॥२०॥

\twolineshloka
{सुप्रजा देवमातेव जानकीव धृतव्रता}
{रविप्रभेव सत्कान्तिश्चन्द्रिकेव मनोरमा} %॥२१॥

\twolineshloka
{दशवर्षसहस्राणि सह भर्त्रा प्रमोदते}
{प्रसूय तनयानष्टौ परं सुखमवाप्स्यति} %॥२२॥

\twolineshloka
{इत्युक्तवन्तं नृपतिर्धनैः सम्पूज्य तं द्विजम्}
{अवाप परमां प्रीतिं तद्वागमृतसेवया} %॥२३॥

\twolineshloka
{अथान्योऽपि द्विजः प्राह धैर्यवानमितद्युतिः}
{एषा चतुर्दशे वर्षे वैधव्यं प्रतिपत्स्यति} %॥२४॥

\twolineshloka
{इत्याकर्ण्य वचस्तस्य वज्रनिर्घातनिष्ठुरम्}
{मुहूर्तमभवद्राजा चिन्ताव्याकुलमानसः} %॥२५॥

\twolineshloka
{अथ सर्वान्समुत्सृज्य ब्राह्मणान्ब्रह्मवत्सलः}
{सर्वं दैवकृतं मत्वा निश्चिन्तः पार्थिवोऽभवत्} %॥२६॥

\twolineshloka
{सापि सीमन्तिनी बाला क्रमेण गतशैशवा}
{वैधव्यमात्मनो भावि शुश्रावाऽऽत्मसखीमुखात्} %॥२७॥

\twolineshloka
{परं निर्वेदमापन्ना चिन्तयामास बालिका}
{याज्ञवल्क्यमुनेः पत्नीं मैत्रेयीं पर्यपृच्छत} %॥२८॥

\twolineshloka
{मातस्त्वच्चरणाम्भोजं प्रपन्नाऽस्मि भयाकुला}
{सौभाग्यवर्धनं कर्म मम शंसितुमर्हसि} %॥२९॥

\twolineshloka
{इति प्रपन्नां नृपतेः कन्यां प्राह मुनेः सती}
{शरणं व्रज तन्वङ्गि पार्वतीं शिवसंयुताम्} %॥३०॥

\twolineshloka
{सोमवारे शिवं गौरीं पूजयस्व समाहिता}
{उपोषिता वा सुस्नाता विरजाम्बरधारिणी} %॥३१॥

\twolineshloka
{यतवाङ्निश्चलमनाः पूजां कृत्वा यथोचिताम्}
{ब्राह्मणान्भोजयित्वाऽथ शिवं सम्यक्प्रसादयत्} %॥३२॥

\twolineshloka
{पापक्षयोऽभिषेकेण साम्राज्यं पीठपूजनात्}
{सौभाग्यमखिलं सौख्यं गन्धमाल्याक्षतार्पणात्} %॥३३॥

\twolineshloka
{धूपदानेन सौगन्ध्यं कान्तिर्दीपप्रदानतः}
{नैवेद्यैश्च महाभोगो लक्ष्मीस्ताम्बूलदानतः} %॥३४॥

\twolineshloka
{धर्मार्थकाममोक्षाश्च नमस्कारप्रदानतः}
{अष्टैश्वर्यादिसिद्धीनां जप एव हि कारणम्} %॥३५॥

\twolineshloka
{होमेन सर्वकामानां समृद्धिरुपजायते}
{सर्वेषामेव देवानां तुष्टिर्ब्राह्मणभोजनात्} %॥३६॥

\twolineshloka
{इत्थमाराधय शिवं सोमवारे शिवामपि}
{अत्यापदमपि प्राप्ता निस्तीर्णाभिभवा भवेः} %॥३७॥

\twolineshloka
{घोराद् घोरं प्रपन्नापि महाक्लेशं भयानकम्}
{शिवपूजाप्रभावेण तरिष्यसि महद्भयम्} %॥३८॥

\twolineshloka
{इत्थं सीमन्तिनीं सम्यगनुशास्य पुनः सती}
{ययौ साऽपि वरारोहा राजपुत्री तथाऽकरोत्} %॥३९॥

\twolineshloka
{दमयन्त्यां नलस्यासीदिन्द्रसेनाभिधः सुतः}
{तस्य चन्द्राङ्गदो नाम पुत्रोऽभूच्चन्द्रसन्निभः} %॥४०॥

\twolineshloka
{चित्रवर्मा नृपश्रेष्ठस्तमाहूय नृपात्मजम्}
{कन्यां सीमन्तिनीं तस्मै प्रायच्छद्गुर्वनुज्ञया} %॥४१॥

\twolineshloka
{सोऽभून्महोत्सवस्तत्र तस्या उद्वाहकर्मणि}
{यत्र सर्वमहीपानां समवायो महानभूत्} %॥४२॥

\twolineshloka
{तस्याः पाणिग्रहं काले कृत्वा चन्द्राङ्गदः कृती}
{उवास कतिचिन्मासांस्तत्रैव श्वशुरालये} %॥४३॥

\twolineshloka
{एकदा यमुनां तर्तुं स राजतनयो बली}
{आरुरोह तरीं कैश्चिद्वयस्यैः सह लीलया} %॥४४॥

\twolineshloka
{तस्मिंस्तरति कालिन्दीं राजपुत्रे विधेर्वशात्}
{ममज्ज सह कैवर्तैरावर्ताभिहता तरी} %॥४५॥

\twolineshloka
{हा हेति शब्दः सुमहानासीत्तस्यास्तटद्वये}
{पश्यतां सर्वसैन्यानां प्रलापो दिवमस्पृशत्} %॥४६॥

\twolineshloka
{मज्जन्तो मम्रिरे केचित्केचिद्ग्राहोदरं गताः}
{राजपुत्रादयः केचिन्नादृश्यन्त महाजले} %॥४७॥

\twolineshloka
{तदुपश्रुत्य राजाऽपि चित्रवर्माऽतिविह्वलः}
{यमुनायास्तटं प्राप्य विचेष्टः समजायत} %॥४८॥

\twolineshloka
{श्रुत्वाऽथ राजपत्न्यश्च बभूवुर्गतचेतनाः}
{सा च सीमन्तिनी श्रुत्वा पपाप भुवि मूर्च्छिता} %॥४९॥

\twolineshloka
{तथाऽन्ये मन्त्रिमुख्याश्च नायकाः सपुरोहिताः}
{विह्वलाः शोकसन्तप्ता विलेपुर्मुक्तमूर्धजाः} %॥५०॥

\twolineshloka
{इन्द्रसेनोऽपि राजेन्द्रः पुत्रवार्त्तां सुदुःखितः}
{आकर्ण्य सह पत्नीभिर्नष्टसंज्ञः पपात ह} %॥५१॥

\twolineshloka
{तन्मन्त्रिणश्च तत्पौरास्तथा तद्देशवासिनः}
{आबालवृद्धवनिताश्चुक्रुशुः शोकविह्वलाः} %॥५२॥

\twolineshloka
{शोकात्केचिदुरो जघ्नुः शिरो जघ्नुश्च केचन}
{हा राजपुत्र हा तात क्वासि क्वासीति बभ्रमुः} %॥५३॥

\twolineshloka
{एवं शोकाकुलं दीनमिन्द्रसेनमहीपतेः}
{नगरं सहसा क्षुब्धं चित्रवर्मपुरं तथा} %॥५४॥

\twolineshloka
{अथ वृद्धैः समाश्वस्तश्चित्रवर्मा महीपतिः}
{शनैर्नगरमागत्य सान्त्वयामास चाऽऽत्मजाम्} %॥५५॥

\twolineshloka
{स राजाऽम्भसिमग्नस्य जामातुस्तस्य बान्धवैः}
{आगतैः कारयामास साकल्यादौर्ध्वदैहिकम्} %॥५६॥

\twolineshloka
{सा च सीमन्तिनी साध्वी भर्तृलोकमतिः सती}
{पित्रा निषिद्धा स्नेहेन वैधव्यं प्रत्यपद्यत} %॥५७॥

\twolineshloka
{मुनेः पत्न्योऽपदिष्टं यत्सोमवारव्रतं शुभम्}
{न तत्याज शुभाचारा वैधव्यं प्राप्तवत्यपि} %॥५८॥

\twolineshloka
{एवं चतुर्दशे वर्षे दुःखं प्राप्य सुदारुणम्}
{ध्यायन्ती शिवपादाब्जं वत्सरत्रयमत्यगात्} %॥५९॥

\twolineshloka
{पुत्रशोकादिवोन्मत्तमिन्द्रसेनं महीपतिम्}
{प्रसह्य तस्य दायादाः सप्ताङ्गं जह्रुरोजसा} %॥६०॥

\twolineshloka
{हृतसिंहासनः शूरैर्दायादैः सोऽप्रजो नृपः}
{निगृह्य काराभवने सपत्नीको निवेशितः॥} %॥६१॥

\twolineshloka
{चन्द्रागदोऽपि तत्पुत्रो निमग्नो यमुनाजले}
{अधोधोमज्जमानोऽसौ ददर्शोरगकामिनीः} %॥६२॥

\twolineshloka
{जलक्रीडासु सक्तास्ता दृष्ट्वा राजकुमार कम्}
{विस्मितास्तमथो निन्युः पातालं पन्नगालयम्} %॥६३॥

\twolineshloka
{स नीयमानस्तरसा पन्नगीभिर्नृपात्मजः}
{तक्षकस्य पुरं रम्यं विवेश परमाद्भुतम्॥} %॥६४॥

\twolineshloka
{सोऽपश्यद्राजतनयो महेन्द्रभवनोपमम्}
{महारत्नपरिभ्राजन्मयूखपरिदीपितम्} %॥६५॥

\twolineshloka
{वज्रवैडूर्यपाचादिप्रासादशतसङ्कुलम्}
{माणिक्यगोपुरद्वारं मुक्तादामभिरुज्ज्वलम्} %॥६६॥

\twolineshloka
{चन्द्रकान्तस्थलं रम्यं हेमद्वारकपाटकम्}
{अनेकशतसाहस्रमणिदीपविराजितम्} %॥६७॥

\twolineshloka
{तत्रापश्यत्सभामध्ये निषण्णं रत्नविष्टरे}
{तक्षकं पन्नगाधीशं फणानेकशतोज्ज्वलम्} %॥६८॥

\twolineshloka
{दिव्याम्बरधरं दीप्तं रत्नकुण्डलराजितम्}
{नानारत्नपरिक्षिप्तमुकुटद्युतिरञ्जितम्} %॥६९॥

\twolineshloka
{फणामणिमयूखाढ्यैरसङ्ख्यैः पन्नगोत्तमैः}
{उपासितं प्राञ्जलिभिश्चित्ररत्नविभूषितैः} %॥७०॥

\twolineshloka
{रूपयौवनमाधुर्यविलासगति शोभिना}
{नागकन्यासहस्रेण समन्तात्परिवारितम्} %॥७१॥

\twolineshloka
{दिव्याभरणदीप्ताङ्गं दिव्यचन्दनचर्चितम्}
{कालाग्निमिव दुर्धर्षं तेजसाऽऽदित्यसन्निभम्॥} %॥७२॥

\twolineshloka
{दृष्ट्वा राजसुतो धीरः प्रणिपत्य सभास्थले}
{उत्थितः प्राञ्जलिस्तस्य तेजसाऽऽक्षिप्तलोचनः} %॥७३॥

\twolineshloka
{नागराजोऽपि तं दृष्ट्वा राजपुत्रं मनोरमम्}
{कोऽयं कस्मादिहायात इति पप्रच्छ पन्नगीः} %॥७४॥

\twolineshloka
{ता ऊचुर्यमुनातोये दृष्टोऽस्माभिर्यदृच्छया}
{अज्ञातकुलनामायमानीतस्तव सन्निधिम्} %॥७५॥

\twolineshloka
{अथ पृष्टो राजपुत्रस्तक्षकेण महात्मना}
{कस्यासि तनयः कस्त्वं को देशः कथमागतः} %॥७६॥

\onelineshloka
{राजपुत्रो वचः श्रुत्वा तक्षकं वाक्यमब्रवीत्} %॥७७॥

\uvacha{राजपुत्र उवाच}
\threelineshloka
{अस्ति भूमण्डले कश्चिद्देशो निषधसंज्ञकः}
{तस्याधिपोऽभवद्राजा नलो नाम महायशाः}
{स पुण्यकीर्तिः क्षितिपो दमयन्तीपतिः शुभः} %॥७८॥

\threelineshloka
{तस्मादपीन्द्रसेनाख्यस्तस्य पुत्रो महाबलः}
{चन्द्राङ्गदोऽस्मि नाम्नाऽहं नवोढः श्वशुरालये}
{विहरन्यमुनातोये निमग्नो देवचोदितः} %॥७९॥

\twolineshloka
{एताभिः पन्नगस्त्रीभिरानीतोऽस्मि तवान्तिकम्}
{दृष्ट्वाऽहं तव पादाब्जं पुण्यैर्जन्मान्तरार्जितैः} %॥८०॥

\twolineshloka
{अद्य धन्योऽस्मि धन्योऽस्मि कृतार्थो पितरौ मम}
{यत्प्रेक्षितोऽहं कारुण्यात्त्वया सम्भाषितोऽपि च} %॥८१॥

\uvacha{सूत उवाच}
\twolineshloka
{इत्युदारमसम्भ्रान्तं वचः श्रुत्वाऽतिपेशलम्}
{तक्षकः पुनरौत्सुक्याद्बभाषे राजनन्दनम्} %॥८२॥

\uvacha{तक्षक उवाच}
\twolineshloka
{भो भो नरेन्द्रदायाद मा भैषीर्धीरतां व्रज}
{सर्वदेवेषु को देवो युष्माभिः पूज्यते सदा} %॥८३॥

\uvacha{राजपुत्र उवाच}
\twolineshloka
{यो देवः सर्वेदेवेषु महादेव इति स्मृतः}
{पूज्यते स हि विश्वात्मा शिवोऽस्माभिरुमापतिः} %॥८४॥

\twolineshloka
{यस्य तेजोंशलेशेन रजसा च प्रजापतिः}
{कृतरूपोऽसृजद्विश्वं स नः पूज्यो महेश्वरः} %॥८५॥

\twolineshloka
{यस्यांशात्सात्त्विकं दिव्यं बिभ्रद्विष्णुः सनातनः}
{विश्वं बिभर्ति भूतात्मा शिवोऽस्माभिः स पूज्यते॥} %॥८६॥

\twolineshloka
{यस्यांशात्तामसाज्जातो रुद्रः कालाग्निसन्निभः}
{विश्वमेतद्धरत्यन्ते स पूज्योऽस्माभिरीश्वरः} %॥८७॥

\twolineshloka
{यो विधाता विधातुश्च कारणस्यापि कारणम्}
{तेजसां परमं तेजः स शिवो नः परा गतिः} %॥८८॥

\twolineshloka
{योऽन्तिकस्थोऽपि दूरस्थः पापोपहृतचेतसाम्}
{अपरिच्छेद्य धामासौ शिवो नः परमा गतिः} %॥८९॥

\twolineshloka
{योऽग्नौ तिष्ठति यो भूमौ यो वायौ सलिले च यः}
{य आकाशे च विश्वात्मा स पूज्यो नः सदाशिवः} %॥९०॥

\twolineshloka
{यः साक्षी सर्वभूतानां य आत्मस्थो निरञ्जनः}
{यस्येच्छावशगो लोकः सोऽस्माभिः पूज्यते शिवः} %॥९१॥

\fourlineindentedshloka
{यमेकमाद्यं पुरुषं पुराणं}
{वदन्ति भिन्नं गुणवैकृतेन}
{क्षेत्रज्ञमेकेऽथ तुरीयमन्ये}
{कूटस्थमन्ये स शिवो गतिर्नः} %॥९२॥

\fourlineindentedshloka
{यं नास्पृशंश्चैत्यमचिन्त्यतत्त्वं}
{दुरन्तधामानमतत्स्वरूपम्}
{मनोवचोवृत्तय आत्मभाजां}
{स एष पूज्यः परमः शिवो नः} %॥९३॥

\fourlineindentedshloka
{यस्य प्रसादं प्रतिलभ्य सन्तो}
{वाञ्छन्ति नैन्द्रं पदमुज्ज्वलं वा}
{निस्तीर्णकर्मार्गलकालचक्राः}
{चरन्त्यभीताः स शिवो गतिर्नः} %॥९४॥

\fourlineindentedshloka
{यस्य स्मृतिः सकलपापरुजां विघातं}
{सद्यः करोत्यपि चु पुल्कसजन्मभाजाम्}
{यस्य स्वरूपमखिलं श्रुतिभिर्विमृग्यं}
{तस्मै शिवाय सततं करवाम पूजाम्} %॥९५॥

\fourlineindentedshloka
{यन्मूर्ध्नि लब्धनिलया सुरलोकसिन्धुः}
{यस्यांङ्गगा भगवती जगदम्बिका च}
{यत्कुण्डले त्वहह तक्षकवासुकी द्वौ}
{सोऽस्माकमेव गतिरर्धशशाङ्कमौलिः} %॥९६॥

\fourlineindentedshloka
{जयति निगमचूडाग्रेषु यस्याङ्घ्रिपद्मं}
{जयति च हृदि नित्यं योगिनां यस्य मूर्तिः}
{जयति सकलतत्त्वोद्भासनं यस्य मूर्तिः}
{स विजितगुणसर्गः पूज्यतेऽस्माभिरीशः} %॥९७॥

\uvacha{सूत उवाच}
\twolineshloka
{इत्याकर्ण्य वचस्तस्य तक्षकः प्रीतमानसः}
{जातभक्तिर्महादेवे राजपुत्रमभाषत} %॥९८॥

\uvacha{तक्षक उवाच}
\twolineshloka
{परितुष्टोऽस्मि भद्रं स्तात् तव राजेन्द्रनन्दन}
{बालोऽपि यत्परं तत्त्वं वेत्सि शैवं परात्परम्} %॥९९॥

\twolineshloka
{एष रत्नमयो लोक एताश्चारुदृशोऽबलाः}
{एते कल्पद्रुमाः सर्वे वाऽप्योमृतरसाम्भसः} %॥१००॥

\twolineshloka
{नात्र मृत्युभयं घोरं न जरारोगपीडनम्}
{यथेष्टं विहरात्रैव भुङ्क्ष्व भोगान्यथोचितान्} %॥१०१॥

\twolineshloka
{इत्युक्तो नागराजेन स राजेन्द्रकुमारकः}
{प्रत्युवाच परं प्रीत्या कृताञ्जलिरुदारधीः} %॥२॥

\twolineshloka
{कृतदारोऽस्म्यहं काले सुव्रता गृहिणी मम}
{शिव पूजापरा नित्यं पितरावेकपुत्रकौ} %॥३॥

\twolineshloka
{ते त्वद्य मां मृतं मत्वा शोकेन महताऽऽवृताः}
{प्रायः प्राणैर्वियुज्यन्ते दैवात्प्राणान्वहन्ति वा} %॥४॥

\twolineshloka
{अतो मया बहुतिथं नात्र स्थेयं कथञ्चन}
{तमेव लोकं कृपया मां प्रापयितुमर्हसि} %॥५॥

\fourlineindentedshloka
{इत्युक्तवन्तं नरदेवपुत्रं}
{दिव्यैर्वरान्नैः सुरपादपोत्थैः}
{आप्याययित्वा वरगन्धवासः}
{स्रग्रत्नदिव्याभरणैर्विचित्रैः} %॥६॥

\fourlineindentedshloka
{सन्तोषयित्वा विविधैश्च भोगैः}
{पुनर्बभाषे भुजगाधिराजः}
{यदा यदा त्वं स्मरसि त्वदग्रे}
{तदा तदाऽऽविष्क्रियते मयेति} %॥७॥

\twolineshloka
{पुनश्च राजपुत्राय तक्षकोऽश्वं च कामगम्}
{नानाद्वीपसमुद्रेषु लोकेषु च निरर्गलम्} %॥८॥

\twolineshloka
{दत्तवान्रत्नाभरणदिव्याभरणवाससाम्}
{वाहनाय ददावेकं राक्षसं पन्नगेश्वरः} %॥९॥

\twolineshloka
{तत्सहायार्थमेकं च पन्नगेन्द्रकुमारकम्}
{नियुज्य तक्षकः प्रीत्या गच्छेति विससर्ज तम्} %॥११०॥

\twolineshloka
{इति चन्द्राङ्गदः सोऽथ सङ्गृह्य विविधं धनम्}
{अश्वं कामगमारुह्य ताभ्यां सह विनिर्ययौ} %॥११॥

\twolineshloka
{स मूहूर्तादिवोन्मज्ज्य तस्मा देव सरिज्जलात्}
{विजहार तटे रम्ये दिव्यमारुह्य वाजिनम्} %॥१२॥

\twolineshloka
{अथास्मिन्समये तन्वी सा च सीमन्तिनी सती}
{स्नातुं समाययौ तत्र सखीभिः परिवारिता} %॥१३॥

\twolineshloka
{सा ददर्श नदीतीरे विहरन्तं नृपात्मजम्}
{रक्षसा नररूपेण नागपुत्रेण चान्वितम्} %॥१४॥

\twolineshloka
{दिव्यरत्नसमाकीर्णं दिव्य माल्यावतंसकम्}
{देहेन दिव्यगन्धेन व्याक्षिप्तदशयोजनम्} %॥१५॥

\twolineshloka
{तमपूर्वाकृतिं वीक्ष्य दिव्याश्वमधिसंस्थितम्}
{जडोन्मत्तेव भीतेव तस्थौ तन्न्यस्तलोचना} %॥१६॥

\twolineshloka
{तां च राजेन्द्रपुत्रोऽसौ दृष्टपूर्वामिति स्मरन्}
{निर्मुक्तकण्ठाभरणां कण्ठसूत्रविवर्जिताम्} %॥१७॥

\twolineshloka
{असंयोजितधम्मिल्लामङ्गरागविवर्जिताम्}
{त्यक्तनीलाञ्जनापाङ्गीं कृशाङ्गीं शोकदूषिताम्} %॥१८॥

\twolineshloka
{दृष्ट्वाऽवतीर्य तुरगादुपविष्टः सरित्तटे}
{तामाहूय वरारोहामुपवेश्येदमब्रवीत्} %॥१९॥

\twolineshloka
{का त्वं कस्य कलत्रं वा कस्यासि तनया सती}
{किमिदं तेऽङ्गने बाल्ये दुःसहं शोकलक्षणम्} %॥१२०॥

\twolineshloka
{इति स्नेहेन सम्पृष्टा सा वधूरश्रुलोचना}
{लज्जिता स्वयमाख्यातुं तत्सखी सर्वमब्रवीत्} %॥२१॥

\twolineshloka
{इयं सीमन्तिनी नाम्ना स्नुषा निषधभूपतेः}
{चन्द्राङ्गदस्य महिषी तनया चित्रवर्मणः} %॥२२॥

\twolineshloka
{अस्याः पतिर्दैवयोगान्निमग्नोऽस्मिन्महाजले}
{तेनेयं प्राप्तवैधव्या बाला दुःखेन शोषिता} %॥२३॥

\twolineshloka
{एवं वर्षत्रयं नीतं शोकेनातिबलीयसा}
{अद्येन्दुवारे सम्प्राप्ते स्नातुमत्र समागता} %॥२४॥

\twolineshloka
{श्वशुरोऽस्याश्च राजेन्द्रो हृतराज्यश्च शत्रुभिः}
{बलाद्गृहीतो बद्धश्च सभार्यस्तद्वशे स्थितः} %॥२५॥

\twolineshloka
{तथाऽप्येषा शुभाचारा सोमवारे महेश्वरम्}
{साम्बिकं परया भक्त्या पूजयत्यमलाशया} %॥२६॥

\uvacha{सूत उवाच}
\twolineshloka
{इत्थं सखीमुखेनैव सर्वमावेद्य सुव्रता}
{ततः सीमन्तिनी प्राह स्वयमेव नृपात्मजम्} %॥२७॥

\twolineshloka
{कस्त्वं कन्दर्पसङ्काशः काविमौ तव पार्श्वगौ}
{देवो नरेन्द्रः सिद्धो वा गन्धर्वो वाऽथ किन्नरः} %॥२८॥

\twolineshloka
{किमर्थं मम वृत्तान्तं स्नेहवानिव पृच्छसि}
{किं मां वेत्सि महाबाहो दृष्टवान्किमु कुत्रचित्} %॥२९॥

\twolineshloka
{दृष्टपूर्व इवाऽऽभासि मया च स्वजनो यथा}
{सर्वं कथय तत्त्वेन सत्यसारा हि साधवः} %॥१३०॥

\uvacha{सूत उवाच}
\fourlineindentedshloka
{एतावदुक्त्वा नरदेवपुत्री}
{सबाष्पकण्ठं सुचिरं रुरोद}
{मुमोह भूमौ पतिता सखीभिः}
{वृता न किञ्चित्कथितुं शशाक} %॥३१॥

\twolineshloka
{श्रुत्वा चन्द्राङ्गदः सर्वं प्रियायाः शोककारणम्}
{मुहूर्तमभवत्तूष्णीं स्वयं शोकसमाकुलः} %॥३२॥

\twolineshloka
{अथाश्वास्य प्रियां तन्वीं विविधैर्वाक्यनैपुणैः}
{सिद्धा नाम वयं देवाः कामगा इति सोऽब्रवीत्} %॥३३॥

\twolineshloka
{ततो बलादिवाकृष्य पाणिग्रहणशङ्किताम्}
{पुलकाञ्चितसर्वाङ्गीं तां कर्णे त्विदमब्रवीत्} %॥३४॥

\twolineshloka
{क्वापि लोके मया दृष्टस्तव भर्ता वरानने}
{त्वद्व्रताचरणात्प्रीतः सद्य एवागमिष्यति॥} %॥३५॥

\twolineshloka
{अपनेष्यति ते शोकं द्वित्रैरेव दिनैर्ध्रुवम्}
{एतच्छंसितुमायातस्तव भर्तुः सखाऽस्म्यहम्} %॥३६॥

\twolineshloka
{अत्र कार्यो न सन्देहः शपामि शिवपादयोः}
{तावत्त्वद्धृदये स्थेयं न प्रकाश्यं च कुत्रचित्} %॥३७॥

\twolineshloka
{सा तु तद्वचनं श्रुत्वा सुधाधाराशताधिकम्}
{सम्भ्रमोद्भ्रान्तनयना तमेव मुहुरैक्षत॥} %॥३८॥

\twolineshloka
{प्रेमबन्धानुगुणितं वाक्यं चाह रसायनम्}
{विभ्रमोदारसहितं मधुरापाङ्गवीक्षणम्} %॥३९॥

\threelineshloka
{स्वपाणिस्पर्शनोद्भिन्नपुलकाञ्चितविग्रहम्}
{पूर्वदृष्टानि चाङ्गेषु लक्षणानि स्वरादिषु}
{वयःप्रमाणं वर्णं च परीक्ष्यैनमतर्कयत्} %॥१४०॥

\twolineshloka
{एष एव पतिर्मे स्याद्ध्रुवं नान्यो भविष्यति}
{अस्मिन्नेव प्रसक्तं मे हृदयं प्रेमकातरम्} %॥१४१॥

\twolineshloka
{परलोकादिहायातः कथमेवं स्वरूपधृक्}
{दुर्भाग्यायाः कथं मे स्याद्भर्तुर्नष्टस्य दर्शनम्} %॥१४२॥

\twolineshloka
{स्वप्नोऽयं किमु न स्वप्नो भ्रमोऽयं किं तु न भ्रमः}
{एष धूर्तोऽथवा कश्चिद् यक्षो गन्धर्व एव वा} %॥१४३॥

\twolineshloka
{मुनिपत्न्या यदुक्तं मे परमापद्गताऽपि च}
{व्रतमेतत्कुरुष्वेति तस्य वा फलमेव वा} %॥४४॥

\twolineshloka
{यो वर्षायुतसौभाग्यं ममेत्याह द्विजोत्तमः}
{नूनं तस्य वचः सत्यं को विद्यादीश्वरं विना} %॥४५॥

\twolineshloka
{निमित्तानि च दृश्यन्ते मङ्गलानि दिनेदिने}
{प्रसन्ने पार्वतीनाथे किमसाध्यं शरीरिणाम्} %॥४६॥

\twolineshloka
{इत्थं विमृश्य बहुधा तां पुनर्मुक्तसंशयाम्}
{लज्जानम्रमुखीं कर्णे शशंसात्मप्रयोजनम्} %॥४७॥

\twolineshloka
{इमं वृत्तान्तमाख्यातुं तत्पित्रोः शोकतप्तयोः}
{गच्छामः स्वस्ति ते भद्रे सद्यः पतिमवाप्स्यसि} %॥४८॥

\twolineshloka
{इत्युक्त्वाऽश्वं समारुह्य जगाम नृपनन्दनः}
{ताभ्यां सह निजं राष्ट्रं प्रत्यपद्यत तत्क्षणात्} %॥४९॥

\twolineshloka
{स पुरोपवनाभ्याशे स्थित्वा तं फणिपुत्रकम्}
{विससर्जाऽऽत्मदायादान्नृपासनगतान्प्रति} %॥१५०॥

\twolineshloka
{स गत्वोवाच ताञ्छीघ्रमिन्द्रसेनो विमुच्यताम्}
{चन्द्राङ्गदस्तस्य सुतः प्राप्तोऽयं पन्नगालयात्} %॥१५१॥

\twolineshloka
{नृपासनं विमुञ्चन्तु भवन्तो न विचार्यताम्}
{नो चेच्चन्द्रागदस्याशु बाणाः प्राणान्हरन्ति वः} %॥१५२॥

\twolineshloka
{स मग्नो यमुनातोये गत्वा तक्षकमन्दिरम्}
{लब्ध्वा च तस्य साहाय्यं पुनर्लोकादिहागतः} %॥१५३॥

\twolineshloka
{इत्याख्यातमशेषेण तद्वृत्तान्तं निशम्य ते}
{साधुसाध्विति सम्भ्रान्ताः शशंसुः परिपन्थिनः} %॥५४॥

\fourlineindentedshloka
{अथेन्द्रसेनाय निवेद्य सत्वरं}
{नष्टस्य पुत्रस्य पुनः समागमम्}
{प्रसाद्य तं प्राप्तनरेश्वरासनं}
{दायादमुख्यास्तु भयं प्रपेदिरे} %॥५५॥

\twolineshloka
{अथ पौरजनाः सर्वे पुरोद्याने नृपात्मजम्}
{दृष्ट्वा राज्ञे द्रुतं प्रोचुर्लेभिरे च महाधनम्} %॥५६॥

\twolineshloka
{आकर्ण्य पुत्रमायान्तं राजाऽऽनन्दजलाप्लुतः}
{न व्यजानादिमं लोकं राज्ञी च परया मुदा} %॥५७॥

\twolineshloka
{अथ नागरिकाः सर्वे मन्त्रिवृद्धाः पुरोधसः}
{प्रत्युद्गम्य परिष्वज्य तमानिन्युर्नृपान्तिकम्} %॥५८॥

\twolineshloka
{अथोत्सवेन महता प्रविश्य निजमन्दिरम्}
{राजपुत्रः स्वपितरौ ववन्दे बाष्पमुत्सृजन्} %॥५९॥

\fourlineindentedshloka
{तं पादमूले पतितं स्वपुत्रं}
{विवेद नासौ पृथिवीपतिः क्षणम्}
{प्रबोधितोऽमात्यजनैः कथञ्चिद्}
{उत्थाय क्लिन्नेन हृदाऽऽलिलिङ्ग} %॥१६०॥

\fourlineindentedshloka
{क्रमेण मातॄरभिवन्द्य ताभिः}
{प्रवर्धिताशीः प्रणयाकुलाभिः}
{आलिङ्गितः पौरजनानशेषान्}
{सम्भावयामास स राजसूनुः} %॥६१॥

\twolineshloka
{तेषां मध्ये समासीनः स्ववृत्तान्तमशेषतः}
{पित्रे निवेदयामास तक्षकस्य च मित्रताम्} %॥६२॥

\twolineshloka
{दत्तं भुजङ्गराजेन रत्नादिधनसञ्चयम्}
{दिव्यं तद्राक्षसानीतं पित्रे सर्वं न्यवेदयत्} %॥६३॥

\twolineshloka
{राजपुत्रस्य चरितं दृष्ट्वा श्रुत्वा च विह्वलः}
{मेने स्नुषायाः सौभाग्यं महेशाराधनार्जितम्} %॥६४॥

\twolineshloka
{सौमाङ्गल्यमयीं वार्तामिमां निषधभूपतिः}
{चारैर्निवेदयामास चित्रवर्ममहीपतेः} %॥६५॥

\twolineshloka
{श्रुत्वाऽमृतमयीं वार्त्तां स समुत्थाय सम्भ्रमात्}
{तेभ्यो दत्त्वा धनं भूरि ननर्ताऽऽनन्दविह्वलः} %॥६६॥

\twolineshloka
{अथाहूय स्वतनयां परिष्वज्याश्रुलोचनः}
{भूषणैर्भूषयामास त्यक्तवैधव्यलक्षणाम्} %॥६७॥

\twolineshloka
{अथोत्सवो महानासीद्राष्ट्रग्रामपुरादिषु}
{सीमन्तिन्याः शुभाचारं शशंसुः सर्वतो जनाः} %॥६८॥

\twolineshloka
{चित्रवर्माऽथ नृपतिः समाहूयेन्द्रसेनजम्}
{पुनर्विवाहविधिना सुतां तस्मै न्यवेदयत्} %॥६९॥

\twolineshloka
{चन्द्राङ्गदोऽपि रत्नाद्यैरानीतैस्तक्षकालयात्}
{स्वां पत्नीं भूषयां चक्रे मर्त्यानामतिदुर्लभैः} %॥१७०॥

\twolineshloka
{अङ्गरागेण दिव्येन तप्तकाञ्चनशोभिना}
{शुशुभे सा सुगन्धेन दशयोजनगामिना} %॥७१॥

\twolineshloka
{अम्लानमालया शश्वत्पद्मकिञ्जल्कवर्णया}
{कल्पद्रुमोत्थया बाला भूषिता शुशुभे सती} %॥७२॥

\twolineshloka
{एवं चन्द्राङ्गदः पत्नीमवाप्य समये शुभे}
{ययौ स्वनगरीं भूयः श्वशुरेणानुमोदितः} %॥७३॥

\twolineshloka
{इन्द्रसेनोऽपि राजेन्द्रो राज्ये स्थाप्य निजात्मजम्}
{तपसा शिवमाराध्य लेभे संयमिनां गतिम्} %॥७४॥

\twolineshloka
{दशवर्षसहस्राणि सीमन्तिन्या स्वभार्यया}
{सार्धं चन्द्राङ्गदो राजा बुभुजे विषयान्बहून्} %॥७५॥

\threelineshloka
{प्रासूत तनयानष्टौ कन्यामेकां वराननाम्}
{रेमे सीमन्तिनी भर्त्रा पूजयन्ती महेश्वरम्}
{दिनेदिने च सौभाग्यं प्राप्तं चैवेन्दुवासरात्} %॥७६॥

\uvacha{सूत उवाच}
\twolineshloka
{विचित्रमिदमाख्यानं मया समनुवर्णितम्}
{भूयोऽपि वक्ष्ये माहात्म्यं सोमवारव्रतोदितम्} %॥१७७॥

{॥इति श्रीस्कान्दे महापुराणे एकाशीतिसाहस्र्यां संहितायां ब्रह्मोत्तरखण्डे सोमवारव्रतवर्णनं नामाष्टमोऽध्यायः॥}



\uvacha{ऋषय ऊचुः}
\resetShloka

\twolineshloka
{साधु साधु महाभाग त्वया कथितमुत्तमम्}
{आख्यानं पुनरन्यत्र विचित्रं वक्तुमर्हसि} %॥१॥

\uvacha{सूत उवाच}
\twolineshloka
{विदर्भविषये पूर्वमासीदेको द्विजोत्तमः}
{वेदमित्र इति ख्यातो वेदशास्त्रार्थवित्सुधीः} %॥२॥

\twolineshloka
{तस्यासीदपरो विप्रः सखा सारस्वताह्वयः}
{तावुभौ परमस्निग्धावेकदेशनिवासिनौ} %॥३॥

\twolineshloka
{वेदमित्रस्य पुत्रोऽभूत्सुमेधा नाम सुव्रतः}
{सारस्वतस्य तनयः सोमवानिति विश्रुतः} %॥४॥

\twolineshloka
{उभौ सवयसौ बालौ समवेषौ समस्थिती}
{समं च कृतसंस्कारौ समविद्यौ बभूवतुः} %॥५॥

\twolineshloka
{साङ्गानधीत्य तौ वेदांस्तर्कव्याकरणानि च}
{इतिहासपुराणानि धर्मशास्त्राणि कृत्स्नशः} %॥६॥

\twolineshloka
{सर्वविद्याकुशलिनौ बाल्य एव मनीषिणौ}
{प्रहर्षमतुलं पित्रोर्ददतुः सकलैर्गुणैः} %॥७॥

\twolineshloka
{तावेकदा स्वतनयौ तावुभौ ब्राह्मणोत्तमौ}
{आहूयावोचतां प्रीत्या षोडशाब्दौ शुभाकृती} %॥८॥

\twolineshloka
{हे पुत्रकौ युवां बाल्ये कृतविद्यौ सुवर्चसौ}
{वैवाहिकोऽयं समयो वर्तते युवयोः समम्} %॥९॥

\twolineshloka
{इमं प्रसाद्य राजानं विदर्भेशं स्वविद्यया}
{ततः प्राप्य धनं भूरि कृतोद्वाहौ भविष्यथः} %॥१०॥

\twolineshloka
{एवमुक्तौ सुतौ ताभ्यां तावुभौ द्विजनन्दनौ}
{विदर्भराजमासाद्य समतोषयतां गुणैः} %॥११॥

\twolineshloka
{विद्यया परितुष्टाय तस्मै द्विजकुमारकौ}
{विवाहार्थं कृतोद्योगौ धनहीनावशंसताम्} %॥१२॥

\twolineshloka
{तयोरपि मतं ज्ञात्वा स विदर्भमहीपतिः}
{प्रहस्य किञ्चित्प्रोवाच लोकतत्त्वविवित्सया} %॥१३॥

\twolineshloka
{आस्ते निषधराजस्य राज्ञी सीमन्तिनी सती}
{सोमवारे महादेवं पूजयत्यम्बिकायुतम्} %॥१४॥

\twolineshloka
{तस्मिन्दिने सपत्नीकान्द्विजाग्र्यान्वेदवित्तमान्}
{सम्पूज्य परया भक्त्या धनं भूरि ददाति च} %॥१५॥

\twolineshloka
{अतोऽत्र युवयोरैको नारीविभ्रमवेषधृक्}
{एकस्तस्या पतिर्भूत्वा जायेतां विप्रदम्पती} %॥१६॥

\twolineshloka
{युवां वधूवरौ भूत्वा प्राप्य सीमन्तिनीगृहम्}
{भुक्त्वा भूरि धनं लब्ध्वा पुनर्यातं ममान्तिकम्} %॥१७॥

\twolineshloka
{इति राज्ञा समादिष्टौ भीतौ द्विजकुमारकौ}
{प्रत्यूचतुरिदं कर्म कर्तुं नौ जायते भयम्} %॥१८॥

\twolineshloka
{देवतासु गुरौ पित्रोस्तथा राजकुलेषु च}
{कौटिल्यमाचरन्मोहात्सद्यो नश्यति सान्वयः} %॥१९॥

\twolineshloka
{कथमन्तर्गृहं राज्ञां छद्मना प्रविशेत्पुमान्}
{गोप्यमानमपिच्छद्म कदाचित्ख्यातिमेष्यति} %॥२०॥

\twolineshloka
{ये गुणाः साधिताः पूर्वं शीलाचारश्रुतादिभिः}
{सद्यस्ते नाशमायान्ति कौटिल्य पथगामिनः} %॥२१॥

\twolineshloka
{पापं निन्दा भयं वैरं चत्वार्येतानि देहिनाम्}
{छद्ममार्गप्रपन्नानां तिष्ठन्त्येव हि सर्वदा} %॥२२॥

\twolineshloka
{अत आवां शुभाचारौ जातौ च शुचिनां कुले}
{वृत्तं धूर्तजनश्लाघ्यं नाश्रयावः कदाचन} %॥२३॥

\uvacha{राजोवाच}
\twolineshloka
{दैवतानां गुरूणां च पित्रोश्च पृथिवीपतेः}
{शासनस्याप्यलङ्घ्यत्वात्प्रत्यादेशो न कर्हिचित्} %॥२४॥

\twolineshloka
{एतैर्यद्यत्समादिष्टं शुभं वा यदि वाऽशुभम्}
{कर्तव्यं नियतं भीतैरप्रमत्तैर्बुभूषुभिः॥} %॥२५॥

\twolineshloka
{अहो वयं हि राजानः प्रजा यूयं हि सम्मताः}
{राजाज्ञया प्रवृत्तानां श्रेयः स्यादन्यथा भयम्} %॥२६॥

\twolineshloka
{अतो मच्छासनं कार्यं भवद्भ्यामविलम्बितम्}
{इत्युक्तौ नरदेवेन तौ तथेत्यूचतुर्भयात्} %॥२७॥

\twolineshloka
{सारस्वतस्य तनयं सामवन्तं नराधिपः}
{स्त्रीरूपधारिणं चक्रे वस्त्राकल्पां जनादिभिः} %॥२८॥

\fourlineindentedshloka
{स कृत्रिमोद्भूतकलत्रभावः}
{प्रयुक्तकर्णाभरणाङ्गरागः}
{स्निग्धाञ्जनाक्षः स्पृहणीयरूपो}
{बभूव सद्यः प्रमदोत्तमाभः} %॥२९॥

\twolineshloka
{तावुभौ दम्पती भूत्वा द्विजपुत्रौ नृपाज्ञया}
{जग्मतुर्नैषधं देशं यद्वा तद्वा भवत्विति} %॥३०॥

\twolineshloka
{उपेत्य राजसदनं सोमवारे द्विजोत्तमैः}
{सपत्नीकैः कृतातिथ्यौ धौतपादौ बभूवतुः} %॥३१॥

\twolineshloka
{सा राज्ञी ब्राह्मणान्सर्वानुपविष्टान्वरासने}
{प्रत्येकमर्चयाञ्चक्रे सपत्नीकान्द्विजोत्तमान्} %॥३२॥

\twolineshloka
{तौ च विप्रसुतौ दृष्ट्वा प्राप्तौ कृतकदम्पती}
{ज्ञात्वा किञ्चिद्विहस्याथ मेने गौरीमहेश्वरौ} %॥३३॥

\twolineshloka
{आवाह्य द्विजमुख्येषु देवदेवं सदाशिवम्}
{पत्नीष्वावाहयामास सा देवीं जगदम्बिकाम्} %॥३४॥

\twolineshloka
{गन्धैर्माल्यैः सुरभिभिर्धूपैर्नीराजनैरपि}
{अर्चयित्वा द्विजश्रेष्ठान्नमश्चक्रे समाहिता} %॥३५॥

\twolineshloka
{हिरण्मयेषु पात्रेषु पायसं घृतसंयुतम्}
{शर्करामधुसंयुक्तं शाकैर्जुष्टं मनोरमैः} %॥३६॥

\twolineshloka
{गन्धशाल्योदनैर्हृद्यैर्मोदकापूपराशिभिः}
{शष्कुलीभिश्च संयावैः कृसरैर्माषपक्वकैः} %॥३७॥

\twolineshloka
{तथान्यैरप्यसङ्ख्यातैर्भक्ष्यैर्भोज्यैर्मनोरमैः}
{सुगन्धैः स्वादुभिः सूपैः पानीयैरपि शीतलैः} %॥३८॥

\twolineshloka
{क्लृप्तमन्नं द्विजाग्र्येभ्यः सा भक्त्या पर्यवेषयत्}
{दध्योदनं निरुपमं निवेद्य समतोषयत्} %॥३९॥

\twolineshloka
{भुक्तवत्सु द्विजाग्र्येषु स्वाचान्तेषु नृपाङ्गना}
{प्रणम्य दत्त्वा ताम्बूलं दक्षिणां च यथार्हतः} %॥४०॥

\twolineshloka
{धेनूर्हिरण्यवासांसि रत्नस्रग्भूषणानि च}
{दत्त्वा भूयो नमस्कृत्य विससर्ज द्विजोत्तमान्} %॥४१॥

\fourlineindentedshloka
{तयोर्द्वयोर्भूसुरवर्यपुत्रयोः}
{एकस्तया हैमवतीधियार्चितः}
{एको महादेवधियाभिपूजितः}
{कृतप्रणामौ ययतुस्तदाज्ञया} %॥४२॥

\twolineshloka
{सा तु विस्मृतपुम्भावा तस्मिन्नेव द्विजोत्तमे}
{जातस्पृहा मदोत्सिक्ता कन्दर्पविवशाऽब्रवीत्} %॥४३॥

\twolineshloka
{अंयि नाथ विशालाक्ष सर्वावयवसुन्दर}
{तिष्ठ तिष्ठ क्व वा यासि मां न पश्यसि ते प्रियाम्} %॥४४॥

\twolineshloka
{इदमग्रे वनं रम्यं सुपुष्पितमहाद्रुमम्}
{अस्मिन्विहर्तुमिच्छामि त्वया सह यथासुखम्} %॥४५॥

\twolineshloka
{इत्थं तयोक्तमाकर्ण्य पुरोऽगच्छद्द्विजात्मजः}
{विचिन्त्य परिहासोक्तिं गच्छति स्म यथा पुरा} %॥४६॥

\twolineshloka
{पुनरप्याह सा बाला तिष्ठतिष्ठ क्व यास्यसि}
{दुरुत्सहस्मरावेशां परिभोक्तुमुपेत्य माम्} %॥४७॥

\twolineshloka
{परिष्वजस्व मां कान्तां पाययस्व तवाधरम्}
{नाहं गन्तुं समर्थाऽस्मि स्मरबाणप्रपीडिता॥} %॥४८॥

\twolineshloka
{इत्थमश्रुतपूर्वां तां निशम्य परिशङ्कितः}
{आयान्तीं पृष्ठतो वीक्ष्य सहसा विस्मयं गतः} %॥४९॥

\twolineshloka
{कैषा पद्मपलाशाक्षी पीनोन्नतपयोधरा}
{कृशोदरी बृहच्छ्रोणी नवपल्लवकोमला} %॥५०॥

\twolineshloka
{स एव मे सखा किं नु जात एव वराङ्गना}
{पृच्छाम्येनमतः सर्वमिति सञ्चिन्त्य सोऽब्रवीत्} %॥५१॥

\twolineshloka
{किमपूर्व इवाऽऽभाषि सखे रूपगुणादिभिः}
{अपूर्वं भाषसे वाक्यं कामिनीव समाकुला} %॥५२॥

\twolineshloka
{यस्त्वं वेदपुराणज्ञो ब्रह्मचारी जितेन्द्रियः}
{सारस्वतात्मजः शान्तः कथमेवं प्रभाषसे} %॥५३॥

\twolineshloka
{इत्युक्ता सा पुनः प्राह नाहमस्मि पुमान्प्रभो}
{नाम्ना सामवती बाला तवास्मि रतिदायिनी} %॥५४॥

\twolineshloka
{यदि ते संशयः कान्त ममाङ्गानि विलोकय}
{इत्युक्तः सहसा मार्गे रहस्येनां व्यलोकयत्} %॥५५॥

\twolineshloka
{तामकृत्रिमधम्मिल्लां जवनस्तनशोभिनीम्}
{सुरूपां वीक्ष्य कामेन किञ्चिद्व्याकुलतामगात्} %॥५६॥

\twolineshloka
{पुनः संस्तभ्य यत्नेन चेतसो विकृतिं बुधः}
{मुहूर्तं विस्मयाविष्टो न किञ्चित्प्रत्यभाषत} %॥५७॥

\uvacha{सामवत्युवाच}
\twolineshloka
{गतस्ते संशयः कश्चित्तर्ह्यागच्छ भजस्व माम्}
{पश्येदं विपिनं कान्त परस्त्रीसुरतोचितम्} %॥५८॥

\uvacha{सुमेधा उवाच}
\twolineshloka
{मैवं कथय मर्यादां मा हिंसीर्मदमत्तवत्}
{आवां विज्ञातशास्त्रार्थौ त्वमेवं भाषसे कथम्} %॥५९॥

\twolineshloka
{अधीतस्य च शास्त्रस्य विवेकस्य कुलस्य च}
{किमेष सदृशो धर्मो जारधर्मनिषेवणम्} %॥६०॥

\twolineshloka
{न त्वं स्त्री पुरुषो विद्वाञ्जानीह्यात्मानमात्मना}
{अयं स्वयङ्कृतोऽनर्थ आवाभ्यां यद्विचेष्टितम्} %॥६१॥

\twolineshloka
{वञ्चयित्वाऽऽत्मपितरौ धूर्त्तराजानुशासनात्}
{कृत्वा चानुचितं कर्म तस्यैतद् भुज्यते फलम्} %॥६२॥

\twolineshloka
{सर्वं त्वनुचितं कर्म नृणां श्रेयोविनाशनम्}
{यस्त्वं विप्रात्मजो विद्वान्गतः स्त्रीत्वं विगर्हितम्} %॥६३॥

\twolineshloka
{मार्गं त्यक्त्वा गतोऽरण्यं नरो विध्येत कण्टकैः}
{बलाद्धिंस्येत वा हिंस्रैर्यदा त्यक्तसमागमः} %॥६४॥

\twolineshloka
{एवं विवेकमाश्रित्य तूष्णीमेहि स्वयं गृहम्}
{देवद्विजप्रसादेन स्त्रीत्वं तव विलीयते} %॥६५॥

\twolineshloka
{अथवा दैवयोगेन स्त्रीत्वमेव भवेत्तव}
{पित्रा दत्ता मया साकं रंस्यसे वरवर्णिनि} %॥६६॥

\twolineshloka
{अहो चित्रमहो दुःखमहो पापबलं महत्}
{अहो राज्ञः प्रभावोऽयं शिवाराधनसम्भृतः} %॥६७॥

\twolineshloka
{इत्युक्ताऽप्यसकृत् तेन सा वधूरतिविह्वला}
{बलेन तं समालिङ्ग्य चुचुम्बाधरपल्लवम्} %॥६८॥

\twolineshloka
{धर्षितोऽपि तया धीरः सुमेधा नूतनस्त्रियम्}
{यत्नादानीय सदनं कृत्स्नं तत्र न्यवेदयत्} %॥६९॥

\twolineshloka
{तदाकर्ण्याथ तौ विप्रौ कुपितौ शोकविह्वलौ}
{ताभ्यां सह कुमाराभ्यां वैदर्भान्तिकमीयतुः} %॥७०॥

\twolineshloka
{ततः सारस्वतः प्राह राजानं धूर्तचेष्टितम्}
{राजन्ममात्मजं पश्य तव शासनयन्त्रितम्} %॥७१॥

\twolineshloka
{एतौ तवाज्ञावशगौ चक्रतुः कर्म गर्हितम्}
{मत्पुत्रस्तत्फलं भुङ्क्ते स्त्रीत्वं प्राप्य जुगुप्सितम्} %॥७२॥

\twolineshloka
{अद्य मे सन्ततिर्नष्टा निराशाः पितरो मम}
{नापुत्रस्य हि लोकोऽस्ति लुप्तपिण्डादिसंस्कृतेः} %॥७३॥

\twolineshloka
{शिखोपवीतमजिनं मौञ्जीं दण्डं कमण्डलुम्}
{ब्रह्मचर्योचितं चिह्नं विहायेमां दशां गतः} %॥७४॥

\twolineshloka
{ब्रह्मसूत्रं च सावित्रीं स्नानं सन्ध्यां जपार्चनम्}
{विसृज्य स्त्रीत्वमाप्तोऽस्य का गतिर्वद पार्थिव} %॥७५॥

\twolineshloka
{त्वया मे सन्ततिर्नष्टा नष्टो वेदपथश्च मे}
{एकात्मजस्य मे राजन्का गतिर्वद शाश्वती} %॥७६॥

\twolineshloka
{इति सारस्वतेनोक्तं वाक्यमाकर्ण्य भूपतिः}
{सीमन्तिन्याः प्रभावेण विस्मयं परमं गतः} %॥७७॥

\twolineshloka
{अथ सर्वान्समाहूय महर्षीनमितद्युतीन्}
{प्रसाद्य प्रार्थयामास तस्य पुंस्त्वं महीपतिः} %॥७८॥

\twolineshloka
{तेऽब्रुवन्नथ पार्वत्याः शिवस्य च समीहितम्}
{तद्भक्तानां च माहात्म्यं कोऽन्यथा कर्तुमीश्वरः} %॥७९॥

\twolineshloka
{अथ राजा भरद्वाजमादाय मुनिपुङ्गवम्}
{ताभ्यां सह द्विजाग्र्याभ्यां तत्सुताभ्यां समन्वितः} %॥८०॥

\twolineshloka
{अम्बिकाभवनं प्राप्य भरद्वाजोपदेशतः}
{तां देवीं नियमैस्तीव्रैरुपास्ते स्म महानिशि} %॥८१॥

\fourlineindentedshloka
{एवं त्रिरात्रं सुविसृष्टभोजनः}
{स पार्वतीध्यानरतो महीपतिः}
{सम्यक्प्रणामैर्विविधैश्च संस्तवैः}
{गौरीं प्रपन्नार्तिहरामतोषयत्} %॥८२॥

\twolineshloka
{ततः प्रसन्ना सा देवी भक्तस्य पृथिवीपतेः}
{स्वरूपं दर्शयामास चन्द्रकोटिसमप्रभम्} %॥८३॥

\twolineshloka
{अथाऽऽह गौरी राजानं किं ते ब्रूहि समीहितम्}
{सोऽप्याह पुंस्त्वमेतस्य कृपया दीयतामिति} %॥८४॥

\twolineshloka
{भूयोऽप्याह महादेवी मद्भक्तैः कर्म यत्कृतम्}
{शक्यते नान्यथा कर्तुं वर्षायुतशतैरपि} %॥८५॥

\uvacha{राजोवाच}
\twolineshloka
{एकात्मजो हि विप्रोयं कर्मणा नष्टसन्ततिः}
{कथं सुखं प्रपद्येत विना पुत्रेण तादृशः} %॥८६॥

\uvacha{देव्युवाच}
\twolineshloka
{तस्यान्यो मत्प्रसादेन भविष्यति सुतोत्तमः}
{विद्या विनयसम्पन्नो दीर्घायुरमलाशयः} %॥८७॥

\twolineshloka
{एषा सामवती नाम सुता तस्य द्विजन्मनः}
{भूत्वा सुमेधसः पत्नी कामभोगेन युज्यताम्} %॥८८॥

\twolineshloka
{इत्युक्त्वाऽन्तर्हिता देवी ते च राजपुरोगमाः}
{गताः स्वं स्वं गृहं सर्वे चक्रुस्तच्छासने स्थितिम्} %॥८९॥

\twolineshloka
{सोऽपि सारस्वतो विप्रः पुत्रं पूर्वसुतोत्तमम्}
{लेभे देव्याः प्रसादेन ह्यचिरादेव कालतः} %॥९०॥

\twolineshloka
{तां च सामवतीं कन्यां ददौ तस्मै सुमेधसे}
{तौ दम्पती चिरं कालं बुभुजाते परं सुखम्} %॥९१॥

\uvacha{सूत उवाच}
\twolineshloka
{इत्येष शिवभक्तायाः सीमन्तिन्या नृपस्त्रियाः}
{प्रभावः कथितः शम्भोर्माहात्म्यमपि वर्णितम्} %॥९२॥

\twolineshloka
{भूयोऽपि शिवभक्तानां प्रभावं विस्मयावहम्}
{समासाद्वर्णयिष्यामि श्रोतॄणां मङ्गलायनम्} %॥९३॥

{॥इति श्रीस्कान्दे महापुराणे एकाशीतिसाहस्र्यां संहितायां तृतीये ब्रह्मोत्तरखण्डे सीमन्तिन्याः प्रभाववर्णनं नाम नवमोऽध्यायः॥} 

