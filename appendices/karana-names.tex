% !TeX program = XeLaTeX
% !TeX root = ../pujavidhanam.tex
\chapter{करण-नामानि}
\label{app:karana_names}

\addtocounter{shlokacount}{81}
\threelineshloka
{बवश्च बालवश्चैव कौलवस्तैतिलस्तथा}
{गरश्च वणिजश्चापि विष्टिश्च शकुनिस्तथा}
{चतुष्पाच्चापि नागश्च किंस्तुघ्न इति कीर्तितम्}

{\fontsize{10}{4}\selectfont —श्रीब्रह्मवैवर्त-महापुराणे श्रीकृष्णजन्मखण्डे उत्तरार्धे नारदनारायणसंवादे राधोद्धवसंवादे कालनिरूपणं नाम षण्ण्वतितमेऽध्याये}

\begin{center}
{\bfseries {चराणि सप्त}}

१. बवम् \hspace{2ex} २. बालवम् \hspace{2ex} ३. कौलवम् \hspace{2ex} ४. तैतिलम्

५. गरजा \hspace{2ex} ६. वणिजा  \hspace{2ex}७. भद्रा

{\bfseries {स्थिराणि चत्वारि}}

१. शकुनिः \hspace{2ex} २. चतुष्पात् \hspace{2ex} ३. नागवान् \hspace{2ex} ४. किंस्तुघ्नम्

\end{center}

% \pagebreak[4]

\dnsub{तिथीनां पूर्वोत्तरार्ध-करणानि}
\begingroup
\normalsize
\begin{longtable}{llcc}
  & तिथिः       & पूर्वार्ध-करणम् & उत्तरार्ध-करणम् \\\endhead
  १.  & शुक्ल-प्रथमा   & किंस्तुघ्नम्     & बवम्          \\
  २.  & शुक्ल-द्वितीया & बालवम्       & कौलवम्        \\
  ३.  & शुक्ल-तृतीया   & तैतिलम्       & गरजा         \\
  ४.  & शुक्ल-चतुर्थी   & वणिजा       & भद्रा         \\
  ५.  & शुक्ल-पञ्चमी   & बवम्         & बालवम्        \\
  ६.  & शुक्ल-षष्ठी    & कौलवम्       & तैतिलम्        \\
  ७.  & शुक्ल-सप्तमी   & गरजा        & वणिजा        \\
  ८.  & शुक्ल-अष्टमी   & भद्रा        & बवम्          \\
  ९.  & शुक्ल-नवमी    & बालवम्       & कौलवम्        \\
  १०. & शुक्ल-दशमी    & तैतिलम्       & गरजा         \\
  ११. & शुक्ल-एकादशी  & वणिजा       & भद्रा         \\
  १२. & शुक्ल-द्वादशी  & बवम्         & बालवम्        \\
  १३. & शुक्ल-त्रयोदशी & कौलवम्       & तैतिलम्        \\
  १४. & शुक्ल-चतुर्दशी  & गरजा        & वणिजा        \\
  १५. & पौर्णमासी    & भद्रा        & बवम्          \\
  १६. & कृष्ण-प्रथमा   & बालवम्       & कौलवम्        \\
  १७. & कृष्ण-द्वितीया & तैतिलम्       & गरजा         \\
  १८. & कृष्ण-तृतीया   & वणिजा       & भद्रा         \\
  १९. & कृष्ण-चतुर्थी   & बवम्         & बालवम्        \\
  २०. & कृष्ण-पञ्चमी   & कौलवम्       & तैतिलम्        \\
  २१. & कृष्ण-षष्ठी    & गरजा        & वणिजा        \\
  २२. & कृष्ण-सप्तमी   & भद्रा        & बवम्          \\
  २३. & कृष्ण-अष्टमी   & बालवम्       & कौलवम्        \\
  २४. & कृष्ण-नवमी    & तैतिलम्       & गरजा         \\
  २५. & कृष्ण-दशमी    & वणिजा       & भद्रा         \\
  २६. & कृष्ण-एकादशी  & बवम्         & बालवम्        \\
  २७. & कृष्ण-द्वादशी  & कौलवम्       & तैतिलम्        \\
  २८. & कृष्ण-त्रयोदशी & गरजा        & वणिजा        \\
  २९. & कृष्ण-चतुर्दशी  & भद्रा        & शकुनिः        \\
  ३०. & अमावास्या    & चतुष्पात्      & नागवान्       \\
\end{longtable}

