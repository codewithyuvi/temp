% !TeX program = XeLaTeX
% !TeX root = ../pujavidhanam.tex
\section{संवत्सर-नामानि}
\label{app:samvatsara_names}

\twolineshloka
{प्रभवो विभवः शुक्लः प्रमोदोऽथ प्रजापतिः}
{अङ्गिराः श्रीमुखो भावः युवा धाता तथैव च}

\twolineshloka
{ईश्वरो बहुधान्यश्च प्रमाथी विक्रमो विषुः}
{चित्रभानुः स्वभानुश्च तारणः पार्थिवो व्ययः}

\twolineshloka
{सर्वजित्सर्वधारी च विरोधी विकृतिः खरः}
{नन्दनो विजयश्चैव जयो मन्मथदुर्मुखौ}

\twolineshloka
{हेमलम्बो विलम्बश्च विकारः शार्वरी प्लवः}
{शुभकृच्छोभनः क्रोधी विश्वावसुपराभवौ}

\twolineshloka
{प्लवङ्गः कीलकः सौम्यः साधारणविरोधिकृत्}
{परिधावी प्रमाधी च आनन्दो राक्षसो नलः}

\twolineshloka
{पिङ्गलः कालसिद्धार्थौ रौद्रिर्वै दुर्मतिस्तथा}
{दुन्दुभी रुधिरोद्गारी रक्ताक्षः क्रोधनोऽक्षयः}

\begin{multicols}{2}
\begin{enumerate}
\item प्रभव
\item विभव
\item शुक्ल
\item प्रमोद
\item प्रजापति
\item आङ्गिरस
\item श्रीमुख
\item भाव
\item युव
\item धातृ
\item ईश्वर
\item बहुधान्य
\item प्रमाधि
\item विक्रम
\item वृष
\item चित्रभानु
\item स्वभानु
\item तारण
\item पार्थिव
\item व्यय
\item सर्वजित्
\item सर्वधारी
\item विरोधी
\item विकृति
\item खर
\item नन्दन
\item विजय
\item जय
\item मन्मथ
\item दुर्मुखी
\item हेविलम्बी
\item विलम्बी
\item विकारी
\item शार्वरी
\item प्लव
\item शुभकृत्
\item शोभकृत्
\item क्रोधी
\item विश्वावसु
\item पराभव
\item प्लवङ्ग
\item कीलक
\item सौम्य
\item साधारण
\item विरोधिकृति
\item परिधावी
\item प्रमादीच
\item आनन्द
\item राक्षस
\item नल
\item पिङ्गल
\item कालयुक्ति
\item सिद्धार्थी
\item रौद्र
\item दुर्मति
\item दुन्दुभि
\item रुधिरोद्गारी
\item रक्ताक्षी
\item क्रोधन
\item अक्षय
\end{enumerate}
\end{multicols}