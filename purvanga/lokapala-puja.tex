% !TeX program = XeLaTeX
% !TeX root = ..\pujavidhanam.tex

\sect{लोकपाल-पूजा}

प्राणान् आयम्य। ममोपात्त-समस्त-दुरित-क्षयद्वारा श्रीपरमेश्वरप्रीत्यर्थम् अद्य-पूर्वोक्त एवं गुण-विशेषेण विशिष्टायाम् अस्यां
अमावास्यायां शुभतिथौ श्रीमहालक्ष्मी-पूजाङ्गभूतां ब्रह्म-विष्णु-त्र्यम्बक-क्षेत्रपाल-पूजां करिष्ये।

अस्मिन् कूर्चे ब्रह्मादीन् ध्यायामि। ब्रह्मन् सरस्वत्या सह इह आगच्छ आगच्छ। सरस्वती-सहित-ब्रह्माणम् आवाहयामि। आसनं समर्पयामि।

लक्ष्मी-विष्णुभ्यां नमः।\\
ध्यायामि। आवाहयामि। आसनं समर्पयामि।

दुर्गा-त्र्यम्बकाभ्यां नमः।\\
ध्यायामि। आवाहयामि। आसनं समर्पयामि।

क्षेत्रपाल-भूमिभ्यां नमः।\\
ध्यायामि। आवाहयामि। आसनं समर्पयामि।

ब्रह्मादिभ्यो नमः पाद्यं समर्पयामि। अर्घ्यं समर्पयामि।
आचमनीयं समर्पयामि। शुद्धोदकस्नानं समर्पयामि। स्नानानन्तरम् आचमनीयं समर्पयामि।
वस्त्रार्थम् अक्षतान् समर्पयामि।
यज्ञोपवीताभरणार्थे अक्षतान् समर्पयामि।
दिव्यपरिमलगन्धान् धारयामि।
गन्धस्योपरि हरिद्राकुङ्कुमं समर्पयामि। अक्षतान् समर्पयामि। \\
पुष्पैः पूजयामि।\\

नैवेद्यम्। \\
कर्पूरताम्बूलं समर्पयामि। कर्पूरनीराजनं दर्शयामि।\\
प्रार्थनाः समर्पयामि।
अनन्तकोटिप्रदक्षिणनमस्कारान् समर्पयामि।\\

ब्रह्मादिभ्यो नमः (अक्षतान् समर्पयित्वा) यथास्थानं प्रतिष्ठापयामि। शोभनार्थे क्षेमाय पुनरागमनाय च।

\dnsub{प्रार्थना}

\twolineshloka*
{विघ्नराजं नमस्कृत्य नमस्कृत्य विधिं परम्}
{विष्णुं रुद्रं श्रियं दुर्गां वन्दे भक्त्या सरस्वतीम्}

\twolineshloka*
{क्षेत्राधिपं नमस्कृत्य दिवानाथं निशाकरम्}
{धरणीगर्भसम्भूतं शशिपुत्रं बृहस्पतिम्}

\twolineshloka*
{दैत्याचार्यं नमस्कृत्य सूर्यपुत्रं महाग्रहम्}
{राहुकेतू नमस्कृत्य यज्ञारम्भे विशेषतः}

\twolineshloka*
{शक्राद्या देवताः सर्वाः मुनींश्च प्रणमाम्यहम्}
{गर्गं मुनिं नमस्कृत्य नारदं मुनिसत्तमम्}

\twolineshloka*
{वसिष्ठं मुनिशार्दूलं विश्वामित्रं भृगोः सुतम्}
{व्यासं मुनिं नमस्कृत्य आचार्यांश्च तपोधनान्}

\twolineshloka*
{सर्वान् तान् प्रणमाम्येवं यज्ञरक्षाकरान् सदा}
{शङ्खचक्रगदाशार्ङ्ग-पद्मपाणिर्जनार्दनः}
\onelineshloka*
{सर्वासु दिक्षु रक्षेन्मां यावत् पूजावसानकम्}
